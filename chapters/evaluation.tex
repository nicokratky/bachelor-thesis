\chapter{Evaluation}
\label{ch:evaluation}

This chapter first introduces the setup that was used for evaluating the different elasticity strategies. Then the results of different tests are presented and discussed.

\section{Testsetup}
\label{sec:testsetup}

In order to test the different elasticity strategies a test environment has to be set up. It was decided to create three virtual machines (VM) that will form a Kubernetes cluster. Because of its easy of use microk8s was chosen as distribution\footnote{\url{https://microk8s.io/}}. All three virtual machines were assigned 10 vCPUs and 10GB of memory. One VM acts as the Kubernetes control plane while the other two join the cluster as worker nodes.

Everything that was deployed into the Kubernetes cluster was built using the infrastructure as code (IaC) tool HashiCorp Terraform\footnote{\url{https://www.terraform.io/}}. This enables rapid changes and reproducibility. Deployed resources include the kube-prometheus-stack\footnote{\raggedright\url{https://artifacthub.io/packages/helm/prometheus-community/kube-prometheus-stack}} for monitoring, the k8ssandra-operator\footnote{\url{https://docs.k8ssandra.io/components/k8ssandra-operator/}} for managing k8ssandra clusters and a definition for a k8ssandra cluster. Additionally, the in \cref{sec:metrics} mentioned Grafana ashboards are also deployed using Terraform.

\Cref{lst:k8c} illustrates a minimal definition of a 3 node k8ssandra cluster. Each node has resource limits of 800 millicpu and 6000 megabytes of memory and 3 gibibytes storage space.

\begin{lstlisting}[caption={Minimal example of a K8ssandraCluster definition.},
                label=lst:k8c,
                captionpos=b,
                float]
apiVersion: k8ssandra.io/v1alpha1
kind: K8ssandraCluster
metadata:
  name: polaris-test-cluster
  namespace: k8ssandra
spec:
  cassandra:
    resources:
      limits:
        cpu: 800m
        memory: 6000M
    datacenters:
      - metadata:
          name: dc1
        size: 3
        storageConfig:
          cassandraDataVolumeClaimSpec:
            resources:
              requests:
                storage: 3Gi
\end{lstlisting}

\section{Benchmarks}

In the following sections, different test scenarios will be discussed. To let k8ssandra experience load, the built-in stress testing tool \texttt{cassandra-stress} was used\footnote{\raggedright\url{https://cassandra.apache.org/doc/stable/cassandra/tools/cassandra_stress.html}}.

\subsection{Stress Testing}
\label{sec:stress-testing}

To set a baseline, three different k8ssandra cluster setups have been stress tested using \texttt{cassandra-stress}. The amount of write requests that the tool will make is set to be 1000000, the exact call is listed in \cref{lst:stress-1000000writes}. These cluster setups merely differ in the cluster size, thus the amount of nodes. All clusters were provisioned with limits of 2 CPUs and 6GB of memory.

\begin{lstlisting}[caption={},
                    captionpos=b,
                    label=lst:stress-1000000writes,
                    float]
./cassandra-stress write n=1000000 -mode native cql3 \
    user='USERNAME' password='PASSWORD'
\end{lstlisting}

The results of these tests are depicted in \cref{fig:stress-1000000writes-1node,fig:stress-1000000writes-2node,,fig:stress-1000000writes-3node}. The write throughput increases with the amount of nodes, but not linearly. This, however, was to be expected as \texttt{cassandra-stress} does not partition data in way that favours linear scalability. The average write throughputs of these different clusters can be seen in \cref{tab:stress-1000000writes-ops}.

\begin{table}[H]
\centering
\begin{tabular}{|l|l|l|}
\hline
\textbf{Cluster size} & \textbf{operations/s} & \textbf{Time to complete} \\ \hline
1                     & 12514                 & 2m38s                     \\ \hline
2                     & 13142                 & 1m57s                     \\ \hline
3                     & 14318                 & 1m50s                     \\ \hline
\end{tabular}
\caption{Average write throughput for different k8ssandra clusters. With increasing cluster size the throughput also increases}
\label{tab:stress-1000000writes-ops}
\end{table}

\begin{figure}
    \centering
    %% Creator: Matplotlib, PGF backend
%%
%% To include the figure in your LaTeX document, write
%%   \input{<filename>.pgf}
%%
%% Make sure the required packages are loaded in your preamble
%%   \usepackage{pgf}
%%
%% Also ensure that all the required font packages are loaded; for instance,
%% the lmodern package is sometimes necessary when using math font.
%%   \usepackage{lmodern}
%%
%% Figures using additional raster images can only be included by \input if
%% they are in the same directory as the main LaTeX file. For loading figures
%% from other directories you can use the `import` package
%%   \usepackage{import}
%%
%% and then include the figures with
%%   \import{<path to file>}{<filename>.pgf}
%%
%% Matplotlib used the following preamble
%%   \def\mathdefault#1{#1}
%%   \everymath=\expandafter{\the\everymath\displaystyle}
%%   
%%   \usepackage{fontspec}
%%   \setmainfont{DejaVuSerif.ttf}[Path=\detokenize{/Users/nkratky/private/polaris-elasticity-strategies/test/scripts/.venv/lib/python3.11/site-packages/matplotlib/mpl-data/fonts/ttf/}]
%%   \setsansfont{Arial.ttf}[Path=\detokenize{/System/Library/Fonts/Supplemental/}]
%%   \setmonofont{DejaVuSansMono.ttf}[Path=\detokenize{/Users/nkratky/private/polaris-elasticity-strategies/test/scripts/.venv/lib/python3.11/site-packages/matplotlib/mpl-data/fonts/ttf/}]
%%   \makeatletter\@ifpackageloaded{underscore}{}{\usepackage[strings]{underscore}}\makeatother
%%
\begingroup%
\makeatletter%
\begin{pgfpicture}%
\pgfpathrectangle{\pgfpointorigin}{\pgfqpoint{5.600000in}{2.500000in}}%
\pgfusepath{use as bounding box, clip}%
\begin{pgfscope}%
\pgfsetbuttcap%
\pgfsetmiterjoin%
\definecolor{currentfill}{rgb}{1.000000,1.000000,1.000000}%
\pgfsetfillcolor{currentfill}%
\pgfsetlinewidth{0.000000pt}%
\definecolor{currentstroke}{rgb}{1.000000,1.000000,1.000000}%
\pgfsetstrokecolor{currentstroke}%
\pgfsetdash{}{0pt}%
\pgfpathmoveto{\pgfqpoint{0.000000in}{0.000000in}}%
\pgfpathlineto{\pgfqpoint{5.600000in}{0.000000in}}%
\pgfpathlineto{\pgfqpoint{5.600000in}{2.500000in}}%
\pgfpathlineto{\pgfqpoint{0.000000in}{2.500000in}}%
\pgfpathlineto{\pgfqpoint{0.000000in}{0.000000in}}%
\pgfpathclose%
\pgfusepath{fill}%
\end{pgfscope}%
\begin{pgfscope}%
\pgfsetbuttcap%
\pgfsetmiterjoin%
\definecolor{currentfill}{rgb}{0.917647,0.917647,0.949020}%
\pgfsetfillcolor{currentfill}%
\pgfsetlinewidth{0.000000pt}%
\definecolor{currentstroke}{rgb}{0.000000,0.000000,0.000000}%
\pgfsetstrokecolor{currentstroke}%
\pgfsetstrokeopacity{0.000000}%
\pgfsetdash{}{0pt}%
\pgfpathmoveto{\pgfqpoint{0.948751in}{0.663635in}}%
\pgfpathlineto{\pgfqpoint{5.420000in}{0.663635in}}%
\pgfpathlineto{\pgfqpoint{5.420000in}{2.320000in}}%
\pgfpathlineto{\pgfqpoint{0.948751in}{2.320000in}}%
\pgfpathlineto{\pgfqpoint{0.948751in}{0.663635in}}%
\pgfpathclose%
\pgfusepath{fill}%
\end{pgfscope}%
\begin{pgfscope}%
\pgfpathrectangle{\pgfqpoint{0.948751in}{0.663635in}}{\pgfqpoint{4.471249in}{1.656365in}}%
\pgfusepath{clip}%
\pgfsetroundcap%
\pgfsetroundjoin%
\pgfsetlinewidth{1.003750pt}%
\definecolor{currentstroke}{rgb}{1.000000,1.000000,1.000000}%
\pgfsetstrokecolor{currentstroke}%
\pgfsetdash{}{0pt}%
\pgfpathmoveto{\pgfqpoint{1.151990in}{0.663635in}}%
\pgfpathlineto{\pgfqpoint{1.151990in}{2.320000in}}%
\pgfusepath{stroke}%
\end{pgfscope}%
\begin{pgfscope}%
\definecolor{textcolor}{rgb}{0.150000,0.150000,0.150000}%
\pgfsetstrokecolor{textcolor}%
\pgfsetfillcolor{textcolor}%
\pgftext[x=1.151990in,y=0.531691in,,top]{\color{textcolor}{\sffamily\fontsize{11.000000}{13.200000}\selectfont\catcode`\^=\active\def^{\ifmmode\sp\else\^{}\fi}\catcode`\%=\active\def%{\%}0}}%
\end{pgfscope}%
\begin{pgfscope}%
\pgfpathrectangle{\pgfqpoint{0.948751in}{0.663635in}}{\pgfqpoint{4.471249in}{1.656365in}}%
\pgfusepath{clip}%
\pgfsetroundcap%
\pgfsetroundjoin%
\pgfsetlinewidth{1.003750pt}%
\definecolor{currentstroke}{rgb}{1.000000,1.000000,1.000000}%
\pgfsetstrokecolor{currentstroke}%
\pgfsetdash{}{0pt}%
\pgfpathmoveto{\pgfqpoint{1.829452in}{0.663635in}}%
\pgfpathlineto{\pgfqpoint{1.829452in}{2.320000in}}%
\pgfusepath{stroke}%
\end{pgfscope}%
\begin{pgfscope}%
\definecolor{textcolor}{rgb}{0.150000,0.150000,0.150000}%
\pgfsetstrokecolor{textcolor}%
\pgfsetfillcolor{textcolor}%
\pgftext[x=1.829452in,y=0.531691in,,top]{\color{textcolor}{\sffamily\fontsize{11.000000}{13.200000}\selectfont\catcode`\^=\active\def^{\ifmmode\sp\else\^{}\fi}\catcode`\%=\active\def%{\%}20}}%
\end{pgfscope}%
\begin{pgfscope}%
\pgfpathrectangle{\pgfqpoint{0.948751in}{0.663635in}}{\pgfqpoint{4.471249in}{1.656365in}}%
\pgfusepath{clip}%
\pgfsetroundcap%
\pgfsetroundjoin%
\pgfsetlinewidth{1.003750pt}%
\definecolor{currentstroke}{rgb}{1.000000,1.000000,1.000000}%
\pgfsetstrokecolor{currentstroke}%
\pgfsetdash{}{0pt}%
\pgfpathmoveto{\pgfqpoint{2.506914in}{0.663635in}}%
\pgfpathlineto{\pgfqpoint{2.506914in}{2.320000in}}%
\pgfusepath{stroke}%
\end{pgfscope}%
\begin{pgfscope}%
\definecolor{textcolor}{rgb}{0.150000,0.150000,0.150000}%
\pgfsetstrokecolor{textcolor}%
\pgfsetfillcolor{textcolor}%
\pgftext[x=2.506914in,y=0.531691in,,top]{\color{textcolor}{\sffamily\fontsize{11.000000}{13.200000}\selectfont\catcode`\^=\active\def^{\ifmmode\sp\else\^{}\fi}\catcode`\%=\active\def%{\%}40}}%
\end{pgfscope}%
\begin{pgfscope}%
\pgfpathrectangle{\pgfqpoint{0.948751in}{0.663635in}}{\pgfqpoint{4.471249in}{1.656365in}}%
\pgfusepath{clip}%
\pgfsetroundcap%
\pgfsetroundjoin%
\pgfsetlinewidth{1.003750pt}%
\definecolor{currentstroke}{rgb}{1.000000,1.000000,1.000000}%
\pgfsetstrokecolor{currentstroke}%
\pgfsetdash{}{0pt}%
\pgfpathmoveto{\pgfqpoint{3.184376in}{0.663635in}}%
\pgfpathlineto{\pgfqpoint{3.184376in}{2.320000in}}%
\pgfusepath{stroke}%
\end{pgfscope}%
\begin{pgfscope}%
\definecolor{textcolor}{rgb}{0.150000,0.150000,0.150000}%
\pgfsetstrokecolor{textcolor}%
\pgfsetfillcolor{textcolor}%
\pgftext[x=3.184376in,y=0.531691in,,top]{\color{textcolor}{\sffamily\fontsize{11.000000}{13.200000}\selectfont\catcode`\^=\active\def^{\ifmmode\sp\else\^{}\fi}\catcode`\%=\active\def%{\%}60}}%
\end{pgfscope}%
\begin{pgfscope}%
\pgfpathrectangle{\pgfqpoint{0.948751in}{0.663635in}}{\pgfqpoint{4.471249in}{1.656365in}}%
\pgfusepath{clip}%
\pgfsetroundcap%
\pgfsetroundjoin%
\pgfsetlinewidth{1.003750pt}%
\definecolor{currentstroke}{rgb}{1.000000,1.000000,1.000000}%
\pgfsetstrokecolor{currentstroke}%
\pgfsetdash{}{0pt}%
\pgfpathmoveto{\pgfqpoint{3.861838in}{0.663635in}}%
\pgfpathlineto{\pgfqpoint{3.861838in}{2.320000in}}%
\pgfusepath{stroke}%
\end{pgfscope}%
\begin{pgfscope}%
\definecolor{textcolor}{rgb}{0.150000,0.150000,0.150000}%
\pgfsetstrokecolor{textcolor}%
\pgfsetfillcolor{textcolor}%
\pgftext[x=3.861838in,y=0.531691in,,top]{\color{textcolor}{\sffamily\fontsize{11.000000}{13.200000}\selectfont\catcode`\^=\active\def^{\ifmmode\sp\else\^{}\fi}\catcode`\%=\active\def%{\%}80}}%
\end{pgfscope}%
\begin{pgfscope}%
\pgfpathrectangle{\pgfqpoint{0.948751in}{0.663635in}}{\pgfqpoint{4.471249in}{1.656365in}}%
\pgfusepath{clip}%
\pgfsetroundcap%
\pgfsetroundjoin%
\pgfsetlinewidth{1.003750pt}%
\definecolor{currentstroke}{rgb}{1.000000,1.000000,1.000000}%
\pgfsetstrokecolor{currentstroke}%
\pgfsetdash{}{0pt}%
\pgfpathmoveto{\pgfqpoint{4.539299in}{0.663635in}}%
\pgfpathlineto{\pgfqpoint{4.539299in}{2.320000in}}%
\pgfusepath{stroke}%
\end{pgfscope}%
\begin{pgfscope}%
\definecolor{textcolor}{rgb}{0.150000,0.150000,0.150000}%
\pgfsetstrokecolor{textcolor}%
\pgfsetfillcolor{textcolor}%
\pgftext[x=4.539299in,y=0.531691in,,top]{\color{textcolor}{\sffamily\fontsize{11.000000}{13.200000}\selectfont\catcode`\^=\active\def^{\ifmmode\sp\else\^{}\fi}\catcode`\%=\active\def%{\%}100}}%
\end{pgfscope}%
\begin{pgfscope}%
\pgfpathrectangle{\pgfqpoint{0.948751in}{0.663635in}}{\pgfqpoint{4.471249in}{1.656365in}}%
\pgfusepath{clip}%
\pgfsetroundcap%
\pgfsetroundjoin%
\pgfsetlinewidth{1.003750pt}%
\definecolor{currentstroke}{rgb}{1.000000,1.000000,1.000000}%
\pgfsetstrokecolor{currentstroke}%
\pgfsetdash{}{0pt}%
\pgfpathmoveto{\pgfqpoint{5.216761in}{0.663635in}}%
\pgfpathlineto{\pgfqpoint{5.216761in}{2.320000in}}%
\pgfusepath{stroke}%
\end{pgfscope}%
\begin{pgfscope}%
\definecolor{textcolor}{rgb}{0.150000,0.150000,0.150000}%
\pgfsetstrokecolor{textcolor}%
\pgfsetfillcolor{textcolor}%
\pgftext[x=5.216761in,y=0.531691in,,top]{\color{textcolor}{\sffamily\fontsize{11.000000}{13.200000}\selectfont\catcode`\^=\active\def^{\ifmmode\sp\else\^{}\fi}\catcode`\%=\active\def%{\%}120}}%
\end{pgfscope}%
\begin{pgfscope}%
\definecolor{textcolor}{rgb}{0.150000,0.150000,0.150000}%
\pgfsetstrokecolor{textcolor}%
\pgfsetfillcolor{textcolor}%
\pgftext[x=3.184376in,y=0.336413in,,top]{\color{textcolor}{\sffamily\fontsize{12.000000}{14.400000}\selectfont\catcode`\^=\active\def^{\ifmmode\sp\else\^{}\fi}\catcode`\%=\active\def%{\%}Time (s)}}%
\end{pgfscope}%
\begin{pgfscope}%
\pgfpathrectangle{\pgfqpoint{0.948751in}{0.663635in}}{\pgfqpoint{4.471249in}{1.656365in}}%
\pgfusepath{clip}%
\pgfsetroundcap%
\pgfsetroundjoin%
\pgfsetlinewidth{1.003750pt}%
\definecolor{currentstroke}{rgb}{1.000000,1.000000,1.000000}%
\pgfsetstrokecolor{currentstroke}%
\pgfsetdash{}{0pt}%
\pgfpathmoveto{\pgfqpoint{0.948751in}{0.738925in}}%
\pgfpathlineto{\pgfqpoint{5.420000in}{0.738925in}}%
\pgfusepath{stroke}%
\end{pgfscope}%
\begin{pgfscope}%
\definecolor{textcolor}{rgb}{0.150000,0.150000,0.150000}%
\pgfsetstrokecolor{textcolor}%
\pgfsetfillcolor{textcolor}%
\pgftext[x=0.731839in, y=0.684244in, left, base]{\color{textcolor}{\sffamily\fontsize{11.000000}{13.200000}\selectfont\catcode`\^=\active\def^{\ifmmode\sp\else\^{}\fi}\catcode`\%=\active\def%{\%}0}}%
\end{pgfscope}%
\begin{pgfscope}%
\pgfpathrectangle{\pgfqpoint{0.948751in}{0.663635in}}{\pgfqpoint{4.471249in}{1.656365in}}%
\pgfusepath{clip}%
\pgfsetroundcap%
\pgfsetroundjoin%
\pgfsetlinewidth{1.003750pt}%
\definecolor{currentstroke}{rgb}{1.000000,1.000000,1.000000}%
\pgfsetstrokecolor{currentstroke}%
\pgfsetdash{}{0pt}%
\pgfpathmoveto{\pgfqpoint{0.948751in}{1.339773in}}%
\pgfpathlineto{\pgfqpoint{5.420000in}{1.339773in}}%
\pgfusepath{stroke}%
\end{pgfscope}%
\begin{pgfscope}%
\definecolor{textcolor}{rgb}{0.150000,0.150000,0.150000}%
\pgfsetstrokecolor{textcolor}%
\pgfsetfillcolor{textcolor}%
\pgftext[x=0.391968in, y=1.285092in, left, base]{\color{textcolor}{\sffamily\fontsize{11.000000}{13.200000}\selectfont\catcode`\^=\active\def^{\ifmmode\sp\else\^{}\fi}\catcode`\%=\active\def%{\%}10000}}%
\end{pgfscope}%
\begin{pgfscope}%
\pgfpathrectangle{\pgfqpoint{0.948751in}{0.663635in}}{\pgfqpoint{4.471249in}{1.656365in}}%
\pgfusepath{clip}%
\pgfsetroundcap%
\pgfsetroundjoin%
\pgfsetlinewidth{1.003750pt}%
\definecolor{currentstroke}{rgb}{1.000000,1.000000,1.000000}%
\pgfsetstrokecolor{currentstroke}%
\pgfsetdash{}{0pt}%
\pgfpathmoveto{\pgfqpoint{0.948751in}{1.940621in}}%
\pgfpathlineto{\pgfqpoint{5.420000in}{1.940621in}}%
\pgfusepath{stroke}%
\end{pgfscope}%
\begin{pgfscope}%
\definecolor{textcolor}{rgb}{0.150000,0.150000,0.150000}%
\pgfsetstrokecolor{textcolor}%
\pgfsetfillcolor{textcolor}%
\pgftext[x=0.391968in, y=1.885941in, left, base]{\color{textcolor}{\sffamily\fontsize{11.000000}{13.200000}\selectfont\catcode`\^=\active\def^{\ifmmode\sp\else\^{}\fi}\catcode`\%=\active\def%{\%}20000}}%
\end{pgfscope}%
\begin{pgfscope}%
\definecolor{textcolor}{rgb}{0.150000,0.150000,0.150000}%
\pgfsetstrokecolor{textcolor}%
\pgfsetfillcolor{textcolor}%
\pgftext[x=0.336413in,y=1.491818in,,bottom,rotate=90.000000]{\color{textcolor}{\sffamily\fontsize{12.000000}{14.400000}\selectfont\catcode`\^=\active\def^{\ifmmode\sp\else\^{}\fi}\catcode`\%=\active\def%{\%}Writes (op/s)}}%
\end{pgfscope}%
\begin{pgfscope}%
\pgfpathrectangle{\pgfqpoint{0.948751in}{0.663635in}}{\pgfqpoint{4.471249in}{1.656365in}}%
\pgfusepath{clip}%
\pgfsetroundcap%
\pgfsetroundjoin%
\pgfsetlinewidth{1.505625pt}%
\definecolor{currentstroke}{rgb}{0.298039,0.447059,0.690196}%
\pgfsetstrokecolor{currentstroke}%
\pgfsetdash{}{0pt}%
\pgfpathmoveto{\pgfqpoint{1.151990in}{0.738925in}}%
\pgfpathlineto{\pgfqpoint{1.321355in}{0.738925in}}%
\pgfpathlineto{\pgfqpoint{1.490721in}{0.738925in}}%
\pgfpathlineto{\pgfqpoint{1.660086in}{1.222487in}}%
\pgfpathlineto{\pgfqpoint{1.829452in}{1.352691in}}%
\pgfpathlineto{\pgfqpoint{1.998817in}{1.482895in}}%
\pgfpathlineto{\pgfqpoint{2.168183in}{1.607451in}}%
\pgfpathlineto{\pgfqpoint{2.337548in}{1.694694in}}%
\pgfpathlineto{\pgfqpoint{2.506914in}{1.694694in}}%
\pgfpathlineto{\pgfqpoint{2.676279in}{1.765955in}}%
\pgfpathlineto{\pgfqpoint{2.845645in}{1.837275in}}%
\pgfpathlineto{\pgfqpoint{3.015010in}{1.837275in}}%
\pgfpathlineto{\pgfqpoint{3.184376in}{1.946450in}}%
\pgfpathlineto{\pgfqpoint{3.353741in}{2.055624in}}%
\pgfpathlineto{\pgfqpoint{3.523107in}{2.055624in}}%
\pgfpathlineto{\pgfqpoint{3.692472in}{2.150137in}}%
\pgfpathlineto{\pgfqpoint{3.861838in}{2.244711in}}%
\pgfpathlineto{\pgfqpoint{4.031203in}{2.244711in}}%
\pgfpathlineto{\pgfqpoint{4.200569in}{1.724496in}}%
\pgfpathlineto{\pgfqpoint{4.369934in}{1.204282in}}%
\pgfpathlineto{\pgfqpoint{4.539299in}{1.204282in}}%
\pgfpathlineto{\pgfqpoint{4.708665in}{0.971573in}}%
\pgfpathlineto{\pgfqpoint{4.878030in}{0.738925in}}%
\pgfpathlineto{\pgfqpoint{5.047396in}{0.738925in}}%
\pgfpathlineto{\pgfqpoint{5.216761in}{0.738925in}}%
\pgfusepath{stroke}%
\end{pgfscope}%
\begin{pgfscope}%
\pgfpathrectangle{\pgfqpoint{0.948751in}{0.663635in}}{\pgfqpoint{4.471249in}{1.656365in}}%
\pgfusepath{clip}%
\pgfsetroundcap%
\pgfsetroundjoin%
\pgfsetlinewidth{1.505625pt}%
\definecolor{currentstroke}{rgb}{1.000000,0.000000,0.000000}%
\pgfsetstrokecolor{currentstroke}%
\pgfsetdash{}{0pt}%
\pgfpathmoveto{\pgfqpoint{0.948751in}{1.469234in}}%
\pgfpathlineto{\pgfqpoint{5.420000in}{1.469234in}}%
\pgfusepath{stroke}%
\end{pgfscope}%
\begin{pgfscope}%
\pgfsetrectcap%
\pgfsetmiterjoin%
\pgfsetlinewidth{1.254687pt}%
\definecolor{currentstroke}{rgb}{1.000000,1.000000,1.000000}%
\pgfsetstrokecolor{currentstroke}%
\pgfsetdash{}{0pt}%
\pgfpathmoveto{\pgfqpoint{0.948751in}{0.663635in}}%
\pgfpathlineto{\pgfqpoint{0.948751in}{2.320000in}}%
\pgfusepath{stroke}%
\end{pgfscope}%
\begin{pgfscope}%
\pgfsetrectcap%
\pgfsetmiterjoin%
\pgfsetlinewidth{1.254687pt}%
\definecolor{currentstroke}{rgb}{1.000000,1.000000,1.000000}%
\pgfsetstrokecolor{currentstroke}%
\pgfsetdash{}{0pt}%
\pgfpathmoveto{\pgfqpoint{5.420000in}{0.663635in}}%
\pgfpathlineto{\pgfqpoint{5.420000in}{2.320000in}}%
\pgfusepath{stroke}%
\end{pgfscope}%
\begin{pgfscope}%
\pgfsetrectcap%
\pgfsetmiterjoin%
\pgfsetlinewidth{1.254687pt}%
\definecolor{currentstroke}{rgb}{1.000000,1.000000,1.000000}%
\pgfsetstrokecolor{currentstroke}%
\pgfsetdash{}{0pt}%
\pgfpathmoveto{\pgfqpoint{0.948751in}{0.663635in}}%
\pgfpathlineto{\pgfqpoint{5.420000in}{0.663635in}}%
\pgfusepath{stroke}%
\end{pgfscope}%
\begin{pgfscope}%
\pgfsetrectcap%
\pgfsetmiterjoin%
\pgfsetlinewidth{1.254687pt}%
\definecolor{currentstroke}{rgb}{1.000000,1.000000,1.000000}%
\pgfsetstrokecolor{currentstroke}%
\pgfsetdash{}{0pt}%
\pgfpathmoveto{\pgfqpoint{0.948751in}{2.320000in}}%
\pgfpathlineto{\pgfqpoint{5.420000in}{2.320000in}}%
\pgfusepath{stroke}%
\end{pgfscope}%
\begin{pgfscope}%
\pgfsetbuttcap%
\pgfsetmiterjoin%
\definecolor{currentfill}{rgb}{0.917647,0.917647,0.949020}%
\pgfsetfillcolor{currentfill}%
\pgfsetfillopacity{0.800000}%
\pgfsetlinewidth{1.003750pt}%
\definecolor{currentstroke}{rgb}{0.800000,0.800000,0.800000}%
\pgfsetstrokecolor{currentstroke}%
\pgfsetstrokeopacity{0.800000}%
\pgfsetdash{}{0pt}%
\pgfpathmoveto{\pgfqpoint{1.026529in}{2.072637in}}%
\pgfpathlineto{\pgfqpoint{3.179448in}{2.072637in}}%
\pgfpathquadraticcurveto{\pgfqpoint{3.201670in}{2.072637in}}{\pgfqpoint{3.201670in}{2.094859in}}%
\pgfpathlineto{\pgfqpoint{3.201670in}{2.242222in}}%
\pgfpathquadraticcurveto{\pgfqpoint{3.201670in}{2.264444in}}{\pgfqpoint{3.179448in}{2.264444in}}%
\pgfpathlineto{\pgfqpoint{1.026529in}{2.264444in}}%
\pgfpathquadraticcurveto{\pgfqpoint{1.004307in}{2.264444in}}{\pgfqpoint{1.004307in}{2.242222in}}%
\pgfpathlineto{\pgfqpoint{1.004307in}{2.094859in}}%
\pgfpathquadraticcurveto{\pgfqpoint{1.004307in}{2.072637in}}{\pgfqpoint{1.026529in}{2.072637in}}%
\pgfpathlineto{\pgfqpoint{1.026529in}{2.072637in}}%
\pgfpathclose%
\pgfusepath{stroke,fill}%
\end{pgfscope}%
\begin{pgfscope}%
\pgfsetroundcap%
\pgfsetroundjoin%
\pgfsetlinewidth{1.505625pt}%
\definecolor{currentstroke}{rgb}{1.000000,0.000000,0.000000}%
\pgfsetstrokecolor{currentstroke}%
\pgfsetdash{}{0pt}%
\pgfpathmoveto{\pgfqpoint{1.048751in}{2.179353in}}%
\pgfpathlineto{\pgfqpoint{1.159862in}{2.179353in}}%
\pgfpathlineto{\pgfqpoint{1.270973in}{2.179353in}}%
\pgfusepath{stroke}%
\end{pgfscope}%
\begin{pgfscope}%
\definecolor{textcolor}{rgb}{0.150000,0.150000,0.150000}%
\pgfsetstrokecolor{textcolor}%
\pgfsetfillcolor{textcolor}%
\pgftext[x=1.359862in,y=2.140464in,left,base]{\color{textcolor}{\sffamily\fontsize{8.000000}{9.600000}\selectfont\catcode`\^=\active\def^{\ifmmode\sp\else\^{}\fi}\catcode`\%=\active\def%{\%}average write operations per second}}%
\end{pgfscope}%
\end{pgfpicture}%
\makeatother%
\endgroup%

    \caption{Stress test of 1 node with 1000000 writes}
    \label{fig:stress-1000000writes-1node}
\end{figure}

\begin{figure}
    \centering
    %% Creator: Matplotlib, PGF backend
%%
%% To include the figure in your LaTeX document, write
%%   \input{<filename>.pgf}
%%
%% Make sure the required packages are loaded in your preamble
%%   \usepackage{pgf}
%%
%% Also ensure that all the required font packages are loaded; for instance,
%% the lmodern package is sometimes necessary when using math font.
%%   \usepackage{lmodern}
%%
%% Figures using additional raster images can only be included by \input if
%% they are in the same directory as the main LaTeX file. For loading figures
%% from other directories you can use the `import` package
%%   \usepackage{import}
%%
%% and then include the figures with
%%   \import{<path to file>}{<filename>.pgf}
%%
%% Matplotlib used the following preamble
%%   \def\mathdefault#1{#1}
%%   \everymath=\expandafter{\the\everymath\displaystyle}
%%   
%%   \usepackage{fontspec}
%%   \setmainfont{DejaVuSerif.ttf}[Path=\detokenize{/Users/nkratky/private/polaris-elasticity-strategies/test/scripts/.venv/lib/python3.11/site-packages/matplotlib/mpl-data/fonts/ttf/}]
%%   \setsansfont{Arial.ttf}[Path=\detokenize{/System/Library/Fonts/Supplemental/}]
%%   \setmonofont{DejaVuSansMono.ttf}[Path=\detokenize{/Users/nkratky/private/polaris-elasticity-strategies/test/scripts/.venv/lib/python3.11/site-packages/matplotlib/mpl-data/fonts/ttf/}]
%%   \makeatletter\@ifpackageloaded{underscore}{}{\usepackage[strings]{underscore}}\makeatother
%%
\begingroup%
\makeatletter%
\begin{pgfpicture}%
\pgfpathrectangle{\pgfpointorigin}{\pgfqpoint{5.600000in}{2.500000in}}%
\pgfusepath{use as bounding box, clip}%
\begin{pgfscope}%
\pgfsetbuttcap%
\pgfsetmiterjoin%
\definecolor{currentfill}{rgb}{1.000000,1.000000,1.000000}%
\pgfsetfillcolor{currentfill}%
\pgfsetlinewidth{0.000000pt}%
\definecolor{currentstroke}{rgb}{1.000000,1.000000,1.000000}%
\pgfsetstrokecolor{currentstroke}%
\pgfsetdash{}{0pt}%
\pgfpathmoveto{\pgfqpoint{0.000000in}{0.000000in}}%
\pgfpathlineto{\pgfqpoint{5.600000in}{0.000000in}}%
\pgfpathlineto{\pgfqpoint{5.600000in}{2.500000in}}%
\pgfpathlineto{\pgfqpoint{0.000000in}{2.500000in}}%
\pgfpathlineto{\pgfqpoint{0.000000in}{0.000000in}}%
\pgfpathclose%
\pgfusepath{fill}%
\end{pgfscope}%
\begin{pgfscope}%
\pgfsetbuttcap%
\pgfsetmiterjoin%
\definecolor{currentfill}{rgb}{0.917647,0.917647,0.949020}%
\pgfsetfillcolor{currentfill}%
\pgfsetlinewidth{0.000000pt}%
\definecolor{currentstroke}{rgb}{0.000000,0.000000,0.000000}%
\pgfsetstrokecolor{currentstroke}%
\pgfsetstrokeopacity{0.000000}%
\pgfsetdash{}{0pt}%
\pgfpathmoveto{\pgfqpoint{0.948751in}{0.663635in}}%
\pgfpathlineto{\pgfqpoint{5.343693in}{0.663635in}}%
\pgfpathlineto{\pgfqpoint{5.343693in}{2.320000in}}%
\pgfpathlineto{\pgfqpoint{0.948751in}{2.320000in}}%
\pgfpathlineto{\pgfqpoint{0.948751in}{0.663635in}}%
\pgfpathclose%
\pgfusepath{fill}%
\end{pgfscope}%
\begin{pgfscope}%
\pgfpathrectangle{\pgfqpoint{0.948751in}{0.663635in}}{\pgfqpoint{4.394942in}{1.656365in}}%
\pgfusepath{clip}%
\pgfsetroundcap%
\pgfsetroundjoin%
\pgfsetlinewidth{1.003750pt}%
\definecolor{currentstroke}{rgb}{1.000000,1.000000,1.000000}%
\pgfsetstrokecolor{currentstroke}%
\pgfsetdash{}{0pt}%
\pgfpathmoveto{\pgfqpoint{1.148521in}{0.663635in}}%
\pgfpathlineto{\pgfqpoint{1.148521in}{2.320000in}}%
\pgfusepath{stroke}%
\end{pgfscope}%
\begin{pgfscope}%
\definecolor{textcolor}{rgb}{0.150000,0.150000,0.150000}%
\pgfsetstrokecolor{textcolor}%
\pgfsetfillcolor{textcolor}%
\pgftext[x=1.148521in,y=0.531691in,,top]{\color{textcolor}{\sffamily\fontsize{11.000000}{13.200000}\selectfont\catcode`\^=\active\def^{\ifmmode\sp\else\^{}\fi}\catcode`\%=\active\def%{\%}0}}%
\end{pgfscope}%
\begin{pgfscope}%
\pgfpathrectangle{\pgfqpoint{0.948751in}{0.663635in}}{\pgfqpoint{4.394942in}{1.656365in}}%
\pgfusepath{clip}%
\pgfsetroundcap%
\pgfsetroundjoin%
\pgfsetlinewidth{1.003750pt}%
\definecolor{currentstroke}{rgb}{1.000000,1.000000,1.000000}%
\pgfsetstrokecolor{currentstroke}%
\pgfsetdash{}{0pt}%
\pgfpathmoveto{\pgfqpoint{1.740433in}{0.663635in}}%
\pgfpathlineto{\pgfqpoint{1.740433in}{2.320000in}}%
\pgfusepath{stroke}%
\end{pgfscope}%
\begin{pgfscope}%
\definecolor{textcolor}{rgb}{0.150000,0.150000,0.150000}%
\pgfsetstrokecolor{textcolor}%
\pgfsetfillcolor{textcolor}%
\pgftext[x=1.740433in,y=0.531691in,,top]{\color{textcolor}{\sffamily\fontsize{11.000000}{13.200000}\selectfont\catcode`\^=\active\def^{\ifmmode\sp\else\^{}\fi}\catcode`\%=\active\def%{\%}20}}%
\end{pgfscope}%
\begin{pgfscope}%
\pgfpathrectangle{\pgfqpoint{0.948751in}{0.663635in}}{\pgfqpoint{4.394942in}{1.656365in}}%
\pgfusepath{clip}%
\pgfsetroundcap%
\pgfsetroundjoin%
\pgfsetlinewidth{1.003750pt}%
\definecolor{currentstroke}{rgb}{1.000000,1.000000,1.000000}%
\pgfsetstrokecolor{currentstroke}%
\pgfsetdash{}{0pt}%
\pgfpathmoveto{\pgfqpoint{2.332344in}{0.663635in}}%
\pgfpathlineto{\pgfqpoint{2.332344in}{2.320000in}}%
\pgfusepath{stroke}%
\end{pgfscope}%
\begin{pgfscope}%
\definecolor{textcolor}{rgb}{0.150000,0.150000,0.150000}%
\pgfsetstrokecolor{textcolor}%
\pgfsetfillcolor{textcolor}%
\pgftext[x=2.332344in,y=0.531691in,,top]{\color{textcolor}{\sffamily\fontsize{11.000000}{13.200000}\selectfont\catcode`\^=\active\def^{\ifmmode\sp\else\^{}\fi}\catcode`\%=\active\def%{\%}40}}%
\end{pgfscope}%
\begin{pgfscope}%
\pgfpathrectangle{\pgfqpoint{0.948751in}{0.663635in}}{\pgfqpoint{4.394942in}{1.656365in}}%
\pgfusepath{clip}%
\pgfsetroundcap%
\pgfsetroundjoin%
\pgfsetlinewidth{1.003750pt}%
\definecolor{currentstroke}{rgb}{1.000000,1.000000,1.000000}%
\pgfsetstrokecolor{currentstroke}%
\pgfsetdash{}{0pt}%
\pgfpathmoveto{\pgfqpoint{2.924255in}{0.663635in}}%
\pgfpathlineto{\pgfqpoint{2.924255in}{2.320000in}}%
\pgfusepath{stroke}%
\end{pgfscope}%
\begin{pgfscope}%
\definecolor{textcolor}{rgb}{0.150000,0.150000,0.150000}%
\pgfsetstrokecolor{textcolor}%
\pgfsetfillcolor{textcolor}%
\pgftext[x=2.924255in,y=0.531691in,,top]{\color{textcolor}{\sffamily\fontsize{11.000000}{13.200000}\selectfont\catcode`\^=\active\def^{\ifmmode\sp\else\^{}\fi}\catcode`\%=\active\def%{\%}60}}%
\end{pgfscope}%
\begin{pgfscope}%
\pgfpathrectangle{\pgfqpoint{0.948751in}{0.663635in}}{\pgfqpoint{4.394942in}{1.656365in}}%
\pgfusepath{clip}%
\pgfsetroundcap%
\pgfsetroundjoin%
\pgfsetlinewidth{1.003750pt}%
\definecolor{currentstroke}{rgb}{1.000000,1.000000,1.000000}%
\pgfsetstrokecolor{currentstroke}%
\pgfsetdash{}{0pt}%
\pgfpathmoveto{\pgfqpoint{3.516167in}{0.663635in}}%
\pgfpathlineto{\pgfqpoint{3.516167in}{2.320000in}}%
\pgfusepath{stroke}%
\end{pgfscope}%
\begin{pgfscope}%
\definecolor{textcolor}{rgb}{0.150000,0.150000,0.150000}%
\pgfsetstrokecolor{textcolor}%
\pgfsetfillcolor{textcolor}%
\pgftext[x=3.516167in,y=0.531691in,,top]{\color{textcolor}{\sffamily\fontsize{11.000000}{13.200000}\selectfont\catcode`\^=\active\def^{\ifmmode\sp\else\^{}\fi}\catcode`\%=\active\def%{\%}80}}%
\end{pgfscope}%
\begin{pgfscope}%
\pgfpathrectangle{\pgfqpoint{0.948751in}{0.663635in}}{\pgfqpoint{4.394942in}{1.656365in}}%
\pgfusepath{clip}%
\pgfsetroundcap%
\pgfsetroundjoin%
\pgfsetlinewidth{1.003750pt}%
\definecolor{currentstroke}{rgb}{1.000000,1.000000,1.000000}%
\pgfsetstrokecolor{currentstroke}%
\pgfsetdash{}{0pt}%
\pgfpathmoveto{\pgfqpoint{4.108078in}{0.663635in}}%
\pgfpathlineto{\pgfqpoint{4.108078in}{2.320000in}}%
\pgfusepath{stroke}%
\end{pgfscope}%
\begin{pgfscope}%
\definecolor{textcolor}{rgb}{0.150000,0.150000,0.150000}%
\pgfsetstrokecolor{textcolor}%
\pgfsetfillcolor{textcolor}%
\pgftext[x=4.108078in,y=0.531691in,,top]{\color{textcolor}{\sffamily\fontsize{11.000000}{13.200000}\selectfont\catcode`\^=\active\def^{\ifmmode\sp\else\^{}\fi}\catcode`\%=\active\def%{\%}100}}%
\end{pgfscope}%
\begin{pgfscope}%
\pgfpathrectangle{\pgfqpoint{0.948751in}{0.663635in}}{\pgfqpoint{4.394942in}{1.656365in}}%
\pgfusepath{clip}%
\pgfsetroundcap%
\pgfsetroundjoin%
\pgfsetlinewidth{1.003750pt}%
\definecolor{currentstroke}{rgb}{1.000000,1.000000,1.000000}%
\pgfsetstrokecolor{currentstroke}%
\pgfsetdash{}{0pt}%
\pgfpathmoveto{\pgfqpoint{4.699990in}{0.663635in}}%
\pgfpathlineto{\pgfqpoint{4.699990in}{2.320000in}}%
\pgfusepath{stroke}%
\end{pgfscope}%
\begin{pgfscope}%
\definecolor{textcolor}{rgb}{0.150000,0.150000,0.150000}%
\pgfsetstrokecolor{textcolor}%
\pgfsetfillcolor{textcolor}%
\pgftext[x=4.699990in,y=0.531691in,,top]{\color{textcolor}{\sffamily\fontsize{11.000000}{13.200000}\selectfont\catcode`\^=\active\def^{\ifmmode\sp\else\^{}\fi}\catcode`\%=\active\def%{\%}120}}%
\end{pgfscope}%
\begin{pgfscope}%
\pgfpathrectangle{\pgfqpoint{0.948751in}{0.663635in}}{\pgfqpoint{4.394942in}{1.656365in}}%
\pgfusepath{clip}%
\pgfsetroundcap%
\pgfsetroundjoin%
\pgfsetlinewidth{1.003750pt}%
\definecolor{currentstroke}{rgb}{1.000000,1.000000,1.000000}%
\pgfsetstrokecolor{currentstroke}%
\pgfsetdash{}{0pt}%
\pgfpathmoveto{\pgfqpoint{5.291901in}{0.663635in}}%
\pgfpathlineto{\pgfqpoint{5.291901in}{2.320000in}}%
\pgfusepath{stroke}%
\end{pgfscope}%
\begin{pgfscope}%
\definecolor{textcolor}{rgb}{0.150000,0.150000,0.150000}%
\pgfsetstrokecolor{textcolor}%
\pgfsetfillcolor{textcolor}%
\pgftext[x=5.291901in,y=0.531691in,,top]{\color{textcolor}{\sffamily\fontsize{11.000000}{13.200000}\selectfont\catcode`\^=\active\def^{\ifmmode\sp\else\^{}\fi}\catcode`\%=\active\def%{\%}140}}%
\end{pgfscope}%
\begin{pgfscope}%
\definecolor{textcolor}{rgb}{0.150000,0.150000,0.150000}%
\pgfsetstrokecolor{textcolor}%
\pgfsetfillcolor{textcolor}%
\pgftext[x=3.146222in,y=0.336413in,,top]{\color{textcolor}{\sffamily\fontsize{12.000000}{14.400000}\selectfont\catcode`\^=\active\def^{\ifmmode\sp\else\^{}\fi}\catcode`\%=\active\def%{\%}Time (s)}}%
\end{pgfscope}%
\begin{pgfscope}%
\pgfpathrectangle{\pgfqpoint{0.948751in}{0.663635in}}{\pgfqpoint{4.394942in}{1.656365in}}%
\pgfusepath{clip}%
\pgfsetroundcap%
\pgfsetroundjoin%
\pgfsetlinewidth{1.003750pt}%
\definecolor{currentstroke}{rgb}{1.000000,1.000000,1.000000}%
\pgfsetstrokecolor{currentstroke}%
\pgfsetdash{}{0pt}%
\pgfpathmoveto{\pgfqpoint{0.948751in}{0.738925in}}%
\pgfpathlineto{\pgfqpoint{5.343693in}{0.738925in}}%
\pgfusepath{stroke}%
\end{pgfscope}%
\begin{pgfscope}%
\definecolor{textcolor}{rgb}{0.150000,0.150000,0.150000}%
\pgfsetstrokecolor{textcolor}%
\pgfsetfillcolor{textcolor}%
\pgftext[x=0.731839in, y=0.684244in, left, base]{\color{textcolor}{\sffamily\fontsize{11.000000}{13.200000}\selectfont\catcode`\^=\active\def^{\ifmmode\sp\else\^{}\fi}\catcode`\%=\active\def%{\%}0}}%
\end{pgfscope}%
\begin{pgfscope}%
\pgfpathrectangle{\pgfqpoint{0.948751in}{0.663635in}}{\pgfqpoint{4.394942in}{1.656365in}}%
\pgfusepath{clip}%
\pgfsetroundcap%
\pgfsetroundjoin%
\pgfsetlinewidth{1.003750pt}%
\definecolor{currentstroke}{rgb}{1.000000,1.000000,1.000000}%
\pgfsetstrokecolor{currentstroke}%
\pgfsetdash{}{0pt}%
\pgfpathmoveto{\pgfqpoint{0.948751in}{1.247104in}}%
\pgfpathlineto{\pgfqpoint{5.343693in}{1.247104in}}%
\pgfusepath{stroke}%
\end{pgfscope}%
\begin{pgfscope}%
\definecolor{textcolor}{rgb}{0.150000,0.150000,0.150000}%
\pgfsetstrokecolor{textcolor}%
\pgfsetfillcolor{textcolor}%
\pgftext[x=0.391968in, y=1.192423in, left, base]{\color{textcolor}{\sffamily\fontsize{11.000000}{13.200000}\selectfont\catcode`\^=\active\def^{\ifmmode\sp\else\^{}\fi}\catcode`\%=\active\def%{\%}10000}}%
\end{pgfscope}%
\begin{pgfscope}%
\pgfpathrectangle{\pgfqpoint{0.948751in}{0.663635in}}{\pgfqpoint{4.394942in}{1.656365in}}%
\pgfusepath{clip}%
\pgfsetroundcap%
\pgfsetroundjoin%
\pgfsetlinewidth{1.003750pt}%
\definecolor{currentstroke}{rgb}{1.000000,1.000000,1.000000}%
\pgfsetstrokecolor{currentstroke}%
\pgfsetdash{}{0pt}%
\pgfpathmoveto{\pgfqpoint{0.948751in}{1.755283in}}%
\pgfpathlineto{\pgfqpoint{5.343693in}{1.755283in}}%
\pgfusepath{stroke}%
\end{pgfscope}%
\begin{pgfscope}%
\definecolor{textcolor}{rgb}{0.150000,0.150000,0.150000}%
\pgfsetstrokecolor{textcolor}%
\pgfsetfillcolor{textcolor}%
\pgftext[x=0.391968in, y=1.700603in, left, base]{\color{textcolor}{\sffamily\fontsize{11.000000}{13.200000}\selectfont\catcode`\^=\active\def^{\ifmmode\sp\else\^{}\fi}\catcode`\%=\active\def%{\%}20000}}%
\end{pgfscope}%
\begin{pgfscope}%
\pgfpathrectangle{\pgfqpoint{0.948751in}{0.663635in}}{\pgfqpoint{4.394942in}{1.656365in}}%
\pgfusepath{clip}%
\pgfsetroundcap%
\pgfsetroundjoin%
\pgfsetlinewidth{1.003750pt}%
\definecolor{currentstroke}{rgb}{1.000000,1.000000,1.000000}%
\pgfsetstrokecolor{currentstroke}%
\pgfsetdash{}{0pt}%
\pgfpathmoveto{\pgfqpoint{0.948751in}{2.263463in}}%
\pgfpathlineto{\pgfqpoint{5.343693in}{2.263463in}}%
\pgfusepath{stroke}%
\end{pgfscope}%
\begin{pgfscope}%
\definecolor{textcolor}{rgb}{0.150000,0.150000,0.150000}%
\pgfsetstrokecolor{textcolor}%
\pgfsetfillcolor{textcolor}%
\pgftext[x=0.391968in, y=2.208782in, left, base]{\color{textcolor}{\sffamily\fontsize{11.000000}{13.200000}\selectfont\catcode`\^=\active\def^{\ifmmode\sp\else\^{}\fi}\catcode`\%=\active\def%{\%}30000}}%
\end{pgfscope}%
\begin{pgfscope}%
\definecolor{textcolor}{rgb}{0.150000,0.150000,0.150000}%
\pgfsetstrokecolor{textcolor}%
\pgfsetfillcolor{textcolor}%
\pgftext[x=0.336413in,y=1.491818in,,bottom,rotate=90.000000]{\color{textcolor}{\sffamily\fontsize{12.000000}{14.400000}\selectfont\catcode`\^=\active\def^{\ifmmode\sp\else\^{}\fi}\catcode`\%=\active\def%{\%}Writes (op/s)}}%
\end{pgfscope}%
\begin{pgfscope}%
\pgfpathrectangle{\pgfqpoint{0.948751in}{0.663635in}}{\pgfqpoint{4.394942in}{1.656365in}}%
\pgfusepath{clip}%
\pgfsetroundcap%
\pgfsetroundjoin%
\pgfsetlinewidth{1.505625pt}%
\definecolor{currentstroke}{rgb}{0.298039,0.447059,0.690196}%
\pgfsetstrokecolor{currentstroke}%
\pgfsetdash{}{0pt}%
\pgfpathmoveto{\pgfqpoint{1.148521in}{0.738925in}}%
\pgfpathlineto{\pgfqpoint{1.296499in}{0.738925in}}%
\pgfpathlineto{\pgfqpoint{1.444477in}{0.738925in}}%
\pgfpathlineto{\pgfqpoint{1.592455in}{0.741339in}}%
\pgfpathlineto{\pgfqpoint{1.740433in}{0.743752in}}%
\pgfpathlineto{\pgfqpoint{1.888411in}{0.743752in}}%
\pgfpathlineto{\pgfqpoint{2.036388in}{1.208177in}}%
\pgfpathlineto{\pgfqpoint{2.184366in}{1.672602in}}%
\pgfpathlineto{\pgfqpoint{2.332344in}{1.672602in}}%
\pgfpathlineto{\pgfqpoint{2.480322in}{1.902350in}}%
\pgfpathlineto{\pgfqpoint{2.628300in}{2.132098in}}%
\pgfpathlineto{\pgfqpoint{2.776278in}{2.132098in}}%
\pgfpathlineto{\pgfqpoint{2.924255in}{2.188404in}}%
\pgfpathlineto{\pgfqpoint{3.072233in}{2.244711in}}%
\pgfpathlineto{\pgfqpoint{3.220211in}{2.244711in}}%
\pgfpathlineto{\pgfqpoint{3.368189in}{2.192317in}}%
\pgfpathlineto{\pgfqpoint{3.516167in}{2.089055in}}%
\pgfpathlineto{\pgfqpoint{3.664145in}{2.089055in}}%
\pgfpathlineto{\pgfqpoint{3.812123in}{2.026905in}}%
\pgfpathlineto{\pgfqpoint{3.960100in}{1.659796in}}%
\pgfpathlineto{\pgfqpoint{4.108078in}{1.507343in}}%
\pgfpathlineto{\pgfqpoint{4.256056in}{1.194203in}}%
\pgfpathlineto{\pgfqpoint{4.404034in}{0.881062in}}%
\pgfpathlineto{\pgfqpoint{4.552012in}{0.881062in}}%
\pgfpathlineto{\pgfqpoint{4.699990in}{0.809968in}}%
\pgfpathlineto{\pgfqpoint{4.847967in}{0.738925in}}%
\pgfpathlineto{\pgfqpoint{4.995945in}{0.738925in}}%
\pgfpathlineto{\pgfqpoint{5.143923in}{0.738925in}}%
\pgfusepath{stroke}%
\end{pgfscope}%
\begin{pgfscope}%
\pgfpathrectangle{\pgfqpoint{0.948751in}{0.663635in}}{\pgfqpoint{4.394942in}{1.656365in}}%
\pgfusepath{clip}%
\pgfsetroundcap%
\pgfsetroundjoin%
\pgfsetlinewidth{1.505625pt}%
\definecolor{currentstroke}{rgb}{1.000000,0.000000,0.000000}%
\pgfsetstrokecolor{currentstroke}%
\pgfsetdash{}{0pt}%
\pgfpathmoveto{\pgfqpoint{0.948751in}{1.406818in}}%
\pgfpathlineto{\pgfqpoint{5.343693in}{1.406818in}}%
\pgfusepath{stroke}%
\end{pgfscope}%
\begin{pgfscope}%
\pgfsetrectcap%
\pgfsetmiterjoin%
\pgfsetlinewidth{1.254687pt}%
\definecolor{currentstroke}{rgb}{1.000000,1.000000,1.000000}%
\pgfsetstrokecolor{currentstroke}%
\pgfsetdash{}{0pt}%
\pgfpathmoveto{\pgfqpoint{0.948751in}{0.663635in}}%
\pgfpathlineto{\pgfqpoint{0.948751in}{2.320000in}}%
\pgfusepath{stroke}%
\end{pgfscope}%
\begin{pgfscope}%
\pgfsetrectcap%
\pgfsetmiterjoin%
\pgfsetlinewidth{1.254687pt}%
\definecolor{currentstroke}{rgb}{1.000000,1.000000,1.000000}%
\pgfsetstrokecolor{currentstroke}%
\pgfsetdash{}{0pt}%
\pgfpathmoveto{\pgfqpoint{5.343693in}{0.663635in}}%
\pgfpathlineto{\pgfqpoint{5.343693in}{2.320000in}}%
\pgfusepath{stroke}%
\end{pgfscope}%
\begin{pgfscope}%
\pgfsetrectcap%
\pgfsetmiterjoin%
\pgfsetlinewidth{1.254687pt}%
\definecolor{currentstroke}{rgb}{1.000000,1.000000,1.000000}%
\pgfsetstrokecolor{currentstroke}%
\pgfsetdash{}{0pt}%
\pgfpathmoveto{\pgfqpoint{0.948751in}{0.663635in}}%
\pgfpathlineto{\pgfqpoint{5.343693in}{0.663635in}}%
\pgfusepath{stroke}%
\end{pgfscope}%
\begin{pgfscope}%
\pgfsetrectcap%
\pgfsetmiterjoin%
\pgfsetlinewidth{1.254687pt}%
\definecolor{currentstroke}{rgb}{1.000000,1.000000,1.000000}%
\pgfsetstrokecolor{currentstroke}%
\pgfsetdash{}{0pt}%
\pgfpathmoveto{\pgfqpoint{0.948751in}{2.320000in}}%
\pgfpathlineto{\pgfqpoint{5.343693in}{2.320000in}}%
\pgfusepath{stroke}%
\end{pgfscope}%
\begin{pgfscope}%
\pgfsetbuttcap%
\pgfsetmiterjoin%
\definecolor{currentfill}{rgb}{0.917647,0.917647,0.949020}%
\pgfsetfillcolor{currentfill}%
\pgfsetfillopacity{0.800000}%
\pgfsetlinewidth{1.003750pt}%
\definecolor{currentstroke}{rgb}{0.800000,0.800000,0.800000}%
\pgfsetstrokecolor{currentstroke}%
\pgfsetstrokeopacity{0.800000}%
\pgfsetdash{}{0pt}%
\pgfpathmoveto{\pgfqpoint{2.069763in}{0.719191in}}%
\pgfpathlineto{\pgfqpoint{4.222682in}{0.719191in}}%
\pgfpathquadraticcurveto{\pgfqpoint{4.244904in}{0.719191in}}{\pgfqpoint{4.244904in}{0.741413in}}%
\pgfpathlineto{\pgfqpoint{4.244904in}{0.888776in}}%
\pgfpathquadraticcurveto{\pgfqpoint{4.244904in}{0.910998in}}{\pgfqpoint{4.222682in}{0.910998in}}%
\pgfpathlineto{\pgfqpoint{2.069763in}{0.910998in}}%
\pgfpathquadraticcurveto{\pgfqpoint{2.047541in}{0.910998in}}{\pgfqpoint{2.047541in}{0.888776in}}%
\pgfpathlineto{\pgfqpoint{2.047541in}{0.741413in}}%
\pgfpathquadraticcurveto{\pgfqpoint{2.047541in}{0.719191in}}{\pgfqpoint{2.069763in}{0.719191in}}%
\pgfpathlineto{\pgfqpoint{2.069763in}{0.719191in}}%
\pgfpathclose%
\pgfusepath{stroke,fill}%
\end{pgfscope}%
\begin{pgfscope}%
\pgfsetroundcap%
\pgfsetroundjoin%
\pgfsetlinewidth{1.505625pt}%
\definecolor{currentstroke}{rgb}{1.000000,0.000000,0.000000}%
\pgfsetstrokecolor{currentstroke}%
\pgfsetdash{}{0pt}%
\pgfpathmoveto{\pgfqpoint{2.091985in}{0.825907in}}%
\pgfpathlineto{\pgfqpoint{2.203096in}{0.825907in}}%
\pgfpathlineto{\pgfqpoint{2.314207in}{0.825907in}}%
\pgfusepath{stroke}%
\end{pgfscope}%
\begin{pgfscope}%
\definecolor{textcolor}{rgb}{0.150000,0.150000,0.150000}%
\pgfsetstrokecolor{textcolor}%
\pgfsetfillcolor{textcolor}%
\pgftext[x=2.403096in,y=0.787018in,left,base]{\color{textcolor}{\sffamily\fontsize{8.000000}{9.600000}\selectfont\catcode`\^=\active\def^{\ifmmode\sp\else\^{}\fi}\catcode`\%=\active\def%{\%}average write operations per second}}%
\end{pgfscope}%
\end{pgfpicture}%
\makeatother%
\endgroup%

    \caption{Stress test of 2 nodes with 1000000 writes}
    \label{fig:stress-1000000writes-2node}
\end{figure}

\begin{figure}
    \centering
    %% Creator: Matplotlib, PGF backend
%%
%% To include the figure in your LaTeX document, write
%%   \input{<filename>.pgf}
%%
%% Make sure the required packages are loaded in your preamble
%%   \usepackage{pgf}
%%
%% Also ensure that all the required font packages are loaded; for instance,
%% the lmodern package is sometimes necessary when using math font.
%%   \usepackage{lmodern}
%%
%% Figures using additional raster images can only be included by \input if
%% they are in the same directory as the main LaTeX file. For loading figures
%% from other directories you can use the `import` package
%%   \usepackage{import}
%%
%% and then include the figures with
%%   \import{<path to file>}{<filename>.pgf}
%%
%% Matplotlib used the following preamble
%%   \def\mathdefault#1{#1}
%%   \everymath=\expandafter{\the\everymath\displaystyle}
%%   
%%   \usepackage{fontspec}
%%   \setmainfont{DejaVuSerif.ttf}[Path=\detokenize{/Users/nkratky/private/polaris-elasticity-strategies/test/scripts/.venv/lib/python3.11/site-packages/matplotlib/mpl-data/fonts/ttf/}]
%%   \setsansfont{Arial.ttf}[Path=\detokenize{/System/Library/Fonts/Supplemental/}]
%%   \setmonofont{DejaVuSansMono.ttf}[Path=\detokenize{/Users/nkratky/private/polaris-elasticity-strategies/test/scripts/.venv/lib/python3.11/site-packages/matplotlib/mpl-data/fonts/ttf/}]
%%   \makeatletter\@ifpackageloaded{underscore}{}{\usepackage[strings]{underscore}}\makeatother
%%
\begingroup%
\makeatletter%
\begin{pgfpicture}%
\pgfpathrectangle{\pgfpointorigin}{\pgfqpoint{5.600000in}{2.500000in}}%
\pgfusepath{use as bounding box, clip}%
\begin{pgfscope}%
\pgfsetbuttcap%
\pgfsetmiterjoin%
\definecolor{currentfill}{rgb}{1.000000,1.000000,1.000000}%
\pgfsetfillcolor{currentfill}%
\pgfsetlinewidth{0.000000pt}%
\definecolor{currentstroke}{rgb}{1.000000,1.000000,1.000000}%
\pgfsetstrokecolor{currentstroke}%
\pgfsetdash{}{0pt}%
\pgfpathmoveto{\pgfqpoint{0.000000in}{0.000000in}}%
\pgfpathlineto{\pgfqpoint{5.600000in}{0.000000in}}%
\pgfpathlineto{\pgfqpoint{5.600000in}{2.500000in}}%
\pgfpathlineto{\pgfqpoint{0.000000in}{2.500000in}}%
\pgfpathlineto{\pgfqpoint{0.000000in}{0.000000in}}%
\pgfpathclose%
\pgfusepath{fill}%
\end{pgfscope}%
\begin{pgfscope}%
\pgfsetbuttcap%
\pgfsetmiterjoin%
\definecolor{currentfill}{rgb}{0.917647,0.917647,0.949020}%
\pgfsetfillcolor{currentfill}%
\pgfsetlinewidth{0.000000pt}%
\definecolor{currentstroke}{rgb}{0.000000,0.000000,0.000000}%
\pgfsetstrokecolor{currentstroke}%
\pgfsetstrokeopacity{0.000000}%
\pgfsetdash{}{0pt}%
\pgfpathmoveto{\pgfqpoint{0.948751in}{0.663635in}}%
\pgfpathlineto{\pgfqpoint{5.420000in}{0.663635in}}%
\pgfpathlineto{\pgfqpoint{5.420000in}{2.320000in}}%
\pgfpathlineto{\pgfqpoint{0.948751in}{2.320000in}}%
\pgfpathlineto{\pgfqpoint{0.948751in}{0.663635in}}%
\pgfpathclose%
\pgfusepath{fill}%
\end{pgfscope}%
\begin{pgfscope}%
\pgfpathrectangle{\pgfqpoint{0.948751in}{0.663635in}}{\pgfqpoint{4.471249in}{1.656365in}}%
\pgfusepath{clip}%
\pgfsetroundcap%
\pgfsetroundjoin%
\pgfsetlinewidth{1.003750pt}%
\definecolor{currentstroke}{rgb}{1.000000,1.000000,1.000000}%
\pgfsetstrokecolor{currentstroke}%
\pgfsetdash{}{0pt}%
\pgfpathmoveto{\pgfqpoint{1.151990in}{0.663635in}}%
\pgfpathlineto{\pgfqpoint{1.151990in}{2.320000in}}%
\pgfusepath{stroke}%
\end{pgfscope}%
\begin{pgfscope}%
\definecolor{textcolor}{rgb}{0.150000,0.150000,0.150000}%
\pgfsetstrokecolor{textcolor}%
\pgfsetfillcolor{textcolor}%
\pgftext[x=1.151990in,y=0.531691in,,top]{\color{textcolor}{\sffamily\fontsize{11.000000}{13.200000}\selectfont\catcode`\^=\active\def^{\ifmmode\sp\else\^{}\fi}\catcode`\%=\active\def%{\%}0}}%
\end{pgfscope}%
\begin{pgfscope}%
\pgfpathrectangle{\pgfqpoint{0.948751in}{0.663635in}}{\pgfqpoint{4.471249in}{1.656365in}}%
\pgfusepath{clip}%
\pgfsetroundcap%
\pgfsetroundjoin%
\pgfsetlinewidth{1.003750pt}%
\definecolor{currentstroke}{rgb}{1.000000,1.000000,1.000000}%
\pgfsetstrokecolor{currentstroke}%
\pgfsetdash{}{0pt}%
\pgfpathmoveto{\pgfqpoint{1.926232in}{0.663635in}}%
\pgfpathlineto{\pgfqpoint{1.926232in}{2.320000in}}%
\pgfusepath{stroke}%
\end{pgfscope}%
\begin{pgfscope}%
\definecolor{textcolor}{rgb}{0.150000,0.150000,0.150000}%
\pgfsetstrokecolor{textcolor}%
\pgfsetfillcolor{textcolor}%
\pgftext[x=1.926232in,y=0.531691in,,top]{\color{textcolor}{\sffamily\fontsize{11.000000}{13.200000}\selectfont\catcode`\^=\active\def^{\ifmmode\sp\else\^{}\fi}\catcode`\%=\active\def%{\%}20}}%
\end{pgfscope}%
\begin{pgfscope}%
\pgfpathrectangle{\pgfqpoint{0.948751in}{0.663635in}}{\pgfqpoint{4.471249in}{1.656365in}}%
\pgfusepath{clip}%
\pgfsetroundcap%
\pgfsetroundjoin%
\pgfsetlinewidth{1.003750pt}%
\definecolor{currentstroke}{rgb}{1.000000,1.000000,1.000000}%
\pgfsetstrokecolor{currentstroke}%
\pgfsetdash{}{0pt}%
\pgfpathmoveto{\pgfqpoint{2.700474in}{0.663635in}}%
\pgfpathlineto{\pgfqpoint{2.700474in}{2.320000in}}%
\pgfusepath{stroke}%
\end{pgfscope}%
\begin{pgfscope}%
\definecolor{textcolor}{rgb}{0.150000,0.150000,0.150000}%
\pgfsetstrokecolor{textcolor}%
\pgfsetfillcolor{textcolor}%
\pgftext[x=2.700474in,y=0.531691in,,top]{\color{textcolor}{\sffamily\fontsize{11.000000}{13.200000}\selectfont\catcode`\^=\active\def^{\ifmmode\sp\else\^{}\fi}\catcode`\%=\active\def%{\%}40}}%
\end{pgfscope}%
\begin{pgfscope}%
\pgfpathrectangle{\pgfqpoint{0.948751in}{0.663635in}}{\pgfqpoint{4.471249in}{1.656365in}}%
\pgfusepath{clip}%
\pgfsetroundcap%
\pgfsetroundjoin%
\pgfsetlinewidth{1.003750pt}%
\definecolor{currentstroke}{rgb}{1.000000,1.000000,1.000000}%
\pgfsetstrokecolor{currentstroke}%
\pgfsetdash{}{0pt}%
\pgfpathmoveto{\pgfqpoint{3.474716in}{0.663635in}}%
\pgfpathlineto{\pgfqpoint{3.474716in}{2.320000in}}%
\pgfusepath{stroke}%
\end{pgfscope}%
\begin{pgfscope}%
\definecolor{textcolor}{rgb}{0.150000,0.150000,0.150000}%
\pgfsetstrokecolor{textcolor}%
\pgfsetfillcolor{textcolor}%
\pgftext[x=3.474716in,y=0.531691in,,top]{\color{textcolor}{\sffamily\fontsize{11.000000}{13.200000}\selectfont\catcode`\^=\active\def^{\ifmmode\sp\else\^{}\fi}\catcode`\%=\active\def%{\%}60}}%
\end{pgfscope}%
\begin{pgfscope}%
\pgfpathrectangle{\pgfqpoint{0.948751in}{0.663635in}}{\pgfqpoint{4.471249in}{1.656365in}}%
\pgfusepath{clip}%
\pgfsetroundcap%
\pgfsetroundjoin%
\pgfsetlinewidth{1.003750pt}%
\definecolor{currentstroke}{rgb}{1.000000,1.000000,1.000000}%
\pgfsetstrokecolor{currentstroke}%
\pgfsetdash{}{0pt}%
\pgfpathmoveto{\pgfqpoint{4.248959in}{0.663635in}}%
\pgfpathlineto{\pgfqpoint{4.248959in}{2.320000in}}%
\pgfusepath{stroke}%
\end{pgfscope}%
\begin{pgfscope}%
\definecolor{textcolor}{rgb}{0.150000,0.150000,0.150000}%
\pgfsetstrokecolor{textcolor}%
\pgfsetfillcolor{textcolor}%
\pgftext[x=4.248959in,y=0.531691in,,top]{\color{textcolor}{\sffamily\fontsize{11.000000}{13.200000}\selectfont\catcode`\^=\active\def^{\ifmmode\sp\else\^{}\fi}\catcode`\%=\active\def%{\%}80}}%
\end{pgfscope}%
\begin{pgfscope}%
\pgfpathrectangle{\pgfqpoint{0.948751in}{0.663635in}}{\pgfqpoint{4.471249in}{1.656365in}}%
\pgfusepath{clip}%
\pgfsetroundcap%
\pgfsetroundjoin%
\pgfsetlinewidth{1.003750pt}%
\definecolor{currentstroke}{rgb}{1.000000,1.000000,1.000000}%
\pgfsetstrokecolor{currentstroke}%
\pgfsetdash{}{0pt}%
\pgfpathmoveto{\pgfqpoint{5.023201in}{0.663635in}}%
\pgfpathlineto{\pgfqpoint{5.023201in}{2.320000in}}%
\pgfusepath{stroke}%
\end{pgfscope}%
\begin{pgfscope}%
\definecolor{textcolor}{rgb}{0.150000,0.150000,0.150000}%
\pgfsetstrokecolor{textcolor}%
\pgfsetfillcolor{textcolor}%
\pgftext[x=5.023201in,y=0.531691in,,top]{\color{textcolor}{\sffamily\fontsize{11.000000}{13.200000}\selectfont\catcode`\^=\active\def^{\ifmmode\sp\else\^{}\fi}\catcode`\%=\active\def%{\%}100}}%
\end{pgfscope}%
\begin{pgfscope}%
\definecolor{textcolor}{rgb}{0.150000,0.150000,0.150000}%
\pgfsetstrokecolor{textcolor}%
\pgfsetfillcolor{textcolor}%
\pgftext[x=3.184376in,y=0.336413in,,top]{\color{textcolor}{\sffamily\fontsize{12.000000}{14.400000}\selectfont\catcode`\^=\active\def^{\ifmmode\sp\else\^{}\fi}\catcode`\%=\active\def%{\%}Time (s)}}%
\end{pgfscope}%
\begin{pgfscope}%
\pgfpathrectangle{\pgfqpoint{0.948751in}{0.663635in}}{\pgfqpoint{4.471249in}{1.656365in}}%
\pgfusepath{clip}%
\pgfsetroundcap%
\pgfsetroundjoin%
\pgfsetlinewidth{1.003750pt}%
\definecolor{currentstroke}{rgb}{1.000000,1.000000,1.000000}%
\pgfsetstrokecolor{currentstroke}%
\pgfsetdash{}{0pt}%
\pgfpathmoveto{\pgfqpoint{0.948751in}{0.738925in}}%
\pgfpathlineto{\pgfqpoint{5.420000in}{0.738925in}}%
\pgfusepath{stroke}%
\end{pgfscope}%
\begin{pgfscope}%
\definecolor{textcolor}{rgb}{0.150000,0.150000,0.150000}%
\pgfsetstrokecolor{textcolor}%
\pgfsetfillcolor{textcolor}%
\pgftext[x=0.731839in, y=0.684244in, left, base]{\color{textcolor}{\sffamily\fontsize{11.000000}{13.200000}\selectfont\catcode`\^=\active\def^{\ifmmode\sp\else\^{}\fi}\catcode`\%=\active\def%{\%}0}}%
\end{pgfscope}%
\begin{pgfscope}%
\pgfpathrectangle{\pgfqpoint{0.948751in}{0.663635in}}{\pgfqpoint{4.471249in}{1.656365in}}%
\pgfusepath{clip}%
\pgfsetroundcap%
\pgfsetroundjoin%
\pgfsetlinewidth{1.003750pt}%
\definecolor{currentstroke}{rgb}{1.000000,1.000000,1.000000}%
\pgfsetstrokecolor{currentstroke}%
\pgfsetdash{}{0pt}%
\pgfpathmoveto{\pgfqpoint{0.948751in}{1.227959in}}%
\pgfpathlineto{\pgfqpoint{5.420000in}{1.227959in}}%
\pgfusepath{stroke}%
\end{pgfscope}%
\begin{pgfscope}%
\definecolor{textcolor}{rgb}{0.150000,0.150000,0.150000}%
\pgfsetstrokecolor{textcolor}%
\pgfsetfillcolor{textcolor}%
\pgftext[x=0.391968in, y=1.173278in, left, base]{\color{textcolor}{\sffamily\fontsize{11.000000}{13.200000}\selectfont\catcode`\^=\active\def^{\ifmmode\sp\else\^{}\fi}\catcode`\%=\active\def%{\%}10000}}%
\end{pgfscope}%
\begin{pgfscope}%
\pgfpathrectangle{\pgfqpoint{0.948751in}{0.663635in}}{\pgfqpoint{4.471249in}{1.656365in}}%
\pgfusepath{clip}%
\pgfsetroundcap%
\pgfsetroundjoin%
\pgfsetlinewidth{1.003750pt}%
\definecolor{currentstroke}{rgb}{1.000000,1.000000,1.000000}%
\pgfsetstrokecolor{currentstroke}%
\pgfsetdash{}{0pt}%
\pgfpathmoveto{\pgfqpoint{0.948751in}{1.716994in}}%
\pgfpathlineto{\pgfqpoint{5.420000in}{1.716994in}}%
\pgfusepath{stroke}%
\end{pgfscope}%
\begin{pgfscope}%
\definecolor{textcolor}{rgb}{0.150000,0.150000,0.150000}%
\pgfsetstrokecolor{textcolor}%
\pgfsetfillcolor{textcolor}%
\pgftext[x=0.391968in, y=1.662313in, left, base]{\color{textcolor}{\sffamily\fontsize{11.000000}{13.200000}\selectfont\catcode`\^=\active\def^{\ifmmode\sp\else\^{}\fi}\catcode`\%=\active\def%{\%}20000}}%
\end{pgfscope}%
\begin{pgfscope}%
\pgfpathrectangle{\pgfqpoint{0.948751in}{0.663635in}}{\pgfqpoint{4.471249in}{1.656365in}}%
\pgfusepath{clip}%
\pgfsetroundcap%
\pgfsetroundjoin%
\pgfsetlinewidth{1.003750pt}%
\definecolor{currentstroke}{rgb}{1.000000,1.000000,1.000000}%
\pgfsetstrokecolor{currentstroke}%
\pgfsetdash{}{0pt}%
\pgfpathmoveto{\pgfqpoint{0.948751in}{2.206028in}}%
\pgfpathlineto{\pgfqpoint{5.420000in}{2.206028in}}%
\pgfusepath{stroke}%
\end{pgfscope}%
\begin{pgfscope}%
\definecolor{textcolor}{rgb}{0.150000,0.150000,0.150000}%
\pgfsetstrokecolor{textcolor}%
\pgfsetfillcolor{textcolor}%
\pgftext[x=0.391968in, y=2.151347in, left, base]{\color{textcolor}{\sffamily\fontsize{11.000000}{13.200000}\selectfont\catcode`\^=\active\def^{\ifmmode\sp\else\^{}\fi}\catcode`\%=\active\def%{\%}30000}}%
\end{pgfscope}%
\begin{pgfscope}%
\definecolor{textcolor}{rgb}{0.150000,0.150000,0.150000}%
\pgfsetstrokecolor{textcolor}%
\pgfsetfillcolor{textcolor}%
\pgftext[x=0.336413in,y=1.491818in,,bottom,rotate=90.000000]{\color{textcolor}{\sffamily\fontsize{12.000000}{14.400000}\selectfont\catcode`\^=\active\def^{\ifmmode\sp\else\^{}\fi}\catcode`\%=\active\def%{\%}Writes (op/s)}}%
\end{pgfscope}%
\begin{pgfscope}%
\pgfpathrectangle{\pgfqpoint{0.948751in}{0.663635in}}{\pgfqpoint{4.471249in}{1.656365in}}%
\pgfusepath{clip}%
\pgfsetroundcap%
\pgfsetroundjoin%
\pgfsetlinewidth{1.505625pt}%
\definecolor{currentstroke}{rgb}{0.298039,0.447059,0.690196}%
\pgfsetstrokecolor{currentstroke}%
\pgfsetdash{}{0pt}%
\pgfpathmoveto{\pgfqpoint{1.151990in}{0.738925in}}%
\pgfpathlineto{\pgfqpoint{1.345550in}{0.738925in}}%
\pgfpathlineto{\pgfqpoint{1.539111in}{0.738925in}}%
\pgfpathlineto{\pgfqpoint{1.732671in}{1.014789in}}%
\pgfpathlineto{\pgfqpoint{1.926232in}{1.290653in}}%
\pgfpathlineto{\pgfqpoint{2.119793in}{1.290653in}}%
\pgfpathlineto{\pgfqpoint{2.313353in}{1.552531in}}%
\pgfpathlineto{\pgfqpoint{2.506914in}{1.814360in}}%
\pgfpathlineto{\pgfqpoint{2.700474in}{1.814360in}}%
\pgfpathlineto{\pgfqpoint{2.894035in}{1.974568in}}%
\pgfpathlineto{\pgfqpoint{3.087595in}{2.134776in}}%
\pgfpathlineto{\pgfqpoint{3.281156in}{2.134776in}}%
\pgfpathlineto{\pgfqpoint{3.474716in}{2.189743in}}%
\pgfpathlineto{\pgfqpoint{3.668277in}{2.244711in}}%
\pgfpathlineto{\pgfqpoint{3.861838in}{2.244711in}}%
\pgfpathlineto{\pgfqpoint{4.055398in}{1.794848in}}%
\pgfpathlineto{\pgfqpoint{4.248959in}{1.344985in}}%
\pgfpathlineto{\pgfqpoint{4.442519in}{1.344985in}}%
\pgfpathlineto{\pgfqpoint{4.636080in}{1.041979in}}%
\pgfpathlineto{\pgfqpoint{4.829640in}{0.738925in}}%
\pgfpathlineto{\pgfqpoint{5.023201in}{0.738925in}}%
\pgfpathlineto{\pgfqpoint{5.216761in}{0.738925in}}%
\pgfusepath{stroke}%
\end{pgfscope}%
\begin{pgfscope}%
\pgfpathrectangle{\pgfqpoint{0.948751in}{0.663635in}}{\pgfqpoint{4.471249in}{1.656365in}}%
\pgfusepath{clip}%
\pgfsetroundcap%
\pgfsetroundjoin%
\pgfsetlinewidth{1.505625pt}%
\definecolor{currentstroke}{rgb}{1.000000,0.000000,0.000000}%
\pgfsetstrokecolor{currentstroke}%
\pgfsetdash{}{0pt}%
\pgfpathmoveto{\pgfqpoint{0.948751in}{1.439135in}}%
\pgfpathlineto{\pgfqpoint{5.420000in}{1.439135in}}%
\pgfusepath{stroke}%
\end{pgfscope}%
\begin{pgfscope}%
\pgfsetrectcap%
\pgfsetmiterjoin%
\pgfsetlinewidth{1.254687pt}%
\definecolor{currentstroke}{rgb}{1.000000,1.000000,1.000000}%
\pgfsetstrokecolor{currentstroke}%
\pgfsetdash{}{0pt}%
\pgfpathmoveto{\pgfqpoint{0.948751in}{0.663635in}}%
\pgfpathlineto{\pgfqpoint{0.948751in}{2.320000in}}%
\pgfusepath{stroke}%
\end{pgfscope}%
\begin{pgfscope}%
\pgfsetrectcap%
\pgfsetmiterjoin%
\pgfsetlinewidth{1.254687pt}%
\definecolor{currentstroke}{rgb}{1.000000,1.000000,1.000000}%
\pgfsetstrokecolor{currentstroke}%
\pgfsetdash{}{0pt}%
\pgfpathmoveto{\pgfqpoint{5.420000in}{0.663635in}}%
\pgfpathlineto{\pgfqpoint{5.420000in}{2.320000in}}%
\pgfusepath{stroke}%
\end{pgfscope}%
\begin{pgfscope}%
\pgfsetrectcap%
\pgfsetmiterjoin%
\pgfsetlinewidth{1.254687pt}%
\definecolor{currentstroke}{rgb}{1.000000,1.000000,1.000000}%
\pgfsetstrokecolor{currentstroke}%
\pgfsetdash{}{0pt}%
\pgfpathmoveto{\pgfqpoint{0.948751in}{0.663635in}}%
\pgfpathlineto{\pgfqpoint{5.420000in}{0.663635in}}%
\pgfusepath{stroke}%
\end{pgfscope}%
\begin{pgfscope}%
\pgfsetrectcap%
\pgfsetmiterjoin%
\pgfsetlinewidth{1.254687pt}%
\definecolor{currentstroke}{rgb}{1.000000,1.000000,1.000000}%
\pgfsetstrokecolor{currentstroke}%
\pgfsetdash{}{0pt}%
\pgfpathmoveto{\pgfqpoint{0.948751in}{2.320000in}}%
\pgfpathlineto{\pgfqpoint{5.420000in}{2.320000in}}%
\pgfusepath{stroke}%
\end{pgfscope}%
\begin{pgfscope}%
\pgfsetbuttcap%
\pgfsetmiterjoin%
\definecolor{currentfill}{rgb}{0.917647,0.917647,0.949020}%
\pgfsetfillcolor{currentfill}%
\pgfsetfillopacity{0.800000}%
\pgfsetlinewidth{1.003750pt}%
\definecolor{currentstroke}{rgb}{0.800000,0.800000,0.800000}%
\pgfsetstrokecolor{currentstroke}%
\pgfsetstrokeopacity{0.800000}%
\pgfsetdash{}{0pt}%
\pgfpathmoveto{\pgfqpoint{2.107916in}{0.719191in}}%
\pgfpathlineto{\pgfqpoint{4.260835in}{0.719191in}}%
\pgfpathquadraticcurveto{\pgfqpoint{4.283057in}{0.719191in}}{\pgfqpoint{4.283057in}{0.741413in}}%
\pgfpathlineto{\pgfqpoint{4.283057in}{0.888776in}}%
\pgfpathquadraticcurveto{\pgfqpoint{4.283057in}{0.910998in}}{\pgfqpoint{4.260835in}{0.910998in}}%
\pgfpathlineto{\pgfqpoint{2.107916in}{0.910998in}}%
\pgfpathquadraticcurveto{\pgfqpoint{2.085694in}{0.910998in}}{\pgfqpoint{2.085694in}{0.888776in}}%
\pgfpathlineto{\pgfqpoint{2.085694in}{0.741413in}}%
\pgfpathquadraticcurveto{\pgfqpoint{2.085694in}{0.719191in}}{\pgfqpoint{2.107916in}{0.719191in}}%
\pgfpathlineto{\pgfqpoint{2.107916in}{0.719191in}}%
\pgfpathclose%
\pgfusepath{stroke,fill}%
\end{pgfscope}%
\begin{pgfscope}%
\pgfsetroundcap%
\pgfsetroundjoin%
\pgfsetlinewidth{1.505625pt}%
\definecolor{currentstroke}{rgb}{1.000000,0.000000,0.000000}%
\pgfsetstrokecolor{currentstroke}%
\pgfsetdash{}{0pt}%
\pgfpathmoveto{\pgfqpoint{2.130138in}{0.825907in}}%
\pgfpathlineto{\pgfqpoint{2.241250in}{0.825907in}}%
\pgfpathlineto{\pgfqpoint{2.352361in}{0.825907in}}%
\pgfusepath{stroke}%
\end{pgfscope}%
\begin{pgfscope}%
\definecolor{textcolor}{rgb}{0.150000,0.150000,0.150000}%
\pgfsetstrokecolor{textcolor}%
\pgfsetfillcolor{textcolor}%
\pgftext[x=2.441250in,y=0.787018in,left,base]{\color{textcolor}{\sffamily\fontsize{8.000000}{9.600000}\selectfont\catcode`\^=\active\def^{\ifmmode\sp\else\^{}\fi}\catcode`\%=\active\def%{\%}average write operations per second}}%
\end{pgfscope}%
\end{pgfpicture}%
\makeatother%
\endgroup%

    \caption{Stress test of 3 nodes with 1000000 writes}
    \label{fig:stress-1000000writes-3node}
\end{figure}

\subsection{Vertical Elasticity Strategy}
\label{sec:evaluation-vertical-elasticity}

As mentioned in \cref{sec:vertical-elasticity} the vertical elasticity strategy adjusts the resource claims of k8ssandra according to its CPU and memory utilization.

As it can bee seen in \cref{fig:simple-limits-vertical} the elasticity strategy controller successfully changes the CPU and memory limits of the k8ssandra cluster once it is operational. \Cref{fig:utilization-vertical} shows the CPU and memory utilization that is used for triggering elasticity processes. Because the CPU utilization stays very low even after scaling takes place, it can be assumed that this metric was not a decisive factor. The memory utilization, however, changes notably. Before starting the elasticity strategy controller the actual memory utilization was off by \(>10\%\) from the target memory utilization. This triggers an elasticity event and the resources are adjusted proportionally.

Interestingly, during reconsiliation the exposed metrics of k8ssandra are not very meaningful. During this process utilization values of far more than 100\% are exposed by the metrics controller. In order to keep the diagram clean, these nonsense-metrics have been filtered out. The reconsiliation process is marked red in \cref{fig:utilization-vertical}.

This elasticity strategy mirrors real-life scenarios. The advantage lies in being able to scale down when demand and therefore CPU and memory utilization is low, thus potentially reducing cost. This obviously only applies when not using dedicated resources.

\begin{figure}
    \centering
    %% Creator: Matplotlib, PGF backend
%%
%% To include the figure in your LaTeX document, write
%%   \input{<filename>.pgf}
%%
%% Make sure the required packages are loaded in your preamble
%%   \usepackage{pgf}
%%
%% Also ensure that all the required font packages are loaded; for instance,
%% the lmodern package is sometimes necessary when using math font.
%%   \usepackage{lmodern}
%%
%% Figures using additional raster images can only be included by \input if
%% they are in the same directory as the main LaTeX file. For loading figures
%% from other directories you can use the `import` package
%%   \usepackage{import}
%%
%% and then include the figures with
%%   \import{<path to file>}{<filename>.pgf}
%%
%% Matplotlib used the following preamble
%%   \def\mathdefault#1{#1}
%%   \everymath=\expandafter{\the\everymath\displaystyle}
%%   
%%   \usepackage{fontspec}
%%   \setmainfont{DejaVuSerif.ttf}[Path=\detokenize{/Users/nkratky/private/polaris-elasticity-strategies/test/scripts/.venv/lib/python3.11/site-packages/matplotlib/mpl-data/fonts/ttf/}]
%%   \setsansfont{Arial.ttf}[Path=\detokenize{/System/Library/Fonts/Supplemental/}]
%%   \setmonofont{DejaVuSansMono.ttf}[Path=\detokenize{/Users/nkratky/private/polaris-elasticity-strategies/test/scripts/.venv/lib/python3.11/site-packages/matplotlib/mpl-data/fonts/ttf/}]
%%   \makeatletter\@ifpackageloaded{underscore}{}{\usepackage[strings]{underscore}}\makeatother
%%
\begingroup%
\makeatletter%
\begin{pgfpicture}%
\pgfpathrectangle{\pgfpointorigin}{\pgfqpoint{5.600000in}{4.000000in}}%
\pgfusepath{use as bounding box, clip}%
\begin{pgfscope}%
\pgfsetbuttcap%
\pgfsetmiterjoin%
\definecolor{currentfill}{rgb}{1.000000,1.000000,1.000000}%
\pgfsetfillcolor{currentfill}%
\pgfsetlinewidth{0.000000pt}%
\definecolor{currentstroke}{rgb}{1.000000,1.000000,1.000000}%
\pgfsetstrokecolor{currentstroke}%
\pgfsetdash{}{0pt}%
\pgfpathmoveto{\pgfqpoint{0.000000in}{0.000000in}}%
\pgfpathlineto{\pgfqpoint{5.600000in}{0.000000in}}%
\pgfpathlineto{\pgfqpoint{5.600000in}{4.000000in}}%
\pgfpathlineto{\pgfqpoint{0.000000in}{4.000000in}}%
\pgfpathlineto{\pgfqpoint{0.000000in}{0.000000in}}%
\pgfpathclose%
\pgfusepath{fill}%
\end{pgfscope}%
\begin{pgfscope}%
\pgfsetbuttcap%
\pgfsetmiterjoin%
\definecolor{currentfill}{rgb}{0.917647,0.917647,0.949020}%
\pgfsetfillcolor{currentfill}%
\pgfsetlinewidth{0.000000pt}%
\definecolor{currentstroke}{rgb}{0.000000,0.000000,0.000000}%
\pgfsetstrokecolor{currentstroke}%
\pgfsetstrokeopacity{0.000000}%
\pgfsetdash{}{0pt}%
\pgfpathmoveto{\pgfqpoint{0.863783in}{2.546295in}}%
\pgfpathlineto{\pgfqpoint{5.420000in}{2.546295in}}%
\pgfpathlineto{\pgfqpoint{5.420000in}{3.765319in}}%
\pgfpathlineto{\pgfqpoint{0.863783in}{3.765319in}}%
\pgfpathlineto{\pgfqpoint{0.863783in}{2.546295in}}%
\pgfpathclose%
\pgfusepath{fill}%
\end{pgfscope}%
\begin{pgfscope}%
\pgfpathrectangle{\pgfqpoint{0.863783in}{2.546295in}}{\pgfqpoint{4.556217in}{1.219024in}}%
\pgfusepath{clip}%
\pgfsetroundcap%
\pgfsetroundjoin%
\pgfsetlinewidth{1.003750pt}%
\definecolor{currentstroke}{rgb}{1.000000,1.000000,1.000000}%
\pgfsetstrokecolor{currentstroke}%
\pgfsetdash{}{0pt}%
\pgfpathmoveto{\pgfqpoint{1.070884in}{2.546295in}}%
\pgfpathlineto{\pgfqpoint{1.070884in}{3.765319in}}%
\pgfusepath{stroke}%
\end{pgfscope}%
\begin{pgfscope}%
\definecolor{textcolor}{rgb}{0.150000,0.150000,0.150000}%
\pgfsetstrokecolor{textcolor}%
\pgfsetfillcolor{textcolor}%
\pgftext[x=1.070884in,y=2.414351in,,top]{\color{textcolor}{\sffamily\fontsize{11.000000}{13.200000}\selectfont\catcode`\^=\active\def^{\ifmmode\sp\else\^{}\fi}\catcode`\%=\active\def%{\%}0}}%
\end{pgfscope}%
\begin{pgfscope}%
\pgfpathrectangle{\pgfqpoint{0.863783in}{2.546295in}}{\pgfqpoint{4.556217in}{1.219024in}}%
\pgfusepath{clip}%
\pgfsetroundcap%
\pgfsetroundjoin%
\pgfsetlinewidth{1.003750pt}%
\definecolor{currentstroke}{rgb}{1.000000,1.000000,1.000000}%
\pgfsetstrokecolor{currentstroke}%
\pgfsetdash{}{0pt}%
\pgfpathmoveto{\pgfqpoint{1.582244in}{2.546295in}}%
\pgfpathlineto{\pgfqpoint{1.582244in}{3.765319in}}%
\pgfusepath{stroke}%
\end{pgfscope}%
\begin{pgfscope}%
\definecolor{textcolor}{rgb}{0.150000,0.150000,0.150000}%
\pgfsetstrokecolor{textcolor}%
\pgfsetfillcolor{textcolor}%
\pgftext[x=1.582244in,y=2.414351in,,top]{\color{textcolor}{\sffamily\fontsize{11.000000}{13.200000}\selectfont\catcode`\^=\active\def^{\ifmmode\sp\else\^{}\fi}\catcode`\%=\active\def%{\%}200}}%
\end{pgfscope}%
\begin{pgfscope}%
\pgfpathrectangle{\pgfqpoint{0.863783in}{2.546295in}}{\pgfqpoint{4.556217in}{1.219024in}}%
\pgfusepath{clip}%
\pgfsetroundcap%
\pgfsetroundjoin%
\pgfsetlinewidth{1.003750pt}%
\definecolor{currentstroke}{rgb}{1.000000,1.000000,1.000000}%
\pgfsetstrokecolor{currentstroke}%
\pgfsetdash{}{0pt}%
\pgfpathmoveto{\pgfqpoint{2.093604in}{2.546295in}}%
\pgfpathlineto{\pgfqpoint{2.093604in}{3.765319in}}%
\pgfusepath{stroke}%
\end{pgfscope}%
\begin{pgfscope}%
\definecolor{textcolor}{rgb}{0.150000,0.150000,0.150000}%
\pgfsetstrokecolor{textcolor}%
\pgfsetfillcolor{textcolor}%
\pgftext[x=2.093604in,y=2.414351in,,top]{\color{textcolor}{\sffamily\fontsize{11.000000}{13.200000}\selectfont\catcode`\^=\active\def^{\ifmmode\sp\else\^{}\fi}\catcode`\%=\active\def%{\%}400}}%
\end{pgfscope}%
\begin{pgfscope}%
\pgfpathrectangle{\pgfqpoint{0.863783in}{2.546295in}}{\pgfqpoint{4.556217in}{1.219024in}}%
\pgfusepath{clip}%
\pgfsetroundcap%
\pgfsetroundjoin%
\pgfsetlinewidth{1.003750pt}%
\definecolor{currentstroke}{rgb}{1.000000,1.000000,1.000000}%
\pgfsetstrokecolor{currentstroke}%
\pgfsetdash{}{0pt}%
\pgfpathmoveto{\pgfqpoint{2.604964in}{2.546295in}}%
\pgfpathlineto{\pgfqpoint{2.604964in}{3.765319in}}%
\pgfusepath{stroke}%
\end{pgfscope}%
\begin{pgfscope}%
\definecolor{textcolor}{rgb}{0.150000,0.150000,0.150000}%
\pgfsetstrokecolor{textcolor}%
\pgfsetfillcolor{textcolor}%
\pgftext[x=2.604964in,y=2.414351in,,top]{\color{textcolor}{\sffamily\fontsize{11.000000}{13.200000}\selectfont\catcode`\^=\active\def^{\ifmmode\sp\else\^{}\fi}\catcode`\%=\active\def%{\%}600}}%
\end{pgfscope}%
\begin{pgfscope}%
\pgfpathrectangle{\pgfqpoint{0.863783in}{2.546295in}}{\pgfqpoint{4.556217in}{1.219024in}}%
\pgfusepath{clip}%
\pgfsetroundcap%
\pgfsetroundjoin%
\pgfsetlinewidth{1.003750pt}%
\definecolor{currentstroke}{rgb}{1.000000,1.000000,1.000000}%
\pgfsetstrokecolor{currentstroke}%
\pgfsetdash{}{0pt}%
\pgfpathmoveto{\pgfqpoint{3.116324in}{2.546295in}}%
\pgfpathlineto{\pgfqpoint{3.116324in}{3.765319in}}%
\pgfusepath{stroke}%
\end{pgfscope}%
\begin{pgfscope}%
\definecolor{textcolor}{rgb}{0.150000,0.150000,0.150000}%
\pgfsetstrokecolor{textcolor}%
\pgfsetfillcolor{textcolor}%
\pgftext[x=3.116324in,y=2.414351in,,top]{\color{textcolor}{\sffamily\fontsize{11.000000}{13.200000}\selectfont\catcode`\^=\active\def^{\ifmmode\sp\else\^{}\fi}\catcode`\%=\active\def%{\%}800}}%
\end{pgfscope}%
\begin{pgfscope}%
\pgfpathrectangle{\pgfqpoint{0.863783in}{2.546295in}}{\pgfqpoint{4.556217in}{1.219024in}}%
\pgfusepath{clip}%
\pgfsetroundcap%
\pgfsetroundjoin%
\pgfsetlinewidth{1.003750pt}%
\definecolor{currentstroke}{rgb}{1.000000,1.000000,1.000000}%
\pgfsetstrokecolor{currentstroke}%
\pgfsetdash{}{0pt}%
\pgfpathmoveto{\pgfqpoint{3.627684in}{2.546295in}}%
\pgfpathlineto{\pgfqpoint{3.627684in}{3.765319in}}%
\pgfusepath{stroke}%
\end{pgfscope}%
\begin{pgfscope}%
\definecolor{textcolor}{rgb}{0.150000,0.150000,0.150000}%
\pgfsetstrokecolor{textcolor}%
\pgfsetfillcolor{textcolor}%
\pgftext[x=3.627684in,y=2.414351in,,top]{\color{textcolor}{\sffamily\fontsize{11.000000}{13.200000}\selectfont\catcode`\^=\active\def^{\ifmmode\sp\else\^{}\fi}\catcode`\%=\active\def%{\%}1000}}%
\end{pgfscope}%
\begin{pgfscope}%
\pgfpathrectangle{\pgfqpoint{0.863783in}{2.546295in}}{\pgfqpoint{4.556217in}{1.219024in}}%
\pgfusepath{clip}%
\pgfsetroundcap%
\pgfsetroundjoin%
\pgfsetlinewidth{1.003750pt}%
\definecolor{currentstroke}{rgb}{1.000000,1.000000,1.000000}%
\pgfsetstrokecolor{currentstroke}%
\pgfsetdash{}{0pt}%
\pgfpathmoveto{\pgfqpoint{4.139044in}{2.546295in}}%
\pgfpathlineto{\pgfqpoint{4.139044in}{3.765319in}}%
\pgfusepath{stroke}%
\end{pgfscope}%
\begin{pgfscope}%
\definecolor{textcolor}{rgb}{0.150000,0.150000,0.150000}%
\pgfsetstrokecolor{textcolor}%
\pgfsetfillcolor{textcolor}%
\pgftext[x=4.139044in,y=2.414351in,,top]{\color{textcolor}{\sffamily\fontsize{11.000000}{13.200000}\selectfont\catcode`\^=\active\def^{\ifmmode\sp\else\^{}\fi}\catcode`\%=\active\def%{\%}1200}}%
\end{pgfscope}%
\begin{pgfscope}%
\pgfpathrectangle{\pgfqpoint{0.863783in}{2.546295in}}{\pgfqpoint{4.556217in}{1.219024in}}%
\pgfusepath{clip}%
\pgfsetroundcap%
\pgfsetroundjoin%
\pgfsetlinewidth{1.003750pt}%
\definecolor{currentstroke}{rgb}{1.000000,1.000000,1.000000}%
\pgfsetstrokecolor{currentstroke}%
\pgfsetdash{}{0pt}%
\pgfpathmoveto{\pgfqpoint{4.650403in}{2.546295in}}%
\pgfpathlineto{\pgfqpoint{4.650403in}{3.765319in}}%
\pgfusepath{stroke}%
\end{pgfscope}%
\begin{pgfscope}%
\definecolor{textcolor}{rgb}{0.150000,0.150000,0.150000}%
\pgfsetstrokecolor{textcolor}%
\pgfsetfillcolor{textcolor}%
\pgftext[x=4.650403in,y=2.414351in,,top]{\color{textcolor}{\sffamily\fontsize{11.000000}{13.200000}\selectfont\catcode`\^=\active\def^{\ifmmode\sp\else\^{}\fi}\catcode`\%=\active\def%{\%}1400}}%
\end{pgfscope}%
\begin{pgfscope}%
\pgfpathrectangle{\pgfqpoint{0.863783in}{2.546295in}}{\pgfqpoint{4.556217in}{1.219024in}}%
\pgfusepath{clip}%
\pgfsetroundcap%
\pgfsetroundjoin%
\pgfsetlinewidth{1.003750pt}%
\definecolor{currentstroke}{rgb}{1.000000,1.000000,1.000000}%
\pgfsetstrokecolor{currentstroke}%
\pgfsetdash{}{0pt}%
\pgfpathmoveto{\pgfqpoint{5.161763in}{2.546295in}}%
\pgfpathlineto{\pgfqpoint{5.161763in}{3.765319in}}%
\pgfusepath{stroke}%
\end{pgfscope}%
\begin{pgfscope}%
\definecolor{textcolor}{rgb}{0.150000,0.150000,0.150000}%
\pgfsetstrokecolor{textcolor}%
\pgfsetfillcolor{textcolor}%
\pgftext[x=5.161763in,y=2.414351in,,top]{\color{textcolor}{\sffamily\fontsize{11.000000}{13.200000}\selectfont\catcode`\^=\active\def^{\ifmmode\sp\else\^{}\fi}\catcode`\%=\active\def%{\%}1600}}%
\end{pgfscope}%
\begin{pgfscope}%
\definecolor{textcolor}{rgb}{0.150000,0.150000,0.150000}%
\pgfsetstrokecolor{textcolor}%
\pgfsetfillcolor{textcolor}%
\pgftext[x=3.141892in,y=2.219072in,,top]{\color{textcolor}{\sffamily\fontsize{12.000000}{14.400000}\selectfont\catcode`\^=\active\def^{\ifmmode\sp\else\^{}\fi}\catcode`\%=\active\def%{\%}Time (s)}}%
\end{pgfscope}%
\begin{pgfscope}%
\pgfpathrectangle{\pgfqpoint{0.863783in}{2.546295in}}{\pgfqpoint{4.556217in}{1.219024in}}%
\pgfusepath{clip}%
\pgfsetroundcap%
\pgfsetroundjoin%
\pgfsetlinewidth{1.003750pt}%
\definecolor{currentstroke}{rgb}{1.000000,1.000000,1.000000}%
\pgfsetstrokecolor{currentstroke}%
\pgfsetdash{}{0pt}%
\pgfpathmoveto{\pgfqpoint{0.863783in}{2.790100in}}%
\pgfpathlineto{\pgfqpoint{5.420000in}{2.790100in}}%
\pgfusepath{stroke}%
\end{pgfscope}%
\begin{pgfscope}%
\definecolor{textcolor}{rgb}{0.150000,0.150000,0.150000}%
\pgfsetstrokecolor{textcolor}%
\pgfsetfillcolor{textcolor}%
\pgftext[x=0.476936in, y=2.735419in, left, base]{\color{textcolor}{\sffamily\fontsize{11.000000}{13.200000}\selectfont\catcode`\^=\active\def^{\ifmmode\sp\else\^{}\fi}\catcode`\%=\active\def%{\%}800}}%
\end{pgfscope}%
\begin{pgfscope}%
\pgfpathrectangle{\pgfqpoint{0.863783in}{2.546295in}}{\pgfqpoint{4.556217in}{1.219024in}}%
\pgfusepath{clip}%
\pgfsetroundcap%
\pgfsetroundjoin%
\pgfsetlinewidth{1.003750pt}%
\definecolor{currentstroke}{rgb}{1.000000,1.000000,1.000000}%
\pgfsetstrokecolor{currentstroke}%
\pgfsetdash{}{0pt}%
\pgfpathmoveto{\pgfqpoint{0.863783in}{3.277710in}}%
\pgfpathlineto{\pgfqpoint{5.420000in}{3.277710in}}%
\pgfusepath{stroke}%
\end{pgfscope}%
\begin{pgfscope}%
\definecolor{textcolor}{rgb}{0.150000,0.150000,0.150000}%
\pgfsetstrokecolor{textcolor}%
\pgfsetfillcolor{textcolor}%
\pgftext[x=0.391968in, y=3.223029in, left, base]{\color{textcolor}{\sffamily\fontsize{11.000000}{13.200000}\selectfont\catcode`\^=\active\def^{\ifmmode\sp\else\^{}\fi}\catcode`\%=\active\def%{\%}1000}}%
\end{pgfscope}%
\begin{pgfscope}%
\pgfpathrectangle{\pgfqpoint{0.863783in}{2.546295in}}{\pgfqpoint{4.556217in}{1.219024in}}%
\pgfusepath{clip}%
\pgfsetroundcap%
\pgfsetroundjoin%
\pgfsetlinewidth{1.003750pt}%
\definecolor{currentstroke}{rgb}{1.000000,1.000000,1.000000}%
\pgfsetstrokecolor{currentstroke}%
\pgfsetdash{}{0pt}%
\pgfpathmoveto{\pgfqpoint{0.863783in}{3.765319in}}%
\pgfpathlineto{\pgfqpoint{5.420000in}{3.765319in}}%
\pgfusepath{stroke}%
\end{pgfscope}%
\begin{pgfscope}%
\definecolor{textcolor}{rgb}{0.150000,0.150000,0.150000}%
\pgfsetstrokecolor{textcolor}%
\pgfsetfillcolor{textcolor}%
\pgftext[x=0.391968in, y=3.710639in, left, base]{\color{textcolor}{\sffamily\fontsize{11.000000}{13.200000}\selectfont\catcode`\^=\active\def^{\ifmmode\sp\else\^{}\fi}\catcode`\%=\active\def%{\%}1200}}%
\end{pgfscope}%
\begin{pgfscope}%
\definecolor{textcolor}{rgb}{0.150000,0.150000,0.150000}%
\pgfsetstrokecolor{textcolor}%
\pgfsetfillcolor{textcolor}%
\pgftext[x=0.336413in,y=3.155807in,,bottom,rotate=90.000000]{\color{textcolor}{\sffamily\fontsize{12.000000}{14.400000}\selectfont\catcode`\^=\active\def^{\ifmmode\sp\else\^{}\fi}\catcode`\%=\active\def%{\%}CPU Limits (milliCPU)}}%
\end{pgfscope}%
\begin{pgfscope}%
\pgfpathrectangle{\pgfqpoint{0.863783in}{2.546295in}}{\pgfqpoint{4.556217in}{1.219024in}}%
\pgfusepath{clip}%
\pgfsetroundcap%
\pgfsetroundjoin%
\pgfsetlinewidth{1.505625pt}%
\definecolor{currentstroke}{rgb}{0.298039,0.447059,0.690196}%
\pgfsetstrokecolor{currentstroke}%
\pgfsetdash{}{0pt}%
\pgfpathmoveto{\pgfqpoint{1.070884in}{3.521514in}}%
\pgfpathlineto{\pgfqpoint{1.684516in}{3.521514in}}%
\pgfpathlineto{\pgfqpoint{1.697300in}{3.033905in}}%
\pgfpathlineto{\pgfqpoint{5.212899in}{3.033905in}}%
\pgfpathlineto{\pgfqpoint{5.212899in}{3.033905in}}%
\pgfusepath{stroke}%
\end{pgfscope}%
\begin{pgfscope}%
\pgfsetrectcap%
\pgfsetmiterjoin%
\pgfsetlinewidth{1.254687pt}%
\definecolor{currentstroke}{rgb}{1.000000,1.000000,1.000000}%
\pgfsetstrokecolor{currentstroke}%
\pgfsetdash{}{0pt}%
\pgfpathmoveto{\pgfqpoint{0.863783in}{2.546295in}}%
\pgfpathlineto{\pgfqpoint{0.863783in}{3.765319in}}%
\pgfusepath{stroke}%
\end{pgfscope}%
\begin{pgfscope}%
\pgfsetrectcap%
\pgfsetmiterjoin%
\pgfsetlinewidth{1.254687pt}%
\definecolor{currentstroke}{rgb}{1.000000,1.000000,1.000000}%
\pgfsetstrokecolor{currentstroke}%
\pgfsetdash{}{0pt}%
\pgfpathmoveto{\pgfqpoint{5.420000in}{2.546295in}}%
\pgfpathlineto{\pgfqpoint{5.420000in}{3.765319in}}%
\pgfusepath{stroke}%
\end{pgfscope}%
\begin{pgfscope}%
\pgfsetrectcap%
\pgfsetmiterjoin%
\pgfsetlinewidth{1.254687pt}%
\definecolor{currentstroke}{rgb}{1.000000,1.000000,1.000000}%
\pgfsetstrokecolor{currentstroke}%
\pgfsetdash{}{0pt}%
\pgfpathmoveto{\pgfqpoint{0.863783in}{2.546295in}}%
\pgfpathlineto{\pgfqpoint{5.420000in}{2.546295in}}%
\pgfusepath{stroke}%
\end{pgfscope}%
\begin{pgfscope}%
\pgfsetrectcap%
\pgfsetmiterjoin%
\pgfsetlinewidth{1.254687pt}%
\definecolor{currentstroke}{rgb}{1.000000,1.000000,1.000000}%
\pgfsetstrokecolor{currentstroke}%
\pgfsetdash{}{0pt}%
\pgfpathmoveto{\pgfqpoint{0.863783in}{3.765319in}}%
\pgfpathlineto{\pgfqpoint{5.420000in}{3.765319in}}%
\pgfusepath{stroke}%
\end{pgfscope}%
\begin{pgfscope}%
\pgfsetbuttcap%
\pgfsetmiterjoin%
\definecolor{currentfill}{rgb}{0.917647,0.917647,0.949020}%
\pgfsetfillcolor{currentfill}%
\pgfsetlinewidth{0.000000pt}%
\definecolor{currentstroke}{rgb}{0.000000,0.000000,0.000000}%
\pgfsetstrokecolor{currentstroke}%
\pgfsetstrokeopacity{0.000000}%
\pgfsetdash{}{0pt}%
\pgfpathmoveto{\pgfqpoint{0.863783in}{0.663635in}}%
\pgfpathlineto{\pgfqpoint{5.420000in}{0.663635in}}%
\pgfpathlineto{\pgfqpoint{5.420000in}{1.882660in}}%
\pgfpathlineto{\pgfqpoint{0.863783in}{1.882660in}}%
\pgfpathlineto{\pgfqpoint{0.863783in}{0.663635in}}%
\pgfpathclose%
\pgfusepath{fill}%
\end{pgfscope}%
\begin{pgfscope}%
\pgfpathrectangle{\pgfqpoint{0.863783in}{0.663635in}}{\pgfqpoint{4.556217in}{1.219024in}}%
\pgfusepath{clip}%
\pgfsetroundcap%
\pgfsetroundjoin%
\pgfsetlinewidth{1.003750pt}%
\definecolor{currentstroke}{rgb}{1.000000,1.000000,1.000000}%
\pgfsetstrokecolor{currentstroke}%
\pgfsetdash{}{0pt}%
\pgfpathmoveto{\pgfqpoint{1.070884in}{0.663635in}}%
\pgfpathlineto{\pgfqpoint{1.070884in}{1.882660in}}%
\pgfusepath{stroke}%
\end{pgfscope}%
\begin{pgfscope}%
\definecolor{textcolor}{rgb}{0.150000,0.150000,0.150000}%
\pgfsetstrokecolor{textcolor}%
\pgfsetfillcolor{textcolor}%
\pgftext[x=1.070884in,y=0.531691in,,top]{\color{textcolor}{\sffamily\fontsize{11.000000}{13.200000}\selectfont\catcode`\^=\active\def^{\ifmmode\sp\else\^{}\fi}\catcode`\%=\active\def%{\%}0}}%
\end{pgfscope}%
\begin{pgfscope}%
\pgfpathrectangle{\pgfqpoint{0.863783in}{0.663635in}}{\pgfqpoint{4.556217in}{1.219024in}}%
\pgfusepath{clip}%
\pgfsetroundcap%
\pgfsetroundjoin%
\pgfsetlinewidth{1.003750pt}%
\definecolor{currentstroke}{rgb}{1.000000,1.000000,1.000000}%
\pgfsetstrokecolor{currentstroke}%
\pgfsetdash{}{0pt}%
\pgfpathmoveto{\pgfqpoint{1.582244in}{0.663635in}}%
\pgfpathlineto{\pgfqpoint{1.582244in}{1.882660in}}%
\pgfusepath{stroke}%
\end{pgfscope}%
\begin{pgfscope}%
\definecolor{textcolor}{rgb}{0.150000,0.150000,0.150000}%
\pgfsetstrokecolor{textcolor}%
\pgfsetfillcolor{textcolor}%
\pgftext[x=1.582244in,y=0.531691in,,top]{\color{textcolor}{\sffamily\fontsize{11.000000}{13.200000}\selectfont\catcode`\^=\active\def^{\ifmmode\sp\else\^{}\fi}\catcode`\%=\active\def%{\%}200}}%
\end{pgfscope}%
\begin{pgfscope}%
\pgfpathrectangle{\pgfqpoint{0.863783in}{0.663635in}}{\pgfqpoint{4.556217in}{1.219024in}}%
\pgfusepath{clip}%
\pgfsetroundcap%
\pgfsetroundjoin%
\pgfsetlinewidth{1.003750pt}%
\definecolor{currentstroke}{rgb}{1.000000,1.000000,1.000000}%
\pgfsetstrokecolor{currentstroke}%
\pgfsetdash{}{0pt}%
\pgfpathmoveto{\pgfqpoint{2.093604in}{0.663635in}}%
\pgfpathlineto{\pgfqpoint{2.093604in}{1.882660in}}%
\pgfusepath{stroke}%
\end{pgfscope}%
\begin{pgfscope}%
\definecolor{textcolor}{rgb}{0.150000,0.150000,0.150000}%
\pgfsetstrokecolor{textcolor}%
\pgfsetfillcolor{textcolor}%
\pgftext[x=2.093604in,y=0.531691in,,top]{\color{textcolor}{\sffamily\fontsize{11.000000}{13.200000}\selectfont\catcode`\^=\active\def^{\ifmmode\sp\else\^{}\fi}\catcode`\%=\active\def%{\%}400}}%
\end{pgfscope}%
\begin{pgfscope}%
\pgfpathrectangle{\pgfqpoint{0.863783in}{0.663635in}}{\pgfqpoint{4.556217in}{1.219024in}}%
\pgfusepath{clip}%
\pgfsetroundcap%
\pgfsetroundjoin%
\pgfsetlinewidth{1.003750pt}%
\definecolor{currentstroke}{rgb}{1.000000,1.000000,1.000000}%
\pgfsetstrokecolor{currentstroke}%
\pgfsetdash{}{0pt}%
\pgfpathmoveto{\pgfqpoint{2.604964in}{0.663635in}}%
\pgfpathlineto{\pgfqpoint{2.604964in}{1.882660in}}%
\pgfusepath{stroke}%
\end{pgfscope}%
\begin{pgfscope}%
\definecolor{textcolor}{rgb}{0.150000,0.150000,0.150000}%
\pgfsetstrokecolor{textcolor}%
\pgfsetfillcolor{textcolor}%
\pgftext[x=2.604964in,y=0.531691in,,top]{\color{textcolor}{\sffamily\fontsize{11.000000}{13.200000}\selectfont\catcode`\^=\active\def^{\ifmmode\sp\else\^{}\fi}\catcode`\%=\active\def%{\%}600}}%
\end{pgfscope}%
\begin{pgfscope}%
\pgfpathrectangle{\pgfqpoint{0.863783in}{0.663635in}}{\pgfqpoint{4.556217in}{1.219024in}}%
\pgfusepath{clip}%
\pgfsetroundcap%
\pgfsetroundjoin%
\pgfsetlinewidth{1.003750pt}%
\definecolor{currentstroke}{rgb}{1.000000,1.000000,1.000000}%
\pgfsetstrokecolor{currentstroke}%
\pgfsetdash{}{0pt}%
\pgfpathmoveto{\pgfqpoint{3.116324in}{0.663635in}}%
\pgfpathlineto{\pgfqpoint{3.116324in}{1.882660in}}%
\pgfusepath{stroke}%
\end{pgfscope}%
\begin{pgfscope}%
\definecolor{textcolor}{rgb}{0.150000,0.150000,0.150000}%
\pgfsetstrokecolor{textcolor}%
\pgfsetfillcolor{textcolor}%
\pgftext[x=3.116324in,y=0.531691in,,top]{\color{textcolor}{\sffamily\fontsize{11.000000}{13.200000}\selectfont\catcode`\^=\active\def^{\ifmmode\sp\else\^{}\fi}\catcode`\%=\active\def%{\%}800}}%
\end{pgfscope}%
\begin{pgfscope}%
\pgfpathrectangle{\pgfqpoint{0.863783in}{0.663635in}}{\pgfqpoint{4.556217in}{1.219024in}}%
\pgfusepath{clip}%
\pgfsetroundcap%
\pgfsetroundjoin%
\pgfsetlinewidth{1.003750pt}%
\definecolor{currentstroke}{rgb}{1.000000,1.000000,1.000000}%
\pgfsetstrokecolor{currentstroke}%
\pgfsetdash{}{0pt}%
\pgfpathmoveto{\pgfqpoint{3.627684in}{0.663635in}}%
\pgfpathlineto{\pgfqpoint{3.627684in}{1.882660in}}%
\pgfusepath{stroke}%
\end{pgfscope}%
\begin{pgfscope}%
\definecolor{textcolor}{rgb}{0.150000,0.150000,0.150000}%
\pgfsetstrokecolor{textcolor}%
\pgfsetfillcolor{textcolor}%
\pgftext[x=3.627684in,y=0.531691in,,top]{\color{textcolor}{\sffamily\fontsize{11.000000}{13.200000}\selectfont\catcode`\^=\active\def^{\ifmmode\sp\else\^{}\fi}\catcode`\%=\active\def%{\%}1000}}%
\end{pgfscope}%
\begin{pgfscope}%
\pgfpathrectangle{\pgfqpoint{0.863783in}{0.663635in}}{\pgfqpoint{4.556217in}{1.219024in}}%
\pgfusepath{clip}%
\pgfsetroundcap%
\pgfsetroundjoin%
\pgfsetlinewidth{1.003750pt}%
\definecolor{currentstroke}{rgb}{1.000000,1.000000,1.000000}%
\pgfsetstrokecolor{currentstroke}%
\pgfsetdash{}{0pt}%
\pgfpathmoveto{\pgfqpoint{4.139044in}{0.663635in}}%
\pgfpathlineto{\pgfqpoint{4.139044in}{1.882660in}}%
\pgfusepath{stroke}%
\end{pgfscope}%
\begin{pgfscope}%
\definecolor{textcolor}{rgb}{0.150000,0.150000,0.150000}%
\pgfsetstrokecolor{textcolor}%
\pgfsetfillcolor{textcolor}%
\pgftext[x=4.139044in,y=0.531691in,,top]{\color{textcolor}{\sffamily\fontsize{11.000000}{13.200000}\selectfont\catcode`\^=\active\def^{\ifmmode\sp\else\^{}\fi}\catcode`\%=\active\def%{\%}1200}}%
\end{pgfscope}%
\begin{pgfscope}%
\pgfpathrectangle{\pgfqpoint{0.863783in}{0.663635in}}{\pgfqpoint{4.556217in}{1.219024in}}%
\pgfusepath{clip}%
\pgfsetroundcap%
\pgfsetroundjoin%
\pgfsetlinewidth{1.003750pt}%
\definecolor{currentstroke}{rgb}{1.000000,1.000000,1.000000}%
\pgfsetstrokecolor{currentstroke}%
\pgfsetdash{}{0pt}%
\pgfpathmoveto{\pgfqpoint{4.650403in}{0.663635in}}%
\pgfpathlineto{\pgfqpoint{4.650403in}{1.882660in}}%
\pgfusepath{stroke}%
\end{pgfscope}%
\begin{pgfscope}%
\definecolor{textcolor}{rgb}{0.150000,0.150000,0.150000}%
\pgfsetstrokecolor{textcolor}%
\pgfsetfillcolor{textcolor}%
\pgftext[x=4.650403in,y=0.531691in,,top]{\color{textcolor}{\sffamily\fontsize{11.000000}{13.200000}\selectfont\catcode`\^=\active\def^{\ifmmode\sp\else\^{}\fi}\catcode`\%=\active\def%{\%}1400}}%
\end{pgfscope}%
\begin{pgfscope}%
\pgfpathrectangle{\pgfqpoint{0.863783in}{0.663635in}}{\pgfqpoint{4.556217in}{1.219024in}}%
\pgfusepath{clip}%
\pgfsetroundcap%
\pgfsetroundjoin%
\pgfsetlinewidth{1.003750pt}%
\definecolor{currentstroke}{rgb}{1.000000,1.000000,1.000000}%
\pgfsetstrokecolor{currentstroke}%
\pgfsetdash{}{0pt}%
\pgfpathmoveto{\pgfqpoint{5.161763in}{0.663635in}}%
\pgfpathlineto{\pgfqpoint{5.161763in}{1.882660in}}%
\pgfusepath{stroke}%
\end{pgfscope}%
\begin{pgfscope}%
\definecolor{textcolor}{rgb}{0.150000,0.150000,0.150000}%
\pgfsetstrokecolor{textcolor}%
\pgfsetfillcolor{textcolor}%
\pgftext[x=5.161763in,y=0.531691in,,top]{\color{textcolor}{\sffamily\fontsize{11.000000}{13.200000}\selectfont\catcode`\^=\active\def^{\ifmmode\sp\else\^{}\fi}\catcode`\%=\active\def%{\%}1600}}%
\end{pgfscope}%
\begin{pgfscope}%
\definecolor{textcolor}{rgb}{0.150000,0.150000,0.150000}%
\pgfsetstrokecolor{textcolor}%
\pgfsetfillcolor{textcolor}%
\pgftext[x=3.141892in,y=0.336413in,,top]{\color{textcolor}{\sffamily\fontsize{12.000000}{14.400000}\selectfont\catcode`\^=\active\def^{\ifmmode\sp\else\^{}\fi}\catcode`\%=\active\def%{\%}Time (s)}}%
\end{pgfscope}%
\begin{pgfscope}%
\pgfpathrectangle{\pgfqpoint{0.863783in}{0.663635in}}{\pgfqpoint{4.556217in}{1.219024in}}%
\pgfusepath{clip}%
\pgfsetroundcap%
\pgfsetroundjoin%
\pgfsetlinewidth{1.003750pt}%
\definecolor{currentstroke}{rgb}{1.000000,1.000000,1.000000}%
\pgfsetstrokecolor{currentstroke}%
\pgfsetdash{}{0pt}%
\pgfpathmoveto{\pgfqpoint{0.863783in}{0.934530in}}%
\pgfpathlineto{\pgfqpoint{5.420000in}{0.934530in}}%
\pgfusepath{stroke}%
\end{pgfscope}%
\begin{pgfscope}%
\definecolor{textcolor}{rgb}{0.150000,0.150000,0.150000}%
\pgfsetstrokecolor{textcolor}%
\pgfsetfillcolor{textcolor}%
\pgftext[x=0.391968in, y=0.879849in, left, base]{\color{textcolor}{\sffamily\fontsize{11.000000}{13.200000}\selectfont\catcode`\^=\active\def^{\ifmmode\sp\else\^{}\fi}\catcode`\%=\active\def%{\%}4000}}%
\end{pgfscope}%
\begin{pgfscope}%
\pgfpathrectangle{\pgfqpoint{0.863783in}{0.663635in}}{\pgfqpoint{4.556217in}{1.219024in}}%
\pgfusepath{clip}%
\pgfsetroundcap%
\pgfsetroundjoin%
\pgfsetlinewidth{1.003750pt}%
\definecolor{currentstroke}{rgb}{1.000000,1.000000,1.000000}%
\pgfsetstrokecolor{currentstroke}%
\pgfsetdash{}{0pt}%
\pgfpathmoveto{\pgfqpoint{0.863783in}{1.476318in}}%
\pgfpathlineto{\pgfqpoint{5.420000in}{1.476318in}}%
\pgfusepath{stroke}%
\end{pgfscope}%
\begin{pgfscope}%
\definecolor{textcolor}{rgb}{0.150000,0.150000,0.150000}%
\pgfsetstrokecolor{textcolor}%
\pgfsetfillcolor{textcolor}%
\pgftext[x=0.391968in, y=1.421638in, left, base]{\color{textcolor}{\sffamily\fontsize{11.000000}{13.200000}\selectfont\catcode`\^=\active\def^{\ifmmode\sp\else\^{}\fi}\catcode`\%=\active\def%{\%}6000}}%
\end{pgfscope}%
\begin{pgfscope}%
\definecolor{textcolor}{rgb}{0.150000,0.150000,0.150000}%
\pgfsetstrokecolor{textcolor}%
\pgfsetfillcolor{textcolor}%
\pgftext[x=0.336413in,y=1.273148in,,bottom,rotate=90.000000]{\color{textcolor}{\sffamily\fontsize{12.000000}{14.400000}\selectfont\catcode`\^=\active\def^{\ifmmode\sp\else\^{}\fi}\catcode`\%=\active\def%{\%}Memory Limits (MiB)}}%
\end{pgfscope}%
\begin{pgfscope}%
\pgfpathrectangle{\pgfqpoint{0.863783in}{0.663635in}}{\pgfqpoint{4.556217in}{1.219024in}}%
\pgfusepath{clip}%
\pgfsetroundcap%
\pgfsetroundjoin%
\pgfsetlinewidth{1.505625pt}%
\definecolor{currentstroke}{rgb}{0.298039,0.447059,0.690196}%
\pgfsetstrokecolor{currentstroke}%
\pgfsetdash{}{0pt}%
\pgfpathmoveto{\pgfqpoint{1.070884in}{1.572609in}}%
\pgfpathlineto{\pgfqpoint{1.684516in}{1.572609in}}%
\pgfpathlineto{\pgfqpoint{1.697300in}{0.924968in}}%
\pgfpathlineto{\pgfqpoint{5.212899in}{0.924968in}}%
\pgfpathlineto{\pgfqpoint{5.212899in}{0.924968in}}%
\pgfusepath{stroke}%
\end{pgfscope}%
\begin{pgfscope}%
\pgfsetrectcap%
\pgfsetmiterjoin%
\pgfsetlinewidth{1.254687pt}%
\definecolor{currentstroke}{rgb}{1.000000,1.000000,1.000000}%
\pgfsetstrokecolor{currentstroke}%
\pgfsetdash{}{0pt}%
\pgfpathmoveto{\pgfqpoint{0.863783in}{0.663635in}}%
\pgfpathlineto{\pgfqpoint{0.863783in}{1.882660in}}%
\pgfusepath{stroke}%
\end{pgfscope}%
\begin{pgfscope}%
\pgfsetrectcap%
\pgfsetmiterjoin%
\pgfsetlinewidth{1.254687pt}%
\definecolor{currentstroke}{rgb}{1.000000,1.000000,1.000000}%
\pgfsetstrokecolor{currentstroke}%
\pgfsetdash{}{0pt}%
\pgfpathmoveto{\pgfqpoint{5.420000in}{0.663635in}}%
\pgfpathlineto{\pgfqpoint{5.420000in}{1.882660in}}%
\pgfusepath{stroke}%
\end{pgfscope}%
\begin{pgfscope}%
\pgfsetrectcap%
\pgfsetmiterjoin%
\pgfsetlinewidth{1.254687pt}%
\definecolor{currentstroke}{rgb}{1.000000,1.000000,1.000000}%
\pgfsetstrokecolor{currentstroke}%
\pgfsetdash{}{0pt}%
\pgfpathmoveto{\pgfqpoint{0.863783in}{0.663635in}}%
\pgfpathlineto{\pgfqpoint{5.420000in}{0.663635in}}%
\pgfusepath{stroke}%
\end{pgfscope}%
\begin{pgfscope}%
\pgfsetrectcap%
\pgfsetmiterjoin%
\pgfsetlinewidth{1.254687pt}%
\definecolor{currentstroke}{rgb}{1.000000,1.000000,1.000000}%
\pgfsetstrokecolor{currentstroke}%
\pgfsetdash{}{0pt}%
\pgfpathmoveto{\pgfqpoint{0.863783in}{1.882660in}}%
\pgfpathlineto{\pgfqpoint{5.420000in}{1.882660in}}%
\pgfusepath{stroke}%
\end{pgfscope}%
\end{pgfpicture}%
\makeatother%
\endgroup%

    \caption{Adjustment of CPU and memory limits by the vertical elasticity strategy controller}
    \label{fig:simple-limits-vertical}
\end{figure}

\begin{figure}
    \centering
    %% Creator: Matplotlib, PGF backend
%%
%% To include the figure in your LaTeX document, write
%%   \input{<filename>.pgf}
%%
%% Make sure the required packages are loaded in your preamble
%%   \usepackage{pgf}
%%
%% Also ensure that all the required font packages are loaded; for instance,
%% the lmodern package is sometimes necessary when using math font.
%%   \usepackage{lmodern}
%%
%% Figures using additional raster images can only be included by \input if
%% they are in the same directory as the main LaTeX file. For loading figures
%% from other directories you can use the `import` package
%%   \usepackage{import}
%%
%% and then include the figures with
%%   \import{<path to file>}{<filename>.pgf}
%%
%% Matplotlib used the following preamble
%%   \def\mathdefault#1{#1}
%%   \everymath=\expandafter{\the\everymath\displaystyle}
%%   
%%   \usepackage{fontspec}
%%   \setmainfont{DejaVuSerif.ttf}[Path=\detokenize{/usr/local/lib/python3.11/site-packages/matplotlib/mpl-data/fonts/ttf/}]
%%   \setsansfont{Arial.ttf}[Path=\detokenize{/System/Library/Fonts/Supplemental/}]
%%   \setmonofont{DejaVuSansMono.ttf}[Path=\detokenize{/usr/local/lib/python3.11/site-packages/matplotlib/mpl-data/fonts/ttf/}]
%%   \makeatletter\@ifpackageloaded{underscore}{}{\usepackage[strings]{underscore}}\makeatother
%%
\begingroup%
\makeatletter%
\begin{pgfpicture}%
\pgfpathrectangle{\pgfpointorigin}{\pgfqpoint{5.600000in}{3.500000in}}%
\pgfusepath{use as bounding box, clip}%
\begin{pgfscope}%
\pgfsetbuttcap%
\pgfsetmiterjoin%
\definecolor{currentfill}{rgb}{1.000000,1.000000,1.000000}%
\pgfsetfillcolor{currentfill}%
\pgfsetlinewidth{0.000000pt}%
\definecolor{currentstroke}{rgb}{1.000000,1.000000,1.000000}%
\pgfsetstrokecolor{currentstroke}%
\pgfsetdash{}{0pt}%
\pgfpathmoveto{\pgfqpoint{0.000000in}{0.000000in}}%
\pgfpathlineto{\pgfqpoint{5.600000in}{0.000000in}}%
\pgfpathlineto{\pgfqpoint{5.600000in}{3.500000in}}%
\pgfpathlineto{\pgfqpoint{0.000000in}{3.500000in}}%
\pgfpathlineto{\pgfqpoint{0.000000in}{0.000000in}}%
\pgfpathclose%
\pgfusepath{fill}%
\end{pgfscope}%
\begin{pgfscope}%
\pgfsetbuttcap%
\pgfsetmiterjoin%
\definecolor{currentfill}{rgb}{0.917647,0.917647,0.949020}%
\pgfsetfillcolor{currentfill}%
\pgfsetlinewidth{0.000000pt}%
\definecolor{currentstroke}{rgb}{0.000000,0.000000,0.000000}%
\pgfsetstrokecolor{currentstroke}%
\pgfsetstrokeopacity{0.000000}%
\pgfsetdash{}{0pt}%
\pgfpathmoveto{\pgfqpoint{0.736295in}{2.323635in}}%
\pgfpathlineto{\pgfqpoint{5.420000in}{2.323635in}}%
\pgfpathlineto{\pgfqpoint{5.420000in}{3.320000in}}%
\pgfpathlineto{\pgfqpoint{0.736295in}{3.320000in}}%
\pgfpathlineto{\pgfqpoint{0.736295in}{2.323635in}}%
\pgfpathclose%
\pgfusepath{fill}%
\end{pgfscope}%
\begin{pgfscope}%
\pgfpathrectangle{\pgfqpoint{0.736295in}{2.323635in}}{\pgfqpoint{4.683705in}{0.996365in}}%
\pgfusepath{clip}%
\pgfsetroundcap%
\pgfsetroundjoin%
\pgfsetlinewidth{1.003750pt}%
\definecolor{currentstroke}{rgb}{1.000000,1.000000,1.000000}%
\pgfsetstrokecolor{currentstroke}%
\pgfsetdash{}{0pt}%
\pgfpathmoveto{\pgfqpoint{0.949190in}{2.323635in}}%
\pgfpathlineto{\pgfqpoint{0.949190in}{3.320000in}}%
\pgfusepath{stroke}%
\end{pgfscope}%
\begin{pgfscope}%
\definecolor{textcolor}{rgb}{0.150000,0.150000,0.150000}%
\pgfsetstrokecolor{textcolor}%
\pgfsetfillcolor{textcolor}%
\pgftext[x=0.949190in,y=2.191691in,,top]{\color{textcolor}{\sffamily\fontsize{11.000000}{13.200000}\selectfont\catcode`\^=\active\def^{\ifmmode\sp\else\^{}\fi}\catcode`\%=\active\def%{\%}0}}%
\end{pgfscope}%
\begin{pgfscope}%
\pgfpathrectangle{\pgfqpoint{0.736295in}{2.323635in}}{\pgfqpoint{4.683705in}{0.996365in}}%
\pgfusepath{clip}%
\pgfsetroundcap%
\pgfsetroundjoin%
\pgfsetlinewidth{1.003750pt}%
\definecolor{currentstroke}{rgb}{1.000000,1.000000,1.000000}%
\pgfsetstrokecolor{currentstroke}%
\pgfsetdash{}{0pt}%
\pgfpathmoveto{\pgfqpoint{1.658843in}{2.323635in}}%
\pgfpathlineto{\pgfqpoint{1.658843in}{3.320000in}}%
\pgfusepath{stroke}%
\end{pgfscope}%
\begin{pgfscope}%
\definecolor{textcolor}{rgb}{0.150000,0.150000,0.150000}%
\pgfsetstrokecolor{textcolor}%
\pgfsetfillcolor{textcolor}%
\pgftext[x=1.658843in,y=2.191691in,,top]{\color{textcolor}{\sffamily\fontsize{11.000000}{13.200000}\selectfont\catcode`\^=\active\def^{\ifmmode\sp\else\^{}\fi}\catcode`\%=\active\def%{\%}200}}%
\end{pgfscope}%
\begin{pgfscope}%
\pgfpathrectangle{\pgfqpoint{0.736295in}{2.323635in}}{\pgfqpoint{4.683705in}{0.996365in}}%
\pgfusepath{clip}%
\pgfsetroundcap%
\pgfsetroundjoin%
\pgfsetlinewidth{1.003750pt}%
\definecolor{currentstroke}{rgb}{1.000000,1.000000,1.000000}%
\pgfsetstrokecolor{currentstroke}%
\pgfsetdash{}{0pt}%
\pgfpathmoveto{\pgfqpoint{2.368495in}{2.323635in}}%
\pgfpathlineto{\pgfqpoint{2.368495in}{3.320000in}}%
\pgfusepath{stroke}%
\end{pgfscope}%
\begin{pgfscope}%
\definecolor{textcolor}{rgb}{0.150000,0.150000,0.150000}%
\pgfsetstrokecolor{textcolor}%
\pgfsetfillcolor{textcolor}%
\pgftext[x=2.368495in,y=2.191691in,,top]{\color{textcolor}{\sffamily\fontsize{11.000000}{13.200000}\selectfont\catcode`\^=\active\def^{\ifmmode\sp\else\^{}\fi}\catcode`\%=\active\def%{\%}400}}%
\end{pgfscope}%
\begin{pgfscope}%
\pgfpathrectangle{\pgfqpoint{0.736295in}{2.323635in}}{\pgfqpoint{4.683705in}{0.996365in}}%
\pgfusepath{clip}%
\pgfsetroundcap%
\pgfsetroundjoin%
\pgfsetlinewidth{1.003750pt}%
\definecolor{currentstroke}{rgb}{1.000000,1.000000,1.000000}%
\pgfsetstrokecolor{currentstroke}%
\pgfsetdash{}{0pt}%
\pgfpathmoveto{\pgfqpoint{3.078147in}{2.323635in}}%
\pgfpathlineto{\pgfqpoint{3.078147in}{3.320000in}}%
\pgfusepath{stroke}%
\end{pgfscope}%
\begin{pgfscope}%
\definecolor{textcolor}{rgb}{0.150000,0.150000,0.150000}%
\pgfsetstrokecolor{textcolor}%
\pgfsetfillcolor{textcolor}%
\pgftext[x=3.078147in,y=2.191691in,,top]{\color{textcolor}{\sffamily\fontsize{11.000000}{13.200000}\selectfont\catcode`\^=\active\def^{\ifmmode\sp\else\^{}\fi}\catcode`\%=\active\def%{\%}600}}%
\end{pgfscope}%
\begin{pgfscope}%
\pgfpathrectangle{\pgfqpoint{0.736295in}{2.323635in}}{\pgfqpoint{4.683705in}{0.996365in}}%
\pgfusepath{clip}%
\pgfsetroundcap%
\pgfsetroundjoin%
\pgfsetlinewidth{1.003750pt}%
\definecolor{currentstroke}{rgb}{1.000000,1.000000,1.000000}%
\pgfsetstrokecolor{currentstroke}%
\pgfsetdash{}{0pt}%
\pgfpathmoveto{\pgfqpoint{3.787800in}{2.323635in}}%
\pgfpathlineto{\pgfqpoint{3.787800in}{3.320000in}}%
\pgfusepath{stroke}%
\end{pgfscope}%
\begin{pgfscope}%
\definecolor{textcolor}{rgb}{0.150000,0.150000,0.150000}%
\pgfsetstrokecolor{textcolor}%
\pgfsetfillcolor{textcolor}%
\pgftext[x=3.787800in,y=2.191691in,,top]{\color{textcolor}{\sffamily\fontsize{11.000000}{13.200000}\selectfont\catcode`\^=\active\def^{\ifmmode\sp\else\^{}\fi}\catcode`\%=\active\def%{\%}800}}%
\end{pgfscope}%
\begin{pgfscope}%
\pgfpathrectangle{\pgfqpoint{0.736295in}{2.323635in}}{\pgfqpoint{4.683705in}{0.996365in}}%
\pgfusepath{clip}%
\pgfsetroundcap%
\pgfsetroundjoin%
\pgfsetlinewidth{1.003750pt}%
\definecolor{currentstroke}{rgb}{1.000000,1.000000,1.000000}%
\pgfsetstrokecolor{currentstroke}%
\pgfsetdash{}{0pt}%
\pgfpathmoveto{\pgfqpoint{4.497452in}{2.323635in}}%
\pgfpathlineto{\pgfqpoint{4.497452in}{3.320000in}}%
\pgfusepath{stroke}%
\end{pgfscope}%
\begin{pgfscope}%
\definecolor{textcolor}{rgb}{0.150000,0.150000,0.150000}%
\pgfsetstrokecolor{textcolor}%
\pgfsetfillcolor{textcolor}%
\pgftext[x=4.497452in,y=2.191691in,,top]{\color{textcolor}{\sffamily\fontsize{11.000000}{13.200000}\selectfont\catcode`\^=\active\def^{\ifmmode\sp\else\^{}\fi}\catcode`\%=\active\def%{\%}1000}}%
\end{pgfscope}%
\begin{pgfscope}%
\pgfpathrectangle{\pgfqpoint{0.736295in}{2.323635in}}{\pgfqpoint{4.683705in}{0.996365in}}%
\pgfusepath{clip}%
\pgfsetroundcap%
\pgfsetroundjoin%
\pgfsetlinewidth{1.003750pt}%
\definecolor{currentstroke}{rgb}{1.000000,1.000000,1.000000}%
\pgfsetstrokecolor{currentstroke}%
\pgfsetdash{}{0pt}%
\pgfpathmoveto{\pgfqpoint{5.207104in}{2.323635in}}%
\pgfpathlineto{\pgfqpoint{5.207104in}{3.320000in}}%
\pgfusepath{stroke}%
\end{pgfscope}%
\begin{pgfscope}%
\definecolor{textcolor}{rgb}{0.150000,0.150000,0.150000}%
\pgfsetstrokecolor{textcolor}%
\pgfsetfillcolor{textcolor}%
\pgftext[x=5.207104in,y=2.191691in,,top]{\color{textcolor}{\sffamily\fontsize{11.000000}{13.200000}\selectfont\catcode`\^=\active\def^{\ifmmode\sp\else\^{}\fi}\catcode`\%=\active\def%{\%}1200}}%
\end{pgfscope}%
\begin{pgfscope}%
\definecolor{textcolor}{rgb}{0.150000,0.150000,0.150000}%
\pgfsetstrokecolor{textcolor}%
\pgfsetfillcolor{textcolor}%
\pgftext[x=3.078147in,y=1.996413in,,top]{\color{textcolor}{\sffamily\fontsize{12.000000}{14.400000}\selectfont\catcode`\^=\active\def^{\ifmmode\sp\else\^{}\fi}\catcode`\%=\active\def%{\%}Time (s)}}%
\end{pgfscope}%
\begin{pgfscope}%
\pgfpathrectangle{\pgfqpoint{0.736295in}{2.323635in}}{\pgfqpoint{4.683705in}{0.996365in}}%
\pgfusepath{clip}%
\pgfsetroundcap%
\pgfsetroundjoin%
\pgfsetlinewidth{1.003750pt}%
\definecolor{currentstroke}{rgb}{1.000000,1.000000,1.000000}%
\pgfsetstrokecolor{currentstroke}%
\pgfsetdash{}{0pt}%
\pgfpathmoveto{\pgfqpoint{0.736295in}{2.406666in}}%
\pgfpathlineto{\pgfqpoint{5.420000in}{2.406666in}}%
\pgfusepath{stroke}%
\end{pgfscope}%
\begin{pgfscope}%
\definecolor{textcolor}{rgb}{0.150000,0.150000,0.150000}%
\pgfsetstrokecolor{textcolor}%
\pgfsetfillcolor{textcolor}%
\pgftext[x=0.391968in, y=2.351985in, left, base]{\color{textcolor}{\sffamily\fontsize{11.000000}{13.200000}\selectfont\catcode`\^=\active\def^{\ifmmode\sp\else\^{}\fi}\catcode`\%=\active\def%{\%}0.0}}%
\end{pgfscope}%
\begin{pgfscope}%
\pgfpathrectangle{\pgfqpoint{0.736295in}{2.323635in}}{\pgfqpoint{4.683705in}{0.996365in}}%
\pgfusepath{clip}%
\pgfsetroundcap%
\pgfsetroundjoin%
\pgfsetlinewidth{1.003750pt}%
\definecolor{currentstroke}{rgb}{1.000000,1.000000,1.000000}%
\pgfsetstrokecolor{currentstroke}%
\pgfsetdash{}{0pt}%
\pgfpathmoveto{\pgfqpoint{0.736295in}{2.821818in}}%
\pgfpathlineto{\pgfqpoint{5.420000in}{2.821818in}}%
\pgfusepath{stroke}%
\end{pgfscope}%
\begin{pgfscope}%
\definecolor{textcolor}{rgb}{0.150000,0.150000,0.150000}%
\pgfsetstrokecolor{textcolor}%
\pgfsetfillcolor{textcolor}%
\pgftext[x=0.391968in, y=2.767137in, left, base]{\color{textcolor}{\sffamily\fontsize{11.000000}{13.200000}\selectfont\catcode`\^=\active\def^{\ifmmode\sp\else\^{}\fi}\catcode`\%=\active\def%{\%}0.5}}%
\end{pgfscope}%
\begin{pgfscope}%
\pgfpathrectangle{\pgfqpoint{0.736295in}{2.323635in}}{\pgfqpoint{4.683705in}{0.996365in}}%
\pgfusepath{clip}%
\pgfsetroundcap%
\pgfsetroundjoin%
\pgfsetlinewidth{1.003750pt}%
\definecolor{currentstroke}{rgb}{1.000000,1.000000,1.000000}%
\pgfsetstrokecolor{currentstroke}%
\pgfsetdash{}{0pt}%
\pgfpathmoveto{\pgfqpoint{0.736295in}{3.236970in}}%
\pgfpathlineto{\pgfqpoint{5.420000in}{3.236970in}}%
\pgfusepath{stroke}%
\end{pgfscope}%
\begin{pgfscope}%
\definecolor{textcolor}{rgb}{0.150000,0.150000,0.150000}%
\pgfsetstrokecolor{textcolor}%
\pgfsetfillcolor{textcolor}%
\pgftext[x=0.391968in, y=3.182289in, left, base]{\color{textcolor}{\sffamily\fontsize{11.000000}{13.200000}\selectfont\catcode`\^=\active\def^{\ifmmode\sp\else\^{}\fi}\catcode`\%=\active\def%{\%}1.0}}%
\end{pgfscope}%
\begin{pgfscope}%
\definecolor{textcolor}{rgb}{0.150000,0.150000,0.150000}%
\pgfsetstrokecolor{textcolor}%
\pgfsetfillcolor{textcolor}%
\pgftext[x=0.336413in,y=2.821818in,,bottom,rotate=90.000000]{\color{textcolor}{\sffamily\fontsize{12.000000}{14.400000}\selectfont\catcode`\^=\active\def^{\ifmmode\sp\else\^{}\fi}\catcode`\%=\active\def%{\%}CPU Utilization (%)}}%
\end{pgfscope}%
\begin{pgfscope}%
\pgfpathrectangle{\pgfqpoint{0.736295in}{2.323635in}}{\pgfqpoint{4.683705in}{0.996365in}}%
\pgfusepath{clip}%
\pgfsetroundcap%
\pgfsetroundjoin%
\pgfsetlinewidth{1.505625pt}%
\definecolor{currentstroke}{rgb}{0.298039,0.447059,0.690196}%
\pgfsetstrokecolor{currentstroke}%
\pgfsetdash{}{0pt}%
\pgfpathmoveto{\pgfqpoint{0.949190in}{2.461330in}}%
\pgfpathlineto{\pgfqpoint{1.144345in}{2.460406in}}%
\pgfpathlineto{\pgfqpoint{1.179827in}{2.459655in}}%
\pgfpathlineto{\pgfqpoint{1.428206in}{2.459341in}}%
\pgfpathlineto{\pgfqpoint{1.445947in}{2.457568in}}%
\pgfpathlineto{\pgfqpoint{1.534653in}{2.457568in}}%
\pgfpathlineto{\pgfqpoint{1.552395in}{2.454686in}}%
\pgfpathlineto{\pgfqpoint{1.605619in}{2.454686in}}%
\pgfpathlineto{\pgfqpoint{1.623360in}{2.453504in}}%
\pgfpathlineto{\pgfqpoint{1.694325in}{2.453434in}}%
\pgfpathlineto{\pgfqpoint{1.712067in}{2.446276in}}%
\pgfpathlineto{\pgfqpoint{1.836256in}{2.447643in}}%
\pgfpathlineto{\pgfqpoint{1.853997in}{2.462667in}}%
\pgfpathlineto{\pgfqpoint{1.907221in}{2.462685in}}%
\pgfpathlineto{\pgfqpoint{1.924962in}{2.464368in}}%
\pgfpathlineto{\pgfqpoint{1.978186in}{2.464521in}}%
\pgfpathlineto{\pgfqpoint{1.995927in}{2.462376in}}%
\pgfpathlineto{\pgfqpoint{2.013669in}{2.462451in}}%
\pgfpathlineto{\pgfqpoint{2.031410in}{2.461045in}}%
\pgfpathlineto{\pgfqpoint{2.084634in}{2.461241in}}%
\pgfpathlineto{\pgfqpoint{2.102375in}{2.459602in}}%
\pgfpathlineto{\pgfqpoint{2.191082in}{2.459005in}}%
\pgfpathlineto{\pgfqpoint{2.244306in}{2.457727in}}%
\pgfpathlineto{\pgfqpoint{2.297530in}{2.458042in}}%
\pgfpathlineto{\pgfqpoint{2.350754in}{2.456965in}}%
\pgfpathlineto{\pgfqpoint{2.403978in}{2.457254in}}%
\pgfpathlineto{\pgfqpoint{2.457201in}{2.456644in}}%
\pgfpathlineto{\pgfqpoint{2.492684in}{2.456885in}}%
\pgfpathlineto{\pgfqpoint{2.545908in}{2.455912in}}%
\pgfpathlineto{\pgfqpoint{2.634615in}{2.454941in}}%
\pgfpathlineto{\pgfqpoint{2.652356in}{2.453077in}}%
\pgfpathlineto{\pgfqpoint{2.670097in}{2.453507in}}%
\pgfpathlineto{\pgfqpoint{2.687839in}{2.451254in}}%
\pgfpathlineto{\pgfqpoint{2.723321in}{2.451254in}}%
\pgfpathlineto{\pgfqpoint{2.741062in}{2.453592in}}%
\pgfpathlineto{\pgfqpoint{2.829769in}{2.453293in}}%
\pgfpathlineto{\pgfqpoint{3.202336in}{2.452847in}}%
\pgfpathlineto{\pgfqpoint{3.220078in}{2.454610in}}%
\pgfpathlineto{\pgfqpoint{3.273302in}{2.454398in}}%
\pgfpathlineto{\pgfqpoint{3.291043in}{2.452926in}}%
\pgfpathlineto{\pgfqpoint{3.308784in}{2.452827in}}%
\pgfpathlineto{\pgfqpoint{3.326526in}{2.459274in}}%
\pgfpathlineto{\pgfqpoint{3.379750in}{2.458990in}}%
\pgfpathlineto{\pgfqpoint{3.397491in}{2.456905in}}%
\pgfpathlineto{\pgfqpoint{3.787800in}{2.455749in}}%
\pgfpathlineto{\pgfqpoint{3.805541in}{2.458019in}}%
\pgfpathlineto{\pgfqpoint{3.858765in}{2.458208in}}%
\pgfpathlineto{\pgfqpoint{3.876506in}{2.456422in}}%
\pgfpathlineto{\pgfqpoint{4.781313in}{2.455947in}}%
\pgfpathlineto{\pgfqpoint{4.799054in}{2.454030in}}%
\pgfpathlineto{\pgfqpoint{4.852278in}{2.454030in}}%
\pgfpathlineto{\pgfqpoint{4.870019in}{2.456044in}}%
\pgfpathlineto{\pgfqpoint{5.207104in}{2.455924in}}%
\pgfpathlineto{\pgfqpoint{5.207104in}{2.455924in}}%
\pgfusepath{stroke}%
\end{pgfscope}%
\begin{pgfscope}%
\pgfpathrectangle{\pgfqpoint{0.736295in}{2.323635in}}{\pgfqpoint{4.683705in}{0.996365in}}%
\pgfusepath{clip}%
\pgfsetbuttcap%
\pgfsetroundjoin%
\pgfsetlinewidth{1.505625pt}%
\definecolor{currentstroke}{rgb}{1.000000,0.647059,0.000000}%
\pgfsetstrokecolor{currentstroke}%
\pgfsetdash{{5.550000pt}{2.400000pt}}{0.000000pt}%
\pgfpathmoveto{\pgfqpoint{1.026542in}{2.323635in}}%
\pgfpathlineto{\pgfqpoint{1.026542in}{3.320000in}}%
\pgfusepath{stroke}%
\end{pgfscope}%
\begin{pgfscope}%
\pgfsetrectcap%
\pgfsetmiterjoin%
\pgfsetlinewidth{1.254687pt}%
\definecolor{currentstroke}{rgb}{1.000000,1.000000,1.000000}%
\pgfsetstrokecolor{currentstroke}%
\pgfsetdash{}{0pt}%
\pgfpathmoveto{\pgfqpoint{0.736295in}{2.323635in}}%
\pgfpathlineto{\pgfqpoint{0.736295in}{3.320000in}}%
\pgfusepath{stroke}%
\end{pgfscope}%
\begin{pgfscope}%
\pgfsetrectcap%
\pgfsetmiterjoin%
\pgfsetlinewidth{1.254687pt}%
\definecolor{currentstroke}{rgb}{1.000000,1.000000,1.000000}%
\pgfsetstrokecolor{currentstroke}%
\pgfsetdash{}{0pt}%
\pgfpathmoveto{\pgfqpoint{5.420000in}{2.323635in}}%
\pgfpathlineto{\pgfqpoint{5.420000in}{3.320000in}}%
\pgfusepath{stroke}%
\end{pgfscope}%
\begin{pgfscope}%
\pgfsetrectcap%
\pgfsetmiterjoin%
\pgfsetlinewidth{1.254687pt}%
\definecolor{currentstroke}{rgb}{1.000000,1.000000,1.000000}%
\pgfsetstrokecolor{currentstroke}%
\pgfsetdash{}{0pt}%
\pgfpathmoveto{\pgfqpoint{0.736295in}{2.323635in}}%
\pgfpathlineto{\pgfqpoint{5.420000in}{2.323635in}}%
\pgfusepath{stroke}%
\end{pgfscope}%
\begin{pgfscope}%
\pgfsetrectcap%
\pgfsetmiterjoin%
\pgfsetlinewidth{1.254687pt}%
\definecolor{currentstroke}{rgb}{1.000000,1.000000,1.000000}%
\pgfsetstrokecolor{currentstroke}%
\pgfsetdash{}{0pt}%
\pgfpathmoveto{\pgfqpoint{0.736295in}{3.320000in}}%
\pgfpathlineto{\pgfqpoint{5.420000in}{3.320000in}}%
\pgfusepath{stroke}%
\end{pgfscope}%
\begin{pgfscope}%
\pgfsetbuttcap%
\pgfsetmiterjoin%
\definecolor{currentfill}{rgb}{0.917647,0.917647,0.949020}%
\pgfsetfillcolor{currentfill}%
\pgfsetfillopacity{0.800000}%
\pgfsetlinewidth{1.003750pt}%
\definecolor{currentstroke}{rgb}{0.800000,0.800000,0.800000}%
\pgfsetstrokecolor{currentstroke}%
\pgfsetstrokeopacity{0.800000}%
\pgfsetdash{}{0pt}%
\pgfpathmoveto{\pgfqpoint{3.276163in}{3.071281in}}%
\pgfpathlineto{\pgfqpoint{5.342222in}{3.071281in}}%
\pgfpathquadraticcurveto{\pgfqpoint{5.364444in}{3.071281in}}{\pgfqpoint{5.364444in}{3.093503in}}%
\pgfpathlineto{\pgfqpoint{5.364444in}{3.242222in}}%
\pgfpathquadraticcurveto{\pgfqpoint{5.364444in}{3.264444in}}{\pgfqpoint{5.342222in}{3.264444in}}%
\pgfpathlineto{\pgfqpoint{3.276163in}{3.264444in}}%
\pgfpathquadraticcurveto{\pgfqpoint{3.253941in}{3.264444in}}{\pgfqpoint{3.253941in}{3.242222in}}%
\pgfpathlineto{\pgfqpoint{3.253941in}{3.093503in}}%
\pgfpathquadraticcurveto{\pgfqpoint{3.253941in}{3.071281in}}{\pgfqpoint{3.276163in}{3.071281in}}%
\pgfpathlineto{\pgfqpoint{3.276163in}{3.071281in}}%
\pgfpathclose%
\pgfusepath{stroke,fill}%
\end{pgfscope}%
\begin{pgfscope}%
\pgfsetbuttcap%
\pgfsetroundjoin%
\pgfsetlinewidth{1.505625pt}%
\definecolor{currentstroke}{rgb}{1.000000,0.647059,0.000000}%
\pgfsetstrokecolor{currentstroke}%
\pgfsetdash{{5.550000pt}{2.400000pt}}{0.000000pt}%
\pgfpathmoveto{\pgfqpoint{3.298385in}{3.177997in}}%
\pgfpathlineto{\pgfqpoint{3.409497in}{3.177997in}}%
\pgfpathlineto{\pgfqpoint{3.520608in}{3.177997in}}%
\pgfusepath{stroke}%
\end{pgfscope}%
\begin{pgfscope}%
\definecolor{textcolor}{rgb}{0.150000,0.150000,0.150000}%
\pgfsetstrokecolor{textcolor}%
\pgfsetfillcolor{textcolor}%
\pgftext[x=3.609497in,y=3.139108in,left,base]{\color{textcolor}{\sffamily\fontsize{8.000000}{9.600000}\selectfont\catcode`\^=\active\def^{\ifmmode\sp\else\^{}\fi}\catcode`\%=\active\def%{\%}start of elasticity strategy controller}}%
\end{pgfscope}%
\begin{pgfscope}%
\pgfsetbuttcap%
\pgfsetmiterjoin%
\definecolor{currentfill}{rgb}{0.917647,0.917647,0.949020}%
\pgfsetfillcolor{currentfill}%
\pgfsetlinewidth{0.000000pt}%
\definecolor{currentstroke}{rgb}{0.000000,0.000000,0.000000}%
\pgfsetstrokecolor{currentstroke}%
\pgfsetstrokeopacity{0.000000}%
\pgfsetdash{}{0pt}%
\pgfpathmoveto{\pgfqpoint{0.736295in}{0.663635in}}%
\pgfpathlineto{\pgfqpoint{5.420000in}{0.663635in}}%
\pgfpathlineto{\pgfqpoint{5.420000in}{1.660000in}}%
\pgfpathlineto{\pgfqpoint{0.736295in}{1.660000in}}%
\pgfpathlineto{\pgfqpoint{0.736295in}{0.663635in}}%
\pgfpathclose%
\pgfusepath{fill}%
\end{pgfscope}%
\begin{pgfscope}%
\pgfpathrectangle{\pgfqpoint{0.736295in}{0.663635in}}{\pgfqpoint{4.683705in}{0.996365in}}%
\pgfusepath{clip}%
\pgfsetroundcap%
\pgfsetroundjoin%
\pgfsetlinewidth{1.003750pt}%
\definecolor{currentstroke}{rgb}{1.000000,1.000000,1.000000}%
\pgfsetstrokecolor{currentstroke}%
\pgfsetdash{}{0pt}%
\pgfpathmoveto{\pgfqpoint{0.949190in}{0.663635in}}%
\pgfpathlineto{\pgfqpoint{0.949190in}{1.660000in}}%
\pgfusepath{stroke}%
\end{pgfscope}%
\begin{pgfscope}%
\definecolor{textcolor}{rgb}{0.150000,0.150000,0.150000}%
\pgfsetstrokecolor{textcolor}%
\pgfsetfillcolor{textcolor}%
\pgftext[x=0.949190in,y=0.531691in,,top]{\color{textcolor}{\sffamily\fontsize{11.000000}{13.200000}\selectfont\catcode`\^=\active\def^{\ifmmode\sp\else\^{}\fi}\catcode`\%=\active\def%{\%}0}}%
\end{pgfscope}%
\begin{pgfscope}%
\pgfpathrectangle{\pgfqpoint{0.736295in}{0.663635in}}{\pgfqpoint{4.683705in}{0.996365in}}%
\pgfusepath{clip}%
\pgfsetroundcap%
\pgfsetroundjoin%
\pgfsetlinewidth{1.003750pt}%
\definecolor{currentstroke}{rgb}{1.000000,1.000000,1.000000}%
\pgfsetstrokecolor{currentstroke}%
\pgfsetdash{}{0pt}%
\pgfpathmoveto{\pgfqpoint{1.658843in}{0.663635in}}%
\pgfpathlineto{\pgfqpoint{1.658843in}{1.660000in}}%
\pgfusepath{stroke}%
\end{pgfscope}%
\begin{pgfscope}%
\definecolor{textcolor}{rgb}{0.150000,0.150000,0.150000}%
\pgfsetstrokecolor{textcolor}%
\pgfsetfillcolor{textcolor}%
\pgftext[x=1.658843in,y=0.531691in,,top]{\color{textcolor}{\sffamily\fontsize{11.000000}{13.200000}\selectfont\catcode`\^=\active\def^{\ifmmode\sp\else\^{}\fi}\catcode`\%=\active\def%{\%}200}}%
\end{pgfscope}%
\begin{pgfscope}%
\pgfpathrectangle{\pgfqpoint{0.736295in}{0.663635in}}{\pgfqpoint{4.683705in}{0.996365in}}%
\pgfusepath{clip}%
\pgfsetroundcap%
\pgfsetroundjoin%
\pgfsetlinewidth{1.003750pt}%
\definecolor{currentstroke}{rgb}{1.000000,1.000000,1.000000}%
\pgfsetstrokecolor{currentstroke}%
\pgfsetdash{}{0pt}%
\pgfpathmoveto{\pgfqpoint{2.368495in}{0.663635in}}%
\pgfpathlineto{\pgfqpoint{2.368495in}{1.660000in}}%
\pgfusepath{stroke}%
\end{pgfscope}%
\begin{pgfscope}%
\definecolor{textcolor}{rgb}{0.150000,0.150000,0.150000}%
\pgfsetstrokecolor{textcolor}%
\pgfsetfillcolor{textcolor}%
\pgftext[x=2.368495in,y=0.531691in,,top]{\color{textcolor}{\sffamily\fontsize{11.000000}{13.200000}\selectfont\catcode`\^=\active\def^{\ifmmode\sp\else\^{}\fi}\catcode`\%=\active\def%{\%}400}}%
\end{pgfscope}%
\begin{pgfscope}%
\pgfpathrectangle{\pgfqpoint{0.736295in}{0.663635in}}{\pgfqpoint{4.683705in}{0.996365in}}%
\pgfusepath{clip}%
\pgfsetroundcap%
\pgfsetroundjoin%
\pgfsetlinewidth{1.003750pt}%
\definecolor{currentstroke}{rgb}{1.000000,1.000000,1.000000}%
\pgfsetstrokecolor{currentstroke}%
\pgfsetdash{}{0pt}%
\pgfpathmoveto{\pgfqpoint{3.078147in}{0.663635in}}%
\pgfpathlineto{\pgfqpoint{3.078147in}{1.660000in}}%
\pgfusepath{stroke}%
\end{pgfscope}%
\begin{pgfscope}%
\definecolor{textcolor}{rgb}{0.150000,0.150000,0.150000}%
\pgfsetstrokecolor{textcolor}%
\pgfsetfillcolor{textcolor}%
\pgftext[x=3.078147in,y=0.531691in,,top]{\color{textcolor}{\sffamily\fontsize{11.000000}{13.200000}\selectfont\catcode`\^=\active\def^{\ifmmode\sp\else\^{}\fi}\catcode`\%=\active\def%{\%}600}}%
\end{pgfscope}%
\begin{pgfscope}%
\pgfpathrectangle{\pgfqpoint{0.736295in}{0.663635in}}{\pgfqpoint{4.683705in}{0.996365in}}%
\pgfusepath{clip}%
\pgfsetroundcap%
\pgfsetroundjoin%
\pgfsetlinewidth{1.003750pt}%
\definecolor{currentstroke}{rgb}{1.000000,1.000000,1.000000}%
\pgfsetstrokecolor{currentstroke}%
\pgfsetdash{}{0pt}%
\pgfpathmoveto{\pgfqpoint{3.787800in}{0.663635in}}%
\pgfpathlineto{\pgfqpoint{3.787800in}{1.660000in}}%
\pgfusepath{stroke}%
\end{pgfscope}%
\begin{pgfscope}%
\definecolor{textcolor}{rgb}{0.150000,0.150000,0.150000}%
\pgfsetstrokecolor{textcolor}%
\pgfsetfillcolor{textcolor}%
\pgftext[x=3.787800in,y=0.531691in,,top]{\color{textcolor}{\sffamily\fontsize{11.000000}{13.200000}\selectfont\catcode`\^=\active\def^{\ifmmode\sp\else\^{}\fi}\catcode`\%=\active\def%{\%}800}}%
\end{pgfscope}%
\begin{pgfscope}%
\pgfpathrectangle{\pgfqpoint{0.736295in}{0.663635in}}{\pgfqpoint{4.683705in}{0.996365in}}%
\pgfusepath{clip}%
\pgfsetroundcap%
\pgfsetroundjoin%
\pgfsetlinewidth{1.003750pt}%
\definecolor{currentstroke}{rgb}{1.000000,1.000000,1.000000}%
\pgfsetstrokecolor{currentstroke}%
\pgfsetdash{}{0pt}%
\pgfpathmoveto{\pgfqpoint{4.497452in}{0.663635in}}%
\pgfpathlineto{\pgfqpoint{4.497452in}{1.660000in}}%
\pgfusepath{stroke}%
\end{pgfscope}%
\begin{pgfscope}%
\definecolor{textcolor}{rgb}{0.150000,0.150000,0.150000}%
\pgfsetstrokecolor{textcolor}%
\pgfsetfillcolor{textcolor}%
\pgftext[x=4.497452in,y=0.531691in,,top]{\color{textcolor}{\sffamily\fontsize{11.000000}{13.200000}\selectfont\catcode`\^=\active\def^{\ifmmode\sp\else\^{}\fi}\catcode`\%=\active\def%{\%}1000}}%
\end{pgfscope}%
\begin{pgfscope}%
\pgfpathrectangle{\pgfqpoint{0.736295in}{0.663635in}}{\pgfqpoint{4.683705in}{0.996365in}}%
\pgfusepath{clip}%
\pgfsetroundcap%
\pgfsetroundjoin%
\pgfsetlinewidth{1.003750pt}%
\definecolor{currentstroke}{rgb}{1.000000,1.000000,1.000000}%
\pgfsetstrokecolor{currentstroke}%
\pgfsetdash{}{0pt}%
\pgfpathmoveto{\pgfqpoint{5.207104in}{0.663635in}}%
\pgfpathlineto{\pgfqpoint{5.207104in}{1.660000in}}%
\pgfusepath{stroke}%
\end{pgfscope}%
\begin{pgfscope}%
\definecolor{textcolor}{rgb}{0.150000,0.150000,0.150000}%
\pgfsetstrokecolor{textcolor}%
\pgfsetfillcolor{textcolor}%
\pgftext[x=5.207104in,y=0.531691in,,top]{\color{textcolor}{\sffamily\fontsize{11.000000}{13.200000}\selectfont\catcode`\^=\active\def^{\ifmmode\sp\else\^{}\fi}\catcode`\%=\active\def%{\%}1200}}%
\end{pgfscope}%
\begin{pgfscope}%
\definecolor{textcolor}{rgb}{0.150000,0.150000,0.150000}%
\pgfsetstrokecolor{textcolor}%
\pgfsetfillcolor{textcolor}%
\pgftext[x=3.078147in,y=0.336413in,,top]{\color{textcolor}{\sffamily\fontsize{12.000000}{14.400000}\selectfont\catcode`\^=\active\def^{\ifmmode\sp\else\^{}\fi}\catcode`\%=\active\def%{\%}Time (s)}}%
\end{pgfscope}%
\begin{pgfscope}%
\pgfpathrectangle{\pgfqpoint{0.736295in}{0.663635in}}{\pgfqpoint{4.683705in}{0.996365in}}%
\pgfusepath{clip}%
\pgfsetroundcap%
\pgfsetroundjoin%
\pgfsetlinewidth{1.003750pt}%
\definecolor{currentstroke}{rgb}{1.000000,1.000000,1.000000}%
\pgfsetstrokecolor{currentstroke}%
\pgfsetdash{}{0pt}%
\pgfpathmoveto{\pgfqpoint{0.736295in}{0.746666in}}%
\pgfpathlineto{\pgfqpoint{5.420000in}{0.746666in}}%
\pgfusepath{stroke}%
\end{pgfscope}%
\begin{pgfscope}%
\definecolor{textcolor}{rgb}{0.150000,0.150000,0.150000}%
\pgfsetstrokecolor{textcolor}%
\pgfsetfillcolor{textcolor}%
\pgftext[x=0.391968in, y=0.691985in, left, base]{\color{textcolor}{\sffamily\fontsize{11.000000}{13.200000}\selectfont\catcode`\^=\active\def^{\ifmmode\sp\else\^{}\fi}\catcode`\%=\active\def%{\%}0.0}}%
\end{pgfscope}%
\begin{pgfscope}%
\pgfpathrectangle{\pgfqpoint{0.736295in}{0.663635in}}{\pgfqpoint{4.683705in}{0.996365in}}%
\pgfusepath{clip}%
\pgfsetroundcap%
\pgfsetroundjoin%
\pgfsetlinewidth{1.003750pt}%
\definecolor{currentstroke}{rgb}{1.000000,1.000000,1.000000}%
\pgfsetstrokecolor{currentstroke}%
\pgfsetdash{}{0pt}%
\pgfpathmoveto{\pgfqpoint{0.736295in}{1.161818in}}%
\pgfpathlineto{\pgfqpoint{5.420000in}{1.161818in}}%
\pgfusepath{stroke}%
\end{pgfscope}%
\begin{pgfscope}%
\definecolor{textcolor}{rgb}{0.150000,0.150000,0.150000}%
\pgfsetstrokecolor{textcolor}%
\pgfsetfillcolor{textcolor}%
\pgftext[x=0.391968in, y=1.107137in, left, base]{\color{textcolor}{\sffamily\fontsize{11.000000}{13.200000}\selectfont\catcode`\^=\active\def^{\ifmmode\sp\else\^{}\fi}\catcode`\%=\active\def%{\%}0.5}}%
\end{pgfscope}%
\begin{pgfscope}%
\pgfpathrectangle{\pgfqpoint{0.736295in}{0.663635in}}{\pgfqpoint{4.683705in}{0.996365in}}%
\pgfusepath{clip}%
\pgfsetroundcap%
\pgfsetroundjoin%
\pgfsetlinewidth{1.003750pt}%
\definecolor{currentstroke}{rgb}{1.000000,1.000000,1.000000}%
\pgfsetstrokecolor{currentstroke}%
\pgfsetdash{}{0pt}%
\pgfpathmoveto{\pgfqpoint{0.736295in}{1.576970in}}%
\pgfpathlineto{\pgfqpoint{5.420000in}{1.576970in}}%
\pgfusepath{stroke}%
\end{pgfscope}%
\begin{pgfscope}%
\definecolor{textcolor}{rgb}{0.150000,0.150000,0.150000}%
\pgfsetstrokecolor{textcolor}%
\pgfsetfillcolor{textcolor}%
\pgftext[x=0.391968in, y=1.522289in, left, base]{\color{textcolor}{\sffamily\fontsize{11.000000}{13.200000}\selectfont\catcode`\^=\active\def^{\ifmmode\sp\else\^{}\fi}\catcode`\%=\active\def%{\%}1.0}}%
\end{pgfscope}%
\begin{pgfscope}%
\definecolor{textcolor}{rgb}{0.150000,0.150000,0.150000}%
\pgfsetstrokecolor{textcolor}%
\pgfsetfillcolor{textcolor}%
\pgftext[x=0.336413in,y=1.161818in,,bottom,rotate=90.000000]{\color{textcolor}{\sffamily\fontsize{12.000000}{14.400000}\selectfont\catcode`\^=\active\def^{\ifmmode\sp\else\^{}\fi}\catcode`\%=\active\def%{\%}Memory Utilization (%)}}%
\end{pgfscope}%
\begin{pgfscope}%
\pgfpathrectangle{\pgfqpoint{0.736295in}{0.663635in}}{\pgfqpoint{4.683705in}{0.996365in}}%
\pgfusepath{clip}%
\pgfsetbuttcap%
\pgfsetmiterjoin%
\definecolor{currentfill}{rgb}{1.000000,0.000000,0.000000}%
\pgfsetfillcolor{currentfill}%
\pgfsetfillopacity{0.300000}%
\pgfsetlinewidth{1.003750pt}%
\definecolor{currentstroke}{rgb}{1.000000,0.000000,0.000000}%
\pgfsetstrokecolor{currentstroke}%
\pgfsetstrokeopacity{0.300000}%
\pgfsetdash{}{0pt}%
\pgfpathmoveto{\pgfqpoint{1.367885in}{0.663635in}}%
\pgfpathlineto{\pgfqpoint{1.367885in}{1.660000in}}%
\pgfpathlineto{\pgfqpoint{2.411074in}{1.660000in}}%
\pgfpathlineto{\pgfqpoint{2.411074in}{0.663635in}}%
\pgfpathlineto{\pgfqpoint{1.367885in}{0.663635in}}%
\pgfpathclose%
\pgfusepath{stroke,fill}%
\end{pgfscope}%
\begin{pgfscope}%
\pgfpathrectangle{\pgfqpoint{0.736295in}{0.663635in}}{\pgfqpoint{4.683705in}{0.996365in}}%
\pgfusepath{clip}%
\pgfsetroundcap%
\pgfsetroundjoin%
\pgfsetlinewidth{1.505625pt}%
\definecolor{currentstroke}{rgb}{0.298039,0.447059,0.690196}%
\pgfsetstrokecolor{currentstroke}%
\pgfsetdash{}{0pt}%
\pgfpathmoveto{\pgfqpoint{0.949190in}{1.175322in}}%
\pgfpathlineto{\pgfqpoint{1.179827in}{1.175111in}}%
\pgfpathlineto{\pgfqpoint{1.197569in}{1.177375in}}%
\pgfpathlineto{\pgfqpoint{1.374982in}{1.178370in}}%
\pgfpathlineto{\pgfqpoint{1.392723in}{1.418450in}}%
\pgfpathlineto{\pgfqpoint{1.445947in}{1.419151in}}%
\pgfpathlineto{\pgfqpoint{1.463688in}{1.576970in}}%
\pgfpathlineto{\pgfqpoint{2.386236in}{1.576970in}}%
\pgfpathlineto{\pgfqpoint{2.403978in}{1.402559in}}%
\pgfpathlineto{\pgfqpoint{2.457201in}{1.402035in}}%
\pgfpathlineto{\pgfqpoint{2.474943in}{1.400621in}}%
\pgfpathlineto{\pgfqpoint{2.510425in}{1.400621in}}%
\pgfpathlineto{\pgfqpoint{2.528167in}{1.404946in}}%
\pgfpathlineto{\pgfqpoint{2.563649in}{1.404946in}}%
\pgfpathlineto{\pgfqpoint{2.581391in}{1.401377in}}%
\pgfpathlineto{\pgfqpoint{2.953958in}{1.400924in}}%
\pgfpathlineto{\pgfqpoint{2.971699in}{1.402976in}}%
\pgfpathlineto{\pgfqpoint{3.024923in}{1.402976in}}%
\pgfpathlineto{\pgfqpoint{3.042665in}{1.400604in}}%
\pgfpathlineto{\pgfqpoint{3.078147in}{1.400604in}}%
\pgfpathlineto{\pgfqpoint{3.095889in}{1.404172in}}%
\pgfpathlineto{\pgfqpoint{3.131371in}{1.404172in}}%
\pgfpathlineto{\pgfqpoint{3.149113in}{1.402557in}}%
\pgfpathlineto{\pgfqpoint{3.184595in}{1.402557in}}%
\pgfpathlineto{\pgfqpoint{3.202336in}{1.400815in}}%
\pgfpathlineto{\pgfqpoint{3.397491in}{1.401466in}}%
\pgfpathlineto{\pgfqpoint{3.415232in}{1.402735in}}%
\pgfpathlineto{\pgfqpoint{3.610387in}{1.402549in}}%
\pgfpathlineto{\pgfqpoint{3.628128in}{1.405441in}}%
\pgfpathlineto{\pgfqpoint{3.787800in}{1.405651in}}%
\pgfpathlineto{\pgfqpoint{3.805541in}{1.404133in}}%
\pgfpathlineto{\pgfqpoint{4.089402in}{1.404731in}}%
\pgfpathlineto{\pgfqpoint{4.107143in}{1.401121in}}%
\pgfpathlineto{\pgfqpoint{4.213591in}{1.401202in}}%
\pgfpathlineto{\pgfqpoint{4.231332in}{1.402726in}}%
\pgfpathlineto{\pgfqpoint{4.266815in}{1.402726in}}%
\pgfpathlineto{\pgfqpoint{4.284556in}{1.404362in}}%
\pgfpathlineto{\pgfqpoint{4.568417in}{1.405176in}}%
\pgfpathlineto{\pgfqpoint{4.586159in}{1.401940in}}%
\pgfpathlineto{\pgfqpoint{4.692606in}{1.401941in}}%
\pgfpathlineto{\pgfqpoint{4.710348in}{1.407260in}}%
\pgfpathlineto{\pgfqpoint{4.745830in}{1.407261in}}%
\pgfpathlineto{\pgfqpoint{4.763572in}{1.404999in}}%
\pgfpathlineto{\pgfqpoint{4.816796in}{1.404999in}}%
\pgfpathlineto{\pgfqpoint{4.834537in}{1.402270in}}%
\pgfpathlineto{\pgfqpoint{4.976467in}{1.402540in}}%
\pgfpathlineto{\pgfqpoint{4.994209in}{1.404280in}}%
\pgfpathlineto{\pgfqpoint{5.207104in}{1.403701in}}%
\pgfpathlineto{\pgfqpoint{5.207104in}{1.403701in}}%
\pgfusepath{stroke}%
\end{pgfscope}%
\begin{pgfscope}%
\pgfpathrectangle{\pgfqpoint{0.736295in}{0.663635in}}{\pgfqpoint{4.683705in}{0.996365in}}%
\pgfusepath{clip}%
\pgfsetbuttcap%
\pgfsetroundjoin%
\pgfsetlinewidth{1.505625pt}%
\definecolor{currentstroke}{rgb}{0.172549,0.627451,0.172549}%
\pgfsetstrokecolor{currentstroke}%
\pgfsetdash{{5.550000pt}{2.400000pt}}{0.000000pt}%
\pgfpathmoveto{\pgfqpoint{0.736295in}{1.327878in}}%
\pgfpathlineto{\pgfqpoint{5.420000in}{1.327878in}}%
\pgfusepath{stroke}%
\end{pgfscope}%
\begin{pgfscope}%
\pgfpathrectangle{\pgfqpoint{0.736295in}{0.663635in}}{\pgfqpoint{4.683705in}{0.996365in}}%
\pgfusepath{clip}%
\pgfsetbuttcap%
\pgfsetroundjoin%
\pgfsetlinewidth{1.505625pt}%
\definecolor{currentstroke}{rgb}{1.000000,0.647059,0.000000}%
\pgfsetstrokecolor{currentstroke}%
\pgfsetdash{{5.550000pt}{2.400000pt}}{0.000000pt}%
\pgfpathmoveto{\pgfqpoint{1.026542in}{0.663635in}}%
\pgfpathlineto{\pgfqpoint{1.026542in}{1.660000in}}%
\pgfusepath{stroke}%
\end{pgfscope}%
\begin{pgfscope}%
\pgfsetrectcap%
\pgfsetmiterjoin%
\pgfsetlinewidth{1.254687pt}%
\definecolor{currentstroke}{rgb}{1.000000,1.000000,1.000000}%
\pgfsetstrokecolor{currentstroke}%
\pgfsetdash{}{0pt}%
\pgfpathmoveto{\pgfqpoint{0.736295in}{0.663635in}}%
\pgfpathlineto{\pgfqpoint{0.736295in}{1.660000in}}%
\pgfusepath{stroke}%
\end{pgfscope}%
\begin{pgfscope}%
\pgfsetrectcap%
\pgfsetmiterjoin%
\pgfsetlinewidth{1.254687pt}%
\definecolor{currentstroke}{rgb}{1.000000,1.000000,1.000000}%
\pgfsetstrokecolor{currentstroke}%
\pgfsetdash{}{0pt}%
\pgfpathmoveto{\pgfqpoint{5.420000in}{0.663635in}}%
\pgfpathlineto{\pgfqpoint{5.420000in}{1.660000in}}%
\pgfusepath{stroke}%
\end{pgfscope}%
\begin{pgfscope}%
\pgfsetrectcap%
\pgfsetmiterjoin%
\pgfsetlinewidth{1.254687pt}%
\definecolor{currentstroke}{rgb}{1.000000,1.000000,1.000000}%
\pgfsetstrokecolor{currentstroke}%
\pgfsetdash{}{0pt}%
\pgfpathmoveto{\pgfqpoint{0.736295in}{0.663635in}}%
\pgfpathlineto{\pgfqpoint{5.420000in}{0.663635in}}%
\pgfusepath{stroke}%
\end{pgfscope}%
\begin{pgfscope}%
\pgfsetrectcap%
\pgfsetmiterjoin%
\pgfsetlinewidth{1.254687pt}%
\definecolor{currentstroke}{rgb}{1.000000,1.000000,1.000000}%
\pgfsetstrokecolor{currentstroke}%
\pgfsetdash{}{0pt}%
\pgfpathmoveto{\pgfqpoint{0.736295in}{1.660000in}}%
\pgfpathlineto{\pgfqpoint{5.420000in}{1.660000in}}%
\pgfusepath{stroke}%
\end{pgfscope}%
\begin{pgfscope}%
\pgfsetbuttcap%
\pgfsetmiterjoin%
\definecolor{currentfill}{rgb}{0.917647,0.917647,0.949020}%
\pgfsetfillcolor{currentfill}%
\pgfsetfillopacity{0.800000}%
\pgfsetlinewidth{1.003750pt}%
\definecolor{currentstroke}{rgb}{0.800000,0.800000,0.800000}%
\pgfsetstrokecolor{currentstroke}%
\pgfsetstrokeopacity{0.800000}%
\pgfsetdash{}{0pt}%
\pgfpathmoveto{\pgfqpoint{3.276163in}{0.719191in}}%
\pgfpathlineto{\pgfqpoint{5.342222in}{0.719191in}}%
\pgfpathquadraticcurveto{\pgfqpoint{5.364444in}{0.719191in}}{\pgfqpoint{5.364444in}{0.741413in}}%
\pgfpathlineto{\pgfqpoint{5.364444in}{1.205779in}}%
\pgfpathquadraticcurveto{\pgfqpoint{5.364444in}{1.228001in}}{\pgfqpoint{5.342222in}{1.228001in}}%
\pgfpathlineto{\pgfqpoint{3.276163in}{1.228001in}}%
\pgfpathquadraticcurveto{\pgfqpoint{3.253941in}{1.228001in}}{\pgfqpoint{3.253941in}{1.205779in}}%
\pgfpathlineto{\pgfqpoint{3.253941in}{0.741413in}}%
\pgfpathquadraticcurveto{\pgfqpoint{3.253941in}{0.719191in}}{\pgfqpoint{3.276163in}{0.719191in}}%
\pgfpathlineto{\pgfqpoint{3.276163in}{0.719191in}}%
\pgfpathclose%
\pgfusepath{stroke,fill}%
\end{pgfscope}%
\begin{pgfscope}%
\pgfsetbuttcap%
\pgfsetmiterjoin%
\definecolor{currentfill}{rgb}{1.000000,0.000000,0.000000}%
\pgfsetfillcolor{currentfill}%
\pgfsetfillopacity{0.300000}%
\pgfsetlinewidth{1.003750pt}%
\definecolor{currentstroke}{rgb}{1.000000,0.000000,0.000000}%
\pgfsetstrokecolor{currentstroke}%
\pgfsetstrokeopacity{0.300000}%
\pgfsetdash{}{0pt}%
\pgfpathmoveto{\pgfqpoint{3.298385in}{1.104021in}}%
\pgfpathlineto{\pgfqpoint{3.520608in}{1.104021in}}%
\pgfpathlineto{\pgfqpoint{3.520608in}{1.181799in}}%
\pgfpathlineto{\pgfqpoint{3.298385in}{1.181799in}}%
\pgfpathlineto{\pgfqpoint{3.298385in}{1.104021in}}%
\pgfpathclose%
\pgfusepath{stroke,fill}%
\end{pgfscope}%
\begin{pgfscope}%
\definecolor{textcolor}{rgb}{0.150000,0.150000,0.150000}%
\pgfsetstrokecolor{textcolor}%
\pgfsetfillcolor{textcolor}%
\pgftext[x=3.609497in,y=1.104021in,left,base]{\color{textcolor}{\sffamily\fontsize{8.000000}{9.600000}\selectfont\catcode`\^=\active\def^{\ifmmode\sp\else\^{}\fi}\catcode`\%=\active\def%{\%}k8ssandra reconsiliation}}%
\end{pgfscope}%
\begin{pgfscope}%
\pgfsetbuttcap%
\pgfsetroundjoin%
\pgfsetlinewidth{1.505625pt}%
\definecolor{currentstroke}{rgb}{0.172549,0.627451,0.172549}%
\pgfsetstrokecolor{currentstroke}%
\pgfsetdash{{5.550000pt}{2.400000pt}}{0.000000pt}%
\pgfpathmoveto{\pgfqpoint{3.298385in}{0.985738in}}%
\pgfpathlineto{\pgfqpoint{3.409497in}{0.985738in}}%
\pgfpathlineto{\pgfqpoint{3.520608in}{0.985738in}}%
\pgfusepath{stroke}%
\end{pgfscope}%
\begin{pgfscope}%
\definecolor{textcolor}{rgb}{0.150000,0.150000,0.150000}%
\pgfsetstrokecolor{textcolor}%
\pgfsetfillcolor{textcolor}%
\pgftext[x=3.609497in,y=0.946849in,left,base]{\color{textcolor}{\sffamily\fontsize{8.000000}{9.600000}\selectfont\catcode`\^=\active\def^{\ifmmode\sp\else\^{}\fi}\catcode`\%=\active\def%{\%}target memory utilization}}%
\end{pgfscope}%
\begin{pgfscope}%
\pgfsetbuttcap%
\pgfsetroundjoin%
\pgfsetlinewidth{1.505625pt}%
\definecolor{currentstroke}{rgb}{1.000000,0.647059,0.000000}%
\pgfsetstrokecolor{currentstroke}%
\pgfsetdash{{5.550000pt}{2.400000pt}}{0.000000pt}%
\pgfpathmoveto{\pgfqpoint{3.298385in}{0.825907in}}%
\pgfpathlineto{\pgfqpoint{3.409497in}{0.825907in}}%
\pgfpathlineto{\pgfqpoint{3.520608in}{0.825907in}}%
\pgfusepath{stroke}%
\end{pgfscope}%
\begin{pgfscope}%
\definecolor{textcolor}{rgb}{0.150000,0.150000,0.150000}%
\pgfsetstrokecolor{textcolor}%
\pgfsetfillcolor{textcolor}%
\pgftext[x=3.609497in,y=0.787018in,left,base]{\color{textcolor}{\sffamily\fontsize{8.000000}{9.600000}\selectfont\catcode`\^=\active\def^{\ifmmode\sp\else\^{}\fi}\catcode`\%=\active\def%{\%}start of elasticity strategy controller}}%
\end{pgfscope}%
\end{pgfpicture}%
\makeatother%
\endgroup%

    \caption{Utilization of CPU and memory during an vertical scaling action}
    \label{fig:utilization-vertical}
\end{figure}

\subsection{Horizontal Elasticity Strategy}
\label{sec:evaluation-horizontal-elasticity}

The horizontal elasticity strategy controller scales the target k8ssandra cluster horizontally, thus adding nodes as demand increases. Demand is measured as write throughput by the metrics controller as described in \cref{sec:metrics-average-write-utilization}.

As in the example stress tests discussed in \cref{sec:stress-testing}, \texttt{cassandra-stress} was used to generate load on the target k8ssandra cluster. During this load generation process, the horizontal elasticity controller was running. The target write load per node was defined in the SLO mapping as 5000. Depicted in \cref{fig:horizontal-elasticity} is the average write load per node metric and the corresponding node count during the testing process. It can be seen that the node count does not increase immediatly when the scaling action takes place. That is because when the k8ssandra CRD is updated by the elasticity strategy controller, first the \texttt{k8ssandra-operator} has to recognize the made changes and adjust the configuration accordingly. When the second k8ssandra node is successfully scheduled it still needs time to start and finally register in the cluster. The final action is the Cassandra reconciliation process.

At approximately 290s a sudden drop in the metric can be observed. This is the point when the scaling action becomes effective and the k8ssandra node is ready. Then, after another few moments the metric drops under the set boundary of 5000. Tests of this kind are difficult to run over an extended period of time because of a limitation of \texttt{cassandra-stress}. When the load generator is started, it collects all available nodes in the cluster through Cassandra's communication protocol \texttt{Gossip}. \texttt{Gossip} is the protocol that Cassandra uses internally for its nodes to communicate with each other\footnote{\raggedright\url{https://docs.datastax.com/en/cassandra-oss/3.x/cassandra/architecture/archGossipAbout.html}}. While \texttt{cassandra-stress} is running, new nodes are not recognized and requests are therefore not sent to added nodes. Possible solutions to this will be discussed in \cref{ch:conclusion}.

\begin{figure}
    \centering
    %% Creator: Matplotlib, PGF backend
%%
%% To include the figure in your LaTeX document, write
%%   \input{<filename>.pgf}
%%
%% Make sure the required packages are loaded in your preamble
%%   \usepackage{pgf}
%%
%% Also ensure that all the required font packages are loaded; for instance,
%% the lmodern package is sometimes necessary when using math font.
%%   \usepackage{lmodern}
%%
%% Figures using additional raster images can only be included by \input if
%% they are in the same directory as the main LaTeX file. For loading figures
%% from other directories you can use the `import` package
%%   \usepackage{import}
%%
%% and then include the figures with
%%   \import{<path to file>}{<filename>.pgf}
%%
%% Matplotlib used the following preamble
%%   \def\mathdefault#1{#1}
%%   \everymath=\expandafter{\the\everymath\displaystyle}
%%   
%%   \usepackage{fontspec}
%%   \setmainfont{DejaVuSerif.ttf}[Path=\detokenize{/Users/nkratky/private/polaris-elasticity-strategies/test/scripts/.venv/lib/python3.11/site-packages/matplotlib/mpl-data/fonts/ttf/}]
%%   \setsansfont{Arial.ttf}[Path=\detokenize{/System/Library/Fonts/Supplemental/}]
%%   \setmonofont{DejaVuSansMono.ttf}[Path=\detokenize{/Users/nkratky/private/polaris-elasticity-strategies/test/scripts/.venv/lib/python3.11/site-packages/matplotlib/mpl-data/fonts/ttf/}]
%%   \makeatletter\@ifpackageloaded{underscore}{}{\usepackage[strings]{underscore}}\makeatother
%%
\begingroup%
\makeatletter%
\begin{pgfpicture}%
\pgfpathrectangle{\pgfpointorigin}{\pgfqpoint{5.600000in}{3.000000in}}%
\pgfusepath{use as bounding box, clip}%
\begin{pgfscope}%
\pgfsetbuttcap%
\pgfsetmiterjoin%
\definecolor{currentfill}{rgb}{1.000000,1.000000,1.000000}%
\pgfsetfillcolor{currentfill}%
\pgfsetlinewidth{0.000000pt}%
\definecolor{currentstroke}{rgb}{1.000000,1.000000,1.000000}%
\pgfsetstrokecolor{currentstroke}%
\pgfsetdash{}{0pt}%
\pgfpathmoveto{\pgfqpoint{0.000000in}{0.000000in}}%
\pgfpathlineto{\pgfqpoint{5.600000in}{0.000000in}}%
\pgfpathlineto{\pgfqpoint{5.600000in}{3.000000in}}%
\pgfpathlineto{\pgfqpoint{0.000000in}{3.000000in}}%
\pgfpathlineto{\pgfqpoint{0.000000in}{0.000000in}}%
\pgfpathclose%
\pgfusepath{fill}%
\end{pgfscope}%
\begin{pgfscope}%
\pgfsetbuttcap%
\pgfsetmiterjoin%
\definecolor{currentfill}{rgb}{0.917647,0.917647,0.949020}%
\pgfsetfillcolor{currentfill}%
\pgfsetlinewidth{0.000000pt}%
\definecolor{currentstroke}{rgb}{0.000000,0.000000,0.000000}%
\pgfsetstrokecolor{currentstroke}%
\pgfsetstrokeopacity{0.000000}%
\pgfsetdash{}{0pt}%
\pgfpathmoveto{\pgfqpoint{0.946717in}{0.663635in}}%
\pgfpathlineto{\pgfqpoint{2.880912in}{0.663635in}}%
\pgfpathlineto{\pgfqpoint{2.880912in}{2.820000in}}%
\pgfpathlineto{\pgfqpoint{0.946717in}{2.820000in}}%
\pgfpathlineto{\pgfqpoint{0.946717in}{0.663635in}}%
\pgfpathclose%
\pgfusepath{fill}%
\end{pgfscope}%
\begin{pgfscope}%
\pgfpathrectangle{\pgfqpoint{0.946717in}{0.663635in}}{\pgfqpoint{1.934195in}{2.156365in}}%
\pgfusepath{clip}%
\pgfsetroundcap%
\pgfsetroundjoin%
\pgfsetlinewidth{1.003750pt}%
\definecolor{currentstroke}{rgb}{1.000000,1.000000,1.000000}%
\pgfsetstrokecolor{currentstroke}%
\pgfsetdash{}{0pt}%
\pgfpathmoveto{\pgfqpoint{1.034635in}{0.663635in}}%
\pgfpathlineto{\pgfqpoint{1.034635in}{2.820000in}}%
\pgfusepath{stroke}%
\end{pgfscope}%
\begin{pgfscope}%
\definecolor{textcolor}{rgb}{0.150000,0.150000,0.150000}%
\pgfsetstrokecolor{textcolor}%
\pgfsetfillcolor{textcolor}%
\pgftext[x=1.034635in,y=0.531691in,,top]{\color{textcolor}{\sffamily\fontsize{11.000000}{13.200000}\selectfont\catcode`\^=\active\def^{\ifmmode\sp\else\^{}\fi}\catcode`\%=\active\def%{\%}0}}%
\end{pgfscope}%
\begin{pgfscope}%
\pgfpathrectangle{\pgfqpoint{0.946717in}{0.663635in}}{\pgfqpoint{1.934195in}{2.156365in}}%
\pgfusepath{clip}%
\pgfsetroundcap%
\pgfsetroundjoin%
\pgfsetlinewidth{1.003750pt}%
\definecolor{currentstroke}{rgb}{1.000000,1.000000,1.000000}%
\pgfsetstrokecolor{currentstroke}%
\pgfsetdash{}{0pt}%
\pgfpathmoveto{\pgfqpoint{1.674038in}{0.663635in}}%
\pgfpathlineto{\pgfqpoint{1.674038in}{2.820000in}}%
\pgfusepath{stroke}%
\end{pgfscope}%
\begin{pgfscope}%
\definecolor{textcolor}{rgb}{0.150000,0.150000,0.150000}%
\pgfsetstrokecolor{textcolor}%
\pgfsetfillcolor{textcolor}%
\pgftext[x=1.674038in,y=0.531691in,,top]{\color{textcolor}{\sffamily\fontsize{11.000000}{13.200000}\selectfont\catcode`\^=\active\def^{\ifmmode\sp\else\^{}\fi}\catcode`\%=\active\def%{\%}200}}%
\end{pgfscope}%
\begin{pgfscope}%
\pgfpathrectangle{\pgfqpoint{0.946717in}{0.663635in}}{\pgfqpoint{1.934195in}{2.156365in}}%
\pgfusepath{clip}%
\pgfsetroundcap%
\pgfsetroundjoin%
\pgfsetlinewidth{1.003750pt}%
\definecolor{currentstroke}{rgb}{1.000000,1.000000,1.000000}%
\pgfsetstrokecolor{currentstroke}%
\pgfsetdash{}{0pt}%
\pgfpathmoveto{\pgfqpoint{2.313441in}{0.663635in}}%
\pgfpathlineto{\pgfqpoint{2.313441in}{2.820000in}}%
\pgfusepath{stroke}%
\end{pgfscope}%
\begin{pgfscope}%
\definecolor{textcolor}{rgb}{0.150000,0.150000,0.150000}%
\pgfsetstrokecolor{textcolor}%
\pgfsetfillcolor{textcolor}%
\pgftext[x=2.313441in,y=0.531691in,,top]{\color{textcolor}{\sffamily\fontsize{11.000000}{13.200000}\selectfont\catcode`\^=\active\def^{\ifmmode\sp\else\^{}\fi}\catcode`\%=\active\def%{\%}400}}%
\end{pgfscope}%
\begin{pgfscope}%
\definecolor{textcolor}{rgb}{0.150000,0.150000,0.150000}%
\pgfsetstrokecolor{textcolor}%
\pgfsetfillcolor{textcolor}%
\pgftext[x=1.913814in,y=0.336413in,,top]{\color{textcolor}{\sffamily\fontsize{12.000000}{14.400000}\selectfont\catcode`\^=\active\def^{\ifmmode\sp\else\^{}\fi}\catcode`\%=\active\def%{\%}Time (s)}}%
\end{pgfscope}%
\begin{pgfscope}%
\pgfpathrectangle{\pgfqpoint{0.946717in}{0.663635in}}{\pgfqpoint{1.934195in}{2.156365in}}%
\pgfusepath{clip}%
\pgfsetroundcap%
\pgfsetroundjoin%
\pgfsetlinewidth{1.003750pt}%
\definecolor{currentstroke}{rgb}{1.000000,1.000000,1.000000}%
\pgfsetstrokecolor{currentstroke}%
\pgfsetdash{}{0pt}%
\pgfpathmoveto{\pgfqpoint{0.946717in}{0.761652in}}%
\pgfpathlineto{\pgfqpoint{2.880912in}{0.761652in}}%
\pgfusepath{stroke}%
\end{pgfscope}%
\begin{pgfscope}%
\definecolor{textcolor}{rgb}{0.150000,0.150000,0.150000}%
\pgfsetstrokecolor{textcolor}%
\pgfsetfillcolor{textcolor}%
\pgftext[x=0.729804in, y=0.706971in, left, base]{\color{textcolor}{\sffamily\fontsize{11.000000}{13.200000}\selectfont\catcode`\^=\active\def^{\ifmmode\sp\else\^{}\fi}\catcode`\%=\active\def%{\%}0}}%
\end{pgfscope}%
\begin{pgfscope}%
\pgfpathrectangle{\pgfqpoint{0.946717in}{0.663635in}}{\pgfqpoint{1.934195in}{2.156365in}}%
\pgfusepath{clip}%
\pgfsetroundcap%
\pgfsetroundjoin%
\pgfsetlinewidth{1.003750pt}%
\definecolor{currentstroke}{rgb}{1.000000,1.000000,1.000000}%
\pgfsetstrokecolor{currentstroke}%
\pgfsetdash{}{0pt}%
\pgfpathmoveto{\pgfqpoint{0.946717in}{1.243426in}}%
\pgfpathlineto{\pgfqpoint{2.880912in}{1.243426in}}%
\pgfusepath{stroke}%
\end{pgfscope}%
\begin{pgfscope}%
\definecolor{textcolor}{rgb}{0.150000,0.150000,0.150000}%
\pgfsetstrokecolor{textcolor}%
\pgfsetfillcolor{textcolor}%
\pgftext[x=0.474901in, y=1.188746in, left, base]{\color{textcolor}{\sffamily\fontsize{11.000000}{13.200000}\selectfont\catcode`\^=\active\def^{\ifmmode\sp\else\^{}\fi}\catcode`\%=\active\def%{\%}5000}}%
\end{pgfscope}%
\begin{pgfscope}%
\pgfpathrectangle{\pgfqpoint{0.946717in}{0.663635in}}{\pgfqpoint{1.934195in}{2.156365in}}%
\pgfusepath{clip}%
\pgfsetroundcap%
\pgfsetroundjoin%
\pgfsetlinewidth{1.003750pt}%
\definecolor{currentstroke}{rgb}{1.000000,1.000000,1.000000}%
\pgfsetstrokecolor{currentstroke}%
\pgfsetdash{}{0pt}%
\pgfpathmoveto{\pgfqpoint{0.946717in}{1.725201in}}%
\pgfpathlineto{\pgfqpoint{2.880912in}{1.725201in}}%
\pgfusepath{stroke}%
\end{pgfscope}%
\begin{pgfscope}%
\definecolor{textcolor}{rgb}{0.150000,0.150000,0.150000}%
\pgfsetstrokecolor{textcolor}%
\pgfsetfillcolor{textcolor}%
\pgftext[x=0.389934in, y=1.670520in, left, base]{\color{textcolor}{\sffamily\fontsize{11.000000}{13.200000}\selectfont\catcode`\^=\active\def^{\ifmmode\sp\else\^{}\fi}\catcode`\%=\active\def%{\%}10000}}%
\end{pgfscope}%
\begin{pgfscope}%
\pgfpathrectangle{\pgfqpoint{0.946717in}{0.663635in}}{\pgfqpoint{1.934195in}{2.156365in}}%
\pgfusepath{clip}%
\pgfsetroundcap%
\pgfsetroundjoin%
\pgfsetlinewidth{1.003750pt}%
\definecolor{currentstroke}{rgb}{1.000000,1.000000,1.000000}%
\pgfsetstrokecolor{currentstroke}%
\pgfsetdash{}{0pt}%
\pgfpathmoveto{\pgfqpoint{0.946717in}{2.206975in}}%
\pgfpathlineto{\pgfqpoint{2.880912in}{2.206975in}}%
\pgfusepath{stroke}%
\end{pgfscope}%
\begin{pgfscope}%
\definecolor{textcolor}{rgb}{0.150000,0.150000,0.150000}%
\pgfsetstrokecolor{textcolor}%
\pgfsetfillcolor{textcolor}%
\pgftext[x=0.389934in, y=2.152295in, left, base]{\color{textcolor}{\sffamily\fontsize{11.000000}{13.200000}\selectfont\catcode`\^=\active\def^{\ifmmode\sp\else\^{}\fi}\catcode`\%=\active\def%{\%}15000}}%
\end{pgfscope}%
\begin{pgfscope}%
\pgfpathrectangle{\pgfqpoint{0.946717in}{0.663635in}}{\pgfqpoint{1.934195in}{2.156365in}}%
\pgfusepath{clip}%
\pgfsetroundcap%
\pgfsetroundjoin%
\pgfsetlinewidth{1.003750pt}%
\definecolor{currentstroke}{rgb}{1.000000,1.000000,1.000000}%
\pgfsetstrokecolor{currentstroke}%
\pgfsetdash{}{0pt}%
\pgfpathmoveto{\pgfqpoint{0.946717in}{2.688750in}}%
\pgfpathlineto{\pgfqpoint{2.880912in}{2.688750in}}%
\pgfusepath{stroke}%
\end{pgfscope}%
\begin{pgfscope}%
\definecolor{textcolor}{rgb}{0.150000,0.150000,0.150000}%
\pgfsetstrokecolor{textcolor}%
\pgfsetfillcolor{textcolor}%
\pgftext[x=0.389934in, y=2.634069in, left, base]{\color{textcolor}{\sffamily\fontsize{11.000000}{13.200000}\selectfont\catcode`\^=\active\def^{\ifmmode\sp\else\^{}\fi}\catcode`\%=\active\def%{\%}20000}}%
\end{pgfscope}%
\begin{pgfscope}%
\definecolor{textcolor}{rgb}{0.150000,0.150000,0.150000}%
\pgfsetstrokecolor{textcolor}%
\pgfsetfillcolor{textcolor}%
\pgftext[x=0.334378in,y=1.741818in,,bottom,rotate=90.000000]{\color{textcolor}{\sffamily\fontsize{12.000000}{14.400000}\selectfont\catcode`\^=\active\def^{\ifmmode\sp\else\^{}\fi}\catcode`\%=\active\def%{\%}Average Write Load Per Node}}%
\end{pgfscope}%
\begin{pgfscope}%
\pgfpathrectangle{\pgfqpoint{0.946717in}{0.663635in}}{\pgfqpoint{1.934195in}{2.156365in}}%
\pgfusepath{clip}%
\pgfsetroundcap%
\pgfsetroundjoin%
\pgfsetlinewidth{1.505625pt}%
\definecolor{currentstroke}{rgb}{0.298039,0.447059,0.690196}%
\pgfsetstrokecolor{currentstroke}%
\pgfsetdash{}{0pt}%
\pgfpathmoveto{\pgfqpoint{1.034635in}{0.761652in}}%
\pgfpathlineto{\pgfqpoint{1.050620in}{0.761652in}}%
\pgfpathlineto{\pgfqpoint{1.066605in}{0.761652in}}%
\pgfpathlineto{\pgfqpoint{1.082590in}{0.761652in}}%
\pgfpathlineto{\pgfqpoint{1.098575in}{0.761652in}}%
\pgfpathlineto{\pgfqpoint{1.114560in}{0.761652in}}%
\pgfpathlineto{\pgfqpoint{1.130545in}{0.761652in}}%
\pgfpathlineto{\pgfqpoint{1.146530in}{0.803544in}}%
\pgfpathlineto{\pgfqpoint{1.162515in}{0.803544in}}%
\pgfpathlineto{\pgfqpoint{1.178500in}{0.803544in}}%
\pgfpathlineto{\pgfqpoint{1.194485in}{0.803544in}}%
\pgfpathlineto{\pgfqpoint{1.210471in}{0.800208in}}%
\pgfpathlineto{\pgfqpoint{1.226456in}{0.800208in}}%
\pgfpathlineto{\pgfqpoint{1.242441in}{0.800208in}}%
\pgfpathlineto{\pgfqpoint{1.258426in}{0.800208in}}%
\pgfpathlineto{\pgfqpoint{1.274411in}{0.966691in}}%
\pgfpathlineto{\pgfqpoint{1.290396in}{0.966691in}}%
\pgfpathlineto{\pgfqpoint{1.306381in}{0.966691in}}%
\pgfpathlineto{\pgfqpoint{1.322366in}{0.966691in}}%
\pgfpathlineto{\pgfqpoint{1.338351in}{1.445737in}}%
\pgfpathlineto{\pgfqpoint{1.354336in}{1.445737in}}%
\pgfpathlineto{\pgfqpoint{1.370321in}{1.445737in}}%
\pgfpathlineto{\pgfqpoint{1.386307in}{1.445737in}}%
\pgfpathlineto{\pgfqpoint{1.402292in}{1.627569in}}%
\pgfpathlineto{\pgfqpoint{1.418277in}{1.627569in}}%
\pgfpathlineto{\pgfqpoint{1.434262in}{1.627569in}}%
\pgfpathlineto{\pgfqpoint{1.450247in}{1.627569in}}%
\pgfpathlineto{\pgfqpoint{1.466232in}{1.645603in}}%
\pgfpathlineto{\pgfqpoint{1.482217in}{1.645603in}}%
\pgfpathlineto{\pgfqpoint{1.498202in}{1.645603in}}%
\pgfpathlineto{\pgfqpoint{1.514187in}{1.645603in}}%
\pgfpathlineto{\pgfqpoint{1.530172in}{2.007291in}}%
\pgfpathlineto{\pgfqpoint{1.546157in}{2.007291in}}%
\pgfpathlineto{\pgfqpoint{1.562142in}{2.007291in}}%
\pgfpathlineto{\pgfqpoint{1.578128in}{2.007291in}}%
\pgfpathlineto{\pgfqpoint{1.594113in}{2.136263in}}%
\pgfpathlineto{\pgfqpoint{1.610098in}{2.136263in}}%
\pgfpathlineto{\pgfqpoint{1.626083in}{2.136263in}}%
\pgfpathlineto{\pgfqpoint{1.642068in}{2.136263in}}%
\pgfpathlineto{\pgfqpoint{1.658053in}{2.176474in}}%
\pgfpathlineto{\pgfqpoint{1.674038in}{2.176474in}}%
\pgfpathlineto{\pgfqpoint{1.690023in}{2.176474in}}%
\pgfpathlineto{\pgfqpoint{1.706008in}{2.176474in}}%
\pgfpathlineto{\pgfqpoint{1.721993in}{2.529427in}}%
\pgfpathlineto{\pgfqpoint{1.737978in}{2.529427in}}%
\pgfpathlineto{\pgfqpoint{1.753963in}{2.529427in}}%
\pgfpathlineto{\pgfqpoint{1.769949in}{2.529427in}}%
\pgfpathlineto{\pgfqpoint{1.785934in}{2.652978in}}%
\pgfpathlineto{\pgfqpoint{1.801919in}{2.652978in}}%
\pgfpathlineto{\pgfqpoint{1.817904in}{2.652978in}}%
\pgfpathlineto{\pgfqpoint{1.833889in}{2.652978in}}%
\pgfpathlineto{\pgfqpoint{1.849874in}{2.721983in}}%
\pgfpathlineto{\pgfqpoint{1.865859in}{2.721983in}}%
\pgfpathlineto{\pgfqpoint{1.881844in}{2.721983in}}%
\pgfpathlineto{\pgfqpoint{1.897829in}{2.721983in}}%
\pgfpathlineto{\pgfqpoint{1.913814in}{2.475762in}}%
\pgfpathlineto{\pgfqpoint{1.929799in}{2.475762in}}%
\pgfpathlineto{\pgfqpoint{1.945785in}{2.475762in}}%
\pgfpathlineto{\pgfqpoint{1.961770in}{2.475762in}}%
\pgfpathlineto{\pgfqpoint{1.977755in}{1.658006in}}%
\pgfpathlineto{\pgfqpoint{1.993740in}{1.658006in}}%
\pgfpathlineto{\pgfqpoint{2.009725in}{1.658006in}}%
\pgfpathlineto{\pgfqpoint{2.025710in}{1.658006in}}%
\pgfpathlineto{\pgfqpoint{2.041695in}{1.653545in}}%
\pgfpathlineto{\pgfqpoint{2.057680in}{1.653545in}}%
\pgfpathlineto{\pgfqpoint{2.073665in}{1.653545in}}%
\pgfpathlineto{\pgfqpoint{2.089650in}{1.653545in}}%
\pgfpathlineto{\pgfqpoint{2.105635in}{1.561695in}}%
\pgfpathlineto{\pgfqpoint{2.121620in}{1.561695in}}%
\pgfpathlineto{\pgfqpoint{2.137606in}{1.561695in}}%
\pgfpathlineto{\pgfqpoint{2.153591in}{1.561695in}}%
\pgfpathlineto{\pgfqpoint{2.169576in}{1.423998in}}%
\pgfpathlineto{\pgfqpoint{2.185561in}{1.423998in}}%
\pgfpathlineto{\pgfqpoint{2.201546in}{1.423998in}}%
\pgfpathlineto{\pgfqpoint{2.217531in}{1.423998in}}%
\pgfpathlineto{\pgfqpoint{2.233516in}{1.358510in}}%
\pgfpathlineto{\pgfqpoint{2.249501in}{1.358510in}}%
\pgfpathlineto{\pgfqpoint{2.265486in}{1.358510in}}%
\pgfpathlineto{\pgfqpoint{2.281471in}{1.358510in}}%
\pgfpathlineto{\pgfqpoint{2.297456in}{1.295104in}}%
\pgfpathlineto{\pgfqpoint{2.313441in}{1.295104in}}%
\pgfpathlineto{\pgfqpoint{2.329427in}{1.295104in}}%
\pgfpathlineto{\pgfqpoint{2.345412in}{1.295104in}}%
\pgfpathlineto{\pgfqpoint{2.361397in}{1.162725in}}%
\pgfpathlineto{\pgfqpoint{2.377382in}{1.162725in}}%
\pgfpathlineto{\pgfqpoint{2.393367in}{1.162725in}}%
\pgfpathlineto{\pgfqpoint{2.409352in}{1.162725in}}%
\pgfpathlineto{\pgfqpoint{2.425337in}{1.069570in}}%
\pgfpathlineto{\pgfqpoint{2.441322in}{1.069570in}}%
\pgfpathlineto{\pgfqpoint{2.457307in}{1.069570in}}%
\pgfpathlineto{\pgfqpoint{2.473292in}{1.069570in}}%
\pgfpathlineto{\pgfqpoint{2.489277in}{1.037435in}}%
\pgfpathlineto{\pgfqpoint{2.505263in}{1.037435in}}%
\pgfpathlineto{\pgfqpoint{2.521248in}{1.037435in}}%
\pgfpathlineto{\pgfqpoint{2.537233in}{1.037435in}}%
\pgfpathlineto{\pgfqpoint{2.553218in}{0.957524in}}%
\pgfpathlineto{\pgfqpoint{2.569203in}{0.957524in}}%
\pgfpathlineto{\pgfqpoint{2.585188in}{0.957524in}}%
\pgfpathlineto{\pgfqpoint{2.601173in}{0.957524in}}%
\pgfpathlineto{\pgfqpoint{2.617158in}{0.898503in}}%
\pgfpathlineto{\pgfqpoint{2.633143in}{0.898503in}}%
\pgfpathlineto{\pgfqpoint{2.649128in}{0.898503in}}%
\pgfpathlineto{\pgfqpoint{2.665113in}{0.898503in}}%
\pgfpathlineto{\pgfqpoint{2.681098in}{0.884924in}}%
\pgfpathlineto{\pgfqpoint{2.697084in}{0.884924in}}%
\pgfpathlineto{\pgfqpoint{2.713069in}{0.884924in}}%
\pgfpathlineto{\pgfqpoint{2.729054in}{0.884924in}}%
\pgfpathlineto{\pgfqpoint{2.745039in}{0.770701in}}%
\pgfpathlineto{\pgfqpoint{2.761024in}{0.770701in}}%
\pgfpathlineto{\pgfqpoint{2.777009in}{0.770701in}}%
\pgfpathlineto{\pgfqpoint{2.792994in}{0.770701in}}%
\pgfusepath{stroke}%
\end{pgfscope}%
\begin{pgfscope}%
\pgfpathrectangle{\pgfqpoint{0.946717in}{0.663635in}}{\pgfqpoint{1.934195in}{2.156365in}}%
\pgfusepath{clip}%
\pgfsetbuttcap%
\pgfsetroundjoin%
\pgfsetlinewidth{1.505625pt}%
\definecolor{currentstroke}{rgb}{1.000000,0.647059,0.000000}%
\pgfsetstrokecolor{currentstroke}%
\pgfsetdash{{5.550000pt}{2.400000pt}}{0.000000pt}%
\pgfpathmoveto{\pgfqpoint{1.343466in}{0.663635in}}%
\pgfpathlineto{\pgfqpoint{1.343466in}{2.820000in}}%
\pgfusepath{stroke}%
\end{pgfscope}%
\begin{pgfscope}%
\pgfsetrectcap%
\pgfsetmiterjoin%
\pgfsetlinewidth{1.254687pt}%
\definecolor{currentstroke}{rgb}{1.000000,1.000000,1.000000}%
\pgfsetstrokecolor{currentstroke}%
\pgfsetdash{}{0pt}%
\pgfpathmoveto{\pgfqpoint{0.946717in}{0.663635in}}%
\pgfpathlineto{\pgfqpoint{0.946717in}{2.820000in}}%
\pgfusepath{stroke}%
\end{pgfscope}%
\begin{pgfscope}%
\pgfsetrectcap%
\pgfsetmiterjoin%
\pgfsetlinewidth{1.254687pt}%
\definecolor{currentstroke}{rgb}{1.000000,1.000000,1.000000}%
\pgfsetstrokecolor{currentstroke}%
\pgfsetdash{}{0pt}%
\pgfpathmoveto{\pgfqpoint{2.880912in}{0.663635in}}%
\pgfpathlineto{\pgfqpoint{2.880912in}{2.820000in}}%
\pgfusepath{stroke}%
\end{pgfscope}%
\begin{pgfscope}%
\pgfsetrectcap%
\pgfsetmiterjoin%
\pgfsetlinewidth{1.254687pt}%
\definecolor{currentstroke}{rgb}{1.000000,1.000000,1.000000}%
\pgfsetstrokecolor{currentstroke}%
\pgfsetdash{}{0pt}%
\pgfpathmoveto{\pgfqpoint{0.946717in}{0.663635in}}%
\pgfpathlineto{\pgfqpoint{2.880912in}{0.663635in}}%
\pgfusepath{stroke}%
\end{pgfscope}%
\begin{pgfscope}%
\pgfsetrectcap%
\pgfsetmiterjoin%
\pgfsetlinewidth{1.254687pt}%
\definecolor{currentstroke}{rgb}{1.000000,1.000000,1.000000}%
\pgfsetstrokecolor{currentstroke}%
\pgfsetdash{}{0pt}%
\pgfpathmoveto{\pgfqpoint{0.946717in}{2.820000in}}%
\pgfpathlineto{\pgfqpoint{2.880912in}{2.820000in}}%
\pgfusepath{stroke}%
\end{pgfscope}%
\begin{pgfscope}%
\pgfsetbuttcap%
\pgfsetmiterjoin%
\definecolor{currentfill}{rgb}{0.917647,0.917647,0.949020}%
\pgfsetfillcolor{currentfill}%
\pgfsetlinewidth{0.000000pt}%
\definecolor{currentstroke}{rgb}{0.000000,0.000000,0.000000}%
\pgfsetstrokecolor{currentstroke}%
\pgfsetstrokeopacity{0.000000}%
\pgfsetdash{}{0pt}%
\pgfpathmoveto{\pgfqpoint{3.485805in}{0.663635in}}%
\pgfpathlineto{\pgfqpoint{5.420000in}{0.663635in}}%
\pgfpathlineto{\pgfqpoint{5.420000in}{2.820000in}}%
\pgfpathlineto{\pgfqpoint{3.485805in}{2.820000in}}%
\pgfpathlineto{\pgfqpoint{3.485805in}{0.663635in}}%
\pgfpathclose%
\pgfusepath{fill}%
\end{pgfscope}%
\begin{pgfscope}%
\pgfpathrectangle{\pgfqpoint{3.485805in}{0.663635in}}{\pgfqpoint{1.934195in}{2.156365in}}%
\pgfusepath{clip}%
\pgfsetroundcap%
\pgfsetroundjoin%
\pgfsetlinewidth{1.003750pt}%
\definecolor{currentstroke}{rgb}{1.000000,1.000000,1.000000}%
\pgfsetstrokecolor{currentstroke}%
\pgfsetdash{}{0pt}%
\pgfpathmoveto{\pgfqpoint{3.573723in}{0.663635in}}%
\pgfpathlineto{\pgfqpoint{3.573723in}{2.820000in}}%
\pgfusepath{stroke}%
\end{pgfscope}%
\begin{pgfscope}%
\definecolor{textcolor}{rgb}{0.150000,0.150000,0.150000}%
\pgfsetstrokecolor{textcolor}%
\pgfsetfillcolor{textcolor}%
\pgftext[x=3.573723in,y=0.531691in,,top]{\color{textcolor}{\sffamily\fontsize{11.000000}{13.200000}\selectfont\catcode`\^=\active\def^{\ifmmode\sp\else\^{}\fi}\catcode`\%=\active\def%{\%}0}}%
\end{pgfscope}%
\begin{pgfscope}%
\pgfpathrectangle{\pgfqpoint{3.485805in}{0.663635in}}{\pgfqpoint{1.934195in}{2.156365in}}%
\pgfusepath{clip}%
\pgfsetroundcap%
\pgfsetroundjoin%
\pgfsetlinewidth{1.003750pt}%
\definecolor{currentstroke}{rgb}{1.000000,1.000000,1.000000}%
\pgfsetstrokecolor{currentstroke}%
\pgfsetdash{}{0pt}%
\pgfpathmoveto{\pgfqpoint{4.190691in}{0.663635in}}%
\pgfpathlineto{\pgfqpoint{4.190691in}{2.820000in}}%
\pgfusepath{stroke}%
\end{pgfscope}%
\begin{pgfscope}%
\definecolor{textcolor}{rgb}{0.150000,0.150000,0.150000}%
\pgfsetstrokecolor{textcolor}%
\pgfsetfillcolor{textcolor}%
\pgftext[x=4.190691in,y=0.531691in,,top]{\color{textcolor}{\sffamily\fontsize{11.000000}{13.200000}\selectfont\catcode`\^=\active\def^{\ifmmode\sp\else\^{}\fi}\catcode`\%=\active\def%{\%}200}}%
\end{pgfscope}%
\begin{pgfscope}%
\pgfpathrectangle{\pgfqpoint{3.485805in}{0.663635in}}{\pgfqpoint{1.934195in}{2.156365in}}%
\pgfusepath{clip}%
\pgfsetroundcap%
\pgfsetroundjoin%
\pgfsetlinewidth{1.003750pt}%
\definecolor{currentstroke}{rgb}{1.000000,1.000000,1.000000}%
\pgfsetstrokecolor{currentstroke}%
\pgfsetdash{}{0pt}%
\pgfpathmoveto{\pgfqpoint{4.807659in}{0.663635in}}%
\pgfpathlineto{\pgfqpoint{4.807659in}{2.820000in}}%
\pgfusepath{stroke}%
\end{pgfscope}%
\begin{pgfscope}%
\definecolor{textcolor}{rgb}{0.150000,0.150000,0.150000}%
\pgfsetstrokecolor{textcolor}%
\pgfsetfillcolor{textcolor}%
\pgftext[x=4.807659in,y=0.531691in,,top]{\color{textcolor}{\sffamily\fontsize{11.000000}{13.200000}\selectfont\catcode`\^=\active\def^{\ifmmode\sp\else\^{}\fi}\catcode`\%=\active\def%{\%}400}}%
\end{pgfscope}%
\begin{pgfscope}%
\definecolor{textcolor}{rgb}{0.150000,0.150000,0.150000}%
\pgfsetstrokecolor{textcolor}%
\pgfsetfillcolor{textcolor}%
\pgftext[x=4.452902in,y=0.336413in,,top]{\color{textcolor}{\sffamily\fontsize{12.000000}{14.400000}\selectfont\catcode`\^=\active\def^{\ifmmode\sp\else\^{}\fi}\catcode`\%=\active\def%{\%}Time (s)}}%
\end{pgfscope}%
\begin{pgfscope}%
\pgfpathrectangle{\pgfqpoint{3.485805in}{0.663635in}}{\pgfqpoint{1.934195in}{2.156365in}}%
\pgfusepath{clip}%
\pgfsetroundcap%
\pgfsetroundjoin%
\pgfsetlinewidth{1.003750pt}%
\definecolor{currentstroke}{rgb}{1.000000,1.000000,1.000000}%
\pgfsetstrokecolor{currentstroke}%
\pgfsetdash{}{0pt}%
\pgfpathmoveto{\pgfqpoint{3.485805in}{1.202727in}}%
\pgfpathlineto{\pgfqpoint{5.420000in}{1.202727in}}%
\pgfusepath{stroke}%
\end{pgfscope}%
\begin{pgfscope}%
\definecolor{textcolor}{rgb}{0.150000,0.150000,0.150000}%
\pgfsetstrokecolor{textcolor}%
\pgfsetfillcolor{textcolor}%
\pgftext[x=3.268892in, y=1.148046in, left, base]{\color{textcolor}{\sffamily\fontsize{11.000000}{13.200000}\selectfont\catcode`\^=\active\def^{\ifmmode\sp\else\^{}\fi}\catcode`\%=\active\def%{\%}1}}%
\end{pgfscope}%
\begin{pgfscope}%
\pgfpathrectangle{\pgfqpoint{3.485805in}{0.663635in}}{\pgfqpoint{1.934195in}{2.156365in}}%
\pgfusepath{clip}%
\pgfsetroundcap%
\pgfsetroundjoin%
\pgfsetlinewidth{1.003750pt}%
\definecolor{currentstroke}{rgb}{1.000000,1.000000,1.000000}%
\pgfsetstrokecolor{currentstroke}%
\pgfsetdash{}{0pt}%
\pgfpathmoveto{\pgfqpoint{3.485805in}{2.280909in}}%
\pgfpathlineto{\pgfqpoint{5.420000in}{2.280909in}}%
\pgfusepath{stroke}%
\end{pgfscope}%
\begin{pgfscope}%
\definecolor{textcolor}{rgb}{0.150000,0.150000,0.150000}%
\pgfsetstrokecolor{textcolor}%
\pgfsetfillcolor{textcolor}%
\pgftext[x=3.268892in, y=2.226228in, left, base]{\color{textcolor}{\sffamily\fontsize{11.000000}{13.200000}\selectfont\catcode`\^=\active\def^{\ifmmode\sp\else\^{}\fi}\catcode`\%=\active\def%{\%}2}}%
\end{pgfscope}%
\begin{pgfscope}%
\definecolor{textcolor}{rgb}{0.150000,0.150000,0.150000}%
\pgfsetstrokecolor{textcolor}%
\pgfsetfillcolor{textcolor}%
\pgftext[x=3.213337in,y=1.741818in,,bottom,rotate=90.000000]{\color{textcolor}{\sffamily\fontsize{12.000000}{14.400000}\selectfont\catcode`\^=\active\def^{\ifmmode\sp\else\^{}\fi}\catcode`\%=\active\def%{\%}\# nodes}}%
\end{pgfscope}%
\begin{pgfscope}%
\pgfpathrectangle{\pgfqpoint{3.485805in}{0.663635in}}{\pgfqpoint{1.934195in}{2.156365in}}%
\pgfusepath{clip}%
\pgfsetroundcap%
\pgfsetroundjoin%
\pgfsetlinewidth{1.505625pt}%
\definecolor{currentstroke}{rgb}{0.298039,0.447059,0.690196}%
\pgfsetstrokecolor{currentstroke}%
\pgfsetdash{}{0pt}%
\pgfpathmoveto{\pgfqpoint{3.573723in}{1.202727in}}%
\pgfpathlineto{\pgfqpoint{3.589147in}{1.202727in}}%
\pgfpathlineto{\pgfqpoint{3.604571in}{1.202727in}}%
\pgfpathlineto{\pgfqpoint{3.619995in}{1.202727in}}%
\pgfpathlineto{\pgfqpoint{3.635419in}{1.202727in}}%
\pgfpathlineto{\pgfqpoint{3.650844in}{1.202727in}}%
\pgfpathlineto{\pgfqpoint{3.666268in}{1.202727in}}%
\pgfpathlineto{\pgfqpoint{3.681692in}{1.202727in}}%
\pgfpathlineto{\pgfqpoint{3.697116in}{1.202727in}}%
\pgfpathlineto{\pgfqpoint{3.712540in}{1.202727in}}%
\pgfpathlineto{\pgfqpoint{3.727965in}{1.202727in}}%
\pgfpathlineto{\pgfqpoint{3.743389in}{1.202727in}}%
\pgfpathlineto{\pgfqpoint{3.758813in}{1.202727in}}%
\pgfpathlineto{\pgfqpoint{3.774237in}{1.202727in}}%
\pgfpathlineto{\pgfqpoint{3.789661in}{1.202727in}}%
\pgfpathlineto{\pgfqpoint{3.805086in}{1.202727in}}%
\pgfpathlineto{\pgfqpoint{3.820510in}{1.202727in}}%
\pgfpathlineto{\pgfqpoint{3.835934in}{1.202727in}}%
\pgfpathlineto{\pgfqpoint{3.851358in}{1.202727in}}%
\pgfpathlineto{\pgfqpoint{3.866783in}{1.202727in}}%
\pgfpathlineto{\pgfqpoint{3.882207in}{1.202727in}}%
\pgfpathlineto{\pgfqpoint{3.897631in}{1.202727in}}%
\pgfpathlineto{\pgfqpoint{3.913055in}{1.202727in}}%
\pgfpathlineto{\pgfqpoint{3.928479in}{1.202727in}}%
\pgfpathlineto{\pgfqpoint{3.943904in}{1.202727in}}%
\pgfpathlineto{\pgfqpoint{3.959328in}{1.202727in}}%
\pgfpathlineto{\pgfqpoint{3.974752in}{1.202727in}}%
\pgfpathlineto{\pgfqpoint{3.990176in}{1.202727in}}%
\pgfpathlineto{\pgfqpoint{4.005600in}{1.202727in}}%
\pgfpathlineto{\pgfqpoint{4.021025in}{1.202727in}}%
\pgfpathlineto{\pgfqpoint{4.036449in}{1.202727in}}%
\pgfpathlineto{\pgfqpoint{4.051873in}{1.202727in}}%
\pgfpathlineto{\pgfqpoint{4.067297in}{1.202727in}}%
\pgfpathlineto{\pgfqpoint{4.082721in}{1.202727in}}%
\pgfpathlineto{\pgfqpoint{4.098146in}{1.202727in}}%
\pgfpathlineto{\pgfqpoint{4.113570in}{1.202727in}}%
\pgfpathlineto{\pgfqpoint{4.128994in}{1.202727in}}%
\pgfpathlineto{\pgfqpoint{4.144418in}{1.202727in}}%
\pgfpathlineto{\pgfqpoint{4.159842in}{1.202727in}}%
\pgfpathlineto{\pgfqpoint{4.175267in}{1.202727in}}%
\pgfpathlineto{\pgfqpoint{4.190691in}{1.202727in}}%
\pgfpathlineto{\pgfqpoint{4.206115in}{1.202727in}}%
\pgfpathlineto{\pgfqpoint{4.221539in}{1.202727in}}%
\pgfpathlineto{\pgfqpoint{4.236963in}{1.202727in}}%
\pgfpathlineto{\pgfqpoint{4.252388in}{1.202727in}}%
\pgfpathlineto{\pgfqpoint{4.267812in}{1.202727in}}%
\pgfpathlineto{\pgfqpoint{4.283236in}{1.202727in}}%
\pgfpathlineto{\pgfqpoint{4.298660in}{1.202727in}}%
\pgfpathlineto{\pgfqpoint{4.314084in}{1.202727in}}%
\pgfpathlineto{\pgfqpoint{4.329509in}{1.202727in}}%
\pgfpathlineto{\pgfqpoint{4.344933in}{1.202727in}}%
\pgfpathlineto{\pgfqpoint{4.360357in}{1.202727in}}%
\pgfpathlineto{\pgfqpoint{4.375781in}{1.202727in}}%
\pgfpathlineto{\pgfqpoint{4.391206in}{1.202727in}}%
\pgfpathlineto{\pgfqpoint{4.406630in}{1.202727in}}%
\pgfpathlineto{\pgfqpoint{4.422054in}{2.280909in}}%
\pgfpathlineto{\pgfqpoint{4.437478in}{2.280909in}}%
\pgfpathlineto{\pgfqpoint{4.452902in}{2.280909in}}%
\pgfpathlineto{\pgfqpoint{4.468327in}{2.280909in}}%
\pgfpathlineto{\pgfqpoint{4.483751in}{2.280909in}}%
\pgfpathlineto{\pgfqpoint{4.499175in}{2.280909in}}%
\pgfpathlineto{\pgfqpoint{4.514599in}{2.280909in}}%
\pgfpathlineto{\pgfqpoint{4.530023in}{2.280909in}}%
\pgfpathlineto{\pgfqpoint{4.545448in}{2.280909in}}%
\pgfpathlineto{\pgfqpoint{4.560872in}{2.280909in}}%
\pgfpathlineto{\pgfqpoint{4.576296in}{2.280909in}}%
\pgfpathlineto{\pgfqpoint{4.591720in}{2.280909in}}%
\pgfpathlineto{\pgfqpoint{4.607144in}{2.280909in}}%
\pgfpathlineto{\pgfqpoint{4.622569in}{2.280909in}}%
\pgfpathlineto{\pgfqpoint{4.637993in}{2.280909in}}%
\pgfpathlineto{\pgfqpoint{4.653417in}{2.280909in}}%
\pgfpathlineto{\pgfqpoint{4.668841in}{2.280909in}}%
\pgfpathlineto{\pgfqpoint{4.684265in}{2.280909in}}%
\pgfpathlineto{\pgfqpoint{4.699690in}{2.280909in}}%
\pgfpathlineto{\pgfqpoint{4.715114in}{2.280909in}}%
\pgfpathlineto{\pgfqpoint{4.730538in}{2.280909in}}%
\pgfpathlineto{\pgfqpoint{4.745962in}{2.280909in}}%
\pgfpathlineto{\pgfqpoint{4.761386in}{2.280909in}}%
\pgfpathlineto{\pgfqpoint{4.776811in}{2.280909in}}%
\pgfpathlineto{\pgfqpoint{4.792235in}{2.280909in}}%
\pgfpathlineto{\pgfqpoint{4.807659in}{2.280909in}}%
\pgfpathlineto{\pgfqpoint{4.823083in}{2.280909in}}%
\pgfpathlineto{\pgfqpoint{4.838507in}{2.280909in}}%
\pgfpathlineto{\pgfqpoint{4.853932in}{2.280909in}}%
\pgfpathlineto{\pgfqpoint{4.869356in}{2.280909in}}%
\pgfpathlineto{\pgfqpoint{4.884780in}{2.280909in}}%
\pgfpathlineto{\pgfqpoint{4.900204in}{2.280909in}}%
\pgfpathlineto{\pgfqpoint{4.915628in}{2.280909in}}%
\pgfpathlineto{\pgfqpoint{4.931053in}{2.280909in}}%
\pgfpathlineto{\pgfqpoint{4.946477in}{2.280909in}}%
\pgfpathlineto{\pgfqpoint{4.961901in}{2.280909in}}%
\pgfpathlineto{\pgfqpoint{4.977325in}{2.280909in}}%
\pgfpathlineto{\pgfqpoint{4.992750in}{2.280909in}}%
\pgfpathlineto{\pgfqpoint{5.008174in}{2.280909in}}%
\pgfpathlineto{\pgfqpoint{5.023598in}{2.280909in}}%
\pgfpathlineto{\pgfqpoint{5.039022in}{2.280909in}}%
\pgfpathlineto{\pgfqpoint{5.054446in}{2.280909in}}%
\pgfpathlineto{\pgfqpoint{5.069871in}{2.280909in}}%
\pgfpathlineto{\pgfqpoint{5.085295in}{2.280909in}}%
\pgfpathlineto{\pgfqpoint{5.100719in}{2.280909in}}%
\pgfpathlineto{\pgfqpoint{5.116143in}{2.280909in}}%
\pgfpathlineto{\pgfqpoint{5.131567in}{2.280909in}}%
\pgfpathlineto{\pgfqpoint{5.146992in}{2.280909in}}%
\pgfpathlineto{\pgfqpoint{5.162416in}{2.280909in}}%
\pgfpathlineto{\pgfqpoint{5.177840in}{2.280909in}}%
\pgfpathlineto{\pgfqpoint{5.193264in}{2.280909in}}%
\pgfpathlineto{\pgfqpoint{5.208688in}{2.280909in}}%
\pgfpathlineto{\pgfqpoint{5.224113in}{2.280909in}}%
\pgfpathlineto{\pgfqpoint{5.239537in}{2.280909in}}%
\pgfpathlineto{\pgfqpoint{5.254961in}{2.280909in}}%
\pgfpathlineto{\pgfqpoint{5.270385in}{2.280909in}}%
\pgfpathlineto{\pgfqpoint{5.285809in}{2.280909in}}%
\pgfpathlineto{\pgfqpoint{5.301234in}{2.280909in}}%
\pgfpathlineto{\pgfqpoint{5.316658in}{2.280909in}}%
\pgfpathlineto{\pgfqpoint{5.332082in}{2.280909in}}%
\pgfusepath{stroke}%
\end{pgfscope}%
\begin{pgfscope}%
\pgfsetrectcap%
\pgfsetmiterjoin%
\pgfsetlinewidth{1.254687pt}%
\definecolor{currentstroke}{rgb}{1.000000,1.000000,1.000000}%
\pgfsetstrokecolor{currentstroke}%
\pgfsetdash{}{0pt}%
\pgfpathmoveto{\pgfqpoint{3.485805in}{0.663635in}}%
\pgfpathlineto{\pgfqpoint{3.485805in}{2.820000in}}%
\pgfusepath{stroke}%
\end{pgfscope}%
\begin{pgfscope}%
\pgfsetrectcap%
\pgfsetmiterjoin%
\pgfsetlinewidth{1.254687pt}%
\definecolor{currentstroke}{rgb}{1.000000,1.000000,1.000000}%
\pgfsetstrokecolor{currentstroke}%
\pgfsetdash{}{0pt}%
\pgfpathmoveto{\pgfqpoint{5.420000in}{0.663635in}}%
\pgfpathlineto{\pgfqpoint{5.420000in}{2.820000in}}%
\pgfusepath{stroke}%
\end{pgfscope}%
\begin{pgfscope}%
\pgfsetrectcap%
\pgfsetmiterjoin%
\pgfsetlinewidth{1.254687pt}%
\definecolor{currentstroke}{rgb}{1.000000,1.000000,1.000000}%
\pgfsetstrokecolor{currentstroke}%
\pgfsetdash{}{0pt}%
\pgfpathmoveto{\pgfqpoint{3.485805in}{0.663635in}}%
\pgfpathlineto{\pgfqpoint{5.420000in}{0.663635in}}%
\pgfusepath{stroke}%
\end{pgfscope}%
\begin{pgfscope}%
\pgfsetrectcap%
\pgfsetmiterjoin%
\pgfsetlinewidth{1.254687pt}%
\definecolor{currentstroke}{rgb}{1.000000,1.000000,1.000000}%
\pgfsetstrokecolor{currentstroke}%
\pgfsetdash{}{0pt}%
\pgfpathmoveto{\pgfqpoint{3.485805in}{2.820000in}}%
\pgfpathlineto{\pgfqpoint{5.420000in}{2.820000in}}%
\pgfusepath{stroke}%
\end{pgfscope}%
\begin{pgfscope}%
\pgfsetbuttcap%
\pgfsetmiterjoin%
\definecolor{currentfill}{rgb}{0.917647,0.917647,0.949020}%
\pgfsetfillcolor{currentfill}%
\pgfsetfillopacity{0.800000}%
\pgfsetlinewidth{1.003750pt}%
\definecolor{currentstroke}{rgb}{0.800000,0.800000,0.800000}%
\pgfsetstrokecolor{currentstroke}%
\pgfsetstrokeopacity{0.800000}%
\pgfsetdash{}{0pt}%
\pgfpathmoveto{\pgfqpoint{2.100358in}{2.751281in}}%
\pgfpathlineto{\pgfqpoint{3.499642in}{2.751281in}}%
\pgfpathquadraticcurveto{\pgfqpoint{3.521864in}{2.751281in}}{\pgfqpoint{3.521864in}{2.773503in}}%
\pgfpathlineto{\pgfqpoint{3.521864in}{2.922222in}}%
\pgfpathquadraticcurveto{\pgfqpoint{3.521864in}{2.944444in}}{\pgfqpoint{3.499642in}{2.944444in}}%
\pgfpathlineto{\pgfqpoint{2.100358in}{2.944444in}}%
\pgfpathquadraticcurveto{\pgfqpoint{2.078136in}{2.944444in}}{\pgfqpoint{2.078136in}{2.922222in}}%
\pgfpathlineto{\pgfqpoint{2.078136in}{2.773503in}}%
\pgfpathquadraticcurveto{\pgfqpoint{2.078136in}{2.751281in}}{\pgfqpoint{2.100358in}{2.751281in}}%
\pgfpathlineto{\pgfqpoint{2.100358in}{2.751281in}}%
\pgfpathclose%
\pgfusepath{stroke,fill}%
\end{pgfscope}%
\begin{pgfscope}%
\pgfsetbuttcap%
\pgfsetroundjoin%
\pgfsetlinewidth{1.505625pt}%
\definecolor{currentstroke}{rgb}{1.000000,0.647059,0.000000}%
\pgfsetstrokecolor{currentstroke}%
\pgfsetdash{{5.550000pt}{2.400000pt}}{0.000000pt}%
\pgfpathmoveto{\pgfqpoint{2.122580in}{2.857997in}}%
\pgfpathlineto{\pgfqpoint{2.233691in}{2.857997in}}%
\pgfpathlineto{\pgfqpoint{2.344803in}{2.857997in}}%
\pgfusepath{stroke}%
\end{pgfscope}%
\begin{pgfscope}%
\definecolor{textcolor}{rgb}{0.150000,0.150000,0.150000}%
\pgfsetstrokecolor{textcolor}%
\pgfsetfillcolor{textcolor}%
\pgftext[x=2.433691in,y=2.819108in,left,base]{\color{textcolor}{\sffamily\fontsize{8.000000}{9.600000}\selectfont\catcode`\^=\active\def^{\ifmmode\sp\else\^{}\fi}\catcode`\%=\active\def%{\%}start of scaling action}}%
\end{pgfscope}%
\end{pgfpicture}%
\makeatother%
\endgroup%

    \caption{Average write load per node and amount of nodes during a horizontal scaling action}
    \label{fig:horizontal-elasticity}
\end{figure}
