\chapter{Evaluation}
\label{ch:evaluation}

This chapter first introduces the setup that was used for evaluating the different elasticity strategies. Then the results of different tests are presented and discussed.

\section{Test Setup}
\label{sec:testsetup}

In order to test the different elasticity strategies a test environment has to be set up. It was decided to create three virtual machines (VM) that will form a Kubernetes cluster. Because of its ease of use microk8s was chosen as distribution\footnote{\url{https://microk8s.io/}}. All three virtual machines were assigned 10 vCPUs and 10GB of memory. One VM acts as the Kubernetes control plane while the other two join the cluster as worker nodes.

Everything that was deployed into the Kubernetes cluster was built using the infrastructure as code (IaC) tool HashiCorp Terraform\footnote{\url{https://www.terraform.io/}}. This enables rapid changes and reproducibility. Deployed resources include the kube-prometheus-stack\footnote{\raggedright\url{https://artifacthub.io/packages/helm/prometheus-community/kube-prometheus-stack}} for monitoring, the k8ssandra-operator\footnote{\url{https://docs.k8ssandra.io/components/k8ssandra-operator/}} for managing k8ssandra clusters and a definition for a k8ssandra cluster. Additionally, the in \cref{sec:metrics} mentioned Grafana dashboards are also deployed using Terraform.

\Cref{lst:k8c} illustrates a minimal definition of a 3 node k8ssandra cluster. Each node has resource limits of 800 millicpu and 6000MB of memory and 3GiB storage space.

\begin{lstlisting}[caption={Minimal example of a K8ssandraCluster definition.},
                label=lst:k8c,
                captionpos=b,
                float]
apiVersion: k8ssandra.io/v1alpha1
kind: K8ssandraCluster
metadata:
  name: polaris-test-cluster
  namespace: k8ssandra
spec:
  cassandra:
    resources:
      limits:
        cpu: 800m
        memory: 6000M
    datacenters:
      - metadata:
          name: dc1
        size: 3
        storageConfig:
          cassandraDataVolumeClaimSpec:
            resources:
              requests:
                storage: 3Gi
\end{lstlisting}

\section{Benchmarks}

In the following sections, different test scenarios will be discussed. To let k8ssandra experience load, the built-in stress testing tool \texttt{cassandra-stress} was used\footnote{\raggedright\url{https://cassandra.apache.org/doc/stable/cassandra/tools/cassandra_stress.html}}.

\subsection{Stress Testing}
\label{sec:stress-testing}

To set a baseline, three different k8ssandra cluster setups have been stress tested using \texttt{cassandra-stress}. The amount of write requests that the tool will make is set to be 1000000, the exact call is listed in \cref{lst:stress-1000000writes}. These cluster setups merely differ in the cluster size, thus the amount of nodes. All clusters were provisioned with limits of 2 CPUs and 6GB of memory.

\begin{lstlisting}[caption={},
                    captionpos=b,
                    label=lst:stress-1000000writes,
                    float]
./cassandra-stress write n=1000000 -mode native cql3 \
    user='USERNAME' password='PASSWORD'
\end{lstlisting}

The results of these tests are depicted in \cref{fig:stress-1000000writes-1node,fig:stress-1000000writes-2node,,fig:stress-1000000writes-3node}. The write throughput increases with the amount of nodes, but not linearly. This, however, was to be expected as \texttt{cassandra-stress} does not partition data in a way that favours linear scalability. The average write throughputs of these different clusters can be seen in \cref{tab:stress-1000000writes-ops}.

\begin{table}[H]
\centering
\begin{tabular}{|l|l|l|}
\hline
\textbf{Cluster size} & \textbf{operations/s} & \textbf{Time to complete} \\ \hline
1                     & 12514                 & 2m38s                     \\ \hline
2                     & 13142                 & 1m57s                     \\ \hline
3                     & 14318                 & 1m50s                     \\ \hline
\end{tabular}
\caption{Average write throughput for different k8ssandra clusters. With increasing cluster size the throughput also increases}
\label{tab:stress-1000000writes-ops}
\end{table}

\begin{figure}
    \centering
    %% Creator: Matplotlib, PGF backend
%%
%% To include the figure in your LaTeX document, write
%%   \input{<filename>.pgf}
%%
%% Make sure the required packages are loaded in your preamble
%%   \usepackage{pgf}
%%
%% Also ensure that all the required font packages are loaded; for instance,
%% the lmodern package is sometimes necessary when using math font.
%%   \usepackage{lmodern}
%%
%% Figures using additional raster images can only be included by \input if
%% they are in the same directory as the main LaTeX file. For loading figures
%% from other directories you can use the `import` package
%%   \usepackage{import}
%%
%% and then include the figures with
%%   \import{<path to file>}{<filename>.pgf}
%%
%% Matplotlib used the following preamble
%%   \def\mathdefault#1{#1}
%%   \everymath=\expandafter{\the\everymath\displaystyle}
%%   
%%   \usepackage{fontspec}
%%   \setmainfont{DejaVuSerif.ttf}[Path=\detokenize{/Users/nkratky/private/polaris-elasticity-strategies/test/scripts/.venv/lib/python3.11/site-packages/matplotlib/mpl-data/fonts/ttf/}]
%%   \setsansfont{Arial.ttf}[Path=\detokenize{/System/Library/Fonts/Supplemental/}]
%%   \setmonofont{DejaVuSansMono.ttf}[Path=\detokenize{/Users/nkratky/private/polaris-elasticity-strategies/test/scripts/.venv/lib/python3.11/site-packages/matplotlib/mpl-data/fonts/ttf/}]
%%   \makeatletter\@ifpackageloaded{underscore}{}{\usepackage[strings]{underscore}}\makeatother
%%
\begingroup%
\makeatletter%
\begin{pgfpicture}%
\pgfpathrectangle{\pgfpointorigin}{\pgfqpoint{5.600000in}{2.500000in}}%
\pgfusepath{use as bounding box, clip}%
\begin{pgfscope}%
\pgfsetbuttcap%
\pgfsetmiterjoin%
\definecolor{currentfill}{rgb}{1.000000,1.000000,1.000000}%
\pgfsetfillcolor{currentfill}%
\pgfsetlinewidth{0.000000pt}%
\definecolor{currentstroke}{rgb}{1.000000,1.000000,1.000000}%
\pgfsetstrokecolor{currentstroke}%
\pgfsetdash{}{0pt}%
\pgfpathmoveto{\pgfqpoint{0.000000in}{0.000000in}}%
\pgfpathlineto{\pgfqpoint{5.600000in}{0.000000in}}%
\pgfpathlineto{\pgfqpoint{5.600000in}{2.500000in}}%
\pgfpathlineto{\pgfqpoint{0.000000in}{2.500000in}}%
\pgfpathlineto{\pgfqpoint{0.000000in}{0.000000in}}%
\pgfpathclose%
\pgfusepath{fill}%
\end{pgfscope}%
\begin{pgfscope}%
\pgfsetbuttcap%
\pgfsetmiterjoin%
\definecolor{currentfill}{rgb}{0.917647,0.917647,0.949020}%
\pgfsetfillcolor{currentfill}%
\pgfsetlinewidth{0.000000pt}%
\definecolor{currentstroke}{rgb}{0.000000,0.000000,0.000000}%
\pgfsetstrokecolor{currentstroke}%
\pgfsetstrokeopacity{0.000000}%
\pgfsetdash{}{0pt}%
\pgfpathmoveto{\pgfqpoint{0.948751in}{0.663635in}}%
\pgfpathlineto{\pgfqpoint{5.420000in}{0.663635in}}%
\pgfpathlineto{\pgfqpoint{5.420000in}{2.320000in}}%
\pgfpathlineto{\pgfqpoint{0.948751in}{2.320000in}}%
\pgfpathlineto{\pgfqpoint{0.948751in}{0.663635in}}%
\pgfpathclose%
\pgfusepath{fill}%
\end{pgfscope}%
\begin{pgfscope}%
\pgfpathrectangle{\pgfqpoint{0.948751in}{0.663635in}}{\pgfqpoint{4.471249in}{1.656365in}}%
\pgfusepath{clip}%
\pgfsetroundcap%
\pgfsetroundjoin%
\pgfsetlinewidth{1.003750pt}%
\definecolor{currentstroke}{rgb}{1.000000,1.000000,1.000000}%
\pgfsetstrokecolor{currentstroke}%
\pgfsetdash{}{0pt}%
\pgfpathmoveto{\pgfqpoint{1.151990in}{0.663635in}}%
\pgfpathlineto{\pgfqpoint{1.151990in}{2.320000in}}%
\pgfusepath{stroke}%
\end{pgfscope}%
\begin{pgfscope}%
\definecolor{textcolor}{rgb}{0.150000,0.150000,0.150000}%
\pgfsetstrokecolor{textcolor}%
\pgfsetfillcolor{textcolor}%
\pgftext[x=1.151990in,y=0.531691in,,top]{\color{textcolor}{\sffamily\fontsize{11.000000}{13.200000}\selectfont\catcode`\^=\active\def^{\ifmmode\sp\else\^{}\fi}\catcode`\%=\active\def%{\%}0}}%
\end{pgfscope}%
\begin{pgfscope}%
\pgfpathrectangle{\pgfqpoint{0.948751in}{0.663635in}}{\pgfqpoint{4.471249in}{1.656365in}}%
\pgfusepath{clip}%
\pgfsetroundcap%
\pgfsetroundjoin%
\pgfsetlinewidth{1.003750pt}%
\definecolor{currentstroke}{rgb}{1.000000,1.000000,1.000000}%
\pgfsetstrokecolor{currentstroke}%
\pgfsetdash{}{0pt}%
\pgfpathmoveto{\pgfqpoint{1.829452in}{0.663635in}}%
\pgfpathlineto{\pgfqpoint{1.829452in}{2.320000in}}%
\pgfusepath{stroke}%
\end{pgfscope}%
\begin{pgfscope}%
\definecolor{textcolor}{rgb}{0.150000,0.150000,0.150000}%
\pgfsetstrokecolor{textcolor}%
\pgfsetfillcolor{textcolor}%
\pgftext[x=1.829452in,y=0.531691in,,top]{\color{textcolor}{\sffamily\fontsize{11.000000}{13.200000}\selectfont\catcode`\^=\active\def^{\ifmmode\sp\else\^{}\fi}\catcode`\%=\active\def%{\%}20}}%
\end{pgfscope}%
\begin{pgfscope}%
\pgfpathrectangle{\pgfqpoint{0.948751in}{0.663635in}}{\pgfqpoint{4.471249in}{1.656365in}}%
\pgfusepath{clip}%
\pgfsetroundcap%
\pgfsetroundjoin%
\pgfsetlinewidth{1.003750pt}%
\definecolor{currentstroke}{rgb}{1.000000,1.000000,1.000000}%
\pgfsetstrokecolor{currentstroke}%
\pgfsetdash{}{0pt}%
\pgfpathmoveto{\pgfqpoint{2.506914in}{0.663635in}}%
\pgfpathlineto{\pgfqpoint{2.506914in}{2.320000in}}%
\pgfusepath{stroke}%
\end{pgfscope}%
\begin{pgfscope}%
\definecolor{textcolor}{rgb}{0.150000,0.150000,0.150000}%
\pgfsetstrokecolor{textcolor}%
\pgfsetfillcolor{textcolor}%
\pgftext[x=2.506914in,y=0.531691in,,top]{\color{textcolor}{\sffamily\fontsize{11.000000}{13.200000}\selectfont\catcode`\^=\active\def^{\ifmmode\sp\else\^{}\fi}\catcode`\%=\active\def%{\%}40}}%
\end{pgfscope}%
\begin{pgfscope}%
\pgfpathrectangle{\pgfqpoint{0.948751in}{0.663635in}}{\pgfqpoint{4.471249in}{1.656365in}}%
\pgfusepath{clip}%
\pgfsetroundcap%
\pgfsetroundjoin%
\pgfsetlinewidth{1.003750pt}%
\definecolor{currentstroke}{rgb}{1.000000,1.000000,1.000000}%
\pgfsetstrokecolor{currentstroke}%
\pgfsetdash{}{0pt}%
\pgfpathmoveto{\pgfqpoint{3.184376in}{0.663635in}}%
\pgfpathlineto{\pgfqpoint{3.184376in}{2.320000in}}%
\pgfusepath{stroke}%
\end{pgfscope}%
\begin{pgfscope}%
\definecolor{textcolor}{rgb}{0.150000,0.150000,0.150000}%
\pgfsetstrokecolor{textcolor}%
\pgfsetfillcolor{textcolor}%
\pgftext[x=3.184376in,y=0.531691in,,top]{\color{textcolor}{\sffamily\fontsize{11.000000}{13.200000}\selectfont\catcode`\^=\active\def^{\ifmmode\sp\else\^{}\fi}\catcode`\%=\active\def%{\%}60}}%
\end{pgfscope}%
\begin{pgfscope}%
\pgfpathrectangle{\pgfqpoint{0.948751in}{0.663635in}}{\pgfqpoint{4.471249in}{1.656365in}}%
\pgfusepath{clip}%
\pgfsetroundcap%
\pgfsetroundjoin%
\pgfsetlinewidth{1.003750pt}%
\definecolor{currentstroke}{rgb}{1.000000,1.000000,1.000000}%
\pgfsetstrokecolor{currentstroke}%
\pgfsetdash{}{0pt}%
\pgfpathmoveto{\pgfqpoint{3.861838in}{0.663635in}}%
\pgfpathlineto{\pgfqpoint{3.861838in}{2.320000in}}%
\pgfusepath{stroke}%
\end{pgfscope}%
\begin{pgfscope}%
\definecolor{textcolor}{rgb}{0.150000,0.150000,0.150000}%
\pgfsetstrokecolor{textcolor}%
\pgfsetfillcolor{textcolor}%
\pgftext[x=3.861838in,y=0.531691in,,top]{\color{textcolor}{\sffamily\fontsize{11.000000}{13.200000}\selectfont\catcode`\^=\active\def^{\ifmmode\sp\else\^{}\fi}\catcode`\%=\active\def%{\%}80}}%
\end{pgfscope}%
\begin{pgfscope}%
\pgfpathrectangle{\pgfqpoint{0.948751in}{0.663635in}}{\pgfqpoint{4.471249in}{1.656365in}}%
\pgfusepath{clip}%
\pgfsetroundcap%
\pgfsetroundjoin%
\pgfsetlinewidth{1.003750pt}%
\definecolor{currentstroke}{rgb}{1.000000,1.000000,1.000000}%
\pgfsetstrokecolor{currentstroke}%
\pgfsetdash{}{0pt}%
\pgfpathmoveto{\pgfqpoint{4.539299in}{0.663635in}}%
\pgfpathlineto{\pgfqpoint{4.539299in}{2.320000in}}%
\pgfusepath{stroke}%
\end{pgfscope}%
\begin{pgfscope}%
\definecolor{textcolor}{rgb}{0.150000,0.150000,0.150000}%
\pgfsetstrokecolor{textcolor}%
\pgfsetfillcolor{textcolor}%
\pgftext[x=4.539299in,y=0.531691in,,top]{\color{textcolor}{\sffamily\fontsize{11.000000}{13.200000}\selectfont\catcode`\^=\active\def^{\ifmmode\sp\else\^{}\fi}\catcode`\%=\active\def%{\%}100}}%
\end{pgfscope}%
\begin{pgfscope}%
\pgfpathrectangle{\pgfqpoint{0.948751in}{0.663635in}}{\pgfqpoint{4.471249in}{1.656365in}}%
\pgfusepath{clip}%
\pgfsetroundcap%
\pgfsetroundjoin%
\pgfsetlinewidth{1.003750pt}%
\definecolor{currentstroke}{rgb}{1.000000,1.000000,1.000000}%
\pgfsetstrokecolor{currentstroke}%
\pgfsetdash{}{0pt}%
\pgfpathmoveto{\pgfqpoint{5.216761in}{0.663635in}}%
\pgfpathlineto{\pgfqpoint{5.216761in}{2.320000in}}%
\pgfusepath{stroke}%
\end{pgfscope}%
\begin{pgfscope}%
\definecolor{textcolor}{rgb}{0.150000,0.150000,0.150000}%
\pgfsetstrokecolor{textcolor}%
\pgfsetfillcolor{textcolor}%
\pgftext[x=5.216761in,y=0.531691in,,top]{\color{textcolor}{\sffamily\fontsize{11.000000}{13.200000}\selectfont\catcode`\^=\active\def^{\ifmmode\sp\else\^{}\fi}\catcode`\%=\active\def%{\%}120}}%
\end{pgfscope}%
\begin{pgfscope}%
\definecolor{textcolor}{rgb}{0.150000,0.150000,0.150000}%
\pgfsetstrokecolor{textcolor}%
\pgfsetfillcolor{textcolor}%
\pgftext[x=3.184376in,y=0.336413in,,top]{\color{textcolor}{\sffamily\fontsize{12.000000}{14.400000}\selectfont\catcode`\^=\active\def^{\ifmmode\sp\else\^{}\fi}\catcode`\%=\active\def%{\%}Time (s)}}%
\end{pgfscope}%
\begin{pgfscope}%
\pgfpathrectangle{\pgfqpoint{0.948751in}{0.663635in}}{\pgfqpoint{4.471249in}{1.656365in}}%
\pgfusepath{clip}%
\pgfsetroundcap%
\pgfsetroundjoin%
\pgfsetlinewidth{1.003750pt}%
\definecolor{currentstroke}{rgb}{1.000000,1.000000,1.000000}%
\pgfsetstrokecolor{currentstroke}%
\pgfsetdash{}{0pt}%
\pgfpathmoveto{\pgfqpoint{0.948751in}{0.738925in}}%
\pgfpathlineto{\pgfqpoint{5.420000in}{0.738925in}}%
\pgfusepath{stroke}%
\end{pgfscope}%
\begin{pgfscope}%
\definecolor{textcolor}{rgb}{0.150000,0.150000,0.150000}%
\pgfsetstrokecolor{textcolor}%
\pgfsetfillcolor{textcolor}%
\pgftext[x=0.731839in, y=0.684244in, left, base]{\color{textcolor}{\sffamily\fontsize{11.000000}{13.200000}\selectfont\catcode`\^=\active\def^{\ifmmode\sp\else\^{}\fi}\catcode`\%=\active\def%{\%}0}}%
\end{pgfscope}%
\begin{pgfscope}%
\pgfpathrectangle{\pgfqpoint{0.948751in}{0.663635in}}{\pgfqpoint{4.471249in}{1.656365in}}%
\pgfusepath{clip}%
\pgfsetroundcap%
\pgfsetroundjoin%
\pgfsetlinewidth{1.003750pt}%
\definecolor{currentstroke}{rgb}{1.000000,1.000000,1.000000}%
\pgfsetstrokecolor{currentstroke}%
\pgfsetdash{}{0pt}%
\pgfpathmoveto{\pgfqpoint{0.948751in}{1.339773in}}%
\pgfpathlineto{\pgfqpoint{5.420000in}{1.339773in}}%
\pgfusepath{stroke}%
\end{pgfscope}%
\begin{pgfscope}%
\definecolor{textcolor}{rgb}{0.150000,0.150000,0.150000}%
\pgfsetstrokecolor{textcolor}%
\pgfsetfillcolor{textcolor}%
\pgftext[x=0.391968in, y=1.285092in, left, base]{\color{textcolor}{\sffamily\fontsize{11.000000}{13.200000}\selectfont\catcode`\^=\active\def^{\ifmmode\sp\else\^{}\fi}\catcode`\%=\active\def%{\%}10000}}%
\end{pgfscope}%
\begin{pgfscope}%
\pgfpathrectangle{\pgfqpoint{0.948751in}{0.663635in}}{\pgfqpoint{4.471249in}{1.656365in}}%
\pgfusepath{clip}%
\pgfsetroundcap%
\pgfsetroundjoin%
\pgfsetlinewidth{1.003750pt}%
\definecolor{currentstroke}{rgb}{1.000000,1.000000,1.000000}%
\pgfsetstrokecolor{currentstroke}%
\pgfsetdash{}{0pt}%
\pgfpathmoveto{\pgfqpoint{0.948751in}{1.940621in}}%
\pgfpathlineto{\pgfqpoint{5.420000in}{1.940621in}}%
\pgfusepath{stroke}%
\end{pgfscope}%
\begin{pgfscope}%
\definecolor{textcolor}{rgb}{0.150000,0.150000,0.150000}%
\pgfsetstrokecolor{textcolor}%
\pgfsetfillcolor{textcolor}%
\pgftext[x=0.391968in, y=1.885941in, left, base]{\color{textcolor}{\sffamily\fontsize{11.000000}{13.200000}\selectfont\catcode`\^=\active\def^{\ifmmode\sp\else\^{}\fi}\catcode`\%=\active\def%{\%}20000}}%
\end{pgfscope}%
\begin{pgfscope}%
\definecolor{textcolor}{rgb}{0.150000,0.150000,0.150000}%
\pgfsetstrokecolor{textcolor}%
\pgfsetfillcolor{textcolor}%
\pgftext[x=0.336413in,y=1.491818in,,bottom,rotate=90.000000]{\color{textcolor}{\sffamily\fontsize{12.000000}{14.400000}\selectfont\catcode`\^=\active\def^{\ifmmode\sp\else\^{}\fi}\catcode`\%=\active\def%{\%}Writes (op/s)}}%
\end{pgfscope}%
\begin{pgfscope}%
\pgfpathrectangle{\pgfqpoint{0.948751in}{0.663635in}}{\pgfqpoint{4.471249in}{1.656365in}}%
\pgfusepath{clip}%
\pgfsetroundcap%
\pgfsetroundjoin%
\pgfsetlinewidth{1.505625pt}%
\definecolor{currentstroke}{rgb}{0.298039,0.447059,0.690196}%
\pgfsetstrokecolor{currentstroke}%
\pgfsetdash{}{0pt}%
\pgfpathmoveto{\pgfqpoint{1.151990in}{0.738925in}}%
\pgfpathlineto{\pgfqpoint{1.321355in}{0.738925in}}%
\pgfpathlineto{\pgfqpoint{1.490721in}{0.738925in}}%
\pgfpathlineto{\pgfqpoint{1.660086in}{1.222487in}}%
\pgfpathlineto{\pgfqpoint{1.829452in}{1.352691in}}%
\pgfpathlineto{\pgfqpoint{1.998817in}{1.482895in}}%
\pgfpathlineto{\pgfqpoint{2.168183in}{1.607451in}}%
\pgfpathlineto{\pgfqpoint{2.337548in}{1.694694in}}%
\pgfpathlineto{\pgfqpoint{2.506914in}{1.694694in}}%
\pgfpathlineto{\pgfqpoint{2.676279in}{1.765955in}}%
\pgfpathlineto{\pgfqpoint{2.845645in}{1.837275in}}%
\pgfpathlineto{\pgfqpoint{3.015010in}{1.837275in}}%
\pgfpathlineto{\pgfqpoint{3.184376in}{1.946450in}}%
\pgfpathlineto{\pgfqpoint{3.353741in}{2.055624in}}%
\pgfpathlineto{\pgfqpoint{3.523107in}{2.055624in}}%
\pgfpathlineto{\pgfqpoint{3.692472in}{2.150137in}}%
\pgfpathlineto{\pgfqpoint{3.861838in}{2.244711in}}%
\pgfpathlineto{\pgfqpoint{4.031203in}{2.244711in}}%
\pgfpathlineto{\pgfqpoint{4.200569in}{1.724496in}}%
\pgfpathlineto{\pgfqpoint{4.369934in}{1.204282in}}%
\pgfpathlineto{\pgfqpoint{4.539299in}{1.204282in}}%
\pgfpathlineto{\pgfqpoint{4.708665in}{0.971573in}}%
\pgfpathlineto{\pgfqpoint{4.878030in}{0.738925in}}%
\pgfpathlineto{\pgfqpoint{5.047396in}{0.738925in}}%
\pgfpathlineto{\pgfqpoint{5.216761in}{0.738925in}}%
\pgfusepath{stroke}%
\end{pgfscope}%
\begin{pgfscope}%
\pgfpathrectangle{\pgfqpoint{0.948751in}{0.663635in}}{\pgfqpoint{4.471249in}{1.656365in}}%
\pgfusepath{clip}%
\pgfsetroundcap%
\pgfsetroundjoin%
\pgfsetlinewidth{1.505625pt}%
\definecolor{currentstroke}{rgb}{1.000000,0.000000,0.000000}%
\pgfsetstrokecolor{currentstroke}%
\pgfsetdash{}{0pt}%
\pgfpathmoveto{\pgfqpoint{0.948751in}{1.469234in}}%
\pgfpathlineto{\pgfqpoint{5.420000in}{1.469234in}}%
\pgfusepath{stroke}%
\end{pgfscope}%
\begin{pgfscope}%
\pgfsetrectcap%
\pgfsetmiterjoin%
\pgfsetlinewidth{1.254687pt}%
\definecolor{currentstroke}{rgb}{1.000000,1.000000,1.000000}%
\pgfsetstrokecolor{currentstroke}%
\pgfsetdash{}{0pt}%
\pgfpathmoveto{\pgfqpoint{0.948751in}{0.663635in}}%
\pgfpathlineto{\pgfqpoint{0.948751in}{2.320000in}}%
\pgfusepath{stroke}%
\end{pgfscope}%
\begin{pgfscope}%
\pgfsetrectcap%
\pgfsetmiterjoin%
\pgfsetlinewidth{1.254687pt}%
\definecolor{currentstroke}{rgb}{1.000000,1.000000,1.000000}%
\pgfsetstrokecolor{currentstroke}%
\pgfsetdash{}{0pt}%
\pgfpathmoveto{\pgfqpoint{5.420000in}{0.663635in}}%
\pgfpathlineto{\pgfqpoint{5.420000in}{2.320000in}}%
\pgfusepath{stroke}%
\end{pgfscope}%
\begin{pgfscope}%
\pgfsetrectcap%
\pgfsetmiterjoin%
\pgfsetlinewidth{1.254687pt}%
\definecolor{currentstroke}{rgb}{1.000000,1.000000,1.000000}%
\pgfsetstrokecolor{currentstroke}%
\pgfsetdash{}{0pt}%
\pgfpathmoveto{\pgfqpoint{0.948751in}{0.663635in}}%
\pgfpathlineto{\pgfqpoint{5.420000in}{0.663635in}}%
\pgfusepath{stroke}%
\end{pgfscope}%
\begin{pgfscope}%
\pgfsetrectcap%
\pgfsetmiterjoin%
\pgfsetlinewidth{1.254687pt}%
\definecolor{currentstroke}{rgb}{1.000000,1.000000,1.000000}%
\pgfsetstrokecolor{currentstroke}%
\pgfsetdash{}{0pt}%
\pgfpathmoveto{\pgfqpoint{0.948751in}{2.320000in}}%
\pgfpathlineto{\pgfqpoint{5.420000in}{2.320000in}}%
\pgfusepath{stroke}%
\end{pgfscope}%
\begin{pgfscope}%
\pgfsetbuttcap%
\pgfsetmiterjoin%
\definecolor{currentfill}{rgb}{0.917647,0.917647,0.949020}%
\pgfsetfillcolor{currentfill}%
\pgfsetfillopacity{0.800000}%
\pgfsetlinewidth{1.003750pt}%
\definecolor{currentstroke}{rgb}{0.800000,0.800000,0.800000}%
\pgfsetstrokecolor{currentstroke}%
\pgfsetstrokeopacity{0.800000}%
\pgfsetdash{}{0pt}%
\pgfpathmoveto{\pgfqpoint{1.026529in}{2.072637in}}%
\pgfpathlineto{\pgfqpoint{3.179448in}{2.072637in}}%
\pgfpathquadraticcurveto{\pgfqpoint{3.201670in}{2.072637in}}{\pgfqpoint{3.201670in}{2.094859in}}%
\pgfpathlineto{\pgfqpoint{3.201670in}{2.242222in}}%
\pgfpathquadraticcurveto{\pgfqpoint{3.201670in}{2.264444in}}{\pgfqpoint{3.179448in}{2.264444in}}%
\pgfpathlineto{\pgfqpoint{1.026529in}{2.264444in}}%
\pgfpathquadraticcurveto{\pgfqpoint{1.004307in}{2.264444in}}{\pgfqpoint{1.004307in}{2.242222in}}%
\pgfpathlineto{\pgfqpoint{1.004307in}{2.094859in}}%
\pgfpathquadraticcurveto{\pgfqpoint{1.004307in}{2.072637in}}{\pgfqpoint{1.026529in}{2.072637in}}%
\pgfpathlineto{\pgfqpoint{1.026529in}{2.072637in}}%
\pgfpathclose%
\pgfusepath{stroke,fill}%
\end{pgfscope}%
\begin{pgfscope}%
\pgfsetroundcap%
\pgfsetroundjoin%
\pgfsetlinewidth{1.505625pt}%
\definecolor{currentstroke}{rgb}{1.000000,0.000000,0.000000}%
\pgfsetstrokecolor{currentstroke}%
\pgfsetdash{}{0pt}%
\pgfpathmoveto{\pgfqpoint{1.048751in}{2.179353in}}%
\pgfpathlineto{\pgfqpoint{1.159862in}{2.179353in}}%
\pgfpathlineto{\pgfqpoint{1.270973in}{2.179353in}}%
\pgfusepath{stroke}%
\end{pgfscope}%
\begin{pgfscope}%
\definecolor{textcolor}{rgb}{0.150000,0.150000,0.150000}%
\pgfsetstrokecolor{textcolor}%
\pgfsetfillcolor{textcolor}%
\pgftext[x=1.359862in,y=2.140464in,left,base]{\color{textcolor}{\sffamily\fontsize{8.000000}{9.600000}\selectfont\catcode`\^=\active\def^{\ifmmode\sp\else\^{}\fi}\catcode`\%=\active\def%{\%}average write operations per second}}%
\end{pgfscope}%
\end{pgfpicture}%
\makeatother%
\endgroup%

    \caption{Stress test of 1 node with 1000000 writes}
    \label{fig:stress-1000000writes-1node}
\end{figure}

\begin{figure}
    \centering
    %% Creator: Matplotlib, PGF backend
%%
%% To include the figure in your LaTeX document, write
%%   \input{<filename>.pgf}
%%
%% Make sure the required packages are loaded in your preamble
%%   \usepackage{pgf}
%%
%% Also ensure that all the required font packages are loaded; for instance,
%% the lmodern package is sometimes necessary when using math font.
%%   \usepackage{lmodern}
%%
%% Figures using additional raster images can only be included by \input if
%% they are in the same directory as the main LaTeX file. For loading figures
%% from other directories you can use the `import` package
%%   \usepackage{import}
%%
%% and then include the figures with
%%   \import{<path to file>}{<filename>.pgf}
%%
%% Matplotlib used the following preamble
%%   \def\mathdefault#1{#1}
%%   \everymath=\expandafter{\the\everymath\displaystyle}
%%   
%%   \usepackage{fontspec}
%%   \setmainfont{DejaVuSerif.ttf}[Path=\detokenize{/Users/nkratky/private/polaris-elasticity-strategies/test/scripts/.venv/lib/python3.11/site-packages/matplotlib/mpl-data/fonts/ttf/}]
%%   \setsansfont{Arial.ttf}[Path=\detokenize{/System/Library/Fonts/Supplemental/}]
%%   \setmonofont{DejaVuSansMono.ttf}[Path=\detokenize{/Users/nkratky/private/polaris-elasticity-strategies/test/scripts/.venv/lib/python3.11/site-packages/matplotlib/mpl-data/fonts/ttf/}]
%%   \makeatletter\@ifpackageloaded{underscore}{}{\usepackage[strings]{underscore}}\makeatother
%%
\begingroup%
\makeatletter%
\begin{pgfpicture}%
\pgfpathrectangle{\pgfpointorigin}{\pgfqpoint{5.600000in}{2.500000in}}%
\pgfusepath{use as bounding box, clip}%
\begin{pgfscope}%
\pgfsetbuttcap%
\pgfsetmiterjoin%
\definecolor{currentfill}{rgb}{1.000000,1.000000,1.000000}%
\pgfsetfillcolor{currentfill}%
\pgfsetlinewidth{0.000000pt}%
\definecolor{currentstroke}{rgb}{1.000000,1.000000,1.000000}%
\pgfsetstrokecolor{currentstroke}%
\pgfsetdash{}{0pt}%
\pgfpathmoveto{\pgfqpoint{0.000000in}{0.000000in}}%
\pgfpathlineto{\pgfqpoint{5.600000in}{0.000000in}}%
\pgfpathlineto{\pgfqpoint{5.600000in}{2.500000in}}%
\pgfpathlineto{\pgfqpoint{0.000000in}{2.500000in}}%
\pgfpathlineto{\pgfqpoint{0.000000in}{0.000000in}}%
\pgfpathclose%
\pgfusepath{fill}%
\end{pgfscope}%
\begin{pgfscope}%
\pgfsetbuttcap%
\pgfsetmiterjoin%
\definecolor{currentfill}{rgb}{0.917647,0.917647,0.949020}%
\pgfsetfillcolor{currentfill}%
\pgfsetlinewidth{0.000000pt}%
\definecolor{currentstroke}{rgb}{0.000000,0.000000,0.000000}%
\pgfsetstrokecolor{currentstroke}%
\pgfsetstrokeopacity{0.000000}%
\pgfsetdash{}{0pt}%
\pgfpathmoveto{\pgfqpoint{0.948751in}{0.663635in}}%
\pgfpathlineto{\pgfqpoint{5.343693in}{0.663635in}}%
\pgfpathlineto{\pgfqpoint{5.343693in}{2.320000in}}%
\pgfpathlineto{\pgfqpoint{0.948751in}{2.320000in}}%
\pgfpathlineto{\pgfqpoint{0.948751in}{0.663635in}}%
\pgfpathclose%
\pgfusepath{fill}%
\end{pgfscope}%
\begin{pgfscope}%
\pgfpathrectangle{\pgfqpoint{0.948751in}{0.663635in}}{\pgfqpoint{4.394942in}{1.656365in}}%
\pgfusepath{clip}%
\pgfsetroundcap%
\pgfsetroundjoin%
\pgfsetlinewidth{1.003750pt}%
\definecolor{currentstroke}{rgb}{1.000000,1.000000,1.000000}%
\pgfsetstrokecolor{currentstroke}%
\pgfsetdash{}{0pt}%
\pgfpathmoveto{\pgfqpoint{1.148521in}{0.663635in}}%
\pgfpathlineto{\pgfqpoint{1.148521in}{2.320000in}}%
\pgfusepath{stroke}%
\end{pgfscope}%
\begin{pgfscope}%
\definecolor{textcolor}{rgb}{0.150000,0.150000,0.150000}%
\pgfsetstrokecolor{textcolor}%
\pgfsetfillcolor{textcolor}%
\pgftext[x=1.148521in,y=0.531691in,,top]{\color{textcolor}{\sffamily\fontsize{11.000000}{13.200000}\selectfont\catcode`\^=\active\def^{\ifmmode\sp\else\^{}\fi}\catcode`\%=\active\def%{\%}0}}%
\end{pgfscope}%
\begin{pgfscope}%
\pgfpathrectangle{\pgfqpoint{0.948751in}{0.663635in}}{\pgfqpoint{4.394942in}{1.656365in}}%
\pgfusepath{clip}%
\pgfsetroundcap%
\pgfsetroundjoin%
\pgfsetlinewidth{1.003750pt}%
\definecolor{currentstroke}{rgb}{1.000000,1.000000,1.000000}%
\pgfsetstrokecolor{currentstroke}%
\pgfsetdash{}{0pt}%
\pgfpathmoveto{\pgfqpoint{1.740433in}{0.663635in}}%
\pgfpathlineto{\pgfqpoint{1.740433in}{2.320000in}}%
\pgfusepath{stroke}%
\end{pgfscope}%
\begin{pgfscope}%
\definecolor{textcolor}{rgb}{0.150000,0.150000,0.150000}%
\pgfsetstrokecolor{textcolor}%
\pgfsetfillcolor{textcolor}%
\pgftext[x=1.740433in,y=0.531691in,,top]{\color{textcolor}{\sffamily\fontsize{11.000000}{13.200000}\selectfont\catcode`\^=\active\def^{\ifmmode\sp\else\^{}\fi}\catcode`\%=\active\def%{\%}20}}%
\end{pgfscope}%
\begin{pgfscope}%
\pgfpathrectangle{\pgfqpoint{0.948751in}{0.663635in}}{\pgfqpoint{4.394942in}{1.656365in}}%
\pgfusepath{clip}%
\pgfsetroundcap%
\pgfsetroundjoin%
\pgfsetlinewidth{1.003750pt}%
\definecolor{currentstroke}{rgb}{1.000000,1.000000,1.000000}%
\pgfsetstrokecolor{currentstroke}%
\pgfsetdash{}{0pt}%
\pgfpathmoveto{\pgfqpoint{2.332344in}{0.663635in}}%
\pgfpathlineto{\pgfqpoint{2.332344in}{2.320000in}}%
\pgfusepath{stroke}%
\end{pgfscope}%
\begin{pgfscope}%
\definecolor{textcolor}{rgb}{0.150000,0.150000,0.150000}%
\pgfsetstrokecolor{textcolor}%
\pgfsetfillcolor{textcolor}%
\pgftext[x=2.332344in,y=0.531691in,,top]{\color{textcolor}{\sffamily\fontsize{11.000000}{13.200000}\selectfont\catcode`\^=\active\def^{\ifmmode\sp\else\^{}\fi}\catcode`\%=\active\def%{\%}40}}%
\end{pgfscope}%
\begin{pgfscope}%
\pgfpathrectangle{\pgfqpoint{0.948751in}{0.663635in}}{\pgfqpoint{4.394942in}{1.656365in}}%
\pgfusepath{clip}%
\pgfsetroundcap%
\pgfsetroundjoin%
\pgfsetlinewidth{1.003750pt}%
\definecolor{currentstroke}{rgb}{1.000000,1.000000,1.000000}%
\pgfsetstrokecolor{currentstroke}%
\pgfsetdash{}{0pt}%
\pgfpathmoveto{\pgfqpoint{2.924255in}{0.663635in}}%
\pgfpathlineto{\pgfqpoint{2.924255in}{2.320000in}}%
\pgfusepath{stroke}%
\end{pgfscope}%
\begin{pgfscope}%
\definecolor{textcolor}{rgb}{0.150000,0.150000,0.150000}%
\pgfsetstrokecolor{textcolor}%
\pgfsetfillcolor{textcolor}%
\pgftext[x=2.924255in,y=0.531691in,,top]{\color{textcolor}{\sffamily\fontsize{11.000000}{13.200000}\selectfont\catcode`\^=\active\def^{\ifmmode\sp\else\^{}\fi}\catcode`\%=\active\def%{\%}60}}%
\end{pgfscope}%
\begin{pgfscope}%
\pgfpathrectangle{\pgfqpoint{0.948751in}{0.663635in}}{\pgfqpoint{4.394942in}{1.656365in}}%
\pgfusepath{clip}%
\pgfsetroundcap%
\pgfsetroundjoin%
\pgfsetlinewidth{1.003750pt}%
\definecolor{currentstroke}{rgb}{1.000000,1.000000,1.000000}%
\pgfsetstrokecolor{currentstroke}%
\pgfsetdash{}{0pt}%
\pgfpathmoveto{\pgfqpoint{3.516167in}{0.663635in}}%
\pgfpathlineto{\pgfqpoint{3.516167in}{2.320000in}}%
\pgfusepath{stroke}%
\end{pgfscope}%
\begin{pgfscope}%
\definecolor{textcolor}{rgb}{0.150000,0.150000,0.150000}%
\pgfsetstrokecolor{textcolor}%
\pgfsetfillcolor{textcolor}%
\pgftext[x=3.516167in,y=0.531691in,,top]{\color{textcolor}{\sffamily\fontsize{11.000000}{13.200000}\selectfont\catcode`\^=\active\def^{\ifmmode\sp\else\^{}\fi}\catcode`\%=\active\def%{\%}80}}%
\end{pgfscope}%
\begin{pgfscope}%
\pgfpathrectangle{\pgfqpoint{0.948751in}{0.663635in}}{\pgfqpoint{4.394942in}{1.656365in}}%
\pgfusepath{clip}%
\pgfsetroundcap%
\pgfsetroundjoin%
\pgfsetlinewidth{1.003750pt}%
\definecolor{currentstroke}{rgb}{1.000000,1.000000,1.000000}%
\pgfsetstrokecolor{currentstroke}%
\pgfsetdash{}{0pt}%
\pgfpathmoveto{\pgfqpoint{4.108078in}{0.663635in}}%
\pgfpathlineto{\pgfqpoint{4.108078in}{2.320000in}}%
\pgfusepath{stroke}%
\end{pgfscope}%
\begin{pgfscope}%
\definecolor{textcolor}{rgb}{0.150000,0.150000,0.150000}%
\pgfsetstrokecolor{textcolor}%
\pgfsetfillcolor{textcolor}%
\pgftext[x=4.108078in,y=0.531691in,,top]{\color{textcolor}{\sffamily\fontsize{11.000000}{13.200000}\selectfont\catcode`\^=\active\def^{\ifmmode\sp\else\^{}\fi}\catcode`\%=\active\def%{\%}100}}%
\end{pgfscope}%
\begin{pgfscope}%
\pgfpathrectangle{\pgfqpoint{0.948751in}{0.663635in}}{\pgfqpoint{4.394942in}{1.656365in}}%
\pgfusepath{clip}%
\pgfsetroundcap%
\pgfsetroundjoin%
\pgfsetlinewidth{1.003750pt}%
\definecolor{currentstroke}{rgb}{1.000000,1.000000,1.000000}%
\pgfsetstrokecolor{currentstroke}%
\pgfsetdash{}{0pt}%
\pgfpathmoveto{\pgfqpoint{4.699990in}{0.663635in}}%
\pgfpathlineto{\pgfqpoint{4.699990in}{2.320000in}}%
\pgfusepath{stroke}%
\end{pgfscope}%
\begin{pgfscope}%
\definecolor{textcolor}{rgb}{0.150000,0.150000,0.150000}%
\pgfsetstrokecolor{textcolor}%
\pgfsetfillcolor{textcolor}%
\pgftext[x=4.699990in,y=0.531691in,,top]{\color{textcolor}{\sffamily\fontsize{11.000000}{13.200000}\selectfont\catcode`\^=\active\def^{\ifmmode\sp\else\^{}\fi}\catcode`\%=\active\def%{\%}120}}%
\end{pgfscope}%
\begin{pgfscope}%
\pgfpathrectangle{\pgfqpoint{0.948751in}{0.663635in}}{\pgfqpoint{4.394942in}{1.656365in}}%
\pgfusepath{clip}%
\pgfsetroundcap%
\pgfsetroundjoin%
\pgfsetlinewidth{1.003750pt}%
\definecolor{currentstroke}{rgb}{1.000000,1.000000,1.000000}%
\pgfsetstrokecolor{currentstroke}%
\pgfsetdash{}{0pt}%
\pgfpathmoveto{\pgfqpoint{5.291901in}{0.663635in}}%
\pgfpathlineto{\pgfqpoint{5.291901in}{2.320000in}}%
\pgfusepath{stroke}%
\end{pgfscope}%
\begin{pgfscope}%
\definecolor{textcolor}{rgb}{0.150000,0.150000,0.150000}%
\pgfsetstrokecolor{textcolor}%
\pgfsetfillcolor{textcolor}%
\pgftext[x=5.291901in,y=0.531691in,,top]{\color{textcolor}{\sffamily\fontsize{11.000000}{13.200000}\selectfont\catcode`\^=\active\def^{\ifmmode\sp\else\^{}\fi}\catcode`\%=\active\def%{\%}140}}%
\end{pgfscope}%
\begin{pgfscope}%
\definecolor{textcolor}{rgb}{0.150000,0.150000,0.150000}%
\pgfsetstrokecolor{textcolor}%
\pgfsetfillcolor{textcolor}%
\pgftext[x=3.146222in,y=0.336413in,,top]{\color{textcolor}{\sffamily\fontsize{12.000000}{14.400000}\selectfont\catcode`\^=\active\def^{\ifmmode\sp\else\^{}\fi}\catcode`\%=\active\def%{\%}Time (s)}}%
\end{pgfscope}%
\begin{pgfscope}%
\pgfpathrectangle{\pgfqpoint{0.948751in}{0.663635in}}{\pgfqpoint{4.394942in}{1.656365in}}%
\pgfusepath{clip}%
\pgfsetroundcap%
\pgfsetroundjoin%
\pgfsetlinewidth{1.003750pt}%
\definecolor{currentstroke}{rgb}{1.000000,1.000000,1.000000}%
\pgfsetstrokecolor{currentstroke}%
\pgfsetdash{}{0pt}%
\pgfpathmoveto{\pgfqpoint{0.948751in}{0.738925in}}%
\pgfpathlineto{\pgfqpoint{5.343693in}{0.738925in}}%
\pgfusepath{stroke}%
\end{pgfscope}%
\begin{pgfscope}%
\definecolor{textcolor}{rgb}{0.150000,0.150000,0.150000}%
\pgfsetstrokecolor{textcolor}%
\pgfsetfillcolor{textcolor}%
\pgftext[x=0.731839in, y=0.684244in, left, base]{\color{textcolor}{\sffamily\fontsize{11.000000}{13.200000}\selectfont\catcode`\^=\active\def^{\ifmmode\sp\else\^{}\fi}\catcode`\%=\active\def%{\%}0}}%
\end{pgfscope}%
\begin{pgfscope}%
\pgfpathrectangle{\pgfqpoint{0.948751in}{0.663635in}}{\pgfqpoint{4.394942in}{1.656365in}}%
\pgfusepath{clip}%
\pgfsetroundcap%
\pgfsetroundjoin%
\pgfsetlinewidth{1.003750pt}%
\definecolor{currentstroke}{rgb}{1.000000,1.000000,1.000000}%
\pgfsetstrokecolor{currentstroke}%
\pgfsetdash{}{0pt}%
\pgfpathmoveto{\pgfqpoint{0.948751in}{1.247104in}}%
\pgfpathlineto{\pgfqpoint{5.343693in}{1.247104in}}%
\pgfusepath{stroke}%
\end{pgfscope}%
\begin{pgfscope}%
\definecolor{textcolor}{rgb}{0.150000,0.150000,0.150000}%
\pgfsetstrokecolor{textcolor}%
\pgfsetfillcolor{textcolor}%
\pgftext[x=0.391968in, y=1.192423in, left, base]{\color{textcolor}{\sffamily\fontsize{11.000000}{13.200000}\selectfont\catcode`\^=\active\def^{\ifmmode\sp\else\^{}\fi}\catcode`\%=\active\def%{\%}10000}}%
\end{pgfscope}%
\begin{pgfscope}%
\pgfpathrectangle{\pgfqpoint{0.948751in}{0.663635in}}{\pgfqpoint{4.394942in}{1.656365in}}%
\pgfusepath{clip}%
\pgfsetroundcap%
\pgfsetroundjoin%
\pgfsetlinewidth{1.003750pt}%
\definecolor{currentstroke}{rgb}{1.000000,1.000000,1.000000}%
\pgfsetstrokecolor{currentstroke}%
\pgfsetdash{}{0pt}%
\pgfpathmoveto{\pgfqpoint{0.948751in}{1.755283in}}%
\pgfpathlineto{\pgfqpoint{5.343693in}{1.755283in}}%
\pgfusepath{stroke}%
\end{pgfscope}%
\begin{pgfscope}%
\definecolor{textcolor}{rgb}{0.150000,0.150000,0.150000}%
\pgfsetstrokecolor{textcolor}%
\pgfsetfillcolor{textcolor}%
\pgftext[x=0.391968in, y=1.700603in, left, base]{\color{textcolor}{\sffamily\fontsize{11.000000}{13.200000}\selectfont\catcode`\^=\active\def^{\ifmmode\sp\else\^{}\fi}\catcode`\%=\active\def%{\%}20000}}%
\end{pgfscope}%
\begin{pgfscope}%
\pgfpathrectangle{\pgfqpoint{0.948751in}{0.663635in}}{\pgfqpoint{4.394942in}{1.656365in}}%
\pgfusepath{clip}%
\pgfsetroundcap%
\pgfsetroundjoin%
\pgfsetlinewidth{1.003750pt}%
\definecolor{currentstroke}{rgb}{1.000000,1.000000,1.000000}%
\pgfsetstrokecolor{currentstroke}%
\pgfsetdash{}{0pt}%
\pgfpathmoveto{\pgfqpoint{0.948751in}{2.263463in}}%
\pgfpathlineto{\pgfqpoint{5.343693in}{2.263463in}}%
\pgfusepath{stroke}%
\end{pgfscope}%
\begin{pgfscope}%
\definecolor{textcolor}{rgb}{0.150000,0.150000,0.150000}%
\pgfsetstrokecolor{textcolor}%
\pgfsetfillcolor{textcolor}%
\pgftext[x=0.391968in, y=2.208782in, left, base]{\color{textcolor}{\sffamily\fontsize{11.000000}{13.200000}\selectfont\catcode`\^=\active\def^{\ifmmode\sp\else\^{}\fi}\catcode`\%=\active\def%{\%}30000}}%
\end{pgfscope}%
\begin{pgfscope}%
\definecolor{textcolor}{rgb}{0.150000,0.150000,0.150000}%
\pgfsetstrokecolor{textcolor}%
\pgfsetfillcolor{textcolor}%
\pgftext[x=0.336413in,y=1.491818in,,bottom,rotate=90.000000]{\color{textcolor}{\sffamily\fontsize{12.000000}{14.400000}\selectfont\catcode`\^=\active\def^{\ifmmode\sp\else\^{}\fi}\catcode`\%=\active\def%{\%}Writes (op/s)}}%
\end{pgfscope}%
\begin{pgfscope}%
\pgfpathrectangle{\pgfqpoint{0.948751in}{0.663635in}}{\pgfqpoint{4.394942in}{1.656365in}}%
\pgfusepath{clip}%
\pgfsetroundcap%
\pgfsetroundjoin%
\pgfsetlinewidth{1.505625pt}%
\definecolor{currentstroke}{rgb}{0.298039,0.447059,0.690196}%
\pgfsetstrokecolor{currentstroke}%
\pgfsetdash{}{0pt}%
\pgfpathmoveto{\pgfqpoint{1.148521in}{0.738925in}}%
\pgfpathlineto{\pgfqpoint{1.296499in}{0.738925in}}%
\pgfpathlineto{\pgfqpoint{1.444477in}{0.738925in}}%
\pgfpathlineto{\pgfqpoint{1.592455in}{0.741339in}}%
\pgfpathlineto{\pgfqpoint{1.740433in}{0.743752in}}%
\pgfpathlineto{\pgfqpoint{1.888411in}{0.743752in}}%
\pgfpathlineto{\pgfqpoint{2.036388in}{1.208177in}}%
\pgfpathlineto{\pgfqpoint{2.184366in}{1.672602in}}%
\pgfpathlineto{\pgfqpoint{2.332344in}{1.672602in}}%
\pgfpathlineto{\pgfqpoint{2.480322in}{1.902350in}}%
\pgfpathlineto{\pgfqpoint{2.628300in}{2.132098in}}%
\pgfpathlineto{\pgfqpoint{2.776278in}{2.132098in}}%
\pgfpathlineto{\pgfqpoint{2.924255in}{2.188404in}}%
\pgfpathlineto{\pgfqpoint{3.072233in}{2.244711in}}%
\pgfpathlineto{\pgfqpoint{3.220211in}{2.244711in}}%
\pgfpathlineto{\pgfqpoint{3.368189in}{2.192317in}}%
\pgfpathlineto{\pgfqpoint{3.516167in}{2.089055in}}%
\pgfpathlineto{\pgfqpoint{3.664145in}{2.089055in}}%
\pgfpathlineto{\pgfqpoint{3.812123in}{2.026905in}}%
\pgfpathlineto{\pgfqpoint{3.960100in}{1.659796in}}%
\pgfpathlineto{\pgfqpoint{4.108078in}{1.507343in}}%
\pgfpathlineto{\pgfqpoint{4.256056in}{1.194203in}}%
\pgfpathlineto{\pgfqpoint{4.404034in}{0.881062in}}%
\pgfpathlineto{\pgfqpoint{4.552012in}{0.881062in}}%
\pgfpathlineto{\pgfqpoint{4.699990in}{0.809968in}}%
\pgfpathlineto{\pgfqpoint{4.847967in}{0.738925in}}%
\pgfpathlineto{\pgfqpoint{4.995945in}{0.738925in}}%
\pgfpathlineto{\pgfqpoint{5.143923in}{0.738925in}}%
\pgfusepath{stroke}%
\end{pgfscope}%
\begin{pgfscope}%
\pgfpathrectangle{\pgfqpoint{0.948751in}{0.663635in}}{\pgfqpoint{4.394942in}{1.656365in}}%
\pgfusepath{clip}%
\pgfsetroundcap%
\pgfsetroundjoin%
\pgfsetlinewidth{1.505625pt}%
\definecolor{currentstroke}{rgb}{1.000000,0.000000,0.000000}%
\pgfsetstrokecolor{currentstroke}%
\pgfsetdash{}{0pt}%
\pgfpathmoveto{\pgfqpoint{0.948751in}{1.406818in}}%
\pgfpathlineto{\pgfqpoint{5.343693in}{1.406818in}}%
\pgfusepath{stroke}%
\end{pgfscope}%
\begin{pgfscope}%
\pgfsetrectcap%
\pgfsetmiterjoin%
\pgfsetlinewidth{1.254687pt}%
\definecolor{currentstroke}{rgb}{1.000000,1.000000,1.000000}%
\pgfsetstrokecolor{currentstroke}%
\pgfsetdash{}{0pt}%
\pgfpathmoveto{\pgfqpoint{0.948751in}{0.663635in}}%
\pgfpathlineto{\pgfqpoint{0.948751in}{2.320000in}}%
\pgfusepath{stroke}%
\end{pgfscope}%
\begin{pgfscope}%
\pgfsetrectcap%
\pgfsetmiterjoin%
\pgfsetlinewidth{1.254687pt}%
\definecolor{currentstroke}{rgb}{1.000000,1.000000,1.000000}%
\pgfsetstrokecolor{currentstroke}%
\pgfsetdash{}{0pt}%
\pgfpathmoveto{\pgfqpoint{5.343693in}{0.663635in}}%
\pgfpathlineto{\pgfqpoint{5.343693in}{2.320000in}}%
\pgfusepath{stroke}%
\end{pgfscope}%
\begin{pgfscope}%
\pgfsetrectcap%
\pgfsetmiterjoin%
\pgfsetlinewidth{1.254687pt}%
\definecolor{currentstroke}{rgb}{1.000000,1.000000,1.000000}%
\pgfsetstrokecolor{currentstroke}%
\pgfsetdash{}{0pt}%
\pgfpathmoveto{\pgfqpoint{0.948751in}{0.663635in}}%
\pgfpathlineto{\pgfqpoint{5.343693in}{0.663635in}}%
\pgfusepath{stroke}%
\end{pgfscope}%
\begin{pgfscope}%
\pgfsetrectcap%
\pgfsetmiterjoin%
\pgfsetlinewidth{1.254687pt}%
\definecolor{currentstroke}{rgb}{1.000000,1.000000,1.000000}%
\pgfsetstrokecolor{currentstroke}%
\pgfsetdash{}{0pt}%
\pgfpathmoveto{\pgfqpoint{0.948751in}{2.320000in}}%
\pgfpathlineto{\pgfqpoint{5.343693in}{2.320000in}}%
\pgfusepath{stroke}%
\end{pgfscope}%
\begin{pgfscope}%
\pgfsetbuttcap%
\pgfsetmiterjoin%
\definecolor{currentfill}{rgb}{0.917647,0.917647,0.949020}%
\pgfsetfillcolor{currentfill}%
\pgfsetfillopacity{0.800000}%
\pgfsetlinewidth{1.003750pt}%
\definecolor{currentstroke}{rgb}{0.800000,0.800000,0.800000}%
\pgfsetstrokecolor{currentstroke}%
\pgfsetstrokeopacity{0.800000}%
\pgfsetdash{}{0pt}%
\pgfpathmoveto{\pgfqpoint{2.069763in}{0.719191in}}%
\pgfpathlineto{\pgfqpoint{4.222682in}{0.719191in}}%
\pgfpathquadraticcurveto{\pgfqpoint{4.244904in}{0.719191in}}{\pgfqpoint{4.244904in}{0.741413in}}%
\pgfpathlineto{\pgfqpoint{4.244904in}{0.888776in}}%
\pgfpathquadraticcurveto{\pgfqpoint{4.244904in}{0.910998in}}{\pgfqpoint{4.222682in}{0.910998in}}%
\pgfpathlineto{\pgfqpoint{2.069763in}{0.910998in}}%
\pgfpathquadraticcurveto{\pgfqpoint{2.047541in}{0.910998in}}{\pgfqpoint{2.047541in}{0.888776in}}%
\pgfpathlineto{\pgfqpoint{2.047541in}{0.741413in}}%
\pgfpathquadraticcurveto{\pgfqpoint{2.047541in}{0.719191in}}{\pgfqpoint{2.069763in}{0.719191in}}%
\pgfpathlineto{\pgfqpoint{2.069763in}{0.719191in}}%
\pgfpathclose%
\pgfusepath{stroke,fill}%
\end{pgfscope}%
\begin{pgfscope}%
\pgfsetroundcap%
\pgfsetroundjoin%
\pgfsetlinewidth{1.505625pt}%
\definecolor{currentstroke}{rgb}{1.000000,0.000000,0.000000}%
\pgfsetstrokecolor{currentstroke}%
\pgfsetdash{}{0pt}%
\pgfpathmoveto{\pgfqpoint{2.091985in}{0.825907in}}%
\pgfpathlineto{\pgfqpoint{2.203096in}{0.825907in}}%
\pgfpathlineto{\pgfqpoint{2.314207in}{0.825907in}}%
\pgfusepath{stroke}%
\end{pgfscope}%
\begin{pgfscope}%
\definecolor{textcolor}{rgb}{0.150000,0.150000,0.150000}%
\pgfsetstrokecolor{textcolor}%
\pgfsetfillcolor{textcolor}%
\pgftext[x=2.403096in,y=0.787018in,left,base]{\color{textcolor}{\sffamily\fontsize{8.000000}{9.600000}\selectfont\catcode`\^=\active\def^{\ifmmode\sp\else\^{}\fi}\catcode`\%=\active\def%{\%}average write operations per second}}%
\end{pgfscope}%
\end{pgfpicture}%
\makeatother%
\endgroup%

    \caption{Stress test of 2 nodes with 1000000 writes}
    \label{fig:stress-1000000writes-2node}
\end{figure}

\begin{figure}
    \centering
    %% Creator: Matplotlib, PGF backend
%%
%% To include the figure in your LaTeX document, write
%%   \input{<filename>.pgf}
%%
%% Make sure the required packages are loaded in your preamble
%%   \usepackage{pgf}
%%
%% Also ensure that all the required font packages are loaded; for instance,
%% the lmodern package is sometimes necessary when using math font.
%%   \usepackage{lmodern}
%%
%% Figures using additional raster images can only be included by \input if
%% they are in the same directory as the main LaTeX file. For loading figures
%% from other directories you can use the `import` package
%%   \usepackage{import}
%%
%% and then include the figures with
%%   \import{<path to file>}{<filename>.pgf}
%%
%% Matplotlib used the following preamble
%%   \def\mathdefault#1{#1}
%%   \everymath=\expandafter{\the\everymath\displaystyle}
%%   
%%   \usepackage{fontspec}
%%   \setmainfont{DejaVuSerif.ttf}[Path=\detokenize{/Users/nkratky/private/polaris-elasticity-strategies/test/scripts/.venv/lib/python3.11/site-packages/matplotlib/mpl-data/fonts/ttf/}]
%%   \setsansfont{Arial.ttf}[Path=\detokenize{/System/Library/Fonts/Supplemental/}]
%%   \setmonofont{DejaVuSansMono.ttf}[Path=\detokenize{/Users/nkratky/private/polaris-elasticity-strategies/test/scripts/.venv/lib/python3.11/site-packages/matplotlib/mpl-data/fonts/ttf/}]
%%   \makeatletter\@ifpackageloaded{underscore}{}{\usepackage[strings]{underscore}}\makeatother
%%
\begingroup%
\makeatletter%
\begin{pgfpicture}%
\pgfpathrectangle{\pgfpointorigin}{\pgfqpoint{5.600000in}{2.500000in}}%
\pgfusepath{use as bounding box, clip}%
\begin{pgfscope}%
\pgfsetbuttcap%
\pgfsetmiterjoin%
\definecolor{currentfill}{rgb}{1.000000,1.000000,1.000000}%
\pgfsetfillcolor{currentfill}%
\pgfsetlinewidth{0.000000pt}%
\definecolor{currentstroke}{rgb}{1.000000,1.000000,1.000000}%
\pgfsetstrokecolor{currentstroke}%
\pgfsetdash{}{0pt}%
\pgfpathmoveto{\pgfqpoint{0.000000in}{0.000000in}}%
\pgfpathlineto{\pgfqpoint{5.600000in}{0.000000in}}%
\pgfpathlineto{\pgfqpoint{5.600000in}{2.500000in}}%
\pgfpathlineto{\pgfqpoint{0.000000in}{2.500000in}}%
\pgfpathlineto{\pgfqpoint{0.000000in}{0.000000in}}%
\pgfpathclose%
\pgfusepath{fill}%
\end{pgfscope}%
\begin{pgfscope}%
\pgfsetbuttcap%
\pgfsetmiterjoin%
\definecolor{currentfill}{rgb}{0.917647,0.917647,0.949020}%
\pgfsetfillcolor{currentfill}%
\pgfsetlinewidth{0.000000pt}%
\definecolor{currentstroke}{rgb}{0.000000,0.000000,0.000000}%
\pgfsetstrokecolor{currentstroke}%
\pgfsetstrokeopacity{0.000000}%
\pgfsetdash{}{0pt}%
\pgfpathmoveto{\pgfqpoint{0.948751in}{0.663635in}}%
\pgfpathlineto{\pgfqpoint{5.420000in}{0.663635in}}%
\pgfpathlineto{\pgfqpoint{5.420000in}{2.320000in}}%
\pgfpathlineto{\pgfqpoint{0.948751in}{2.320000in}}%
\pgfpathlineto{\pgfqpoint{0.948751in}{0.663635in}}%
\pgfpathclose%
\pgfusepath{fill}%
\end{pgfscope}%
\begin{pgfscope}%
\pgfpathrectangle{\pgfqpoint{0.948751in}{0.663635in}}{\pgfqpoint{4.471249in}{1.656365in}}%
\pgfusepath{clip}%
\pgfsetroundcap%
\pgfsetroundjoin%
\pgfsetlinewidth{1.003750pt}%
\definecolor{currentstroke}{rgb}{1.000000,1.000000,1.000000}%
\pgfsetstrokecolor{currentstroke}%
\pgfsetdash{}{0pt}%
\pgfpathmoveto{\pgfqpoint{1.151990in}{0.663635in}}%
\pgfpathlineto{\pgfqpoint{1.151990in}{2.320000in}}%
\pgfusepath{stroke}%
\end{pgfscope}%
\begin{pgfscope}%
\definecolor{textcolor}{rgb}{0.150000,0.150000,0.150000}%
\pgfsetstrokecolor{textcolor}%
\pgfsetfillcolor{textcolor}%
\pgftext[x=1.151990in,y=0.531691in,,top]{\color{textcolor}{\sffamily\fontsize{11.000000}{13.200000}\selectfont\catcode`\^=\active\def^{\ifmmode\sp\else\^{}\fi}\catcode`\%=\active\def%{\%}0}}%
\end{pgfscope}%
\begin{pgfscope}%
\pgfpathrectangle{\pgfqpoint{0.948751in}{0.663635in}}{\pgfqpoint{4.471249in}{1.656365in}}%
\pgfusepath{clip}%
\pgfsetroundcap%
\pgfsetroundjoin%
\pgfsetlinewidth{1.003750pt}%
\definecolor{currentstroke}{rgb}{1.000000,1.000000,1.000000}%
\pgfsetstrokecolor{currentstroke}%
\pgfsetdash{}{0pt}%
\pgfpathmoveto{\pgfqpoint{1.926232in}{0.663635in}}%
\pgfpathlineto{\pgfqpoint{1.926232in}{2.320000in}}%
\pgfusepath{stroke}%
\end{pgfscope}%
\begin{pgfscope}%
\definecolor{textcolor}{rgb}{0.150000,0.150000,0.150000}%
\pgfsetstrokecolor{textcolor}%
\pgfsetfillcolor{textcolor}%
\pgftext[x=1.926232in,y=0.531691in,,top]{\color{textcolor}{\sffamily\fontsize{11.000000}{13.200000}\selectfont\catcode`\^=\active\def^{\ifmmode\sp\else\^{}\fi}\catcode`\%=\active\def%{\%}20}}%
\end{pgfscope}%
\begin{pgfscope}%
\pgfpathrectangle{\pgfqpoint{0.948751in}{0.663635in}}{\pgfqpoint{4.471249in}{1.656365in}}%
\pgfusepath{clip}%
\pgfsetroundcap%
\pgfsetroundjoin%
\pgfsetlinewidth{1.003750pt}%
\definecolor{currentstroke}{rgb}{1.000000,1.000000,1.000000}%
\pgfsetstrokecolor{currentstroke}%
\pgfsetdash{}{0pt}%
\pgfpathmoveto{\pgfqpoint{2.700474in}{0.663635in}}%
\pgfpathlineto{\pgfqpoint{2.700474in}{2.320000in}}%
\pgfusepath{stroke}%
\end{pgfscope}%
\begin{pgfscope}%
\definecolor{textcolor}{rgb}{0.150000,0.150000,0.150000}%
\pgfsetstrokecolor{textcolor}%
\pgfsetfillcolor{textcolor}%
\pgftext[x=2.700474in,y=0.531691in,,top]{\color{textcolor}{\sffamily\fontsize{11.000000}{13.200000}\selectfont\catcode`\^=\active\def^{\ifmmode\sp\else\^{}\fi}\catcode`\%=\active\def%{\%}40}}%
\end{pgfscope}%
\begin{pgfscope}%
\pgfpathrectangle{\pgfqpoint{0.948751in}{0.663635in}}{\pgfqpoint{4.471249in}{1.656365in}}%
\pgfusepath{clip}%
\pgfsetroundcap%
\pgfsetroundjoin%
\pgfsetlinewidth{1.003750pt}%
\definecolor{currentstroke}{rgb}{1.000000,1.000000,1.000000}%
\pgfsetstrokecolor{currentstroke}%
\pgfsetdash{}{0pt}%
\pgfpathmoveto{\pgfqpoint{3.474716in}{0.663635in}}%
\pgfpathlineto{\pgfqpoint{3.474716in}{2.320000in}}%
\pgfusepath{stroke}%
\end{pgfscope}%
\begin{pgfscope}%
\definecolor{textcolor}{rgb}{0.150000,0.150000,0.150000}%
\pgfsetstrokecolor{textcolor}%
\pgfsetfillcolor{textcolor}%
\pgftext[x=3.474716in,y=0.531691in,,top]{\color{textcolor}{\sffamily\fontsize{11.000000}{13.200000}\selectfont\catcode`\^=\active\def^{\ifmmode\sp\else\^{}\fi}\catcode`\%=\active\def%{\%}60}}%
\end{pgfscope}%
\begin{pgfscope}%
\pgfpathrectangle{\pgfqpoint{0.948751in}{0.663635in}}{\pgfqpoint{4.471249in}{1.656365in}}%
\pgfusepath{clip}%
\pgfsetroundcap%
\pgfsetroundjoin%
\pgfsetlinewidth{1.003750pt}%
\definecolor{currentstroke}{rgb}{1.000000,1.000000,1.000000}%
\pgfsetstrokecolor{currentstroke}%
\pgfsetdash{}{0pt}%
\pgfpathmoveto{\pgfqpoint{4.248959in}{0.663635in}}%
\pgfpathlineto{\pgfqpoint{4.248959in}{2.320000in}}%
\pgfusepath{stroke}%
\end{pgfscope}%
\begin{pgfscope}%
\definecolor{textcolor}{rgb}{0.150000,0.150000,0.150000}%
\pgfsetstrokecolor{textcolor}%
\pgfsetfillcolor{textcolor}%
\pgftext[x=4.248959in,y=0.531691in,,top]{\color{textcolor}{\sffamily\fontsize{11.000000}{13.200000}\selectfont\catcode`\^=\active\def^{\ifmmode\sp\else\^{}\fi}\catcode`\%=\active\def%{\%}80}}%
\end{pgfscope}%
\begin{pgfscope}%
\pgfpathrectangle{\pgfqpoint{0.948751in}{0.663635in}}{\pgfqpoint{4.471249in}{1.656365in}}%
\pgfusepath{clip}%
\pgfsetroundcap%
\pgfsetroundjoin%
\pgfsetlinewidth{1.003750pt}%
\definecolor{currentstroke}{rgb}{1.000000,1.000000,1.000000}%
\pgfsetstrokecolor{currentstroke}%
\pgfsetdash{}{0pt}%
\pgfpathmoveto{\pgfqpoint{5.023201in}{0.663635in}}%
\pgfpathlineto{\pgfqpoint{5.023201in}{2.320000in}}%
\pgfusepath{stroke}%
\end{pgfscope}%
\begin{pgfscope}%
\definecolor{textcolor}{rgb}{0.150000,0.150000,0.150000}%
\pgfsetstrokecolor{textcolor}%
\pgfsetfillcolor{textcolor}%
\pgftext[x=5.023201in,y=0.531691in,,top]{\color{textcolor}{\sffamily\fontsize{11.000000}{13.200000}\selectfont\catcode`\^=\active\def^{\ifmmode\sp\else\^{}\fi}\catcode`\%=\active\def%{\%}100}}%
\end{pgfscope}%
\begin{pgfscope}%
\definecolor{textcolor}{rgb}{0.150000,0.150000,0.150000}%
\pgfsetstrokecolor{textcolor}%
\pgfsetfillcolor{textcolor}%
\pgftext[x=3.184376in,y=0.336413in,,top]{\color{textcolor}{\sffamily\fontsize{12.000000}{14.400000}\selectfont\catcode`\^=\active\def^{\ifmmode\sp\else\^{}\fi}\catcode`\%=\active\def%{\%}Time (s)}}%
\end{pgfscope}%
\begin{pgfscope}%
\pgfpathrectangle{\pgfqpoint{0.948751in}{0.663635in}}{\pgfqpoint{4.471249in}{1.656365in}}%
\pgfusepath{clip}%
\pgfsetroundcap%
\pgfsetroundjoin%
\pgfsetlinewidth{1.003750pt}%
\definecolor{currentstroke}{rgb}{1.000000,1.000000,1.000000}%
\pgfsetstrokecolor{currentstroke}%
\pgfsetdash{}{0pt}%
\pgfpathmoveto{\pgfqpoint{0.948751in}{0.738925in}}%
\pgfpathlineto{\pgfqpoint{5.420000in}{0.738925in}}%
\pgfusepath{stroke}%
\end{pgfscope}%
\begin{pgfscope}%
\definecolor{textcolor}{rgb}{0.150000,0.150000,0.150000}%
\pgfsetstrokecolor{textcolor}%
\pgfsetfillcolor{textcolor}%
\pgftext[x=0.731839in, y=0.684244in, left, base]{\color{textcolor}{\sffamily\fontsize{11.000000}{13.200000}\selectfont\catcode`\^=\active\def^{\ifmmode\sp\else\^{}\fi}\catcode`\%=\active\def%{\%}0}}%
\end{pgfscope}%
\begin{pgfscope}%
\pgfpathrectangle{\pgfqpoint{0.948751in}{0.663635in}}{\pgfqpoint{4.471249in}{1.656365in}}%
\pgfusepath{clip}%
\pgfsetroundcap%
\pgfsetroundjoin%
\pgfsetlinewidth{1.003750pt}%
\definecolor{currentstroke}{rgb}{1.000000,1.000000,1.000000}%
\pgfsetstrokecolor{currentstroke}%
\pgfsetdash{}{0pt}%
\pgfpathmoveto{\pgfqpoint{0.948751in}{1.227959in}}%
\pgfpathlineto{\pgfqpoint{5.420000in}{1.227959in}}%
\pgfusepath{stroke}%
\end{pgfscope}%
\begin{pgfscope}%
\definecolor{textcolor}{rgb}{0.150000,0.150000,0.150000}%
\pgfsetstrokecolor{textcolor}%
\pgfsetfillcolor{textcolor}%
\pgftext[x=0.391968in, y=1.173278in, left, base]{\color{textcolor}{\sffamily\fontsize{11.000000}{13.200000}\selectfont\catcode`\^=\active\def^{\ifmmode\sp\else\^{}\fi}\catcode`\%=\active\def%{\%}10000}}%
\end{pgfscope}%
\begin{pgfscope}%
\pgfpathrectangle{\pgfqpoint{0.948751in}{0.663635in}}{\pgfqpoint{4.471249in}{1.656365in}}%
\pgfusepath{clip}%
\pgfsetroundcap%
\pgfsetroundjoin%
\pgfsetlinewidth{1.003750pt}%
\definecolor{currentstroke}{rgb}{1.000000,1.000000,1.000000}%
\pgfsetstrokecolor{currentstroke}%
\pgfsetdash{}{0pt}%
\pgfpathmoveto{\pgfqpoint{0.948751in}{1.716994in}}%
\pgfpathlineto{\pgfqpoint{5.420000in}{1.716994in}}%
\pgfusepath{stroke}%
\end{pgfscope}%
\begin{pgfscope}%
\definecolor{textcolor}{rgb}{0.150000,0.150000,0.150000}%
\pgfsetstrokecolor{textcolor}%
\pgfsetfillcolor{textcolor}%
\pgftext[x=0.391968in, y=1.662313in, left, base]{\color{textcolor}{\sffamily\fontsize{11.000000}{13.200000}\selectfont\catcode`\^=\active\def^{\ifmmode\sp\else\^{}\fi}\catcode`\%=\active\def%{\%}20000}}%
\end{pgfscope}%
\begin{pgfscope}%
\pgfpathrectangle{\pgfqpoint{0.948751in}{0.663635in}}{\pgfqpoint{4.471249in}{1.656365in}}%
\pgfusepath{clip}%
\pgfsetroundcap%
\pgfsetroundjoin%
\pgfsetlinewidth{1.003750pt}%
\definecolor{currentstroke}{rgb}{1.000000,1.000000,1.000000}%
\pgfsetstrokecolor{currentstroke}%
\pgfsetdash{}{0pt}%
\pgfpathmoveto{\pgfqpoint{0.948751in}{2.206028in}}%
\pgfpathlineto{\pgfqpoint{5.420000in}{2.206028in}}%
\pgfusepath{stroke}%
\end{pgfscope}%
\begin{pgfscope}%
\definecolor{textcolor}{rgb}{0.150000,0.150000,0.150000}%
\pgfsetstrokecolor{textcolor}%
\pgfsetfillcolor{textcolor}%
\pgftext[x=0.391968in, y=2.151347in, left, base]{\color{textcolor}{\sffamily\fontsize{11.000000}{13.200000}\selectfont\catcode`\^=\active\def^{\ifmmode\sp\else\^{}\fi}\catcode`\%=\active\def%{\%}30000}}%
\end{pgfscope}%
\begin{pgfscope}%
\definecolor{textcolor}{rgb}{0.150000,0.150000,0.150000}%
\pgfsetstrokecolor{textcolor}%
\pgfsetfillcolor{textcolor}%
\pgftext[x=0.336413in,y=1.491818in,,bottom,rotate=90.000000]{\color{textcolor}{\sffamily\fontsize{12.000000}{14.400000}\selectfont\catcode`\^=\active\def^{\ifmmode\sp\else\^{}\fi}\catcode`\%=\active\def%{\%}Writes (op/s)}}%
\end{pgfscope}%
\begin{pgfscope}%
\pgfpathrectangle{\pgfqpoint{0.948751in}{0.663635in}}{\pgfqpoint{4.471249in}{1.656365in}}%
\pgfusepath{clip}%
\pgfsetroundcap%
\pgfsetroundjoin%
\pgfsetlinewidth{1.505625pt}%
\definecolor{currentstroke}{rgb}{0.298039,0.447059,0.690196}%
\pgfsetstrokecolor{currentstroke}%
\pgfsetdash{}{0pt}%
\pgfpathmoveto{\pgfqpoint{1.151990in}{0.738925in}}%
\pgfpathlineto{\pgfqpoint{1.345550in}{0.738925in}}%
\pgfpathlineto{\pgfqpoint{1.539111in}{0.738925in}}%
\pgfpathlineto{\pgfqpoint{1.732671in}{1.014789in}}%
\pgfpathlineto{\pgfqpoint{1.926232in}{1.290653in}}%
\pgfpathlineto{\pgfqpoint{2.119793in}{1.290653in}}%
\pgfpathlineto{\pgfqpoint{2.313353in}{1.552531in}}%
\pgfpathlineto{\pgfqpoint{2.506914in}{1.814360in}}%
\pgfpathlineto{\pgfqpoint{2.700474in}{1.814360in}}%
\pgfpathlineto{\pgfqpoint{2.894035in}{1.974568in}}%
\pgfpathlineto{\pgfqpoint{3.087595in}{2.134776in}}%
\pgfpathlineto{\pgfqpoint{3.281156in}{2.134776in}}%
\pgfpathlineto{\pgfqpoint{3.474716in}{2.189743in}}%
\pgfpathlineto{\pgfqpoint{3.668277in}{2.244711in}}%
\pgfpathlineto{\pgfqpoint{3.861838in}{2.244711in}}%
\pgfpathlineto{\pgfqpoint{4.055398in}{1.794848in}}%
\pgfpathlineto{\pgfqpoint{4.248959in}{1.344985in}}%
\pgfpathlineto{\pgfqpoint{4.442519in}{1.344985in}}%
\pgfpathlineto{\pgfqpoint{4.636080in}{1.041979in}}%
\pgfpathlineto{\pgfqpoint{4.829640in}{0.738925in}}%
\pgfpathlineto{\pgfqpoint{5.023201in}{0.738925in}}%
\pgfpathlineto{\pgfqpoint{5.216761in}{0.738925in}}%
\pgfusepath{stroke}%
\end{pgfscope}%
\begin{pgfscope}%
\pgfpathrectangle{\pgfqpoint{0.948751in}{0.663635in}}{\pgfqpoint{4.471249in}{1.656365in}}%
\pgfusepath{clip}%
\pgfsetroundcap%
\pgfsetroundjoin%
\pgfsetlinewidth{1.505625pt}%
\definecolor{currentstroke}{rgb}{1.000000,0.000000,0.000000}%
\pgfsetstrokecolor{currentstroke}%
\pgfsetdash{}{0pt}%
\pgfpathmoveto{\pgfqpoint{0.948751in}{1.439135in}}%
\pgfpathlineto{\pgfqpoint{5.420000in}{1.439135in}}%
\pgfusepath{stroke}%
\end{pgfscope}%
\begin{pgfscope}%
\pgfsetrectcap%
\pgfsetmiterjoin%
\pgfsetlinewidth{1.254687pt}%
\definecolor{currentstroke}{rgb}{1.000000,1.000000,1.000000}%
\pgfsetstrokecolor{currentstroke}%
\pgfsetdash{}{0pt}%
\pgfpathmoveto{\pgfqpoint{0.948751in}{0.663635in}}%
\pgfpathlineto{\pgfqpoint{0.948751in}{2.320000in}}%
\pgfusepath{stroke}%
\end{pgfscope}%
\begin{pgfscope}%
\pgfsetrectcap%
\pgfsetmiterjoin%
\pgfsetlinewidth{1.254687pt}%
\definecolor{currentstroke}{rgb}{1.000000,1.000000,1.000000}%
\pgfsetstrokecolor{currentstroke}%
\pgfsetdash{}{0pt}%
\pgfpathmoveto{\pgfqpoint{5.420000in}{0.663635in}}%
\pgfpathlineto{\pgfqpoint{5.420000in}{2.320000in}}%
\pgfusepath{stroke}%
\end{pgfscope}%
\begin{pgfscope}%
\pgfsetrectcap%
\pgfsetmiterjoin%
\pgfsetlinewidth{1.254687pt}%
\definecolor{currentstroke}{rgb}{1.000000,1.000000,1.000000}%
\pgfsetstrokecolor{currentstroke}%
\pgfsetdash{}{0pt}%
\pgfpathmoveto{\pgfqpoint{0.948751in}{0.663635in}}%
\pgfpathlineto{\pgfqpoint{5.420000in}{0.663635in}}%
\pgfusepath{stroke}%
\end{pgfscope}%
\begin{pgfscope}%
\pgfsetrectcap%
\pgfsetmiterjoin%
\pgfsetlinewidth{1.254687pt}%
\definecolor{currentstroke}{rgb}{1.000000,1.000000,1.000000}%
\pgfsetstrokecolor{currentstroke}%
\pgfsetdash{}{0pt}%
\pgfpathmoveto{\pgfqpoint{0.948751in}{2.320000in}}%
\pgfpathlineto{\pgfqpoint{5.420000in}{2.320000in}}%
\pgfusepath{stroke}%
\end{pgfscope}%
\begin{pgfscope}%
\pgfsetbuttcap%
\pgfsetmiterjoin%
\definecolor{currentfill}{rgb}{0.917647,0.917647,0.949020}%
\pgfsetfillcolor{currentfill}%
\pgfsetfillopacity{0.800000}%
\pgfsetlinewidth{1.003750pt}%
\definecolor{currentstroke}{rgb}{0.800000,0.800000,0.800000}%
\pgfsetstrokecolor{currentstroke}%
\pgfsetstrokeopacity{0.800000}%
\pgfsetdash{}{0pt}%
\pgfpathmoveto{\pgfqpoint{2.107916in}{0.719191in}}%
\pgfpathlineto{\pgfqpoint{4.260835in}{0.719191in}}%
\pgfpathquadraticcurveto{\pgfqpoint{4.283057in}{0.719191in}}{\pgfqpoint{4.283057in}{0.741413in}}%
\pgfpathlineto{\pgfqpoint{4.283057in}{0.888776in}}%
\pgfpathquadraticcurveto{\pgfqpoint{4.283057in}{0.910998in}}{\pgfqpoint{4.260835in}{0.910998in}}%
\pgfpathlineto{\pgfqpoint{2.107916in}{0.910998in}}%
\pgfpathquadraticcurveto{\pgfqpoint{2.085694in}{0.910998in}}{\pgfqpoint{2.085694in}{0.888776in}}%
\pgfpathlineto{\pgfqpoint{2.085694in}{0.741413in}}%
\pgfpathquadraticcurveto{\pgfqpoint{2.085694in}{0.719191in}}{\pgfqpoint{2.107916in}{0.719191in}}%
\pgfpathlineto{\pgfqpoint{2.107916in}{0.719191in}}%
\pgfpathclose%
\pgfusepath{stroke,fill}%
\end{pgfscope}%
\begin{pgfscope}%
\pgfsetroundcap%
\pgfsetroundjoin%
\pgfsetlinewidth{1.505625pt}%
\definecolor{currentstroke}{rgb}{1.000000,0.000000,0.000000}%
\pgfsetstrokecolor{currentstroke}%
\pgfsetdash{}{0pt}%
\pgfpathmoveto{\pgfqpoint{2.130138in}{0.825907in}}%
\pgfpathlineto{\pgfqpoint{2.241250in}{0.825907in}}%
\pgfpathlineto{\pgfqpoint{2.352361in}{0.825907in}}%
\pgfusepath{stroke}%
\end{pgfscope}%
\begin{pgfscope}%
\definecolor{textcolor}{rgb}{0.150000,0.150000,0.150000}%
\pgfsetstrokecolor{textcolor}%
\pgfsetfillcolor{textcolor}%
\pgftext[x=2.441250in,y=0.787018in,left,base]{\color{textcolor}{\sffamily\fontsize{8.000000}{9.600000}\selectfont\catcode`\^=\active\def^{\ifmmode\sp\else\^{}\fi}\catcode`\%=\active\def%{\%}average write operations per second}}%
\end{pgfscope}%
\end{pgfpicture}%
\makeatother%
\endgroup%

    \caption{Stress test of 3 nodes with 1000000 writes}
    \label{fig:stress-1000000writes-3node}
\end{figure}

\subsection{Vertical Elasticity Strategy}
\label{sec:evaluation-vertical-elasticity}

As mentioned in \cref{sec:vertical-elasticity} the vertical elasticity strategy adjusts the resource claims of k8ssandra according to its CPU and memory utilization.

As it can bee seen in \cref{fig:simple-limits-vertical} the elasticity strategy controller successfully changes the CPU and memory limits of the k8ssandra cluster once it is operational. \Cref{fig:utilization-vertical} shows the CPU and memory utilization that is used for triggering elasticity processes. Because the CPU utilization stays very low even after scaling takes place, it can be assumed that this metric was not a decisive factor. The memory utilization, however, changes notably. Before starting the elasticity strategy controller the actual memory utilization was off by \(>10\%\) from the target memory utilization. This triggers an elasticity event and the resources are adjusted proportionally.

Interestingly, during reconsiliation the exposed metrics of k8ssandra are not very meaningful. During this process utilization values of far more than 100\% are exposed by the metrics controller. In order to keep the diagram clean, these nonsense-metrics have been filtered out. The reconsiliation process is marked red in \cref{fig:utilization-vertical}.

This elasticity strategy mirrors real-life scenarios. The advantage lies in being able to scale down when demand and therefore CPU and memory utilization is low, thus potentially reducing cost. This obviously only applies when not using dedicated resources.

\begin{figure}
    \centering
    %% Creator: Matplotlib, PGF backend
%%
%% To include the figure in your LaTeX document, write
%%   \input{<filename>.pgf}
%%
%% Make sure the required packages are loaded in your preamble
%%   \usepackage{pgf}
%%
%% Also ensure that all the required font packages are loaded; for instance,
%% the lmodern package is sometimes necessary when using math font.
%%   \usepackage{lmodern}
%%
%% Figures using additional raster images can only be included by \input if
%% they are in the same directory as the main LaTeX file. For loading figures
%% from other directories you can use the `import` package
%%   \usepackage{import}
%%
%% and then include the figures with
%%   \import{<path to file>}{<filename>.pgf}
%%
%% Matplotlib used the following preamble
%%   \def\mathdefault#1{#1}
%%   \everymath=\expandafter{\the\everymath\displaystyle}
%%   
%%   \usepackage{fontspec}
%%   \setmainfont{DejaVuSerif.ttf}[Path=\detokenize{/Users/nkratky/private/polaris-elasticity-strategies/test/scripts/.venv/lib/python3.11/site-packages/matplotlib/mpl-data/fonts/ttf/}]
%%   \setsansfont{Arial.ttf}[Path=\detokenize{/System/Library/Fonts/Supplemental/}]
%%   \setmonofont{DejaVuSansMono.ttf}[Path=\detokenize{/Users/nkratky/private/polaris-elasticity-strategies/test/scripts/.venv/lib/python3.11/site-packages/matplotlib/mpl-data/fonts/ttf/}]
%%   \makeatletter\@ifpackageloaded{underscore}{}{\usepackage[strings]{underscore}}\makeatother
%%
\begingroup%
\makeatletter%
\begin{pgfpicture}%
\pgfpathrectangle{\pgfpointorigin}{\pgfqpoint{5.600000in}{4.000000in}}%
\pgfusepath{use as bounding box, clip}%
\begin{pgfscope}%
\pgfsetbuttcap%
\pgfsetmiterjoin%
\definecolor{currentfill}{rgb}{1.000000,1.000000,1.000000}%
\pgfsetfillcolor{currentfill}%
\pgfsetlinewidth{0.000000pt}%
\definecolor{currentstroke}{rgb}{1.000000,1.000000,1.000000}%
\pgfsetstrokecolor{currentstroke}%
\pgfsetdash{}{0pt}%
\pgfpathmoveto{\pgfqpoint{0.000000in}{0.000000in}}%
\pgfpathlineto{\pgfqpoint{5.600000in}{0.000000in}}%
\pgfpathlineto{\pgfqpoint{5.600000in}{4.000000in}}%
\pgfpathlineto{\pgfqpoint{0.000000in}{4.000000in}}%
\pgfpathlineto{\pgfqpoint{0.000000in}{0.000000in}}%
\pgfpathclose%
\pgfusepath{fill}%
\end{pgfscope}%
\begin{pgfscope}%
\pgfsetbuttcap%
\pgfsetmiterjoin%
\definecolor{currentfill}{rgb}{0.917647,0.917647,0.949020}%
\pgfsetfillcolor{currentfill}%
\pgfsetlinewidth{0.000000pt}%
\definecolor{currentstroke}{rgb}{0.000000,0.000000,0.000000}%
\pgfsetstrokecolor{currentstroke}%
\pgfsetstrokeopacity{0.000000}%
\pgfsetdash{}{0pt}%
\pgfpathmoveto{\pgfqpoint{0.863783in}{2.546295in}}%
\pgfpathlineto{\pgfqpoint{5.420000in}{2.546295in}}%
\pgfpathlineto{\pgfqpoint{5.420000in}{3.765319in}}%
\pgfpathlineto{\pgfqpoint{0.863783in}{3.765319in}}%
\pgfpathlineto{\pgfqpoint{0.863783in}{2.546295in}}%
\pgfpathclose%
\pgfusepath{fill}%
\end{pgfscope}%
\begin{pgfscope}%
\pgfpathrectangle{\pgfqpoint{0.863783in}{2.546295in}}{\pgfqpoint{4.556217in}{1.219024in}}%
\pgfusepath{clip}%
\pgfsetroundcap%
\pgfsetroundjoin%
\pgfsetlinewidth{1.003750pt}%
\definecolor{currentstroke}{rgb}{1.000000,1.000000,1.000000}%
\pgfsetstrokecolor{currentstroke}%
\pgfsetdash{}{0pt}%
\pgfpathmoveto{\pgfqpoint{1.070884in}{2.546295in}}%
\pgfpathlineto{\pgfqpoint{1.070884in}{3.765319in}}%
\pgfusepath{stroke}%
\end{pgfscope}%
\begin{pgfscope}%
\definecolor{textcolor}{rgb}{0.150000,0.150000,0.150000}%
\pgfsetstrokecolor{textcolor}%
\pgfsetfillcolor{textcolor}%
\pgftext[x=1.070884in,y=2.414351in,,top]{\color{textcolor}{\sffamily\fontsize{11.000000}{13.200000}\selectfont\catcode`\^=\active\def^{\ifmmode\sp\else\^{}\fi}\catcode`\%=\active\def%{\%}0}}%
\end{pgfscope}%
\begin{pgfscope}%
\pgfpathrectangle{\pgfqpoint{0.863783in}{2.546295in}}{\pgfqpoint{4.556217in}{1.219024in}}%
\pgfusepath{clip}%
\pgfsetroundcap%
\pgfsetroundjoin%
\pgfsetlinewidth{1.003750pt}%
\definecolor{currentstroke}{rgb}{1.000000,1.000000,1.000000}%
\pgfsetstrokecolor{currentstroke}%
\pgfsetdash{}{0pt}%
\pgfpathmoveto{\pgfqpoint{1.582244in}{2.546295in}}%
\pgfpathlineto{\pgfqpoint{1.582244in}{3.765319in}}%
\pgfusepath{stroke}%
\end{pgfscope}%
\begin{pgfscope}%
\definecolor{textcolor}{rgb}{0.150000,0.150000,0.150000}%
\pgfsetstrokecolor{textcolor}%
\pgfsetfillcolor{textcolor}%
\pgftext[x=1.582244in,y=2.414351in,,top]{\color{textcolor}{\sffamily\fontsize{11.000000}{13.200000}\selectfont\catcode`\^=\active\def^{\ifmmode\sp\else\^{}\fi}\catcode`\%=\active\def%{\%}200}}%
\end{pgfscope}%
\begin{pgfscope}%
\pgfpathrectangle{\pgfqpoint{0.863783in}{2.546295in}}{\pgfqpoint{4.556217in}{1.219024in}}%
\pgfusepath{clip}%
\pgfsetroundcap%
\pgfsetroundjoin%
\pgfsetlinewidth{1.003750pt}%
\definecolor{currentstroke}{rgb}{1.000000,1.000000,1.000000}%
\pgfsetstrokecolor{currentstroke}%
\pgfsetdash{}{0pt}%
\pgfpathmoveto{\pgfqpoint{2.093604in}{2.546295in}}%
\pgfpathlineto{\pgfqpoint{2.093604in}{3.765319in}}%
\pgfusepath{stroke}%
\end{pgfscope}%
\begin{pgfscope}%
\definecolor{textcolor}{rgb}{0.150000,0.150000,0.150000}%
\pgfsetstrokecolor{textcolor}%
\pgfsetfillcolor{textcolor}%
\pgftext[x=2.093604in,y=2.414351in,,top]{\color{textcolor}{\sffamily\fontsize{11.000000}{13.200000}\selectfont\catcode`\^=\active\def^{\ifmmode\sp\else\^{}\fi}\catcode`\%=\active\def%{\%}400}}%
\end{pgfscope}%
\begin{pgfscope}%
\pgfpathrectangle{\pgfqpoint{0.863783in}{2.546295in}}{\pgfqpoint{4.556217in}{1.219024in}}%
\pgfusepath{clip}%
\pgfsetroundcap%
\pgfsetroundjoin%
\pgfsetlinewidth{1.003750pt}%
\definecolor{currentstroke}{rgb}{1.000000,1.000000,1.000000}%
\pgfsetstrokecolor{currentstroke}%
\pgfsetdash{}{0pt}%
\pgfpathmoveto{\pgfqpoint{2.604964in}{2.546295in}}%
\pgfpathlineto{\pgfqpoint{2.604964in}{3.765319in}}%
\pgfusepath{stroke}%
\end{pgfscope}%
\begin{pgfscope}%
\definecolor{textcolor}{rgb}{0.150000,0.150000,0.150000}%
\pgfsetstrokecolor{textcolor}%
\pgfsetfillcolor{textcolor}%
\pgftext[x=2.604964in,y=2.414351in,,top]{\color{textcolor}{\sffamily\fontsize{11.000000}{13.200000}\selectfont\catcode`\^=\active\def^{\ifmmode\sp\else\^{}\fi}\catcode`\%=\active\def%{\%}600}}%
\end{pgfscope}%
\begin{pgfscope}%
\pgfpathrectangle{\pgfqpoint{0.863783in}{2.546295in}}{\pgfqpoint{4.556217in}{1.219024in}}%
\pgfusepath{clip}%
\pgfsetroundcap%
\pgfsetroundjoin%
\pgfsetlinewidth{1.003750pt}%
\definecolor{currentstroke}{rgb}{1.000000,1.000000,1.000000}%
\pgfsetstrokecolor{currentstroke}%
\pgfsetdash{}{0pt}%
\pgfpathmoveto{\pgfqpoint{3.116324in}{2.546295in}}%
\pgfpathlineto{\pgfqpoint{3.116324in}{3.765319in}}%
\pgfusepath{stroke}%
\end{pgfscope}%
\begin{pgfscope}%
\definecolor{textcolor}{rgb}{0.150000,0.150000,0.150000}%
\pgfsetstrokecolor{textcolor}%
\pgfsetfillcolor{textcolor}%
\pgftext[x=3.116324in,y=2.414351in,,top]{\color{textcolor}{\sffamily\fontsize{11.000000}{13.200000}\selectfont\catcode`\^=\active\def^{\ifmmode\sp\else\^{}\fi}\catcode`\%=\active\def%{\%}800}}%
\end{pgfscope}%
\begin{pgfscope}%
\pgfpathrectangle{\pgfqpoint{0.863783in}{2.546295in}}{\pgfqpoint{4.556217in}{1.219024in}}%
\pgfusepath{clip}%
\pgfsetroundcap%
\pgfsetroundjoin%
\pgfsetlinewidth{1.003750pt}%
\definecolor{currentstroke}{rgb}{1.000000,1.000000,1.000000}%
\pgfsetstrokecolor{currentstroke}%
\pgfsetdash{}{0pt}%
\pgfpathmoveto{\pgfqpoint{3.627684in}{2.546295in}}%
\pgfpathlineto{\pgfqpoint{3.627684in}{3.765319in}}%
\pgfusepath{stroke}%
\end{pgfscope}%
\begin{pgfscope}%
\definecolor{textcolor}{rgb}{0.150000,0.150000,0.150000}%
\pgfsetstrokecolor{textcolor}%
\pgfsetfillcolor{textcolor}%
\pgftext[x=3.627684in,y=2.414351in,,top]{\color{textcolor}{\sffamily\fontsize{11.000000}{13.200000}\selectfont\catcode`\^=\active\def^{\ifmmode\sp\else\^{}\fi}\catcode`\%=\active\def%{\%}1000}}%
\end{pgfscope}%
\begin{pgfscope}%
\pgfpathrectangle{\pgfqpoint{0.863783in}{2.546295in}}{\pgfqpoint{4.556217in}{1.219024in}}%
\pgfusepath{clip}%
\pgfsetroundcap%
\pgfsetroundjoin%
\pgfsetlinewidth{1.003750pt}%
\definecolor{currentstroke}{rgb}{1.000000,1.000000,1.000000}%
\pgfsetstrokecolor{currentstroke}%
\pgfsetdash{}{0pt}%
\pgfpathmoveto{\pgfqpoint{4.139044in}{2.546295in}}%
\pgfpathlineto{\pgfqpoint{4.139044in}{3.765319in}}%
\pgfusepath{stroke}%
\end{pgfscope}%
\begin{pgfscope}%
\definecolor{textcolor}{rgb}{0.150000,0.150000,0.150000}%
\pgfsetstrokecolor{textcolor}%
\pgfsetfillcolor{textcolor}%
\pgftext[x=4.139044in,y=2.414351in,,top]{\color{textcolor}{\sffamily\fontsize{11.000000}{13.200000}\selectfont\catcode`\^=\active\def^{\ifmmode\sp\else\^{}\fi}\catcode`\%=\active\def%{\%}1200}}%
\end{pgfscope}%
\begin{pgfscope}%
\pgfpathrectangle{\pgfqpoint{0.863783in}{2.546295in}}{\pgfqpoint{4.556217in}{1.219024in}}%
\pgfusepath{clip}%
\pgfsetroundcap%
\pgfsetroundjoin%
\pgfsetlinewidth{1.003750pt}%
\definecolor{currentstroke}{rgb}{1.000000,1.000000,1.000000}%
\pgfsetstrokecolor{currentstroke}%
\pgfsetdash{}{0pt}%
\pgfpathmoveto{\pgfqpoint{4.650403in}{2.546295in}}%
\pgfpathlineto{\pgfqpoint{4.650403in}{3.765319in}}%
\pgfusepath{stroke}%
\end{pgfscope}%
\begin{pgfscope}%
\definecolor{textcolor}{rgb}{0.150000,0.150000,0.150000}%
\pgfsetstrokecolor{textcolor}%
\pgfsetfillcolor{textcolor}%
\pgftext[x=4.650403in,y=2.414351in,,top]{\color{textcolor}{\sffamily\fontsize{11.000000}{13.200000}\selectfont\catcode`\^=\active\def^{\ifmmode\sp\else\^{}\fi}\catcode`\%=\active\def%{\%}1400}}%
\end{pgfscope}%
\begin{pgfscope}%
\pgfpathrectangle{\pgfqpoint{0.863783in}{2.546295in}}{\pgfqpoint{4.556217in}{1.219024in}}%
\pgfusepath{clip}%
\pgfsetroundcap%
\pgfsetroundjoin%
\pgfsetlinewidth{1.003750pt}%
\definecolor{currentstroke}{rgb}{1.000000,1.000000,1.000000}%
\pgfsetstrokecolor{currentstroke}%
\pgfsetdash{}{0pt}%
\pgfpathmoveto{\pgfqpoint{5.161763in}{2.546295in}}%
\pgfpathlineto{\pgfqpoint{5.161763in}{3.765319in}}%
\pgfusepath{stroke}%
\end{pgfscope}%
\begin{pgfscope}%
\definecolor{textcolor}{rgb}{0.150000,0.150000,0.150000}%
\pgfsetstrokecolor{textcolor}%
\pgfsetfillcolor{textcolor}%
\pgftext[x=5.161763in,y=2.414351in,,top]{\color{textcolor}{\sffamily\fontsize{11.000000}{13.200000}\selectfont\catcode`\^=\active\def^{\ifmmode\sp\else\^{}\fi}\catcode`\%=\active\def%{\%}1600}}%
\end{pgfscope}%
\begin{pgfscope}%
\definecolor{textcolor}{rgb}{0.150000,0.150000,0.150000}%
\pgfsetstrokecolor{textcolor}%
\pgfsetfillcolor{textcolor}%
\pgftext[x=3.141892in,y=2.219072in,,top]{\color{textcolor}{\sffamily\fontsize{12.000000}{14.400000}\selectfont\catcode`\^=\active\def^{\ifmmode\sp\else\^{}\fi}\catcode`\%=\active\def%{\%}Time (s)}}%
\end{pgfscope}%
\begin{pgfscope}%
\pgfpathrectangle{\pgfqpoint{0.863783in}{2.546295in}}{\pgfqpoint{4.556217in}{1.219024in}}%
\pgfusepath{clip}%
\pgfsetroundcap%
\pgfsetroundjoin%
\pgfsetlinewidth{1.003750pt}%
\definecolor{currentstroke}{rgb}{1.000000,1.000000,1.000000}%
\pgfsetstrokecolor{currentstroke}%
\pgfsetdash{}{0pt}%
\pgfpathmoveto{\pgfqpoint{0.863783in}{2.790100in}}%
\pgfpathlineto{\pgfqpoint{5.420000in}{2.790100in}}%
\pgfusepath{stroke}%
\end{pgfscope}%
\begin{pgfscope}%
\definecolor{textcolor}{rgb}{0.150000,0.150000,0.150000}%
\pgfsetstrokecolor{textcolor}%
\pgfsetfillcolor{textcolor}%
\pgftext[x=0.476936in, y=2.735419in, left, base]{\color{textcolor}{\sffamily\fontsize{11.000000}{13.200000}\selectfont\catcode`\^=\active\def^{\ifmmode\sp\else\^{}\fi}\catcode`\%=\active\def%{\%}800}}%
\end{pgfscope}%
\begin{pgfscope}%
\pgfpathrectangle{\pgfqpoint{0.863783in}{2.546295in}}{\pgfqpoint{4.556217in}{1.219024in}}%
\pgfusepath{clip}%
\pgfsetroundcap%
\pgfsetroundjoin%
\pgfsetlinewidth{1.003750pt}%
\definecolor{currentstroke}{rgb}{1.000000,1.000000,1.000000}%
\pgfsetstrokecolor{currentstroke}%
\pgfsetdash{}{0pt}%
\pgfpathmoveto{\pgfqpoint{0.863783in}{3.277710in}}%
\pgfpathlineto{\pgfqpoint{5.420000in}{3.277710in}}%
\pgfusepath{stroke}%
\end{pgfscope}%
\begin{pgfscope}%
\definecolor{textcolor}{rgb}{0.150000,0.150000,0.150000}%
\pgfsetstrokecolor{textcolor}%
\pgfsetfillcolor{textcolor}%
\pgftext[x=0.391968in, y=3.223029in, left, base]{\color{textcolor}{\sffamily\fontsize{11.000000}{13.200000}\selectfont\catcode`\^=\active\def^{\ifmmode\sp\else\^{}\fi}\catcode`\%=\active\def%{\%}1000}}%
\end{pgfscope}%
\begin{pgfscope}%
\pgfpathrectangle{\pgfqpoint{0.863783in}{2.546295in}}{\pgfqpoint{4.556217in}{1.219024in}}%
\pgfusepath{clip}%
\pgfsetroundcap%
\pgfsetroundjoin%
\pgfsetlinewidth{1.003750pt}%
\definecolor{currentstroke}{rgb}{1.000000,1.000000,1.000000}%
\pgfsetstrokecolor{currentstroke}%
\pgfsetdash{}{0pt}%
\pgfpathmoveto{\pgfqpoint{0.863783in}{3.765319in}}%
\pgfpathlineto{\pgfqpoint{5.420000in}{3.765319in}}%
\pgfusepath{stroke}%
\end{pgfscope}%
\begin{pgfscope}%
\definecolor{textcolor}{rgb}{0.150000,0.150000,0.150000}%
\pgfsetstrokecolor{textcolor}%
\pgfsetfillcolor{textcolor}%
\pgftext[x=0.391968in, y=3.710639in, left, base]{\color{textcolor}{\sffamily\fontsize{11.000000}{13.200000}\selectfont\catcode`\^=\active\def^{\ifmmode\sp\else\^{}\fi}\catcode`\%=\active\def%{\%}1200}}%
\end{pgfscope}%
\begin{pgfscope}%
\definecolor{textcolor}{rgb}{0.150000,0.150000,0.150000}%
\pgfsetstrokecolor{textcolor}%
\pgfsetfillcolor{textcolor}%
\pgftext[x=0.336413in,y=3.155807in,,bottom,rotate=90.000000]{\color{textcolor}{\sffamily\fontsize{12.000000}{14.400000}\selectfont\catcode`\^=\active\def^{\ifmmode\sp\else\^{}\fi}\catcode`\%=\active\def%{\%}CPU Limits (milliCPU)}}%
\end{pgfscope}%
\begin{pgfscope}%
\pgfpathrectangle{\pgfqpoint{0.863783in}{2.546295in}}{\pgfqpoint{4.556217in}{1.219024in}}%
\pgfusepath{clip}%
\pgfsetroundcap%
\pgfsetroundjoin%
\pgfsetlinewidth{1.505625pt}%
\definecolor{currentstroke}{rgb}{0.298039,0.447059,0.690196}%
\pgfsetstrokecolor{currentstroke}%
\pgfsetdash{}{0pt}%
\pgfpathmoveto{\pgfqpoint{1.070884in}{3.521514in}}%
\pgfpathlineto{\pgfqpoint{1.684516in}{3.521514in}}%
\pgfpathlineto{\pgfqpoint{1.697300in}{3.033905in}}%
\pgfpathlineto{\pgfqpoint{5.212899in}{3.033905in}}%
\pgfpathlineto{\pgfqpoint{5.212899in}{3.033905in}}%
\pgfusepath{stroke}%
\end{pgfscope}%
\begin{pgfscope}%
\pgfsetrectcap%
\pgfsetmiterjoin%
\pgfsetlinewidth{1.254687pt}%
\definecolor{currentstroke}{rgb}{1.000000,1.000000,1.000000}%
\pgfsetstrokecolor{currentstroke}%
\pgfsetdash{}{0pt}%
\pgfpathmoveto{\pgfqpoint{0.863783in}{2.546295in}}%
\pgfpathlineto{\pgfqpoint{0.863783in}{3.765319in}}%
\pgfusepath{stroke}%
\end{pgfscope}%
\begin{pgfscope}%
\pgfsetrectcap%
\pgfsetmiterjoin%
\pgfsetlinewidth{1.254687pt}%
\definecolor{currentstroke}{rgb}{1.000000,1.000000,1.000000}%
\pgfsetstrokecolor{currentstroke}%
\pgfsetdash{}{0pt}%
\pgfpathmoveto{\pgfqpoint{5.420000in}{2.546295in}}%
\pgfpathlineto{\pgfqpoint{5.420000in}{3.765319in}}%
\pgfusepath{stroke}%
\end{pgfscope}%
\begin{pgfscope}%
\pgfsetrectcap%
\pgfsetmiterjoin%
\pgfsetlinewidth{1.254687pt}%
\definecolor{currentstroke}{rgb}{1.000000,1.000000,1.000000}%
\pgfsetstrokecolor{currentstroke}%
\pgfsetdash{}{0pt}%
\pgfpathmoveto{\pgfqpoint{0.863783in}{2.546295in}}%
\pgfpathlineto{\pgfqpoint{5.420000in}{2.546295in}}%
\pgfusepath{stroke}%
\end{pgfscope}%
\begin{pgfscope}%
\pgfsetrectcap%
\pgfsetmiterjoin%
\pgfsetlinewidth{1.254687pt}%
\definecolor{currentstroke}{rgb}{1.000000,1.000000,1.000000}%
\pgfsetstrokecolor{currentstroke}%
\pgfsetdash{}{0pt}%
\pgfpathmoveto{\pgfqpoint{0.863783in}{3.765319in}}%
\pgfpathlineto{\pgfqpoint{5.420000in}{3.765319in}}%
\pgfusepath{stroke}%
\end{pgfscope}%
\begin{pgfscope}%
\pgfsetbuttcap%
\pgfsetmiterjoin%
\definecolor{currentfill}{rgb}{0.917647,0.917647,0.949020}%
\pgfsetfillcolor{currentfill}%
\pgfsetlinewidth{0.000000pt}%
\definecolor{currentstroke}{rgb}{0.000000,0.000000,0.000000}%
\pgfsetstrokecolor{currentstroke}%
\pgfsetstrokeopacity{0.000000}%
\pgfsetdash{}{0pt}%
\pgfpathmoveto{\pgfqpoint{0.863783in}{0.663635in}}%
\pgfpathlineto{\pgfqpoint{5.420000in}{0.663635in}}%
\pgfpathlineto{\pgfqpoint{5.420000in}{1.882660in}}%
\pgfpathlineto{\pgfqpoint{0.863783in}{1.882660in}}%
\pgfpathlineto{\pgfqpoint{0.863783in}{0.663635in}}%
\pgfpathclose%
\pgfusepath{fill}%
\end{pgfscope}%
\begin{pgfscope}%
\pgfpathrectangle{\pgfqpoint{0.863783in}{0.663635in}}{\pgfqpoint{4.556217in}{1.219024in}}%
\pgfusepath{clip}%
\pgfsetroundcap%
\pgfsetroundjoin%
\pgfsetlinewidth{1.003750pt}%
\definecolor{currentstroke}{rgb}{1.000000,1.000000,1.000000}%
\pgfsetstrokecolor{currentstroke}%
\pgfsetdash{}{0pt}%
\pgfpathmoveto{\pgfqpoint{1.070884in}{0.663635in}}%
\pgfpathlineto{\pgfqpoint{1.070884in}{1.882660in}}%
\pgfusepath{stroke}%
\end{pgfscope}%
\begin{pgfscope}%
\definecolor{textcolor}{rgb}{0.150000,0.150000,0.150000}%
\pgfsetstrokecolor{textcolor}%
\pgfsetfillcolor{textcolor}%
\pgftext[x=1.070884in,y=0.531691in,,top]{\color{textcolor}{\sffamily\fontsize{11.000000}{13.200000}\selectfont\catcode`\^=\active\def^{\ifmmode\sp\else\^{}\fi}\catcode`\%=\active\def%{\%}0}}%
\end{pgfscope}%
\begin{pgfscope}%
\pgfpathrectangle{\pgfqpoint{0.863783in}{0.663635in}}{\pgfqpoint{4.556217in}{1.219024in}}%
\pgfusepath{clip}%
\pgfsetroundcap%
\pgfsetroundjoin%
\pgfsetlinewidth{1.003750pt}%
\definecolor{currentstroke}{rgb}{1.000000,1.000000,1.000000}%
\pgfsetstrokecolor{currentstroke}%
\pgfsetdash{}{0pt}%
\pgfpathmoveto{\pgfqpoint{1.582244in}{0.663635in}}%
\pgfpathlineto{\pgfqpoint{1.582244in}{1.882660in}}%
\pgfusepath{stroke}%
\end{pgfscope}%
\begin{pgfscope}%
\definecolor{textcolor}{rgb}{0.150000,0.150000,0.150000}%
\pgfsetstrokecolor{textcolor}%
\pgfsetfillcolor{textcolor}%
\pgftext[x=1.582244in,y=0.531691in,,top]{\color{textcolor}{\sffamily\fontsize{11.000000}{13.200000}\selectfont\catcode`\^=\active\def^{\ifmmode\sp\else\^{}\fi}\catcode`\%=\active\def%{\%}200}}%
\end{pgfscope}%
\begin{pgfscope}%
\pgfpathrectangle{\pgfqpoint{0.863783in}{0.663635in}}{\pgfqpoint{4.556217in}{1.219024in}}%
\pgfusepath{clip}%
\pgfsetroundcap%
\pgfsetroundjoin%
\pgfsetlinewidth{1.003750pt}%
\definecolor{currentstroke}{rgb}{1.000000,1.000000,1.000000}%
\pgfsetstrokecolor{currentstroke}%
\pgfsetdash{}{0pt}%
\pgfpathmoveto{\pgfqpoint{2.093604in}{0.663635in}}%
\pgfpathlineto{\pgfqpoint{2.093604in}{1.882660in}}%
\pgfusepath{stroke}%
\end{pgfscope}%
\begin{pgfscope}%
\definecolor{textcolor}{rgb}{0.150000,0.150000,0.150000}%
\pgfsetstrokecolor{textcolor}%
\pgfsetfillcolor{textcolor}%
\pgftext[x=2.093604in,y=0.531691in,,top]{\color{textcolor}{\sffamily\fontsize{11.000000}{13.200000}\selectfont\catcode`\^=\active\def^{\ifmmode\sp\else\^{}\fi}\catcode`\%=\active\def%{\%}400}}%
\end{pgfscope}%
\begin{pgfscope}%
\pgfpathrectangle{\pgfqpoint{0.863783in}{0.663635in}}{\pgfqpoint{4.556217in}{1.219024in}}%
\pgfusepath{clip}%
\pgfsetroundcap%
\pgfsetroundjoin%
\pgfsetlinewidth{1.003750pt}%
\definecolor{currentstroke}{rgb}{1.000000,1.000000,1.000000}%
\pgfsetstrokecolor{currentstroke}%
\pgfsetdash{}{0pt}%
\pgfpathmoveto{\pgfqpoint{2.604964in}{0.663635in}}%
\pgfpathlineto{\pgfqpoint{2.604964in}{1.882660in}}%
\pgfusepath{stroke}%
\end{pgfscope}%
\begin{pgfscope}%
\definecolor{textcolor}{rgb}{0.150000,0.150000,0.150000}%
\pgfsetstrokecolor{textcolor}%
\pgfsetfillcolor{textcolor}%
\pgftext[x=2.604964in,y=0.531691in,,top]{\color{textcolor}{\sffamily\fontsize{11.000000}{13.200000}\selectfont\catcode`\^=\active\def^{\ifmmode\sp\else\^{}\fi}\catcode`\%=\active\def%{\%}600}}%
\end{pgfscope}%
\begin{pgfscope}%
\pgfpathrectangle{\pgfqpoint{0.863783in}{0.663635in}}{\pgfqpoint{4.556217in}{1.219024in}}%
\pgfusepath{clip}%
\pgfsetroundcap%
\pgfsetroundjoin%
\pgfsetlinewidth{1.003750pt}%
\definecolor{currentstroke}{rgb}{1.000000,1.000000,1.000000}%
\pgfsetstrokecolor{currentstroke}%
\pgfsetdash{}{0pt}%
\pgfpathmoveto{\pgfqpoint{3.116324in}{0.663635in}}%
\pgfpathlineto{\pgfqpoint{3.116324in}{1.882660in}}%
\pgfusepath{stroke}%
\end{pgfscope}%
\begin{pgfscope}%
\definecolor{textcolor}{rgb}{0.150000,0.150000,0.150000}%
\pgfsetstrokecolor{textcolor}%
\pgfsetfillcolor{textcolor}%
\pgftext[x=3.116324in,y=0.531691in,,top]{\color{textcolor}{\sffamily\fontsize{11.000000}{13.200000}\selectfont\catcode`\^=\active\def^{\ifmmode\sp\else\^{}\fi}\catcode`\%=\active\def%{\%}800}}%
\end{pgfscope}%
\begin{pgfscope}%
\pgfpathrectangle{\pgfqpoint{0.863783in}{0.663635in}}{\pgfqpoint{4.556217in}{1.219024in}}%
\pgfusepath{clip}%
\pgfsetroundcap%
\pgfsetroundjoin%
\pgfsetlinewidth{1.003750pt}%
\definecolor{currentstroke}{rgb}{1.000000,1.000000,1.000000}%
\pgfsetstrokecolor{currentstroke}%
\pgfsetdash{}{0pt}%
\pgfpathmoveto{\pgfqpoint{3.627684in}{0.663635in}}%
\pgfpathlineto{\pgfqpoint{3.627684in}{1.882660in}}%
\pgfusepath{stroke}%
\end{pgfscope}%
\begin{pgfscope}%
\definecolor{textcolor}{rgb}{0.150000,0.150000,0.150000}%
\pgfsetstrokecolor{textcolor}%
\pgfsetfillcolor{textcolor}%
\pgftext[x=3.627684in,y=0.531691in,,top]{\color{textcolor}{\sffamily\fontsize{11.000000}{13.200000}\selectfont\catcode`\^=\active\def^{\ifmmode\sp\else\^{}\fi}\catcode`\%=\active\def%{\%}1000}}%
\end{pgfscope}%
\begin{pgfscope}%
\pgfpathrectangle{\pgfqpoint{0.863783in}{0.663635in}}{\pgfqpoint{4.556217in}{1.219024in}}%
\pgfusepath{clip}%
\pgfsetroundcap%
\pgfsetroundjoin%
\pgfsetlinewidth{1.003750pt}%
\definecolor{currentstroke}{rgb}{1.000000,1.000000,1.000000}%
\pgfsetstrokecolor{currentstroke}%
\pgfsetdash{}{0pt}%
\pgfpathmoveto{\pgfqpoint{4.139044in}{0.663635in}}%
\pgfpathlineto{\pgfqpoint{4.139044in}{1.882660in}}%
\pgfusepath{stroke}%
\end{pgfscope}%
\begin{pgfscope}%
\definecolor{textcolor}{rgb}{0.150000,0.150000,0.150000}%
\pgfsetstrokecolor{textcolor}%
\pgfsetfillcolor{textcolor}%
\pgftext[x=4.139044in,y=0.531691in,,top]{\color{textcolor}{\sffamily\fontsize{11.000000}{13.200000}\selectfont\catcode`\^=\active\def^{\ifmmode\sp\else\^{}\fi}\catcode`\%=\active\def%{\%}1200}}%
\end{pgfscope}%
\begin{pgfscope}%
\pgfpathrectangle{\pgfqpoint{0.863783in}{0.663635in}}{\pgfqpoint{4.556217in}{1.219024in}}%
\pgfusepath{clip}%
\pgfsetroundcap%
\pgfsetroundjoin%
\pgfsetlinewidth{1.003750pt}%
\definecolor{currentstroke}{rgb}{1.000000,1.000000,1.000000}%
\pgfsetstrokecolor{currentstroke}%
\pgfsetdash{}{0pt}%
\pgfpathmoveto{\pgfqpoint{4.650403in}{0.663635in}}%
\pgfpathlineto{\pgfqpoint{4.650403in}{1.882660in}}%
\pgfusepath{stroke}%
\end{pgfscope}%
\begin{pgfscope}%
\definecolor{textcolor}{rgb}{0.150000,0.150000,0.150000}%
\pgfsetstrokecolor{textcolor}%
\pgfsetfillcolor{textcolor}%
\pgftext[x=4.650403in,y=0.531691in,,top]{\color{textcolor}{\sffamily\fontsize{11.000000}{13.200000}\selectfont\catcode`\^=\active\def^{\ifmmode\sp\else\^{}\fi}\catcode`\%=\active\def%{\%}1400}}%
\end{pgfscope}%
\begin{pgfscope}%
\pgfpathrectangle{\pgfqpoint{0.863783in}{0.663635in}}{\pgfqpoint{4.556217in}{1.219024in}}%
\pgfusepath{clip}%
\pgfsetroundcap%
\pgfsetroundjoin%
\pgfsetlinewidth{1.003750pt}%
\definecolor{currentstroke}{rgb}{1.000000,1.000000,1.000000}%
\pgfsetstrokecolor{currentstroke}%
\pgfsetdash{}{0pt}%
\pgfpathmoveto{\pgfqpoint{5.161763in}{0.663635in}}%
\pgfpathlineto{\pgfqpoint{5.161763in}{1.882660in}}%
\pgfusepath{stroke}%
\end{pgfscope}%
\begin{pgfscope}%
\definecolor{textcolor}{rgb}{0.150000,0.150000,0.150000}%
\pgfsetstrokecolor{textcolor}%
\pgfsetfillcolor{textcolor}%
\pgftext[x=5.161763in,y=0.531691in,,top]{\color{textcolor}{\sffamily\fontsize{11.000000}{13.200000}\selectfont\catcode`\^=\active\def^{\ifmmode\sp\else\^{}\fi}\catcode`\%=\active\def%{\%}1600}}%
\end{pgfscope}%
\begin{pgfscope}%
\definecolor{textcolor}{rgb}{0.150000,0.150000,0.150000}%
\pgfsetstrokecolor{textcolor}%
\pgfsetfillcolor{textcolor}%
\pgftext[x=3.141892in,y=0.336413in,,top]{\color{textcolor}{\sffamily\fontsize{12.000000}{14.400000}\selectfont\catcode`\^=\active\def^{\ifmmode\sp\else\^{}\fi}\catcode`\%=\active\def%{\%}Time (s)}}%
\end{pgfscope}%
\begin{pgfscope}%
\pgfpathrectangle{\pgfqpoint{0.863783in}{0.663635in}}{\pgfqpoint{4.556217in}{1.219024in}}%
\pgfusepath{clip}%
\pgfsetroundcap%
\pgfsetroundjoin%
\pgfsetlinewidth{1.003750pt}%
\definecolor{currentstroke}{rgb}{1.000000,1.000000,1.000000}%
\pgfsetstrokecolor{currentstroke}%
\pgfsetdash{}{0pt}%
\pgfpathmoveto{\pgfqpoint{0.863783in}{0.934530in}}%
\pgfpathlineto{\pgfqpoint{5.420000in}{0.934530in}}%
\pgfusepath{stroke}%
\end{pgfscope}%
\begin{pgfscope}%
\definecolor{textcolor}{rgb}{0.150000,0.150000,0.150000}%
\pgfsetstrokecolor{textcolor}%
\pgfsetfillcolor{textcolor}%
\pgftext[x=0.391968in, y=0.879849in, left, base]{\color{textcolor}{\sffamily\fontsize{11.000000}{13.200000}\selectfont\catcode`\^=\active\def^{\ifmmode\sp\else\^{}\fi}\catcode`\%=\active\def%{\%}4000}}%
\end{pgfscope}%
\begin{pgfscope}%
\pgfpathrectangle{\pgfqpoint{0.863783in}{0.663635in}}{\pgfqpoint{4.556217in}{1.219024in}}%
\pgfusepath{clip}%
\pgfsetroundcap%
\pgfsetroundjoin%
\pgfsetlinewidth{1.003750pt}%
\definecolor{currentstroke}{rgb}{1.000000,1.000000,1.000000}%
\pgfsetstrokecolor{currentstroke}%
\pgfsetdash{}{0pt}%
\pgfpathmoveto{\pgfqpoint{0.863783in}{1.476318in}}%
\pgfpathlineto{\pgfqpoint{5.420000in}{1.476318in}}%
\pgfusepath{stroke}%
\end{pgfscope}%
\begin{pgfscope}%
\definecolor{textcolor}{rgb}{0.150000,0.150000,0.150000}%
\pgfsetstrokecolor{textcolor}%
\pgfsetfillcolor{textcolor}%
\pgftext[x=0.391968in, y=1.421638in, left, base]{\color{textcolor}{\sffamily\fontsize{11.000000}{13.200000}\selectfont\catcode`\^=\active\def^{\ifmmode\sp\else\^{}\fi}\catcode`\%=\active\def%{\%}6000}}%
\end{pgfscope}%
\begin{pgfscope}%
\definecolor{textcolor}{rgb}{0.150000,0.150000,0.150000}%
\pgfsetstrokecolor{textcolor}%
\pgfsetfillcolor{textcolor}%
\pgftext[x=0.336413in,y=1.273148in,,bottom,rotate=90.000000]{\color{textcolor}{\sffamily\fontsize{12.000000}{14.400000}\selectfont\catcode`\^=\active\def^{\ifmmode\sp\else\^{}\fi}\catcode`\%=\active\def%{\%}Memory Limits (MiB)}}%
\end{pgfscope}%
\begin{pgfscope}%
\pgfpathrectangle{\pgfqpoint{0.863783in}{0.663635in}}{\pgfqpoint{4.556217in}{1.219024in}}%
\pgfusepath{clip}%
\pgfsetroundcap%
\pgfsetroundjoin%
\pgfsetlinewidth{1.505625pt}%
\definecolor{currentstroke}{rgb}{0.298039,0.447059,0.690196}%
\pgfsetstrokecolor{currentstroke}%
\pgfsetdash{}{0pt}%
\pgfpathmoveto{\pgfqpoint{1.070884in}{1.572609in}}%
\pgfpathlineto{\pgfqpoint{1.684516in}{1.572609in}}%
\pgfpathlineto{\pgfqpoint{1.697300in}{0.924968in}}%
\pgfpathlineto{\pgfqpoint{5.212899in}{0.924968in}}%
\pgfpathlineto{\pgfqpoint{5.212899in}{0.924968in}}%
\pgfusepath{stroke}%
\end{pgfscope}%
\begin{pgfscope}%
\pgfsetrectcap%
\pgfsetmiterjoin%
\pgfsetlinewidth{1.254687pt}%
\definecolor{currentstroke}{rgb}{1.000000,1.000000,1.000000}%
\pgfsetstrokecolor{currentstroke}%
\pgfsetdash{}{0pt}%
\pgfpathmoveto{\pgfqpoint{0.863783in}{0.663635in}}%
\pgfpathlineto{\pgfqpoint{0.863783in}{1.882660in}}%
\pgfusepath{stroke}%
\end{pgfscope}%
\begin{pgfscope}%
\pgfsetrectcap%
\pgfsetmiterjoin%
\pgfsetlinewidth{1.254687pt}%
\definecolor{currentstroke}{rgb}{1.000000,1.000000,1.000000}%
\pgfsetstrokecolor{currentstroke}%
\pgfsetdash{}{0pt}%
\pgfpathmoveto{\pgfqpoint{5.420000in}{0.663635in}}%
\pgfpathlineto{\pgfqpoint{5.420000in}{1.882660in}}%
\pgfusepath{stroke}%
\end{pgfscope}%
\begin{pgfscope}%
\pgfsetrectcap%
\pgfsetmiterjoin%
\pgfsetlinewidth{1.254687pt}%
\definecolor{currentstroke}{rgb}{1.000000,1.000000,1.000000}%
\pgfsetstrokecolor{currentstroke}%
\pgfsetdash{}{0pt}%
\pgfpathmoveto{\pgfqpoint{0.863783in}{0.663635in}}%
\pgfpathlineto{\pgfqpoint{5.420000in}{0.663635in}}%
\pgfusepath{stroke}%
\end{pgfscope}%
\begin{pgfscope}%
\pgfsetrectcap%
\pgfsetmiterjoin%
\pgfsetlinewidth{1.254687pt}%
\definecolor{currentstroke}{rgb}{1.000000,1.000000,1.000000}%
\pgfsetstrokecolor{currentstroke}%
\pgfsetdash{}{0pt}%
\pgfpathmoveto{\pgfqpoint{0.863783in}{1.882660in}}%
\pgfpathlineto{\pgfqpoint{5.420000in}{1.882660in}}%
\pgfusepath{stroke}%
\end{pgfscope}%
\end{pgfpicture}%
\makeatother%
\endgroup%

    \caption{Adjustment of CPU and memory limits by the vertical elasticity strategy controller}
    \label{fig:simple-limits-vertical}
\end{figure}

\begin{figure}
    \centering
    %% Creator: Matplotlib, PGF backend
%%
%% To include the figure in your LaTeX document, write
%%   \input{<filename>.pgf}
%%
%% Make sure the required packages are loaded in your preamble
%%   \usepackage{pgf}
%%
%% Also ensure that all the required font packages are loaded; for instance,
%% the lmodern package is sometimes necessary when using math font.
%%   \usepackage{lmodern}
%%
%% Figures using additional raster images can only be included by \input if
%% they are in the same directory as the main LaTeX file. For loading figures
%% from other directories you can use the `import` package
%%   \usepackage{import}
%%
%% and then include the figures with
%%   \import{<path to file>}{<filename>.pgf}
%%
%% Matplotlib used the following preamble
%%   \def\mathdefault#1{#1}
%%   \everymath=\expandafter{\the\everymath\displaystyle}
%%   
%%   \usepackage{fontspec}
%%   \setmainfont{DejaVuSerif.ttf}[Path=\detokenize{/usr/local/lib/python3.11/site-packages/matplotlib/mpl-data/fonts/ttf/}]
%%   \setsansfont{Arial.ttf}[Path=\detokenize{/System/Library/Fonts/Supplemental/}]
%%   \setmonofont{DejaVuSansMono.ttf}[Path=\detokenize{/usr/local/lib/python3.11/site-packages/matplotlib/mpl-data/fonts/ttf/}]
%%   \makeatletter\@ifpackageloaded{underscore}{}{\usepackage[strings]{underscore}}\makeatother
%%
\begingroup%
\makeatletter%
\begin{pgfpicture}%
\pgfpathrectangle{\pgfpointorigin}{\pgfqpoint{5.600000in}{3.500000in}}%
\pgfusepath{use as bounding box, clip}%
\begin{pgfscope}%
\pgfsetbuttcap%
\pgfsetmiterjoin%
\definecolor{currentfill}{rgb}{1.000000,1.000000,1.000000}%
\pgfsetfillcolor{currentfill}%
\pgfsetlinewidth{0.000000pt}%
\definecolor{currentstroke}{rgb}{1.000000,1.000000,1.000000}%
\pgfsetstrokecolor{currentstroke}%
\pgfsetdash{}{0pt}%
\pgfpathmoveto{\pgfqpoint{0.000000in}{0.000000in}}%
\pgfpathlineto{\pgfqpoint{5.600000in}{0.000000in}}%
\pgfpathlineto{\pgfqpoint{5.600000in}{3.500000in}}%
\pgfpathlineto{\pgfqpoint{0.000000in}{3.500000in}}%
\pgfpathlineto{\pgfqpoint{0.000000in}{0.000000in}}%
\pgfpathclose%
\pgfusepath{fill}%
\end{pgfscope}%
\begin{pgfscope}%
\pgfsetbuttcap%
\pgfsetmiterjoin%
\definecolor{currentfill}{rgb}{0.917647,0.917647,0.949020}%
\pgfsetfillcolor{currentfill}%
\pgfsetlinewidth{0.000000pt}%
\definecolor{currentstroke}{rgb}{0.000000,0.000000,0.000000}%
\pgfsetstrokecolor{currentstroke}%
\pgfsetstrokeopacity{0.000000}%
\pgfsetdash{}{0pt}%
\pgfpathmoveto{\pgfqpoint{0.736295in}{2.323635in}}%
\pgfpathlineto{\pgfqpoint{5.420000in}{2.323635in}}%
\pgfpathlineto{\pgfqpoint{5.420000in}{3.320000in}}%
\pgfpathlineto{\pgfqpoint{0.736295in}{3.320000in}}%
\pgfpathlineto{\pgfqpoint{0.736295in}{2.323635in}}%
\pgfpathclose%
\pgfusepath{fill}%
\end{pgfscope}%
\begin{pgfscope}%
\pgfpathrectangle{\pgfqpoint{0.736295in}{2.323635in}}{\pgfqpoint{4.683705in}{0.996365in}}%
\pgfusepath{clip}%
\pgfsetroundcap%
\pgfsetroundjoin%
\pgfsetlinewidth{1.003750pt}%
\definecolor{currentstroke}{rgb}{1.000000,1.000000,1.000000}%
\pgfsetstrokecolor{currentstroke}%
\pgfsetdash{}{0pt}%
\pgfpathmoveto{\pgfqpoint{0.949190in}{2.323635in}}%
\pgfpathlineto{\pgfqpoint{0.949190in}{3.320000in}}%
\pgfusepath{stroke}%
\end{pgfscope}%
\begin{pgfscope}%
\definecolor{textcolor}{rgb}{0.150000,0.150000,0.150000}%
\pgfsetstrokecolor{textcolor}%
\pgfsetfillcolor{textcolor}%
\pgftext[x=0.949190in,y=2.191691in,,top]{\color{textcolor}{\sffamily\fontsize{11.000000}{13.200000}\selectfont\catcode`\^=\active\def^{\ifmmode\sp\else\^{}\fi}\catcode`\%=\active\def%{\%}0}}%
\end{pgfscope}%
\begin{pgfscope}%
\pgfpathrectangle{\pgfqpoint{0.736295in}{2.323635in}}{\pgfqpoint{4.683705in}{0.996365in}}%
\pgfusepath{clip}%
\pgfsetroundcap%
\pgfsetroundjoin%
\pgfsetlinewidth{1.003750pt}%
\definecolor{currentstroke}{rgb}{1.000000,1.000000,1.000000}%
\pgfsetstrokecolor{currentstroke}%
\pgfsetdash{}{0pt}%
\pgfpathmoveto{\pgfqpoint{1.658843in}{2.323635in}}%
\pgfpathlineto{\pgfqpoint{1.658843in}{3.320000in}}%
\pgfusepath{stroke}%
\end{pgfscope}%
\begin{pgfscope}%
\definecolor{textcolor}{rgb}{0.150000,0.150000,0.150000}%
\pgfsetstrokecolor{textcolor}%
\pgfsetfillcolor{textcolor}%
\pgftext[x=1.658843in,y=2.191691in,,top]{\color{textcolor}{\sffamily\fontsize{11.000000}{13.200000}\selectfont\catcode`\^=\active\def^{\ifmmode\sp\else\^{}\fi}\catcode`\%=\active\def%{\%}200}}%
\end{pgfscope}%
\begin{pgfscope}%
\pgfpathrectangle{\pgfqpoint{0.736295in}{2.323635in}}{\pgfqpoint{4.683705in}{0.996365in}}%
\pgfusepath{clip}%
\pgfsetroundcap%
\pgfsetroundjoin%
\pgfsetlinewidth{1.003750pt}%
\definecolor{currentstroke}{rgb}{1.000000,1.000000,1.000000}%
\pgfsetstrokecolor{currentstroke}%
\pgfsetdash{}{0pt}%
\pgfpathmoveto{\pgfqpoint{2.368495in}{2.323635in}}%
\pgfpathlineto{\pgfqpoint{2.368495in}{3.320000in}}%
\pgfusepath{stroke}%
\end{pgfscope}%
\begin{pgfscope}%
\definecolor{textcolor}{rgb}{0.150000,0.150000,0.150000}%
\pgfsetstrokecolor{textcolor}%
\pgfsetfillcolor{textcolor}%
\pgftext[x=2.368495in,y=2.191691in,,top]{\color{textcolor}{\sffamily\fontsize{11.000000}{13.200000}\selectfont\catcode`\^=\active\def^{\ifmmode\sp\else\^{}\fi}\catcode`\%=\active\def%{\%}400}}%
\end{pgfscope}%
\begin{pgfscope}%
\pgfpathrectangle{\pgfqpoint{0.736295in}{2.323635in}}{\pgfqpoint{4.683705in}{0.996365in}}%
\pgfusepath{clip}%
\pgfsetroundcap%
\pgfsetroundjoin%
\pgfsetlinewidth{1.003750pt}%
\definecolor{currentstroke}{rgb}{1.000000,1.000000,1.000000}%
\pgfsetstrokecolor{currentstroke}%
\pgfsetdash{}{0pt}%
\pgfpathmoveto{\pgfqpoint{3.078147in}{2.323635in}}%
\pgfpathlineto{\pgfqpoint{3.078147in}{3.320000in}}%
\pgfusepath{stroke}%
\end{pgfscope}%
\begin{pgfscope}%
\definecolor{textcolor}{rgb}{0.150000,0.150000,0.150000}%
\pgfsetstrokecolor{textcolor}%
\pgfsetfillcolor{textcolor}%
\pgftext[x=3.078147in,y=2.191691in,,top]{\color{textcolor}{\sffamily\fontsize{11.000000}{13.200000}\selectfont\catcode`\^=\active\def^{\ifmmode\sp\else\^{}\fi}\catcode`\%=\active\def%{\%}600}}%
\end{pgfscope}%
\begin{pgfscope}%
\pgfpathrectangle{\pgfqpoint{0.736295in}{2.323635in}}{\pgfqpoint{4.683705in}{0.996365in}}%
\pgfusepath{clip}%
\pgfsetroundcap%
\pgfsetroundjoin%
\pgfsetlinewidth{1.003750pt}%
\definecolor{currentstroke}{rgb}{1.000000,1.000000,1.000000}%
\pgfsetstrokecolor{currentstroke}%
\pgfsetdash{}{0pt}%
\pgfpathmoveto{\pgfqpoint{3.787800in}{2.323635in}}%
\pgfpathlineto{\pgfqpoint{3.787800in}{3.320000in}}%
\pgfusepath{stroke}%
\end{pgfscope}%
\begin{pgfscope}%
\definecolor{textcolor}{rgb}{0.150000,0.150000,0.150000}%
\pgfsetstrokecolor{textcolor}%
\pgfsetfillcolor{textcolor}%
\pgftext[x=3.787800in,y=2.191691in,,top]{\color{textcolor}{\sffamily\fontsize{11.000000}{13.200000}\selectfont\catcode`\^=\active\def^{\ifmmode\sp\else\^{}\fi}\catcode`\%=\active\def%{\%}800}}%
\end{pgfscope}%
\begin{pgfscope}%
\pgfpathrectangle{\pgfqpoint{0.736295in}{2.323635in}}{\pgfqpoint{4.683705in}{0.996365in}}%
\pgfusepath{clip}%
\pgfsetroundcap%
\pgfsetroundjoin%
\pgfsetlinewidth{1.003750pt}%
\definecolor{currentstroke}{rgb}{1.000000,1.000000,1.000000}%
\pgfsetstrokecolor{currentstroke}%
\pgfsetdash{}{0pt}%
\pgfpathmoveto{\pgfqpoint{4.497452in}{2.323635in}}%
\pgfpathlineto{\pgfqpoint{4.497452in}{3.320000in}}%
\pgfusepath{stroke}%
\end{pgfscope}%
\begin{pgfscope}%
\definecolor{textcolor}{rgb}{0.150000,0.150000,0.150000}%
\pgfsetstrokecolor{textcolor}%
\pgfsetfillcolor{textcolor}%
\pgftext[x=4.497452in,y=2.191691in,,top]{\color{textcolor}{\sffamily\fontsize{11.000000}{13.200000}\selectfont\catcode`\^=\active\def^{\ifmmode\sp\else\^{}\fi}\catcode`\%=\active\def%{\%}1000}}%
\end{pgfscope}%
\begin{pgfscope}%
\pgfpathrectangle{\pgfqpoint{0.736295in}{2.323635in}}{\pgfqpoint{4.683705in}{0.996365in}}%
\pgfusepath{clip}%
\pgfsetroundcap%
\pgfsetroundjoin%
\pgfsetlinewidth{1.003750pt}%
\definecolor{currentstroke}{rgb}{1.000000,1.000000,1.000000}%
\pgfsetstrokecolor{currentstroke}%
\pgfsetdash{}{0pt}%
\pgfpathmoveto{\pgfqpoint{5.207104in}{2.323635in}}%
\pgfpathlineto{\pgfqpoint{5.207104in}{3.320000in}}%
\pgfusepath{stroke}%
\end{pgfscope}%
\begin{pgfscope}%
\definecolor{textcolor}{rgb}{0.150000,0.150000,0.150000}%
\pgfsetstrokecolor{textcolor}%
\pgfsetfillcolor{textcolor}%
\pgftext[x=5.207104in,y=2.191691in,,top]{\color{textcolor}{\sffamily\fontsize{11.000000}{13.200000}\selectfont\catcode`\^=\active\def^{\ifmmode\sp\else\^{}\fi}\catcode`\%=\active\def%{\%}1200}}%
\end{pgfscope}%
\begin{pgfscope}%
\definecolor{textcolor}{rgb}{0.150000,0.150000,0.150000}%
\pgfsetstrokecolor{textcolor}%
\pgfsetfillcolor{textcolor}%
\pgftext[x=3.078147in,y=1.996413in,,top]{\color{textcolor}{\sffamily\fontsize{12.000000}{14.400000}\selectfont\catcode`\^=\active\def^{\ifmmode\sp\else\^{}\fi}\catcode`\%=\active\def%{\%}Time (s)}}%
\end{pgfscope}%
\begin{pgfscope}%
\pgfpathrectangle{\pgfqpoint{0.736295in}{2.323635in}}{\pgfqpoint{4.683705in}{0.996365in}}%
\pgfusepath{clip}%
\pgfsetroundcap%
\pgfsetroundjoin%
\pgfsetlinewidth{1.003750pt}%
\definecolor{currentstroke}{rgb}{1.000000,1.000000,1.000000}%
\pgfsetstrokecolor{currentstroke}%
\pgfsetdash{}{0pt}%
\pgfpathmoveto{\pgfqpoint{0.736295in}{2.406666in}}%
\pgfpathlineto{\pgfqpoint{5.420000in}{2.406666in}}%
\pgfusepath{stroke}%
\end{pgfscope}%
\begin{pgfscope}%
\definecolor{textcolor}{rgb}{0.150000,0.150000,0.150000}%
\pgfsetstrokecolor{textcolor}%
\pgfsetfillcolor{textcolor}%
\pgftext[x=0.391968in, y=2.351985in, left, base]{\color{textcolor}{\sffamily\fontsize{11.000000}{13.200000}\selectfont\catcode`\^=\active\def^{\ifmmode\sp\else\^{}\fi}\catcode`\%=\active\def%{\%}0.0}}%
\end{pgfscope}%
\begin{pgfscope}%
\pgfpathrectangle{\pgfqpoint{0.736295in}{2.323635in}}{\pgfqpoint{4.683705in}{0.996365in}}%
\pgfusepath{clip}%
\pgfsetroundcap%
\pgfsetroundjoin%
\pgfsetlinewidth{1.003750pt}%
\definecolor{currentstroke}{rgb}{1.000000,1.000000,1.000000}%
\pgfsetstrokecolor{currentstroke}%
\pgfsetdash{}{0pt}%
\pgfpathmoveto{\pgfqpoint{0.736295in}{2.821818in}}%
\pgfpathlineto{\pgfqpoint{5.420000in}{2.821818in}}%
\pgfusepath{stroke}%
\end{pgfscope}%
\begin{pgfscope}%
\definecolor{textcolor}{rgb}{0.150000,0.150000,0.150000}%
\pgfsetstrokecolor{textcolor}%
\pgfsetfillcolor{textcolor}%
\pgftext[x=0.391968in, y=2.767137in, left, base]{\color{textcolor}{\sffamily\fontsize{11.000000}{13.200000}\selectfont\catcode`\^=\active\def^{\ifmmode\sp\else\^{}\fi}\catcode`\%=\active\def%{\%}0.5}}%
\end{pgfscope}%
\begin{pgfscope}%
\pgfpathrectangle{\pgfqpoint{0.736295in}{2.323635in}}{\pgfqpoint{4.683705in}{0.996365in}}%
\pgfusepath{clip}%
\pgfsetroundcap%
\pgfsetroundjoin%
\pgfsetlinewidth{1.003750pt}%
\definecolor{currentstroke}{rgb}{1.000000,1.000000,1.000000}%
\pgfsetstrokecolor{currentstroke}%
\pgfsetdash{}{0pt}%
\pgfpathmoveto{\pgfqpoint{0.736295in}{3.236970in}}%
\pgfpathlineto{\pgfqpoint{5.420000in}{3.236970in}}%
\pgfusepath{stroke}%
\end{pgfscope}%
\begin{pgfscope}%
\definecolor{textcolor}{rgb}{0.150000,0.150000,0.150000}%
\pgfsetstrokecolor{textcolor}%
\pgfsetfillcolor{textcolor}%
\pgftext[x=0.391968in, y=3.182289in, left, base]{\color{textcolor}{\sffamily\fontsize{11.000000}{13.200000}\selectfont\catcode`\^=\active\def^{\ifmmode\sp\else\^{}\fi}\catcode`\%=\active\def%{\%}1.0}}%
\end{pgfscope}%
\begin{pgfscope}%
\definecolor{textcolor}{rgb}{0.150000,0.150000,0.150000}%
\pgfsetstrokecolor{textcolor}%
\pgfsetfillcolor{textcolor}%
\pgftext[x=0.336413in,y=2.821818in,,bottom,rotate=90.000000]{\color{textcolor}{\sffamily\fontsize{12.000000}{14.400000}\selectfont\catcode`\^=\active\def^{\ifmmode\sp\else\^{}\fi}\catcode`\%=\active\def%{\%}CPU Utilization (%)}}%
\end{pgfscope}%
\begin{pgfscope}%
\pgfpathrectangle{\pgfqpoint{0.736295in}{2.323635in}}{\pgfqpoint{4.683705in}{0.996365in}}%
\pgfusepath{clip}%
\pgfsetroundcap%
\pgfsetroundjoin%
\pgfsetlinewidth{1.505625pt}%
\definecolor{currentstroke}{rgb}{0.298039,0.447059,0.690196}%
\pgfsetstrokecolor{currentstroke}%
\pgfsetdash{}{0pt}%
\pgfpathmoveto{\pgfqpoint{0.949190in}{2.461330in}}%
\pgfpathlineto{\pgfqpoint{1.144345in}{2.460406in}}%
\pgfpathlineto{\pgfqpoint{1.179827in}{2.459655in}}%
\pgfpathlineto{\pgfqpoint{1.428206in}{2.459341in}}%
\pgfpathlineto{\pgfqpoint{1.445947in}{2.457568in}}%
\pgfpathlineto{\pgfqpoint{1.534653in}{2.457568in}}%
\pgfpathlineto{\pgfqpoint{1.552395in}{2.454686in}}%
\pgfpathlineto{\pgfqpoint{1.605619in}{2.454686in}}%
\pgfpathlineto{\pgfqpoint{1.623360in}{2.453504in}}%
\pgfpathlineto{\pgfqpoint{1.694325in}{2.453434in}}%
\pgfpathlineto{\pgfqpoint{1.712067in}{2.446276in}}%
\pgfpathlineto{\pgfqpoint{1.836256in}{2.447643in}}%
\pgfpathlineto{\pgfqpoint{1.853997in}{2.462667in}}%
\pgfpathlineto{\pgfqpoint{1.907221in}{2.462685in}}%
\pgfpathlineto{\pgfqpoint{1.924962in}{2.464368in}}%
\pgfpathlineto{\pgfqpoint{1.978186in}{2.464521in}}%
\pgfpathlineto{\pgfqpoint{1.995927in}{2.462376in}}%
\pgfpathlineto{\pgfqpoint{2.013669in}{2.462451in}}%
\pgfpathlineto{\pgfqpoint{2.031410in}{2.461045in}}%
\pgfpathlineto{\pgfqpoint{2.084634in}{2.461241in}}%
\pgfpathlineto{\pgfqpoint{2.102375in}{2.459602in}}%
\pgfpathlineto{\pgfqpoint{2.191082in}{2.459005in}}%
\pgfpathlineto{\pgfqpoint{2.244306in}{2.457727in}}%
\pgfpathlineto{\pgfqpoint{2.297530in}{2.458042in}}%
\pgfpathlineto{\pgfqpoint{2.350754in}{2.456965in}}%
\pgfpathlineto{\pgfqpoint{2.403978in}{2.457254in}}%
\pgfpathlineto{\pgfqpoint{2.457201in}{2.456644in}}%
\pgfpathlineto{\pgfqpoint{2.492684in}{2.456885in}}%
\pgfpathlineto{\pgfqpoint{2.545908in}{2.455912in}}%
\pgfpathlineto{\pgfqpoint{2.634615in}{2.454941in}}%
\pgfpathlineto{\pgfqpoint{2.652356in}{2.453077in}}%
\pgfpathlineto{\pgfqpoint{2.670097in}{2.453507in}}%
\pgfpathlineto{\pgfqpoint{2.687839in}{2.451254in}}%
\pgfpathlineto{\pgfqpoint{2.723321in}{2.451254in}}%
\pgfpathlineto{\pgfqpoint{2.741062in}{2.453592in}}%
\pgfpathlineto{\pgfqpoint{2.829769in}{2.453293in}}%
\pgfpathlineto{\pgfqpoint{3.202336in}{2.452847in}}%
\pgfpathlineto{\pgfqpoint{3.220078in}{2.454610in}}%
\pgfpathlineto{\pgfqpoint{3.273302in}{2.454398in}}%
\pgfpathlineto{\pgfqpoint{3.291043in}{2.452926in}}%
\pgfpathlineto{\pgfqpoint{3.308784in}{2.452827in}}%
\pgfpathlineto{\pgfqpoint{3.326526in}{2.459274in}}%
\pgfpathlineto{\pgfqpoint{3.379750in}{2.458990in}}%
\pgfpathlineto{\pgfqpoint{3.397491in}{2.456905in}}%
\pgfpathlineto{\pgfqpoint{3.787800in}{2.455749in}}%
\pgfpathlineto{\pgfqpoint{3.805541in}{2.458019in}}%
\pgfpathlineto{\pgfqpoint{3.858765in}{2.458208in}}%
\pgfpathlineto{\pgfqpoint{3.876506in}{2.456422in}}%
\pgfpathlineto{\pgfqpoint{4.781313in}{2.455947in}}%
\pgfpathlineto{\pgfqpoint{4.799054in}{2.454030in}}%
\pgfpathlineto{\pgfqpoint{4.852278in}{2.454030in}}%
\pgfpathlineto{\pgfqpoint{4.870019in}{2.456044in}}%
\pgfpathlineto{\pgfqpoint{5.207104in}{2.455924in}}%
\pgfpathlineto{\pgfqpoint{5.207104in}{2.455924in}}%
\pgfusepath{stroke}%
\end{pgfscope}%
\begin{pgfscope}%
\pgfpathrectangle{\pgfqpoint{0.736295in}{2.323635in}}{\pgfqpoint{4.683705in}{0.996365in}}%
\pgfusepath{clip}%
\pgfsetbuttcap%
\pgfsetroundjoin%
\pgfsetlinewidth{1.505625pt}%
\definecolor{currentstroke}{rgb}{1.000000,0.647059,0.000000}%
\pgfsetstrokecolor{currentstroke}%
\pgfsetdash{{5.550000pt}{2.400000pt}}{0.000000pt}%
\pgfpathmoveto{\pgfqpoint{1.026542in}{2.323635in}}%
\pgfpathlineto{\pgfqpoint{1.026542in}{3.320000in}}%
\pgfusepath{stroke}%
\end{pgfscope}%
\begin{pgfscope}%
\pgfsetrectcap%
\pgfsetmiterjoin%
\pgfsetlinewidth{1.254687pt}%
\definecolor{currentstroke}{rgb}{1.000000,1.000000,1.000000}%
\pgfsetstrokecolor{currentstroke}%
\pgfsetdash{}{0pt}%
\pgfpathmoveto{\pgfqpoint{0.736295in}{2.323635in}}%
\pgfpathlineto{\pgfqpoint{0.736295in}{3.320000in}}%
\pgfusepath{stroke}%
\end{pgfscope}%
\begin{pgfscope}%
\pgfsetrectcap%
\pgfsetmiterjoin%
\pgfsetlinewidth{1.254687pt}%
\definecolor{currentstroke}{rgb}{1.000000,1.000000,1.000000}%
\pgfsetstrokecolor{currentstroke}%
\pgfsetdash{}{0pt}%
\pgfpathmoveto{\pgfqpoint{5.420000in}{2.323635in}}%
\pgfpathlineto{\pgfqpoint{5.420000in}{3.320000in}}%
\pgfusepath{stroke}%
\end{pgfscope}%
\begin{pgfscope}%
\pgfsetrectcap%
\pgfsetmiterjoin%
\pgfsetlinewidth{1.254687pt}%
\definecolor{currentstroke}{rgb}{1.000000,1.000000,1.000000}%
\pgfsetstrokecolor{currentstroke}%
\pgfsetdash{}{0pt}%
\pgfpathmoveto{\pgfqpoint{0.736295in}{2.323635in}}%
\pgfpathlineto{\pgfqpoint{5.420000in}{2.323635in}}%
\pgfusepath{stroke}%
\end{pgfscope}%
\begin{pgfscope}%
\pgfsetrectcap%
\pgfsetmiterjoin%
\pgfsetlinewidth{1.254687pt}%
\definecolor{currentstroke}{rgb}{1.000000,1.000000,1.000000}%
\pgfsetstrokecolor{currentstroke}%
\pgfsetdash{}{0pt}%
\pgfpathmoveto{\pgfqpoint{0.736295in}{3.320000in}}%
\pgfpathlineto{\pgfqpoint{5.420000in}{3.320000in}}%
\pgfusepath{stroke}%
\end{pgfscope}%
\begin{pgfscope}%
\pgfsetbuttcap%
\pgfsetmiterjoin%
\definecolor{currentfill}{rgb}{0.917647,0.917647,0.949020}%
\pgfsetfillcolor{currentfill}%
\pgfsetfillopacity{0.800000}%
\pgfsetlinewidth{1.003750pt}%
\definecolor{currentstroke}{rgb}{0.800000,0.800000,0.800000}%
\pgfsetstrokecolor{currentstroke}%
\pgfsetstrokeopacity{0.800000}%
\pgfsetdash{}{0pt}%
\pgfpathmoveto{\pgfqpoint{3.276163in}{3.071281in}}%
\pgfpathlineto{\pgfqpoint{5.342222in}{3.071281in}}%
\pgfpathquadraticcurveto{\pgfqpoint{5.364444in}{3.071281in}}{\pgfqpoint{5.364444in}{3.093503in}}%
\pgfpathlineto{\pgfqpoint{5.364444in}{3.242222in}}%
\pgfpathquadraticcurveto{\pgfqpoint{5.364444in}{3.264444in}}{\pgfqpoint{5.342222in}{3.264444in}}%
\pgfpathlineto{\pgfqpoint{3.276163in}{3.264444in}}%
\pgfpathquadraticcurveto{\pgfqpoint{3.253941in}{3.264444in}}{\pgfqpoint{3.253941in}{3.242222in}}%
\pgfpathlineto{\pgfqpoint{3.253941in}{3.093503in}}%
\pgfpathquadraticcurveto{\pgfqpoint{3.253941in}{3.071281in}}{\pgfqpoint{3.276163in}{3.071281in}}%
\pgfpathlineto{\pgfqpoint{3.276163in}{3.071281in}}%
\pgfpathclose%
\pgfusepath{stroke,fill}%
\end{pgfscope}%
\begin{pgfscope}%
\pgfsetbuttcap%
\pgfsetroundjoin%
\pgfsetlinewidth{1.505625pt}%
\definecolor{currentstroke}{rgb}{1.000000,0.647059,0.000000}%
\pgfsetstrokecolor{currentstroke}%
\pgfsetdash{{5.550000pt}{2.400000pt}}{0.000000pt}%
\pgfpathmoveto{\pgfqpoint{3.298385in}{3.177997in}}%
\pgfpathlineto{\pgfqpoint{3.409497in}{3.177997in}}%
\pgfpathlineto{\pgfqpoint{3.520608in}{3.177997in}}%
\pgfusepath{stroke}%
\end{pgfscope}%
\begin{pgfscope}%
\definecolor{textcolor}{rgb}{0.150000,0.150000,0.150000}%
\pgfsetstrokecolor{textcolor}%
\pgfsetfillcolor{textcolor}%
\pgftext[x=3.609497in,y=3.139108in,left,base]{\color{textcolor}{\sffamily\fontsize{8.000000}{9.600000}\selectfont\catcode`\^=\active\def^{\ifmmode\sp\else\^{}\fi}\catcode`\%=\active\def%{\%}start of elasticity strategy controller}}%
\end{pgfscope}%
\begin{pgfscope}%
\pgfsetbuttcap%
\pgfsetmiterjoin%
\definecolor{currentfill}{rgb}{0.917647,0.917647,0.949020}%
\pgfsetfillcolor{currentfill}%
\pgfsetlinewidth{0.000000pt}%
\definecolor{currentstroke}{rgb}{0.000000,0.000000,0.000000}%
\pgfsetstrokecolor{currentstroke}%
\pgfsetstrokeopacity{0.000000}%
\pgfsetdash{}{0pt}%
\pgfpathmoveto{\pgfqpoint{0.736295in}{0.663635in}}%
\pgfpathlineto{\pgfqpoint{5.420000in}{0.663635in}}%
\pgfpathlineto{\pgfqpoint{5.420000in}{1.660000in}}%
\pgfpathlineto{\pgfqpoint{0.736295in}{1.660000in}}%
\pgfpathlineto{\pgfqpoint{0.736295in}{0.663635in}}%
\pgfpathclose%
\pgfusepath{fill}%
\end{pgfscope}%
\begin{pgfscope}%
\pgfpathrectangle{\pgfqpoint{0.736295in}{0.663635in}}{\pgfqpoint{4.683705in}{0.996365in}}%
\pgfusepath{clip}%
\pgfsetroundcap%
\pgfsetroundjoin%
\pgfsetlinewidth{1.003750pt}%
\definecolor{currentstroke}{rgb}{1.000000,1.000000,1.000000}%
\pgfsetstrokecolor{currentstroke}%
\pgfsetdash{}{0pt}%
\pgfpathmoveto{\pgfqpoint{0.949190in}{0.663635in}}%
\pgfpathlineto{\pgfqpoint{0.949190in}{1.660000in}}%
\pgfusepath{stroke}%
\end{pgfscope}%
\begin{pgfscope}%
\definecolor{textcolor}{rgb}{0.150000,0.150000,0.150000}%
\pgfsetstrokecolor{textcolor}%
\pgfsetfillcolor{textcolor}%
\pgftext[x=0.949190in,y=0.531691in,,top]{\color{textcolor}{\sffamily\fontsize{11.000000}{13.200000}\selectfont\catcode`\^=\active\def^{\ifmmode\sp\else\^{}\fi}\catcode`\%=\active\def%{\%}0}}%
\end{pgfscope}%
\begin{pgfscope}%
\pgfpathrectangle{\pgfqpoint{0.736295in}{0.663635in}}{\pgfqpoint{4.683705in}{0.996365in}}%
\pgfusepath{clip}%
\pgfsetroundcap%
\pgfsetroundjoin%
\pgfsetlinewidth{1.003750pt}%
\definecolor{currentstroke}{rgb}{1.000000,1.000000,1.000000}%
\pgfsetstrokecolor{currentstroke}%
\pgfsetdash{}{0pt}%
\pgfpathmoveto{\pgfqpoint{1.658843in}{0.663635in}}%
\pgfpathlineto{\pgfqpoint{1.658843in}{1.660000in}}%
\pgfusepath{stroke}%
\end{pgfscope}%
\begin{pgfscope}%
\definecolor{textcolor}{rgb}{0.150000,0.150000,0.150000}%
\pgfsetstrokecolor{textcolor}%
\pgfsetfillcolor{textcolor}%
\pgftext[x=1.658843in,y=0.531691in,,top]{\color{textcolor}{\sffamily\fontsize{11.000000}{13.200000}\selectfont\catcode`\^=\active\def^{\ifmmode\sp\else\^{}\fi}\catcode`\%=\active\def%{\%}200}}%
\end{pgfscope}%
\begin{pgfscope}%
\pgfpathrectangle{\pgfqpoint{0.736295in}{0.663635in}}{\pgfqpoint{4.683705in}{0.996365in}}%
\pgfusepath{clip}%
\pgfsetroundcap%
\pgfsetroundjoin%
\pgfsetlinewidth{1.003750pt}%
\definecolor{currentstroke}{rgb}{1.000000,1.000000,1.000000}%
\pgfsetstrokecolor{currentstroke}%
\pgfsetdash{}{0pt}%
\pgfpathmoveto{\pgfqpoint{2.368495in}{0.663635in}}%
\pgfpathlineto{\pgfqpoint{2.368495in}{1.660000in}}%
\pgfusepath{stroke}%
\end{pgfscope}%
\begin{pgfscope}%
\definecolor{textcolor}{rgb}{0.150000,0.150000,0.150000}%
\pgfsetstrokecolor{textcolor}%
\pgfsetfillcolor{textcolor}%
\pgftext[x=2.368495in,y=0.531691in,,top]{\color{textcolor}{\sffamily\fontsize{11.000000}{13.200000}\selectfont\catcode`\^=\active\def^{\ifmmode\sp\else\^{}\fi}\catcode`\%=\active\def%{\%}400}}%
\end{pgfscope}%
\begin{pgfscope}%
\pgfpathrectangle{\pgfqpoint{0.736295in}{0.663635in}}{\pgfqpoint{4.683705in}{0.996365in}}%
\pgfusepath{clip}%
\pgfsetroundcap%
\pgfsetroundjoin%
\pgfsetlinewidth{1.003750pt}%
\definecolor{currentstroke}{rgb}{1.000000,1.000000,1.000000}%
\pgfsetstrokecolor{currentstroke}%
\pgfsetdash{}{0pt}%
\pgfpathmoveto{\pgfqpoint{3.078147in}{0.663635in}}%
\pgfpathlineto{\pgfqpoint{3.078147in}{1.660000in}}%
\pgfusepath{stroke}%
\end{pgfscope}%
\begin{pgfscope}%
\definecolor{textcolor}{rgb}{0.150000,0.150000,0.150000}%
\pgfsetstrokecolor{textcolor}%
\pgfsetfillcolor{textcolor}%
\pgftext[x=3.078147in,y=0.531691in,,top]{\color{textcolor}{\sffamily\fontsize{11.000000}{13.200000}\selectfont\catcode`\^=\active\def^{\ifmmode\sp\else\^{}\fi}\catcode`\%=\active\def%{\%}600}}%
\end{pgfscope}%
\begin{pgfscope}%
\pgfpathrectangle{\pgfqpoint{0.736295in}{0.663635in}}{\pgfqpoint{4.683705in}{0.996365in}}%
\pgfusepath{clip}%
\pgfsetroundcap%
\pgfsetroundjoin%
\pgfsetlinewidth{1.003750pt}%
\definecolor{currentstroke}{rgb}{1.000000,1.000000,1.000000}%
\pgfsetstrokecolor{currentstroke}%
\pgfsetdash{}{0pt}%
\pgfpathmoveto{\pgfqpoint{3.787800in}{0.663635in}}%
\pgfpathlineto{\pgfqpoint{3.787800in}{1.660000in}}%
\pgfusepath{stroke}%
\end{pgfscope}%
\begin{pgfscope}%
\definecolor{textcolor}{rgb}{0.150000,0.150000,0.150000}%
\pgfsetstrokecolor{textcolor}%
\pgfsetfillcolor{textcolor}%
\pgftext[x=3.787800in,y=0.531691in,,top]{\color{textcolor}{\sffamily\fontsize{11.000000}{13.200000}\selectfont\catcode`\^=\active\def^{\ifmmode\sp\else\^{}\fi}\catcode`\%=\active\def%{\%}800}}%
\end{pgfscope}%
\begin{pgfscope}%
\pgfpathrectangle{\pgfqpoint{0.736295in}{0.663635in}}{\pgfqpoint{4.683705in}{0.996365in}}%
\pgfusepath{clip}%
\pgfsetroundcap%
\pgfsetroundjoin%
\pgfsetlinewidth{1.003750pt}%
\definecolor{currentstroke}{rgb}{1.000000,1.000000,1.000000}%
\pgfsetstrokecolor{currentstroke}%
\pgfsetdash{}{0pt}%
\pgfpathmoveto{\pgfqpoint{4.497452in}{0.663635in}}%
\pgfpathlineto{\pgfqpoint{4.497452in}{1.660000in}}%
\pgfusepath{stroke}%
\end{pgfscope}%
\begin{pgfscope}%
\definecolor{textcolor}{rgb}{0.150000,0.150000,0.150000}%
\pgfsetstrokecolor{textcolor}%
\pgfsetfillcolor{textcolor}%
\pgftext[x=4.497452in,y=0.531691in,,top]{\color{textcolor}{\sffamily\fontsize{11.000000}{13.200000}\selectfont\catcode`\^=\active\def^{\ifmmode\sp\else\^{}\fi}\catcode`\%=\active\def%{\%}1000}}%
\end{pgfscope}%
\begin{pgfscope}%
\pgfpathrectangle{\pgfqpoint{0.736295in}{0.663635in}}{\pgfqpoint{4.683705in}{0.996365in}}%
\pgfusepath{clip}%
\pgfsetroundcap%
\pgfsetroundjoin%
\pgfsetlinewidth{1.003750pt}%
\definecolor{currentstroke}{rgb}{1.000000,1.000000,1.000000}%
\pgfsetstrokecolor{currentstroke}%
\pgfsetdash{}{0pt}%
\pgfpathmoveto{\pgfqpoint{5.207104in}{0.663635in}}%
\pgfpathlineto{\pgfqpoint{5.207104in}{1.660000in}}%
\pgfusepath{stroke}%
\end{pgfscope}%
\begin{pgfscope}%
\definecolor{textcolor}{rgb}{0.150000,0.150000,0.150000}%
\pgfsetstrokecolor{textcolor}%
\pgfsetfillcolor{textcolor}%
\pgftext[x=5.207104in,y=0.531691in,,top]{\color{textcolor}{\sffamily\fontsize{11.000000}{13.200000}\selectfont\catcode`\^=\active\def^{\ifmmode\sp\else\^{}\fi}\catcode`\%=\active\def%{\%}1200}}%
\end{pgfscope}%
\begin{pgfscope}%
\definecolor{textcolor}{rgb}{0.150000,0.150000,0.150000}%
\pgfsetstrokecolor{textcolor}%
\pgfsetfillcolor{textcolor}%
\pgftext[x=3.078147in,y=0.336413in,,top]{\color{textcolor}{\sffamily\fontsize{12.000000}{14.400000}\selectfont\catcode`\^=\active\def^{\ifmmode\sp\else\^{}\fi}\catcode`\%=\active\def%{\%}Time (s)}}%
\end{pgfscope}%
\begin{pgfscope}%
\pgfpathrectangle{\pgfqpoint{0.736295in}{0.663635in}}{\pgfqpoint{4.683705in}{0.996365in}}%
\pgfusepath{clip}%
\pgfsetroundcap%
\pgfsetroundjoin%
\pgfsetlinewidth{1.003750pt}%
\definecolor{currentstroke}{rgb}{1.000000,1.000000,1.000000}%
\pgfsetstrokecolor{currentstroke}%
\pgfsetdash{}{0pt}%
\pgfpathmoveto{\pgfqpoint{0.736295in}{0.746666in}}%
\pgfpathlineto{\pgfqpoint{5.420000in}{0.746666in}}%
\pgfusepath{stroke}%
\end{pgfscope}%
\begin{pgfscope}%
\definecolor{textcolor}{rgb}{0.150000,0.150000,0.150000}%
\pgfsetstrokecolor{textcolor}%
\pgfsetfillcolor{textcolor}%
\pgftext[x=0.391968in, y=0.691985in, left, base]{\color{textcolor}{\sffamily\fontsize{11.000000}{13.200000}\selectfont\catcode`\^=\active\def^{\ifmmode\sp\else\^{}\fi}\catcode`\%=\active\def%{\%}0.0}}%
\end{pgfscope}%
\begin{pgfscope}%
\pgfpathrectangle{\pgfqpoint{0.736295in}{0.663635in}}{\pgfqpoint{4.683705in}{0.996365in}}%
\pgfusepath{clip}%
\pgfsetroundcap%
\pgfsetroundjoin%
\pgfsetlinewidth{1.003750pt}%
\definecolor{currentstroke}{rgb}{1.000000,1.000000,1.000000}%
\pgfsetstrokecolor{currentstroke}%
\pgfsetdash{}{0pt}%
\pgfpathmoveto{\pgfqpoint{0.736295in}{1.161818in}}%
\pgfpathlineto{\pgfqpoint{5.420000in}{1.161818in}}%
\pgfusepath{stroke}%
\end{pgfscope}%
\begin{pgfscope}%
\definecolor{textcolor}{rgb}{0.150000,0.150000,0.150000}%
\pgfsetstrokecolor{textcolor}%
\pgfsetfillcolor{textcolor}%
\pgftext[x=0.391968in, y=1.107137in, left, base]{\color{textcolor}{\sffamily\fontsize{11.000000}{13.200000}\selectfont\catcode`\^=\active\def^{\ifmmode\sp\else\^{}\fi}\catcode`\%=\active\def%{\%}0.5}}%
\end{pgfscope}%
\begin{pgfscope}%
\pgfpathrectangle{\pgfqpoint{0.736295in}{0.663635in}}{\pgfqpoint{4.683705in}{0.996365in}}%
\pgfusepath{clip}%
\pgfsetroundcap%
\pgfsetroundjoin%
\pgfsetlinewidth{1.003750pt}%
\definecolor{currentstroke}{rgb}{1.000000,1.000000,1.000000}%
\pgfsetstrokecolor{currentstroke}%
\pgfsetdash{}{0pt}%
\pgfpathmoveto{\pgfqpoint{0.736295in}{1.576970in}}%
\pgfpathlineto{\pgfqpoint{5.420000in}{1.576970in}}%
\pgfusepath{stroke}%
\end{pgfscope}%
\begin{pgfscope}%
\definecolor{textcolor}{rgb}{0.150000,0.150000,0.150000}%
\pgfsetstrokecolor{textcolor}%
\pgfsetfillcolor{textcolor}%
\pgftext[x=0.391968in, y=1.522289in, left, base]{\color{textcolor}{\sffamily\fontsize{11.000000}{13.200000}\selectfont\catcode`\^=\active\def^{\ifmmode\sp\else\^{}\fi}\catcode`\%=\active\def%{\%}1.0}}%
\end{pgfscope}%
\begin{pgfscope}%
\definecolor{textcolor}{rgb}{0.150000,0.150000,0.150000}%
\pgfsetstrokecolor{textcolor}%
\pgfsetfillcolor{textcolor}%
\pgftext[x=0.336413in,y=1.161818in,,bottom,rotate=90.000000]{\color{textcolor}{\sffamily\fontsize{12.000000}{14.400000}\selectfont\catcode`\^=\active\def^{\ifmmode\sp\else\^{}\fi}\catcode`\%=\active\def%{\%}Memory Utilization (%)}}%
\end{pgfscope}%
\begin{pgfscope}%
\pgfpathrectangle{\pgfqpoint{0.736295in}{0.663635in}}{\pgfqpoint{4.683705in}{0.996365in}}%
\pgfusepath{clip}%
\pgfsetbuttcap%
\pgfsetmiterjoin%
\definecolor{currentfill}{rgb}{1.000000,0.000000,0.000000}%
\pgfsetfillcolor{currentfill}%
\pgfsetfillopacity{0.300000}%
\pgfsetlinewidth{1.003750pt}%
\definecolor{currentstroke}{rgb}{1.000000,0.000000,0.000000}%
\pgfsetstrokecolor{currentstroke}%
\pgfsetstrokeopacity{0.300000}%
\pgfsetdash{}{0pt}%
\pgfpathmoveto{\pgfqpoint{1.367885in}{0.663635in}}%
\pgfpathlineto{\pgfqpoint{1.367885in}{1.660000in}}%
\pgfpathlineto{\pgfqpoint{2.411074in}{1.660000in}}%
\pgfpathlineto{\pgfqpoint{2.411074in}{0.663635in}}%
\pgfpathlineto{\pgfqpoint{1.367885in}{0.663635in}}%
\pgfpathclose%
\pgfusepath{stroke,fill}%
\end{pgfscope}%
\begin{pgfscope}%
\pgfpathrectangle{\pgfqpoint{0.736295in}{0.663635in}}{\pgfqpoint{4.683705in}{0.996365in}}%
\pgfusepath{clip}%
\pgfsetroundcap%
\pgfsetroundjoin%
\pgfsetlinewidth{1.505625pt}%
\definecolor{currentstroke}{rgb}{0.298039,0.447059,0.690196}%
\pgfsetstrokecolor{currentstroke}%
\pgfsetdash{}{0pt}%
\pgfpathmoveto{\pgfqpoint{0.949190in}{1.175322in}}%
\pgfpathlineto{\pgfqpoint{1.179827in}{1.175111in}}%
\pgfpathlineto{\pgfqpoint{1.197569in}{1.177375in}}%
\pgfpathlineto{\pgfqpoint{1.374982in}{1.178370in}}%
\pgfpathlineto{\pgfqpoint{1.392723in}{1.418450in}}%
\pgfpathlineto{\pgfqpoint{1.445947in}{1.419151in}}%
\pgfpathlineto{\pgfqpoint{1.463688in}{1.576970in}}%
\pgfpathlineto{\pgfqpoint{2.386236in}{1.576970in}}%
\pgfpathlineto{\pgfqpoint{2.403978in}{1.402559in}}%
\pgfpathlineto{\pgfqpoint{2.457201in}{1.402035in}}%
\pgfpathlineto{\pgfqpoint{2.474943in}{1.400621in}}%
\pgfpathlineto{\pgfqpoint{2.510425in}{1.400621in}}%
\pgfpathlineto{\pgfqpoint{2.528167in}{1.404946in}}%
\pgfpathlineto{\pgfqpoint{2.563649in}{1.404946in}}%
\pgfpathlineto{\pgfqpoint{2.581391in}{1.401377in}}%
\pgfpathlineto{\pgfqpoint{2.953958in}{1.400924in}}%
\pgfpathlineto{\pgfqpoint{2.971699in}{1.402976in}}%
\pgfpathlineto{\pgfqpoint{3.024923in}{1.402976in}}%
\pgfpathlineto{\pgfqpoint{3.042665in}{1.400604in}}%
\pgfpathlineto{\pgfqpoint{3.078147in}{1.400604in}}%
\pgfpathlineto{\pgfqpoint{3.095889in}{1.404172in}}%
\pgfpathlineto{\pgfqpoint{3.131371in}{1.404172in}}%
\pgfpathlineto{\pgfqpoint{3.149113in}{1.402557in}}%
\pgfpathlineto{\pgfqpoint{3.184595in}{1.402557in}}%
\pgfpathlineto{\pgfqpoint{3.202336in}{1.400815in}}%
\pgfpathlineto{\pgfqpoint{3.397491in}{1.401466in}}%
\pgfpathlineto{\pgfqpoint{3.415232in}{1.402735in}}%
\pgfpathlineto{\pgfqpoint{3.610387in}{1.402549in}}%
\pgfpathlineto{\pgfqpoint{3.628128in}{1.405441in}}%
\pgfpathlineto{\pgfqpoint{3.787800in}{1.405651in}}%
\pgfpathlineto{\pgfqpoint{3.805541in}{1.404133in}}%
\pgfpathlineto{\pgfqpoint{4.089402in}{1.404731in}}%
\pgfpathlineto{\pgfqpoint{4.107143in}{1.401121in}}%
\pgfpathlineto{\pgfqpoint{4.213591in}{1.401202in}}%
\pgfpathlineto{\pgfqpoint{4.231332in}{1.402726in}}%
\pgfpathlineto{\pgfqpoint{4.266815in}{1.402726in}}%
\pgfpathlineto{\pgfqpoint{4.284556in}{1.404362in}}%
\pgfpathlineto{\pgfqpoint{4.568417in}{1.405176in}}%
\pgfpathlineto{\pgfqpoint{4.586159in}{1.401940in}}%
\pgfpathlineto{\pgfqpoint{4.692606in}{1.401941in}}%
\pgfpathlineto{\pgfqpoint{4.710348in}{1.407260in}}%
\pgfpathlineto{\pgfqpoint{4.745830in}{1.407261in}}%
\pgfpathlineto{\pgfqpoint{4.763572in}{1.404999in}}%
\pgfpathlineto{\pgfqpoint{4.816796in}{1.404999in}}%
\pgfpathlineto{\pgfqpoint{4.834537in}{1.402270in}}%
\pgfpathlineto{\pgfqpoint{4.976467in}{1.402540in}}%
\pgfpathlineto{\pgfqpoint{4.994209in}{1.404280in}}%
\pgfpathlineto{\pgfqpoint{5.207104in}{1.403701in}}%
\pgfpathlineto{\pgfqpoint{5.207104in}{1.403701in}}%
\pgfusepath{stroke}%
\end{pgfscope}%
\begin{pgfscope}%
\pgfpathrectangle{\pgfqpoint{0.736295in}{0.663635in}}{\pgfqpoint{4.683705in}{0.996365in}}%
\pgfusepath{clip}%
\pgfsetbuttcap%
\pgfsetroundjoin%
\pgfsetlinewidth{1.505625pt}%
\definecolor{currentstroke}{rgb}{0.172549,0.627451,0.172549}%
\pgfsetstrokecolor{currentstroke}%
\pgfsetdash{{5.550000pt}{2.400000pt}}{0.000000pt}%
\pgfpathmoveto{\pgfqpoint{0.736295in}{1.327878in}}%
\pgfpathlineto{\pgfqpoint{5.420000in}{1.327878in}}%
\pgfusepath{stroke}%
\end{pgfscope}%
\begin{pgfscope}%
\pgfpathrectangle{\pgfqpoint{0.736295in}{0.663635in}}{\pgfqpoint{4.683705in}{0.996365in}}%
\pgfusepath{clip}%
\pgfsetbuttcap%
\pgfsetroundjoin%
\pgfsetlinewidth{1.505625pt}%
\definecolor{currentstroke}{rgb}{1.000000,0.647059,0.000000}%
\pgfsetstrokecolor{currentstroke}%
\pgfsetdash{{5.550000pt}{2.400000pt}}{0.000000pt}%
\pgfpathmoveto{\pgfqpoint{1.026542in}{0.663635in}}%
\pgfpathlineto{\pgfqpoint{1.026542in}{1.660000in}}%
\pgfusepath{stroke}%
\end{pgfscope}%
\begin{pgfscope}%
\pgfsetrectcap%
\pgfsetmiterjoin%
\pgfsetlinewidth{1.254687pt}%
\definecolor{currentstroke}{rgb}{1.000000,1.000000,1.000000}%
\pgfsetstrokecolor{currentstroke}%
\pgfsetdash{}{0pt}%
\pgfpathmoveto{\pgfqpoint{0.736295in}{0.663635in}}%
\pgfpathlineto{\pgfqpoint{0.736295in}{1.660000in}}%
\pgfusepath{stroke}%
\end{pgfscope}%
\begin{pgfscope}%
\pgfsetrectcap%
\pgfsetmiterjoin%
\pgfsetlinewidth{1.254687pt}%
\definecolor{currentstroke}{rgb}{1.000000,1.000000,1.000000}%
\pgfsetstrokecolor{currentstroke}%
\pgfsetdash{}{0pt}%
\pgfpathmoveto{\pgfqpoint{5.420000in}{0.663635in}}%
\pgfpathlineto{\pgfqpoint{5.420000in}{1.660000in}}%
\pgfusepath{stroke}%
\end{pgfscope}%
\begin{pgfscope}%
\pgfsetrectcap%
\pgfsetmiterjoin%
\pgfsetlinewidth{1.254687pt}%
\definecolor{currentstroke}{rgb}{1.000000,1.000000,1.000000}%
\pgfsetstrokecolor{currentstroke}%
\pgfsetdash{}{0pt}%
\pgfpathmoveto{\pgfqpoint{0.736295in}{0.663635in}}%
\pgfpathlineto{\pgfqpoint{5.420000in}{0.663635in}}%
\pgfusepath{stroke}%
\end{pgfscope}%
\begin{pgfscope}%
\pgfsetrectcap%
\pgfsetmiterjoin%
\pgfsetlinewidth{1.254687pt}%
\definecolor{currentstroke}{rgb}{1.000000,1.000000,1.000000}%
\pgfsetstrokecolor{currentstroke}%
\pgfsetdash{}{0pt}%
\pgfpathmoveto{\pgfqpoint{0.736295in}{1.660000in}}%
\pgfpathlineto{\pgfqpoint{5.420000in}{1.660000in}}%
\pgfusepath{stroke}%
\end{pgfscope}%
\begin{pgfscope}%
\pgfsetbuttcap%
\pgfsetmiterjoin%
\definecolor{currentfill}{rgb}{0.917647,0.917647,0.949020}%
\pgfsetfillcolor{currentfill}%
\pgfsetfillopacity{0.800000}%
\pgfsetlinewidth{1.003750pt}%
\definecolor{currentstroke}{rgb}{0.800000,0.800000,0.800000}%
\pgfsetstrokecolor{currentstroke}%
\pgfsetstrokeopacity{0.800000}%
\pgfsetdash{}{0pt}%
\pgfpathmoveto{\pgfqpoint{3.276163in}{0.719191in}}%
\pgfpathlineto{\pgfqpoint{5.342222in}{0.719191in}}%
\pgfpathquadraticcurveto{\pgfqpoint{5.364444in}{0.719191in}}{\pgfqpoint{5.364444in}{0.741413in}}%
\pgfpathlineto{\pgfqpoint{5.364444in}{1.205779in}}%
\pgfpathquadraticcurveto{\pgfqpoint{5.364444in}{1.228001in}}{\pgfqpoint{5.342222in}{1.228001in}}%
\pgfpathlineto{\pgfqpoint{3.276163in}{1.228001in}}%
\pgfpathquadraticcurveto{\pgfqpoint{3.253941in}{1.228001in}}{\pgfqpoint{3.253941in}{1.205779in}}%
\pgfpathlineto{\pgfqpoint{3.253941in}{0.741413in}}%
\pgfpathquadraticcurveto{\pgfqpoint{3.253941in}{0.719191in}}{\pgfqpoint{3.276163in}{0.719191in}}%
\pgfpathlineto{\pgfqpoint{3.276163in}{0.719191in}}%
\pgfpathclose%
\pgfusepath{stroke,fill}%
\end{pgfscope}%
\begin{pgfscope}%
\pgfsetbuttcap%
\pgfsetmiterjoin%
\definecolor{currentfill}{rgb}{1.000000,0.000000,0.000000}%
\pgfsetfillcolor{currentfill}%
\pgfsetfillopacity{0.300000}%
\pgfsetlinewidth{1.003750pt}%
\definecolor{currentstroke}{rgb}{1.000000,0.000000,0.000000}%
\pgfsetstrokecolor{currentstroke}%
\pgfsetstrokeopacity{0.300000}%
\pgfsetdash{}{0pt}%
\pgfpathmoveto{\pgfqpoint{3.298385in}{1.104021in}}%
\pgfpathlineto{\pgfqpoint{3.520608in}{1.104021in}}%
\pgfpathlineto{\pgfqpoint{3.520608in}{1.181799in}}%
\pgfpathlineto{\pgfqpoint{3.298385in}{1.181799in}}%
\pgfpathlineto{\pgfqpoint{3.298385in}{1.104021in}}%
\pgfpathclose%
\pgfusepath{stroke,fill}%
\end{pgfscope}%
\begin{pgfscope}%
\definecolor{textcolor}{rgb}{0.150000,0.150000,0.150000}%
\pgfsetstrokecolor{textcolor}%
\pgfsetfillcolor{textcolor}%
\pgftext[x=3.609497in,y=1.104021in,left,base]{\color{textcolor}{\sffamily\fontsize{8.000000}{9.600000}\selectfont\catcode`\^=\active\def^{\ifmmode\sp\else\^{}\fi}\catcode`\%=\active\def%{\%}k8ssandra reconsiliation}}%
\end{pgfscope}%
\begin{pgfscope}%
\pgfsetbuttcap%
\pgfsetroundjoin%
\pgfsetlinewidth{1.505625pt}%
\definecolor{currentstroke}{rgb}{0.172549,0.627451,0.172549}%
\pgfsetstrokecolor{currentstroke}%
\pgfsetdash{{5.550000pt}{2.400000pt}}{0.000000pt}%
\pgfpathmoveto{\pgfqpoint{3.298385in}{0.985738in}}%
\pgfpathlineto{\pgfqpoint{3.409497in}{0.985738in}}%
\pgfpathlineto{\pgfqpoint{3.520608in}{0.985738in}}%
\pgfusepath{stroke}%
\end{pgfscope}%
\begin{pgfscope}%
\definecolor{textcolor}{rgb}{0.150000,0.150000,0.150000}%
\pgfsetstrokecolor{textcolor}%
\pgfsetfillcolor{textcolor}%
\pgftext[x=3.609497in,y=0.946849in,left,base]{\color{textcolor}{\sffamily\fontsize{8.000000}{9.600000}\selectfont\catcode`\^=\active\def^{\ifmmode\sp\else\^{}\fi}\catcode`\%=\active\def%{\%}target memory utilization}}%
\end{pgfscope}%
\begin{pgfscope}%
\pgfsetbuttcap%
\pgfsetroundjoin%
\pgfsetlinewidth{1.505625pt}%
\definecolor{currentstroke}{rgb}{1.000000,0.647059,0.000000}%
\pgfsetstrokecolor{currentstroke}%
\pgfsetdash{{5.550000pt}{2.400000pt}}{0.000000pt}%
\pgfpathmoveto{\pgfqpoint{3.298385in}{0.825907in}}%
\pgfpathlineto{\pgfqpoint{3.409497in}{0.825907in}}%
\pgfpathlineto{\pgfqpoint{3.520608in}{0.825907in}}%
\pgfusepath{stroke}%
\end{pgfscope}%
\begin{pgfscope}%
\definecolor{textcolor}{rgb}{0.150000,0.150000,0.150000}%
\pgfsetstrokecolor{textcolor}%
\pgfsetfillcolor{textcolor}%
\pgftext[x=3.609497in,y=0.787018in,left,base]{\color{textcolor}{\sffamily\fontsize{8.000000}{9.600000}\selectfont\catcode`\^=\active\def^{\ifmmode\sp\else\^{}\fi}\catcode`\%=\active\def%{\%}start of elasticity strategy controller}}%
\end{pgfscope}%
\end{pgfpicture}%
\makeatother%
\endgroup%

    \caption{Utilization of CPU and memory during an vertical scaling action}
    \label{fig:utilization-vertical}
\end{figure}

\subsection{Horizontal Elasticity Strategy}
\label{sec:evaluation-horizontal-elasticity}

The horizontal elasticity strategy controller scales the target k8ssandra cluster horizontally, thus adding nodes as demand increases. Demand is measured as write throughput by the metrics controller as described in \cref{sec:metrics-average-write-utilization}.

As in the example stress tests discussed in \cref{sec:stress-testing}, \texttt{cassandra-stress} was used to generate load on the target k8ssandra cluster. During this load generation process, the horizontal elasticity controller was running. The target write load per node was defined in the SLO mapping as 5000. Depicted in \cref{fig:horizontal-elasticity} is the average write load per node metric and the corresponding node count during the testing process. It can be seen that the node count does not increase immediatly when the scaling action takes place. That is because when the k8ssandra CRD is updated by the elasticity strategy controller, first the \texttt{k8ssandra-operator} has to recognize the made changes and adjust the configuration accordingly. When the second k8ssandra node is successfully scheduled it still needs time to start and finally register in the cluster. The final action is the Cassandra reconciliation process.

At approximately 290s a sudden drop in the metric can be observed. This is the point when the scaling action becomes effective and the k8ssandra node is ready. Then, after another few moments the metric drops under the set boundary of 5000. Tests of this kind are difficult to run over an extended period of time because of a limitation of \texttt{cassandra-stress}. When the load generator is started, it collects all available nodes in the cluster through Cassandra's communication protocol \texttt{Gossip}. \texttt{Gossip} is the protocol that Cassandra uses internally for its nodes to communicate with each other\footnote{\raggedright\url{https://docs.datastax.com/en/cassandra-oss/3.x/cassandra/architecture/archGossipAbout.html}}. While \texttt{cassandra-stress} is running, new nodes are not recognized and requests are therefore not sent to added nodes. Possible solutions to this will be discussed in \cref{ch:conclusion}.

\begin{figure}
    \centering
    %% Creator: Matplotlib, PGF backend
%%
%% To include the figure in your LaTeX document, write
%%   \input{<filename>.pgf}
%%
%% Make sure the required packages are loaded in your preamble
%%   \usepackage{pgf}
%%
%% Also ensure that all the required font packages are loaded; for instance,
%% the lmodern package is sometimes necessary when using math font.
%%   \usepackage{lmodern}
%%
%% Figures using additional raster images can only be included by \input if
%% they are in the same directory as the main LaTeX file. For loading figures
%% from other directories you can use the `import` package
%%   \usepackage{import}
%%
%% and then include the figures with
%%   \import{<path to file>}{<filename>.pgf}
%%
%% Matplotlib used the following preamble
%%   \def\mathdefault#1{#1}
%%   \everymath=\expandafter{\the\everymath\displaystyle}
%%   
%%   \usepackage{fontspec}
%%   \setmainfont{DejaVuSerif.ttf}[Path=\detokenize{/Users/nkratky/private/polaris-elasticity-strategies/test/scripts/.venv/lib/python3.11/site-packages/matplotlib/mpl-data/fonts/ttf/}]
%%   \setsansfont{Arial.ttf}[Path=\detokenize{/System/Library/Fonts/Supplemental/}]
%%   \setmonofont{DejaVuSansMono.ttf}[Path=\detokenize{/Users/nkratky/private/polaris-elasticity-strategies/test/scripts/.venv/lib/python3.11/site-packages/matplotlib/mpl-data/fonts/ttf/}]
%%   \makeatletter\@ifpackageloaded{underscore}{}{\usepackage[strings]{underscore}}\makeatother
%%
\begingroup%
\makeatletter%
\begin{pgfpicture}%
\pgfpathrectangle{\pgfpointorigin}{\pgfqpoint{5.600000in}{3.000000in}}%
\pgfusepath{use as bounding box, clip}%
\begin{pgfscope}%
\pgfsetbuttcap%
\pgfsetmiterjoin%
\definecolor{currentfill}{rgb}{1.000000,1.000000,1.000000}%
\pgfsetfillcolor{currentfill}%
\pgfsetlinewidth{0.000000pt}%
\definecolor{currentstroke}{rgb}{1.000000,1.000000,1.000000}%
\pgfsetstrokecolor{currentstroke}%
\pgfsetdash{}{0pt}%
\pgfpathmoveto{\pgfqpoint{0.000000in}{0.000000in}}%
\pgfpathlineto{\pgfqpoint{5.600000in}{0.000000in}}%
\pgfpathlineto{\pgfqpoint{5.600000in}{3.000000in}}%
\pgfpathlineto{\pgfqpoint{0.000000in}{3.000000in}}%
\pgfpathlineto{\pgfqpoint{0.000000in}{0.000000in}}%
\pgfpathclose%
\pgfusepath{fill}%
\end{pgfscope}%
\begin{pgfscope}%
\pgfsetbuttcap%
\pgfsetmiterjoin%
\definecolor{currentfill}{rgb}{0.917647,0.917647,0.949020}%
\pgfsetfillcolor{currentfill}%
\pgfsetlinewidth{0.000000pt}%
\definecolor{currentstroke}{rgb}{0.000000,0.000000,0.000000}%
\pgfsetstrokecolor{currentstroke}%
\pgfsetstrokeopacity{0.000000}%
\pgfsetdash{}{0pt}%
\pgfpathmoveto{\pgfqpoint{0.946717in}{0.663635in}}%
\pgfpathlineto{\pgfqpoint{2.880912in}{0.663635in}}%
\pgfpathlineto{\pgfqpoint{2.880912in}{2.820000in}}%
\pgfpathlineto{\pgfqpoint{0.946717in}{2.820000in}}%
\pgfpathlineto{\pgfqpoint{0.946717in}{0.663635in}}%
\pgfpathclose%
\pgfusepath{fill}%
\end{pgfscope}%
\begin{pgfscope}%
\pgfpathrectangle{\pgfqpoint{0.946717in}{0.663635in}}{\pgfqpoint{1.934195in}{2.156365in}}%
\pgfusepath{clip}%
\pgfsetroundcap%
\pgfsetroundjoin%
\pgfsetlinewidth{1.003750pt}%
\definecolor{currentstroke}{rgb}{1.000000,1.000000,1.000000}%
\pgfsetstrokecolor{currentstroke}%
\pgfsetdash{}{0pt}%
\pgfpathmoveto{\pgfqpoint{1.034635in}{0.663635in}}%
\pgfpathlineto{\pgfqpoint{1.034635in}{2.820000in}}%
\pgfusepath{stroke}%
\end{pgfscope}%
\begin{pgfscope}%
\definecolor{textcolor}{rgb}{0.150000,0.150000,0.150000}%
\pgfsetstrokecolor{textcolor}%
\pgfsetfillcolor{textcolor}%
\pgftext[x=1.034635in,y=0.531691in,,top]{\color{textcolor}{\sffamily\fontsize{11.000000}{13.200000}\selectfont\catcode`\^=\active\def^{\ifmmode\sp\else\^{}\fi}\catcode`\%=\active\def%{\%}0}}%
\end{pgfscope}%
\begin{pgfscope}%
\pgfpathrectangle{\pgfqpoint{0.946717in}{0.663635in}}{\pgfqpoint{1.934195in}{2.156365in}}%
\pgfusepath{clip}%
\pgfsetroundcap%
\pgfsetroundjoin%
\pgfsetlinewidth{1.003750pt}%
\definecolor{currentstroke}{rgb}{1.000000,1.000000,1.000000}%
\pgfsetstrokecolor{currentstroke}%
\pgfsetdash{}{0pt}%
\pgfpathmoveto{\pgfqpoint{1.674038in}{0.663635in}}%
\pgfpathlineto{\pgfqpoint{1.674038in}{2.820000in}}%
\pgfusepath{stroke}%
\end{pgfscope}%
\begin{pgfscope}%
\definecolor{textcolor}{rgb}{0.150000,0.150000,0.150000}%
\pgfsetstrokecolor{textcolor}%
\pgfsetfillcolor{textcolor}%
\pgftext[x=1.674038in,y=0.531691in,,top]{\color{textcolor}{\sffamily\fontsize{11.000000}{13.200000}\selectfont\catcode`\^=\active\def^{\ifmmode\sp\else\^{}\fi}\catcode`\%=\active\def%{\%}200}}%
\end{pgfscope}%
\begin{pgfscope}%
\pgfpathrectangle{\pgfqpoint{0.946717in}{0.663635in}}{\pgfqpoint{1.934195in}{2.156365in}}%
\pgfusepath{clip}%
\pgfsetroundcap%
\pgfsetroundjoin%
\pgfsetlinewidth{1.003750pt}%
\definecolor{currentstroke}{rgb}{1.000000,1.000000,1.000000}%
\pgfsetstrokecolor{currentstroke}%
\pgfsetdash{}{0pt}%
\pgfpathmoveto{\pgfqpoint{2.313441in}{0.663635in}}%
\pgfpathlineto{\pgfqpoint{2.313441in}{2.820000in}}%
\pgfusepath{stroke}%
\end{pgfscope}%
\begin{pgfscope}%
\definecolor{textcolor}{rgb}{0.150000,0.150000,0.150000}%
\pgfsetstrokecolor{textcolor}%
\pgfsetfillcolor{textcolor}%
\pgftext[x=2.313441in,y=0.531691in,,top]{\color{textcolor}{\sffamily\fontsize{11.000000}{13.200000}\selectfont\catcode`\^=\active\def^{\ifmmode\sp\else\^{}\fi}\catcode`\%=\active\def%{\%}400}}%
\end{pgfscope}%
\begin{pgfscope}%
\definecolor{textcolor}{rgb}{0.150000,0.150000,0.150000}%
\pgfsetstrokecolor{textcolor}%
\pgfsetfillcolor{textcolor}%
\pgftext[x=1.913814in,y=0.336413in,,top]{\color{textcolor}{\sffamily\fontsize{12.000000}{14.400000}\selectfont\catcode`\^=\active\def^{\ifmmode\sp\else\^{}\fi}\catcode`\%=\active\def%{\%}Time (s)}}%
\end{pgfscope}%
\begin{pgfscope}%
\pgfpathrectangle{\pgfqpoint{0.946717in}{0.663635in}}{\pgfqpoint{1.934195in}{2.156365in}}%
\pgfusepath{clip}%
\pgfsetroundcap%
\pgfsetroundjoin%
\pgfsetlinewidth{1.003750pt}%
\definecolor{currentstroke}{rgb}{1.000000,1.000000,1.000000}%
\pgfsetstrokecolor{currentstroke}%
\pgfsetdash{}{0pt}%
\pgfpathmoveto{\pgfqpoint{0.946717in}{0.761652in}}%
\pgfpathlineto{\pgfqpoint{2.880912in}{0.761652in}}%
\pgfusepath{stroke}%
\end{pgfscope}%
\begin{pgfscope}%
\definecolor{textcolor}{rgb}{0.150000,0.150000,0.150000}%
\pgfsetstrokecolor{textcolor}%
\pgfsetfillcolor{textcolor}%
\pgftext[x=0.729804in, y=0.706971in, left, base]{\color{textcolor}{\sffamily\fontsize{11.000000}{13.200000}\selectfont\catcode`\^=\active\def^{\ifmmode\sp\else\^{}\fi}\catcode`\%=\active\def%{\%}0}}%
\end{pgfscope}%
\begin{pgfscope}%
\pgfpathrectangle{\pgfqpoint{0.946717in}{0.663635in}}{\pgfqpoint{1.934195in}{2.156365in}}%
\pgfusepath{clip}%
\pgfsetroundcap%
\pgfsetroundjoin%
\pgfsetlinewidth{1.003750pt}%
\definecolor{currentstroke}{rgb}{1.000000,1.000000,1.000000}%
\pgfsetstrokecolor{currentstroke}%
\pgfsetdash{}{0pt}%
\pgfpathmoveto{\pgfqpoint{0.946717in}{1.243426in}}%
\pgfpathlineto{\pgfqpoint{2.880912in}{1.243426in}}%
\pgfusepath{stroke}%
\end{pgfscope}%
\begin{pgfscope}%
\definecolor{textcolor}{rgb}{0.150000,0.150000,0.150000}%
\pgfsetstrokecolor{textcolor}%
\pgfsetfillcolor{textcolor}%
\pgftext[x=0.474901in, y=1.188746in, left, base]{\color{textcolor}{\sffamily\fontsize{11.000000}{13.200000}\selectfont\catcode`\^=\active\def^{\ifmmode\sp\else\^{}\fi}\catcode`\%=\active\def%{\%}5000}}%
\end{pgfscope}%
\begin{pgfscope}%
\pgfpathrectangle{\pgfqpoint{0.946717in}{0.663635in}}{\pgfqpoint{1.934195in}{2.156365in}}%
\pgfusepath{clip}%
\pgfsetroundcap%
\pgfsetroundjoin%
\pgfsetlinewidth{1.003750pt}%
\definecolor{currentstroke}{rgb}{1.000000,1.000000,1.000000}%
\pgfsetstrokecolor{currentstroke}%
\pgfsetdash{}{0pt}%
\pgfpathmoveto{\pgfqpoint{0.946717in}{1.725201in}}%
\pgfpathlineto{\pgfqpoint{2.880912in}{1.725201in}}%
\pgfusepath{stroke}%
\end{pgfscope}%
\begin{pgfscope}%
\definecolor{textcolor}{rgb}{0.150000,0.150000,0.150000}%
\pgfsetstrokecolor{textcolor}%
\pgfsetfillcolor{textcolor}%
\pgftext[x=0.389934in, y=1.670520in, left, base]{\color{textcolor}{\sffamily\fontsize{11.000000}{13.200000}\selectfont\catcode`\^=\active\def^{\ifmmode\sp\else\^{}\fi}\catcode`\%=\active\def%{\%}10000}}%
\end{pgfscope}%
\begin{pgfscope}%
\pgfpathrectangle{\pgfqpoint{0.946717in}{0.663635in}}{\pgfqpoint{1.934195in}{2.156365in}}%
\pgfusepath{clip}%
\pgfsetroundcap%
\pgfsetroundjoin%
\pgfsetlinewidth{1.003750pt}%
\definecolor{currentstroke}{rgb}{1.000000,1.000000,1.000000}%
\pgfsetstrokecolor{currentstroke}%
\pgfsetdash{}{0pt}%
\pgfpathmoveto{\pgfqpoint{0.946717in}{2.206975in}}%
\pgfpathlineto{\pgfqpoint{2.880912in}{2.206975in}}%
\pgfusepath{stroke}%
\end{pgfscope}%
\begin{pgfscope}%
\definecolor{textcolor}{rgb}{0.150000,0.150000,0.150000}%
\pgfsetstrokecolor{textcolor}%
\pgfsetfillcolor{textcolor}%
\pgftext[x=0.389934in, y=2.152295in, left, base]{\color{textcolor}{\sffamily\fontsize{11.000000}{13.200000}\selectfont\catcode`\^=\active\def^{\ifmmode\sp\else\^{}\fi}\catcode`\%=\active\def%{\%}15000}}%
\end{pgfscope}%
\begin{pgfscope}%
\pgfpathrectangle{\pgfqpoint{0.946717in}{0.663635in}}{\pgfqpoint{1.934195in}{2.156365in}}%
\pgfusepath{clip}%
\pgfsetroundcap%
\pgfsetroundjoin%
\pgfsetlinewidth{1.003750pt}%
\definecolor{currentstroke}{rgb}{1.000000,1.000000,1.000000}%
\pgfsetstrokecolor{currentstroke}%
\pgfsetdash{}{0pt}%
\pgfpathmoveto{\pgfqpoint{0.946717in}{2.688750in}}%
\pgfpathlineto{\pgfqpoint{2.880912in}{2.688750in}}%
\pgfusepath{stroke}%
\end{pgfscope}%
\begin{pgfscope}%
\definecolor{textcolor}{rgb}{0.150000,0.150000,0.150000}%
\pgfsetstrokecolor{textcolor}%
\pgfsetfillcolor{textcolor}%
\pgftext[x=0.389934in, y=2.634069in, left, base]{\color{textcolor}{\sffamily\fontsize{11.000000}{13.200000}\selectfont\catcode`\^=\active\def^{\ifmmode\sp\else\^{}\fi}\catcode`\%=\active\def%{\%}20000}}%
\end{pgfscope}%
\begin{pgfscope}%
\definecolor{textcolor}{rgb}{0.150000,0.150000,0.150000}%
\pgfsetstrokecolor{textcolor}%
\pgfsetfillcolor{textcolor}%
\pgftext[x=0.334378in,y=1.741818in,,bottom,rotate=90.000000]{\color{textcolor}{\sffamily\fontsize{12.000000}{14.400000}\selectfont\catcode`\^=\active\def^{\ifmmode\sp\else\^{}\fi}\catcode`\%=\active\def%{\%}Average Write Load Per Node}}%
\end{pgfscope}%
\begin{pgfscope}%
\pgfpathrectangle{\pgfqpoint{0.946717in}{0.663635in}}{\pgfqpoint{1.934195in}{2.156365in}}%
\pgfusepath{clip}%
\pgfsetroundcap%
\pgfsetroundjoin%
\pgfsetlinewidth{1.505625pt}%
\definecolor{currentstroke}{rgb}{0.298039,0.447059,0.690196}%
\pgfsetstrokecolor{currentstroke}%
\pgfsetdash{}{0pt}%
\pgfpathmoveto{\pgfqpoint{1.034635in}{0.761652in}}%
\pgfpathlineto{\pgfqpoint{1.050620in}{0.761652in}}%
\pgfpathlineto{\pgfqpoint{1.066605in}{0.761652in}}%
\pgfpathlineto{\pgfqpoint{1.082590in}{0.761652in}}%
\pgfpathlineto{\pgfqpoint{1.098575in}{0.761652in}}%
\pgfpathlineto{\pgfqpoint{1.114560in}{0.761652in}}%
\pgfpathlineto{\pgfqpoint{1.130545in}{0.761652in}}%
\pgfpathlineto{\pgfqpoint{1.146530in}{0.803544in}}%
\pgfpathlineto{\pgfqpoint{1.162515in}{0.803544in}}%
\pgfpathlineto{\pgfqpoint{1.178500in}{0.803544in}}%
\pgfpathlineto{\pgfqpoint{1.194485in}{0.803544in}}%
\pgfpathlineto{\pgfqpoint{1.210471in}{0.800208in}}%
\pgfpathlineto{\pgfqpoint{1.226456in}{0.800208in}}%
\pgfpathlineto{\pgfqpoint{1.242441in}{0.800208in}}%
\pgfpathlineto{\pgfqpoint{1.258426in}{0.800208in}}%
\pgfpathlineto{\pgfqpoint{1.274411in}{0.966691in}}%
\pgfpathlineto{\pgfqpoint{1.290396in}{0.966691in}}%
\pgfpathlineto{\pgfqpoint{1.306381in}{0.966691in}}%
\pgfpathlineto{\pgfqpoint{1.322366in}{0.966691in}}%
\pgfpathlineto{\pgfqpoint{1.338351in}{1.445737in}}%
\pgfpathlineto{\pgfqpoint{1.354336in}{1.445737in}}%
\pgfpathlineto{\pgfqpoint{1.370321in}{1.445737in}}%
\pgfpathlineto{\pgfqpoint{1.386307in}{1.445737in}}%
\pgfpathlineto{\pgfqpoint{1.402292in}{1.627569in}}%
\pgfpathlineto{\pgfqpoint{1.418277in}{1.627569in}}%
\pgfpathlineto{\pgfqpoint{1.434262in}{1.627569in}}%
\pgfpathlineto{\pgfqpoint{1.450247in}{1.627569in}}%
\pgfpathlineto{\pgfqpoint{1.466232in}{1.645603in}}%
\pgfpathlineto{\pgfqpoint{1.482217in}{1.645603in}}%
\pgfpathlineto{\pgfqpoint{1.498202in}{1.645603in}}%
\pgfpathlineto{\pgfqpoint{1.514187in}{1.645603in}}%
\pgfpathlineto{\pgfqpoint{1.530172in}{2.007291in}}%
\pgfpathlineto{\pgfqpoint{1.546157in}{2.007291in}}%
\pgfpathlineto{\pgfqpoint{1.562142in}{2.007291in}}%
\pgfpathlineto{\pgfqpoint{1.578128in}{2.007291in}}%
\pgfpathlineto{\pgfqpoint{1.594113in}{2.136263in}}%
\pgfpathlineto{\pgfqpoint{1.610098in}{2.136263in}}%
\pgfpathlineto{\pgfqpoint{1.626083in}{2.136263in}}%
\pgfpathlineto{\pgfqpoint{1.642068in}{2.136263in}}%
\pgfpathlineto{\pgfqpoint{1.658053in}{2.176474in}}%
\pgfpathlineto{\pgfqpoint{1.674038in}{2.176474in}}%
\pgfpathlineto{\pgfqpoint{1.690023in}{2.176474in}}%
\pgfpathlineto{\pgfqpoint{1.706008in}{2.176474in}}%
\pgfpathlineto{\pgfqpoint{1.721993in}{2.529427in}}%
\pgfpathlineto{\pgfqpoint{1.737978in}{2.529427in}}%
\pgfpathlineto{\pgfqpoint{1.753963in}{2.529427in}}%
\pgfpathlineto{\pgfqpoint{1.769949in}{2.529427in}}%
\pgfpathlineto{\pgfqpoint{1.785934in}{2.652978in}}%
\pgfpathlineto{\pgfqpoint{1.801919in}{2.652978in}}%
\pgfpathlineto{\pgfqpoint{1.817904in}{2.652978in}}%
\pgfpathlineto{\pgfqpoint{1.833889in}{2.652978in}}%
\pgfpathlineto{\pgfqpoint{1.849874in}{2.721983in}}%
\pgfpathlineto{\pgfqpoint{1.865859in}{2.721983in}}%
\pgfpathlineto{\pgfqpoint{1.881844in}{2.721983in}}%
\pgfpathlineto{\pgfqpoint{1.897829in}{2.721983in}}%
\pgfpathlineto{\pgfqpoint{1.913814in}{2.475762in}}%
\pgfpathlineto{\pgfqpoint{1.929799in}{2.475762in}}%
\pgfpathlineto{\pgfqpoint{1.945785in}{2.475762in}}%
\pgfpathlineto{\pgfqpoint{1.961770in}{2.475762in}}%
\pgfpathlineto{\pgfqpoint{1.977755in}{1.658006in}}%
\pgfpathlineto{\pgfqpoint{1.993740in}{1.658006in}}%
\pgfpathlineto{\pgfqpoint{2.009725in}{1.658006in}}%
\pgfpathlineto{\pgfqpoint{2.025710in}{1.658006in}}%
\pgfpathlineto{\pgfqpoint{2.041695in}{1.653545in}}%
\pgfpathlineto{\pgfqpoint{2.057680in}{1.653545in}}%
\pgfpathlineto{\pgfqpoint{2.073665in}{1.653545in}}%
\pgfpathlineto{\pgfqpoint{2.089650in}{1.653545in}}%
\pgfpathlineto{\pgfqpoint{2.105635in}{1.561695in}}%
\pgfpathlineto{\pgfqpoint{2.121620in}{1.561695in}}%
\pgfpathlineto{\pgfqpoint{2.137606in}{1.561695in}}%
\pgfpathlineto{\pgfqpoint{2.153591in}{1.561695in}}%
\pgfpathlineto{\pgfqpoint{2.169576in}{1.423998in}}%
\pgfpathlineto{\pgfqpoint{2.185561in}{1.423998in}}%
\pgfpathlineto{\pgfqpoint{2.201546in}{1.423998in}}%
\pgfpathlineto{\pgfqpoint{2.217531in}{1.423998in}}%
\pgfpathlineto{\pgfqpoint{2.233516in}{1.358510in}}%
\pgfpathlineto{\pgfqpoint{2.249501in}{1.358510in}}%
\pgfpathlineto{\pgfqpoint{2.265486in}{1.358510in}}%
\pgfpathlineto{\pgfqpoint{2.281471in}{1.358510in}}%
\pgfpathlineto{\pgfqpoint{2.297456in}{1.295104in}}%
\pgfpathlineto{\pgfqpoint{2.313441in}{1.295104in}}%
\pgfpathlineto{\pgfqpoint{2.329427in}{1.295104in}}%
\pgfpathlineto{\pgfqpoint{2.345412in}{1.295104in}}%
\pgfpathlineto{\pgfqpoint{2.361397in}{1.162725in}}%
\pgfpathlineto{\pgfqpoint{2.377382in}{1.162725in}}%
\pgfpathlineto{\pgfqpoint{2.393367in}{1.162725in}}%
\pgfpathlineto{\pgfqpoint{2.409352in}{1.162725in}}%
\pgfpathlineto{\pgfqpoint{2.425337in}{1.069570in}}%
\pgfpathlineto{\pgfqpoint{2.441322in}{1.069570in}}%
\pgfpathlineto{\pgfqpoint{2.457307in}{1.069570in}}%
\pgfpathlineto{\pgfqpoint{2.473292in}{1.069570in}}%
\pgfpathlineto{\pgfqpoint{2.489277in}{1.037435in}}%
\pgfpathlineto{\pgfqpoint{2.505263in}{1.037435in}}%
\pgfpathlineto{\pgfqpoint{2.521248in}{1.037435in}}%
\pgfpathlineto{\pgfqpoint{2.537233in}{1.037435in}}%
\pgfpathlineto{\pgfqpoint{2.553218in}{0.957524in}}%
\pgfpathlineto{\pgfqpoint{2.569203in}{0.957524in}}%
\pgfpathlineto{\pgfqpoint{2.585188in}{0.957524in}}%
\pgfpathlineto{\pgfqpoint{2.601173in}{0.957524in}}%
\pgfpathlineto{\pgfqpoint{2.617158in}{0.898503in}}%
\pgfpathlineto{\pgfqpoint{2.633143in}{0.898503in}}%
\pgfpathlineto{\pgfqpoint{2.649128in}{0.898503in}}%
\pgfpathlineto{\pgfqpoint{2.665113in}{0.898503in}}%
\pgfpathlineto{\pgfqpoint{2.681098in}{0.884924in}}%
\pgfpathlineto{\pgfqpoint{2.697084in}{0.884924in}}%
\pgfpathlineto{\pgfqpoint{2.713069in}{0.884924in}}%
\pgfpathlineto{\pgfqpoint{2.729054in}{0.884924in}}%
\pgfpathlineto{\pgfqpoint{2.745039in}{0.770701in}}%
\pgfpathlineto{\pgfqpoint{2.761024in}{0.770701in}}%
\pgfpathlineto{\pgfqpoint{2.777009in}{0.770701in}}%
\pgfpathlineto{\pgfqpoint{2.792994in}{0.770701in}}%
\pgfusepath{stroke}%
\end{pgfscope}%
\begin{pgfscope}%
\pgfpathrectangle{\pgfqpoint{0.946717in}{0.663635in}}{\pgfqpoint{1.934195in}{2.156365in}}%
\pgfusepath{clip}%
\pgfsetbuttcap%
\pgfsetroundjoin%
\pgfsetlinewidth{1.505625pt}%
\definecolor{currentstroke}{rgb}{1.000000,0.647059,0.000000}%
\pgfsetstrokecolor{currentstroke}%
\pgfsetdash{{5.550000pt}{2.400000pt}}{0.000000pt}%
\pgfpathmoveto{\pgfqpoint{1.343466in}{0.663635in}}%
\pgfpathlineto{\pgfqpoint{1.343466in}{2.820000in}}%
\pgfusepath{stroke}%
\end{pgfscope}%
\begin{pgfscope}%
\pgfsetrectcap%
\pgfsetmiterjoin%
\pgfsetlinewidth{1.254687pt}%
\definecolor{currentstroke}{rgb}{1.000000,1.000000,1.000000}%
\pgfsetstrokecolor{currentstroke}%
\pgfsetdash{}{0pt}%
\pgfpathmoveto{\pgfqpoint{0.946717in}{0.663635in}}%
\pgfpathlineto{\pgfqpoint{0.946717in}{2.820000in}}%
\pgfusepath{stroke}%
\end{pgfscope}%
\begin{pgfscope}%
\pgfsetrectcap%
\pgfsetmiterjoin%
\pgfsetlinewidth{1.254687pt}%
\definecolor{currentstroke}{rgb}{1.000000,1.000000,1.000000}%
\pgfsetstrokecolor{currentstroke}%
\pgfsetdash{}{0pt}%
\pgfpathmoveto{\pgfqpoint{2.880912in}{0.663635in}}%
\pgfpathlineto{\pgfqpoint{2.880912in}{2.820000in}}%
\pgfusepath{stroke}%
\end{pgfscope}%
\begin{pgfscope}%
\pgfsetrectcap%
\pgfsetmiterjoin%
\pgfsetlinewidth{1.254687pt}%
\definecolor{currentstroke}{rgb}{1.000000,1.000000,1.000000}%
\pgfsetstrokecolor{currentstroke}%
\pgfsetdash{}{0pt}%
\pgfpathmoveto{\pgfqpoint{0.946717in}{0.663635in}}%
\pgfpathlineto{\pgfqpoint{2.880912in}{0.663635in}}%
\pgfusepath{stroke}%
\end{pgfscope}%
\begin{pgfscope}%
\pgfsetrectcap%
\pgfsetmiterjoin%
\pgfsetlinewidth{1.254687pt}%
\definecolor{currentstroke}{rgb}{1.000000,1.000000,1.000000}%
\pgfsetstrokecolor{currentstroke}%
\pgfsetdash{}{0pt}%
\pgfpathmoveto{\pgfqpoint{0.946717in}{2.820000in}}%
\pgfpathlineto{\pgfqpoint{2.880912in}{2.820000in}}%
\pgfusepath{stroke}%
\end{pgfscope}%
\begin{pgfscope}%
\pgfsetbuttcap%
\pgfsetmiterjoin%
\definecolor{currentfill}{rgb}{0.917647,0.917647,0.949020}%
\pgfsetfillcolor{currentfill}%
\pgfsetlinewidth{0.000000pt}%
\definecolor{currentstroke}{rgb}{0.000000,0.000000,0.000000}%
\pgfsetstrokecolor{currentstroke}%
\pgfsetstrokeopacity{0.000000}%
\pgfsetdash{}{0pt}%
\pgfpathmoveto{\pgfqpoint{3.485805in}{0.663635in}}%
\pgfpathlineto{\pgfqpoint{5.420000in}{0.663635in}}%
\pgfpathlineto{\pgfqpoint{5.420000in}{2.820000in}}%
\pgfpathlineto{\pgfqpoint{3.485805in}{2.820000in}}%
\pgfpathlineto{\pgfqpoint{3.485805in}{0.663635in}}%
\pgfpathclose%
\pgfusepath{fill}%
\end{pgfscope}%
\begin{pgfscope}%
\pgfpathrectangle{\pgfqpoint{3.485805in}{0.663635in}}{\pgfqpoint{1.934195in}{2.156365in}}%
\pgfusepath{clip}%
\pgfsetroundcap%
\pgfsetroundjoin%
\pgfsetlinewidth{1.003750pt}%
\definecolor{currentstroke}{rgb}{1.000000,1.000000,1.000000}%
\pgfsetstrokecolor{currentstroke}%
\pgfsetdash{}{0pt}%
\pgfpathmoveto{\pgfqpoint{3.573723in}{0.663635in}}%
\pgfpathlineto{\pgfqpoint{3.573723in}{2.820000in}}%
\pgfusepath{stroke}%
\end{pgfscope}%
\begin{pgfscope}%
\definecolor{textcolor}{rgb}{0.150000,0.150000,0.150000}%
\pgfsetstrokecolor{textcolor}%
\pgfsetfillcolor{textcolor}%
\pgftext[x=3.573723in,y=0.531691in,,top]{\color{textcolor}{\sffamily\fontsize{11.000000}{13.200000}\selectfont\catcode`\^=\active\def^{\ifmmode\sp\else\^{}\fi}\catcode`\%=\active\def%{\%}0}}%
\end{pgfscope}%
\begin{pgfscope}%
\pgfpathrectangle{\pgfqpoint{3.485805in}{0.663635in}}{\pgfqpoint{1.934195in}{2.156365in}}%
\pgfusepath{clip}%
\pgfsetroundcap%
\pgfsetroundjoin%
\pgfsetlinewidth{1.003750pt}%
\definecolor{currentstroke}{rgb}{1.000000,1.000000,1.000000}%
\pgfsetstrokecolor{currentstroke}%
\pgfsetdash{}{0pt}%
\pgfpathmoveto{\pgfqpoint{4.190691in}{0.663635in}}%
\pgfpathlineto{\pgfqpoint{4.190691in}{2.820000in}}%
\pgfusepath{stroke}%
\end{pgfscope}%
\begin{pgfscope}%
\definecolor{textcolor}{rgb}{0.150000,0.150000,0.150000}%
\pgfsetstrokecolor{textcolor}%
\pgfsetfillcolor{textcolor}%
\pgftext[x=4.190691in,y=0.531691in,,top]{\color{textcolor}{\sffamily\fontsize{11.000000}{13.200000}\selectfont\catcode`\^=\active\def^{\ifmmode\sp\else\^{}\fi}\catcode`\%=\active\def%{\%}200}}%
\end{pgfscope}%
\begin{pgfscope}%
\pgfpathrectangle{\pgfqpoint{3.485805in}{0.663635in}}{\pgfqpoint{1.934195in}{2.156365in}}%
\pgfusepath{clip}%
\pgfsetroundcap%
\pgfsetroundjoin%
\pgfsetlinewidth{1.003750pt}%
\definecolor{currentstroke}{rgb}{1.000000,1.000000,1.000000}%
\pgfsetstrokecolor{currentstroke}%
\pgfsetdash{}{0pt}%
\pgfpathmoveto{\pgfqpoint{4.807659in}{0.663635in}}%
\pgfpathlineto{\pgfqpoint{4.807659in}{2.820000in}}%
\pgfusepath{stroke}%
\end{pgfscope}%
\begin{pgfscope}%
\definecolor{textcolor}{rgb}{0.150000,0.150000,0.150000}%
\pgfsetstrokecolor{textcolor}%
\pgfsetfillcolor{textcolor}%
\pgftext[x=4.807659in,y=0.531691in,,top]{\color{textcolor}{\sffamily\fontsize{11.000000}{13.200000}\selectfont\catcode`\^=\active\def^{\ifmmode\sp\else\^{}\fi}\catcode`\%=\active\def%{\%}400}}%
\end{pgfscope}%
\begin{pgfscope}%
\definecolor{textcolor}{rgb}{0.150000,0.150000,0.150000}%
\pgfsetstrokecolor{textcolor}%
\pgfsetfillcolor{textcolor}%
\pgftext[x=4.452902in,y=0.336413in,,top]{\color{textcolor}{\sffamily\fontsize{12.000000}{14.400000}\selectfont\catcode`\^=\active\def^{\ifmmode\sp\else\^{}\fi}\catcode`\%=\active\def%{\%}Time (s)}}%
\end{pgfscope}%
\begin{pgfscope}%
\pgfpathrectangle{\pgfqpoint{3.485805in}{0.663635in}}{\pgfqpoint{1.934195in}{2.156365in}}%
\pgfusepath{clip}%
\pgfsetroundcap%
\pgfsetroundjoin%
\pgfsetlinewidth{1.003750pt}%
\definecolor{currentstroke}{rgb}{1.000000,1.000000,1.000000}%
\pgfsetstrokecolor{currentstroke}%
\pgfsetdash{}{0pt}%
\pgfpathmoveto{\pgfqpoint{3.485805in}{1.202727in}}%
\pgfpathlineto{\pgfqpoint{5.420000in}{1.202727in}}%
\pgfusepath{stroke}%
\end{pgfscope}%
\begin{pgfscope}%
\definecolor{textcolor}{rgb}{0.150000,0.150000,0.150000}%
\pgfsetstrokecolor{textcolor}%
\pgfsetfillcolor{textcolor}%
\pgftext[x=3.268892in, y=1.148046in, left, base]{\color{textcolor}{\sffamily\fontsize{11.000000}{13.200000}\selectfont\catcode`\^=\active\def^{\ifmmode\sp\else\^{}\fi}\catcode`\%=\active\def%{\%}1}}%
\end{pgfscope}%
\begin{pgfscope}%
\pgfpathrectangle{\pgfqpoint{3.485805in}{0.663635in}}{\pgfqpoint{1.934195in}{2.156365in}}%
\pgfusepath{clip}%
\pgfsetroundcap%
\pgfsetroundjoin%
\pgfsetlinewidth{1.003750pt}%
\definecolor{currentstroke}{rgb}{1.000000,1.000000,1.000000}%
\pgfsetstrokecolor{currentstroke}%
\pgfsetdash{}{0pt}%
\pgfpathmoveto{\pgfqpoint{3.485805in}{2.280909in}}%
\pgfpathlineto{\pgfqpoint{5.420000in}{2.280909in}}%
\pgfusepath{stroke}%
\end{pgfscope}%
\begin{pgfscope}%
\definecolor{textcolor}{rgb}{0.150000,0.150000,0.150000}%
\pgfsetstrokecolor{textcolor}%
\pgfsetfillcolor{textcolor}%
\pgftext[x=3.268892in, y=2.226228in, left, base]{\color{textcolor}{\sffamily\fontsize{11.000000}{13.200000}\selectfont\catcode`\^=\active\def^{\ifmmode\sp\else\^{}\fi}\catcode`\%=\active\def%{\%}2}}%
\end{pgfscope}%
\begin{pgfscope}%
\definecolor{textcolor}{rgb}{0.150000,0.150000,0.150000}%
\pgfsetstrokecolor{textcolor}%
\pgfsetfillcolor{textcolor}%
\pgftext[x=3.213337in,y=1.741818in,,bottom,rotate=90.000000]{\color{textcolor}{\sffamily\fontsize{12.000000}{14.400000}\selectfont\catcode`\^=\active\def^{\ifmmode\sp\else\^{}\fi}\catcode`\%=\active\def%{\%}\# nodes}}%
\end{pgfscope}%
\begin{pgfscope}%
\pgfpathrectangle{\pgfqpoint{3.485805in}{0.663635in}}{\pgfqpoint{1.934195in}{2.156365in}}%
\pgfusepath{clip}%
\pgfsetroundcap%
\pgfsetroundjoin%
\pgfsetlinewidth{1.505625pt}%
\definecolor{currentstroke}{rgb}{0.298039,0.447059,0.690196}%
\pgfsetstrokecolor{currentstroke}%
\pgfsetdash{}{0pt}%
\pgfpathmoveto{\pgfqpoint{3.573723in}{1.202727in}}%
\pgfpathlineto{\pgfqpoint{3.589147in}{1.202727in}}%
\pgfpathlineto{\pgfqpoint{3.604571in}{1.202727in}}%
\pgfpathlineto{\pgfqpoint{3.619995in}{1.202727in}}%
\pgfpathlineto{\pgfqpoint{3.635419in}{1.202727in}}%
\pgfpathlineto{\pgfqpoint{3.650844in}{1.202727in}}%
\pgfpathlineto{\pgfqpoint{3.666268in}{1.202727in}}%
\pgfpathlineto{\pgfqpoint{3.681692in}{1.202727in}}%
\pgfpathlineto{\pgfqpoint{3.697116in}{1.202727in}}%
\pgfpathlineto{\pgfqpoint{3.712540in}{1.202727in}}%
\pgfpathlineto{\pgfqpoint{3.727965in}{1.202727in}}%
\pgfpathlineto{\pgfqpoint{3.743389in}{1.202727in}}%
\pgfpathlineto{\pgfqpoint{3.758813in}{1.202727in}}%
\pgfpathlineto{\pgfqpoint{3.774237in}{1.202727in}}%
\pgfpathlineto{\pgfqpoint{3.789661in}{1.202727in}}%
\pgfpathlineto{\pgfqpoint{3.805086in}{1.202727in}}%
\pgfpathlineto{\pgfqpoint{3.820510in}{1.202727in}}%
\pgfpathlineto{\pgfqpoint{3.835934in}{1.202727in}}%
\pgfpathlineto{\pgfqpoint{3.851358in}{1.202727in}}%
\pgfpathlineto{\pgfqpoint{3.866783in}{1.202727in}}%
\pgfpathlineto{\pgfqpoint{3.882207in}{1.202727in}}%
\pgfpathlineto{\pgfqpoint{3.897631in}{1.202727in}}%
\pgfpathlineto{\pgfqpoint{3.913055in}{1.202727in}}%
\pgfpathlineto{\pgfqpoint{3.928479in}{1.202727in}}%
\pgfpathlineto{\pgfqpoint{3.943904in}{1.202727in}}%
\pgfpathlineto{\pgfqpoint{3.959328in}{1.202727in}}%
\pgfpathlineto{\pgfqpoint{3.974752in}{1.202727in}}%
\pgfpathlineto{\pgfqpoint{3.990176in}{1.202727in}}%
\pgfpathlineto{\pgfqpoint{4.005600in}{1.202727in}}%
\pgfpathlineto{\pgfqpoint{4.021025in}{1.202727in}}%
\pgfpathlineto{\pgfqpoint{4.036449in}{1.202727in}}%
\pgfpathlineto{\pgfqpoint{4.051873in}{1.202727in}}%
\pgfpathlineto{\pgfqpoint{4.067297in}{1.202727in}}%
\pgfpathlineto{\pgfqpoint{4.082721in}{1.202727in}}%
\pgfpathlineto{\pgfqpoint{4.098146in}{1.202727in}}%
\pgfpathlineto{\pgfqpoint{4.113570in}{1.202727in}}%
\pgfpathlineto{\pgfqpoint{4.128994in}{1.202727in}}%
\pgfpathlineto{\pgfqpoint{4.144418in}{1.202727in}}%
\pgfpathlineto{\pgfqpoint{4.159842in}{1.202727in}}%
\pgfpathlineto{\pgfqpoint{4.175267in}{1.202727in}}%
\pgfpathlineto{\pgfqpoint{4.190691in}{1.202727in}}%
\pgfpathlineto{\pgfqpoint{4.206115in}{1.202727in}}%
\pgfpathlineto{\pgfqpoint{4.221539in}{1.202727in}}%
\pgfpathlineto{\pgfqpoint{4.236963in}{1.202727in}}%
\pgfpathlineto{\pgfqpoint{4.252388in}{1.202727in}}%
\pgfpathlineto{\pgfqpoint{4.267812in}{1.202727in}}%
\pgfpathlineto{\pgfqpoint{4.283236in}{1.202727in}}%
\pgfpathlineto{\pgfqpoint{4.298660in}{1.202727in}}%
\pgfpathlineto{\pgfqpoint{4.314084in}{1.202727in}}%
\pgfpathlineto{\pgfqpoint{4.329509in}{1.202727in}}%
\pgfpathlineto{\pgfqpoint{4.344933in}{1.202727in}}%
\pgfpathlineto{\pgfqpoint{4.360357in}{1.202727in}}%
\pgfpathlineto{\pgfqpoint{4.375781in}{1.202727in}}%
\pgfpathlineto{\pgfqpoint{4.391206in}{1.202727in}}%
\pgfpathlineto{\pgfqpoint{4.406630in}{1.202727in}}%
\pgfpathlineto{\pgfqpoint{4.422054in}{2.280909in}}%
\pgfpathlineto{\pgfqpoint{4.437478in}{2.280909in}}%
\pgfpathlineto{\pgfqpoint{4.452902in}{2.280909in}}%
\pgfpathlineto{\pgfqpoint{4.468327in}{2.280909in}}%
\pgfpathlineto{\pgfqpoint{4.483751in}{2.280909in}}%
\pgfpathlineto{\pgfqpoint{4.499175in}{2.280909in}}%
\pgfpathlineto{\pgfqpoint{4.514599in}{2.280909in}}%
\pgfpathlineto{\pgfqpoint{4.530023in}{2.280909in}}%
\pgfpathlineto{\pgfqpoint{4.545448in}{2.280909in}}%
\pgfpathlineto{\pgfqpoint{4.560872in}{2.280909in}}%
\pgfpathlineto{\pgfqpoint{4.576296in}{2.280909in}}%
\pgfpathlineto{\pgfqpoint{4.591720in}{2.280909in}}%
\pgfpathlineto{\pgfqpoint{4.607144in}{2.280909in}}%
\pgfpathlineto{\pgfqpoint{4.622569in}{2.280909in}}%
\pgfpathlineto{\pgfqpoint{4.637993in}{2.280909in}}%
\pgfpathlineto{\pgfqpoint{4.653417in}{2.280909in}}%
\pgfpathlineto{\pgfqpoint{4.668841in}{2.280909in}}%
\pgfpathlineto{\pgfqpoint{4.684265in}{2.280909in}}%
\pgfpathlineto{\pgfqpoint{4.699690in}{2.280909in}}%
\pgfpathlineto{\pgfqpoint{4.715114in}{2.280909in}}%
\pgfpathlineto{\pgfqpoint{4.730538in}{2.280909in}}%
\pgfpathlineto{\pgfqpoint{4.745962in}{2.280909in}}%
\pgfpathlineto{\pgfqpoint{4.761386in}{2.280909in}}%
\pgfpathlineto{\pgfqpoint{4.776811in}{2.280909in}}%
\pgfpathlineto{\pgfqpoint{4.792235in}{2.280909in}}%
\pgfpathlineto{\pgfqpoint{4.807659in}{2.280909in}}%
\pgfpathlineto{\pgfqpoint{4.823083in}{2.280909in}}%
\pgfpathlineto{\pgfqpoint{4.838507in}{2.280909in}}%
\pgfpathlineto{\pgfqpoint{4.853932in}{2.280909in}}%
\pgfpathlineto{\pgfqpoint{4.869356in}{2.280909in}}%
\pgfpathlineto{\pgfqpoint{4.884780in}{2.280909in}}%
\pgfpathlineto{\pgfqpoint{4.900204in}{2.280909in}}%
\pgfpathlineto{\pgfqpoint{4.915628in}{2.280909in}}%
\pgfpathlineto{\pgfqpoint{4.931053in}{2.280909in}}%
\pgfpathlineto{\pgfqpoint{4.946477in}{2.280909in}}%
\pgfpathlineto{\pgfqpoint{4.961901in}{2.280909in}}%
\pgfpathlineto{\pgfqpoint{4.977325in}{2.280909in}}%
\pgfpathlineto{\pgfqpoint{4.992750in}{2.280909in}}%
\pgfpathlineto{\pgfqpoint{5.008174in}{2.280909in}}%
\pgfpathlineto{\pgfqpoint{5.023598in}{2.280909in}}%
\pgfpathlineto{\pgfqpoint{5.039022in}{2.280909in}}%
\pgfpathlineto{\pgfqpoint{5.054446in}{2.280909in}}%
\pgfpathlineto{\pgfqpoint{5.069871in}{2.280909in}}%
\pgfpathlineto{\pgfqpoint{5.085295in}{2.280909in}}%
\pgfpathlineto{\pgfqpoint{5.100719in}{2.280909in}}%
\pgfpathlineto{\pgfqpoint{5.116143in}{2.280909in}}%
\pgfpathlineto{\pgfqpoint{5.131567in}{2.280909in}}%
\pgfpathlineto{\pgfqpoint{5.146992in}{2.280909in}}%
\pgfpathlineto{\pgfqpoint{5.162416in}{2.280909in}}%
\pgfpathlineto{\pgfqpoint{5.177840in}{2.280909in}}%
\pgfpathlineto{\pgfqpoint{5.193264in}{2.280909in}}%
\pgfpathlineto{\pgfqpoint{5.208688in}{2.280909in}}%
\pgfpathlineto{\pgfqpoint{5.224113in}{2.280909in}}%
\pgfpathlineto{\pgfqpoint{5.239537in}{2.280909in}}%
\pgfpathlineto{\pgfqpoint{5.254961in}{2.280909in}}%
\pgfpathlineto{\pgfqpoint{5.270385in}{2.280909in}}%
\pgfpathlineto{\pgfqpoint{5.285809in}{2.280909in}}%
\pgfpathlineto{\pgfqpoint{5.301234in}{2.280909in}}%
\pgfpathlineto{\pgfqpoint{5.316658in}{2.280909in}}%
\pgfpathlineto{\pgfqpoint{5.332082in}{2.280909in}}%
\pgfusepath{stroke}%
\end{pgfscope}%
\begin{pgfscope}%
\pgfsetrectcap%
\pgfsetmiterjoin%
\pgfsetlinewidth{1.254687pt}%
\definecolor{currentstroke}{rgb}{1.000000,1.000000,1.000000}%
\pgfsetstrokecolor{currentstroke}%
\pgfsetdash{}{0pt}%
\pgfpathmoveto{\pgfqpoint{3.485805in}{0.663635in}}%
\pgfpathlineto{\pgfqpoint{3.485805in}{2.820000in}}%
\pgfusepath{stroke}%
\end{pgfscope}%
\begin{pgfscope}%
\pgfsetrectcap%
\pgfsetmiterjoin%
\pgfsetlinewidth{1.254687pt}%
\definecolor{currentstroke}{rgb}{1.000000,1.000000,1.000000}%
\pgfsetstrokecolor{currentstroke}%
\pgfsetdash{}{0pt}%
\pgfpathmoveto{\pgfqpoint{5.420000in}{0.663635in}}%
\pgfpathlineto{\pgfqpoint{5.420000in}{2.820000in}}%
\pgfusepath{stroke}%
\end{pgfscope}%
\begin{pgfscope}%
\pgfsetrectcap%
\pgfsetmiterjoin%
\pgfsetlinewidth{1.254687pt}%
\definecolor{currentstroke}{rgb}{1.000000,1.000000,1.000000}%
\pgfsetstrokecolor{currentstroke}%
\pgfsetdash{}{0pt}%
\pgfpathmoveto{\pgfqpoint{3.485805in}{0.663635in}}%
\pgfpathlineto{\pgfqpoint{5.420000in}{0.663635in}}%
\pgfusepath{stroke}%
\end{pgfscope}%
\begin{pgfscope}%
\pgfsetrectcap%
\pgfsetmiterjoin%
\pgfsetlinewidth{1.254687pt}%
\definecolor{currentstroke}{rgb}{1.000000,1.000000,1.000000}%
\pgfsetstrokecolor{currentstroke}%
\pgfsetdash{}{0pt}%
\pgfpathmoveto{\pgfqpoint{3.485805in}{2.820000in}}%
\pgfpathlineto{\pgfqpoint{5.420000in}{2.820000in}}%
\pgfusepath{stroke}%
\end{pgfscope}%
\begin{pgfscope}%
\pgfsetbuttcap%
\pgfsetmiterjoin%
\definecolor{currentfill}{rgb}{0.917647,0.917647,0.949020}%
\pgfsetfillcolor{currentfill}%
\pgfsetfillopacity{0.800000}%
\pgfsetlinewidth{1.003750pt}%
\definecolor{currentstroke}{rgb}{0.800000,0.800000,0.800000}%
\pgfsetstrokecolor{currentstroke}%
\pgfsetstrokeopacity{0.800000}%
\pgfsetdash{}{0pt}%
\pgfpathmoveto{\pgfqpoint{2.100358in}{2.751281in}}%
\pgfpathlineto{\pgfqpoint{3.499642in}{2.751281in}}%
\pgfpathquadraticcurveto{\pgfqpoint{3.521864in}{2.751281in}}{\pgfqpoint{3.521864in}{2.773503in}}%
\pgfpathlineto{\pgfqpoint{3.521864in}{2.922222in}}%
\pgfpathquadraticcurveto{\pgfqpoint{3.521864in}{2.944444in}}{\pgfqpoint{3.499642in}{2.944444in}}%
\pgfpathlineto{\pgfqpoint{2.100358in}{2.944444in}}%
\pgfpathquadraticcurveto{\pgfqpoint{2.078136in}{2.944444in}}{\pgfqpoint{2.078136in}{2.922222in}}%
\pgfpathlineto{\pgfqpoint{2.078136in}{2.773503in}}%
\pgfpathquadraticcurveto{\pgfqpoint{2.078136in}{2.751281in}}{\pgfqpoint{2.100358in}{2.751281in}}%
\pgfpathlineto{\pgfqpoint{2.100358in}{2.751281in}}%
\pgfpathclose%
\pgfusepath{stroke,fill}%
\end{pgfscope}%
\begin{pgfscope}%
\pgfsetbuttcap%
\pgfsetroundjoin%
\pgfsetlinewidth{1.505625pt}%
\definecolor{currentstroke}{rgb}{1.000000,0.647059,0.000000}%
\pgfsetstrokecolor{currentstroke}%
\pgfsetdash{{5.550000pt}{2.400000pt}}{0.000000pt}%
\pgfpathmoveto{\pgfqpoint{2.122580in}{2.857997in}}%
\pgfpathlineto{\pgfqpoint{2.233691in}{2.857997in}}%
\pgfpathlineto{\pgfqpoint{2.344803in}{2.857997in}}%
\pgfusepath{stroke}%
\end{pgfscope}%
\begin{pgfscope}%
\definecolor{textcolor}{rgb}{0.150000,0.150000,0.150000}%
\pgfsetstrokecolor{textcolor}%
\pgfsetfillcolor{textcolor}%
\pgftext[x=2.433691in,y=2.819108in,left,base]{\color{textcolor}{\sffamily\fontsize{8.000000}{9.600000}\selectfont\catcode`\^=\active\def^{\ifmmode\sp\else\^{}\fi}\catcode`\%=\active\def%{\%}start of scaling action}}%
\end{pgfscope}%
\end{pgfpicture}%
\makeatother%
\endgroup%

    \caption{Average write load per node and amount of nodes during a horizontal scaling action}
    \label{fig:horizontal-elasticity}
\end{figure}

\subsection{Diagonal Elasticity Strategy}

As explained earlier, the diagonal elasticity strategy combines the capabilites of the vertical and horizontal elasticity strategy into one single elasticity strategy.

\Cref{fig:diagonal-elasticity} summarizes all metrics into a single illustration. The starting configuration was set to be a single k8ssandra node with resources of 2 CPUs and 6GB of memory. After starting the elasticity strategy controller it can be seen in \cref{fig:diagonal-elasticity-limits} that the controller immediatly reduces both CPU and memory resources. The reason for that can be seen in \cref{fig:diagonal-elasticity}, subfigure \texttt{c} and \texttt{d}. Right at the start, both CPU and memory utilization was not within the tolerance range of the target utilization. Therefore both CPU and memory limits were reduced. After the inital adjustment, the CPU utilization was still far away from the targeted amount. That is because the CPU resources hit the statically set lower bounds. The memory utilization however climbed above the targeted amount, therefore it was reduced again during the second scaling action.

Similarly to \cref{sec:evaluation-vertical-elasticity}, during Cassandra reconsiliation metrics are not very useful. This is again highlighted in red in \cref{fig:diagonal-elasticity}.

During the second scaling action it can be seen that vertical and horizontal scaling indeed can happen simultaneously. In subfigure \texttt{b} of \cref{fig:diagonal-elasticity} the node count increased to 2, whereas in \cref{fig:diagonal-elasticity-limits} the memory limits increased. Note, that the \texttt{k8ssandra-operator} adjusts those values one at a time. This means that first the second k8ssandra node is started and then both Pods will get its resources updated accordingly.

Horizontal scaling action are taken when the write load per node reaches a certain threshold, 5000 in this example. In \cref{fig:diagonal-elasticity}, subfigure \texttt{a} it can be seen that after the second, third and fourth scaling event, the k8ssandra cluster size is increased, thus an additional node is started. During the fifth and last scaling event no additional node is started, because the statically set maximum amount of nodes is reached. It is also visible, that the total write throughput increases with increasing node count. This can be further illustrated by multiplying the estimated peak write load with the current node count. \(18000 * 1 = 18000,\ 12000 * 2 = 24000,\ 9000 * 3 = 27000 \rightarrow 18000 < 24000 < 27000\).

Because of the in \cref{sec:evaluation-horizontal-elasticity} addressed drawback of \texttt{cassandra-stress}, which does not detect changes to the cluster architecture, stress tests were cancelled after new nodes were added, and restarted when Cassandra had finished its reconsiliation process.

The advantage of this elasticity strategy is its ability to scale vertically and horizontally independently. This means that during times of low demand resources can be saved or used by other applications. A lower amount of resources also implies lower costs. During high demand times resources can be claimed again to provide a sufficient service level. If k8ssandra reports a high amount of writes the elasticity strategy can the also decide to scale-out horizontally by adding more nodes. As it was shown in \cref{sec:stress-testing} this increases the total throughput. As mentioned before, horizontal scale-in is not implemented in this project. This will be further addressed in \cref{sec:future-work}.

\begin{figure}
    \centering
    %% Creator: Matplotlib, PGF backend
%%
%% To include the figure in your LaTeX document, write
%%   \input{<filename>.pgf}
%%
%% Make sure the required packages are loaded in your preamble
%%   \usepackage{pgf}
%%
%% Also ensure that all the required font packages are loaded; for instance,
%% the lmodern package is sometimes necessary when using math font.
%%   \usepackage{lmodern}
%%
%% Figures using additional raster images can only be included by \input if
%% they are in the same directory as the main LaTeX file. For loading figures
%% from other directories you can use the `import` package
%%   \usepackage{import}
%%
%% and then include the figures with
%%   \import{<path to file>}{<filename>.pgf}
%%
%% Matplotlib used the following preamble
%%   \def\mathdefault#1{#1}
%%   \everymath=\expandafter{\the\everymath\displaystyle}
%%   
%%   \usepackage{fontspec}
%%   \setmainfont{DejaVuSerif.ttf}[Path=\detokenize{/usr/local/lib/python3.11/site-packages/matplotlib/mpl-data/fonts/ttf/}]
%%   \setsansfont{Arial.ttf}[Path=\detokenize{/System/Library/Fonts/Supplemental/}]
%%   \setmonofont{DejaVuSansMono.ttf}[Path=\detokenize{/usr/local/lib/python3.11/site-packages/matplotlib/mpl-data/fonts/ttf/}]
%%   \makeatletter\@ifpackageloaded{underscore}{}{\usepackage[strings]{underscore}}\makeatother
%%
\begingroup%
\makeatletter%
\begin{pgfpicture}%
\pgfpathrectangle{\pgfpointorigin}{\pgfqpoint{5.600000in}{5.000000in}}%
\pgfusepath{use as bounding box, clip}%
\begin{pgfscope}%
\pgfsetbuttcap%
\pgfsetmiterjoin%
\definecolor{currentfill}{rgb}{1.000000,1.000000,1.000000}%
\pgfsetfillcolor{currentfill}%
\pgfsetlinewidth{0.000000pt}%
\definecolor{currentstroke}{rgb}{1.000000,1.000000,1.000000}%
\pgfsetstrokecolor{currentstroke}%
\pgfsetdash{}{0pt}%
\pgfpathmoveto{\pgfqpoint{0.000000in}{0.000000in}}%
\pgfpathlineto{\pgfqpoint{5.600000in}{0.000000in}}%
\pgfpathlineto{\pgfqpoint{5.600000in}{5.000000in}}%
\pgfpathlineto{\pgfqpoint{0.000000in}{5.000000in}}%
\pgfpathlineto{\pgfqpoint{0.000000in}{0.000000in}}%
\pgfpathclose%
\pgfusepath{fill}%
\end{pgfscope}%
\begin{pgfscope}%
\pgfsetbuttcap%
\pgfsetmiterjoin%
\definecolor{currentfill}{rgb}{0.917647,0.917647,0.949020}%
\pgfsetfillcolor{currentfill}%
\pgfsetlinewidth{0.000000pt}%
\definecolor{currentstroke}{rgb}{0.000000,0.000000,0.000000}%
\pgfsetstrokecolor{currentstroke}%
\pgfsetstrokeopacity{0.000000}%
\pgfsetdash{}{0pt}%
\pgfpathmoveto{\pgfqpoint{0.946717in}{3.073635in}}%
\pgfpathlineto{\pgfqpoint{2.772727in}{3.073635in}}%
\pgfpathlineto{\pgfqpoint{2.772727in}{4.615329in}}%
\pgfpathlineto{\pgfqpoint{0.946717in}{4.615329in}}%
\pgfpathlineto{\pgfqpoint{0.946717in}{3.073635in}}%
\pgfpathclose%
\pgfusepath{fill}%
\end{pgfscope}%
\begin{pgfscope}%
\pgfpathrectangle{\pgfqpoint{0.946717in}{3.073635in}}{\pgfqpoint{1.826010in}{1.541693in}}%
\pgfusepath{clip}%
\pgfsetroundcap%
\pgfsetroundjoin%
\pgfsetlinewidth{1.003750pt}%
\definecolor{currentstroke}{rgb}{1.000000,1.000000,1.000000}%
\pgfsetstrokecolor{currentstroke}%
\pgfsetdash{}{0pt}%
\pgfpathmoveto{\pgfqpoint{1.029717in}{3.073635in}}%
\pgfpathlineto{\pgfqpoint{1.029717in}{4.615329in}}%
\pgfusepath{stroke}%
\end{pgfscope}%
\begin{pgfscope}%
\definecolor{textcolor}{rgb}{0.150000,0.150000,0.150000}%
\pgfsetstrokecolor{textcolor}%
\pgfsetfillcolor{textcolor}%
\pgftext[x=1.029717in,y=2.941691in,,top]{\color{textcolor}{\sffamily\fontsize{11.000000}{13.200000}\selectfont\catcode`\^=\active\def^{\ifmmode\sp\else\^{}\fi}\catcode`\%=\active\def%{\%}0}}%
\end{pgfscope}%
\begin{pgfscope}%
\pgfpathrectangle{\pgfqpoint{0.946717in}{3.073635in}}{\pgfqpoint{1.826010in}{1.541693in}}%
\pgfusepath{clip}%
\pgfsetroundcap%
\pgfsetroundjoin%
\pgfsetlinewidth{1.003750pt}%
\definecolor{currentstroke}{rgb}{1.000000,1.000000,1.000000}%
\pgfsetstrokecolor{currentstroke}%
\pgfsetdash{}{0pt}%
\pgfpathmoveto{\pgfqpoint{2.048128in}{3.073635in}}%
\pgfpathlineto{\pgfqpoint{2.048128in}{4.615329in}}%
\pgfusepath{stroke}%
\end{pgfscope}%
\begin{pgfscope}%
\definecolor{textcolor}{rgb}{0.150000,0.150000,0.150000}%
\pgfsetstrokecolor{textcolor}%
\pgfsetfillcolor{textcolor}%
\pgftext[x=2.048128in,y=2.941691in,,top]{\color{textcolor}{\sffamily\fontsize{11.000000}{13.200000}\selectfont\catcode`\^=\active\def^{\ifmmode\sp\else\^{}\fi}\catcode`\%=\active\def%{\%}2000}}%
\end{pgfscope}%
\begin{pgfscope}%
\definecolor{textcolor}{rgb}{0.150000,0.150000,0.150000}%
\pgfsetstrokecolor{textcolor}%
\pgfsetfillcolor{textcolor}%
\pgftext[x=1.859722in,y=2.746413in,,top]{\color{textcolor}{\sffamily\fontsize{12.000000}{14.400000}\selectfont\catcode`\^=\active\def^{\ifmmode\sp\else\^{}\fi}\catcode`\%=\active\def%{\%}Time (s)}}%
\end{pgfscope}%
\begin{pgfscope}%
\pgfpathrectangle{\pgfqpoint{0.946717in}{3.073635in}}{\pgfqpoint{1.826010in}{1.541693in}}%
\pgfusepath{clip}%
\pgfsetroundcap%
\pgfsetroundjoin%
\pgfsetlinewidth{1.003750pt}%
\definecolor{currentstroke}{rgb}{1.000000,1.000000,1.000000}%
\pgfsetstrokecolor{currentstroke}%
\pgfsetdash{}{0pt}%
\pgfpathmoveto{\pgfqpoint{0.946717in}{3.143712in}}%
\pgfpathlineto{\pgfqpoint{2.772727in}{3.143712in}}%
\pgfusepath{stroke}%
\end{pgfscope}%
\begin{pgfscope}%
\definecolor{textcolor}{rgb}{0.150000,0.150000,0.150000}%
\pgfsetstrokecolor{textcolor}%
\pgfsetfillcolor{textcolor}%
\pgftext[x=0.729804in, y=3.089032in, left, base]{\color{textcolor}{\sffamily\fontsize{11.000000}{13.200000}\selectfont\catcode`\^=\active\def^{\ifmmode\sp\else\^{}\fi}\catcode`\%=\active\def%{\%}0}}%
\end{pgfscope}%
\begin{pgfscope}%
\pgfpathrectangle{\pgfqpoint{0.946717in}{3.073635in}}{\pgfqpoint{1.826010in}{1.541693in}}%
\pgfusepath{clip}%
\pgfsetroundcap%
\pgfsetroundjoin%
\pgfsetlinewidth{1.003750pt}%
\definecolor{currentstroke}{rgb}{1.000000,1.000000,1.000000}%
\pgfsetstrokecolor{currentstroke}%
\pgfsetdash{}{0pt}%
\pgfpathmoveto{\pgfqpoint{0.946717in}{3.515177in}}%
\pgfpathlineto{\pgfqpoint{2.772727in}{3.515177in}}%
\pgfusepath{stroke}%
\end{pgfscope}%
\begin{pgfscope}%
\definecolor{textcolor}{rgb}{0.150000,0.150000,0.150000}%
\pgfsetstrokecolor{textcolor}%
\pgfsetfillcolor{textcolor}%
\pgftext[x=0.474901in, y=3.460496in, left, base]{\color{textcolor}{\sffamily\fontsize{11.000000}{13.200000}\selectfont\catcode`\^=\active\def^{\ifmmode\sp\else\^{}\fi}\catcode`\%=\active\def%{\%}5000}}%
\end{pgfscope}%
\begin{pgfscope}%
\pgfpathrectangle{\pgfqpoint{0.946717in}{3.073635in}}{\pgfqpoint{1.826010in}{1.541693in}}%
\pgfusepath{clip}%
\pgfsetroundcap%
\pgfsetroundjoin%
\pgfsetlinewidth{1.003750pt}%
\definecolor{currentstroke}{rgb}{1.000000,1.000000,1.000000}%
\pgfsetstrokecolor{currentstroke}%
\pgfsetdash{}{0pt}%
\pgfpathmoveto{\pgfqpoint{0.946717in}{3.886641in}}%
\pgfpathlineto{\pgfqpoint{2.772727in}{3.886641in}}%
\pgfusepath{stroke}%
\end{pgfscope}%
\begin{pgfscope}%
\definecolor{textcolor}{rgb}{0.150000,0.150000,0.150000}%
\pgfsetstrokecolor{textcolor}%
\pgfsetfillcolor{textcolor}%
\pgftext[x=0.389934in, y=3.831960in, left, base]{\color{textcolor}{\sffamily\fontsize{11.000000}{13.200000}\selectfont\catcode`\^=\active\def^{\ifmmode\sp\else\^{}\fi}\catcode`\%=\active\def%{\%}10000}}%
\end{pgfscope}%
\begin{pgfscope}%
\pgfpathrectangle{\pgfqpoint{0.946717in}{3.073635in}}{\pgfqpoint{1.826010in}{1.541693in}}%
\pgfusepath{clip}%
\pgfsetroundcap%
\pgfsetroundjoin%
\pgfsetlinewidth{1.003750pt}%
\definecolor{currentstroke}{rgb}{1.000000,1.000000,1.000000}%
\pgfsetstrokecolor{currentstroke}%
\pgfsetdash{}{0pt}%
\pgfpathmoveto{\pgfqpoint{0.946717in}{4.258105in}}%
\pgfpathlineto{\pgfqpoint{2.772727in}{4.258105in}}%
\pgfusepath{stroke}%
\end{pgfscope}%
\begin{pgfscope}%
\definecolor{textcolor}{rgb}{0.150000,0.150000,0.150000}%
\pgfsetstrokecolor{textcolor}%
\pgfsetfillcolor{textcolor}%
\pgftext[x=0.389934in, y=4.203424in, left, base]{\color{textcolor}{\sffamily\fontsize{11.000000}{13.200000}\selectfont\catcode`\^=\active\def^{\ifmmode\sp\else\^{}\fi}\catcode`\%=\active\def%{\%}15000}}%
\end{pgfscope}%
\begin{pgfscope}%
\definecolor{textcolor}{rgb}{0.150000,0.150000,0.150000}%
\pgfsetstrokecolor{textcolor}%
\pgfsetfillcolor{textcolor}%
\pgftext[x=0.334378in,y=3.844482in,,bottom,rotate=90.000000]{\color{textcolor}{\sffamily\fontsize{12.000000}{14.400000}\selectfont\catcode`\^=\active\def^{\ifmmode\sp\else\^{}\fi}\catcode`\%=\active\def%{\%}Average Write Load Per Node}}%
\end{pgfscope}%
\begin{pgfscope}%
\pgfpathrectangle{\pgfqpoint{0.946717in}{3.073635in}}{\pgfqpoint{1.826010in}{1.541693in}}%
\pgfusepath{clip}%
\pgfsetbuttcap%
\pgfsetroundjoin%
\pgfsetlinewidth{1.505625pt}%
\definecolor{currentstroke}{rgb}{1.000000,0.647059,0.000000}%
\pgfsetstrokecolor{currentstroke}%
\pgfsetdash{{1.500000pt}{2.475000pt}}{0.000000pt}%
\pgfpathmoveto{\pgfqpoint{1.174331in}{3.073635in}}%
\pgfpathlineto{\pgfqpoint{1.174331in}{4.615329in}}%
\pgfusepath{stroke}%
\end{pgfscope}%
\begin{pgfscope}%
\pgfpathrectangle{\pgfqpoint{0.946717in}{3.073635in}}{\pgfqpoint{1.826010in}{1.541693in}}%
\pgfusepath{clip}%
\pgfsetbuttcap%
\pgfsetroundjoin%
\pgfsetlinewidth{1.505625pt}%
\definecolor{currentstroke}{rgb}{1.000000,0.647059,0.000000}%
\pgfsetstrokecolor{currentstroke}%
\pgfsetdash{{1.500000pt}{2.475000pt}}{0.000000pt}%
\pgfpathmoveto{\pgfqpoint{1.482401in}{3.073635in}}%
\pgfpathlineto{\pgfqpoint{1.482401in}{4.615329in}}%
\pgfusepath{stroke}%
\end{pgfscope}%
\begin{pgfscope}%
\pgfpathrectangle{\pgfqpoint{0.946717in}{3.073635in}}{\pgfqpoint{1.826010in}{1.541693in}}%
\pgfusepath{clip}%
\pgfsetbuttcap%
\pgfsetroundjoin%
\pgfsetlinewidth{1.505625pt}%
\definecolor{currentstroke}{rgb}{1.000000,0.647059,0.000000}%
\pgfsetstrokecolor{currentstroke}%
\pgfsetdash{{1.500000pt}{2.475000pt}}{0.000000pt}%
\pgfpathmoveto{\pgfqpoint{1.788433in}{3.073635in}}%
\pgfpathlineto{\pgfqpoint{1.788433in}{4.615329in}}%
\pgfusepath{stroke}%
\end{pgfscope}%
\begin{pgfscope}%
\pgfpathrectangle{\pgfqpoint{0.946717in}{3.073635in}}{\pgfqpoint{1.826010in}{1.541693in}}%
\pgfusepath{clip}%
\pgfsetbuttcap%
\pgfsetroundjoin%
\pgfsetlinewidth{1.505625pt}%
\definecolor{currentstroke}{rgb}{1.000000,0.647059,0.000000}%
\pgfsetstrokecolor{currentstroke}%
\pgfsetdash{{1.500000pt}{2.475000pt}}{0.000000pt}%
\pgfpathmoveto{\pgfqpoint{2.094975in}{3.073635in}}%
\pgfpathlineto{\pgfqpoint{2.094975in}{4.615329in}}%
\pgfusepath{stroke}%
\end{pgfscope}%
\begin{pgfscope}%
\pgfpathrectangle{\pgfqpoint{0.946717in}{3.073635in}}{\pgfqpoint{1.826010in}{1.541693in}}%
\pgfusepath{clip}%
\pgfsetbuttcap%
\pgfsetroundjoin%
\pgfsetlinewidth{1.505625pt}%
\definecolor{currentstroke}{rgb}{1.000000,0.647059,0.000000}%
\pgfsetstrokecolor{currentstroke}%
\pgfsetdash{{1.500000pt}{2.475000pt}}{0.000000pt}%
\pgfpathmoveto{\pgfqpoint{2.400498in}{3.073635in}}%
\pgfpathlineto{\pgfqpoint{2.400498in}{4.615329in}}%
\pgfusepath{stroke}%
\end{pgfscope}%
\begin{pgfscope}%
\pgfpathrectangle{\pgfqpoint{0.946717in}{3.073635in}}{\pgfqpoint{1.826010in}{1.541693in}}%
\pgfusepath{clip}%
\pgfsetroundcap%
\pgfsetroundjoin%
\pgfsetlinewidth{1.505625pt}%
\definecolor{currentstroke}{rgb}{0.298039,0.447059,0.690196}%
\pgfsetstrokecolor{currentstroke}%
\pgfsetdash{}{0pt}%
\pgfpathmoveto{\pgfqpoint{1.029717in}{3.143712in}}%
\pgfpathlineto{\pgfqpoint{1.345424in}{3.143712in}}%
\pgfpathlineto{\pgfqpoint{1.347971in}{3.158604in}}%
\pgfpathlineto{\pgfqpoint{1.355609in}{3.158604in}}%
\pgfpathlineto{\pgfqpoint{1.358155in}{3.230184in}}%
\pgfpathlineto{\pgfqpoint{1.365793in}{3.230184in}}%
\pgfpathlineto{\pgfqpoint{1.368339in}{3.325679in}}%
\pgfpathlineto{\pgfqpoint{1.396345in}{3.326570in}}%
\pgfpathlineto{\pgfqpoint{1.398891in}{3.328107in}}%
\pgfpathlineto{\pgfqpoint{1.406529in}{3.328107in}}%
\pgfpathlineto{\pgfqpoint{1.409075in}{3.441047in}}%
\pgfpathlineto{\pgfqpoint{1.426897in}{3.441047in}}%
\pgfpathlineto{\pgfqpoint{1.429443in}{3.704511in}}%
\pgfpathlineto{\pgfqpoint{1.437081in}{3.704511in}}%
\pgfpathlineto{\pgfqpoint{1.439627in}{3.904847in}}%
\pgfpathlineto{\pgfqpoint{1.447266in}{3.904847in}}%
\pgfpathlineto{\pgfqpoint{1.449812in}{4.022609in}}%
\pgfpathlineto{\pgfqpoint{1.457450in}{4.022609in}}%
\pgfpathlineto{\pgfqpoint{1.459996in}{4.175209in}}%
\pgfpathlineto{\pgfqpoint{1.467634in}{4.175209in}}%
\pgfpathlineto{\pgfqpoint{1.470180in}{4.260429in}}%
\pgfpathlineto{\pgfqpoint{1.477818in}{4.260429in}}%
\pgfpathlineto{\pgfqpoint{1.480364in}{4.384592in}}%
\pgfpathlineto{\pgfqpoint{1.488002in}{4.384592in}}%
\pgfpathlineto{\pgfqpoint{1.490548in}{4.545252in}}%
\pgfpathlineto{\pgfqpoint{1.498186in}{4.545252in}}%
\pgfpathlineto{\pgfqpoint{1.500732in}{4.481242in}}%
\pgfpathlineto{\pgfqpoint{1.508370in}{4.481242in}}%
\pgfpathlineto{\pgfqpoint{1.510916in}{4.367637in}}%
\pgfpathlineto{\pgfqpoint{1.518554in}{4.367637in}}%
\pgfpathlineto{\pgfqpoint{1.521100in}{4.133257in}}%
\pgfpathlineto{\pgfqpoint{1.528738in}{4.133257in}}%
\pgfpathlineto{\pgfqpoint{1.531284in}{4.153455in}}%
\pgfpathlineto{\pgfqpoint{1.538923in}{4.153455in}}%
\pgfpathlineto{\pgfqpoint{1.541469in}{4.102588in}}%
\pgfpathlineto{\pgfqpoint{1.549107in}{4.102588in}}%
\pgfpathlineto{\pgfqpoint{1.551653in}{3.980361in}}%
\pgfpathlineto{\pgfqpoint{1.559291in}{3.980361in}}%
\pgfpathlineto{\pgfqpoint{1.561837in}{3.940022in}}%
\pgfpathlineto{\pgfqpoint{1.569475in}{3.940022in}}%
\pgfpathlineto{\pgfqpoint{1.572021in}{3.782894in}}%
\pgfpathlineto{\pgfqpoint{1.579659in}{3.782894in}}%
\pgfpathlineto{\pgfqpoint{1.582205in}{3.611803in}}%
\pgfpathlineto{\pgfqpoint{1.589843in}{3.611803in}}%
\pgfpathlineto{\pgfqpoint{1.592389in}{3.589019in}}%
\pgfpathlineto{\pgfqpoint{1.600027in}{3.589019in}}%
\pgfpathlineto{\pgfqpoint{1.602573in}{3.409127in}}%
\pgfpathlineto{\pgfqpoint{1.610211in}{3.409127in}}%
\pgfpathlineto{\pgfqpoint{1.612757in}{3.188053in}}%
\pgfpathlineto{\pgfqpoint{1.620395in}{3.188053in}}%
\pgfpathlineto{\pgfqpoint{1.622941in}{3.184504in}}%
\pgfpathlineto{\pgfqpoint{1.630579in}{3.184504in}}%
\pgfpathlineto{\pgfqpoint{1.633126in}{3.143713in}}%
\pgfpathlineto{\pgfqpoint{1.834262in}{3.143712in}}%
\pgfpathlineto{\pgfqpoint{1.836808in}{3.150334in}}%
\pgfpathlineto{\pgfqpoint{1.844446in}{3.150334in}}%
\pgfpathlineto{\pgfqpoint{1.846992in}{3.169542in}}%
\pgfpathlineto{\pgfqpoint{1.854630in}{3.169542in}}%
\pgfpathlineto{\pgfqpoint{1.857176in}{3.193126in}}%
\pgfpathlineto{\pgfqpoint{1.864814in}{3.193126in}}%
\pgfpathlineto{\pgfqpoint{1.867360in}{3.252127in}}%
\pgfpathlineto{\pgfqpoint{1.874998in}{3.252127in}}%
\pgfpathlineto{\pgfqpoint{1.877544in}{3.294064in}}%
\pgfpathlineto{\pgfqpoint{1.885182in}{3.294064in}}%
\pgfpathlineto{\pgfqpoint{1.887728in}{3.335186in}}%
\pgfpathlineto{\pgfqpoint{1.895366in}{3.335186in}}%
\pgfpathlineto{\pgfqpoint{1.897912in}{3.429337in}}%
\pgfpathlineto{\pgfqpoint{1.905550in}{3.429337in}}%
\pgfpathlineto{\pgfqpoint{1.908096in}{3.490808in}}%
\pgfpathlineto{\pgfqpoint{1.915735in}{3.490808in}}%
\pgfpathlineto{\pgfqpoint{1.918281in}{3.486618in}}%
\pgfpathlineto{\pgfqpoint{1.925919in}{3.486618in}}%
\pgfpathlineto{\pgfqpoint{1.928465in}{3.591193in}}%
\pgfpathlineto{\pgfqpoint{1.936103in}{3.591193in}}%
\pgfpathlineto{\pgfqpoint{1.938649in}{3.694351in}}%
\pgfpathlineto{\pgfqpoint{1.946287in}{3.694351in}}%
\pgfpathlineto{\pgfqpoint{1.948833in}{3.687280in}}%
\pgfpathlineto{\pgfqpoint{1.956471in}{3.687280in}}%
\pgfpathlineto{\pgfqpoint{1.959017in}{3.745061in}}%
\pgfpathlineto{\pgfqpoint{1.966655in}{3.745061in}}%
\pgfpathlineto{\pgfqpoint{1.969201in}{3.777427in}}%
\pgfpathlineto{\pgfqpoint{1.976839in}{3.777427in}}%
\pgfpathlineto{\pgfqpoint{1.979385in}{3.826823in}}%
\pgfpathlineto{\pgfqpoint{1.987023in}{3.826823in}}%
\pgfpathlineto{\pgfqpoint{1.989569in}{3.976143in}}%
\pgfpathlineto{\pgfqpoint{1.997207in}{3.976143in}}%
\pgfpathlineto{\pgfqpoint{1.999753in}{4.001743in}}%
\pgfpathlineto{\pgfqpoint{2.007391in}{4.001743in}}%
\pgfpathlineto{\pgfqpoint{2.009937in}{3.962403in}}%
\pgfpathlineto{\pgfqpoint{2.017576in}{3.962403in}}%
\pgfpathlineto{\pgfqpoint{2.020122in}{4.044956in}}%
\pgfpathlineto{\pgfqpoint{2.027760in}{4.044956in}}%
\pgfpathlineto{\pgfqpoint{2.030306in}{4.055442in}}%
\pgfpathlineto{\pgfqpoint{2.037944in}{4.055442in}}%
\pgfpathlineto{\pgfqpoint{2.040490in}{4.012724in}}%
\pgfpathlineto{\pgfqpoint{2.048128in}{4.012724in}}%
\pgfpathlineto{\pgfqpoint{2.050674in}{4.065681in}}%
\pgfpathlineto{\pgfqpoint{2.058312in}{4.065681in}}%
\pgfpathlineto{\pgfqpoint{2.060858in}{4.069955in}}%
\pgfpathlineto{\pgfqpoint{2.068496in}{4.069955in}}%
\pgfpathlineto{\pgfqpoint{2.071042in}{4.028993in}}%
\pgfpathlineto{\pgfqpoint{2.078680in}{4.028993in}}%
\pgfpathlineto{\pgfqpoint{2.081226in}{4.065725in}}%
\pgfpathlineto{\pgfqpoint{2.088864in}{4.065725in}}%
\pgfpathlineto{\pgfqpoint{2.091410in}{3.992072in}}%
\pgfpathlineto{\pgfqpoint{2.099048in}{3.992072in}}%
\pgfpathlineto{\pgfqpoint{2.101594in}{4.012084in}}%
\pgfpathlineto{\pgfqpoint{2.109233in}{4.012084in}}%
\pgfpathlineto{\pgfqpoint{2.111779in}{3.944202in}}%
\pgfpathlineto{\pgfqpoint{2.119417in}{3.944202in}}%
\pgfpathlineto{\pgfqpoint{2.121963in}{3.860907in}}%
\pgfpathlineto{\pgfqpoint{2.129601in}{3.860907in}}%
\pgfpathlineto{\pgfqpoint{2.132147in}{3.806081in}}%
\pgfpathlineto{\pgfqpoint{2.139785in}{3.806081in}}%
\pgfpathlineto{\pgfqpoint{2.142331in}{3.733038in}}%
\pgfpathlineto{\pgfqpoint{2.149969in}{3.733038in}}%
\pgfpathlineto{\pgfqpoint{2.152515in}{3.636018in}}%
\pgfpathlineto{\pgfqpoint{2.160153in}{3.636018in}}%
\pgfpathlineto{\pgfqpoint{2.162699in}{3.577450in}}%
\pgfpathlineto{\pgfqpoint{2.170337in}{3.577450in}}%
\pgfpathlineto{\pgfqpoint{2.172883in}{3.384412in}}%
\pgfpathlineto{\pgfqpoint{2.180521in}{3.384412in}}%
\pgfpathlineto{\pgfqpoint{2.183067in}{3.317805in}}%
\pgfpathlineto{\pgfqpoint{2.190705in}{3.317805in}}%
\pgfpathlineto{\pgfqpoint{2.193251in}{3.284332in}}%
\pgfpathlineto{\pgfqpoint{2.200890in}{3.284332in}}%
\pgfpathlineto{\pgfqpoint{2.203436in}{3.246485in}}%
\pgfpathlineto{\pgfqpoint{2.211074in}{3.246485in}}%
\pgfpathlineto{\pgfqpoint{2.213620in}{3.186206in}}%
\pgfpathlineto{\pgfqpoint{2.221258in}{3.186206in}}%
\pgfpathlineto{\pgfqpoint{2.223804in}{3.160619in}}%
\pgfpathlineto{\pgfqpoint{2.231442in}{3.160619in}}%
\pgfpathlineto{\pgfqpoint{2.233988in}{3.143713in}}%
\pgfpathlineto{\pgfqpoint{2.292546in}{3.143713in}}%
\pgfpathlineto{\pgfqpoint{2.295093in}{3.155726in}}%
\pgfpathlineto{\pgfqpoint{2.302731in}{3.155726in}}%
\pgfpathlineto{\pgfqpoint{2.305277in}{3.172750in}}%
\pgfpathlineto{\pgfqpoint{2.312915in}{3.172750in}}%
\pgfpathlineto{\pgfqpoint{2.315461in}{3.209737in}}%
\pgfpathlineto{\pgfqpoint{2.323099in}{3.209737in}}%
\pgfpathlineto{\pgfqpoint{2.325645in}{3.273511in}}%
\pgfpathlineto{\pgfqpoint{2.333283in}{3.273511in}}%
\pgfpathlineto{\pgfqpoint{2.335829in}{3.327104in}}%
\pgfpathlineto{\pgfqpoint{2.343467in}{3.327104in}}%
\pgfpathlineto{\pgfqpoint{2.346013in}{3.357443in}}%
\pgfpathlineto{\pgfqpoint{2.353651in}{3.357443in}}%
\pgfpathlineto{\pgfqpoint{2.356197in}{3.430893in}}%
\pgfpathlineto{\pgfqpoint{2.363835in}{3.430893in}}%
\pgfpathlineto{\pgfqpoint{2.366381in}{3.483452in}}%
\pgfpathlineto{\pgfqpoint{2.374019in}{3.483452in}}%
\pgfpathlineto{\pgfqpoint{2.376565in}{3.517672in}}%
\pgfpathlineto{\pgfqpoint{2.384203in}{3.517672in}}%
\pgfpathlineto{\pgfqpoint{2.386749in}{3.576414in}}%
\pgfpathlineto{\pgfqpoint{2.394388in}{3.576414in}}%
\pgfpathlineto{\pgfqpoint{2.396934in}{3.607149in}}%
\pgfpathlineto{\pgfqpoint{2.404572in}{3.607149in}}%
\pgfpathlineto{\pgfqpoint{2.407118in}{3.629033in}}%
\pgfpathlineto{\pgfqpoint{2.414756in}{3.629033in}}%
\pgfpathlineto{\pgfqpoint{2.417302in}{3.667188in}}%
\pgfpathlineto{\pgfqpoint{2.424940in}{3.667188in}}%
\pgfpathlineto{\pgfqpoint{2.427486in}{3.714440in}}%
\pgfpathlineto{\pgfqpoint{2.435124in}{3.714440in}}%
\pgfpathlineto{\pgfqpoint{2.437670in}{3.764119in}}%
\pgfpathlineto{\pgfqpoint{2.445308in}{3.764119in}}%
\pgfpathlineto{\pgfqpoint{2.447854in}{3.784660in}}%
\pgfpathlineto{\pgfqpoint{2.455492in}{3.784660in}}%
\pgfpathlineto{\pgfqpoint{2.458038in}{3.776924in}}%
\pgfpathlineto{\pgfqpoint{2.465676in}{3.776924in}}%
\pgfpathlineto{\pgfqpoint{2.468222in}{3.766661in}}%
\pgfpathlineto{\pgfqpoint{2.475860in}{3.766661in}}%
\pgfpathlineto{\pgfqpoint{2.478406in}{3.775350in}}%
\pgfpathlineto{\pgfqpoint{2.486045in}{3.775350in}}%
\pgfpathlineto{\pgfqpoint{2.488591in}{3.731102in}}%
\pgfpathlineto{\pgfqpoint{2.496229in}{3.731102in}}%
\pgfpathlineto{\pgfqpoint{2.498775in}{3.709295in}}%
\pgfpathlineto{\pgfqpoint{2.506413in}{3.709295in}}%
\pgfpathlineto{\pgfqpoint{2.508959in}{3.651785in}}%
\pgfpathlineto{\pgfqpoint{2.516597in}{3.651785in}}%
\pgfpathlineto{\pgfqpoint{2.519143in}{3.589265in}}%
\pgfpathlineto{\pgfqpoint{2.526781in}{3.589265in}}%
\pgfpathlineto{\pgfqpoint{2.529327in}{3.563465in}}%
\pgfpathlineto{\pgfqpoint{2.536965in}{3.563465in}}%
\pgfpathlineto{\pgfqpoint{2.539511in}{3.515025in}}%
\pgfpathlineto{\pgfqpoint{2.547149in}{3.515025in}}%
\pgfpathlineto{\pgfqpoint{2.549695in}{3.433818in}}%
\pgfpathlineto{\pgfqpoint{2.557333in}{3.433818in}}%
\pgfpathlineto{\pgfqpoint{2.559879in}{3.410842in}}%
\pgfpathlineto{\pgfqpoint{2.567517in}{3.410842in}}%
\pgfpathlineto{\pgfqpoint{2.570063in}{3.380870in}}%
\pgfpathlineto{\pgfqpoint{2.577701in}{3.380870in}}%
\pgfpathlineto{\pgfqpoint{2.580248in}{3.313050in}}%
\pgfpathlineto{\pgfqpoint{2.587886in}{3.313050in}}%
\pgfpathlineto{\pgfqpoint{2.590432in}{3.286796in}}%
\pgfpathlineto{\pgfqpoint{2.598070in}{3.286796in}}%
\pgfpathlineto{\pgfqpoint{2.600616in}{3.240662in}}%
\pgfpathlineto{\pgfqpoint{2.608254in}{3.240662in}}%
\pgfpathlineto{\pgfqpoint{2.610800in}{3.175896in}}%
\pgfpathlineto{\pgfqpoint{2.618438in}{3.175896in}}%
\pgfpathlineto{\pgfqpoint{2.620984in}{3.158698in}}%
\pgfpathlineto{\pgfqpoint{2.628622in}{3.158698in}}%
\pgfpathlineto{\pgfqpoint{2.631168in}{3.143712in}}%
\pgfpathlineto{\pgfqpoint{2.689727in}{3.143712in}}%
\pgfpathlineto{\pgfqpoint{2.689727in}{3.143712in}}%
\pgfusepath{stroke}%
\end{pgfscope}%
\begin{pgfscope}%
\pgfsetrectcap%
\pgfsetmiterjoin%
\pgfsetlinewidth{1.254687pt}%
\definecolor{currentstroke}{rgb}{1.000000,1.000000,1.000000}%
\pgfsetstrokecolor{currentstroke}%
\pgfsetdash{}{0pt}%
\pgfpathmoveto{\pgfqpoint{0.946717in}{3.073635in}}%
\pgfpathlineto{\pgfqpoint{0.946717in}{4.615329in}}%
\pgfusepath{stroke}%
\end{pgfscope}%
\begin{pgfscope}%
\pgfsetrectcap%
\pgfsetmiterjoin%
\pgfsetlinewidth{1.254687pt}%
\definecolor{currentstroke}{rgb}{1.000000,1.000000,1.000000}%
\pgfsetstrokecolor{currentstroke}%
\pgfsetdash{}{0pt}%
\pgfpathmoveto{\pgfqpoint{2.772727in}{3.073635in}}%
\pgfpathlineto{\pgfqpoint{2.772727in}{4.615329in}}%
\pgfusepath{stroke}%
\end{pgfscope}%
\begin{pgfscope}%
\pgfsetrectcap%
\pgfsetmiterjoin%
\pgfsetlinewidth{1.254687pt}%
\definecolor{currentstroke}{rgb}{1.000000,1.000000,1.000000}%
\pgfsetstrokecolor{currentstroke}%
\pgfsetdash{}{0pt}%
\pgfpathmoveto{\pgfqpoint{0.946717in}{3.073635in}}%
\pgfpathlineto{\pgfqpoint{2.772727in}{3.073635in}}%
\pgfusepath{stroke}%
\end{pgfscope}%
\begin{pgfscope}%
\pgfsetrectcap%
\pgfsetmiterjoin%
\pgfsetlinewidth{1.254687pt}%
\definecolor{currentstroke}{rgb}{1.000000,1.000000,1.000000}%
\pgfsetstrokecolor{currentstroke}%
\pgfsetdash{}{0pt}%
\pgfpathmoveto{\pgfqpoint{0.946717in}{4.615329in}}%
\pgfpathlineto{\pgfqpoint{2.772727in}{4.615329in}}%
\pgfusepath{stroke}%
\end{pgfscope}%
\begin{pgfscope}%
\definecolor{textcolor}{rgb}{0.150000,0.150000,0.150000}%
\pgfsetstrokecolor{textcolor}%
\pgfsetfillcolor{textcolor}%
\pgftext[x=1.859722in,y=4.698662in,,base]{\color{textcolor}{\sffamily\fontsize{12.000000}{14.400000}\selectfont\catcode`\^=\active\def^{\ifmmode\sp\else\^{}\fi}\catcode`\%=\active\def%{\%}a)}}%
\end{pgfscope}%
\begin{pgfscope}%
\pgfsetbuttcap%
\pgfsetmiterjoin%
\definecolor{currentfill}{rgb}{0.917647,0.917647,0.949020}%
\pgfsetfillcolor{currentfill}%
\pgfsetlinewidth{0.000000pt}%
\definecolor{currentstroke}{rgb}{0.000000,0.000000,0.000000}%
\pgfsetstrokecolor{currentstroke}%
\pgfsetstrokeopacity{0.000000}%
\pgfsetdash{}{0pt}%
\pgfpathmoveto{\pgfqpoint{3.593990in}{3.073635in}}%
\pgfpathlineto{\pgfqpoint{5.420000in}{3.073635in}}%
\pgfpathlineto{\pgfqpoint{5.420000in}{4.615329in}}%
\pgfpathlineto{\pgfqpoint{3.593990in}{4.615329in}}%
\pgfpathlineto{\pgfqpoint{3.593990in}{3.073635in}}%
\pgfpathclose%
\pgfusepath{fill}%
\end{pgfscope}%
\begin{pgfscope}%
\pgfpathrectangle{\pgfqpoint{3.593990in}{3.073635in}}{\pgfqpoint{1.826010in}{1.541693in}}%
\pgfusepath{clip}%
\pgfsetroundcap%
\pgfsetroundjoin%
\pgfsetlinewidth{1.003750pt}%
\definecolor{currentstroke}{rgb}{1.000000,1.000000,1.000000}%
\pgfsetstrokecolor{currentstroke}%
\pgfsetdash{}{0pt}%
\pgfpathmoveto{\pgfqpoint{3.676990in}{3.073635in}}%
\pgfpathlineto{\pgfqpoint{3.676990in}{4.615329in}}%
\pgfusepath{stroke}%
\end{pgfscope}%
\begin{pgfscope}%
\definecolor{textcolor}{rgb}{0.150000,0.150000,0.150000}%
\pgfsetstrokecolor{textcolor}%
\pgfsetfillcolor{textcolor}%
\pgftext[x=3.676990in,y=2.941691in,,top]{\color{textcolor}{\sffamily\fontsize{11.000000}{13.200000}\selectfont\catcode`\^=\active\def^{\ifmmode\sp\else\^{}\fi}\catcode`\%=\active\def%{\%}0}}%
\end{pgfscope}%
\begin{pgfscope}%
\pgfpathrectangle{\pgfqpoint{3.593990in}{3.073635in}}{\pgfqpoint{1.826010in}{1.541693in}}%
\pgfusepath{clip}%
\pgfsetroundcap%
\pgfsetroundjoin%
\pgfsetlinewidth{1.003750pt}%
\definecolor{currentstroke}{rgb}{1.000000,1.000000,1.000000}%
\pgfsetstrokecolor{currentstroke}%
\pgfsetdash{}{0pt}%
\pgfpathmoveto{\pgfqpoint{4.683056in}{3.073635in}}%
\pgfpathlineto{\pgfqpoint{4.683056in}{4.615329in}}%
\pgfusepath{stroke}%
\end{pgfscope}%
\begin{pgfscope}%
\definecolor{textcolor}{rgb}{0.150000,0.150000,0.150000}%
\pgfsetstrokecolor{textcolor}%
\pgfsetfillcolor{textcolor}%
\pgftext[x=4.683056in,y=2.941691in,,top]{\color{textcolor}{\sffamily\fontsize{11.000000}{13.200000}\selectfont\catcode`\^=\active\def^{\ifmmode\sp\else\^{}\fi}\catcode`\%=\active\def%{\%}2000}}%
\end{pgfscope}%
\begin{pgfscope}%
\definecolor{textcolor}{rgb}{0.150000,0.150000,0.150000}%
\pgfsetstrokecolor{textcolor}%
\pgfsetfillcolor{textcolor}%
\pgftext[x=4.506995in,y=2.746413in,,top]{\color{textcolor}{\sffamily\fontsize{12.000000}{14.400000}\selectfont\catcode`\^=\active\def^{\ifmmode\sp\else\^{}\fi}\catcode`\%=\active\def%{\%}Time (s)}}%
\end{pgfscope}%
\begin{pgfscope}%
\pgfpathrectangle{\pgfqpoint{3.593990in}{3.073635in}}{\pgfqpoint{1.826010in}{1.541693in}}%
\pgfusepath{clip}%
\pgfsetroundcap%
\pgfsetroundjoin%
\pgfsetlinewidth{1.003750pt}%
\definecolor{currentstroke}{rgb}{1.000000,1.000000,1.000000}%
\pgfsetstrokecolor{currentstroke}%
\pgfsetdash{}{0pt}%
\pgfpathmoveto{\pgfqpoint{3.593990in}{3.330584in}}%
\pgfpathlineto{\pgfqpoint{5.420000in}{3.330584in}}%
\pgfusepath{stroke}%
\end{pgfscope}%
\begin{pgfscope}%
\definecolor{textcolor}{rgb}{0.150000,0.150000,0.150000}%
\pgfsetstrokecolor{textcolor}%
\pgfsetfillcolor{textcolor}%
\pgftext[x=3.377077in, y=3.275904in, left, base]{\color{textcolor}{\sffamily\fontsize{11.000000}{13.200000}\selectfont\catcode`\^=\active\def^{\ifmmode\sp\else\^{}\fi}\catcode`\%=\active\def%{\%}1}}%
\end{pgfscope}%
\begin{pgfscope}%
\pgfpathrectangle{\pgfqpoint{3.593990in}{3.073635in}}{\pgfqpoint{1.826010in}{1.541693in}}%
\pgfusepath{clip}%
\pgfsetroundcap%
\pgfsetroundjoin%
\pgfsetlinewidth{1.003750pt}%
\definecolor{currentstroke}{rgb}{1.000000,1.000000,1.000000}%
\pgfsetstrokecolor{currentstroke}%
\pgfsetdash{}{0pt}%
\pgfpathmoveto{\pgfqpoint{3.593990in}{3.844482in}}%
\pgfpathlineto{\pgfqpoint{5.420000in}{3.844482in}}%
\pgfusepath{stroke}%
\end{pgfscope}%
\begin{pgfscope}%
\definecolor{textcolor}{rgb}{0.150000,0.150000,0.150000}%
\pgfsetstrokecolor{textcolor}%
\pgfsetfillcolor{textcolor}%
\pgftext[x=3.377077in, y=3.789801in, left, base]{\color{textcolor}{\sffamily\fontsize{11.000000}{13.200000}\selectfont\catcode`\^=\active\def^{\ifmmode\sp\else\^{}\fi}\catcode`\%=\active\def%{\%}2}}%
\end{pgfscope}%
\begin{pgfscope}%
\pgfpathrectangle{\pgfqpoint{3.593990in}{3.073635in}}{\pgfqpoint{1.826010in}{1.541693in}}%
\pgfusepath{clip}%
\pgfsetroundcap%
\pgfsetroundjoin%
\pgfsetlinewidth{1.003750pt}%
\definecolor{currentstroke}{rgb}{1.000000,1.000000,1.000000}%
\pgfsetstrokecolor{currentstroke}%
\pgfsetdash{}{0pt}%
\pgfpathmoveto{\pgfqpoint{3.593990in}{4.358380in}}%
\pgfpathlineto{\pgfqpoint{5.420000in}{4.358380in}}%
\pgfusepath{stroke}%
\end{pgfscope}%
\begin{pgfscope}%
\definecolor{textcolor}{rgb}{0.150000,0.150000,0.150000}%
\pgfsetstrokecolor{textcolor}%
\pgfsetfillcolor{textcolor}%
\pgftext[x=3.377077in, y=4.303699in, left, base]{\color{textcolor}{\sffamily\fontsize{11.000000}{13.200000}\selectfont\catcode`\^=\active\def^{\ifmmode\sp\else\^{}\fi}\catcode`\%=\active\def%{\%}3}}%
\end{pgfscope}%
\begin{pgfscope}%
\definecolor{textcolor}{rgb}{0.150000,0.150000,0.150000}%
\pgfsetstrokecolor{textcolor}%
\pgfsetfillcolor{textcolor}%
\pgftext[x=3.321522in,y=3.844482in,,bottom,rotate=90.000000]{\color{textcolor}{\sffamily\fontsize{12.000000}{14.400000}\selectfont\catcode`\^=\active\def^{\ifmmode\sp\else\^{}\fi}\catcode`\%=\active\def%{\%}\# nodes}}%
\end{pgfscope}%
\begin{pgfscope}%
\pgfpathrectangle{\pgfqpoint{3.593990in}{3.073635in}}{\pgfqpoint{1.826010in}{1.541693in}}%
\pgfusepath{clip}%
\pgfsetbuttcap%
\pgfsetroundjoin%
\pgfsetlinewidth{1.505625pt}%
\definecolor{currentstroke}{rgb}{1.000000,0.647059,0.000000}%
\pgfsetstrokecolor{currentstroke}%
\pgfsetdash{{1.500000pt}{2.475000pt}}{0.000000pt}%
\pgfpathmoveto{\pgfqpoint{3.819851in}{3.073635in}}%
\pgfpathlineto{\pgfqpoint{3.819851in}{4.615329in}}%
\pgfusepath{stroke}%
\end{pgfscope}%
\begin{pgfscope}%
\pgfpathrectangle{\pgfqpoint{3.593990in}{3.073635in}}{\pgfqpoint{1.826010in}{1.541693in}}%
\pgfusepath{clip}%
\pgfsetbuttcap%
\pgfsetroundjoin%
\pgfsetlinewidth{1.505625pt}%
\definecolor{currentstroke}{rgb}{1.000000,0.647059,0.000000}%
\pgfsetstrokecolor{currentstroke}%
\pgfsetdash{{1.500000pt}{2.475000pt}}{0.000000pt}%
\pgfpathmoveto{\pgfqpoint{4.124186in}{3.073635in}}%
\pgfpathlineto{\pgfqpoint{4.124186in}{4.615329in}}%
\pgfusepath{stroke}%
\end{pgfscope}%
\begin{pgfscope}%
\pgfpathrectangle{\pgfqpoint{3.593990in}{3.073635in}}{\pgfqpoint{1.826010in}{1.541693in}}%
\pgfusepath{clip}%
\pgfsetbuttcap%
\pgfsetroundjoin%
\pgfsetlinewidth{1.505625pt}%
\definecolor{currentstroke}{rgb}{1.000000,0.647059,0.000000}%
\pgfsetstrokecolor{currentstroke}%
\pgfsetdash{{1.500000pt}{2.475000pt}}{0.000000pt}%
\pgfpathmoveto{\pgfqpoint{4.426509in}{3.073635in}}%
\pgfpathlineto{\pgfqpoint{4.426509in}{4.615329in}}%
\pgfusepath{stroke}%
\end{pgfscope}%
\begin{pgfscope}%
\pgfpathrectangle{\pgfqpoint{3.593990in}{3.073635in}}{\pgfqpoint{1.826010in}{1.541693in}}%
\pgfusepath{clip}%
\pgfsetbuttcap%
\pgfsetroundjoin%
\pgfsetlinewidth{1.505625pt}%
\definecolor{currentstroke}{rgb}{1.000000,0.647059,0.000000}%
\pgfsetstrokecolor{currentstroke}%
\pgfsetdash{{1.500000pt}{2.475000pt}}{0.000000pt}%
\pgfpathmoveto{\pgfqpoint{4.729335in}{3.073635in}}%
\pgfpathlineto{\pgfqpoint{4.729335in}{4.615329in}}%
\pgfusepath{stroke}%
\end{pgfscope}%
\begin{pgfscope}%
\pgfpathrectangle{\pgfqpoint{3.593990in}{3.073635in}}{\pgfqpoint{1.826010in}{1.541693in}}%
\pgfusepath{clip}%
\pgfsetbuttcap%
\pgfsetroundjoin%
\pgfsetlinewidth{1.505625pt}%
\definecolor{currentstroke}{rgb}{1.000000,0.647059,0.000000}%
\pgfsetstrokecolor{currentstroke}%
\pgfsetdash{{1.500000pt}{2.475000pt}}{0.000000pt}%
\pgfpathmoveto{\pgfqpoint{5.031155in}{3.073635in}}%
\pgfpathlineto{\pgfqpoint{5.031155in}{4.615329in}}%
\pgfusepath{stroke}%
\end{pgfscope}%
\begin{pgfscope}%
\pgfpathrectangle{\pgfqpoint{3.593990in}{3.073635in}}{\pgfqpoint{1.826010in}{1.541693in}}%
\pgfusepath{clip}%
\pgfsetroundcap%
\pgfsetroundjoin%
\pgfsetlinewidth{1.505625pt}%
\definecolor{currentstroke}{rgb}{0.298039,0.447059,0.690196}%
\pgfsetstrokecolor{currentstroke}%
\pgfsetdash{}{0pt}%
\pgfpathmoveto{\pgfqpoint{3.676990in}{3.330584in}}%
\pgfpathlineto{\pgfqpoint{4.237872in}{3.330584in}}%
\pgfpathlineto{\pgfqpoint{4.240387in}{3.844482in}}%
\pgfpathlineto{\pgfqpoint{4.806300in}{3.844482in}}%
\pgfpathlineto{\pgfqpoint{4.808815in}{4.358380in}}%
\pgfpathlineto{\pgfqpoint{5.337000in}{4.358380in}}%
\pgfpathlineto{\pgfqpoint{5.337000in}{4.358380in}}%
\pgfusepath{stroke}%
\end{pgfscope}%
\begin{pgfscope}%
\pgfsetrectcap%
\pgfsetmiterjoin%
\pgfsetlinewidth{1.254687pt}%
\definecolor{currentstroke}{rgb}{1.000000,1.000000,1.000000}%
\pgfsetstrokecolor{currentstroke}%
\pgfsetdash{}{0pt}%
\pgfpathmoveto{\pgfqpoint{3.593990in}{3.073635in}}%
\pgfpathlineto{\pgfqpoint{3.593990in}{4.615329in}}%
\pgfusepath{stroke}%
\end{pgfscope}%
\begin{pgfscope}%
\pgfsetrectcap%
\pgfsetmiterjoin%
\pgfsetlinewidth{1.254687pt}%
\definecolor{currentstroke}{rgb}{1.000000,1.000000,1.000000}%
\pgfsetstrokecolor{currentstroke}%
\pgfsetdash{}{0pt}%
\pgfpathmoveto{\pgfqpoint{5.420000in}{3.073635in}}%
\pgfpathlineto{\pgfqpoint{5.420000in}{4.615329in}}%
\pgfusepath{stroke}%
\end{pgfscope}%
\begin{pgfscope}%
\pgfsetrectcap%
\pgfsetmiterjoin%
\pgfsetlinewidth{1.254687pt}%
\definecolor{currentstroke}{rgb}{1.000000,1.000000,1.000000}%
\pgfsetstrokecolor{currentstroke}%
\pgfsetdash{}{0pt}%
\pgfpathmoveto{\pgfqpoint{3.593990in}{3.073635in}}%
\pgfpathlineto{\pgfqpoint{5.420000in}{3.073635in}}%
\pgfusepath{stroke}%
\end{pgfscope}%
\begin{pgfscope}%
\pgfsetrectcap%
\pgfsetmiterjoin%
\pgfsetlinewidth{1.254687pt}%
\definecolor{currentstroke}{rgb}{1.000000,1.000000,1.000000}%
\pgfsetstrokecolor{currentstroke}%
\pgfsetdash{}{0pt}%
\pgfpathmoveto{\pgfqpoint{3.593990in}{4.615329in}}%
\pgfpathlineto{\pgfqpoint{5.420000in}{4.615329in}}%
\pgfusepath{stroke}%
\end{pgfscope}%
\begin{pgfscope}%
\definecolor{textcolor}{rgb}{0.150000,0.150000,0.150000}%
\pgfsetstrokecolor{textcolor}%
\pgfsetfillcolor{textcolor}%
\pgftext[x=4.506995in,y=4.698662in,,base]{\color{textcolor}{\sffamily\fontsize{12.000000}{14.400000}\selectfont\catcode`\^=\active\def^{\ifmmode\sp\else\^{}\fi}\catcode`\%=\active\def%{\%}b)}}%
\end{pgfscope}%
\begin{pgfscope}%
\pgfsetbuttcap%
\pgfsetmiterjoin%
\definecolor{currentfill}{rgb}{0.917647,0.917647,0.949020}%
\pgfsetfillcolor{currentfill}%
\pgfsetlinewidth{0.000000pt}%
\definecolor{currentstroke}{rgb}{0.000000,0.000000,0.000000}%
\pgfsetstrokecolor{currentstroke}%
\pgfsetstrokeopacity{0.000000}%
\pgfsetdash{}{0pt}%
\pgfpathmoveto{\pgfqpoint{0.946717in}{0.663635in}}%
\pgfpathlineto{\pgfqpoint{2.772727in}{0.663635in}}%
\pgfpathlineto{\pgfqpoint{2.772727in}{2.205329in}}%
\pgfpathlineto{\pgfqpoint{0.946717in}{2.205329in}}%
\pgfpathlineto{\pgfqpoint{0.946717in}{0.663635in}}%
\pgfpathclose%
\pgfusepath{fill}%
\end{pgfscope}%
\begin{pgfscope}%
\pgfpathrectangle{\pgfqpoint{0.946717in}{0.663635in}}{\pgfqpoint{1.826010in}{1.541693in}}%
\pgfusepath{clip}%
\pgfsetroundcap%
\pgfsetroundjoin%
\pgfsetlinewidth{1.003750pt}%
\definecolor{currentstroke}{rgb}{1.000000,1.000000,1.000000}%
\pgfsetstrokecolor{currentstroke}%
\pgfsetdash{}{0pt}%
\pgfpathmoveto{\pgfqpoint{1.029717in}{0.663635in}}%
\pgfpathlineto{\pgfqpoint{1.029717in}{2.205329in}}%
\pgfusepath{stroke}%
\end{pgfscope}%
\begin{pgfscope}%
\definecolor{textcolor}{rgb}{0.150000,0.150000,0.150000}%
\pgfsetstrokecolor{textcolor}%
\pgfsetfillcolor{textcolor}%
\pgftext[x=1.029717in,y=0.531691in,,top]{\color{textcolor}{\sffamily\fontsize{11.000000}{13.200000}\selectfont\catcode`\^=\active\def^{\ifmmode\sp\else\^{}\fi}\catcode`\%=\active\def%{\%}0}}%
\end{pgfscope}%
\begin{pgfscope}%
\pgfpathrectangle{\pgfqpoint{0.946717in}{0.663635in}}{\pgfqpoint{1.826010in}{1.541693in}}%
\pgfusepath{clip}%
\pgfsetroundcap%
\pgfsetroundjoin%
\pgfsetlinewidth{1.003750pt}%
\definecolor{currentstroke}{rgb}{1.000000,1.000000,1.000000}%
\pgfsetstrokecolor{currentstroke}%
\pgfsetdash{}{0pt}%
\pgfpathmoveto{\pgfqpoint{2.048128in}{0.663635in}}%
\pgfpathlineto{\pgfqpoint{2.048128in}{2.205329in}}%
\pgfusepath{stroke}%
\end{pgfscope}%
\begin{pgfscope}%
\definecolor{textcolor}{rgb}{0.150000,0.150000,0.150000}%
\pgfsetstrokecolor{textcolor}%
\pgfsetfillcolor{textcolor}%
\pgftext[x=2.048128in,y=0.531691in,,top]{\color{textcolor}{\sffamily\fontsize{11.000000}{13.200000}\selectfont\catcode`\^=\active\def^{\ifmmode\sp\else\^{}\fi}\catcode`\%=\active\def%{\%}2000}}%
\end{pgfscope}%
\begin{pgfscope}%
\definecolor{textcolor}{rgb}{0.150000,0.150000,0.150000}%
\pgfsetstrokecolor{textcolor}%
\pgfsetfillcolor{textcolor}%
\pgftext[x=1.859722in,y=0.336413in,,top]{\color{textcolor}{\sffamily\fontsize{12.000000}{14.400000}\selectfont\catcode`\^=\active\def^{\ifmmode\sp\else\^{}\fi}\catcode`\%=\active\def%{\%}Time (s)}}%
\end{pgfscope}%
\begin{pgfscope}%
\pgfpathrectangle{\pgfqpoint{0.946717in}{0.663635in}}{\pgfqpoint{1.826010in}{1.541693in}}%
\pgfusepath{clip}%
\pgfsetroundcap%
\pgfsetroundjoin%
\pgfsetlinewidth{1.003750pt}%
\definecolor{currentstroke}{rgb}{1.000000,1.000000,1.000000}%
\pgfsetstrokecolor{currentstroke}%
\pgfsetdash{}{0pt}%
\pgfpathmoveto{\pgfqpoint{0.946717in}{0.792110in}}%
\pgfpathlineto{\pgfqpoint{2.772727in}{0.792110in}}%
\pgfusepath{stroke}%
\end{pgfscope}%
\begin{pgfscope}%
\definecolor{textcolor}{rgb}{0.150000,0.150000,0.150000}%
\pgfsetstrokecolor{textcolor}%
\pgfsetfillcolor{textcolor}%
\pgftext[x=0.517423in, y=0.737429in, left, base]{\color{textcolor}{\sffamily\fontsize{11.000000}{13.200000}\selectfont\catcode`\^=\active\def^{\ifmmode\sp\else\^{}\fi}\catcode`\%=\active\def%{\%}0.00}}%
\end{pgfscope}%
\begin{pgfscope}%
\pgfpathrectangle{\pgfqpoint{0.946717in}{0.663635in}}{\pgfqpoint{1.826010in}{1.541693in}}%
\pgfusepath{clip}%
\pgfsetroundcap%
\pgfsetroundjoin%
\pgfsetlinewidth{1.003750pt}%
\definecolor{currentstroke}{rgb}{1.000000,1.000000,1.000000}%
\pgfsetstrokecolor{currentstroke}%
\pgfsetdash{}{0pt}%
\pgfpathmoveto{\pgfqpoint{0.946717in}{1.113296in}}%
\pgfpathlineto{\pgfqpoint{2.772727in}{1.113296in}}%
\pgfusepath{stroke}%
\end{pgfscope}%
\begin{pgfscope}%
\definecolor{textcolor}{rgb}{0.150000,0.150000,0.150000}%
\pgfsetstrokecolor{textcolor}%
\pgfsetfillcolor{textcolor}%
\pgftext[x=0.517423in, y=1.058615in, left, base]{\color{textcolor}{\sffamily\fontsize{11.000000}{13.200000}\selectfont\catcode`\^=\active\def^{\ifmmode\sp\else\^{}\fi}\catcode`\%=\active\def%{\%}0.25}}%
\end{pgfscope}%
\begin{pgfscope}%
\pgfpathrectangle{\pgfqpoint{0.946717in}{0.663635in}}{\pgfqpoint{1.826010in}{1.541693in}}%
\pgfusepath{clip}%
\pgfsetroundcap%
\pgfsetroundjoin%
\pgfsetlinewidth{1.003750pt}%
\definecolor{currentstroke}{rgb}{1.000000,1.000000,1.000000}%
\pgfsetstrokecolor{currentstroke}%
\pgfsetdash{}{0pt}%
\pgfpathmoveto{\pgfqpoint{0.946717in}{1.434482in}}%
\pgfpathlineto{\pgfqpoint{2.772727in}{1.434482in}}%
\pgfusepath{stroke}%
\end{pgfscope}%
\begin{pgfscope}%
\definecolor{textcolor}{rgb}{0.150000,0.150000,0.150000}%
\pgfsetstrokecolor{textcolor}%
\pgfsetfillcolor{textcolor}%
\pgftext[x=0.517423in, y=1.379801in, left, base]{\color{textcolor}{\sffamily\fontsize{11.000000}{13.200000}\selectfont\catcode`\^=\active\def^{\ifmmode\sp\else\^{}\fi}\catcode`\%=\active\def%{\%}0.50}}%
\end{pgfscope}%
\begin{pgfscope}%
\pgfpathrectangle{\pgfqpoint{0.946717in}{0.663635in}}{\pgfqpoint{1.826010in}{1.541693in}}%
\pgfusepath{clip}%
\pgfsetroundcap%
\pgfsetroundjoin%
\pgfsetlinewidth{1.003750pt}%
\definecolor{currentstroke}{rgb}{1.000000,1.000000,1.000000}%
\pgfsetstrokecolor{currentstroke}%
\pgfsetdash{}{0pt}%
\pgfpathmoveto{\pgfqpoint{0.946717in}{1.755668in}}%
\pgfpathlineto{\pgfqpoint{2.772727in}{1.755668in}}%
\pgfusepath{stroke}%
\end{pgfscope}%
\begin{pgfscope}%
\definecolor{textcolor}{rgb}{0.150000,0.150000,0.150000}%
\pgfsetstrokecolor{textcolor}%
\pgfsetfillcolor{textcolor}%
\pgftext[x=0.517423in, y=1.700987in, left, base]{\color{textcolor}{\sffamily\fontsize{11.000000}{13.200000}\selectfont\catcode`\^=\active\def^{\ifmmode\sp\else\^{}\fi}\catcode`\%=\active\def%{\%}0.75}}%
\end{pgfscope}%
\begin{pgfscope}%
\pgfpathrectangle{\pgfqpoint{0.946717in}{0.663635in}}{\pgfqpoint{1.826010in}{1.541693in}}%
\pgfusepath{clip}%
\pgfsetroundcap%
\pgfsetroundjoin%
\pgfsetlinewidth{1.003750pt}%
\definecolor{currentstroke}{rgb}{1.000000,1.000000,1.000000}%
\pgfsetstrokecolor{currentstroke}%
\pgfsetdash{}{0pt}%
\pgfpathmoveto{\pgfqpoint{0.946717in}{2.076854in}}%
\pgfpathlineto{\pgfqpoint{2.772727in}{2.076854in}}%
\pgfusepath{stroke}%
\end{pgfscope}%
\begin{pgfscope}%
\definecolor{textcolor}{rgb}{0.150000,0.150000,0.150000}%
\pgfsetstrokecolor{textcolor}%
\pgfsetfillcolor{textcolor}%
\pgftext[x=0.517423in, y=2.022174in, left, base]{\color{textcolor}{\sffamily\fontsize{11.000000}{13.200000}\selectfont\catcode`\^=\active\def^{\ifmmode\sp\else\^{}\fi}\catcode`\%=\active\def%{\%}1.00}}%
\end{pgfscope}%
\begin{pgfscope}%
\definecolor{textcolor}{rgb}{0.150000,0.150000,0.150000}%
\pgfsetstrokecolor{textcolor}%
\pgfsetfillcolor{textcolor}%
\pgftext[x=0.461867in,y=1.434482in,,bottom,rotate=90.000000]{\color{textcolor}{\sffamily\fontsize{12.000000}{14.400000}\selectfont\catcode`\^=\active\def^{\ifmmode\sp\else\^{}\fi}\catcode`\%=\active\def%{\%}CPU Utilization (%)}}%
\end{pgfscope}%
\begin{pgfscope}%
\pgfpathrectangle{\pgfqpoint{0.946717in}{0.663635in}}{\pgfqpoint{1.826010in}{1.541693in}}%
\pgfusepath{clip}%
\pgfsetbuttcap%
\pgfsetroundjoin%
\pgfsetlinewidth{1.505625pt}%
\definecolor{currentstroke}{rgb}{1.000000,0.647059,0.000000}%
\pgfsetstrokecolor{currentstroke}%
\pgfsetdash{{1.500000pt}{2.475000pt}}{0.000000pt}%
\pgfpathmoveto{\pgfqpoint{1.174331in}{0.663635in}}%
\pgfpathlineto{\pgfqpoint{1.174331in}{2.205329in}}%
\pgfusepath{stroke}%
\end{pgfscope}%
\begin{pgfscope}%
\pgfpathrectangle{\pgfqpoint{0.946717in}{0.663635in}}{\pgfqpoint{1.826010in}{1.541693in}}%
\pgfusepath{clip}%
\pgfsetbuttcap%
\pgfsetroundjoin%
\pgfsetlinewidth{1.505625pt}%
\definecolor{currentstroke}{rgb}{1.000000,0.647059,0.000000}%
\pgfsetstrokecolor{currentstroke}%
\pgfsetdash{{1.500000pt}{2.475000pt}}{0.000000pt}%
\pgfpathmoveto{\pgfqpoint{1.482401in}{0.663635in}}%
\pgfpathlineto{\pgfqpoint{1.482401in}{2.205329in}}%
\pgfusepath{stroke}%
\end{pgfscope}%
\begin{pgfscope}%
\pgfpathrectangle{\pgfqpoint{0.946717in}{0.663635in}}{\pgfqpoint{1.826010in}{1.541693in}}%
\pgfusepath{clip}%
\pgfsetbuttcap%
\pgfsetroundjoin%
\pgfsetlinewidth{1.505625pt}%
\definecolor{currentstroke}{rgb}{1.000000,0.647059,0.000000}%
\pgfsetstrokecolor{currentstroke}%
\pgfsetdash{{1.500000pt}{2.475000pt}}{0.000000pt}%
\pgfpathmoveto{\pgfqpoint{1.788433in}{0.663635in}}%
\pgfpathlineto{\pgfqpoint{1.788433in}{2.205329in}}%
\pgfusepath{stroke}%
\end{pgfscope}%
\begin{pgfscope}%
\pgfpathrectangle{\pgfqpoint{0.946717in}{0.663635in}}{\pgfqpoint{1.826010in}{1.541693in}}%
\pgfusepath{clip}%
\pgfsetbuttcap%
\pgfsetroundjoin%
\pgfsetlinewidth{1.505625pt}%
\definecolor{currentstroke}{rgb}{1.000000,0.647059,0.000000}%
\pgfsetstrokecolor{currentstroke}%
\pgfsetdash{{1.500000pt}{2.475000pt}}{0.000000pt}%
\pgfpathmoveto{\pgfqpoint{2.094975in}{0.663635in}}%
\pgfpathlineto{\pgfqpoint{2.094975in}{2.205329in}}%
\pgfusepath{stroke}%
\end{pgfscope}%
\begin{pgfscope}%
\pgfpathrectangle{\pgfqpoint{0.946717in}{0.663635in}}{\pgfqpoint{1.826010in}{1.541693in}}%
\pgfusepath{clip}%
\pgfsetbuttcap%
\pgfsetroundjoin%
\pgfsetlinewidth{1.505625pt}%
\definecolor{currentstroke}{rgb}{1.000000,0.647059,0.000000}%
\pgfsetstrokecolor{currentstroke}%
\pgfsetdash{{1.500000pt}{2.475000pt}}{0.000000pt}%
\pgfpathmoveto{\pgfqpoint{2.400498in}{0.663635in}}%
\pgfpathlineto{\pgfqpoint{2.400498in}{2.205329in}}%
\pgfusepath{stroke}%
\end{pgfscope}%
\begin{pgfscope}%
\pgfpathrectangle{\pgfqpoint{0.946717in}{0.663635in}}{\pgfqpoint{1.826010in}{1.541693in}}%
\pgfusepath{clip}%
\pgfsetroundcap%
\pgfsetroundjoin%
\pgfsetlinewidth{1.505625pt}%
\definecolor{currentstroke}{rgb}{0.298039,0.447059,0.690196}%
\pgfsetstrokecolor{currentstroke}%
\pgfsetdash{}{0pt}%
\pgfpathmoveto{\pgfqpoint{1.029717in}{0.866259in}}%
\pgfpathlineto{\pgfqpoint{1.032263in}{0.862796in}}%
\pgfpathlineto{\pgfqpoint{1.050085in}{0.862716in}}%
\pgfpathlineto{\pgfqpoint{1.052631in}{0.865894in}}%
\pgfpathlineto{\pgfqpoint{1.121374in}{0.865024in}}%
\pgfpathlineto{\pgfqpoint{1.123920in}{0.863109in}}%
\pgfpathlineto{\pgfqpoint{1.131558in}{0.863109in}}%
\pgfpathlineto{\pgfqpoint{1.134104in}{0.856711in}}%
\pgfpathlineto{\pgfqpoint{1.141742in}{0.856711in}}%
\pgfpathlineto{\pgfqpoint{1.144288in}{0.865146in}}%
\pgfpathlineto{\pgfqpoint{1.182479in}{0.865396in}}%
\pgfpathlineto{\pgfqpoint{1.185025in}{0.862536in}}%
\pgfpathlineto{\pgfqpoint{1.213031in}{0.863623in}}%
\pgfpathlineto{\pgfqpoint{1.215577in}{0.865027in}}%
\pgfpathlineto{\pgfqpoint{1.223215in}{0.865027in}}%
\pgfpathlineto{\pgfqpoint{1.225761in}{0.868627in}}%
\pgfpathlineto{\pgfqpoint{1.233399in}{0.868627in}}%
\pgfpathlineto{\pgfqpoint{1.235945in}{0.892589in}}%
\pgfpathlineto{\pgfqpoint{1.243583in}{0.892589in}}%
\pgfpathlineto{\pgfqpoint{1.246129in}{0.886992in}}%
\pgfpathlineto{\pgfqpoint{1.253768in}{0.886992in}}%
\pgfpathlineto{\pgfqpoint{1.256314in}{0.880761in}}%
\pgfpathlineto{\pgfqpoint{1.274136in}{0.881468in}}%
\pgfpathlineto{\pgfqpoint{1.276682in}{0.873684in}}%
\pgfpathlineto{\pgfqpoint{1.284320in}{0.873684in}}%
\pgfpathlineto{\pgfqpoint{1.286866in}{0.866329in}}%
\pgfpathlineto{\pgfqpoint{1.294504in}{0.866329in}}%
\pgfpathlineto{\pgfqpoint{1.297050in}{0.869762in}}%
\pgfpathlineto{\pgfqpoint{1.304688in}{0.869762in}}%
\pgfpathlineto{\pgfqpoint{1.307234in}{0.874550in}}%
\pgfpathlineto{\pgfqpoint{1.314872in}{0.874550in}}%
\pgfpathlineto{\pgfqpoint{1.317418in}{0.870937in}}%
\pgfpathlineto{\pgfqpoint{1.325056in}{0.870937in}}%
\pgfpathlineto{\pgfqpoint{1.327602in}{0.872898in}}%
\pgfpathlineto{\pgfqpoint{1.335240in}{0.872898in}}%
\pgfpathlineto{\pgfqpoint{1.337786in}{0.891206in}}%
\pgfpathlineto{\pgfqpoint{1.345424in}{0.891206in}}%
\pgfpathlineto{\pgfqpoint{1.347971in}{0.886765in}}%
\pgfpathlineto{\pgfqpoint{1.355609in}{0.886765in}}%
\pgfpathlineto{\pgfqpoint{1.358155in}{0.890076in}}%
\pgfpathlineto{\pgfqpoint{1.365793in}{0.890076in}}%
\pgfpathlineto{\pgfqpoint{1.368339in}{0.935678in}}%
\pgfpathlineto{\pgfqpoint{1.375977in}{0.935678in}}%
\pgfpathlineto{\pgfqpoint{1.378523in}{0.944991in}}%
\pgfpathlineto{\pgfqpoint{1.386161in}{0.944991in}}%
\pgfpathlineto{\pgfqpoint{1.388707in}{0.947885in}}%
\pgfpathlineto{\pgfqpoint{1.396345in}{0.947885in}}%
\pgfpathlineto{\pgfqpoint{1.398891in}{0.956280in}}%
\pgfpathlineto{\pgfqpoint{1.406529in}{0.956280in}}%
\pgfpathlineto{\pgfqpoint{1.409075in}{0.970287in}}%
\pgfpathlineto{\pgfqpoint{1.416713in}{0.970287in}}%
\pgfpathlineto{\pgfqpoint{1.419259in}{0.906994in}}%
\pgfpathlineto{\pgfqpoint{1.426897in}{0.906994in}}%
\pgfpathlineto{\pgfqpoint{1.429443in}{1.008196in}}%
\pgfpathlineto{\pgfqpoint{1.437081in}{1.008196in}}%
\pgfpathlineto{\pgfqpoint{1.439627in}{0.915053in}}%
\pgfpathlineto{\pgfqpoint{1.447266in}{0.915053in}}%
\pgfpathlineto{\pgfqpoint{1.449812in}{1.043680in}}%
\pgfpathlineto{\pgfqpoint{1.457450in}{1.043680in}}%
\pgfpathlineto{\pgfqpoint{1.459996in}{1.064998in}}%
\pgfpathlineto{\pgfqpoint{1.467634in}{1.064998in}}%
\pgfpathlineto{\pgfqpoint{1.470180in}{1.079168in}}%
\pgfpathlineto{\pgfqpoint{1.477818in}{1.079168in}}%
\pgfpathlineto{\pgfqpoint{1.480364in}{1.082047in}}%
\pgfpathlineto{\pgfqpoint{1.498186in}{1.082969in}}%
\pgfpathlineto{\pgfqpoint{1.500732in}{0.995310in}}%
\pgfpathlineto{\pgfqpoint{1.508370in}{0.995310in}}%
\pgfpathlineto{\pgfqpoint{1.510916in}{1.055334in}}%
\pgfpathlineto{\pgfqpoint{1.518554in}{1.055334in}}%
\pgfpathlineto{\pgfqpoint{1.521100in}{1.039204in}}%
\pgfpathlineto{\pgfqpoint{1.528738in}{1.039204in}}%
\pgfpathlineto{\pgfqpoint{1.531284in}{1.020840in}}%
\pgfpathlineto{\pgfqpoint{1.538923in}{1.020840in}}%
\pgfpathlineto{\pgfqpoint{1.541469in}{1.015728in}}%
\pgfpathlineto{\pgfqpoint{1.549107in}{1.015728in}}%
\pgfpathlineto{\pgfqpoint{1.551653in}{0.991635in}}%
\pgfpathlineto{\pgfqpoint{1.559291in}{0.991635in}}%
\pgfpathlineto{\pgfqpoint{1.561837in}{0.978089in}}%
\pgfpathlineto{\pgfqpoint{1.569475in}{0.978089in}}%
\pgfpathlineto{\pgfqpoint{1.572021in}{0.974174in}}%
\pgfpathlineto{\pgfqpoint{1.579659in}{0.974174in}}%
\pgfpathlineto{\pgfqpoint{1.582205in}{0.945379in}}%
\pgfpathlineto{\pgfqpoint{1.589843in}{0.945379in}}%
\pgfpathlineto{\pgfqpoint{1.592389in}{0.943138in}}%
\pgfpathlineto{\pgfqpoint{1.610211in}{0.942662in}}%
\pgfpathlineto{\pgfqpoint{1.612757in}{0.952734in}}%
\pgfpathlineto{\pgfqpoint{1.620395in}{0.952734in}}%
\pgfpathlineto{\pgfqpoint{1.622941in}{0.947795in}}%
\pgfpathlineto{\pgfqpoint{1.630579in}{0.947795in}}%
\pgfpathlineto{\pgfqpoint{1.633126in}{0.912657in}}%
\pgfpathlineto{\pgfqpoint{1.640764in}{0.912657in}}%
\pgfpathlineto{\pgfqpoint{1.643310in}{0.906420in}}%
\pgfpathlineto{\pgfqpoint{1.650948in}{0.906420in}}%
\pgfpathlineto{\pgfqpoint{1.653494in}{0.959943in}}%
\pgfpathlineto{\pgfqpoint{1.661132in}{0.959943in}}%
\pgfpathlineto{\pgfqpoint{1.663678in}{0.918051in}}%
\pgfpathlineto{\pgfqpoint{1.671316in}{0.918051in}}%
\pgfpathlineto{\pgfqpoint{1.673862in}{0.912174in}}%
\pgfpathlineto{\pgfqpoint{1.681500in}{0.912174in}}%
\pgfpathlineto{\pgfqpoint{1.684046in}{0.909091in}}%
\pgfpathlineto{\pgfqpoint{1.691684in}{0.909091in}}%
\pgfpathlineto{\pgfqpoint{1.694230in}{0.906363in}}%
\pgfpathlineto{\pgfqpoint{1.701868in}{0.906363in}}%
\pgfpathlineto{\pgfqpoint{1.704414in}{0.908082in}}%
\pgfpathlineto{\pgfqpoint{1.712052in}{0.908082in}}%
\pgfpathlineto{\pgfqpoint{1.714598in}{0.909609in}}%
\pgfpathlineto{\pgfqpoint{1.722236in}{0.909609in}}%
\pgfpathlineto{\pgfqpoint{1.724782in}{0.941701in}}%
\pgfpathlineto{\pgfqpoint{1.732421in}{0.941701in}}%
\pgfpathlineto{\pgfqpoint{1.734967in}{0.922381in}}%
\pgfpathlineto{\pgfqpoint{1.742605in}{0.922381in}}%
\pgfpathlineto{\pgfqpoint{1.745151in}{0.916257in}}%
\pgfpathlineto{\pgfqpoint{1.752789in}{0.916257in}}%
\pgfpathlineto{\pgfqpoint{1.755335in}{0.913047in}}%
\pgfpathlineto{\pgfqpoint{1.762973in}{0.913047in}}%
\pgfpathlineto{\pgfqpoint{1.765519in}{0.911260in}}%
\pgfpathlineto{\pgfqpoint{1.783341in}{0.910332in}}%
\pgfpathlineto{\pgfqpoint{1.785887in}{0.905027in}}%
\pgfpathlineto{\pgfqpoint{1.793525in}{0.905027in}}%
\pgfpathlineto{\pgfqpoint{1.796071in}{0.903241in}}%
\pgfpathlineto{\pgfqpoint{1.803709in}{0.903241in}}%
\pgfpathlineto{\pgfqpoint{1.806255in}{0.906615in}}%
\pgfpathlineto{\pgfqpoint{1.824078in}{0.906523in}}%
\pgfpathlineto{\pgfqpoint{1.826624in}{0.902718in}}%
\pgfpathlineto{\pgfqpoint{1.834262in}{0.902718in}}%
\pgfpathlineto{\pgfqpoint{1.836808in}{0.914739in}}%
\pgfpathlineto{\pgfqpoint{1.844446in}{0.914739in}}%
\pgfpathlineto{\pgfqpoint{1.846992in}{0.926129in}}%
\pgfpathlineto{\pgfqpoint{1.854630in}{0.926129in}}%
\pgfpathlineto{\pgfqpoint{1.857176in}{0.932991in}}%
\pgfpathlineto{\pgfqpoint{1.864814in}{0.932991in}}%
\pgfpathlineto{\pgfqpoint{1.867360in}{0.944601in}}%
\pgfpathlineto{\pgfqpoint{1.874998in}{0.944601in}}%
\pgfpathlineto{\pgfqpoint{1.877544in}{0.955987in}}%
\pgfpathlineto{\pgfqpoint{1.885182in}{0.955987in}}%
\pgfpathlineto{\pgfqpoint{1.887728in}{0.938362in}}%
\pgfpathlineto{\pgfqpoint{1.895366in}{0.938362in}}%
\pgfpathlineto{\pgfqpoint{1.897912in}{0.975177in}}%
\pgfpathlineto{\pgfqpoint{1.905550in}{0.975177in}}%
\pgfpathlineto{\pgfqpoint{1.908096in}{0.986757in}}%
\pgfpathlineto{\pgfqpoint{1.915735in}{0.986757in}}%
\pgfpathlineto{\pgfqpoint{1.918281in}{0.997830in}}%
\pgfpathlineto{\pgfqpoint{1.925919in}{0.997830in}}%
\pgfpathlineto{\pgfqpoint{1.928465in}{1.007275in}}%
\pgfpathlineto{\pgfqpoint{1.936103in}{1.007275in}}%
\pgfpathlineto{\pgfqpoint{1.938649in}{0.961563in}}%
\pgfpathlineto{\pgfqpoint{1.946287in}{0.961563in}}%
\pgfpathlineto{\pgfqpoint{1.948833in}{0.964367in}}%
\pgfpathlineto{\pgfqpoint{1.956471in}{0.964367in}}%
\pgfpathlineto{\pgfqpoint{1.959017in}{1.038433in}}%
\pgfpathlineto{\pgfqpoint{1.966655in}{1.038433in}}%
\pgfpathlineto{\pgfqpoint{1.969201in}{1.051416in}}%
\pgfpathlineto{\pgfqpoint{1.987023in}{1.052222in}}%
\pgfpathlineto{\pgfqpoint{1.989569in}{0.984802in}}%
\pgfpathlineto{\pgfqpoint{1.997207in}{0.984802in}}%
\pgfpathlineto{\pgfqpoint{1.999753in}{1.054291in}}%
\pgfpathlineto{\pgfqpoint{2.027760in}{1.054884in}}%
\pgfpathlineto{\pgfqpoint{2.030306in}{1.050929in}}%
\pgfpathlineto{\pgfqpoint{2.048128in}{1.049883in}}%
\pgfpathlineto{\pgfqpoint{2.050674in}{1.057242in}}%
\pgfpathlineto{\pgfqpoint{2.058312in}{1.057242in}}%
\pgfpathlineto{\pgfqpoint{2.060858in}{1.022536in}}%
\pgfpathlineto{\pgfqpoint{2.068496in}{1.022536in}}%
\pgfpathlineto{\pgfqpoint{2.071042in}{1.057241in}}%
\pgfpathlineto{\pgfqpoint{2.088864in}{1.056601in}}%
\pgfpathlineto{\pgfqpoint{2.091410in}{1.044251in}}%
\pgfpathlineto{\pgfqpoint{2.099048in}{1.044251in}}%
\pgfpathlineto{\pgfqpoint{2.101594in}{1.046028in}}%
\pgfpathlineto{\pgfqpoint{2.109233in}{1.046028in}}%
\pgfpathlineto{\pgfqpoint{2.111779in}{1.044659in}}%
\pgfpathlineto{\pgfqpoint{2.119417in}{1.044659in}}%
\pgfpathlineto{\pgfqpoint{2.121963in}{1.027114in}}%
\pgfpathlineto{\pgfqpoint{2.129601in}{1.027114in}}%
\pgfpathlineto{\pgfqpoint{2.132147in}{1.018788in}}%
\pgfpathlineto{\pgfqpoint{2.139785in}{1.018788in}}%
\pgfpathlineto{\pgfqpoint{2.142331in}{1.013025in}}%
\pgfpathlineto{\pgfqpoint{2.149969in}{1.013025in}}%
\pgfpathlineto{\pgfqpoint{2.152515in}{1.002854in}}%
\pgfpathlineto{\pgfqpoint{2.160153in}{1.002854in}}%
\pgfpathlineto{\pgfqpoint{2.162699in}{0.954696in}}%
\pgfpathlineto{\pgfqpoint{2.170337in}{0.954696in}}%
\pgfpathlineto{\pgfqpoint{2.172883in}{0.972913in}}%
\pgfpathlineto{\pgfqpoint{2.180521in}{0.972913in}}%
\pgfpathlineto{\pgfqpoint{2.183067in}{0.958554in}}%
\pgfpathlineto{\pgfqpoint{2.190705in}{0.958554in}}%
\pgfpathlineto{\pgfqpoint{2.193251in}{0.987772in}}%
\pgfpathlineto{\pgfqpoint{2.200890in}{0.987772in}}%
\pgfpathlineto{\pgfqpoint{2.203436in}{0.947067in}}%
\pgfpathlineto{\pgfqpoint{2.211074in}{0.947067in}}%
\pgfpathlineto{\pgfqpoint{2.213620in}{0.971514in}}%
\pgfpathlineto{\pgfqpoint{2.221258in}{0.971514in}}%
\pgfpathlineto{\pgfqpoint{2.223804in}{0.928224in}}%
\pgfpathlineto{\pgfqpoint{2.231442in}{0.928224in}}%
\pgfpathlineto{\pgfqpoint{2.233988in}{0.961534in}}%
\pgfpathlineto{\pgfqpoint{2.241626in}{0.961534in}}%
\pgfpathlineto{\pgfqpoint{2.244172in}{0.956379in}}%
\pgfpathlineto{\pgfqpoint{2.251810in}{0.956379in}}%
\pgfpathlineto{\pgfqpoint{2.254356in}{0.950527in}}%
\pgfpathlineto{\pgfqpoint{2.261994in}{0.950527in}}%
\pgfpathlineto{\pgfqpoint{2.264540in}{0.913183in}}%
\pgfpathlineto{\pgfqpoint{2.272178in}{0.913183in}}%
\pgfpathlineto{\pgfqpoint{2.274724in}{0.910900in}}%
\pgfpathlineto{\pgfqpoint{2.292546in}{0.911098in}}%
\pgfpathlineto{\pgfqpoint{2.295093in}{0.916662in}}%
\pgfpathlineto{\pgfqpoint{2.312915in}{0.917655in}}%
\pgfpathlineto{\pgfqpoint{2.315461in}{0.922202in}}%
\pgfpathlineto{\pgfqpoint{2.323099in}{0.922202in}}%
\pgfpathlineto{\pgfqpoint{2.325645in}{0.930061in}}%
\pgfpathlineto{\pgfqpoint{2.333283in}{0.930061in}}%
\pgfpathlineto{\pgfqpoint{2.335829in}{0.941220in}}%
\pgfpathlineto{\pgfqpoint{2.343467in}{0.941220in}}%
\pgfpathlineto{\pgfqpoint{2.346013in}{0.951623in}}%
\pgfpathlineto{\pgfqpoint{2.353651in}{0.951623in}}%
\pgfpathlineto{\pgfqpoint{2.356197in}{0.961916in}}%
\pgfpathlineto{\pgfqpoint{2.363835in}{0.961916in}}%
\pgfpathlineto{\pgfqpoint{2.366381in}{0.977231in}}%
\pgfpathlineto{\pgfqpoint{2.374019in}{0.977231in}}%
\pgfpathlineto{\pgfqpoint{2.376565in}{0.982889in}}%
\pgfpathlineto{\pgfqpoint{2.384203in}{0.982889in}}%
\pgfpathlineto{\pgfqpoint{2.386749in}{0.993027in}}%
\pgfpathlineto{\pgfqpoint{2.394388in}{0.993027in}}%
\pgfpathlineto{\pgfqpoint{2.396934in}{1.006080in}}%
\pgfpathlineto{\pgfqpoint{2.404572in}{1.006080in}}%
\pgfpathlineto{\pgfqpoint{2.407118in}{1.010945in}}%
\pgfpathlineto{\pgfqpoint{2.414756in}{1.010945in}}%
\pgfpathlineto{\pgfqpoint{2.417302in}{0.963347in}}%
\pgfpathlineto{\pgfqpoint{2.424940in}{0.963347in}}%
\pgfpathlineto{\pgfqpoint{2.427486in}{0.969372in}}%
\pgfpathlineto{\pgfqpoint{2.435124in}{0.969372in}}%
\pgfpathlineto{\pgfqpoint{2.437670in}{1.033217in}}%
\pgfpathlineto{\pgfqpoint{2.445308in}{1.033217in}}%
\pgfpathlineto{\pgfqpoint{2.447854in}{1.035497in}}%
\pgfpathlineto{\pgfqpoint{2.465676in}{1.035738in}}%
\pgfpathlineto{\pgfqpoint{2.468222in}{1.034145in}}%
\pgfpathlineto{\pgfqpoint{2.475860in}{1.034145in}}%
\pgfpathlineto{\pgfqpoint{2.478406in}{1.030803in}}%
\pgfpathlineto{\pgfqpoint{2.486045in}{1.030803in}}%
\pgfpathlineto{\pgfqpoint{2.488591in}{1.021082in}}%
\pgfpathlineto{\pgfqpoint{2.496229in}{1.021082in}}%
\pgfpathlineto{\pgfqpoint{2.498775in}{1.017811in}}%
\pgfpathlineto{\pgfqpoint{2.506413in}{1.017811in}}%
\pgfpathlineto{\pgfqpoint{2.508959in}{1.007034in}}%
\pgfpathlineto{\pgfqpoint{2.516597in}{1.007034in}}%
\pgfpathlineto{\pgfqpoint{2.519143in}{0.994367in}}%
\pgfpathlineto{\pgfqpoint{2.526781in}{0.994367in}}%
\pgfpathlineto{\pgfqpoint{2.529327in}{0.986386in}}%
\pgfpathlineto{\pgfqpoint{2.536965in}{0.986386in}}%
\pgfpathlineto{\pgfqpoint{2.539511in}{0.978107in}}%
\pgfpathlineto{\pgfqpoint{2.547149in}{0.978107in}}%
\pgfpathlineto{\pgfqpoint{2.549695in}{0.986617in}}%
\pgfpathlineto{\pgfqpoint{2.567517in}{0.986297in}}%
\pgfpathlineto{\pgfqpoint{2.570063in}{0.940975in}}%
\pgfpathlineto{\pgfqpoint{2.577701in}{0.940975in}}%
\pgfpathlineto{\pgfqpoint{2.580248in}{0.930225in}}%
\pgfpathlineto{\pgfqpoint{2.587886in}{0.930225in}}%
\pgfpathlineto{\pgfqpoint{2.590432in}{0.980214in}}%
\pgfpathlineto{\pgfqpoint{2.598070in}{0.980214in}}%
\pgfpathlineto{\pgfqpoint{2.600616in}{0.914228in}}%
\pgfpathlineto{\pgfqpoint{2.608254in}{0.914228in}}%
\pgfpathlineto{\pgfqpoint{2.610800in}{0.969652in}}%
\pgfpathlineto{\pgfqpoint{2.618438in}{0.969652in}}%
\pgfpathlineto{\pgfqpoint{2.620984in}{0.894887in}}%
\pgfpathlineto{\pgfqpoint{2.628622in}{0.894887in}}%
\pgfpathlineto{\pgfqpoint{2.631168in}{0.959945in}}%
\pgfpathlineto{\pgfqpoint{2.638806in}{0.959945in}}%
\pgfpathlineto{\pgfqpoint{2.641352in}{0.885372in}}%
\pgfpathlineto{\pgfqpoint{2.648990in}{0.885372in}}%
\pgfpathlineto{\pgfqpoint{2.651536in}{0.889654in}}%
\pgfpathlineto{\pgfqpoint{2.669358in}{0.889336in}}%
\pgfpathlineto{\pgfqpoint{2.671904in}{0.939182in}}%
\pgfpathlineto{\pgfqpoint{2.679543in}{0.939182in}}%
\pgfpathlineto{\pgfqpoint{2.682089in}{0.890077in}}%
\pgfpathlineto{\pgfqpoint{2.689727in}{0.890077in}}%
\pgfpathlineto{\pgfqpoint{2.689727in}{0.890077in}}%
\pgfusepath{stroke}%
\end{pgfscope}%
\begin{pgfscope}%
\pgfpathrectangle{\pgfqpoint{0.946717in}{0.663635in}}{\pgfqpoint{1.826010in}{1.541693in}}%
\pgfusepath{clip}%
\pgfsetbuttcap%
\pgfsetroundjoin%
\pgfsetlinewidth{1.505625pt}%
\definecolor{currentstroke}{rgb}{0.580392,0.403922,0.741176}%
\pgfsetstrokecolor{currentstroke}%
\pgfsetdash{{5.550000pt}{2.400000pt}}{0.000000pt}%
\pgfpathmoveto{\pgfqpoint{0.946717in}{1.562957in}}%
\pgfpathlineto{\pgfqpoint{2.772727in}{1.562957in}}%
\pgfusepath{stroke}%
\end{pgfscope}%
\begin{pgfscope}%
\pgfsetrectcap%
\pgfsetmiterjoin%
\pgfsetlinewidth{1.254687pt}%
\definecolor{currentstroke}{rgb}{1.000000,1.000000,1.000000}%
\pgfsetstrokecolor{currentstroke}%
\pgfsetdash{}{0pt}%
\pgfpathmoveto{\pgfqpoint{0.946717in}{0.663635in}}%
\pgfpathlineto{\pgfqpoint{0.946717in}{2.205329in}}%
\pgfusepath{stroke}%
\end{pgfscope}%
\begin{pgfscope}%
\pgfsetrectcap%
\pgfsetmiterjoin%
\pgfsetlinewidth{1.254687pt}%
\definecolor{currentstroke}{rgb}{1.000000,1.000000,1.000000}%
\pgfsetstrokecolor{currentstroke}%
\pgfsetdash{}{0pt}%
\pgfpathmoveto{\pgfqpoint{2.772727in}{0.663635in}}%
\pgfpathlineto{\pgfqpoint{2.772727in}{2.205329in}}%
\pgfusepath{stroke}%
\end{pgfscope}%
\begin{pgfscope}%
\pgfsetrectcap%
\pgfsetmiterjoin%
\pgfsetlinewidth{1.254687pt}%
\definecolor{currentstroke}{rgb}{1.000000,1.000000,1.000000}%
\pgfsetstrokecolor{currentstroke}%
\pgfsetdash{}{0pt}%
\pgfpathmoveto{\pgfqpoint{0.946717in}{0.663635in}}%
\pgfpathlineto{\pgfqpoint{2.772727in}{0.663635in}}%
\pgfusepath{stroke}%
\end{pgfscope}%
\begin{pgfscope}%
\pgfsetrectcap%
\pgfsetmiterjoin%
\pgfsetlinewidth{1.254687pt}%
\definecolor{currentstroke}{rgb}{1.000000,1.000000,1.000000}%
\pgfsetstrokecolor{currentstroke}%
\pgfsetdash{}{0pt}%
\pgfpathmoveto{\pgfqpoint{0.946717in}{2.205329in}}%
\pgfpathlineto{\pgfqpoint{2.772727in}{2.205329in}}%
\pgfusepath{stroke}%
\end{pgfscope}%
\begin{pgfscope}%
\definecolor{textcolor}{rgb}{0.150000,0.150000,0.150000}%
\pgfsetstrokecolor{textcolor}%
\pgfsetfillcolor{textcolor}%
\pgftext[x=1.859722in,y=2.288662in,,base]{\color{textcolor}{\sffamily\fontsize{12.000000}{14.400000}\selectfont\catcode`\^=\active\def^{\ifmmode\sp\else\^{}\fi}\catcode`\%=\active\def%{\%}c)}}%
\end{pgfscope}%
\begin{pgfscope}%
\pgfsetbuttcap%
\pgfsetmiterjoin%
\definecolor{currentfill}{rgb}{0.917647,0.917647,0.949020}%
\pgfsetfillcolor{currentfill}%
\pgfsetlinewidth{0.000000pt}%
\definecolor{currentstroke}{rgb}{0.000000,0.000000,0.000000}%
\pgfsetstrokecolor{currentstroke}%
\pgfsetstrokeopacity{0.000000}%
\pgfsetdash{}{0pt}%
\pgfpathmoveto{\pgfqpoint{3.593990in}{0.663635in}}%
\pgfpathlineto{\pgfqpoint{5.420000in}{0.663635in}}%
\pgfpathlineto{\pgfqpoint{5.420000in}{2.205329in}}%
\pgfpathlineto{\pgfqpoint{3.593990in}{2.205329in}}%
\pgfpathlineto{\pgfqpoint{3.593990in}{0.663635in}}%
\pgfpathclose%
\pgfusepath{fill}%
\end{pgfscope}%
\begin{pgfscope}%
\pgfpathrectangle{\pgfqpoint{3.593990in}{0.663635in}}{\pgfqpoint{1.826010in}{1.541693in}}%
\pgfusepath{clip}%
\pgfsetroundcap%
\pgfsetroundjoin%
\pgfsetlinewidth{1.003750pt}%
\definecolor{currentstroke}{rgb}{1.000000,1.000000,1.000000}%
\pgfsetstrokecolor{currentstroke}%
\pgfsetdash{}{0pt}%
\pgfpathmoveto{\pgfqpoint{3.676990in}{0.663635in}}%
\pgfpathlineto{\pgfqpoint{3.676990in}{2.205329in}}%
\pgfusepath{stroke}%
\end{pgfscope}%
\begin{pgfscope}%
\definecolor{textcolor}{rgb}{0.150000,0.150000,0.150000}%
\pgfsetstrokecolor{textcolor}%
\pgfsetfillcolor{textcolor}%
\pgftext[x=3.676990in,y=0.531691in,,top]{\color{textcolor}{\sffamily\fontsize{11.000000}{13.200000}\selectfont\catcode`\^=\active\def^{\ifmmode\sp\else\^{}\fi}\catcode`\%=\active\def%{\%}0}}%
\end{pgfscope}%
\begin{pgfscope}%
\pgfpathrectangle{\pgfqpoint{3.593990in}{0.663635in}}{\pgfqpoint{1.826010in}{1.541693in}}%
\pgfusepath{clip}%
\pgfsetroundcap%
\pgfsetroundjoin%
\pgfsetlinewidth{1.003750pt}%
\definecolor{currentstroke}{rgb}{1.000000,1.000000,1.000000}%
\pgfsetstrokecolor{currentstroke}%
\pgfsetdash{}{0pt}%
\pgfpathmoveto{\pgfqpoint{4.695401in}{0.663635in}}%
\pgfpathlineto{\pgfqpoint{4.695401in}{2.205329in}}%
\pgfusepath{stroke}%
\end{pgfscope}%
\begin{pgfscope}%
\definecolor{textcolor}{rgb}{0.150000,0.150000,0.150000}%
\pgfsetstrokecolor{textcolor}%
\pgfsetfillcolor{textcolor}%
\pgftext[x=4.695401in,y=0.531691in,,top]{\color{textcolor}{\sffamily\fontsize{11.000000}{13.200000}\selectfont\catcode`\^=\active\def^{\ifmmode\sp\else\^{}\fi}\catcode`\%=\active\def%{\%}2000}}%
\end{pgfscope}%
\begin{pgfscope}%
\definecolor{textcolor}{rgb}{0.150000,0.150000,0.150000}%
\pgfsetstrokecolor{textcolor}%
\pgfsetfillcolor{textcolor}%
\pgftext[x=4.506995in,y=0.336413in,,top]{\color{textcolor}{\sffamily\fontsize{12.000000}{14.400000}\selectfont\catcode`\^=\active\def^{\ifmmode\sp\else\^{}\fi}\catcode`\%=\active\def%{\%}Time (s)}}%
\end{pgfscope}%
\begin{pgfscope}%
\pgfpathrectangle{\pgfqpoint{3.593990in}{0.663635in}}{\pgfqpoint{1.826010in}{1.541693in}}%
\pgfusepath{clip}%
\pgfsetroundcap%
\pgfsetroundjoin%
\pgfsetlinewidth{1.003750pt}%
\definecolor{currentstroke}{rgb}{1.000000,1.000000,1.000000}%
\pgfsetstrokecolor{currentstroke}%
\pgfsetdash{}{0pt}%
\pgfpathmoveto{\pgfqpoint{3.593990in}{0.792110in}}%
\pgfpathlineto{\pgfqpoint{5.420000in}{0.792110in}}%
\pgfusepath{stroke}%
\end{pgfscope}%
\begin{pgfscope}%
\definecolor{textcolor}{rgb}{0.150000,0.150000,0.150000}%
\pgfsetstrokecolor{textcolor}%
\pgfsetfillcolor{textcolor}%
\pgftext[x=3.164695in, y=0.737429in, left, base]{\color{textcolor}{\sffamily\fontsize{11.000000}{13.200000}\selectfont\catcode`\^=\active\def^{\ifmmode\sp\else\^{}\fi}\catcode`\%=\active\def%{\%}0.00}}%
\end{pgfscope}%
\begin{pgfscope}%
\pgfpathrectangle{\pgfqpoint{3.593990in}{0.663635in}}{\pgfqpoint{1.826010in}{1.541693in}}%
\pgfusepath{clip}%
\pgfsetroundcap%
\pgfsetroundjoin%
\pgfsetlinewidth{1.003750pt}%
\definecolor{currentstroke}{rgb}{1.000000,1.000000,1.000000}%
\pgfsetstrokecolor{currentstroke}%
\pgfsetdash{}{0pt}%
\pgfpathmoveto{\pgfqpoint{3.593990in}{1.113296in}}%
\pgfpathlineto{\pgfqpoint{5.420000in}{1.113296in}}%
\pgfusepath{stroke}%
\end{pgfscope}%
\begin{pgfscope}%
\definecolor{textcolor}{rgb}{0.150000,0.150000,0.150000}%
\pgfsetstrokecolor{textcolor}%
\pgfsetfillcolor{textcolor}%
\pgftext[x=3.164695in, y=1.058615in, left, base]{\color{textcolor}{\sffamily\fontsize{11.000000}{13.200000}\selectfont\catcode`\^=\active\def^{\ifmmode\sp\else\^{}\fi}\catcode`\%=\active\def%{\%}0.25}}%
\end{pgfscope}%
\begin{pgfscope}%
\pgfpathrectangle{\pgfqpoint{3.593990in}{0.663635in}}{\pgfqpoint{1.826010in}{1.541693in}}%
\pgfusepath{clip}%
\pgfsetroundcap%
\pgfsetroundjoin%
\pgfsetlinewidth{1.003750pt}%
\definecolor{currentstroke}{rgb}{1.000000,1.000000,1.000000}%
\pgfsetstrokecolor{currentstroke}%
\pgfsetdash{}{0pt}%
\pgfpathmoveto{\pgfqpoint{3.593990in}{1.434482in}}%
\pgfpathlineto{\pgfqpoint{5.420000in}{1.434482in}}%
\pgfusepath{stroke}%
\end{pgfscope}%
\begin{pgfscope}%
\definecolor{textcolor}{rgb}{0.150000,0.150000,0.150000}%
\pgfsetstrokecolor{textcolor}%
\pgfsetfillcolor{textcolor}%
\pgftext[x=3.164695in, y=1.379801in, left, base]{\color{textcolor}{\sffamily\fontsize{11.000000}{13.200000}\selectfont\catcode`\^=\active\def^{\ifmmode\sp\else\^{}\fi}\catcode`\%=\active\def%{\%}0.50}}%
\end{pgfscope}%
\begin{pgfscope}%
\pgfpathrectangle{\pgfqpoint{3.593990in}{0.663635in}}{\pgfqpoint{1.826010in}{1.541693in}}%
\pgfusepath{clip}%
\pgfsetroundcap%
\pgfsetroundjoin%
\pgfsetlinewidth{1.003750pt}%
\definecolor{currentstroke}{rgb}{1.000000,1.000000,1.000000}%
\pgfsetstrokecolor{currentstroke}%
\pgfsetdash{}{0pt}%
\pgfpathmoveto{\pgfqpoint{3.593990in}{1.755668in}}%
\pgfpathlineto{\pgfqpoint{5.420000in}{1.755668in}}%
\pgfusepath{stroke}%
\end{pgfscope}%
\begin{pgfscope}%
\definecolor{textcolor}{rgb}{0.150000,0.150000,0.150000}%
\pgfsetstrokecolor{textcolor}%
\pgfsetfillcolor{textcolor}%
\pgftext[x=3.164695in, y=1.700987in, left, base]{\color{textcolor}{\sffamily\fontsize{11.000000}{13.200000}\selectfont\catcode`\^=\active\def^{\ifmmode\sp\else\^{}\fi}\catcode`\%=\active\def%{\%}0.75}}%
\end{pgfscope}%
\begin{pgfscope}%
\pgfpathrectangle{\pgfqpoint{3.593990in}{0.663635in}}{\pgfqpoint{1.826010in}{1.541693in}}%
\pgfusepath{clip}%
\pgfsetroundcap%
\pgfsetroundjoin%
\pgfsetlinewidth{1.003750pt}%
\definecolor{currentstroke}{rgb}{1.000000,1.000000,1.000000}%
\pgfsetstrokecolor{currentstroke}%
\pgfsetdash{}{0pt}%
\pgfpathmoveto{\pgfqpoint{3.593990in}{2.076854in}}%
\pgfpathlineto{\pgfqpoint{5.420000in}{2.076854in}}%
\pgfusepath{stroke}%
\end{pgfscope}%
\begin{pgfscope}%
\definecolor{textcolor}{rgb}{0.150000,0.150000,0.150000}%
\pgfsetstrokecolor{textcolor}%
\pgfsetfillcolor{textcolor}%
\pgftext[x=3.164695in, y=2.022174in, left, base]{\color{textcolor}{\sffamily\fontsize{11.000000}{13.200000}\selectfont\catcode`\^=\active\def^{\ifmmode\sp\else\^{}\fi}\catcode`\%=\active\def%{\%}1.00}}%
\end{pgfscope}%
\begin{pgfscope}%
\definecolor{textcolor}{rgb}{0.150000,0.150000,0.150000}%
\pgfsetstrokecolor{textcolor}%
\pgfsetfillcolor{textcolor}%
\pgftext[x=3.109140in,y=1.434482in,,bottom,rotate=90.000000]{\color{textcolor}{\sffamily\fontsize{12.000000}{14.400000}\selectfont\catcode`\^=\active\def^{\ifmmode\sp\else\^{}\fi}\catcode`\%=\active\def%{\%}Memory Utilization (%)}}%
\end{pgfscope}%
\begin{pgfscope}%
\pgfpathrectangle{\pgfqpoint{3.593990in}{0.663635in}}{\pgfqpoint{1.826010in}{1.541693in}}%
\pgfusepath{clip}%
\pgfsetbuttcap%
\pgfsetmiterjoin%
\definecolor{currentfill}{rgb}{1.000000,0.000000,0.000000}%
\pgfsetfillcolor{currentfill}%
\pgfsetfillopacity{0.300000}%
\pgfsetlinewidth{1.003750pt}%
\definecolor{currentstroke}{rgb}{1.000000,0.000000,0.000000}%
\pgfsetstrokecolor{currentstroke}%
\pgfsetstrokeopacity{0.300000}%
\pgfsetdash{}{0pt}%
\pgfpathmoveto{\pgfqpoint{3.847574in}{0.663635in}}%
\pgfpathlineto{\pgfqpoint{3.847574in}{2.205329in}}%
\pgfpathlineto{\pgfqpoint{3.987605in}{2.205329in}}%
\pgfpathlineto{\pgfqpoint{3.987605in}{0.663635in}}%
\pgfpathlineto{\pgfqpoint{3.847574in}{0.663635in}}%
\pgfpathclose%
\pgfusepath{stroke,fill}%
\end{pgfscope}%
\begin{pgfscope}%
\pgfpathrectangle{\pgfqpoint{3.593990in}{0.663635in}}{\pgfqpoint{1.826010in}{1.541693in}}%
\pgfusepath{clip}%
\pgfsetbuttcap%
\pgfsetmiterjoin%
\definecolor{currentfill}{rgb}{1.000000,0.000000,0.000000}%
\pgfsetfillcolor{currentfill}%
\pgfsetfillopacity{0.300000}%
\pgfsetlinewidth{1.003750pt}%
\definecolor{currentstroke}{rgb}{1.000000,0.000000,0.000000}%
\pgfsetstrokecolor{currentstroke}%
\pgfsetstrokeopacity{0.300000}%
\pgfsetdash{}{0pt}%
\pgfpathmoveto{\pgfqpoint{4.211656in}{0.663635in}}%
\pgfpathlineto{\pgfqpoint{4.211656in}{2.205329in}}%
\pgfpathlineto{\pgfqpoint{4.389878in}{2.205329in}}%
\pgfpathlineto{\pgfqpoint{4.389878in}{0.663635in}}%
\pgfpathlineto{\pgfqpoint{4.211656in}{0.663635in}}%
\pgfpathclose%
\pgfusepath{stroke,fill}%
\end{pgfscope}%
\begin{pgfscope}%
\pgfpathrectangle{\pgfqpoint{3.593990in}{0.663635in}}{\pgfqpoint{1.826010in}{1.541693in}}%
\pgfusepath{clip}%
\pgfsetbuttcap%
\pgfsetroundjoin%
\pgfsetlinewidth{1.505625pt}%
\definecolor{currentstroke}{rgb}{1.000000,0.647059,0.000000}%
\pgfsetstrokecolor{currentstroke}%
\pgfsetdash{{1.500000pt}{2.475000pt}}{0.000000pt}%
\pgfpathmoveto{\pgfqpoint{3.821604in}{0.663635in}}%
\pgfpathlineto{\pgfqpoint{3.821604in}{2.205329in}}%
\pgfusepath{stroke}%
\end{pgfscope}%
\begin{pgfscope}%
\pgfpathrectangle{\pgfqpoint{3.593990in}{0.663635in}}{\pgfqpoint{1.826010in}{1.541693in}}%
\pgfusepath{clip}%
\pgfsetbuttcap%
\pgfsetroundjoin%
\pgfsetlinewidth{1.505625pt}%
\definecolor{currentstroke}{rgb}{1.000000,0.647059,0.000000}%
\pgfsetstrokecolor{currentstroke}%
\pgfsetdash{{1.500000pt}{2.475000pt}}{0.000000pt}%
\pgfpathmoveto{\pgfqpoint{4.129674in}{0.663635in}}%
\pgfpathlineto{\pgfqpoint{4.129674in}{2.205329in}}%
\pgfusepath{stroke}%
\end{pgfscope}%
\begin{pgfscope}%
\pgfpathrectangle{\pgfqpoint{3.593990in}{0.663635in}}{\pgfqpoint{1.826010in}{1.541693in}}%
\pgfusepath{clip}%
\pgfsetbuttcap%
\pgfsetroundjoin%
\pgfsetlinewidth{1.505625pt}%
\definecolor{currentstroke}{rgb}{1.000000,0.647059,0.000000}%
\pgfsetstrokecolor{currentstroke}%
\pgfsetdash{{1.500000pt}{2.475000pt}}{0.000000pt}%
\pgfpathmoveto{\pgfqpoint{4.435706in}{0.663635in}}%
\pgfpathlineto{\pgfqpoint{4.435706in}{2.205329in}}%
\pgfusepath{stroke}%
\end{pgfscope}%
\begin{pgfscope}%
\pgfpathrectangle{\pgfqpoint{3.593990in}{0.663635in}}{\pgfqpoint{1.826010in}{1.541693in}}%
\pgfusepath{clip}%
\pgfsetbuttcap%
\pgfsetroundjoin%
\pgfsetlinewidth{1.505625pt}%
\definecolor{currentstroke}{rgb}{1.000000,0.647059,0.000000}%
\pgfsetstrokecolor{currentstroke}%
\pgfsetdash{{1.500000pt}{2.475000pt}}{0.000000pt}%
\pgfpathmoveto{\pgfqpoint{4.742248in}{0.663635in}}%
\pgfpathlineto{\pgfqpoint{4.742248in}{2.205329in}}%
\pgfusepath{stroke}%
\end{pgfscope}%
\begin{pgfscope}%
\pgfpathrectangle{\pgfqpoint{3.593990in}{0.663635in}}{\pgfqpoint{1.826010in}{1.541693in}}%
\pgfusepath{clip}%
\pgfsetbuttcap%
\pgfsetroundjoin%
\pgfsetlinewidth{1.505625pt}%
\definecolor{currentstroke}{rgb}{1.000000,0.647059,0.000000}%
\pgfsetstrokecolor{currentstroke}%
\pgfsetdash{{1.500000pt}{2.475000pt}}{0.000000pt}%
\pgfpathmoveto{\pgfqpoint{5.047771in}{0.663635in}}%
\pgfpathlineto{\pgfqpoint{5.047771in}{2.205329in}}%
\pgfusepath{stroke}%
\end{pgfscope}%
\begin{pgfscope}%
\pgfpathrectangle{\pgfqpoint{3.593990in}{0.663635in}}{\pgfqpoint{1.826010in}{1.541693in}}%
\pgfusepath{clip}%
\pgfsetroundcap%
\pgfsetroundjoin%
\pgfsetlinewidth{1.505625pt}%
\definecolor{currentstroke}{rgb}{0.298039,0.447059,0.690196}%
\pgfsetstrokecolor{currentstroke}%
\pgfsetdash{}{0pt}%
\pgfpathmoveto{\pgfqpoint{3.676990in}{1.456257in}}%
\pgfpathlineto{\pgfqpoint{3.694812in}{1.456482in}}%
\pgfpathlineto{\pgfqpoint{3.697358in}{1.456482in}}%
\pgfpathlineto{\pgfqpoint{3.699904in}{1.454167in}}%
\pgfpathlineto{\pgfqpoint{3.748279in}{1.453987in}}%
\pgfpathlineto{\pgfqpoint{3.750825in}{1.455761in}}%
\pgfpathlineto{\pgfqpoint{3.829752in}{1.455691in}}%
\pgfpathlineto{\pgfqpoint{3.832298in}{1.823900in}}%
\pgfpathlineto{\pgfqpoint{3.839936in}{1.823900in}}%
\pgfpathlineto{\pgfqpoint{3.842482in}{1.826405in}}%
\pgfpathlineto{\pgfqpoint{3.850120in}{1.826405in}}%
\pgfpathlineto{\pgfqpoint{3.851169in}{2.215329in}}%
\pgfpathmoveto{\pgfqpoint{3.984273in}{2.215329in}}%
\pgfpathlineto{\pgfqpoint{3.985059in}{1.905119in}}%
\pgfpathlineto{\pgfqpoint{3.992697in}{1.905119in}}%
\pgfpathlineto{\pgfqpoint{3.995243in}{1.946038in}}%
\pgfpathlineto{\pgfqpoint{4.002881in}{1.946038in}}%
\pgfpathlineto{\pgfqpoint{4.005427in}{1.956183in}}%
\pgfpathlineto{\pgfqpoint{4.013066in}{1.956183in}}%
\pgfpathlineto{\pgfqpoint{4.015612in}{1.954042in}}%
\pgfpathlineto{\pgfqpoint{4.033434in}{1.953436in}}%
\pgfpathlineto{\pgfqpoint{4.035980in}{1.958472in}}%
\pgfpathlineto{\pgfqpoint{4.053802in}{1.957796in}}%
\pgfpathlineto{\pgfqpoint{4.056348in}{1.963118in}}%
\pgfpathlineto{\pgfqpoint{4.063986in}{1.963118in}}%
\pgfpathlineto{\pgfqpoint{4.066532in}{1.977779in}}%
\pgfpathlineto{\pgfqpoint{4.074170in}{1.977779in}}%
\pgfpathlineto{\pgfqpoint{4.076716in}{1.976000in}}%
\pgfpathlineto{\pgfqpoint{4.084354in}{1.976000in}}%
\pgfpathlineto{\pgfqpoint{4.086900in}{1.969789in}}%
\pgfpathlineto{\pgfqpoint{4.094538in}{1.969789in}}%
\pgfpathlineto{\pgfqpoint{4.097084in}{1.980070in}}%
\pgfpathlineto{\pgfqpoint{4.104723in}{1.980070in}}%
\pgfpathlineto{\pgfqpoint{4.107269in}{1.974144in}}%
\pgfpathlineto{\pgfqpoint{4.125091in}{1.974343in}}%
\pgfpathlineto{\pgfqpoint{4.127637in}{1.983816in}}%
\pgfpathlineto{\pgfqpoint{4.135275in}{1.983816in}}%
\pgfpathlineto{\pgfqpoint{4.137821in}{1.852594in}}%
\pgfpathlineto{\pgfqpoint{4.145459in}{1.852594in}}%
\pgfpathlineto{\pgfqpoint{4.148005in}{1.856296in}}%
\pgfpathlineto{\pgfqpoint{4.155643in}{1.856296in}}%
\pgfpathlineto{\pgfqpoint{4.158189in}{1.853202in}}%
\pgfpathlineto{\pgfqpoint{4.165827in}{1.853202in}}%
\pgfpathlineto{\pgfqpoint{4.168373in}{1.849885in}}%
\pgfpathlineto{\pgfqpoint{4.186195in}{1.849886in}}%
\pgfpathlineto{\pgfqpoint{4.191287in}{1.851114in}}%
\pgfpathlineto{\pgfqpoint{4.196379in}{1.851114in}}%
\pgfpathlineto{\pgfqpoint{4.198926in}{1.853070in}}%
\pgfpathlineto{\pgfqpoint{4.206564in}{1.853070in}}%
\pgfpathlineto{\pgfqpoint{4.209110in}{1.848955in}}%
\pgfpathlineto{\pgfqpoint{4.216748in}{1.848955in}}%
\pgfpathlineto{\pgfqpoint{4.218968in}{2.215329in}}%
\pgfpathmoveto{\pgfqpoint{4.390661in}{2.215329in}}%
\pgfpathlineto{\pgfqpoint{4.392424in}{1.623128in}}%
\pgfpathlineto{\pgfqpoint{4.410246in}{1.622524in}}%
\pgfpathlineto{\pgfqpoint{4.412792in}{1.625428in}}%
\pgfpathlineto{\pgfqpoint{4.420430in}{1.625428in}}%
\pgfpathlineto{\pgfqpoint{4.422976in}{1.622397in}}%
\pgfpathlineto{\pgfqpoint{4.440798in}{1.623252in}}%
\pgfpathlineto{\pgfqpoint{4.445890in}{1.624091in}}%
\pgfpathlineto{\pgfqpoint{4.450982in}{1.624091in}}%
\pgfpathlineto{\pgfqpoint{4.453528in}{1.626954in}}%
\pgfpathlineto{\pgfqpoint{4.471350in}{1.626134in}}%
\pgfpathlineto{\pgfqpoint{4.473896in}{1.629369in}}%
\pgfpathlineto{\pgfqpoint{4.481534in}{1.629369in}}%
\pgfpathlineto{\pgfqpoint{4.484081in}{1.720583in}}%
\pgfpathlineto{\pgfqpoint{4.491719in}{1.720583in}}%
\pgfpathlineto{\pgfqpoint{4.494265in}{1.741237in}}%
\pgfpathlineto{\pgfqpoint{4.522271in}{1.740695in}}%
\pgfpathlineto{\pgfqpoint{4.524817in}{1.749738in}}%
\pgfpathlineto{\pgfqpoint{4.532455in}{1.749738in}}%
\pgfpathlineto{\pgfqpoint{4.535001in}{1.754076in}}%
\pgfpathlineto{\pgfqpoint{4.542639in}{1.754076in}}%
\pgfpathlineto{\pgfqpoint{4.545185in}{1.752721in}}%
\pgfpathlineto{\pgfqpoint{4.552823in}{1.752721in}}%
\pgfpathlineto{\pgfqpoint{4.555369in}{1.754723in}}%
\pgfpathlineto{\pgfqpoint{4.563007in}{1.754723in}}%
\pgfpathlineto{\pgfqpoint{4.565553in}{1.756842in}}%
\pgfpathlineto{\pgfqpoint{4.583376in}{1.755963in}}%
\pgfpathlineto{\pgfqpoint{4.585922in}{1.760567in}}%
\pgfpathlineto{\pgfqpoint{4.593560in}{1.760567in}}%
\pgfpathlineto{\pgfqpoint{4.596106in}{1.768671in}}%
\pgfpathlineto{\pgfqpoint{4.603744in}{1.768671in}}%
\pgfpathlineto{\pgfqpoint{4.606290in}{1.763945in}}%
\pgfpathlineto{\pgfqpoint{4.613928in}{1.763945in}}%
\pgfpathlineto{\pgfqpoint{4.616474in}{1.766077in}}%
\pgfpathlineto{\pgfqpoint{4.624112in}{1.766077in}}%
\pgfpathlineto{\pgfqpoint{4.626658in}{1.771198in}}%
\pgfpathlineto{\pgfqpoint{4.634296in}{1.771198in}}%
\pgfpathlineto{\pgfqpoint{4.636842in}{1.769701in}}%
\pgfpathlineto{\pgfqpoint{4.644480in}{1.769701in}}%
\pgfpathlineto{\pgfqpoint{4.647026in}{1.776581in}}%
\pgfpathlineto{\pgfqpoint{4.664848in}{1.777049in}}%
\pgfpathlineto{\pgfqpoint{4.667394in}{1.779755in}}%
\pgfpathlineto{\pgfqpoint{4.675033in}{1.779755in}}%
\pgfpathlineto{\pgfqpoint{4.680125in}{1.778552in}}%
\pgfpathlineto{\pgfqpoint{4.685217in}{1.778552in}}%
\pgfpathlineto{\pgfqpoint{4.690309in}{1.777400in}}%
\pgfpathlineto{\pgfqpoint{4.705585in}{1.776610in}}%
\pgfpathlineto{\pgfqpoint{4.708131in}{1.785847in}}%
\pgfpathlineto{\pgfqpoint{4.715769in}{1.785847in}}%
\pgfpathlineto{\pgfqpoint{4.718315in}{1.777602in}}%
\pgfpathlineto{\pgfqpoint{4.725953in}{1.777602in}}%
\pgfpathlineto{\pgfqpoint{4.728499in}{1.789458in}}%
\pgfpathlineto{\pgfqpoint{4.736137in}{1.789458in}}%
\pgfpathlineto{\pgfqpoint{4.738683in}{1.661705in}}%
\pgfpathlineto{\pgfqpoint{4.746321in}{1.661705in}}%
\pgfpathlineto{\pgfqpoint{4.748867in}{1.659297in}}%
\pgfpathlineto{\pgfqpoint{4.756505in}{1.659297in}}%
\pgfpathlineto{\pgfqpoint{4.759051in}{1.663695in}}%
\pgfpathlineto{\pgfqpoint{4.766689in}{1.663695in}}%
\pgfpathlineto{\pgfqpoint{4.769236in}{1.666360in}}%
\pgfpathlineto{\pgfqpoint{4.776874in}{1.666360in}}%
\pgfpathlineto{\pgfqpoint{4.779420in}{1.661953in}}%
\pgfpathlineto{\pgfqpoint{4.797242in}{1.662133in}}%
\pgfpathlineto{\pgfqpoint{4.799788in}{1.669862in}}%
\pgfpathlineto{\pgfqpoint{4.807426in}{1.669862in}}%
\pgfpathlineto{\pgfqpoint{4.809972in}{1.672239in}}%
\pgfpathlineto{\pgfqpoint{4.817610in}{1.672239in}}%
\pgfpathlineto{\pgfqpoint{4.820156in}{1.662035in}}%
\pgfpathlineto{\pgfqpoint{4.827794in}{1.662035in}}%
\pgfpathlineto{\pgfqpoint{4.832886in}{1.663253in}}%
\pgfpathlineto{\pgfqpoint{4.899083in}{1.663893in}}%
\pgfpathlineto{\pgfqpoint{4.906721in}{1.664380in}}%
\pgfpathlineto{\pgfqpoint{4.929635in}{1.664400in}}%
\pgfpathlineto{\pgfqpoint{4.932181in}{1.752529in}}%
\pgfpathlineto{\pgfqpoint{4.939819in}{1.752529in}}%
\pgfpathlineto{\pgfqpoint{4.942365in}{1.761003in}}%
\pgfpathlineto{\pgfqpoint{4.950003in}{1.761003in}}%
\pgfpathlineto{\pgfqpoint{4.952549in}{1.779409in}}%
\pgfpathlineto{\pgfqpoint{4.960188in}{1.779409in}}%
\pgfpathlineto{\pgfqpoint{4.962734in}{1.782799in}}%
\pgfpathlineto{\pgfqpoint{4.980556in}{1.783057in}}%
\pgfpathlineto{\pgfqpoint{4.983102in}{1.794422in}}%
\pgfpathlineto{\pgfqpoint{4.990740in}{1.794422in}}%
\pgfpathlineto{\pgfqpoint{4.993286in}{1.790988in}}%
\pgfpathlineto{\pgfqpoint{5.000924in}{1.790988in}}%
\pgfpathlineto{\pgfqpoint{5.003470in}{1.795197in}}%
\pgfpathlineto{\pgfqpoint{5.011108in}{1.795197in}}%
\pgfpathlineto{\pgfqpoint{5.013654in}{1.792161in}}%
\pgfpathlineto{\pgfqpoint{5.021292in}{1.792161in}}%
\pgfpathlineto{\pgfqpoint{5.023838in}{1.794286in}}%
\pgfpathlineto{\pgfqpoint{5.062029in}{1.794497in}}%
\pgfpathlineto{\pgfqpoint{5.064575in}{1.798133in}}%
\pgfpathlineto{\pgfqpoint{5.082397in}{1.798220in}}%
\pgfpathlineto{\pgfqpoint{5.084943in}{1.794368in}}%
\pgfpathlineto{\pgfqpoint{5.102765in}{1.795218in}}%
\pgfpathlineto{\pgfqpoint{5.112949in}{1.795584in}}%
\pgfpathlineto{\pgfqpoint{5.118041in}{1.794800in}}%
\pgfpathlineto{\pgfqpoint{5.123133in}{1.794800in}}%
\pgfpathlineto{\pgfqpoint{5.125679in}{1.797588in}}%
\pgfpathlineto{\pgfqpoint{5.133317in}{1.797588in}}%
\pgfpathlineto{\pgfqpoint{5.135863in}{1.665033in}}%
\pgfpathlineto{\pgfqpoint{5.184238in}{1.664633in}}%
\pgfpathlineto{\pgfqpoint{5.189330in}{1.663409in}}%
\pgfpathlineto{\pgfqpoint{5.265711in}{1.663505in}}%
\pgfpathlineto{\pgfqpoint{5.268257in}{1.670644in}}%
\pgfpathlineto{\pgfqpoint{5.286079in}{1.670634in}}%
\pgfpathlineto{\pgfqpoint{5.288625in}{1.665950in}}%
\pgfpathlineto{\pgfqpoint{5.306447in}{1.666008in}}%
\pgfpathlineto{\pgfqpoint{5.308993in}{1.671815in}}%
\pgfpathlineto{\pgfqpoint{5.337000in}{1.671809in}}%
\pgfpathlineto{\pgfqpoint{5.337000in}{1.671809in}}%
\pgfusepath{stroke}%
\end{pgfscope}%
\begin{pgfscope}%
\pgfpathrectangle{\pgfqpoint{3.593990in}{0.663635in}}{\pgfqpoint{1.826010in}{1.541693in}}%
\pgfusepath{clip}%
\pgfsetbuttcap%
\pgfsetroundjoin%
\pgfsetlinewidth{1.505625pt}%
\definecolor{currentstroke}{rgb}{0.172549,0.627451,0.172549}%
\pgfsetstrokecolor{currentstroke}%
\pgfsetdash{{5.550000pt}{2.400000pt}}{0.000000pt}%
\pgfpathmoveto{\pgfqpoint{3.593990in}{1.691431in}}%
\pgfpathlineto{\pgfqpoint{5.420000in}{1.691431in}}%
\pgfusepath{stroke}%
\end{pgfscope}%
\begin{pgfscope}%
\pgfsetrectcap%
\pgfsetmiterjoin%
\pgfsetlinewidth{1.254687pt}%
\definecolor{currentstroke}{rgb}{1.000000,1.000000,1.000000}%
\pgfsetstrokecolor{currentstroke}%
\pgfsetdash{}{0pt}%
\pgfpathmoveto{\pgfqpoint{3.593990in}{0.663635in}}%
\pgfpathlineto{\pgfqpoint{3.593990in}{2.205329in}}%
\pgfusepath{stroke}%
\end{pgfscope}%
\begin{pgfscope}%
\pgfsetrectcap%
\pgfsetmiterjoin%
\pgfsetlinewidth{1.254687pt}%
\definecolor{currentstroke}{rgb}{1.000000,1.000000,1.000000}%
\pgfsetstrokecolor{currentstroke}%
\pgfsetdash{}{0pt}%
\pgfpathmoveto{\pgfqpoint{5.420000in}{0.663635in}}%
\pgfpathlineto{\pgfqpoint{5.420000in}{2.205329in}}%
\pgfusepath{stroke}%
\end{pgfscope}%
\begin{pgfscope}%
\pgfsetrectcap%
\pgfsetmiterjoin%
\pgfsetlinewidth{1.254687pt}%
\definecolor{currentstroke}{rgb}{1.000000,1.000000,1.000000}%
\pgfsetstrokecolor{currentstroke}%
\pgfsetdash{}{0pt}%
\pgfpathmoveto{\pgfqpoint{3.593990in}{0.663635in}}%
\pgfpathlineto{\pgfqpoint{5.420000in}{0.663635in}}%
\pgfusepath{stroke}%
\end{pgfscope}%
\begin{pgfscope}%
\pgfsetrectcap%
\pgfsetmiterjoin%
\pgfsetlinewidth{1.254687pt}%
\definecolor{currentstroke}{rgb}{1.000000,1.000000,1.000000}%
\pgfsetstrokecolor{currentstroke}%
\pgfsetdash{}{0pt}%
\pgfpathmoveto{\pgfqpoint{3.593990in}{2.205329in}}%
\pgfpathlineto{\pgfqpoint{5.420000in}{2.205329in}}%
\pgfusepath{stroke}%
\end{pgfscope}%
\begin{pgfscope}%
\definecolor{textcolor}{rgb}{0.150000,0.150000,0.150000}%
\pgfsetstrokecolor{textcolor}%
\pgfsetfillcolor{textcolor}%
\pgftext[x=4.506995in,y=2.288662in,,base]{\color{textcolor}{\sffamily\fontsize{12.000000}{14.400000}\selectfont\catcode`\^=\active\def^{\ifmmode\sp\else\^{}\fi}\catcode`\%=\active\def%{\%}d)}}%
\end{pgfscope}%
\begin{pgfscope}%
\pgfsetbuttcap%
\pgfsetmiterjoin%
\definecolor{currentfill}{rgb}{0.917647,0.917647,0.949020}%
\pgfsetfillcolor{currentfill}%
\pgfsetfillopacity{0.800000}%
\pgfsetlinewidth{1.003750pt}%
\definecolor{currentstroke}{rgb}{0.800000,0.800000,0.800000}%
\pgfsetstrokecolor{currentstroke}%
\pgfsetstrokeopacity{0.800000}%
\pgfsetdash{}{0pt}%
\pgfpathmoveto{\pgfqpoint{2.017052in}{4.277160in}}%
\pgfpathlineto{\pgfqpoint{3.582948in}{4.277160in}}%
\pgfpathquadraticcurveto{\pgfqpoint{3.605170in}{4.277160in}}{\pgfqpoint{3.605170in}{4.299382in}}%
\pgfpathlineto{\pgfqpoint{3.605170in}{4.922222in}}%
\pgfpathquadraticcurveto{\pgfqpoint{3.605170in}{4.944444in}}{\pgfqpoint{3.582948in}{4.944444in}}%
\pgfpathlineto{\pgfqpoint{2.017052in}{4.944444in}}%
\pgfpathquadraticcurveto{\pgfqpoint{1.994830in}{4.944444in}}{\pgfqpoint{1.994830in}{4.922222in}}%
\pgfpathlineto{\pgfqpoint{1.994830in}{4.299382in}}%
\pgfpathquadraticcurveto{\pgfqpoint{1.994830in}{4.277160in}}{\pgfqpoint{2.017052in}{4.277160in}}%
\pgfpathlineto{\pgfqpoint{2.017052in}{4.277160in}}%
\pgfpathclose%
\pgfusepath{stroke,fill}%
\end{pgfscope}%
\begin{pgfscope}%
\pgfsetbuttcap%
\pgfsetroundjoin%
\pgfsetlinewidth{1.505625pt}%
\definecolor{currentstroke}{rgb}{1.000000,0.647059,0.000000}%
\pgfsetstrokecolor{currentstroke}%
\pgfsetdash{{1.500000pt}{2.475000pt}}{0.000000pt}%
\pgfpathmoveto{\pgfqpoint{2.039274in}{4.859353in}}%
\pgfpathlineto{\pgfqpoint{2.150385in}{4.859353in}}%
\pgfpathlineto{\pgfqpoint{2.261496in}{4.859353in}}%
\pgfusepath{stroke}%
\end{pgfscope}%
\begin{pgfscope}%
\definecolor{textcolor}{rgb}{0.150000,0.150000,0.150000}%
\pgfsetstrokecolor{textcolor}%
\pgfsetfillcolor{textcolor}%
\pgftext[x=2.350385in,y=4.820464in,left,base]{\color{textcolor}{\sffamily\fontsize{8.000000}{9.600000}\selectfont\catcode`\^=\active\def^{\ifmmode\sp\else\^{}\fi}\catcode`\%=\active\def%{\%}scaling event}}%
\end{pgfscope}%
\begin{pgfscope}%
\pgfsetbuttcap%
\pgfsetroundjoin%
\pgfsetlinewidth{1.505625pt}%
\definecolor{currentstroke}{rgb}{0.580392,0.403922,0.741176}%
\pgfsetstrokecolor{currentstroke}%
\pgfsetdash{{5.550000pt}{2.400000pt}}{0.000000pt}%
\pgfpathmoveto{\pgfqpoint{2.039274in}{4.699523in}}%
\pgfpathlineto{\pgfqpoint{2.150385in}{4.699523in}}%
\pgfpathlineto{\pgfqpoint{2.261496in}{4.699523in}}%
\pgfusepath{stroke}%
\end{pgfscope}%
\begin{pgfscope}%
\definecolor{textcolor}{rgb}{0.150000,0.150000,0.150000}%
\pgfsetstrokecolor{textcolor}%
\pgfsetfillcolor{textcolor}%
\pgftext[x=2.350385in,y=4.660634in,left,base]{\color{textcolor}{\sffamily\fontsize{8.000000}{9.600000}\selectfont\catcode`\^=\active\def^{\ifmmode\sp\else\^{}\fi}\catcode`\%=\active\def%{\%}target CPU utilization}}%
\end{pgfscope}%
\begin{pgfscope}%
\pgfsetbuttcap%
\pgfsetmiterjoin%
\definecolor{currentfill}{rgb}{1.000000,0.000000,0.000000}%
\pgfsetfillcolor{currentfill}%
\pgfsetfillopacity{0.300000}%
\pgfsetlinewidth{1.003750pt}%
\definecolor{currentstroke}{rgb}{1.000000,0.000000,0.000000}%
\pgfsetstrokecolor{currentstroke}%
\pgfsetstrokeopacity{0.300000}%
\pgfsetdash{}{0pt}%
\pgfpathmoveto{\pgfqpoint{2.039274in}{4.502160in}}%
\pgfpathlineto{\pgfqpoint{2.261496in}{4.502160in}}%
\pgfpathlineto{\pgfqpoint{2.261496in}{4.579938in}}%
\pgfpathlineto{\pgfqpoint{2.039274in}{4.579938in}}%
\pgfpathlineto{\pgfqpoint{2.039274in}{4.502160in}}%
\pgfpathclose%
\pgfusepath{stroke,fill}%
\end{pgfscope}%
\begin{pgfscope}%
\definecolor{textcolor}{rgb}{0.150000,0.150000,0.150000}%
\pgfsetstrokecolor{textcolor}%
\pgfsetfillcolor{textcolor}%
\pgftext[x=2.350385in,y=4.502160in,left,base]{\color{textcolor}{\sffamily\fontsize{8.000000}{9.600000}\selectfont\catcode`\^=\active\def^{\ifmmode\sp\else\^{}\fi}\catcode`\%=\active\def%{\%}k8ssandra reconsiliation}}%
\end{pgfscope}%
\begin{pgfscope}%
\pgfsetbuttcap%
\pgfsetroundjoin%
\pgfsetlinewidth{1.505625pt}%
\definecolor{currentstroke}{rgb}{0.172549,0.627451,0.172549}%
\pgfsetstrokecolor{currentstroke}%
\pgfsetdash{{5.550000pt}{2.400000pt}}{0.000000pt}%
\pgfpathmoveto{\pgfqpoint{2.039274in}{4.383877in}}%
\pgfpathlineto{\pgfqpoint{2.150385in}{4.383877in}}%
\pgfpathlineto{\pgfqpoint{2.261496in}{4.383877in}}%
\pgfusepath{stroke}%
\end{pgfscope}%
\begin{pgfscope}%
\definecolor{textcolor}{rgb}{0.150000,0.150000,0.150000}%
\pgfsetstrokecolor{textcolor}%
\pgfsetfillcolor{textcolor}%
\pgftext[x=2.350385in,y=4.344988in,left,base]{\color{textcolor}{\sffamily\fontsize{8.000000}{9.600000}\selectfont\catcode`\^=\active\def^{\ifmmode\sp\else\^{}\fi}\catcode`\%=\active\def%{\%}target memory utilization}}%
\end{pgfscope}%
\end{pgfpicture}%
\makeatother%
\endgroup%

    \caption{Adjustment of CPU and memory resources as well as cluster size during diagonal elasticity}
    \label{fig:diagonal-elasticity}
\end{figure}

\begin{figure}
    \centering
    %% Creator: Matplotlib, PGF backend
%%
%% To include the figure in your LaTeX document, write
%%   \input{<filename>.pgf}
%%
%% Make sure the required packages are loaded in your preamble
%%   \usepackage{pgf}
%%
%% Also ensure that all the required font packages are loaded; for instance,
%% the lmodern package is sometimes necessary when using math font.
%%   \usepackage{lmodern}
%%
%% Figures using additional raster images can only be included by \input if
%% they are in the same directory as the main LaTeX file. For loading figures
%% from other directories you can use the `import` package
%%   \usepackage{import}
%%
%% and then include the figures with
%%   \import{<path to file>}{<filename>.pgf}
%%
%% Matplotlib used the following preamble
%%   \def\mathdefault#1{#1}
%%   \everymath=\expandafter{\the\everymath\displaystyle}
%%   
%%   \usepackage{fontspec}
%%   \setmainfont{DejaVuSerif.ttf}[Path=\detokenize{/Users/nkratky/private/polaris-elasticity-strategies/test/scripts/.venv/lib/python3.11/site-packages/matplotlib/mpl-data/fonts/ttf/}]
%%   \setsansfont{Arial.ttf}[Path=\detokenize{/System/Library/Fonts/Supplemental/}]
%%   \setmonofont{DejaVuSansMono.ttf}[Path=\detokenize{/Users/nkratky/private/polaris-elasticity-strategies/test/scripts/.venv/lib/python3.11/site-packages/matplotlib/mpl-data/fonts/ttf/}]
%%   \makeatletter\@ifpackageloaded{underscore}{}{\usepackage[strings]{underscore}}\makeatother
%%
\begingroup%
\makeatletter%
\begin{pgfpicture}%
\pgfpathrectangle{\pgfpointorigin}{\pgfqpoint{5.600000in}{4.000000in}}%
\pgfusepath{use as bounding box, clip}%
\begin{pgfscope}%
\pgfsetbuttcap%
\pgfsetmiterjoin%
\definecolor{currentfill}{rgb}{1.000000,1.000000,1.000000}%
\pgfsetfillcolor{currentfill}%
\pgfsetlinewidth{0.000000pt}%
\definecolor{currentstroke}{rgb}{1.000000,1.000000,1.000000}%
\pgfsetstrokecolor{currentstroke}%
\pgfsetdash{}{0pt}%
\pgfpathmoveto{\pgfqpoint{0.000000in}{0.000000in}}%
\pgfpathlineto{\pgfqpoint{5.600000in}{0.000000in}}%
\pgfpathlineto{\pgfqpoint{5.600000in}{4.000000in}}%
\pgfpathlineto{\pgfqpoint{0.000000in}{4.000000in}}%
\pgfpathlineto{\pgfqpoint{0.000000in}{0.000000in}}%
\pgfpathclose%
\pgfusepath{fill}%
\end{pgfscope}%
\begin{pgfscope}%
\pgfsetbuttcap%
\pgfsetmiterjoin%
\definecolor{currentfill}{rgb}{0.917647,0.917647,0.949020}%
\pgfsetfillcolor{currentfill}%
\pgfsetlinewidth{0.000000pt}%
\definecolor{currentstroke}{rgb}{0.000000,0.000000,0.000000}%
\pgfsetstrokecolor{currentstroke}%
\pgfsetstrokeopacity{0.000000}%
\pgfsetdash{}{0pt}%
\pgfpathmoveto{\pgfqpoint{0.863783in}{2.573635in}}%
\pgfpathlineto{\pgfqpoint{5.420000in}{2.573635in}}%
\pgfpathlineto{\pgfqpoint{5.420000in}{3.765319in}}%
\pgfpathlineto{\pgfqpoint{0.863783in}{3.765319in}}%
\pgfpathlineto{\pgfqpoint{0.863783in}{2.573635in}}%
\pgfpathclose%
\pgfusepath{fill}%
\end{pgfscope}%
\begin{pgfscope}%
\pgfpathrectangle{\pgfqpoint{0.863783in}{2.573635in}}{\pgfqpoint{4.556217in}{1.191684in}}%
\pgfusepath{clip}%
\pgfsetroundcap%
\pgfsetroundjoin%
\pgfsetlinewidth{1.003750pt}%
\definecolor{currentstroke}{rgb}{1.000000,1.000000,1.000000}%
\pgfsetstrokecolor{currentstroke}%
\pgfsetdash{}{0pt}%
\pgfpathmoveto{\pgfqpoint{1.070884in}{2.573635in}}%
\pgfpathlineto{\pgfqpoint{1.070884in}{3.765319in}}%
\pgfusepath{stroke}%
\end{pgfscope}%
\begin{pgfscope}%
\definecolor{textcolor}{rgb}{0.150000,0.150000,0.150000}%
\pgfsetstrokecolor{textcolor}%
\pgfsetfillcolor{textcolor}%
\pgftext[x=1.070884in,y=2.441691in,,top]{\color{textcolor}{\sffamily\fontsize{11.000000}{13.200000}\selectfont\catcode`\^=\active\def^{\ifmmode\sp\else\^{}\fi}\catcode`\%=\active\def%{\%}0}}%
\end{pgfscope}%
\begin{pgfscope}%
\pgfpathrectangle{\pgfqpoint{0.863783in}{2.573635in}}{\pgfqpoint{4.556217in}{1.191684in}}%
\pgfusepath{clip}%
\pgfsetroundcap%
\pgfsetroundjoin%
\pgfsetlinewidth{1.003750pt}%
\definecolor{currentstroke}{rgb}{1.000000,1.000000,1.000000}%
\pgfsetstrokecolor{currentstroke}%
\pgfsetdash{}{0pt}%
\pgfpathmoveto{\pgfqpoint{1.698462in}{2.573635in}}%
\pgfpathlineto{\pgfqpoint{1.698462in}{3.765319in}}%
\pgfusepath{stroke}%
\end{pgfscope}%
\begin{pgfscope}%
\definecolor{textcolor}{rgb}{0.150000,0.150000,0.150000}%
\pgfsetstrokecolor{textcolor}%
\pgfsetfillcolor{textcolor}%
\pgftext[x=1.698462in,y=2.441691in,,top]{\color{textcolor}{\sffamily\fontsize{11.000000}{13.200000}\selectfont\catcode`\^=\active\def^{\ifmmode\sp\else\^{}\fi}\catcode`\%=\active\def%{\%}500}}%
\end{pgfscope}%
\begin{pgfscope}%
\pgfpathrectangle{\pgfqpoint{0.863783in}{2.573635in}}{\pgfqpoint{4.556217in}{1.191684in}}%
\pgfusepath{clip}%
\pgfsetroundcap%
\pgfsetroundjoin%
\pgfsetlinewidth{1.003750pt}%
\definecolor{currentstroke}{rgb}{1.000000,1.000000,1.000000}%
\pgfsetstrokecolor{currentstroke}%
\pgfsetdash{}{0pt}%
\pgfpathmoveto{\pgfqpoint{2.326040in}{2.573635in}}%
\pgfpathlineto{\pgfqpoint{2.326040in}{3.765319in}}%
\pgfusepath{stroke}%
\end{pgfscope}%
\begin{pgfscope}%
\definecolor{textcolor}{rgb}{0.150000,0.150000,0.150000}%
\pgfsetstrokecolor{textcolor}%
\pgfsetfillcolor{textcolor}%
\pgftext[x=2.326040in,y=2.441691in,,top]{\color{textcolor}{\sffamily\fontsize{11.000000}{13.200000}\selectfont\catcode`\^=\active\def^{\ifmmode\sp\else\^{}\fi}\catcode`\%=\active\def%{\%}1000}}%
\end{pgfscope}%
\begin{pgfscope}%
\pgfpathrectangle{\pgfqpoint{0.863783in}{2.573635in}}{\pgfqpoint{4.556217in}{1.191684in}}%
\pgfusepath{clip}%
\pgfsetroundcap%
\pgfsetroundjoin%
\pgfsetlinewidth{1.003750pt}%
\definecolor{currentstroke}{rgb}{1.000000,1.000000,1.000000}%
\pgfsetstrokecolor{currentstroke}%
\pgfsetdash{}{0pt}%
\pgfpathmoveto{\pgfqpoint{2.953618in}{2.573635in}}%
\pgfpathlineto{\pgfqpoint{2.953618in}{3.765319in}}%
\pgfusepath{stroke}%
\end{pgfscope}%
\begin{pgfscope}%
\definecolor{textcolor}{rgb}{0.150000,0.150000,0.150000}%
\pgfsetstrokecolor{textcolor}%
\pgfsetfillcolor{textcolor}%
\pgftext[x=2.953618in,y=2.441691in,,top]{\color{textcolor}{\sffamily\fontsize{11.000000}{13.200000}\selectfont\catcode`\^=\active\def^{\ifmmode\sp\else\^{}\fi}\catcode`\%=\active\def%{\%}1500}}%
\end{pgfscope}%
\begin{pgfscope}%
\pgfpathrectangle{\pgfqpoint{0.863783in}{2.573635in}}{\pgfqpoint{4.556217in}{1.191684in}}%
\pgfusepath{clip}%
\pgfsetroundcap%
\pgfsetroundjoin%
\pgfsetlinewidth{1.003750pt}%
\definecolor{currentstroke}{rgb}{1.000000,1.000000,1.000000}%
\pgfsetstrokecolor{currentstroke}%
\pgfsetdash{}{0pt}%
\pgfpathmoveto{\pgfqpoint{3.581196in}{2.573635in}}%
\pgfpathlineto{\pgfqpoint{3.581196in}{3.765319in}}%
\pgfusepath{stroke}%
\end{pgfscope}%
\begin{pgfscope}%
\definecolor{textcolor}{rgb}{0.150000,0.150000,0.150000}%
\pgfsetstrokecolor{textcolor}%
\pgfsetfillcolor{textcolor}%
\pgftext[x=3.581196in,y=2.441691in,,top]{\color{textcolor}{\sffamily\fontsize{11.000000}{13.200000}\selectfont\catcode`\^=\active\def^{\ifmmode\sp\else\^{}\fi}\catcode`\%=\active\def%{\%}2000}}%
\end{pgfscope}%
\begin{pgfscope}%
\pgfpathrectangle{\pgfqpoint{0.863783in}{2.573635in}}{\pgfqpoint{4.556217in}{1.191684in}}%
\pgfusepath{clip}%
\pgfsetroundcap%
\pgfsetroundjoin%
\pgfsetlinewidth{1.003750pt}%
\definecolor{currentstroke}{rgb}{1.000000,1.000000,1.000000}%
\pgfsetstrokecolor{currentstroke}%
\pgfsetdash{}{0pt}%
\pgfpathmoveto{\pgfqpoint{4.208774in}{2.573635in}}%
\pgfpathlineto{\pgfqpoint{4.208774in}{3.765319in}}%
\pgfusepath{stroke}%
\end{pgfscope}%
\begin{pgfscope}%
\definecolor{textcolor}{rgb}{0.150000,0.150000,0.150000}%
\pgfsetstrokecolor{textcolor}%
\pgfsetfillcolor{textcolor}%
\pgftext[x=4.208774in,y=2.441691in,,top]{\color{textcolor}{\sffamily\fontsize{11.000000}{13.200000}\selectfont\catcode`\^=\active\def^{\ifmmode\sp\else\^{}\fi}\catcode`\%=\active\def%{\%}2500}}%
\end{pgfscope}%
\begin{pgfscope}%
\pgfpathrectangle{\pgfqpoint{0.863783in}{2.573635in}}{\pgfqpoint{4.556217in}{1.191684in}}%
\pgfusepath{clip}%
\pgfsetroundcap%
\pgfsetroundjoin%
\pgfsetlinewidth{1.003750pt}%
\definecolor{currentstroke}{rgb}{1.000000,1.000000,1.000000}%
\pgfsetstrokecolor{currentstroke}%
\pgfsetdash{}{0pt}%
\pgfpathmoveto{\pgfqpoint{4.836352in}{2.573635in}}%
\pgfpathlineto{\pgfqpoint{4.836352in}{3.765319in}}%
\pgfusepath{stroke}%
\end{pgfscope}%
\begin{pgfscope}%
\definecolor{textcolor}{rgb}{0.150000,0.150000,0.150000}%
\pgfsetstrokecolor{textcolor}%
\pgfsetfillcolor{textcolor}%
\pgftext[x=4.836352in,y=2.441691in,,top]{\color{textcolor}{\sffamily\fontsize{11.000000}{13.200000}\selectfont\catcode`\^=\active\def^{\ifmmode\sp\else\^{}\fi}\catcode`\%=\active\def%{\%}3000}}%
\end{pgfscope}%
\begin{pgfscope}%
\definecolor{textcolor}{rgb}{0.150000,0.150000,0.150000}%
\pgfsetstrokecolor{textcolor}%
\pgfsetfillcolor{textcolor}%
\pgftext[x=3.141892in,y=2.246413in,,top]{\color{textcolor}{\sffamily\fontsize{12.000000}{14.400000}\selectfont\catcode`\^=\active\def^{\ifmmode\sp\else\^{}\fi}\catcode`\%=\active\def%{\%}Time (s)}}%
\end{pgfscope}%
\begin{pgfscope}%
\pgfpathrectangle{\pgfqpoint{0.863783in}{2.573635in}}{\pgfqpoint{4.556217in}{1.191684in}}%
\pgfusepath{clip}%
\pgfsetroundcap%
\pgfsetroundjoin%
\pgfsetlinewidth{1.003750pt}%
\definecolor{currentstroke}{rgb}{1.000000,1.000000,1.000000}%
\pgfsetstrokecolor{currentstroke}%
\pgfsetdash{}{0pt}%
\pgfpathmoveto{\pgfqpoint{0.863783in}{2.871556in}}%
\pgfpathlineto{\pgfqpoint{5.420000in}{2.871556in}}%
\pgfusepath{stroke}%
\end{pgfscope}%
\begin{pgfscope}%
\definecolor{textcolor}{rgb}{0.150000,0.150000,0.150000}%
\pgfsetstrokecolor{textcolor}%
\pgfsetfillcolor{textcolor}%
\pgftext[x=0.391968in, y=2.816876in, left, base]{\color{textcolor}{\sffamily\fontsize{11.000000}{13.200000}\selectfont\catcode`\^=\active\def^{\ifmmode\sp\else\^{}\fi}\catcode`\%=\active\def%{\%}1800}}%
\end{pgfscope}%
\begin{pgfscope}%
\pgfpathrectangle{\pgfqpoint{0.863783in}{2.573635in}}{\pgfqpoint{4.556217in}{1.191684in}}%
\pgfusepath{clip}%
\pgfsetroundcap%
\pgfsetroundjoin%
\pgfsetlinewidth{1.003750pt}%
\definecolor{currentstroke}{rgb}{1.000000,1.000000,1.000000}%
\pgfsetstrokecolor{currentstroke}%
\pgfsetdash{}{0pt}%
\pgfpathmoveto{\pgfqpoint{0.863783in}{3.467398in}}%
\pgfpathlineto{\pgfqpoint{5.420000in}{3.467398in}}%
\pgfusepath{stroke}%
\end{pgfscope}%
\begin{pgfscope}%
\definecolor{textcolor}{rgb}{0.150000,0.150000,0.150000}%
\pgfsetstrokecolor{textcolor}%
\pgfsetfillcolor{textcolor}%
\pgftext[x=0.391968in, y=3.412718in, left, base]{\color{textcolor}{\sffamily\fontsize{11.000000}{13.200000}\selectfont\catcode`\^=\active\def^{\ifmmode\sp\else\^{}\fi}\catcode`\%=\active\def%{\%}2000}}%
\end{pgfscope}%
\begin{pgfscope}%
\definecolor{textcolor}{rgb}{0.150000,0.150000,0.150000}%
\pgfsetstrokecolor{textcolor}%
\pgfsetfillcolor{textcolor}%
\pgftext[x=0.336413in,y=3.169477in,,bottom,rotate=90.000000]{\color{textcolor}{\sffamily\fontsize{12.000000}{14.400000}\selectfont\catcode`\^=\active\def^{\ifmmode\sp\else\^{}\fi}\catcode`\%=\active\def%{\%}CPU Limits (milliCPU)}}%
\end{pgfscope}%
\begin{pgfscope}%
\pgfpathrectangle{\pgfqpoint{0.863783in}{2.573635in}}{\pgfqpoint{4.556217in}{1.191684in}}%
\pgfusepath{clip}%
\pgfsetbuttcap%
\pgfsetroundjoin%
\pgfsetlinewidth{1.505625pt}%
\definecolor{currentstroke}{rgb}{1.000000,0.647059,0.000000}%
\pgfsetstrokecolor{currentstroke}%
\pgfsetdash{{1.500000pt}{2.475000pt}}{0.000000pt}%
\pgfpathmoveto{\pgfqpoint{1.427349in}{2.573635in}}%
\pgfpathlineto{\pgfqpoint{1.427349in}{3.765319in}}%
\pgfusepath{stroke}%
\end{pgfscope}%
\begin{pgfscope}%
\pgfpathrectangle{\pgfqpoint{0.863783in}{2.573635in}}{\pgfqpoint{4.556217in}{1.191684in}}%
\pgfusepath{clip}%
\pgfsetbuttcap%
\pgfsetroundjoin%
\pgfsetlinewidth{1.505625pt}%
\definecolor{currentstroke}{rgb}{1.000000,0.647059,0.000000}%
\pgfsetstrokecolor{currentstroke}%
\pgfsetdash{{1.500000pt}{2.475000pt}}{0.000000pt}%
\pgfpathmoveto{\pgfqpoint{2.186718in}{2.573635in}}%
\pgfpathlineto{\pgfqpoint{2.186718in}{3.765319in}}%
\pgfusepath{stroke}%
\end{pgfscope}%
\begin{pgfscope}%
\pgfpathrectangle{\pgfqpoint{0.863783in}{2.573635in}}{\pgfqpoint{4.556217in}{1.191684in}}%
\pgfusepath{clip}%
\pgfsetbuttcap%
\pgfsetroundjoin%
\pgfsetlinewidth{1.505625pt}%
\definecolor{currentstroke}{rgb}{1.000000,0.647059,0.000000}%
\pgfsetstrokecolor{currentstroke}%
\pgfsetdash{{1.500000pt}{2.475000pt}}{0.000000pt}%
\pgfpathmoveto{\pgfqpoint{2.941067in}{2.573635in}}%
\pgfpathlineto{\pgfqpoint{2.941067in}{3.765319in}}%
\pgfusepath{stroke}%
\end{pgfscope}%
\begin{pgfscope}%
\pgfpathrectangle{\pgfqpoint{0.863783in}{2.573635in}}{\pgfqpoint{4.556217in}{1.191684in}}%
\pgfusepath{clip}%
\pgfsetbuttcap%
\pgfsetroundjoin%
\pgfsetlinewidth{1.505625pt}%
\definecolor{currentstroke}{rgb}{1.000000,0.647059,0.000000}%
\pgfsetstrokecolor{currentstroke}%
\pgfsetdash{{1.500000pt}{2.475000pt}}{0.000000pt}%
\pgfpathmoveto{\pgfqpoint{3.696671in}{2.573635in}}%
\pgfpathlineto{\pgfqpoint{3.696671in}{3.765319in}}%
\pgfusepath{stroke}%
\end{pgfscope}%
\begin{pgfscope}%
\pgfpathrectangle{\pgfqpoint{0.863783in}{2.573635in}}{\pgfqpoint{4.556217in}{1.191684in}}%
\pgfusepath{clip}%
\pgfsetbuttcap%
\pgfsetroundjoin%
\pgfsetlinewidth{1.505625pt}%
\definecolor{currentstroke}{rgb}{1.000000,0.647059,0.000000}%
\pgfsetstrokecolor{currentstroke}%
\pgfsetdash{{1.500000pt}{2.475000pt}}{0.000000pt}%
\pgfpathmoveto{\pgfqpoint{4.449764in}{2.573635in}}%
\pgfpathlineto{\pgfqpoint{4.449764in}{3.765319in}}%
\pgfusepath{stroke}%
\end{pgfscope}%
\begin{pgfscope}%
\pgfpathrectangle{\pgfqpoint{0.863783in}{2.573635in}}{\pgfqpoint{4.556217in}{1.191684in}}%
\pgfusepath{clip}%
\pgfsetroundcap%
\pgfsetroundjoin%
\pgfsetlinewidth{1.505625pt}%
\definecolor{currentstroke}{rgb}{0.298039,0.447059,0.690196}%
\pgfsetstrokecolor{currentstroke}%
\pgfsetdash{}{0pt}%
\pgfpathmoveto{\pgfqpoint{1.070884in}{3.467398in}}%
\pgfpathlineto{\pgfqpoint{1.434879in}{3.467398in}}%
\pgfpathlineto{\pgfqpoint{1.441155in}{3.169477in}}%
\pgfpathlineto{\pgfqpoint{5.212899in}{3.169477in}}%
\pgfpathlineto{\pgfqpoint{5.212899in}{3.169477in}}%
\pgfusepath{stroke}%
\end{pgfscope}%
\begin{pgfscope}%
\pgfsetrectcap%
\pgfsetmiterjoin%
\pgfsetlinewidth{1.254687pt}%
\definecolor{currentstroke}{rgb}{1.000000,1.000000,1.000000}%
\pgfsetstrokecolor{currentstroke}%
\pgfsetdash{}{0pt}%
\pgfpathmoveto{\pgfqpoint{0.863783in}{2.573635in}}%
\pgfpathlineto{\pgfqpoint{0.863783in}{3.765319in}}%
\pgfusepath{stroke}%
\end{pgfscope}%
\begin{pgfscope}%
\pgfsetrectcap%
\pgfsetmiterjoin%
\pgfsetlinewidth{1.254687pt}%
\definecolor{currentstroke}{rgb}{1.000000,1.000000,1.000000}%
\pgfsetstrokecolor{currentstroke}%
\pgfsetdash{}{0pt}%
\pgfpathmoveto{\pgfqpoint{5.420000in}{2.573635in}}%
\pgfpathlineto{\pgfqpoint{5.420000in}{3.765319in}}%
\pgfusepath{stroke}%
\end{pgfscope}%
\begin{pgfscope}%
\pgfsetrectcap%
\pgfsetmiterjoin%
\pgfsetlinewidth{1.254687pt}%
\definecolor{currentstroke}{rgb}{1.000000,1.000000,1.000000}%
\pgfsetstrokecolor{currentstroke}%
\pgfsetdash{}{0pt}%
\pgfpathmoveto{\pgfqpoint{0.863783in}{2.573635in}}%
\pgfpathlineto{\pgfqpoint{5.420000in}{2.573635in}}%
\pgfusepath{stroke}%
\end{pgfscope}%
\begin{pgfscope}%
\pgfsetrectcap%
\pgfsetmiterjoin%
\pgfsetlinewidth{1.254687pt}%
\definecolor{currentstroke}{rgb}{1.000000,1.000000,1.000000}%
\pgfsetstrokecolor{currentstroke}%
\pgfsetdash{}{0pt}%
\pgfpathmoveto{\pgfqpoint{0.863783in}{3.765319in}}%
\pgfpathlineto{\pgfqpoint{5.420000in}{3.765319in}}%
\pgfusepath{stroke}%
\end{pgfscope}%
\begin{pgfscope}%
\pgfsetbuttcap%
\pgfsetmiterjoin%
\definecolor{currentfill}{rgb}{0.917647,0.917647,0.949020}%
\pgfsetfillcolor{currentfill}%
\pgfsetlinewidth{0.000000pt}%
\definecolor{currentstroke}{rgb}{0.000000,0.000000,0.000000}%
\pgfsetstrokecolor{currentstroke}%
\pgfsetstrokeopacity{0.000000}%
\pgfsetdash{}{0pt}%
\pgfpathmoveto{\pgfqpoint{0.863783in}{0.663635in}}%
\pgfpathlineto{\pgfqpoint{5.420000in}{0.663635in}}%
\pgfpathlineto{\pgfqpoint{5.420000in}{1.855319in}}%
\pgfpathlineto{\pgfqpoint{0.863783in}{1.855319in}}%
\pgfpathlineto{\pgfqpoint{0.863783in}{0.663635in}}%
\pgfpathclose%
\pgfusepath{fill}%
\end{pgfscope}%
\begin{pgfscope}%
\pgfpathrectangle{\pgfqpoint{0.863783in}{0.663635in}}{\pgfqpoint{4.556217in}{1.191684in}}%
\pgfusepath{clip}%
\pgfsetroundcap%
\pgfsetroundjoin%
\pgfsetlinewidth{1.003750pt}%
\definecolor{currentstroke}{rgb}{1.000000,1.000000,1.000000}%
\pgfsetstrokecolor{currentstroke}%
\pgfsetdash{}{0pt}%
\pgfpathmoveto{\pgfqpoint{1.070884in}{0.663635in}}%
\pgfpathlineto{\pgfqpoint{1.070884in}{1.855319in}}%
\pgfusepath{stroke}%
\end{pgfscope}%
\begin{pgfscope}%
\definecolor{textcolor}{rgb}{0.150000,0.150000,0.150000}%
\pgfsetstrokecolor{textcolor}%
\pgfsetfillcolor{textcolor}%
\pgftext[x=1.070884in,y=0.531691in,,top]{\color{textcolor}{\sffamily\fontsize{11.000000}{13.200000}\selectfont\catcode`\^=\active\def^{\ifmmode\sp\else\^{}\fi}\catcode`\%=\active\def%{\%}0}}%
\end{pgfscope}%
\begin{pgfscope}%
\pgfpathrectangle{\pgfqpoint{0.863783in}{0.663635in}}{\pgfqpoint{4.556217in}{1.191684in}}%
\pgfusepath{clip}%
\pgfsetroundcap%
\pgfsetroundjoin%
\pgfsetlinewidth{1.003750pt}%
\definecolor{currentstroke}{rgb}{1.000000,1.000000,1.000000}%
\pgfsetstrokecolor{currentstroke}%
\pgfsetdash{}{0pt}%
\pgfpathmoveto{\pgfqpoint{1.698462in}{0.663635in}}%
\pgfpathlineto{\pgfqpoint{1.698462in}{1.855319in}}%
\pgfusepath{stroke}%
\end{pgfscope}%
\begin{pgfscope}%
\definecolor{textcolor}{rgb}{0.150000,0.150000,0.150000}%
\pgfsetstrokecolor{textcolor}%
\pgfsetfillcolor{textcolor}%
\pgftext[x=1.698462in,y=0.531691in,,top]{\color{textcolor}{\sffamily\fontsize{11.000000}{13.200000}\selectfont\catcode`\^=\active\def^{\ifmmode\sp\else\^{}\fi}\catcode`\%=\active\def%{\%}500}}%
\end{pgfscope}%
\begin{pgfscope}%
\pgfpathrectangle{\pgfqpoint{0.863783in}{0.663635in}}{\pgfqpoint{4.556217in}{1.191684in}}%
\pgfusepath{clip}%
\pgfsetroundcap%
\pgfsetroundjoin%
\pgfsetlinewidth{1.003750pt}%
\definecolor{currentstroke}{rgb}{1.000000,1.000000,1.000000}%
\pgfsetstrokecolor{currentstroke}%
\pgfsetdash{}{0pt}%
\pgfpathmoveto{\pgfqpoint{2.326040in}{0.663635in}}%
\pgfpathlineto{\pgfqpoint{2.326040in}{1.855319in}}%
\pgfusepath{stroke}%
\end{pgfscope}%
\begin{pgfscope}%
\definecolor{textcolor}{rgb}{0.150000,0.150000,0.150000}%
\pgfsetstrokecolor{textcolor}%
\pgfsetfillcolor{textcolor}%
\pgftext[x=2.326040in,y=0.531691in,,top]{\color{textcolor}{\sffamily\fontsize{11.000000}{13.200000}\selectfont\catcode`\^=\active\def^{\ifmmode\sp\else\^{}\fi}\catcode`\%=\active\def%{\%}1000}}%
\end{pgfscope}%
\begin{pgfscope}%
\pgfpathrectangle{\pgfqpoint{0.863783in}{0.663635in}}{\pgfqpoint{4.556217in}{1.191684in}}%
\pgfusepath{clip}%
\pgfsetroundcap%
\pgfsetroundjoin%
\pgfsetlinewidth{1.003750pt}%
\definecolor{currentstroke}{rgb}{1.000000,1.000000,1.000000}%
\pgfsetstrokecolor{currentstroke}%
\pgfsetdash{}{0pt}%
\pgfpathmoveto{\pgfqpoint{2.953618in}{0.663635in}}%
\pgfpathlineto{\pgfqpoint{2.953618in}{1.855319in}}%
\pgfusepath{stroke}%
\end{pgfscope}%
\begin{pgfscope}%
\definecolor{textcolor}{rgb}{0.150000,0.150000,0.150000}%
\pgfsetstrokecolor{textcolor}%
\pgfsetfillcolor{textcolor}%
\pgftext[x=2.953618in,y=0.531691in,,top]{\color{textcolor}{\sffamily\fontsize{11.000000}{13.200000}\selectfont\catcode`\^=\active\def^{\ifmmode\sp\else\^{}\fi}\catcode`\%=\active\def%{\%}1500}}%
\end{pgfscope}%
\begin{pgfscope}%
\pgfpathrectangle{\pgfqpoint{0.863783in}{0.663635in}}{\pgfqpoint{4.556217in}{1.191684in}}%
\pgfusepath{clip}%
\pgfsetroundcap%
\pgfsetroundjoin%
\pgfsetlinewidth{1.003750pt}%
\definecolor{currentstroke}{rgb}{1.000000,1.000000,1.000000}%
\pgfsetstrokecolor{currentstroke}%
\pgfsetdash{}{0pt}%
\pgfpathmoveto{\pgfqpoint{3.581196in}{0.663635in}}%
\pgfpathlineto{\pgfqpoint{3.581196in}{1.855319in}}%
\pgfusepath{stroke}%
\end{pgfscope}%
\begin{pgfscope}%
\definecolor{textcolor}{rgb}{0.150000,0.150000,0.150000}%
\pgfsetstrokecolor{textcolor}%
\pgfsetfillcolor{textcolor}%
\pgftext[x=3.581196in,y=0.531691in,,top]{\color{textcolor}{\sffamily\fontsize{11.000000}{13.200000}\selectfont\catcode`\^=\active\def^{\ifmmode\sp\else\^{}\fi}\catcode`\%=\active\def%{\%}2000}}%
\end{pgfscope}%
\begin{pgfscope}%
\pgfpathrectangle{\pgfqpoint{0.863783in}{0.663635in}}{\pgfqpoint{4.556217in}{1.191684in}}%
\pgfusepath{clip}%
\pgfsetroundcap%
\pgfsetroundjoin%
\pgfsetlinewidth{1.003750pt}%
\definecolor{currentstroke}{rgb}{1.000000,1.000000,1.000000}%
\pgfsetstrokecolor{currentstroke}%
\pgfsetdash{}{0pt}%
\pgfpathmoveto{\pgfqpoint{4.208774in}{0.663635in}}%
\pgfpathlineto{\pgfqpoint{4.208774in}{1.855319in}}%
\pgfusepath{stroke}%
\end{pgfscope}%
\begin{pgfscope}%
\definecolor{textcolor}{rgb}{0.150000,0.150000,0.150000}%
\pgfsetstrokecolor{textcolor}%
\pgfsetfillcolor{textcolor}%
\pgftext[x=4.208774in,y=0.531691in,,top]{\color{textcolor}{\sffamily\fontsize{11.000000}{13.200000}\selectfont\catcode`\^=\active\def^{\ifmmode\sp\else\^{}\fi}\catcode`\%=\active\def%{\%}2500}}%
\end{pgfscope}%
\begin{pgfscope}%
\pgfpathrectangle{\pgfqpoint{0.863783in}{0.663635in}}{\pgfqpoint{4.556217in}{1.191684in}}%
\pgfusepath{clip}%
\pgfsetroundcap%
\pgfsetroundjoin%
\pgfsetlinewidth{1.003750pt}%
\definecolor{currentstroke}{rgb}{1.000000,1.000000,1.000000}%
\pgfsetstrokecolor{currentstroke}%
\pgfsetdash{}{0pt}%
\pgfpathmoveto{\pgfqpoint{4.836352in}{0.663635in}}%
\pgfpathlineto{\pgfqpoint{4.836352in}{1.855319in}}%
\pgfusepath{stroke}%
\end{pgfscope}%
\begin{pgfscope}%
\definecolor{textcolor}{rgb}{0.150000,0.150000,0.150000}%
\pgfsetstrokecolor{textcolor}%
\pgfsetfillcolor{textcolor}%
\pgftext[x=4.836352in,y=0.531691in,,top]{\color{textcolor}{\sffamily\fontsize{11.000000}{13.200000}\selectfont\catcode`\^=\active\def^{\ifmmode\sp\else\^{}\fi}\catcode`\%=\active\def%{\%}3000}}%
\end{pgfscope}%
\begin{pgfscope}%
\definecolor{textcolor}{rgb}{0.150000,0.150000,0.150000}%
\pgfsetstrokecolor{textcolor}%
\pgfsetfillcolor{textcolor}%
\pgftext[x=3.141892in,y=0.336413in,,top]{\color{textcolor}{\sffamily\fontsize{12.000000}{14.400000}\selectfont\catcode`\^=\active\def^{\ifmmode\sp\else\^{}\fi}\catcode`\%=\active\def%{\%}Time (s)}}%
\end{pgfscope}%
\begin{pgfscope}%
\pgfpathrectangle{\pgfqpoint{0.863783in}{0.663635in}}{\pgfqpoint{4.556217in}{1.191684in}}%
\pgfusepath{clip}%
\pgfsetroundcap%
\pgfsetroundjoin%
\pgfsetlinewidth{1.003750pt}%
\definecolor{currentstroke}{rgb}{1.000000,1.000000,1.000000}%
\pgfsetstrokecolor{currentstroke}%
\pgfsetdash{}{0pt}%
\pgfpathmoveto{\pgfqpoint{0.863783in}{0.961556in}}%
\pgfpathlineto{\pgfqpoint{5.420000in}{0.961556in}}%
\pgfusepath{stroke}%
\end{pgfscope}%
\begin{pgfscope}%
\definecolor{textcolor}{rgb}{0.150000,0.150000,0.150000}%
\pgfsetstrokecolor{textcolor}%
\pgfsetfillcolor{textcolor}%
\pgftext[x=0.391968in, y=0.906876in, left, base]{\color{textcolor}{\sffamily\fontsize{11.000000}{13.200000}\selectfont\catcode`\^=\active\def^{\ifmmode\sp\else\^{}\fi}\catcode`\%=\active\def%{\%}4000}}%
\end{pgfscope}%
\begin{pgfscope}%
\pgfpathrectangle{\pgfqpoint{0.863783in}{0.663635in}}{\pgfqpoint{4.556217in}{1.191684in}}%
\pgfusepath{clip}%
\pgfsetroundcap%
\pgfsetroundjoin%
\pgfsetlinewidth{1.003750pt}%
\definecolor{currentstroke}{rgb}{1.000000,1.000000,1.000000}%
\pgfsetstrokecolor{currentstroke}%
\pgfsetdash{}{0pt}%
\pgfpathmoveto{\pgfqpoint{0.863783in}{1.557398in}}%
\pgfpathlineto{\pgfqpoint{5.420000in}{1.557398in}}%
\pgfusepath{stroke}%
\end{pgfscope}%
\begin{pgfscope}%
\definecolor{textcolor}{rgb}{0.150000,0.150000,0.150000}%
\pgfsetstrokecolor{textcolor}%
\pgfsetfillcolor{textcolor}%
\pgftext[x=0.391968in, y=1.502718in, left, base]{\color{textcolor}{\sffamily\fontsize{11.000000}{13.200000}\selectfont\catcode`\^=\active\def^{\ifmmode\sp\else\^{}\fi}\catcode`\%=\active\def%{\%}6000}}%
\end{pgfscope}%
\begin{pgfscope}%
\definecolor{textcolor}{rgb}{0.150000,0.150000,0.150000}%
\pgfsetstrokecolor{textcolor}%
\pgfsetfillcolor{textcolor}%
\pgftext[x=0.336413in,y=1.259477in,,bottom,rotate=90.000000]{\color{textcolor}{\sffamily\fontsize{12.000000}{14.400000}\selectfont\catcode`\^=\active\def^{\ifmmode\sp\else\^{}\fi}\catcode`\%=\active\def%{\%}Memory Limits (MiB)}}%
\end{pgfscope}%
\begin{pgfscope}%
\pgfpathrectangle{\pgfqpoint{0.863783in}{0.663635in}}{\pgfqpoint{4.556217in}{1.191684in}}%
\pgfusepath{clip}%
\pgfsetbuttcap%
\pgfsetroundjoin%
\pgfsetlinewidth{1.505625pt}%
\definecolor{currentstroke}{rgb}{1.000000,0.647059,0.000000}%
\pgfsetstrokecolor{currentstroke}%
\pgfsetdash{{1.500000pt}{2.475000pt}}{0.000000pt}%
\pgfpathmoveto{\pgfqpoint{1.427349in}{0.663635in}}%
\pgfpathlineto{\pgfqpoint{1.427349in}{1.855319in}}%
\pgfusepath{stroke}%
\end{pgfscope}%
\begin{pgfscope}%
\pgfpathrectangle{\pgfqpoint{0.863783in}{0.663635in}}{\pgfqpoint{4.556217in}{1.191684in}}%
\pgfusepath{clip}%
\pgfsetbuttcap%
\pgfsetroundjoin%
\pgfsetlinewidth{1.505625pt}%
\definecolor{currentstroke}{rgb}{1.000000,0.647059,0.000000}%
\pgfsetstrokecolor{currentstroke}%
\pgfsetdash{{1.500000pt}{2.475000pt}}{0.000000pt}%
\pgfpathmoveto{\pgfqpoint{2.186718in}{0.663635in}}%
\pgfpathlineto{\pgfqpoint{2.186718in}{1.855319in}}%
\pgfusepath{stroke}%
\end{pgfscope}%
\begin{pgfscope}%
\pgfpathrectangle{\pgfqpoint{0.863783in}{0.663635in}}{\pgfqpoint{4.556217in}{1.191684in}}%
\pgfusepath{clip}%
\pgfsetbuttcap%
\pgfsetroundjoin%
\pgfsetlinewidth{1.505625pt}%
\definecolor{currentstroke}{rgb}{1.000000,0.647059,0.000000}%
\pgfsetstrokecolor{currentstroke}%
\pgfsetdash{{1.500000pt}{2.475000pt}}{0.000000pt}%
\pgfpathmoveto{\pgfqpoint{2.941067in}{0.663635in}}%
\pgfpathlineto{\pgfqpoint{2.941067in}{1.855319in}}%
\pgfusepath{stroke}%
\end{pgfscope}%
\begin{pgfscope}%
\pgfpathrectangle{\pgfqpoint{0.863783in}{0.663635in}}{\pgfqpoint{4.556217in}{1.191684in}}%
\pgfusepath{clip}%
\pgfsetbuttcap%
\pgfsetroundjoin%
\pgfsetlinewidth{1.505625pt}%
\definecolor{currentstroke}{rgb}{1.000000,0.647059,0.000000}%
\pgfsetstrokecolor{currentstroke}%
\pgfsetdash{{1.500000pt}{2.475000pt}}{0.000000pt}%
\pgfpathmoveto{\pgfqpoint{3.696671in}{0.663635in}}%
\pgfpathlineto{\pgfqpoint{3.696671in}{1.855319in}}%
\pgfusepath{stroke}%
\end{pgfscope}%
\begin{pgfscope}%
\pgfpathrectangle{\pgfqpoint{0.863783in}{0.663635in}}{\pgfqpoint{4.556217in}{1.191684in}}%
\pgfusepath{clip}%
\pgfsetbuttcap%
\pgfsetroundjoin%
\pgfsetlinewidth{1.505625pt}%
\definecolor{currentstroke}{rgb}{1.000000,0.647059,0.000000}%
\pgfsetstrokecolor{currentstroke}%
\pgfsetdash{{1.500000pt}{2.475000pt}}{0.000000pt}%
\pgfpathmoveto{\pgfqpoint{4.449764in}{0.663635in}}%
\pgfpathlineto{\pgfqpoint{4.449764in}{1.855319in}}%
\pgfusepath{stroke}%
\end{pgfscope}%
\begin{pgfscope}%
\pgfpathrectangle{\pgfqpoint{0.863783in}{0.663635in}}{\pgfqpoint{4.556217in}{1.191684in}}%
\pgfusepath{clip}%
\pgfsetroundcap%
\pgfsetroundjoin%
\pgfsetlinewidth{1.505625pt}%
\definecolor{currentstroke}{rgb}{0.298039,0.447059,0.690196}%
\pgfsetstrokecolor{currentstroke}%
\pgfsetdash{}{0pt}%
\pgfpathmoveto{\pgfqpoint{1.070884in}{1.663296in}}%
\pgfpathlineto{\pgfqpoint{1.434879in}{1.663296in}}%
\pgfpathlineto{\pgfqpoint{1.441155in}{0.988528in}}%
\pgfpathlineto{\pgfqpoint{2.457832in}{0.988528in}}%
\pgfpathlineto{\pgfqpoint{2.464107in}{1.276554in}}%
\pgfpathlineto{\pgfqpoint{5.212899in}{1.276554in}}%
\pgfpathlineto{\pgfqpoint{5.212899in}{1.276554in}}%
\pgfusepath{stroke}%
\end{pgfscope}%
\begin{pgfscope}%
\pgfsetrectcap%
\pgfsetmiterjoin%
\pgfsetlinewidth{1.254687pt}%
\definecolor{currentstroke}{rgb}{1.000000,1.000000,1.000000}%
\pgfsetstrokecolor{currentstroke}%
\pgfsetdash{}{0pt}%
\pgfpathmoveto{\pgfqpoint{0.863783in}{0.663635in}}%
\pgfpathlineto{\pgfqpoint{0.863783in}{1.855319in}}%
\pgfusepath{stroke}%
\end{pgfscope}%
\begin{pgfscope}%
\pgfsetrectcap%
\pgfsetmiterjoin%
\pgfsetlinewidth{1.254687pt}%
\definecolor{currentstroke}{rgb}{1.000000,1.000000,1.000000}%
\pgfsetstrokecolor{currentstroke}%
\pgfsetdash{}{0pt}%
\pgfpathmoveto{\pgfqpoint{5.420000in}{0.663635in}}%
\pgfpathlineto{\pgfqpoint{5.420000in}{1.855319in}}%
\pgfusepath{stroke}%
\end{pgfscope}%
\begin{pgfscope}%
\pgfsetrectcap%
\pgfsetmiterjoin%
\pgfsetlinewidth{1.254687pt}%
\definecolor{currentstroke}{rgb}{1.000000,1.000000,1.000000}%
\pgfsetstrokecolor{currentstroke}%
\pgfsetdash{}{0pt}%
\pgfpathmoveto{\pgfqpoint{0.863783in}{0.663635in}}%
\pgfpathlineto{\pgfqpoint{5.420000in}{0.663635in}}%
\pgfusepath{stroke}%
\end{pgfscope}%
\begin{pgfscope}%
\pgfsetrectcap%
\pgfsetmiterjoin%
\pgfsetlinewidth{1.254687pt}%
\definecolor{currentstroke}{rgb}{1.000000,1.000000,1.000000}%
\pgfsetstrokecolor{currentstroke}%
\pgfsetdash{}{0pt}%
\pgfpathmoveto{\pgfqpoint{0.863783in}{1.855319in}}%
\pgfpathlineto{\pgfqpoint{5.420000in}{1.855319in}}%
\pgfusepath{stroke}%
\end{pgfscope}%
\begin{pgfscope}%
\pgfsetbuttcap%
\pgfsetmiterjoin%
\definecolor{currentfill}{rgb}{0.917647,0.917647,0.949020}%
\pgfsetfillcolor{currentfill}%
\pgfsetfillopacity{0.800000}%
\pgfsetlinewidth{1.003750pt}%
\definecolor{currentstroke}{rgb}{0.800000,0.800000,0.800000}%
\pgfsetstrokecolor{currentstroke}%
\pgfsetstrokeopacity{0.800000}%
\pgfsetdash{}{0pt}%
\pgfpathmoveto{\pgfqpoint{4.518121in}{3.752637in}}%
\pgfpathlineto{\pgfqpoint{5.522222in}{3.752637in}}%
\pgfpathquadraticcurveto{\pgfqpoint{5.544444in}{3.752637in}}{\pgfqpoint{5.544444in}{3.774859in}}%
\pgfpathlineto{\pgfqpoint{5.544444in}{3.922222in}}%
\pgfpathquadraticcurveto{\pgfqpoint{5.544444in}{3.944444in}}{\pgfqpoint{5.522222in}{3.944444in}}%
\pgfpathlineto{\pgfqpoint{4.518121in}{3.944444in}}%
\pgfpathquadraticcurveto{\pgfqpoint{4.495898in}{3.944444in}}{\pgfqpoint{4.495898in}{3.922222in}}%
\pgfpathlineto{\pgfqpoint{4.495898in}{3.774859in}}%
\pgfpathquadraticcurveto{\pgfqpoint{4.495898in}{3.752637in}}{\pgfqpoint{4.518121in}{3.752637in}}%
\pgfpathlineto{\pgfqpoint{4.518121in}{3.752637in}}%
\pgfpathclose%
\pgfusepath{stroke,fill}%
\end{pgfscope}%
\begin{pgfscope}%
\pgfsetbuttcap%
\pgfsetroundjoin%
\pgfsetlinewidth{1.505625pt}%
\definecolor{currentstroke}{rgb}{1.000000,0.647059,0.000000}%
\pgfsetstrokecolor{currentstroke}%
\pgfsetdash{{1.500000pt}{2.475000pt}}{0.000000pt}%
\pgfpathmoveto{\pgfqpoint{4.540343in}{3.859353in}}%
\pgfpathlineto{\pgfqpoint{4.651454in}{3.859353in}}%
\pgfpathlineto{\pgfqpoint{4.762565in}{3.859353in}}%
\pgfusepath{stroke}%
\end{pgfscope}%
\begin{pgfscope}%
\definecolor{textcolor}{rgb}{0.150000,0.150000,0.150000}%
\pgfsetstrokecolor{textcolor}%
\pgfsetfillcolor{textcolor}%
\pgftext[x=4.851454in,y=3.820464in,left,base]{\color{textcolor}{\sffamily\fontsize{8.000000}{9.600000}\selectfont\catcode`\^=\active\def^{\ifmmode\sp\else\^{}\fi}\catcode`\%=\active\def%{\%}scaling event}}%
\end{pgfscope}%
\end{pgfpicture}%
\makeatother%
\endgroup%

    \caption{CPU and memory limits while the diagonal elasticity strategy controller is running}
    \label{fig:diagonal-elasticity-limits}
\end{figure}
