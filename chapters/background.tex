\chapter{Background}
\label{ch:background}

Over the last years and decades cloud computing emerged as a paradigm which allows customers to receive compute power in a pay-as-you-go and on-demand manner.

This chapter introduces some terminology and concepts that are used throughout this thesis. First the cloud computing concepts are defined and then the used framework is introduced.

\section{Elasticity in Cloud Computing}
\label{sec:elasticity}

Elasticity is one of the core concepts that solves a big problem of cloud computing: providing limited resources for potentially unlimited use. The solution is to scale workloads up and down as needed, to claim resources when bigger load is experienced and release resources when they are not needed, therefore making them available to other workloads.

The term elasticity in computing is conceptually similiar to the term in physics. Wikipedia, for example, defines elasticity as follows: ``In physics and materials science, elasticity is the ability of a body to resist a distorting influence and to return to its original size and shape when that influence or force is removed. Solid objects will deform when adequate loads are applied to them; if the material is elastic, the object will return to its initial shape and size after removal.''\footnote{\url{https://en.wikipedia.org/wiki/Elasticity_(physics)}}

The formula - which takes a more mathematical approach - of elasticity can be defined as \[ e(Y, X) = \diff{Y}{X} \frac{X}{Y}, \] where \(e(Y, X)\) is the elasticity of \(Y\) with respect to \(X\) \cite{dustdarPrinciplesElasticProcesses2011}.

To illustrate this imagine an application that serves some content to its customers. These customers typically interact with the application during the day. This means that the application experiences significantly less load during the night. Once people wake up in the morning the load rises until it peaks in the afternoon. Then the load falls again when people go to sleep in the evening. Using this example it can be seen in \cref{fig:elasticity-application-no-scaling} that during the night the resources of the application are overprovisioned and during the day the resoures are underprovisioned.

\begin{figure}
    \centering
    %% Creator: Matplotlib, PGF backend
%%
%% To include the figure in your LaTeX document, write
%%   \input{<filename>.pgf}
%%
%% Make sure the required packages are loaded in your preamble
%%   \usepackage{pgf}
%%
%% Also ensure that all the required font packages are loaded; for instance,
%% the lmodern package is sometimes necessary when using math font.
%%   \usepackage{lmodern}
%%
%% Figures using additional raster images can only be included by \input if
%% they are in the same directory as the main LaTeX file. For loading figures
%% from other directories you can use the `import` package
%%   \usepackage{import}
%%
%% and then include the figures with
%%   \import{<path to file>}{<filename>.pgf}
%%
%% Matplotlib used the following preamble
%%   \def\mathdefault#1{#1}
%%   \everymath=\expandafter{\the\everymath\displaystyle}
%%   
%%   \usepackage{fontspec}
%%   \setmainfont{DejaVuSerif.ttf}[Path=\detokenize{/Users/nkratky/private/polaris-elasticity-strategies/test/scripts/.venv/lib/python3.11/site-packages/matplotlib/mpl-data/fonts/ttf/}]
%%   \setsansfont{Arial.ttf}[Path=\detokenize{/System/Library/Fonts/Supplemental/}]
%%   \setmonofont{DejaVuSansMono.ttf}[Path=\detokenize{/Users/nkratky/private/polaris-elasticity-strategies/test/scripts/.venv/lib/python3.11/site-packages/matplotlib/mpl-data/fonts/ttf/}]
%%   \makeatletter\@ifpackageloaded{underscore}{}{\usepackage[strings]{underscore}}\makeatother
%%
\begingroup%
\makeatletter%
\begin{pgfpicture}%
\pgfpathrectangle{\pgfpointorigin}{\pgfqpoint{5.600000in}{4.500000in}}%
\pgfusepath{use as bounding box, clip}%
\begin{pgfscope}%
\pgfsetbuttcap%
\pgfsetmiterjoin%
\definecolor{currentfill}{rgb}{1.000000,1.000000,1.000000}%
\pgfsetfillcolor{currentfill}%
\pgfsetlinewidth{0.000000pt}%
\definecolor{currentstroke}{rgb}{1.000000,1.000000,1.000000}%
\pgfsetstrokecolor{currentstroke}%
\pgfsetdash{}{0pt}%
\pgfpathmoveto{\pgfqpoint{0.000000in}{0.000000in}}%
\pgfpathlineto{\pgfqpoint{5.600000in}{0.000000in}}%
\pgfpathlineto{\pgfqpoint{5.600000in}{4.500000in}}%
\pgfpathlineto{\pgfqpoint{0.000000in}{4.500000in}}%
\pgfpathlineto{\pgfqpoint{0.000000in}{0.000000in}}%
\pgfpathclose%
\pgfusepath{fill}%
\end{pgfscope}%
\begin{pgfscope}%
\pgfsetbuttcap%
\pgfsetmiterjoin%
\definecolor{currentfill}{rgb}{0.917647,0.917647,0.949020}%
\pgfsetfillcolor{currentfill}%
\pgfsetlinewidth{0.000000pt}%
\definecolor{currentstroke}{rgb}{0.000000,0.000000,0.000000}%
\pgfsetstrokecolor{currentstroke}%
\pgfsetstrokeopacity{0.000000}%
\pgfsetdash{}{0pt}%
\pgfpathmoveto{\pgfqpoint{0.700000in}{0.495000in}}%
\pgfpathlineto{\pgfqpoint{5.040000in}{0.495000in}}%
\pgfpathlineto{\pgfqpoint{5.040000in}{3.960000in}}%
\pgfpathlineto{\pgfqpoint{0.700000in}{3.960000in}}%
\pgfpathlineto{\pgfqpoint{0.700000in}{0.495000in}}%
\pgfpathclose%
\pgfusepath{fill}%
\end{pgfscope}%
\begin{pgfscope}%
\pgfpathrectangle{\pgfqpoint{0.700000in}{0.495000in}}{\pgfqpoint{4.340000in}{3.465000in}}%
\pgfusepath{clip}%
\pgfsetroundcap%
\pgfsetroundjoin%
\pgfsetlinewidth{1.003750pt}%
\definecolor{currentstroke}{rgb}{1.000000,1.000000,1.000000}%
\pgfsetstrokecolor{currentstroke}%
\pgfsetdash{}{0pt}%
\pgfpathmoveto{\pgfqpoint{0.897273in}{0.495000in}}%
\pgfpathlineto{\pgfqpoint{0.897273in}{3.960000in}}%
\pgfusepath{stroke}%
\end{pgfscope}%
\begin{pgfscope}%
\definecolor{textcolor}{rgb}{0.150000,0.150000,0.150000}%
\pgfsetstrokecolor{textcolor}%
\pgfsetfillcolor{textcolor}%
\pgftext[x=0.897273in,y=0.363056in,,top]{\color{textcolor}{\sffamily\fontsize{11.000000}{13.200000}\selectfont\catcode`\^=\active\def^{\ifmmode\sp\else\^{}\fi}\catcode`\%=\active\def%{\%}00:00}}%
\end{pgfscope}%
\begin{pgfscope}%
\pgfpathrectangle{\pgfqpoint{0.700000in}{0.495000in}}{\pgfqpoint{4.340000in}{3.465000in}}%
\pgfusepath{clip}%
\pgfsetroundcap%
\pgfsetroundjoin%
\pgfsetlinewidth{1.003750pt}%
\definecolor{currentstroke}{rgb}{1.000000,1.000000,1.000000}%
\pgfsetstrokecolor{currentstroke}%
\pgfsetdash{}{0pt}%
\pgfpathmoveto{\pgfqpoint{1.390455in}{0.495000in}}%
\pgfpathlineto{\pgfqpoint{1.390455in}{3.960000in}}%
\pgfusepath{stroke}%
\end{pgfscope}%
\begin{pgfscope}%
\definecolor{textcolor}{rgb}{0.150000,0.150000,0.150000}%
\pgfsetstrokecolor{textcolor}%
\pgfsetfillcolor{textcolor}%
\pgftext[x=1.390455in,y=0.363056in,,top]{\color{textcolor}{\sffamily\fontsize{11.000000}{13.200000}\selectfont\catcode`\^=\active\def^{\ifmmode\sp\else\^{}\fi}\catcode`\%=\active\def%{\%}03:00}}%
\end{pgfscope}%
\begin{pgfscope}%
\pgfpathrectangle{\pgfqpoint{0.700000in}{0.495000in}}{\pgfqpoint{4.340000in}{3.465000in}}%
\pgfusepath{clip}%
\pgfsetroundcap%
\pgfsetroundjoin%
\pgfsetlinewidth{1.003750pt}%
\definecolor{currentstroke}{rgb}{1.000000,1.000000,1.000000}%
\pgfsetstrokecolor{currentstroke}%
\pgfsetdash{}{0pt}%
\pgfpathmoveto{\pgfqpoint{1.883636in}{0.495000in}}%
\pgfpathlineto{\pgfqpoint{1.883636in}{3.960000in}}%
\pgfusepath{stroke}%
\end{pgfscope}%
\begin{pgfscope}%
\definecolor{textcolor}{rgb}{0.150000,0.150000,0.150000}%
\pgfsetstrokecolor{textcolor}%
\pgfsetfillcolor{textcolor}%
\pgftext[x=1.883636in,y=0.363056in,,top]{\color{textcolor}{\sffamily\fontsize{11.000000}{13.200000}\selectfont\catcode`\^=\active\def^{\ifmmode\sp\else\^{}\fi}\catcode`\%=\active\def%{\%}06:00}}%
\end{pgfscope}%
\begin{pgfscope}%
\pgfpathrectangle{\pgfqpoint{0.700000in}{0.495000in}}{\pgfqpoint{4.340000in}{3.465000in}}%
\pgfusepath{clip}%
\pgfsetroundcap%
\pgfsetroundjoin%
\pgfsetlinewidth{1.003750pt}%
\definecolor{currentstroke}{rgb}{1.000000,1.000000,1.000000}%
\pgfsetstrokecolor{currentstroke}%
\pgfsetdash{}{0pt}%
\pgfpathmoveto{\pgfqpoint{2.376818in}{0.495000in}}%
\pgfpathlineto{\pgfqpoint{2.376818in}{3.960000in}}%
\pgfusepath{stroke}%
\end{pgfscope}%
\begin{pgfscope}%
\definecolor{textcolor}{rgb}{0.150000,0.150000,0.150000}%
\pgfsetstrokecolor{textcolor}%
\pgfsetfillcolor{textcolor}%
\pgftext[x=2.376818in,y=0.363056in,,top]{\color{textcolor}{\sffamily\fontsize{11.000000}{13.200000}\selectfont\catcode`\^=\active\def^{\ifmmode\sp\else\^{}\fi}\catcode`\%=\active\def%{\%}09:00}}%
\end{pgfscope}%
\begin{pgfscope}%
\pgfpathrectangle{\pgfqpoint{0.700000in}{0.495000in}}{\pgfqpoint{4.340000in}{3.465000in}}%
\pgfusepath{clip}%
\pgfsetroundcap%
\pgfsetroundjoin%
\pgfsetlinewidth{1.003750pt}%
\definecolor{currentstroke}{rgb}{1.000000,1.000000,1.000000}%
\pgfsetstrokecolor{currentstroke}%
\pgfsetdash{}{0pt}%
\pgfpathmoveto{\pgfqpoint{2.870000in}{0.495000in}}%
\pgfpathlineto{\pgfqpoint{2.870000in}{3.960000in}}%
\pgfusepath{stroke}%
\end{pgfscope}%
\begin{pgfscope}%
\definecolor{textcolor}{rgb}{0.150000,0.150000,0.150000}%
\pgfsetstrokecolor{textcolor}%
\pgfsetfillcolor{textcolor}%
\pgftext[x=2.870000in,y=0.363056in,,top]{\color{textcolor}{\sffamily\fontsize{11.000000}{13.200000}\selectfont\catcode`\^=\active\def^{\ifmmode\sp\else\^{}\fi}\catcode`\%=\active\def%{\%}12:00}}%
\end{pgfscope}%
\begin{pgfscope}%
\pgfpathrectangle{\pgfqpoint{0.700000in}{0.495000in}}{\pgfqpoint{4.340000in}{3.465000in}}%
\pgfusepath{clip}%
\pgfsetroundcap%
\pgfsetroundjoin%
\pgfsetlinewidth{1.003750pt}%
\definecolor{currentstroke}{rgb}{1.000000,1.000000,1.000000}%
\pgfsetstrokecolor{currentstroke}%
\pgfsetdash{}{0pt}%
\pgfpathmoveto{\pgfqpoint{3.363182in}{0.495000in}}%
\pgfpathlineto{\pgfqpoint{3.363182in}{3.960000in}}%
\pgfusepath{stroke}%
\end{pgfscope}%
\begin{pgfscope}%
\definecolor{textcolor}{rgb}{0.150000,0.150000,0.150000}%
\pgfsetstrokecolor{textcolor}%
\pgfsetfillcolor{textcolor}%
\pgftext[x=3.363182in,y=0.363056in,,top]{\color{textcolor}{\sffamily\fontsize{11.000000}{13.200000}\selectfont\catcode`\^=\active\def^{\ifmmode\sp\else\^{}\fi}\catcode`\%=\active\def%{\%}15:00}}%
\end{pgfscope}%
\begin{pgfscope}%
\pgfpathrectangle{\pgfqpoint{0.700000in}{0.495000in}}{\pgfqpoint{4.340000in}{3.465000in}}%
\pgfusepath{clip}%
\pgfsetroundcap%
\pgfsetroundjoin%
\pgfsetlinewidth{1.003750pt}%
\definecolor{currentstroke}{rgb}{1.000000,1.000000,1.000000}%
\pgfsetstrokecolor{currentstroke}%
\pgfsetdash{}{0pt}%
\pgfpathmoveto{\pgfqpoint{3.856364in}{0.495000in}}%
\pgfpathlineto{\pgfqpoint{3.856364in}{3.960000in}}%
\pgfusepath{stroke}%
\end{pgfscope}%
\begin{pgfscope}%
\definecolor{textcolor}{rgb}{0.150000,0.150000,0.150000}%
\pgfsetstrokecolor{textcolor}%
\pgfsetfillcolor{textcolor}%
\pgftext[x=3.856364in,y=0.363056in,,top]{\color{textcolor}{\sffamily\fontsize{11.000000}{13.200000}\selectfont\catcode`\^=\active\def^{\ifmmode\sp\else\^{}\fi}\catcode`\%=\active\def%{\%}18:00}}%
\end{pgfscope}%
\begin{pgfscope}%
\pgfpathrectangle{\pgfqpoint{0.700000in}{0.495000in}}{\pgfqpoint{4.340000in}{3.465000in}}%
\pgfusepath{clip}%
\pgfsetroundcap%
\pgfsetroundjoin%
\pgfsetlinewidth{1.003750pt}%
\definecolor{currentstroke}{rgb}{1.000000,1.000000,1.000000}%
\pgfsetstrokecolor{currentstroke}%
\pgfsetdash{}{0pt}%
\pgfpathmoveto{\pgfqpoint{4.349545in}{0.495000in}}%
\pgfpathlineto{\pgfqpoint{4.349545in}{3.960000in}}%
\pgfusepath{stroke}%
\end{pgfscope}%
\begin{pgfscope}%
\definecolor{textcolor}{rgb}{0.150000,0.150000,0.150000}%
\pgfsetstrokecolor{textcolor}%
\pgfsetfillcolor{textcolor}%
\pgftext[x=4.349545in,y=0.363056in,,top]{\color{textcolor}{\sffamily\fontsize{11.000000}{13.200000}\selectfont\catcode`\^=\active\def^{\ifmmode\sp\else\^{}\fi}\catcode`\%=\active\def%{\%}21:00}}%
\end{pgfscope}%
\begin{pgfscope}%
\pgfpathrectangle{\pgfqpoint{0.700000in}{0.495000in}}{\pgfqpoint{4.340000in}{3.465000in}}%
\pgfusepath{clip}%
\pgfsetroundcap%
\pgfsetroundjoin%
\pgfsetlinewidth{1.003750pt}%
\definecolor{currentstroke}{rgb}{1.000000,1.000000,1.000000}%
\pgfsetstrokecolor{currentstroke}%
\pgfsetdash{}{0pt}%
\pgfpathmoveto{\pgfqpoint{4.842727in}{0.495000in}}%
\pgfpathlineto{\pgfqpoint{4.842727in}{3.960000in}}%
\pgfusepath{stroke}%
\end{pgfscope}%
\begin{pgfscope}%
\definecolor{textcolor}{rgb}{0.150000,0.150000,0.150000}%
\pgfsetstrokecolor{textcolor}%
\pgfsetfillcolor{textcolor}%
\pgftext[x=4.842727in,y=0.363056in,,top]{\color{textcolor}{\sffamily\fontsize{11.000000}{13.200000}\selectfont\catcode`\^=\active\def^{\ifmmode\sp\else\^{}\fi}\catcode`\%=\active\def%{\%}24:00}}%
\end{pgfscope}%
\begin{pgfscope}%
\definecolor{textcolor}{rgb}{0.150000,0.150000,0.150000}%
\pgfsetstrokecolor{textcolor}%
\pgfsetfillcolor{textcolor}%
\pgftext[x=2.870000in,y=0.167777in,,top]{\color{textcolor}{\sffamily\fontsize{12.000000}{14.400000}\selectfont\catcode`\^=\active\def^{\ifmmode\sp\else\^{}\fi}\catcode`\%=\active\def%{\%}Time}}%
\end{pgfscope}%
\begin{pgfscope}%
\pgfpathrectangle{\pgfqpoint{0.700000in}{0.495000in}}{\pgfqpoint{4.340000in}{3.465000in}}%
\pgfusepath{clip}%
\pgfsetroundcap%
\pgfsetroundjoin%
\pgfsetlinewidth{1.003750pt}%
\definecolor{currentstroke}{rgb}{1.000000,1.000000,1.000000}%
\pgfsetstrokecolor{currentstroke}%
\pgfsetdash{}{0pt}%
\pgfpathmoveto{\pgfqpoint{0.700000in}{0.677992in}}%
\pgfpathlineto{\pgfqpoint{5.040000in}{0.677992in}}%
\pgfusepath{stroke}%
\end{pgfscope}%
\begin{pgfscope}%
\pgfpathrectangle{\pgfqpoint{0.700000in}{0.495000in}}{\pgfqpoint{4.340000in}{3.465000in}}%
\pgfusepath{clip}%
\pgfsetroundcap%
\pgfsetroundjoin%
\pgfsetlinewidth{1.003750pt}%
\definecolor{currentstroke}{rgb}{1.000000,1.000000,1.000000}%
\pgfsetstrokecolor{currentstroke}%
\pgfsetdash{}{0pt}%
\pgfpathmoveto{\pgfqpoint{0.700000in}{1.242032in}}%
\pgfpathlineto{\pgfqpoint{5.040000in}{1.242032in}}%
\pgfusepath{stroke}%
\end{pgfscope}%
\begin{pgfscope}%
\pgfpathrectangle{\pgfqpoint{0.700000in}{0.495000in}}{\pgfqpoint{4.340000in}{3.465000in}}%
\pgfusepath{clip}%
\pgfsetroundcap%
\pgfsetroundjoin%
\pgfsetlinewidth{1.003750pt}%
\definecolor{currentstroke}{rgb}{1.000000,1.000000,1.000000}%
\pgfsetstrokecolor{currentstroke}%
\pgfsetdash{}{0pt}%
\pgfpathmoveto{\pgfqpoint{0.700000in}{1.806071in}}%
\pgfpathlineto{\pgfqpoint{5.040000in}{1.806071in}}%
\pgfusepath{stroke}%
\end{pgfscope}%
\begin{pgfscope}%
\pgfpathrectangle{\pgfqpoint{0.700000in}{0.495000in}}{\pgfqpoint{4.340000in}{3.465000in}}%
\pgfusepath{clip}%
\pgfsetroundcap%
\pgfsetroundjoin%
\pgfsetlinewidth{1.003750pt}%
\definecolor{currentstroke}{rgb}{1.000000,1.000000,1.000000}%
\pgfsetstrokecolor{currentstroke}%
\pgfsetdash{}{0pt}%
\pgfpathmoveto{\pgfqpoint{0.700000in}{2.370111in}}%
\pgfpathlineto{\pgfqpoint{5.040000in}{2.370111in}}%
\pgfusepath{stroke}%
\end{pgfscope}%
\begin{pgfscope}%
\pgfpathrectangle{\pgfqpoint{0.700000in}{0.495000in}}{\pgfqpoint{4.340000in}{3.465000in}}%
\pgfusepath{clip}%
\pgfsetroundcap%
\pgfsetroundjoin%
\pgfsetlinewidth{1.003750pt}%
\definecolor{currentstroke}{rgb}{1.000000,1.000000,1.000000}%
\pgfsetstrokecolor{currentstroke}%
\pgfsetdash{}{0pt}%
\pgfpathmoveto{\pgfqpoint{0.700000in}{2.934151in}}%
\pgfpathlineto{\pgfqpoint{5.040000in}{2.934151in}}%
\pgfusepath{stroke}%
\end{pgfscope}%
\begin{pgfscope}%
\pgfpathrectangle{\pgfqpoint{0.700000in}{0.495000in}}{\pgfqpoint{4.340000in}{3.465000in}}%
\pgfusepath{clip}%
\pgfsetroundcap%
\pgfsetroundjoin%
\pgfsetlinewidth{1.003750pt}%
\definecolor{currentstroke}{rgb}{1.000000,1.000000,1.000000}%
\pgfsetstrokecolor{currentstroke}%
\pgfsetdash{}{0pt}%
\pgfpathmoveto{\pgfqpoint{0.700000in}{3.498191in}}%
\pgfpathlineto{\pgfqpoint{5.040000in}{3.498191in}}%
\pgfusepath{stroke}%
\end{pgfscope}%
\begin{pgfscope}%
\definecolor{textcolor}{rgb}{0.150000,0.150000,0.150000}%
\pgfsetstrokecolor{textcolor}%
\pgfsetfillcolor{textcolor}%
\pgftext[x=0.512500in,y=2.227500in,,bottom,rotate=90.000000]{\color{textcolor}{\sffamily\fontsize{12.000000}{14.400000}\selectfont\catcode`\^=\active\def^{\ifmmode\sp\else\^{}\fi}\catcode`\%=\active\def%{\%}Resources}}%
\end{pgfscope}%
\begin{pgfscope}%
\pgfpathrectangle{\pgfqpoint{0.700000in}{0.495000in}}{\pgfqpoint{4.340000in}{3.465000in}}%
\pgfusepath{clip}%
\pgfsetbuttcap%
\pgfsetroundjoin%
\definecolor{currentfill}{rgb}{0.172549,0.627451,0.172549}%
\pgfsetfillcolor{currentfill}%
\pgfsetfillopacity{0.300000}%
\pgfsetlinewidth{1.003750pt}%
\definecolor{currentstroke}{rgb}{0.172549,0.627451,0.172549}%
\pgfsetstrokecolor{currentstroke}%
\pgfsetstrokeopacity{0.300000}%
\pgfsetdash{}{0pt}%
\pgfpathmoveto{\pgfqpoint{0.897273in}{2.088091in}}%
\pgfpathlineto{\pgfqpoint{0.897273in}{0.677992in}}%
\pgfpathlineto{\pgfqpoint{0.905179in}{0.677487in}}%
\pgfpathlineto{\pgfqpoint{0.913086in}{0.676935in}}%
\pgfpathlineto{\pgfqpoint{0.920993in}{0.676340in}}%
\pgfpathlineto{\pgfqpoint{0.928900in}{0.675706in}}%
\pgfpathlineto{\pgfqpoint{0.936806in}{0.675034in}}%
\pgfpathlineto{\pgfqpoint{0.944713in}{0.674329in}}%
\pgfpathlineto{\pgfqpoint{0.952620in}{0.673593in}}%
\pgfpathlineto{\pgfqpoint{0.960527in}{0.672830in}}%
\pgfpathlineto{\pgfqpoint{0.968433in}{0.672042in}}%
\pgfpathlineto{\pgfqpoint{0.976340in}{0.671233in}}%
\pgfpathlineto{\pgfqpoint{0.984247in}{0.670405in}}%
\pgfpathlineto{\pgfqpoint{0.992153in}{0.669563in}}%
\pgfpathlineto{\pgfqpoint{1.000060in}{0.668708in}}%
\pgfpathlineto{\pgfqpoint{1.007967in}{0.667845in}}%
\pgfpathlineto{\pgfqpoint{1.015874in}{0.666976in}}%
\pgfpathlineto{\pgfqpoint{1.023780in}{0.666104in}}%
\pgfpathlineto{\pgfqpoint{1.031687in}{0.665233in}}%
\pgfpathlineto{\pgfqpoint{1.039594in}{0.664366in}}%
\pgfpathlineto{\pgfqpoint{1.047500in}{0.663505in}}%
\pgfpathlineto{\pgfqpoint{1.055407in}{0.662654in}}%
\pgfpathlineto{\pgfqpoint{1.063314in}{0.661816in}}%
\pgfpathlineto{\pgfqpoint{1.071221in}{0.660994in}}%
\pgfpathlineto{\pgfqpoint{1.079127in}{0.660191in}}%
\pgfpathlineto{\pgfqpoint{1.087034in}{0.659411in}}%
\pgfpathlineto{\pgfqpoint{1.094941in}{0.658656in}}%
\pgfpathlineto{\pgfqpoint{1.102848in}{0.657930in}}%
\pgfpathlineto{\pgfqpoint{1.110754in}{0.657235in}}%
\pgfpathlineto{\pgfqpoint{1.118661in}{0.656575in}}%
\pgfpathlineto{\pgfqpoint{1.126568in}{0.655953in}}%
\pgfpathlineto{\pgfqpoint{1.134474in}{0.655371in}}%
\pgfpathlineto{\pgfqpoint{1.142381in}{0.654834in}}%
\pgfpathlineto{\pgfqpoint{1.150288in}{0.654345in}}%
\pgfpathlineto{\pgfqpoint{1.158195in}{0.653905in}}%
\pgfpathlineto{\pgfqpoint{1.166101in}{0.653519in}}%
\pgfpathlineto{\pgfqpoint{1.174008in}{0.653190in}}%
\pgfpathlineto{\pgfqpoint{1.181915in}{0.652920in}}%
\pgfpathlineto{\pgfqpoint{1.189821in}{0.652713in}}%
\pgfpathlineto{\pgfqpoint{1.197728in}{0.652572in}}%
\pgfpathlineto{\pgfqpoint{1.205635in}{0.652500in}}%
\pgfpathlineto{\pgfqpoint{1.213542in}{0.652500in}}%
\pgfpathlineto{\pgfqpoint{1.221448in}{0.652576in}}%
\pgfpathlineto{\pgfqpoint{1.229355in}{0.652729in}}%
\pgfpathlineto{\pgfqpoint{1.237262in}{0.652965in}}%
\pgfpathlineto{\pgfqpoint{1.245169in}{0.653285in}}%
\pgfpathlineto{\pgfqpoint{1.253075in}{0.653692in}}%
\pgfpathlineto{\pgfqpoint{1.260982in}{0.654191in}}%
\pgfpathlineto{\pgfqpoint{1.268889in}{0.654783in}}%
\pgfpathlineto{\pgfqpoint{1.276795in}{0.655473in}}%
\pgfpathlineto{\pgfqpoint{1.284702in}{0.656263in}}%
\pgfpathlineto{\pgfqpoint{1.292609in}{0.657156in}}%
\pgfpathlineto{\pgfqpoint{1.300516in}{0.658155in}}%
\pgfpathlineto{\pgfqpoint{1.308422in}{0.659264in}}%
\pgfpathlineto{\pgfqpoint{1.316329in}{0.660486in}}%
\pgfpathlineto{\pgfqpoint{1.324236in}{0.661823in}}%
\pgfpathlineto{\pgfqpoint{1.332142in}{0.663280in}}%
\pgfpathlineto{\pgfqpoint{1.340049in}{0.664858in}}%
\pgfpathlineto{\pgfqpoint{1.347956in}{0.666561in}}%
\pgfpathlineto{\pgfqpoint{1.355863in}{0.668393in}}%
\pgfpathlineto{\pgfqpoint{1.363769in}{0.670356in}}%
\pgfpathlineto{\pgfqpoint{1.371676in}{0.672453in}}%
\pgfpathlineto{\pgfqpoint{1.379583in}{0.674688in}}%
\pgfpathlineto{\pgfqpoint{1.387490in}{0.677064in}}%
\pgfpathlineto{\pgfqpoint{1.395396in}{0.679584in}}%
\pgfpathlineto{\pgfqpoint{1.403303in}{0.682250in}}%
\pgfpathlineto{\pgfqpoint{1.411210in}{0.685067in}}%
\pgfpathlineto{\pgfqpoint{1.419116in}{0.688036in}}%
\pgfpathlineto{\pgfqpoint{1.427023in}{0.691162in}}%
\pgfpathlineto{\pgfqpoint{1.434930in}{0.694447in}}%
\pgfpathlineto{\pgfqpoint{1.442837in}{0.697895in}}%
\pgfpathlineto{\pgfqpoint{1.450743in}{0.701508in}}%
\pgfpathlineto{\pgfqpoint{1.458650in}{0.705291in}}%
\pgfpathlineto{\pgfqpoint{1.466557in}{0.709245in}}%
\pgfpathlineto{\pgfqpoint{1.474463in}{0.713373in}}%
\pgfpathlineto{\pgfqpoint{1.482370in}{0.717680in}}%
\pgfpathlineto{\pgfqpoint{1.490277in}{0.722169in}}%
\pgfpathlineto{\pgfqpoint{1.498184in}{0.726841in}}%
\pgfpathlineto{\pgfqpoint{1.506090in}{0.731701in}}%
\pgfpathlineto{\pgfqpoint{1.513997in}{0.736751in}}%
\pgfpathlineto{\pgfqpoint{1.521904in}{0.741996in}}%
\pgfpathlineto{\pgfqpoint{1.529811in}{0.747436in}}%
\pgfpathlineto{\pgfqpoint{1.537717in}{0.753077in}}%
\pgfpathlineto{\pgfqpoint{1.545624in}{0.758921in}}%
\pgfpathlineto{\pgfqpoint{1.553531in}{0.764971in}}%
\pgfpathlineto{\pgfqpoint{1.561437in}{0.771230in}}%
\pgfpathlineto{\pgfqpoint{1.569344in}{0.777701in}}%
\pgfpathlineto{\pgfqpoint{1.577251in}{0.784388in}}%
\pgfpathlineto{\pgfqpoint{1.585158in}{0.791293in}}%
\pgfpathlineto{\pgfqpoint{1.593064in}{0.798421in}}%
\pgfpathlineto{\pgfqpoint{1.600971in}{0.805773in}}%
\pgfpathlineto{\pgfqpoint{1.608878in}{0.813352in}}%
\pgfpathlineto{\pgfqpoint{1.616784in}{0.821163in}}%
\pgfpathlineto{\pgfqpoint{1.624691in}{0.829208in}}%
\pgfpathlineto{\pgfqpoint{1.632598in}{0.837491in}}%
\pgfpathlineto{\pgfqpoint{1.640505in}{0.846013in}}%
\pgfpathlineto{\pgfqpoint{1.648411in}{0.854780in}}%
\pgfpathlineto{\pgfqpoint{1.656318in}{0.863792in}}%
\pgfpathlineto{\pgfqpoint{1.664225in}{0.873055in}}%
\pgfpathlineto{\pgfqpoint{1.672132in}{0.882570in}}%
\pgfpathlineto{\pgfqpoint{1.680038in}{0.892342in}}%
\pgfpathlineto{\pgfqpoint{1.687945in}{0.902372in}}%
\pgfpathlineto{\pgfqpoint{1.695852in}{0.912665in}}%
\pgfpathlineto{\pgfqpoint{1.703758in}{0.923222in}}%
\pgfpathlineto{\pgfqpoint{1.711665in}{0.934049in}}%
\pgfpathlineto{\pgfqpoint{1.719572in}{0.945147in}}%
\pgfpathlineto{\pgfqpoint{1.727479in}{0.956519in}}%
\pgfpathlineto{\pgfqpoint{1.735385in}{0.968169in}}%
\pgfpathlineto{\pgfqpoint{1.743292in}{0.980101in}}%
\pgfpathlineto{\pgfqpoint{1.751199in}{0.992316in}}%
\pgfpathlineto{\pgfqpoint{1.759105in}{1.004818in}}%
\pgfpathlineto{\pgfqpoint{1.767012in}{1.017610in}}%
\pgfpathlineto{\pgfqpoint{1.774919in}{1.030696in}}%
\pgfpathlineto{\pgfqpoint{1.782826in}{1.044079in}}%
\pgfpathlineto{\pgfqpoint{1.790732in}{1.057760in}}%
\pgfpathlineto{\pgfqpoint{1.798639in}{1.071745in}}%
\pgfpathlineto{\pgfqpoint{1.806546in}{1.086035in}}%
\pgfpathlineto{\pgfqpoint{1.814453in}{1.100635in}}%
\pgfpathlineto{\pgfqpoint{1.822359in}{1.115546in}}%
\pgfpathlineto{\pgfqpoint{1.830266in}{1.130772in}}%
\pgfpathlineto{\pgfqpoint{1.838173in}{1.146317in}}%
\pgfpathlineto{\pgfqpoint{1.846079in}{1.162183in}}%
\pgfpathlineto{\pgfqpoint{1.853986in}{1.178374in}}%
\pgfpathlineto{\pgfqpoint{1.861893in}{1.194892in}}%
\pgfpathlineto{\pgfqpoint{1.869800in}{1.211741in}}%
\pgfpathlineto{\pgfqpoint{1.877706in}{1.228924in}}%
\pgfpathlineto{\pgfqpoint{1.885613in}{1.246443in}}%
\pgfpathlineto{\pgfqpoint{1.893520in}{1.264299in}}%
\pgfpathlineto{\pgfqpoint{1.901426in}{1.282484in}}%
\pgfpathlineto{\pgfqpoint{1.909333in}{1.300988in}}%
\pgfpathlineto{\pgfqpoint{1.917240in}{1.319804in}}%
\pgfpathlineto{\pgfqpoint{1.925147in}{1.338923in}}%
\pgfpathlineto{\pgfqpoint{1.933053in}{1.358335in}}%
\pgfpathlineto{\pgfqpoint{1.940960in}{1.378034in}}%
\pgfpathlineto{\pgfqpoint{1.948867in}{1.398009in}}%
\pgfpathlineto{\pgfqpoint{1.956774in}{1.418252in}}%
\pgfpathlineto{\pgfqpoint{1.964680in}{1.438755in}}%
\pgfpathlineto{\pgfqpoint{1.972587in}{1.459509in}}%
\pgfpathlineto{\pgfqpoint{1.980494in}{1.480506in}}%
\pgfpathlineto{\pgfqpoint{1.988400in}{1.501737in}}%
\pgfpathlineto{\pgfqpoint{1.996307in}{1.523193in}}%
\pgfpathlineto{\pgfqpoint{2.004214in}{1.544865in}}%
\pgfpathlineto{\pgfqpoint{2.012121in}{1.566746in}}%
\pgfpathlineto{\pgfqpoint{2.020027in}{1.588826in}}%
\pgfpathlineto{\pgfqpoint{2.027934in}{1.611097in}}%
\pgfpathlineto{\pgfqpoint{2.035841in}{1.633550in}}%
\pgfpathlineto{\pgfqpoint{2.043747in}{1.656177in}}%
\pgfpathlineto{\pgfqpoint{2.051654in}{1.678969in}}%
\pgfpathlineto{\pgfqpoint{2.059561in}{1.701918in}}%
\pgfpathlineto{\pgfqpoint{2.067468in}{1.725014in}}%
\pgfpathlineto{\pgfqpoint{2.075374in}{1.748250in}}%
\pgfpathlineto{\pgfqpoint{2.083281in}{1.771616in}}%
\pgfpathlineto{\pgfqpoint{2.091188in}{1.795104in}}%
\pgfpathlineto{\pgfqpoint{2.099095in}{1.818706in}}%
\pgfpathlineto{\pgfqpoint{2.107001in}{1.842412in}}%
\pgfpathlineto{\pgfqpoint{2.114908in}{1.866215in}}%
\pgfpathlineto{\pgfqpoint{2.122815in}{1.890105in}}%
\pgfpathlineto{\pgfqpoint{2.130721in}{1.914075in}}%
\pgfpathlineto{\pgfqpoint{2.138628in}{1.938115in}}%
\pgfpathlineto{\pgfqpoint{2.146535in}{1.962216in}}%
\pgfpathlineto{\pgfqpoint{2.154442in}{1.986371in}}%
\pgfpathlineto{\pgfqpoint{2.162348in}{2.010571in}}%
\pgfpathlineto{\pgfqpoint{2.170255in}{2.034806in}}%
\pgfpathlineto{\pgfqpoint{2.178162in}{2.059069in}}%
\pgfpathlineto{\pgfqpoint{2.186069in}{2.083350in}}%
\pgfpathlineto{\pgfqpoint{2.186069in}{2.088091in}}%
\pgfpathlineto{\pgfqpoint{2.186069in}{2.088091in}}%
\pgfpathlineto{\pgfqpoint{2.178162in}{2.088091in}}%
\pgfpathlineto{\pgfqpoint{2.170255in}{2.088091in}}%
\pgfpathlineto{\pgfqpoint{2.162348in}{2.088091in}}%
\pgfpathlineto{\pgfqpoint{2.154442in}{2.088091in}}%
\pgfpathlineto{\pgfqpoint{2.146535in}{2.088091in}}%
\pgfpathlineto{\pgfqpoint{2.138628in}{2.088091in}}%
\pgfpathlineto{\pgfqpoint{2.130721in}{2.088091in}}%
\pgfpathlineto{\pgfqpoint{2.122815in}{2.088091in}}%
\pgfpathlineto{\pgfqpoint{2.114908in}{2.088091in}}%
\pgfpathlineto{\pgfqpoint{2.107001in}{2.088091in}}%
\pgfpathlineto{\pgfqpoint{2.099095in}{2.088091in}}%
\pgfpathlineto{\pgfqpoint{2.091188in}{2.088091in}}%
\pgfpathlineto{\pgfqpoint{2.083281in}{2.088091in}}%
\pgfpathlineto{\pgfqpoint{2.075374in}{2.088091in}}%
\pgfpathlineto{\pgfqpoint{2.067468in}{2.088091in}}%
\pgfpathlineto{\pgfqpoint{2.059561in}{2.088091in}}%
\pgfpathlineto{\pgfqpoint{2.051654in}{2.088091in}}%
\pgfpathlineto{\pgfqpoint{2.043747in}{2.088091in}}%
\pgfpathlineto{\pgfqpoint{2.035841in}{2.088091in}}%
\pgfpathlineto{\pgfqpoint{2.027934in}{2.088091in}}%
\pgfpathlineto{\pgfqpoint{2.020027in}{2.088091in}}%
\pgfpathlineto{\pgfqpoint{2.012121in}{2.088091in}}%
\pgfpathlineto{\pgfqpoint{2.004214in}{2.088091in}}%
\pgfpathlineto{\pgfqpoint{1.996307in}{2.088091in}}%
\pgfpathlineto{\pgfqpoint{1.988400in}{2.088091in}}%
\pgfpathlineto{\pgfqpoint{1.980494in}{2.088091in}}%
\pgfpathlineto{\pgfqpoint{1.972587in}{2.088091in}}%
\pgfpathlineto{\pgfqpoint{1.964680in}{2.088091in}}%
\pgfpathlineto{\pgfqpoint{1.956774in}{2.088091in}}%
\pgfpathlineto{\pgfqpoint{1.948867in}{2.088091in}}%
\pgfpathlineto{\pgfqpoint{1.940960in}{2.088091in}}%
\pgfpathlineto{\pgfqpoint{1.933053in}{2.088091in}}%
\pgfpathlineto{\pgfqpoint{1.925147in}{2.088091in}}%
\pgfpathlineto{\pgfqpoint{1.917240in}{2.088091in}}%
\pgfpathlineto{\pgfqpoint{1.909333in}{2.088091in}}%
\pgfpathlineto{\pgfqpoint{1.901426in}{2.088091in}}%
\pgfpathlineto{\pgfqpoint{1.893520in}{2.088091in}}%
\pgfpathlineto{\pgfqpoint{1.885613in}{2.088091in}}%
\pgfpathlineto{\pgfqpoint{1.877706in}{2.088091in}}%
\pgfpathlineto{\pgfqpoint{1.869800in}{2.088091in}}%
\pgfpathlineto{\pgfqpoint{1.861893in}{2.088091in}}%
\pgfpathlineto{\pgfqpoint{1.853986in}{2.088091in}}%
\pgfpathlineto{\pgfqpoint{1.846079in}{2.088091in}}%
\pgfpathlineto{\pgfqpoint{1.838173in}{2.088091in}}%
\pgfpathlineto{\pgfqpoint{1.830266in}{2.088091in}}%
\pgfpathlineto{\pgfqpoint{1.822359in}{2.088091in}}%
\pgfpathlineto{\pgfqpoint{1.814453in}{2.088091in}}%
\pgfpathlineto{\pgfqpoint{1.806546in}{2.088091in}}%
\pgfpathlineto{\pgfqpoint{1.798639in}{2.088091in}}%
\pgfpathlineto{\pgfqpoint{1.790732in}{2.088091in}}%
\pgfpathlineto{\pgfqpoint{1.782826in}{2.088091in}}%
\pgfpathlineto{\pgfqpoint{1.774919in}{2.088091in}}%
\pgfpathlineto{\pgfqpoint{1.767012in}{2.088091in}}%
\pgfpathlineto{\pgfqpoint{1.759105in}{2.088091in}}%
\pgfpathlineto{\pgfqpoint{1.751199in}{2.088091in}}%
\pgfpathlineto{\pgfqpoint{1.743292in}{2.088091in}}%
\pgfpathlineto{\pgfqpoint{1.735385in}{2.088091in}}%
\pgfpathlineto{\pgfqpoint{1.727479in}{2.088091in}}%
\pgfpathlineto{\pgfqpoint{1.719572in}{2.088091in}}%
\pgfpathlineto{\pgfqpoint{1.711665in}{2.088091in}}%
\pgfpathlineto{\pgfqpoint{1.703758in}{2.088091in}}%
\pgfpathlineto{\pgfqpoint{1.695852in}{2.088091in}}%
\pgfpathlineto{\pgfqpoint{1.687945in}{2.088091in}}%
\pgfpathlineto{\pgfqpoint{1.680038in}{2.088091in}}%
\pgfpathlineto{\pgfqpoint{1.672132in}{2.088091in}}%
\pgfpathlineto{\pgfqpoint{1.664225in}{2.088091in}}%
\pgfpathlineto{\pgfqpoint{1.656318in}{2.088091in}}%
\pgfpathlineto{\pgfqpoint{1.648411in}{2.088091in}}%
\pgfpathlineto{\pgfqpoint{1.640505in}{2.088091in}}%
\pgfpathlineto{\pgfqpoint{1.632598in}{2.088091in}}%
\pgfpathlineto{\pgfqpoint{1.624691in}{2.088091in}}%
\pgfpathlineto{\pgfqpoint{1.616784in}{2.088091in}}%
\pgfpathlineto{\pgfqpoint{1.608878in}{2.088091in}}%
\pgfpathlineto{\pgfqpoint{1.600971in}{2.088091in}}%
\pgfpathlineto{\pgfqpoint{1.593064in}{2.088091in}}%
\pgfpathlineto{\pgfqpoint{1.585158in}{2.088091in}}%
\pgfpathlineto{\pgfqpoint{1.577251in}{2.088091in}}%
\pgfpathlineto{\pgfqpoint{1.569344in}{2.088091in}}%
\pgfpathlineto{\pgfqpoint{1.561437in}{2.088091in}}%
\pgfpathlineto{\pgfqpoint{1.553531in}{2.088091in}}%
\pgfpathlineto{\pgfqpoint{1.545624in}{2.088091in}}%
\pgfpathlineto{\pgfqpoint{1.537717in}{2.088091in}}%
\pgfpathlineto{\pgfqpoint{1.529811in}{2.088091in}}%
\pgfpathlineto{\pgfqpoint{1.521904in}{2.088091in}}%
\pgfpathlineto{\pgfqpoint{1.513997in}{2.088091in}}%
\pgfpathlineto{\pgfqpoint{1.506090in}{2.088091in}}%
\pgfpathlineto{\pgfqpoint{1.498184in}{2.088091in}}%
\pgfpathlineto{\pgfqpoint{1.490277in}{2.088091in}}%
\pgfpathlineto{\pgfqpoint{1.482370in}{2.088091in}}%
\pgfpathlineto{\pgfqpoint{1.474463in}{2.088091in}}%
\pgfpathlineto{\pgfqpoint{1.466557in}{2.088091in}}%
\pgfpathlineto{\pgfqpoint{1.458650in}{2.088091in}}%
\pgfpathlineto{\pgfqpoint{1.450743in}{2.088091in}}%
\pgfpathlineto{\pgfqpoint{1.442837in}{2.088091in}}%
\pgfpathlineto{\pgfqpoint{1.434930in}{2.088091in}}%
\pgfpathlineto{\pgfqpoint{1.427023in}{2.088091in}}%
\pgfpathlineto{\pgfqpoint{1.419116in}{2.088091in}}%
\pgfpathlineto{\pgfqpoint{1.411210in}{2.088091in}}%
\pgfpathlineto{\pgfqpoint{1.403303in}{2.088091in}}%
\pgfpathlineto{\pgfqpoint{1.395396in}{2.088091in}}%
\pgfpathlineto{\pgfqpoint{1.387490in}{2.088091in}}%
\pgfpathlineto{\pgfqpoint{1.379583in}{2.088091in}}%
\pgfpathlineto{\pgfqpoint{1.371676in}{2.088091in}}%
\pgfpathlineto{\pgfqpoint{1.363769in}{2.088091in}}%
\pgfpathlineto{\pgfqpoint{1.355863in}{2.088091in}}%
\pgfpathlineto{\pgfqpoint{1.347956in}{2.088091in}}%
\pgfpathlineto{\pgfqpoint{1.340049in}{2.088091in}}%
\pgfpathlineto{\pgfqpoint{1.332142in}{2.088091in}}%
\pgfpathlineto{\pgfqpoint{1.324236in}{2.088091in}}%
\pgfpathlineto{\pgfqpoint{1.316329in}{2.088091in}}%
\pgfpathlineto{\pgfqpoint{1.308422in}{2.088091in}}%
\pgfpathlineto{\pgfqpoint{1.300516in}{2.088091in}}%
\pgfpathlineto{\pgfqpoint{1.292609in}{2.088091in}}%
\pgfpathlineto{\pgfqpoint{1.284702in}{2.088091in}}%
\pgfpathlineto{\pgfqpoint{1.276795in}{2.088091in}}%
\pgfpathlineto{\pgfqpoint{1.268889in}{2.088091in}}%
\pgfpathlineto{\pgfqpoint{1.260982in}{2.088091in}}%
\pgfpathlineto{\pgfqpoint{1.253075in}{2.088091in}}%
\pgfpathlineto{\pgfqpoint{1.245169in}{2.088091in}}%
\pgfpathlineto{\pgfqpoint{1.237262in}{2.088091in}}%
\pgfpathlineto{\pgfqpoint{1.229355in}{2.088091in}}%
\pgfpathlineto{\pgfqpoint{1.221448in}{2.088091in}}%
\pgfpathlineto{\pgfqpoint{1.213542in}{2.088091in}}%
\pgfpathlineto{\pgfqpoint{1.205635in}{2.088091in}}%
\pgfpathlineto{\pgfqpoint{1.197728in}{2.088091in}}%
\pgfpathlineto{\pgfqpoint{1.189821in}{2.088091in}}%
\pgfpathlineto{\pgfqpoint{1.181915in}{2.088091in}}%
\pgfpathlineto{\pgfqpoint{1.174008in}{2.088091in}}%
\pgfpathlineto{\pgfqpoint{1.166101in}{2.088091in}}%
\pgfpathlineto{\pgfqpoint{1.158195in}{2.088091in}}%
\pgfpathlineto{\pgfqpoint{1.150288in}{2.088091in}}%
\pgfpathlineto{\pgfqpoint{1.142381in}{2.088091in}}%
\pgfpathlineto{\pgfqpoint{1.134474in}{2.088091in}}%
\pgfpathlineto{\pgfqpoint{1.126568in}{2.088091in}}%
\pgfpathlineto{\pgfqpoint{1.118661in}{2.088091in}}%
\pgfpathlineto{\pgfqpoint{1.110754in}{2.088091in}}%
\pgfpathlineto{\pgfqpoint{1.102848in}{2.088091in}}%
\pgfpathlineto{\pgfqpoint{1.094941in}{2.088091in}}%
\pgfpathlineto{\pgfqpoint{1.087034in}{2.088091in}}%
\pgfpathlineto{\pgfqpoint{1.079127in}{2.088091in}}%
\pgfpathlineto{\pgfqpoint{1.071221in}{2.088091in}}%
\pgfpathlineto{\pgfqpoint{1.063314in}{2.088091in}}%
\pgfpathlineto{\pgfqpoint{1.055407in}{2.088091in}}%
\pgfpathlineto{\pgfqpoint{1.047500in}{2.088091in}}%
\pgfpathlineto{\pgfqpoint{1.039594in}{2.088091in}}%
\pgfpathlineto{\pgfqpoint{1.031687in}{2.088091in}}%
\pgfpathlineto{\pgfqpoint{1.023780in}{2.088091in}}%
\pgfpathlineto{\pgfqpoint{1.015874in}{2.088091in}}%
\pgfpathlineto{\pgfqpoint{1.007967in}{2.088091in}}%
\pgfpathlineto{\pgfqpoint{1.000060in}{2.088091in}}%
\pgfpathlineto{\pgfqpoint{0.992153in}{2.088091in}}%
\pgfpathlineto{\pgfqpoint{0.984247in}{2.088091in}}%
\pgfpathlineto{\pgfqpoint{0.976340in}{2.088091in}}%
\pgfpathlineto{\pgfqpoint{0.968433in}{2.088091in}}%
\pgfpathlineto{\pgfqpoint{0.960527in}{2.088091in}}%
\pgfpathlineto{\pgfqpoint{0.952620in}{2.088091in}}%
\pgfpathlineto{\pgfqpoint{0.944713in}{2.088091in}}%
\pgfpathlineto{\pgfqpoint{0.936806in}{2.088091in}}%
\pgfpathlineto{\pgfqpoint{0.928900in}{2.088091in}}%
\pgfpathlineto{\pgfqpoint{0.920993in}{2.088091in}}%
\pgfpathlineto{\pgfqpoint{0.913086in}{2.088091in}}%
\pgfpathlineto{\pgfqpoint{0.905179in}{2.088091in}}%
\pgfpathlineto{\pgfqpoint{0.897273in}{2.088091in}}%
\pgfpathlineto{\pgfqpoint{0.897273in}{2.088091in}}%
\pgfpathclose%
\pgfusepath{stroke,fill}%
\end{pgfscope}%
\begin{pgfscope}%
\pgfpathrectangle{\pgfqpoint{0.700000in}{0.495000in}}{\pgfqpoint{4.340000in}{3.465000in}}%
\pgfusepath{clip}%
\pgfsetbuttcap%
\pgfsetroundjoin%
\definecolor{currentfill}{rgb}{0.172549,0.627451,0.172549}%
\pgfsetfillcolor{currentfill}%
\pgfsetfillopacity{0.300000}%
\pgfsetlinewidth{1.003750pt}%
\definecolor{currentstroke}{rgb}{0.172549,0.627451,0.172549}%
\pgfsetstrokecolor{currentstroke}%
\pgfsetstrokeopacity{0.300000}%
\pgfsetdash{}{0pt}%
\pgfpathmoveto{\pgfqpoint{4.407858in}{2.088091in}}%
\pgfpathlineto{\pgfqpoint{4.407858in}{2.082164in}}%
\pgfpathlineto{\pgfqpoint{4.415764in}{2.043295in}}%
\pgfpathlineto{\pgfqpoint{4.423671in}{2.004536in}}%
\pgfpathlineto{\pgfqpoint{4.431578in}{1.965908in}}%
\pgfpathlineto{\pgfqpoint{4.439484in}{1.927434in}}%
\pgfpathlineto{\pgfqpoint{4.447391in}{1.889134in}}%
\pgfpathlineto{\pgfqpoint{4.455298in}{1.851031in}}%
\pgfpathlineto{\pgfqpoint{4.463205in}{1.813147in}}%
\pgfpathlineto{\pgfqpoint{4.471111in}{1.775503in}}%
\pgfpathlineto{\pgfqpoint{4.479018in}{1.738120in}}%
\pgfpathlineto{\pgfqpoint{4.486925in}{1.701022in}}%
\pgfpathlineto{\pgfqpoint{4.494831in}{1.664229in}}%
\pgfpathlineto{\pgfqpoint{4.502738in}{1.627763in}}%
\pgfpathlineto{\pgfqpoint{4.510645in}{1.591646in}}%
\pgfpathlineto{\pgfqpoint{4.518552in}{1.555900in}}%
\pgfpathlineto{\pgfqpoint{4.526458in}{1.520547in}}%
\pgfpathlineto{\pgfqpoint{4.534365in}{1.485608in}}%
\pgfpathlineto{\pgfqpoint{4.542272in}{1.451104in}}%
\pgfpathlineto{\pgfqpoint{4.550179in}{1.417059in}}%
\pgfpathlineto{\pgfqpoint{4.558085in}{1.383493in}}%
\pgfpathlineto{\pgfqpoint{4.565992in}{1.350428in}}%
\pgfpathlineto{\pgfqpoint{4.573899in}{1.317887in}}%
\pgfpathlineto{\pgfqpoint{4.581805in}{1.285890in}}%
\pgfpathlineto{\pgfqpoint{4.589712in}{1.254459in}}%
\pgfpathlineto{\pgfqpoint{4.597619in}{1.223617in}}%
\pgfpathlineto{\pgfqpoint{4.605526in}{1.193385in}}%
\pgfpathlineto{\pgfqpoint{4.613432in}{1.163785in}}%
\pgfpathlineto{\pgfqpoint{4.621339in}{1.134838in}}%
\pgfpathlineto{\pgfqpoint{4.629246in}{1.106567in}}%
\pgfpathlineto{\pgfqpoint{4.637152in}{1.078993in}}%
\pgfpathlineto{\pgfqpoint{4.645059in}{1.052137in}}%
\pgfpathlineto{\pgfqpoint{4.652966in}{1.026022in}}%
\pgfpathlineto{\pgfqpoint{4.660873in}{1.000669in}}%
\pgfpathlineto{\pgfqpoint{4.668779in}{0.976101in}}%
\pgfpathlineto{\pgfqpoint{4.676686in}{0.952338in}}%
\pgfpathlineto{\pgfqpoint{4.684593in}{0.929403in}}%
\pgfpathlineto{\pgfqpoint{4.692500in}{0.907317in}}%
\pgfpathlineto{\pgfqpoint{4.700406in}{0.886102in}}%
\pgfpathlineto{\pgfqpoint{4.708313in}{0.865780in}}%
\pgfpathlineto{\pgfqpoint{4.716220in}{0.846373in}}%
\pgfpathlineto{\pgfqpoint{4.724126in}{0.827902in}}%
\pgfpathlineto{\pgfqpoint{4.732033in}{0.810389in}}%
\pgfpathlineto{\pgfqpoint{4.739940in}{0.793856in}}%
\pgfpathlineto{\pgfqpoint{4.747847in}{0.778325in}}%
\pgfpathlineto{\pgfqpoint{4.755753in}{0.763817in}}%
\pgfpathlineto{\pgfqpoint{4.763660in}{0.750354in}}%
\pgfpathlineto{\pgfqpoint{4.771567in}{0.737958in}}%
\pgfpathlineto{\pgfqpoint{4.779473in}{0.726650in}}%
\pgfpathlineto{\pgfqpoint{4.787380in}{0.716453in}}%
\pgfpathlineto{\pgfqpoint{4.795287in}{0.707388in}}%
\pgfpathlineto{\pgfqpoint{4.803194in}{0.699477in}}%
\pgfpathlineto{\pgfqpoint{4.811100in}{0.692742in}}%
\pgfpathlineto{\pgfqpoint{4.819007in}{0.687204in}}%
\pgfpathlineto{\pgfqpoint{4.826914in}{0.682885in}}%
\pgfpathlineto{\pgfqpoint{4.834821in}{0.679807in}}%
\pgfpathlineto{\pgfqpoint{4.842727in}{0.677992in}}%
\pgfpathlineto{\pgfqpoint{4.842727in}{2.088091in}}%
\pgfpathlineto{\pgfqpoint{4.842727in}{2.088091in}}%
\pgfpathlineto{\pgfqpoint{4.834821in}{2.088091in}}%
\pgfpathlineto{\pgfqpoint{4.826914in}{2.088091in}}%
\pgfpathlineto{\pgfqpoint{4.819007in}{2.088091in}}%
\pgfpathlineto{\pgfqpoint{4.811100in}{2.088091in}}%
\pgfpathlineto{\pgfqpoint{4.803194in}{2.088091in}}%
\pgfpathlineto{\pgfqpoint{4.795287in}{2.088091in}}%
\pgfpathlineto{\pgfqpoint{4.787380in}{2.088091in}}%
\pgfpathlineto{\pgfqpoint{4.779473in}{2.088091in}}%
\pgfpathlineto{\pgfqpoint{4.771567in}{2.088091in}}%
\pgfpathlineto{\pgfqpoint{4.763660in}{2.088091in}}%
\pgfpathlineto{\pgfqpoint{4.755753in}{2.088091in}}%
\pgfpathlineto{\pgfqpoint{4.747847in}{2.088091in}}%
\pgfpathlineto{\pgfqpoint{4.739940in}{2.088091in}}%
\pgfpathlineto{\pgfqpoint{4.732033in}{2.088091in}}%
\pgfpathlineto{\pgfqpoint{4.724126in}{2.088091in}}%
\pgfpathlineto{\pgfqpoint{4.716220in}{2.088091in}}%
\pgfpathlineto{\pgfqpoint{4.708313in}{2.088091in}}%
\pgfpathlineto{\pgfqpoint{4.700406in}{2.088091in}}%
\pgfpathlineto{\pgfqpoint{4.692500in}{2.088091in}}%
\pgfpathlineto{\pgfqpoint{4.684593in}{2.088091in}}%
\pgfpathlineto{\pgfqpoint{4.676686in}{2.088091in}}%
\pgfpathlineto{\pgfqpoint{4.668779in}{2.088091in}}%
\pgfpathlineto{\pgfqpoint{4.660873in}{2.088091in}}%
\pgfpathlineto{\pgfqpoint{4.652966in}{2.088091in}}%
\pgfpathlineto{\pgfqpoint{4.645059in}{2.088091in}}%
\pgfpathlineto{\pgfqpoint{4.637152in}{2.088091in}}%
\pgfpathlineto{\pgfqpoint{4.629246in}{2.088091in}}%
\pgfpathlineto{\pgfqpoint{4.621339in}{2.088091in}}%
\pgfpathlineto{\pgfqpoint{4.613432in}{2.088091in}}%
\pgfpathlineto{\pgfqpoint{4.605526in}{2.088091in}}%
\pgfpathlineto{\pgfqpoint{4.597619in}{2.088091in}}%
\pgfpathlineto{\pgfqpoint{4.589712in}{2.088091in}}%
\pgfpathlineto{\pgfqpoint{4.581805in}{2.088091in}}%
\pgfpathlineto{\pgfqpoint{4.573899in}{2.088091in}}%
\pgfpathlineto{\pgfqpoint{4.565992in}{2.088091in}}%
\pgfpathlineto{\pgfqpoint{4.558085in}{2.088091in}}%
\pgfpathlineto{\pgfqpoint{4.550179in}{2.088091in}}%
\pgfpathlineto{\pgfqpoint{4.542272in}{2.088091in}}%
\pgfpathlineto{\pgfqpoint{4.534365in}{2.088091in}}%
\pgfpathlineto{\pgfqpoint{4.526458in}{2.088091in}}%
\pgfpathlineto{\pgfqpoint{4.518552in}{2.088091in}}%
\pgfpathlineto{\pgfqpoint{4.510645in}{2.088091in}}%
\pgfpathlineto{\pgfqpoint{4.502738in}{2.088091in}}%
\pgfpathlineto{\pgfqpoint{4.494831in}{2.088091in}}%
\pgfpathlineto{\pgfqpoint{4.486925in}{2.088091in}}%
\pgfpathlineto{\pgfqpoint{4.479018in}{2.088091in}}%
\pgfpathlineto{\pgfqpoint{4.471111in}{2.088091in}}%
\pgfpathlineto{\pgfqpoint{4.463205in}{2.088091in}}%
\pgfpathlineto{\pgfqpoint{4.455298in}{2.088091in}}%
\pgfpathlineto{\pgfqpoint{4.447391in}{2.088091in}}%
\pgfpathlineto{\pgfqpoint{4.439484in}{2.088091in}}%
\pgfpathlineto{\pgfqpoint{4.431578in}{2.088091in}}%
\pgfpathlineto{\pgfqpoint{4.423671in}{2.088091in}}%
\pgfpathlineto{\pgfqpoint{4.415764in}{2.088091in}}%
\pgfpathlineto{\pgfqpoint{4.407858in}{2.088091in}}%
\pgfpathlineto{\pgfqpoint{4.407858in}{2.088091in}}%
\pgfpathclose%
\pgfusepath{stroke,fill}%
\end{pgfscope}%
\begin{pgfscope}%
\pgfpathrectangle{\pgfqpoint{0.700000in}{0.495000in}}{\pgfqpoint{4.340000in}{3.465000in}}%
\pgfusepath{clip}%
\pgfsetbuttcap%
\pgfsetroundjoin%
\definecolor{currentfill}{rgb}{0.839216,0.152941,0.156863}%
\pgfsetfillcolor{currentfill}%
\pgfsetfillopacity{0.300000}%
\pgfsetlinewidth{1.003750pt}%
\definecolor{currentstroke}{rgb}{0.839216,0.152941,0.156863}%
\pgfsetstrokecolor{currentstroke}%
\pgfsetstrokeopacity{0.300000}%
\pgfsetdash{}{0pt}%
\pgfsys@defobject{currentmarker}{\pgfqpoint{2.193975in}{2.088091in}}{\pgfqpoint{4.399951in}{3.802500in}}{%
\pgfpathmoveto{\pgfqpoint{2.193975in}{2.088091in}}%
\pgfpathlineto{\pgfqpoint{2.193975in}{2.107642in}}%
\pgfpathlineto{\pgfqpoint{2.201882in}{2.131936in}}%
\pgfpathlineto{\pgfqpoint{2.209789in}{2.156222in}}%
\pgfpathlineto{\pgfqpoint{2.217695in}{2.180493in}}%
\pgfpathlineto{\pgfqpoint{2.225602in}{2.204739in}}%
\pgfpathlineto{\pgfqpoint{2.233509in}{2.228953in}}%
\pgfpathlineto{\pgfqpoint{2.241416in}{2.253125in}}%
\pgfpathlineto{\pgfqpoint{2.249322in}{2.277247in}}%
\pgfpathlineto{\pgfqpoint{2.257229in}{2.301311in}}%
\pgfpathlineto{\pgfqpoint{2.265136in}{2.325307in}}%
\pgfpathlineto{\pgfqpoint{2.273042in}{2.349228in}}%
\pgfpathlineto{\pgfqpoint{2.280949in}{2.373064in}}%
\pgfpathlineto{\pgfqpoint{2.288856in}{2.396807in}}%
\pgfpathlineto{\pgfqpoint{2.296763in}{2.420448in}}%
\pgfpathlineto{\pgfqpoint{2.304669in}{2.443979in}}%
\pgfpathlineto{\pgfqpoint{2.312576in}{2.467391in}}%
\pgfpathlineto{\pgfqpoint{2.320483in}{2.490676in}}%
\pgfpathlineto{\pgfqpoint{2.328390in}{2.513824in}}%
\pgfpathlineto{\pgfqpoint{2.336296in}{2.536828in}}%
\pgfpathlineto{\pgfqpoint{2.344203in}{2.559679in}}%
\pgfpathlineto{\pgfqpoint{2.352110in}{2.582367in}}%
\pgfpathlineto{\pgfqpoint{2.360016in}{2.604885in}}%
\pgfpathlineto{\pgfqpoint{2.367923in}{2.627224in}}%
\pgfpathlineto{\pgfqpoint{2.375830in}{2.649376in}}%
\pgfpathlineto{\pgfqpoint{2.383737in}{2.671331in}}%
\pgfpathlineto{\pgfqpoint{2.391643in}{2.693079in}}%
\pgfpathlineto{\pgfqpoint{2.399550in}{2.714610in}}%
\pgfpathlineto{\pgfqpoint{2.407457in}{2.735913in}}%
\pgfpathlineto{\pgfqpoint{2.415363in}{2.756979in}}%
\pgfpathlineto{\pgfqpoint{2.423270in}{2.777795in}}%
\pgfpathlineto{\pgfqpoint{2.431177in}{2.798352in}}%
\pgfpathlineto{\pgfqpoint{2.439084in}{2.818640in}}%
\pgfpathlineto{\pgfqpoint{2.446990in}{2.838647in}}%
\pgfpathlineto{\pgfqpoint{2.454897in}{2.858363in}}%
\pgfpathlineto{\pgfqpoint{2.462804in}{2.877778in}}%
\pgfpathlineto{\pgfqpoint{2.470711in}{2.896880in}}%
\pgfpathlineto{\pgfqpoint{2.478617in}{2.915661in}}%
\pgfpathlineto{\pgfqpoint{2.486524in}{2.934108in}}%
\pgfpathlineto{\pgfqpoint{2.494431in}{2.952211in}}%
\pgfpathlineto{\pgfqpoint{2.502337in}{2.969961in}}%
\pgfpathlineto{\pgfqpoint{2.510244in}{2.987346in}}%
\pgfpathlineto{\pgfqpoint{2.518151in}{3.004356in}}%
\pgfpathlineto{\pgfqpoint{2.526058in}{3.020980in}}%
\pgfpathlineto{\pgfqpoint{2.533964in}{3.037208in}}%
\pgfpathlineto{\pgfqpoint{2.541871in}{3.053029in}}%
\pgfpathlineto{\pgfqpoint{2.549778in}{3.068433in}}%
\pgfpathlineto{\pgfqpoint{2.557684in}{3.083409in}}%
\pgfpathlineto{\pgfqpoint{2.565591in}{3.097946in}}%
\pgfpathlineto{\pgfqpoint{2.573498in}{3.112035in}}%
\pgfpathlineto{\pgfqpoint{2.581405in}{3.125664in}}%
\pgfpathlineto{\pgfqpoint{2.589311in}{3.138823in}}%
\pgfpathlineto{\pgfqpoint{2.597218in}{3.151502in}}%
\pgfpathlineto{\pgfqpoint{2.605125in}{3.163690in}}%
\pgfpathlineto{\pgfqpoint{2.613032in}{3.175376in}}%
\pgfpathlineto{\pgfqpoint{2.620938in}{3.186550in}}%
\pgfpathlineto{\pgfqpoint{2.628845in}{3.197201in}}%
\pgfpathlineto{\pgfqpoint{2.636752in}{3.207319in}}%
\pgfpathlineto{\pgfqpoint{2.644658in}{3.216894in}}%
\pgfpathlineto{\pgfqpoint{2.652565in}{3.225914in}}%
\pgfpathlineto{\pgfqpoint{2.660472in}{3.234369in}}%
\pgfpathlineto{\pgfqpoint{2.668379in}{3.242249in}}%
\pgfpathlineto{\pgfqpoint{2.676285in}{3.249543in}}%
\pgfpathlineto{\pgfqpoint{2.684192in}{3.256241in}}%
\pgfpathlineto{\pgfqpoint{2.692099in}{3.262332in}}%
\pgfpathlineto{\pgfqpoint{2.700005in}{3.267805in}}%
\pgfpathlineto{\pgfqpoint{2.707912in}{3.272650in}}%
\pgfpathlineto{\pgfqpoint{2.715819in}{3.276857in}}%
\pgfpathlineto{\pgfqpoint{2.723726in}{3.280415in}}%
\pgfpathlineto{\pgfqpoint{2.731632in}{3.283313in}}%
\pgfpathlineto{\pgfqpoint{2.739539in}{3.285541in}}%
\pgfpathlineto{\pgfqpoint{2.747446in}{3.287088in}}%
\pgfpathlineto{\pgfqpoint{2.755353in}{3.287944in}}%
\pgfpathlineto{\pgfqpoint{2.763259in}{3.288099in}}%
\pgfpathlineto{\pgfqpoint{2.771166in}{3.287541in}}%
\pgfpathlineto{\pgfqpoint{2.779073in}{3.286260in}}%
\pgfpathlineto{\pgfqpoint{2.786979in}{3.284246in}}%
\pgfpathlineto{\pgfqpoint{2.794886in}{3.281488in}}%
\pgfpathlineto{\pgfqpoint{2.802793in}{3.277975in}}%
\pgfpathlineto{\pgfqpoint{2.810700in}{3.273698in}}%
\pgfpathlineto{\pgfqpoint{2.818606in}{3.268645in}}%
\pgfpathlineto{\pgfqpoint{2.826513in}{3.262806in}}%
\pgfpathlineto{\pgfqpoint{2.834420in}{3.256171in}}%
\pgfpathlineto{\pgfqpoint{2.842326in}{3.248728in}}%
\pgfpathlineto{\pgfqpoint{2.850233in}{3.240468in}}%
\pgfpathlineto{\pgfqpoint{2.858140in}{3.231379in}}%
\pgfpathlineto{\pgfqpoint{2.866047in}{3.221452in}}%
\pgfpathlineto{\pgfqpoint{2.873953in}{3.210677in}}%
\pgfpathlineto{\pgfqpoint{2.881860in}{3.199066in}}%
\pgfpathlineto{\pgfqpoint{2.889767in}{3.186657in}}%
\pgfpathlineto{\pgfqpoint{2.897674in}{3.173484in}}%
\pgfpathlineto{\pgfqpoint{2.905580in}{3.159586in}}%
\pgfpathlineto{\pgfqpoint{2.913487in}{3.144999in}}%
\pgfpathlineto{\pgfqpoint{2.921394in}{3.129760in}}%
\pgfpathlineto{\pgfqpoint{2.929300in}{3.113906in}}%
\pgfpathlineto{\pgfqpoint{2.937207in}{3.097473in}}%
\pgfpathlineto{\pgfqpoint{2.945114in}{3.080498in}}%
\pgfpathlineto{\pgfqpoint{2.953021in}{3.063019in}}%
\pgfpathlineto{\pgfqpoint{2.960927in}{3.045071in}}%
\pgfpathlineto{\pgfqpoint{2.968834in}{3.026692in}}%
\pgfpathlineto{\pgfqpoint{2.976741in}{3.007919in}}%
\pgfpathlineto{\pgfqpoint{2.984647in}{2.988788in}}%
\pgfpathlineto{\pgfqpoint{2.992554in}{2.969336in}}%
\pgfpathlineto{\pgfqpoint{3.000461in}{2.949599in}}%
\pgfpathlineto{\pgfqpoint{3.008368in}{2.929616in}}%
\pgfpathlineto{\pgfqpoint{3.016274in}{2.909422in}}%
\pgfpathlineto{\pgfqpoint{3.024181in}{2.889054in}}%
\pgfpathlineto{\pgfqpoint{3.032088in}{2.868549in}}%
\pgfpathlineto{\pgfqpoint{3.039995in}{2.847944in}}%
\pgfpathlineto{\pgfqpoint{3.047901in}{2.827276in}}%
\pgfpathlineto{\pgfqpoint{3.055808in}{2.806581in}}%
\pgfpathlineto{\pgfqpoint{3.063715in}{2.785897in}}%
\pgfpathlineto{\pgfqpoint{3.071621in}{2.765259in}}%
\pgfpathlineto{\pgfqpoint{3.079528in}{2.744706in}}%
\pgfpathlineto{\pgfqpoint{3.087435in}{2.724273in}}%
\pgfpathlineto{\pgfqpoint{3.095342in}{2.703997in}}%
\pgfpathlineto{\pgfqpoint{3.103248in}{2.683916in}}%
\pgfpathlineto{\pgfqpoint{3.111155in}{2.664066in}}%
\pgfpathlineto{\pgfqpoint{3.119062in}{2.644483in}}%
\pgfpathlineto{\pgfqpoint{3.126968in}{2.625206in}}%
\pgfpathlineto{\pgfqpoint{3.134875in}{2.606270in}}%
\pgfpathlineto{\pgfqpoint{3.142782in}{2.587712in}}%
\pgfpathlineto{\pgfqpoint{3.150689in}{2.569569in}}%
\pgfpathlineto{\pgfqpoint{3.158595in}{2.551878in}}%
\pgfpathlineto{\pgfqpoint{3.166502in}{2.534675in}}%
\pgfpathlineto{\pgfqpoint{3.174409in}{2.517998in}}%
\pgfpathlineto{\pgfqpoint{3.182316in}{2.501883in}}%
\pgfpathlineto{\pgfqpoint{3.190222in}{2.486368in}}%
\pgfpathlineto{\pgfqpoint{3.198129in}{2.471488in}}%
\pgfpathlineto{\pgfqpoint{3.206036in}{2.457281in}}%
\pgfpathlineto{\pgfqpoint{3.213942in}{2.443783in}}%
\pgfpathlineto{\pgfqpoint{3.221849in}{2.431031in}}%
\pgfpathlineto{\pgfqpoint{3.229756in}{2.419063in}}%
\pgfpathlineto{\pgfqpoint{3.237663in}{2.407914in}}%
\pgfpathlineto{\pgfqpoint{3.245569in}{2.397622in}}%
\pgfpathlineto{\pgfqpoint{3.253476in}{2.388224in}}%
\pgfpathlineto{\pgfqpoint{3.261383in}{2.379755in}}%
\pgfpathlineto{\pgfqpoint{3.269289in}{2.372254in}}%
\pgfpathlineto{\pgfqpoint{3.277196in}{2.365757in}}%
\pgfpathlineto{\pgfqpoint{3.285103in}{2.360300in}}%
\pgfpathlineto{\pgfqpoint{3.293010in}{2.355920in}}%
\pgfpathlineto{\pgfqpoint{3.300916in}{2.352655in}}%
\pgfpathlineto{\pgfqpoint{3.308823in}{2.350541in}}%
\pgfpathlineto{\pgfqpoint{3.316730in}{2.349614in}}%
\pgfpathlineto{\pgfqpoint{3.324637in}{2.349913in}}%
\pgfpathlineto{\pgfqpoint{3.332543in}{2.351472in}}%
\pgfpathlineto{\pgfqpoint{3.340450in}{2.354330in}}%
\pgfpathlineto{\pgfqpoint{3.348357in}{2.358523in}}%
\pgfpathlineto{\pgfqpoint{3.356263in}{2.364088in}}%
\pgfpathlineto{\pgfqpoint{3.364170in}{2.371061in}}%
\pgfpathlineto{\pgfqpoint{3.372077in}{2.379460in}}%
\pgfpathlineto{\pgfqpoint{3.379984in}{2.389248in}}%
\pgfpathlineto{\pgfqpoint{3.387890in}{2.400380in}}%
\pgfpathlineto{\pgfqpoint{3.395797in}{2.412808in}}%
\pgfpathlineto{\pgfqpoint{3.403704in}{2.426487in}}%
\pgfpathlineto{\pgfqpoint{3.411610in}{2.441371in}}%
\pgfpathlineto{\pgfqpoint{3.419517in}{2.457414in}}%
\pgfpathlineto{\pgfqpoint{3.427424in}{2.474571in}}%
\pgfpathlineto{\pgfqpoint{3.435331in}{2.492794in}}%
\pgfpathlineto{\pgfqpoint{3.443237in}{2.512038in}}%
\pgfpathlineto{\pgfqpoint{3.451144in}{2.532257in}}%
\pgfpathlineto{\pgfqpoint{3.459051in}{2.553404in}}%
\pgfpathlineto{\pgfqpoint{3.466958in}{2.575435in}}%
\pgfpathlineto{\pgfqpoint{3.474864in}{2.598303in}}%
\pgfpathlineto{\pgfqpoint{3.482771in}{2.621961in}}%
\pgfpathlineto{\pgfqpoint{3.490678in}{2.646364in}}%
\pgfpathlineto{\pgfqpoint{3.498584in}{2.671466in}}%
\pgfpathlineto{\pgfqpoint{3.506491in}{2.697221in}}%
\pgfpathlineto{\pgfqpoint{3.514398in}{2.723583in}}%
\pgfpathlineto{\pgfqpoint{3.522305in}{2.750505in}}%
\pgfpathlineto{\pgfqpoint{3.530211in}{2.777942in}}%
\pgfpathlineto{\pgfqpoint{3.538118in}{2.805848in}}%
\pgfpathlineto{\pgfqpoint{3.546025in}{2.834177in}}%
\pgfpathlineto{\pgfqpoint{3.553931in}{2.862882in}}%
\pgfpathlineto{\pgfqpoint{3.561838in}{2.891918in}}%
\pgfpathlineto{\pgfqpoint{3.569745in}{2.921239in}}%
\pgfpathlineto{\pgfqpoint{3.577652in}{2.950798in}}%
\pgfpathlineto{\pgfqpoint{3.585558in}{2.980550in}}%
\pgfpathlineto{\pgfqpoint{3.593465in}{3.010449in}}%
\pgfpathlineto{\pgfqpoint{3.601372in}{3.040448in}}%
\pgfpathlineto{\pgfqpoint{3.609279in}{3.070502in}}%
\pgfpathlineto{\pgfqpoint{3.617185in}{3.100565in}}%
\pgfpathlineto{\pgfqpoint{3.625092in}{3.130590in}}%
\pgfpathlineto{\pgfqpoint{3.632999in}{3.160532in}}%
\pgfpathlineto{\pgfqpoint{3.640905in}{3.190344in}}%
\pgfpathlineto{\pgfqpoint{3.648812in}{3.219980in}}%
\pgfpathlineto{\pgfqpoint{3.656719in}{3.249396in}}%
\pgfpathlineto{\pgfqpoint{3.664626in}{3.278543in}}%
\pgfpathlineto{\pgfqpoint{3.672532in}{3.307378in}}%
\pgfpathlineto{\pgfqpoint{3.680439in}{3.335852in}}%
\pgfpathlineto{\pgfqpoint{3.688346in}{3.363921in}}%
\pgfpathlineto{\pgfqpoint{3.696253in}{3.391539in}}%
\pgfpathlineto{\pgfqpoint{3.704159in}{3.418659in}}%
\pgfpathlineto{\pgfqpoint{3.712066in}{3.445235in}}%
\pgfpathlineto{\pgfqpoint{3.719973in}{3.471222in}}%
\pgfpathlineto{\pgfqpoint{3.727879in}{3.496573in}}%
\pgfpathlineto{\pgfqpoint{3.735786in}{3.521243in}}%
\pgfpathlineto{\pgfqpoint{3.743693in}{3.545185in}}%
\pgfpathlineto{\pgfqpoint{3.751600in}{3.568353in}}%
\pgfpathlineto{\pgfqpoint{3.759506in}{3.590702in}}%
\pgfpathlineto{\pgfqpoint{3.767413in}{3.612185in}}%
\pgfpathlineto{\pgfqpoint{3.775320in}{3.632756in}}%
\pgfpathlineto{\pgfqpoint{3.783226in}{3.652369in}}%
\pgfpathlineto{\pgfqpoint{3.791133in}{3.670979in}}%
\pgfpathlineto{\pgfqpoint{3.799040in}{3.688539in}}%
\pgfpathlineto{\pgfqpoint{3.806947in}{3.705003in}}%
\pgfpathlineto{\pgfqpoint{3.814853in}{3.720325in}}%
\pgfpathlineto{\pgfqpoint{3.822760in}{3.734460in}}%
\pgfpathlineto{\pgfqpoint{3.830667in}{3.747360in}}%
\pgfpathlineto{\pgfqpoint{3.838574in}{3.758981in}}%
\pgfpathlineto{\pgfqpoint{3.846480in}{3.769276in}}%
\pgfpathlineto{\pgfqpoint{3.854387in}{3.778200in}}%
\pgfpathlineto{\pgfqpoint{3.862294in}{3.785710in}}%
\pgfpathlineto{\pgfqpoint{3.870200in}{3.791807in}}%
\pgfpathlineto{\pgfqpoint{3.878107in}{3.796513in}}%
\pgfpathlineto{\pgfqpoint{3.886014in}{3.799849in}}%
\pgfpathlineto{\pgfqpoint{3.893921in}{3.801838in}}%
\pgfpathlineto{\pgfqpoint{3.901827in}{3.802500in}}%
\pgfpathlineto{\pgfqpoint{3.909734in}{3.801858in}}%
\pgfpathlineto{\pgfqpoint{3.917641in}{3.799934in}}%
\pgfpathlineto{\pgfqpoint{3.925547in}{3.796749in}}%
\pgfpathlineto{\pgfqpoint{3.933454in}{3.792324in}}%
\pgfpathlineto{\pgfqpoint{3.941361in}{3.786683in}}%
\pgfpathlineto{\pgfqpoint{3.949268in}{3.779846in}}%
\pgfpathlineto{\pgfqpoint{3.957174in}{3.771835in}}%
\pgfpathlineto{\pgfqpoint{3.965081in}{3.762672in}}%
\pgfpathlineto{\pgfqpoint{3.972988in}{3.752379in}}%
\pgfpathlineto{\pgfqpoint{3.980895in}{3.740977in}}%
\pgfpathlineto{\pgfqpoint{3.988801in}{3.728489in}}%
\pgfpathlineto{\pgfqpoint{3.996708in}{3.714936in}}%
\pgfpathlineto{\pgfqpoint{4.004615in}{3.700339in}}%
\pgfpathlineto{\pgfqpoint{4.012521in}{3.684721in}}%
\pgfpathlineto{\pgfqpoint{4.020428in}{3.668103in}}%
\pgfpathlineto{\pgfqpoint{4.028335in}{3.650507in}}%
\pgfpathlineto{\pgfqpoint{4.036242in}{3.631955in}}%
\pgfpathlineto{\pgfqpoint{4.044148in}{3.612469in}}%
\pgfpathlineto{\pgfqpoint{4.052055in}{3.592069in}}%
\pgfpathlineto{\pgfqpoint{4.059962in}{3.570779in}}%
\pgfpathlineto{\pgfqpoint{4.067868in}{3.548619in}}%
\pgfpathlineto{\pgfqpoint{4.075775in}{3.525612in}}%
\pgfpathlineto{\pgfqpoint{4.083682in}{3.501780in}}%
\pgfpathlineto{\pgfqpoint{4.091589in}{3.477143in}}%
\pgfpathlineto{\pgfqpoint{4.099495in}{3.451724in}}%
\pgfpathlineto{\pgfqpoint{4.107402in}{3.425545in}}%
\pgfpathlineto{\pgfqpoint{4.115309in}{3.398627in}}%
\pgfpathlineto{\pgfqpoint{4.123216in}{3.370992in}}%
\pgfpathlineto{\pgfqpoint{4.131122in}{3.342662in}}%
\pgfpathlineto{\pgfqpoint{4.139029in}{3.313658in}}%
\pgfpathlineto{\pgfqpoint{4.146936in}{3.284003in}}%
\pgfpathlineto{\pgfqpoint{4.154842in}{3.253718in}}%
\pgfpathlineto{\pgfqpoint{4.162749in}{3.222825in}}%
\pgfpathlineto{\pgfqpoint{4.170656in}{3.191345in}}%
\pgfpathlineto{\pgfqpoint{4.178563in}{3.159300in}}%
\pgfpathlineto{\pgfqpoint{4.186469in}{3.126713in}}%
\pgfpathlineto{\pgfqpoint{4.194376in}{3.093604in}}%
\pgfpathlineto{\pgfqpoint{4.202283in}{3.059996in}}%
\pgfpathlineto{\pgfqpoint{4.210189in}{3.025911in}}%
\pgfpathlineto{\pgfqpoint{4.218096in}{2.991369in}}%
\pgfpathlineto{\pgfqpoint{4.226003in}{2.956394in}}%
\pgfpathlineto{\pgfqpoint{4.233910in}{2.921006in}}%
\pgfpathlineto{\pgfqpoint{4.241816in}{2.885227in}}%
\pgfpathlineto{\pgfqpoint{4.249723in}{2.849080in}}%
\pgfpathlineto{\pgfqpoint{4.257630in}{2.812585in}}%
\pgfpathlineto{\pgfqpoint{4.265537in}{2.775765in}}%
\pgfpathlineto{\pgfqpoint{4.273443in}{2.738641in}}%
\pgfpathlineto{\pgfqpoint{4.281350in}{2.701236in}}%
\pgfpathlineto{\pgfqpoint{4.289257in}{2.663570in}}%
\pgfpathlineto{\pgfqpoint{4.297163in}{2.625666in}}%
\pgfpathlineto{\pgfqpoint{4.305070in}{2.587545in}}%
\pgfpathlineto{\pgfqpoint{4.312977in}{2.549230in}}%
\pgfpathlineto{\pgfqpoint{4.320884in}{2.510741in}}%
\pgfpathlineto{\pgfqpoint{4.328790in}{2.472101in}}%
\pgfpathlineto{\pgfqpoint{4.336697in}{2.433332in}}%
\pgfpathlineto{\pgfqpoint{4.344604in}{2.394455in}}%
\pgfpathlineto{\pgfqpoint{4.352510in}{2.355491in}}%
\pgfpathlineto{\pgfqpoint{4.360417in}{2.316463in}}%
\pgfpathlineto{\pgfqpoint{4.368324in}{2.277393in}}%
\pgfpathlineto{\pgfqpoint{4.376231in}{2.238302in}}%
\pgfpathlineto{\pgfqpoint{4.384137in}{2.199212in}}%
\pgfpathlineto{\pgfqpoint{4.392044in}{2.160144in}}%
\pgfpathlineto{\pgfqpoint{4.399951in}{2.121121in}}%
\pgfpathlineto{\pgfqpoint{4.399951in}{2.088091in}}%
\pgfpathlineto{\pgfqpoint{4.399951in}{2.088091in}}%
\pgfpathlineto{\pgfqpoint{4.392044in}{2.088091in}}%
\pgfpathlineto{\pgfqpoint{4.384137in}{2.088091in}}%
\pgfpathlineto{\pgfqpoint{4.376231in}{2.088091in}}%
\pgfpathlineto{\pgfqpoint{4.368324in}{2.088091in}}%
\pgfpathlineto{\pgfqpoint{4.360417in}{2.088091in}}%
\pgfpathlineto{\pgfqpoint{4.352510in}{2.088091in}}%
\pgfpathlineto{\pgfqpoint{4.344604in}{2.088091in}}%
\pgfpathlineto{\pgfqpoint{4.336697in}{2.088091in}}%
\pgfpathlineto{\pgfqpoint{4.328790in}{2.088091in}}%
\pgfpathlineto{\pgfqpoint{4.320884in}{2.088091in}}%
\pgfpathlineto{\pgfqpoint{4.312977in}{2.088091in}}%
\pgfpathlineto{\pgfqpoint{4.305070in}{2.088091in}}%
\pgfpathlineto{\pgfqpoint{4.297163in}{2.088091in}}%
\pgfpathlineto{\pgfqpoint{4.289257in}{2.088091in}}%
\pgfpathlineto{\pgfqpoint{4.281350in}{2.088091in}}%
\pgfpathlineto{\pgfqpoint{4.273443in}{2.088091in}}%
\pgfpathlineto{\pgfqpoint{4.265537in}{2.088091in}}%
\pgfpathlineto{\pgfqpoint{4.257630in}{2.088091in}}%
\pgfpathlineto{\pgfqpoint{4.249723in}{2.088091in}}%
\pgfpathlineto{\pgfqpoint{4.241816in}{2.088091in}}%
\pgfpathlineto{\pgfqpoint{4.233910in}{2.088091in}}%
\pgfpathlineto{\pgfqpoint{4.226003in}{2.088091in}}%
\pgfpathlineto{\pgfqpoint{4.218096in}{2.088091in}}%
\pgfpathlineto{\pgfqpoint{4.210189in}{2.088091in}}%
\pgfpathlineto{\pgfqpoint{4.202283in}{2.088091in}}%
\pgfpathlineto{\pgfqpoint{4.194376in}{2.088091in}}%
\pgfpathlineto{\pgfqpoint{4.186469in}{2.088091in}}%
\pgfpathlineto{\pgfqpoint{4.178563in}{2.088091in}}%
\pgfpathlineto{\pgfqpoint{4.170656in}{2.088091in}}%
\pgfpathlineto{\pgfqpoint{4.162749in}{2.088091in}}%
\pgfpathlineto{\pgfqpoint{4.154842in}{2.088091in}}%
\pgfpathlineto{\pgfqpoint{4.146936in}{2.088091in}}%
\pgfpathlineto{\pgfqpoint{4.139029in}{2.088091in}}%
\pgfpathlineto{\pgfqpoint{4.131122in}{2.088091in}}%
\pgfpathlineto{\pgfqpoint{4.123216in}{2.088091in}}%
\pgfpathlineto{\pgfqpoint{4.115309in}{2.088091in}}%
\pgfpathlineto{\pgfqpoint{4.107402in}{2.088091in}}%
\pgfpathlineto{\pgfqpoint{4.099495in}{2.088091in}}%
\pgfpathlineto{\pgfqpoint{4.091589in}{2.088091in}}%
\pgfpathlineto{\pgfqpoint{4.083682in}{2.088091in}}%
\pgfpathlineto{\pgfqpoint{4.075775in}{2.088091in}}%
\pgfpathlineto{\pgfqpoint{4.067868in}{2.088091in}}%
\pgfpathlineto{\pgfqpoint{4.059962in}{2.088091in}}%
\pgfpathlineto{\pgfqpoint{4.052055in}{2.088091in}}%
\pgfpathlineto{\pgfqpoint{4.044148in}{2.088091in}}%
\pgfpathlineto{\pgfqpoint{4.036242in}{2.088091in}}%
\pgfpathlineto{\pgfqpoint{4.028335in}{2.088091in}}%
\pgfpathlineto{\pgfqpoint{4.020428in}{2.088091in}}%
\pgfpathlineto{\pgfqpoint{4.012521in}{2.088091in}}%
\pgfpathlineto{\pgfqpoint{4.004615in}{2.088091in}}%
\pgfpathlineto{\pgfqpoint{3.996708in}{2.088091in}}%
\pgfpathlineto{\pgfqpoint{3.988801in}{2.088091in}}%
\pgfpathlineto{\pgfqpoint{3.980895in}{2.088091in}}%
\pgfpathlineto{\pgfqpoint{3.972988in}{2.088091in}}%
\pgfpathlineto{\pgfqpoint{3.965081in}{2.088091in}}%
\pgfpathlineto{\pgfqpoint{3.957174in}{2.088091in}}%
\pgfpathlineto{\pgfqpoint{3.949268in}{2.088091in}}%
\pgfpathlineto{\pgfqpoint{3.941361in}{2.088091in}}%
\pgfpathlineto{\pgfqpoint{3.933454in}{2.088091in}}%
\pgfpathlineto{\pgfqpoint{3.925547in}{2.088091in}}%
\pgfpathlineto{\pgfqpoint{3.917641in}{2.088091in}}%
\pgfpathlineto{\pgfqpoint{3.909734in}{2.088091in}}%
\pgfpathlineto{\pgfqpoint{3.901827in}{2.088091in}}%
\pgfpathlineto{\pgfqpoint{3.893921in}{2.088091in}}%
\pgfpathlineto{\pgfqpoint{3.886014in}{2.088091in}}%
\pgfpathlineto{\pgfqpoint{3.878107in}{2.088091in}}%
\pgfpathlineto{\pgfqpoint{3.870200in}{2.088091in}}%
\pgfpathlineto{\pgfqpoint{3.862294in}{2.088091in}}%
\pgfpathlineto{\pgfqpoint{3.854387in}{2.088091in}}%
\pgfpathlineto{\pgfqpoint{3.846480in}{2.088091in}}%
\pgfpathlineto{\pgfqpoint{3.838574in}{2.088091in}}%
\pgfpathlineto{\pgfqpoint{3.830667in}{2.088091in}}%
\pgfpathlineto{\pgfqpoint{3.822760in}{2.088091in}}%
\pgfpathlineto{\pgfqpoint{3.814853in}{2.088091in}}%
\pgfpathlineto{\pgfqpoint{3.806947in}{2.088091in}}%
\pgfpathlineto{\pgfqpoint{3.799040in}{2.088091in}}%
\pgfpathlineto{\pgfqpoint{3.791133in}{2.088091in}}%
\pgfpathlineto{\pgfqpoint{3.783226in}{2.088091in}}%
\pgfpathlineto{\pgfqpoint{3.775320in}{2.088091in}}%
\pgfpathlineto{\pgfqpoint{3.767413in}{2.088091in}}%
\pgfpathlineto{\pgfqpoint{3.759506in}{2.088091in}}%
\pgfpathlineto{\pgfqpoint{3.751600in}{2.088091in}}%
\pgfpathlineto{\pgfqpoint{3.743693in}{2.088091in}}%
\pgfpathlineto{\pgfqpoint{3.735786in}{2.088091in}}%
\pgfpathlineto{\pgfqpoint{3.727879in}{2.088091in}}%
\pgfpathlineto{\pgfqpoint{3.719973in}{2.088091in}}%
\pgfpathlineto{\pgfqpoint{3.712066in}{2.088091in}}%
\pgfpathlineto{\pgfqpoint{3.704159in}{2.088091in}}%
\pgfpathlineto{\pgfqpoint{3.696253in}{2.088091in}}%
\pgfpathlineto{\pgfqpoint{3.688346in}{2.088091in}}%
\pgfpathlineto{\pgfqpoint{3.680439in}{2.088091in}}%
\pgfpathlineto{\pgfqpoint{3.672532in}{2.088091in}}%
\pgfpathlineto{\pgfqpoint{3.664626in}{2.088091in}}%
\pgfpathlineto{\pgfqpoint{3.656719in}{2.088091in}}%
\pgfpathlineto{\pgfqpoint{3.648812in}{2.088091in}}%
\pgfpathlineto{\pgfqpoint{3.640905in}{2.088091in}}%
\pgfpathlineto{\pgfqpoint{3.632999in}{2.088091in}}%
\pgfpathlineto{\pgfqpoint{3.625092in}{2.088091in}}%
\pgfpathlineto{\pgfqpoint{3.617185in}{2.088091in}}%
\pgfpathlineto{\pgfqpoint{3.609279in}{2.088091in}}%
\pgfpathlineto{\pgfqpoint{3.601372in}{2.088091in}}%
\pgfpathlineto{\pgfqpoint{3.593465in}{2.088091in}}%
\pgfpathlineto{\pgfqpoint{3.585558in}{2.088091in}}%
\pgfpathlineto{\pgfqpoint{3.577652in}{2.088091in}}%
\pgfpathlineto{\pgfqpoint{3.569745in}{2.088091in}}%
\pgfpathlineto{\pgfqpoint{3.561838in}{2.088091in}}%
\pgfpathlineto{\pgfqpoint{3.553931in}{2.088091in}}%
\pgfpathlineto{\pgfqpoint{3.546025in}{2.088091in}}%
\pgfpathlineto{\pgfqpoint{3.538118in}{2.088091in}}%
\pgfpathlineto{\pgfqpoint{3.530211in}{2.088091in}}%
\pgfpathlineto{\pgfqpoint{3.522305in}{2.088091in}}%
\pgfpathlineto{\pgfqpoint{3.514398in}{2.088091in}}%
\pgfpathlineto{\pgfqpoint{3.506491in}{2.088091in}}%
\pgfpathlineto{\pgfqpoint{3.498584in}{2.088091in}}%
\pgfpathlineto{\pgfqpoint{3.490678in}{2.088091in}}%
\pgfpathlineto{\pgfqpoint{3.482771in}{2.088091in}}%
\pgfpathlineto{\pgfqpoint{3.474864in}{2.088091in}}%
\pgfpathlineto{\pgfqpoint{3.466958in}{2.088091in}}%
\pgfpathlineto{\pgfqpoint{3.459051in}{2.088091in}}%
\pgfpathlineto{\pgfqpoint{3.451144in}{2.088091in}}%
\pgfpathlineto{\pgfqpoint{3.443237in}{2.088091in}}%
\pgfpathlineto{\pgfqpoint{3.435331in}{2.088091in}}%
\pgfpathlineto{\pgfqpoint{3.427424in}{2.088091in}}%
\pgfpathlineto{\pgfqpoint{3.419517in}{2.088091in}}%
\pgfpathlineto{\pgfqpoint{3.411610in}{2.088091in}}%
\pgfpathlineto{\pgfqpoint{3.403704in}{2.088091in}}%
\pgfpathlineto{\pgfqpoint{3.395797in}{2.088091in}}%
\pgfpathlineto{\pgfqpoint{3.387890in}{2.088091in}}%
\pgfpathlineto{\pgfqpoint{3.379984in}{2.088091in}}%
\pgfpathlineto{\pgfqpoint{3.372077in}{2.088091in}}%
\pgfpathlineto{\pgfqpoint{3.364170in}{2.088091in}}%
\pgfpathlineto{\pgfqpoint{3.356263in}{2.088091in}}%
\pgfpathlineto{\pgfqpoint{3.348357in}{2.088091in}}%
\pgfpathlineto{\pgfqpoint{3.340450in}{2.088091in}}%
\pgfpathlineto{\pgfqpoint{3.332543in}{2.088091in}}%
\pgfpathlineto{\pgfqpoint{3.324637in}{2.088091in}}%
\pgfpathlineto{\pgfqpoint{3.316730in}{2.088091in}}%
\pgfpathlineto{\pgfqpoint{3.308823in}{2.088091in}}%
\pgfpathlineto{\pgfqpoint{3.300916in}{2.088091in}}%
\pgfpathlineto{\pgfqpoint{3.293010in}{2.088091in}}%
\pgfpathlineto{\pgfqpoint{3.285103in}{2.088091in}}%
\pgfpathlineto{\pgfqpoint{3.277196in}{2.088091in}}%
\pgfpathlineto{\pgfqpoint{3.269289in}{2.088091in}}%
\pgfpathlineto{\pgfqpoint{3.261383in}{2.088091in}}%
\pgfpathlineto{\pgfqpoint{3.253476in}{2.088091in}}%
\pgfpathlineto{\pgfqpoint{3.245569in}{2.088091in}}%
\pgfpathlineto{\pgfqpoint{3.237663in}{2.088091in}}%
\pgfpathlineto{\pgfqpoint{3.229756in}{2.088091in}}%
\pgfpathlineto{\pgfqpoint{3.221849in}{2.088091in}}%
\pgfpathlineto{\pgfqpoint{3.213942in}{2.088091in}}%
\pgfpathlineto{\pgfqpoint{3.206036in}{2.088091in}}%
\pgfpathlineto{\pgfqpoint{3.198129in}{2.088091in}}%
\pgfpathlineto{\pgfqpoint{3.190222in}{2.088091in}}%
\pgfpathlineto{\pgfqpoint{3.182316in}{2.088091in}}%
\pgfpathlineto{\pgfqpoint{3.174409in}{2.088091in}}%
\pgfpathlineto{\pgfqpoint{3.166502in}{2.088091in}}%
\pgfpathlineto{\pgfqpoint{3.158595in}{2.088091in}}%
\pgfpathlineto{\pgfqpoint{3.150689in}{2.088091in}}%
\pgfpathlineto{\pgfqpoint{3.142782in}{2.088091in}}%
\pgfpathlineto{\pgfqpoint{3.134875in}{2.088091in}}%
\pgfpathlineto{\pgfqpoint{3.126968in}{2.088091in}}%
\pgfpathlineto{\pgfqpoint{3.119062in}{2.088091in}}%
\pgfpathlineto{\pgfqpoint{3.111155in}{2.088091in}}%
\pgfpathlineto{\pgfqpoint{3.103248in}{2.088091in}}%
\pgfpathlineto{\pgfqpoint{3.095342in}{2.088091in}}%
\pgfpathlineto{\pgfqpoint{3.087435in}{2.088091in}}%
\pgfpathlineto{\pgfqpoint{3.079528in}{2.088091in}}%
\pgfpathlineto{\pgfqpoint{3.071621in}{2.088091in}}%
\pgfpathlineto{\pgfqpoint{3.063715in}{2.088091in}}%
\pgfpathlineto{\pgfqpoint{3.055808in}{2.088091in}}%
\pgfpathlineto{\pgfqpoint{3.047901in}{2.088091in}}%
\pgfpathlineto{\pgfqpoint{3.039995in}{2.088091in}}%
\pgfpathlineto{\pgfqpoint{3.032088in}{2.088091in}}%
\pgfpathlineto{\pgfqpoint{3.024181in}{2.088091in}}%
\pgfpathlineto{\pgfqpoint{3.016274in}{2.088091in}}%
\pgfpathlineto{\pgfqpoint{3.008368in}{2.088091in}}%
\pgfpathlineto{\pgfqpoint{3.000461in}{2.088091in}}%
\pgfpathlineto{\pgfqpoint{2.992554in}{2.088091in}}%
\pgfpathlineto{\pgfqpoint{2.984647in}{2.088091in}}%
\pgfpathlineto{\pgfqpoint{2.976741in}{2.088091in}}%
\pgfpathlineto{\pgfqpoint{2.968834in}{2.088091in}}%
\pgfpathlineto{\pgfqpoint{2.960927in}{2.088091in}}%
\pgfpathlineto{\pgfqpoint{2.953021in}{2.088091in}}%
\pgfpathlineto{\pgfqpoint{2.945114in}{2.088091in}}%
\pgfpathlineto{\pgfqpoint{2.937207in}{2.088091in}}%
\pgfpathlineto{\pgfqpoint{2.929300in}{2.088091in}}%
\pgfpathlineto{\pgfqpoint{2.921394in}{2.088091in}}%
\pgfpathlineto{\pgfqpoint{2.913487in}{2.088091in}}%
\pgfpathlineto{\pgfqpoint{2.905580in}{2.088091in}}%
\pgfpathlineto{\pgfqpoint{2.897674in}{2.088091in}}%
\pgfpathlineto{\pgfqpoint{2.889767in}{2.088091in}}%
\pgfpathlineto{\pgfqpoint{2.881860in}{2.088091in}}%
\pgfpathlineto{\pgfqpoint{2.873953in}{2.088091in}}%
\pgfpathlineto{\pgfqpoint{2.866047in}{2.088091in}}%
\pgfpathlineto{\pgfqpoint{2.858140in}{2.088091in}}%
\pgfpathlineto{\pgfqpoint{2.850233in}{2.088091in}}%
\pgfpathlineto{\pgfqpoint{2.842326in}{2.088091in}}%
\pgfpathlineto{\pgfqpoint{2.834420in}{2.088091in}}%
\pgfpathlineto{\pgfqpoint{2.826513in}{2.088091in}}%
\pgfpathlineto{\pgfqpoint{2.818606in}{2.088091in}}%
\pgfpathlineto{\pgfqpoint{2.810700in}{2.088091in}}%
\pgfpathlineto{\pgfqpoint{2.802793in}{2.088091in}}%
\pgfpathlineto{\pgfqpoint{2.794886in}{2.088091in}}%
\pgfpathlineto{\pgfqpoint{2.786979in}{2.088091in}}%
\pgfpathlineto{\pgfqpoint{2.779073in}{2.088091in}}%
\pgfpathlineto{\pgfqpoint{2.771166in}{2.088091in}}%
\pgfpathlineto{\pgfqpoint{2.763259in}{2.088091in}}%
\pgfpathlineto{\pgfqpoint{2.755353in}{2.088091in}}%
\pgfpathlineto{\pgfqpoint{2.747446in}{2.088091in}}%
\pgfpathlineto{\pgfqpoint{2.739539in}{2.088091in}}%
\pgfpathlineto{\pgfqpoint{2.731632in}{2.088091in}}%
\pgfpathlineto{\pgfqpoint{2.723726in}{2.088091in}}%
\pgfpathlineto{\pgfqpoint{2.715819in}{2.088091in}}%
\pgfpathlineto{\pgfqpoint{2.707912in}{2.088091in}}%
\pgfpathlineto{\pgfqpoint{2.700005in}{2.088091in}}%
\pgfpathlineto{\pgfqpoint{2.692099in}{2.088091in}}%
\pgfpathlineto{\pgfqpoint{2.684192in}{2.088091in}}%
\pgfpathlineto{\pgfqpoint{2.676285in}{2.088091in}}%
\pgfpathlineto{\pgfqpoint{2.668379in}{2.088091in}}%
\pgfpathlineto{\pgfqpoint{2.660472in}{2.088091in}}%
\pgfpathlineto{\pgfqpoint{2.652565in}{2.088091in}}%
\pgfpathlineto{\pgfqpoint{2.644658in}{2.088091in}}%
\pgfpathlineto{\pgfqpoint{2.636752in}{2.088091in}}%
\pgfpathlineto{\pgfqpoint{2.628845in}{2.088091in}}%
\pgfpathlineto{\pgfqpoint{2.620938in}{2.088091in}}%
\pgfpathlineto{\pgfqpoint{2.613032in}{2.088091in}}%
\pgfpathlineto{\pgfqpoint{2.605125in}{2.088091in}}%
\pgfpathlineto{\pgfqpoint{2.597218in}{2.088091in}}%
\pgfpathlineto{\pgfqpoint{2.589311in}{2.088091in}}%
\pgfpathlineto{\pgfqpoint{2.581405in}{2.088091in}}%
\pgfpathlineto{\pgfqpoint{2.573498in}{2.088091in}}%
\pgfpathlineto{\pgfqpoint{2.565591in}{2.088091in}}%
\pgfpathlineto{\pgfqpoint{2.557684in}{2.088091in}}%
\pgfpathlineto{\pgfqpoint{2.549778in}{2.088091in}}%
\pgfpathlineto{\pgfqpoint{2.541871in}{2.088091in}}%
\pgfpathlineto{\pgfqpoint{2.533964in}{2.088091in}}%
\pgfpathlineto{\pgfqpoint{2.526058in}{2.088091in}}%
\pgfpathlineto{\pgfqpoint{2.518151in}{2.088091in}}%
\pgfpathlineto{\pgfqpoint{2.510244in}{2.088091in}}%
\pgfpathlineto{\pgfqpoint{2.502337in}{2.088091in}}%
\pgfpathlineto{\pgfqpoint{2.494431in}{2.088091in}}%
\pgfpathlineto{\pgfqpoint{2.486524in}{2.088091in}}%
\pgfpathlineto{\pgfqpoint{2.478617in}{2.088091in}}%
\pgfpathlineto{\pgfqpoint{2.470711in}{2.088091in}}%
\pgfpathlineto{\pgfqpoint{2.462804in}{2.088091in}}%
\pgfpathlineto{\pgfqpoint{2.454897in}{2.088091in}}%
\pgfpathlineto{\pgfqpoint{2.446990in}{2.088091in}}%
\pgfpathlineto{\pgfqpoint{2.439084in}{2.088091in}}%
\pgfpathlineto{\pgfqpoint{2.431177in}{2.088091in}}%
\pgfpathlineto{\pgfqpoint{2.423270in}{2.088091in}}%
\pgfpathlineto{\pgfqpoint{2.415363in}{2.088091in}}%
\pgfpathlineto{\pgfqpoint{2.407457in}{2.088091in}}%
\pgfpathlineto{\pgfqpoint{2.399550in}{2.088091in}}%
\pgfpathlineto{\pgfqpoint{2.391643in}{2.088091in}}%
\pgfpathlineto{\pgfqpoint{2.383737in}{2.088091in}}%
\pgfpathlineto{\pgfqpoint{2.375830in}{2.088091in}}%
\pgfpathlineto{\pgfqpoint{2.367923in}{2.088091in}}%
\pgfpathlineto{\pgfqpoint{2.360016in}{2.088091in}}%
\pgfpathlineto{\pgfqpoint{2.352110in}{2.088091in}}%
\pgfpathlineto{\pgfqpoint{2.344203in}{2.088091in}}%
\pgfpathlineto{\pgfqpoint{2.336296in}{2.088091in}}%
\pgfpathlineto{\pgfqpoint{2.328390in}{2.088091in}}%
\pgfpathlineto{\pgfqpoint{2.320483in}{2.088091in}}%
\pgfpathlineto{\pgfqpoint{2.312576in}{2.088091in}}%
\pgfpathlineto{\pgfqpoint{2.304669in}{2.088091in}}%
\pgfpathlineto{\pgfqpoint{2.296763in}{2.088091in}}%
\pgfpathlineto{\pgfqpoint{2.288856in}{2.088091in}}%
\pgfpathlineto{\pgfqpoint{2.280949in}{2.088091in}}%
\pgfpathlineto{\pgfqpoint{2.273042in}{2.088091in}}%
\pgfpathlineto{\pgfqpoint{2.265136in}{2.088091in}}%
\pgfpathlineto{\pgfqpoint{2.257229in}{2.088091in}}%
\pgfpathlineto{\pgfqpoint{2.249322in}{2.088091in}}%
\pgfpathlineto{\pgfqpoint{2.241416in}{2.088091in}}%
\pgfpathlineto{\pgfqpoint{2.233509in}{2.088091in}}%
\pgfpathlineto{\pgfqpoint{2.225602in}{2.088091in}}%
\pgfpathlineto{\pgfqpoint{2.217695in}{2.088091in}}%
\pgfpathlineto{\pgfqpoint{2.209789in}{2.088091in}}%
\pgfpathlineto{\pgfqpoint{2.201882in}{2.088091in}}%
\pgfpathlineto{\pgfqpoint{2.193975in}{2.088091in}}%
\pgfpathlineto{\pgfqpoint{2.193975in}{2.088091in}}%
\pgfpathclose%
\pgfusepath{stroke,fill}%
}%
\begin{pgfscope}%
\pgfsys@transformshift{0.000000in}{0.000000in}%
\pgfsys@useobject{currentmarker}{}%
\end{pgfscope}%
\end{pgfscope}%
\begin{pgfscope}%
\pgfpathrectangle{\pgfqpoint{0.700000in}{0.495000in}}{\pgfqpoint{4.340000in}{3.465000in}}%
\pgfusepath{clip}%
\pgfsetroundcap%
\pgfsetroundjoin%
\pgfsetlinewidth{1.505625pt}%
\definecolor{currentstroke}{rgb}{0.298039,0.447059,0.690196}%
\pgfsetstrokecolor{currentstroke}%
\pgfsetdash{}{0pt}%
\pgfpathmoveto{\pgfqpoint{0.897273in}{0.677992in}}%
\pgfpathlineto{\pgfqpoint{0.928900in}{0.675706in}}%
\pgfpathlineto{\pgfqpoint{0.960527in}{0.672830in}}%
\pgfpathlineto{\pgfqpoint{1.007967in}{0.667845in}}%
\pgfpathlineto{\pgfqpoint{1.087034in}{0.659411in}}%
\pgfpathlineto{\pgfqpoint{1.118661in}{0.656575in}}%
\pgfpathlineto{\pgfqpoint{1.150288in}{0.654345in}}%
\pgfpathlineto{\pgfqpoint{1.174008in}{0.653190in}}%
\pgfpathlineto{\pgfqpoint{1.197728in}{0.652572in}}%
\pgfpathlineto{\pgfqpoint{1.221448in}{0.652576in}}%
\pgfpathlineto{\pgfqpoint{1.245169in}{0.653285in}}%
\pgfpathlineto{\pgfqpoint{1.268889in}{0.654783in}}%
\pgfpathlineto{\pgfqpoint{1.284702in}{0.656263in}}%
\pgfpathlineto{\pgfqpoint{1.300516in}{0.658155in}}%
\pgfpathlineto{\pgfqpoint{1.316329in}{0.660486in}}%
\pgfpathlineto{\pgfqpoint{1.332142in}{0.663280in}}%
\pgfpathlineto{\pgfqpoint{1.347956in}{0.666561in}}%
\pgfpathlineto{\pgfqpoint{1.363769in}{0.670356in}}%
\pgfpathlineto{\pgfqpoint{1.379583in}{0.674688in}}%
\pgfpathlineto{\pgfqpoint{1.395396in}{0.679584in}}%
\pgfpathlineto{\pgfqpoint{1.411210in}{0.685067in}}%
\pgfpathlineto{\pgfqpoint{1.427023in}{0.691162in}}%
\pgfpathlineto{\pgfqpoint{1.442837in}{0.697895in}}%
\pgfpathlineto{\pgfqpoint{1.458650in}{0.705291in}}%
\pgfpathlineto{\pgfqpoint{1.474463in}{0.713373in}}%
\pgfpathlineto{\pgfqpoint{1.490277in}{0.722169in}}%
\pgfpathlineto{\pgfqpoint{1.506090in}{0.731701in}}%
\pgfpathlineto{\pgfqpoint{1.521904in}{0.741996in}}%
\pgfpathlineto{\pgfqpoint{1.537717in}{0.753077in}}%
\pgfpathlineto{\pgfqpoint{1.553531in}{0.764971in}}%
\pgfpathlineto{\pgfqpoint{1.569344in}{0.777701in}}%
\pgfpathlineto{\pgfqpoint{1.585158in}{0.791293in}}%
\pgfpathlineto{\pgfqpoint{1.600971in}{0.805773in}}%
\pgfpathlineto{\pgfqpoint{1.616784in}{0.821163in}}%
\pgfpathlineto{\pgfqpoint{1.632598in}{0.837491in}}%
\pgfpathlineto{\pgfqpoint{1.648411in}{0.854780in}}%
\pgfpathlineto{\pgfqpoint{1.664225in}{0.873055in}}%
\pgfpathlineto{\pgfqpoint{1.680038in}{0.892342in}}%
\pgfpathlineto{\pgfqpoint{1.695852in}{0.912665in}}%
\pgfpathlineto{\pgfqpoint{1.711665in}{0.934049in}}%
\pgfpathlineto{\pgfqpoint{1.727479in}{0.956519in}}%
\pgfpathlineto{\pgfqpoint{1.743292in}{0.980101in}}%
\pgfpathlineto{\pgfqpoint{1.759105in}{1.004818in}}%
\pgfpathlineto{\pgfqpoint{1.774919in}{1.030696in}}%
\pgfpathlineto{\pgfqpoint{1.790732in}{1.057760in}}%
\pgfpathlineto{\pgfqpoint{1.806546in}{1.086035in}}%
\pgfpathlineto{\pgfqpoint{1.822359in}{1.115546in}}%
\pgfpathlineto{\pgfqpoint{1.838173in}{1.146317in}}%
\pgfpathlineto{\pgfqpoint{1.853986in}{1.178374in}}%
\pgfpathlineto{\pgfqpoint{1.869800in}{1.211741in}}%
\pgfpathlineto{\pgfqpoint{1.885613in}{1.246443in}}%
\pgfpathlineto{\pgfqpoint{1.901426in}{1.282484in}}%
\pgfpathlineto{\pgfqpoint{1.917240in}{1.319804in}}%
\pgfpathlineto{\pgfqpoint{1.933053in}{1.358335in}}%
\pgfpathlineto{\pgfqpoint{1.948867in}{1.398009in}}%
\pgfpathlineto{\pgfqpoint{1.964680in}{1.438755in}}%
\pgfpathlineto{\pgfqpoint{1.988400in}{1.501737in}}%
\pgfpathlineto{\pgfqpoint{2.012121in}{1.566746in}}%
\pgfpathlineto{\pgfqpoint{2.035841in}{1.633550in}}%
\pgfpathlineto{\pgfqpoint{2.059561in}{1.701918in}}%
\pgfpathlineto{\pgfqpoint{2.091188in}{1.795104in}}%
\pgfpathlineto{\pgfqpoint{2.122815in}{1.890105in}}%
\pgfpathlineto{\pgfqpoint{2.170255in}{2.034806in}}%
\pgfpathlineto{\pgfqpoint{2.265136in}{2.325307in}}%
\pgfpathlineto{\pgfqpoint{2.296763in}{2.420448in}}%
\pgfpathlineto{\pgfqpoint{2.328390in}{2.513824in}}%
\pgfpathlineto{\pgfqpoint{2.352110in}{2.582367in}}%
\pgfpathlineto{\pgfqpoint{2.375830in}{2.649376in}}%
\pgfpathlineto{\pgfqpoint{2.399550in}{2.714610in}}%
\pgfpathlineto{\pgfqpoint{2.423270in}{2.777795in}}%
\pgfpathlineto{\pgfqpoint{2.439084in}{2.818640in}}%
\pgfpathlineto{\pgfqpoint{2.454897in}{2.858363in}}%
\pgfpathlineto{\pgfqpoint{2.470711in}{2.896880in}}%
\pgfpathlineto{\pgfqpoint{2.486524in}{2.934108in}}%
\pgfpathlineto{\pgfqpoint{2.502337in}{2.969961in}}%
\pgfpathlineto{\pgfqpoint{2.518151in}{3.004356in}}%
\pgfpathlineto{\pgfqpoint{2.533964in}{3.037208in}}%
\pgfpathlineto{\pgfqpoint{2.549778in}{3.068433in}}%
\pgfpathlineto{\pgfqpoint{2.565591in}{3.097946in}}%
\pgfpathlineto{\pgfqpoint{2.581405in}{3.125664in}}%
\pgfpathlineto{\pgfqpoint{2.597218in}{3.151502in}}%
\pgfpathlineto{\pgfqpoint{2.613032in}{3.175376in}}%
\pgfpathlineto{\pgfqpoint{2.620938in}{3.186550in}}%
\pgfpathlineto{\pgfqpoint{2.628845in}{3.197201in}}%
\pgfpathlineto{\pgfqpoint{2.636752in}{3.207319in}}%
\pgfpathlineto{\pgfqpoint{2.644658in}{3.216894in}}%
\pgfpathlineto{\pgfqpoint{2.652565in}{3.225914in}}%
\pgfpathlineto{\pgfqpoint{2.660472in}{3.234369in}}%
\pgfpathlineto{\pgfqpoint{2.668379in}{3.242249in}}%
\pgfpathlineto{\pgfqpoint{2.676285in}{3.249543in}}%
\pgfpathlineto{\pgfqpoint{2.684192in}{3.256241in}}%
\pgfpathlineto{\pgfqpoint{2.692099in}{3.262332in}}%
\pgfpathlineto{\pgfqpoint{2.700005in}{3.267805in}}%
\pgfpathlineto{\pgfqpoint{2.707912in}{3.272650in}}%
\pgfpathlineto{\pgfqpoint{2.715819in}{3.276857in}}%
\pgfpathlineto{\pgfqpoint{2.723726in}{3.280415in}}%
\pgfpathlineto{\pgfqpoint{2.731632in}{3.283313in}}%
\pgfpathlineto{\pgfqpoint{2.739539in}{3.285541in}}%
\pgfpathlineto{\pgfqpoint{2.747446in}{3.287088in}}%
\pgfpathlineto{\pgfqpoint{2.755353in}{3.287944in}}%
\pgfpathlineto{\pgfqpoint{2.763259in}{3.288099in}}%
\pgfpathlineto{\pgfqpoint{2.771166in}{3.287541in}}%
\pgfpathlineto{\pgfqpoint{2.779073in}{3.286260in}}%
\pgfpathlineto{\pgfqpoint{2.786979in}{3.284246in}}%
\pgfpathlineto{\pgfqpoint{2.794886in}{3.281488in}}%
\pgfpathlineto{\pgfqpoint{2.802793in}{3.277975in}}%
\pgfpathlineto{\pgfqpoint{2.810700in}{3.273698in}}%
\pgfpathlineto{\pgfqpoint{2.818606in}{3.268645in}}%
\pgfpathlineto{\pgfqpoint{2.826513in}{3.262806in}}%
\pgfpathlineto{\pgfqpoint{2.834420in}{3.256171in}}%
\pgfpathlineto{\pgfqpoint{2.842326in}{3.248728in}}%
\pgfpathlineto{\pgfqpoint{2.850233in}{3.240468in}}%
\pgfpathlineto{\pgfqpoint{2.858140in}{3.231379in}}%
\pgfpathlineto{\pgfqpoint{2.866047in}{3.221452in}}%
\pgfpathlineto{\pgfqpoint{2.873953in}{3.210677in}}%
\pgfpathlineto{\pgfqpoint{2.881860in}{3.199066in}}%
\pgfpathlineto{\pgfqpoint{2.889767in}{3.186657in}}%
\pgfpathlineto{\pgfqpoint{2.897674in}{3.173484in}}%
\pgfpathlineto{\pgfqpoint{2.905580in}{3.159586in}}%
\pgfpathlineto{\pgfqpoint{2.913487in}{3.144999in}}%
\pgfpathlineto{\pgfqpoint{2.921394in}{3.129760in}}%
\pgfpathlineto{\pgfqpoint{2.937207in}{3.097473in}}%
\pgfpathlineto{\pgfqpoint{2.953021in}{3.063019in}}%
\pgfpathlineto{\pgfqpoint{2.968834in}{3.026692in}}%
\pgfpathlineto{\pgfqpoint{2.984647in}{2.988788in}}%
\pgfpathlineto{\pgfqpoint{3.000461in}{2.949599in}}%
\pgfpathlineto{\pgfqpoint{3.024181in}{2.889054in}}%
\pgfpathlineto{\pgfqpoint{3.071621in}{2.765259in}}%
\pgfpathlineto{\pgfqpoint{3.095342in}{2.703997in}}%
\pgfpathlineto{\pgfqpoint{3.119062in}{2.644483in}}%
\pgfpathlineto{\pgfqpoint{3.134875in}{2.606270in}}%
\pgfpathlineto{\pgfqpoint{3.150689in}{2.569569in}}%
\pgfpathlineto{\pgfqpoint{3.166502in}{2.534675in}}%
\pgfpathlineto{\pgfqpoint{3.182316in}{2.501883in}}%
\pgfpathlineto{\pgfqpoint{3.190222in}{2.486368in}}%
\pgfpathlineto{\pgfqpoint{3.198129in}{2.471488in}}%
\pgfpathlineto{\pgfqpoint{3.206036in}{2.457281in}}%
\pgfpathlineto{\pgfqpoint{3.213942in}{2.443783in}}%
\pgfpathlineto{\pgfqpoint{3.221849in}{2.431031in}}%
\pgfpathlineto{\pgfqpoint{3.229756in}{2.419063in}}%
\pgfpathlineto{\pgfqpoint{3.237663in}{2.407914in}}%
\pgfpathlineto{\pgfqpoint{3.245569in}{2.397622in}}%
\pgfpathlineto{\pgfqpoint{3.253476in}{2.388224in}}%
\pgfpathlineto{\pgfqpoint{3.261383in}{2.379755in}}%
\pgfpathlineto{\pgfqpoint{3.269289in}{2.372254in}}%
\pgfpathlineto{\pgfqpoint{3.277196in}{2.365757in}}%
\pgfpathlineto{\pgfqpoint{3.285103in}{2.360300in}}%
\pgfpathlineto{\pgfqpoint{3.293010in}{2.355920in}}%
\pgfpathlineto{\pgfqpoint{3.300916in}{2.352655in}}%
\pgfpathlineto{\pgfqpoint{3.308823in}{2.350541in}}%
\pgfpathlineto{\pgfqpoint{3.316730in}{2.349614in}}%
\pgfpathlineto{\pgfqpoint{3.324637in}{2.349913in}}%
\pgfpathlineto{\pgfqpoint{3.332543in}{2.351472in}}%
\pgfpathlineto{\pgfqpoint{3.340450in}{2.354330in}}%
\pgfpathlineto{\pgfqpoint{3.348357in}{2.358523in}}%
\pgfpathlineto{\pgfqpoint{3.356263in}{2.364088in}}%
\pgfpathlineto{\pgfqpoint{3.364170in}{2.371061in}}%
\pgfpathlineto{\pgfqpoint{3.372077in}{2.379460in}}%
\pgfpathlineto{\pgfqpoint{3.379984in}{2.389248in}}%
\pgfpathlineto{\pgfqpoint{3.387890in}{2.400380in}}%
\pgfpathlineto{\pgfqpoint{3.395797in}{2.412808in}}%
\pgfpathlineto{\pgfqpoint{3.403704in}{2.426487in}}%
\pgfpathlineto{\pgfqpoint{3.411610in}{2.441371in}}%
\pgfpathlineto{\pgfqpoint{3.419517in}{2.457414in}}%
\pgfpathlineto{\pgfqpoint{3.427424in}{2.474571in}}%
\pgfpathlineto{\pgfqpoint{3.435331in}{2.492794in}}%
\pgfpathlineto{\pgfqpoint{3.443237in}{2.512038in}}%
\pgfpathlineto{\pgfqpoint{3.451144in}{2.532257in}}%
\pgfpathlineto{\pgfqpoint{3.459051in}{2.553404in}}%
\pgfpathlineto{\pgfqpoint{3.466958in}{2.575435in}}%
\pgfpathlineto{\pgfqpoint{3.482771in}{2.621961in}}%
\pgfpathlineto{\pgfqpoint{3.498584in}{2.671466in}}%
\pgfpathlineto{\pgfqpoint{3.514398in}{2.723583in}}%
\pgfpathlineto{\pgfqpoint{3.530211in}{2.777942in}}%
\pgfpathlineto{\pgfqpoint{3.546025in}{2.834177in}}%
\pgfpathlineto{\pgfqpoint{3.569745in}{2.921239in}}%
\pgfpathlineto{\pgfqpoint{3.601372in}{3.040448in}}%
\pgfpathlineto{\pgfqpoint{3.648812in}{3.219980in}}%
\pgfpathlineto{\pgfqpoint{3.672532in}{3.307378in}}%
\pgfpathlineto{\pgfqpoint{3.688346in}{3.363921in}}%
\pgfpathlineto{\pgfqpoint{3.704159in}{3.418659in}}%
\pgfpathlineto{\pgfqpoint{3.719973in}{3.471222in}}%
\pgfpathlineto{\pgfqpoint{3.735786in}{3.521243in}}%
\pgfpathlineto{\pgfqpoint{3.751600in}{3.568353in}}%
\pgfpathlineto{\pgfqpoint{3.759506in}{3.590702in}}%
\pgfpathlineto{\pgfqpoint{3.767413in}{3.612185in}}%
\pgfpathlineto{\pgfqpoint{3.775320in}{3.632756in}}%
\pgfpathlineto{\pgfqpoint{3.783226in}{3.652369in}}%
\pgfpathlineto{\pgfqpoint{3.791133in}{3.670979in}}%
\pgfpathlineto{\pgfqpoint{3.799040in}{3.688539in}}%
\pgfpathlineto{\pgfqpoint{3.806947in}{3.705003in}}%
\pgfpathlineto{\pgfqpoint{3.814853in}{3.720325in}}%
\pgfpathlineto{\pgfqpoint{3.822760in}{3.734460in}}%
\pgfpathlineto{\pgfqpoint{3.830667in}{3.747360in}}%
\pgfpathlineto{\pgfqpoint{3.838574in}{3.758981in}}%
\pgfpathlineto{\pgfqpoint{3.846480in}{3.769276in}}%
\pgfpathlineto{\pgfqpoint{3.854387in}{3.778200in}}%
\pgfpathlineto{\pgfqpoint{3.862294in}{3.785710in}}%
\pgfpathlineto{\pgfqpoint{3.870200in}{3.791807in}}%
\pgfpathlineto{\pgfqpoint{3.878107in}{3.796513in}}%
\pgfpathlineto{\pgfqpoint{3.886014in}{3.799849in}}%
\pgfpathlineto{\pgfqpoint{3.893921in}{3.801838in}}%
\pgfpathlineto{\pgfqpoint{3.901827in}{3.802500in}}%
\pgfpathlineto{\pgfqpoint{3.909734in}{3.801858in}}%
\pgfpathlineto{\pgfqpoint{3.917641in}{3.799934in}}%
\pgfpathlineto{\pgfqpoint{3.925547in}{3.796749in}}%
\pgfpathlineto{\pgfqpoint{3.933454in}{3.792324in}}%
\pgfpathlineto{\pgfqpoint{3.941361in}{3.786683in}}%
\pgfpathlineto{\pgfqpoint{3.949268in}{3.779846in}}%
\pgfpathlineto{\pgfqpoint{3.957174in}{3.771835in}}%
\pgfpathlineto{\pgfqpoint{3.965081in}{3.762672in}}%
\pgfpathlineto{\pgfqpoint{3.972988in}{3.752379in}}%
\pgfpathlineto{\pgfqpoint{3.980895in}{3.740977in}}%
\pgfpathlineto{\pgfqpoint{3.988801in}{3.728489in}}%
\pgfpathlineto{\pgfqpoint{3.996708in}{3.714936in}}%
\pgfpathlineto{\pgfqpoint{4.004615in}{3.700339in}}%
\pgfpathlineto{\pgfqpoint{4.012521in}{3.684721in}}%
\pgfpathlineto{\pgfqpoint{4.020428in}{3.668103in}}%
\pgfpathlineto{\pgfqpoint{4.028335in}{3.650507in}}%
\pgfpathlineto{\pgfqpoint{4.036242in}{3.631955in}}%
\pgfpathlineto{\pgfqpoint{4.044148in}{3.612469in}}%
\pgfpathlineto{\pgfqpoint{4.052055in}{3.592069in}}%
\pgfpathlineto{\pgfqpoint{4.059962in}{3.570779in}}%
\pgfpathlineto{\pgfqpoint{4.067868in}{3.548619in}}%
\pgfpathlineto{\pgfqpoint{4.083682in}{3.501780in}}%
\pgfpathlineto{\pgfqpoint{4.099495in}{3.451724in}}%
\pgfpathlineto{\pgfqpoint{4.115309in}{3.398627in}}%
\pgfpathlineto{\pgfqpoint{4.131122in}{3.342662in}}%
\pgfpathlineto{\pgfqpoint{4.146936in}{3.284003in}}%
\pgfpathlineto{\pgfqpoint{4.162749in}{3.222825in}}%
\pgfpathlineto{\pgfqpoint{4.178563in}{3.159300in}}%
\pgfpathlineto{\pgfqpoint{4.194376in}{3.093604in}}%
\pgfpathlineto{\pgfqpoint{4.210189in}{3.025911in}}%
\pgfpathlineto{\pgfqpoint{4.226003in}{2.956394in}}%
\pgfpathlineto{\pgfqpoint{4.249723in}{2.849080in}}%
\pgfpathlineto{\pgfqpoint{4.273443in}{2.738641in}}%
\pgfpathlineto{\pgfqpoint{4.297163in}{2.625666in}}%
\pgfpathlineto{\pgfqpoint{4.328790in}{2.472101in}}%
\pgfpathlineto{\pgfqpoint{4.376231in}{2.238302in}}%
\pgfpathlineto{\pgfqpoint{4.431578in}{1.965908in}}%
\pgfpathlineto{\pgfqpoint{4.463205in}{1.813147in}}%
\pgfpathlineto{\pgfqpoint{4.486925in}{1.701022in}}%
\pgfpathlineto{\pgfqpoint{4.510645in}{1.591646in}}%
\pgfpathlineto{\pgfqpoint{4.534365in}{1.485608in}}%
\pgfpathlineto{\pgfqpoint{4.550179in}{1.417059in}}%
\pgfpathlineto{\pgfqpoint{4.565992in}{1.350428in}}%
\pgfpathlineto{\pgfqpoint{4.581805in}{1.285890in}}%
\pgfpathlineto{\pgfqpoint{4.597619in}{1.223617in}}%
\pgfpathlineto{\pgfqpoint{4.613432in}{1.163785in}}%
\pgfpathlineto{\pgfqpoint{4.629246in}{1.106567in}}%
\pgfpathlineto{\pgfqpoint{4.645059in}{1.052137in}}%
\pgfpathlineto{\pgfqpoint{4.660873in}{1.000669in}}%
\pgfpathlineto{\pgfqpoint{4.676686in}{0.952338in}}%
\pgfpathlineto{\pgfqpoint{4.692500in}{0.907317in}}%
\pgfpathlineto{\pgfqpoint{4.700406in}{0.886102in}}%
\pgfpathlineto{\pgfqpoint{4.708313in}{0.865780in}}%
\pgfpathlineto{\pgfqpoint{4.716220in}{0.846373in}}%
\pgfpathlineto{\pgfqpoint{4.724126in}{0.827902in}}%
\pgfpathlineto{\pgfqpoint{4.732033in}{0.810389in}}%
\pgfpathlineto{\pgfqpoint{4.739940in}{0.793856in}}%
\pgfpathlineto{\pgfqpoint{4.747847in}{0.778325in}}%
\pgfpathlineto{\pgfqpoint{4.755753in}{0.763817in}}%
\pgfpathlineto{\pgfqpoint{4.763660in}{0.750354in}}%
\pgfpathlineto{\pgfqpoint{4.771567in}{0.737958in}}%
\pgfpathlineto{\pgfqpoint{4.779473in}{0.726650in}}%
\pgfpathlineto{\pgfqpoint{4.787380in}{0.716453in}}%
\pgfpathlineto{\pgfqpoint{4.795287in}{0.707388in}}%
\pgfpathlineto{\pgfqpoint{4.803194in}{0.699477in}}%
\pgfpathlineto{\pgfqpoint{4.811100in}{0.692742in}}%
\pgfpathlineto{\pgfqpoint{4.819007in}{0.687204in}}%
\pgfpathlineto{\pgfqpoint{4.826914in}{0.682885in}}%
\pgfpathlineto{\pgfqpoint{4.834821in}{0.679807in}}%
\pgfpathlineto{\pgfqpoint{4.842727in}{0.677992in}}%
\pgfpathlineto{\pgfqpoint{4.842727in}{0.677992in}}%
\pgfusepath{stroke}%
\end{pgfscope}%
\begin{pgfscope}%
\pgfpathrectangle{\pgfqpoint{0.700000in}{0.495000in}}{\pgfqpoint{4.340000in}{3.465000in}}%
\pgfusepath{clip}%
\pgfsetroundcap%
\pgfsetroundjoin%
\pgfsetlinewidth{2.007500pt}%
\definecolor{currentstroke}{rgb}{1.000000,0.647059,0.000000}%
\pgfsetstrokecolor{currentstroke}%
\pgfsetdash{}{0pt}%
\pgfpathmoveto{\pgfqpoint{0.700000in}{2.088091in}}%
\pgfpathlineto{\pgfqpoint{5.040000in}{2.088091in}}%
\pgfusepath{stroke}%
\end{pgfscope}%
\begin{pgfscope}%
\pgfsetrectcap%
\pgfsetmiterjoin%
\pgfsetlinewidth{1.254687pt}%
\definecolor{currentstroke}{rgb}{1.000000,1.000000,1.000000}%
\pgfsetstrokecolor{currentstroke}%
\pgfsetdash{}{0pt}%
\pgfpathmoveto{\pgfqpoint{0.700000in}{0.495000in}}%
\pgfpathlineto{\pgfqpoint{0.700000in}{3.960000in}}%
\pgfusepath{stroke}%
\end{pgfscope}%
\begin{pgfscope}%
\pgfsetrectcap%
\pgfsetmiterjoin%
\pgfsetlinewidth{1.254687pt}%
\definecolor{currentstroke}{rgb}{1.000000,1.000000,1.000000}%
\pgfsetstrokecolor{currentstroke}%
\pgfsetdash{}{0pt}%
\pgfpathmoveto{\pgfqpoint{5.040000in}{0.495000in}}%
\pgfpathlineto{\pgfqpoint{5.040000in}{3.960000in}}%
\pgfusepath{stroke}%
\end{pgfscope}%
\begin{pgfscope}%
\pgfsetrectcap%
\pgfsetmiterjoin%
\pgfsetlinewidth{1.254687pt}%
\definecolor{currentstroke}{rgb}{1.000000,1.000000,1.000000}%
\pgfsetstrokecolor{currentstroke}%
\pgfsetdash{}{0pt}%
\pgfpathmoveto{\pgfqpoint{0.700000in}{0.495000in}}%
\pgfpathlineto{\pgfqpoint{5.040000in}{0.495000in}}%
\pgfusepath{stroke}%
\end{pgfscope}%
\begin{pgfscope}%
\pgfsetrectcap%
\pgfsetmiterjoin%
\pgfsetlinewidth{1.254687pt}%
\definecolor{currentstroke}{rgb}{1.000000,1.000000,1.000000}%
\pgfsetstrokecolor{currentstroke}%
\pgfsetdash{}{0pt}%
\pgfpathmoveto{\pgfqpoint{0.700000in}{3.960000in}}%
\pgfpathlineto{\pgfqpoint{5.040000in}{3.960000in}}%
\pgfusepath{stroke}%
\end{pgfscope}%
\begin{pgfscope}%
\pgfsetbuttcap%
\pgfsetmiterjoin%
\definecolor{currentfill}{rgb}{0.917647,0.917647,0.949020}%
\pgfsetfillcolor{currentfill}%
\pgfsetfillopacity{0.800000}%
\pgfsetlinewidth{1.003750pt}%
\definecolor{currentstroke}{rgb}{0.800000,0.800000,0.800000}%
\pgfsetstrokecolor{currentstroke}%
\pgfsetstrokeopacity{0.800000}%
\pgfsetdash{}{0pt}%
\pgfpathmoveto{\pgfqpoint{0.806944in}{2.967960in}}%
\pgfpathlineto{\pgfqpoint{2.501868in}{2.967960in}}%
\pgfpathquadraticcurveto{\pgfqpoint{2.532423in}{2.967960in}}{\pgfqpoint{2.532423in}{2.998515in}}%
\pgfpathlineto{\pgfqpoint{2.532423in}{3.853056in}}%
\pgfpathquadraticcurveto{\pgfqpoint{2.532423in}{3.883611in}}{\pgfqpoint{2.501868in}{3.883611in}}%
\pgfpathlineto{\pgfqpoint{0.806944in}{3.883611in}}%
\pgfpathquadraticcurveto{\pgfqpoint{0.776389in}{3.883611in}}{\pgfqpoint{0.776389in}{3.853056in}}%
\pgfpathlineto{\pgfqpoint{0.776389in}{2.998515in}}%
\pgfpathquadraticcurveto{\pgfqpoint{0.776389in}{2.967960in}}{\pgfqpoint{0.806944in}{2.967960in}}%
\pgfpathlineto{\pgfqpoint{0.806944in}{2.967960in}}%
\pgfpathclose%
\pgfusepath{stroke,fill}%
\end{pgfscope}%
\begin{pgfscope}%
\pgfsetroundcap%
\pgfsetroundjoin%
\pgfsetlinewidth{1.505625pt}%
\definecolor{currentstroke}{rgb}{0.298039,0.447059,0.690196}%
\pgfsetstrokecolor{currentstroke}%
\pgfsetdash{}{0pt}%
\pgfpathmoveto{\pgfqpoint{0.837500in}{3.766611in}}%
\pgfpathlineto{\pgfqpoint{0.990278in}{3.766611in}}%
\pgfpathlineto{\pgfqpoint{1.143056in}{3.766611in}}%
\pgfusepath{stroke}%
\end{pgfscope}%
\begin{pgfscope}%
\definecolor{textcolor}{rgb}{0.150000,0.150000,0.150000}%
\pgfsetstrokecolor{textcolor}%
\pgfsetfillcolor{textcolor}%
\pgftext[x=1.265278in,y=3.713139in,left,base]{\color{textcolor}{\sffamily\fontsize{11.000000}{13.200000}\selectfont\catcode`\^=\active\def^{\ifmmode\sp\else\^{}\fi}\catcode`\%=\active\def%{\%}resource demand}}%
\end{pgfscope}%
\begin{pgfscope}%
\pgfsetroundcap%
\pgfsetroundjoin%
\pgfsetlinewidth{2.007500pt}%
\definecolor{currentstroke}{rgb}{1.000000,0.647059,0.000000}%
\pgfsetstrokecolor{currentstroke}%
\pgfsetdash{}{0pt}%
\pgfpathmoveto{\pgfqpoint{0.837500in}{3.550499in}}%
\pgfpathlineto{\pgfqpoint{0.990278in}{3.550499in}}%
\pgfpathlineto{\pgfqpoint{1.143056in}{3.550499in}}%
\pgfusepath{stroke}%
\end{pgfscope}%
\begin{pgfscope}%
\definecolor{textcolor}{rgb}{0.150000,0.150000,0.150000}%
\pgfsetstrokecolor{textcolor}%
\pgfsetfillcolor{textcolor}%
\pgftext[x=1.265278in,y=3.497027in,left,base]{\color{textcolor}{\sffamily\fontsize{11.000000}{13.200000}\selectfont\catcode`\^=\active\def^{\ifmmode\sp\else\^{}\fi}\catcode`\%=\active\def%{\%}resource supply}}%
\end{pgfscope}%
\begin{pgfscope}%
\pgfsetbuttcap%
\pgfsetmiterjoin%
\definecolor{currentfill}{rgb}{0.172549,0.627451,0.172549}%
\pgfsetfillcolor{currentfill}%
\pgfsetfillopacity{0.300000}%
\pgfsetlinewidth{1.003750pt}%
\definecolor{currentstroke}{rgb}{0.172549,0.627451,0.172549}%
\pgfsetstrokecolor{currentstroke}%
\pgfsetstrokeopacity{0.300000}%
\pgfsetdash{}{0pt}%
\pgfpathmoveto{\pgfqpoint{0.837500in}{3.279125in}}%
\pgfpathlineto{\pgfqpoint{1.143056in}{3.279125in}}%
\pgfpathlineto{\pgfqpoint{1.143056in}{3.386069in}}%
\pgfpathlineto{\pgfqpoint{0.837500in}{3.386069in}}%
\pgfpathlineto{\pgfqpoint{0.837500in}{3.279125in}}%
\pgfpathclose%
\pgfusepath{stroke,fill}%
\end{pgfscope}%
\begin{pgfscope}%
\definecolor{textcolor}{rgb}{0.150000,0.150000,0.150000}%
\pgfsetstrokecolor{textcolor}%
\pgfsetfillcolor{textcolor}%
\pgftext[x=1.265278in,y=3.279125in,left,base]{\color{textcolor}{\sffamily\fontsize{11.000000}{13.200000}\selectfont\catcode`\^=\active\def^{\ifmmode\sp\else\^{}\fi}\catcode`\%=\active\def%{\%}overprovisioning}}%
\end{pgfscope}%
\begin{pgfscope}%
\pgfsetbuttcap%
\pgfsetmiterjoin%
\definecolor{currentfill}{rgb}{0.839216,0.152941,0.156863}%
\pgfsetfillcolor{currentfill}%
\pgfsetfillopacity{0.300000}%
\pgfsetlinewidth{1.003750pt}%
\definecolor{currentstroke}{rgb}{0.839216,0.152941,0.156863}%
\pgfsetstrokecolor{currentstroke}%
\pgfsetstrokeopacity{0.300000}%
\pgfsetdash{}{0pt}%
\pgfpathmoveto{\pgfqpoint{0.837500in}{3.061223in}}%
\pgfpathlineto{\pgfqpoint{1.143056in}{3.061223in}}%
\pgfpathlineto{\pgfqpoint{1.143056in}{3.168167in}}%
\pgfpathlineto{\pgfqpoint{0.837500in}{3.168167in}}%
\pgfpathlineto{\pgfqpoint{0.837500in}{3.061223in}}%
\pgfpathclose%
\pgfusepath{stroke,fill}%
\end{pgfscope}%
\begin{pgfscope}%
\definecolor{textcolor}{rgb}{0.150000,0.150000,0.150000}%
\pgfsetstrokecolor{textcolor}%
\pgfsetfillcolor{textcolor}%
\pgftext[x=1.265278in,y=3.061223in,left,base]{\color{textcolor}{\sffamily\fontsize{11.000000}{13.200000}\selectfont\catcode`\^=\active\def^{\ifmmode\sp\else\^{}\fi}\catcode`\%=\active\def%{\%}underprovisioning}}%
\end{pgfscope}%
\end{pgfpicture}%
\makeatother%
\endgroup%

    \caption{Resource demand and supply for a website during a typical day.}
    \label{fig:elasticity-application-no-scaling}
\end{figure}

If the concept of elasticity is applied to this example, resources can be released during the night (so called \textit{scale-in}) and more resources can be claimed as they are needed during the day (so called \textit{scale-out}). This is illustrated in \cref{fig:elasticity-application-scaling}.

\begin{figure}
    \centering
    %% Creator: Matplotlib, PGF backend
%%
%% To include the figure in your LaTeX document, write
%%   \input{<filename>.pgf}
%%
%% Make sure the required packages are loaded in your preamble
%%   \usepackage{pgf}
%%
%% Also ensure that all the required font packages are loaded; for instance,
%% the lmodern package is sometimes necessary when using math font.
%%   \usepackage{lmodern}
%%
%% Figures using additional raster images can only be included by \input if
%% they are in the same directory as the main LaTeX file. For loading figures
%% from other directories you can use the `import` package
%%   \usepackage{import}
%%
%% and then include the figures with
%%   \import{<path to file>}{<filename>.pgf}
%%
%% Matplotlib used the following preamble
%%   \def\mathdefault#1{#1}
%%   \everymath=\expandafter{\the\everymath\displaystyle}
%%   
%%   \usepackage{fontspec}
%%   \setmainfont{DejaVuSerif.ttf}[Path=\detokenize{/Users/nkratky/private/polaris-elasticity-strategies/test/scripts/.venv/lib/python3.11/site-packages/matplotlib/mpl-data/fonts/ttf/}]
%%   \setsansfont{Arial.ttf}[Path=\detokenize{/System/Library/Fonts/Supplemental/}]
%%   \setmonofont{DejaVuSansMono.ttf}[Path=\detokenize{/Users/nkratky/private/polaris-elasticity-strategies/test/scripts/.venv/lib/python3.11/site-packages/matplotlib/mpl-data/fonts/ttf/}]
%%   \makeatletter\@ifpackageloaded{underscore}{}{\usepackage[strings]{underscore}}\makeatother
%%
\begingroup%
\makeatletter%
\begin{pgfpicture}%
\pgfpathrectangle{\pgfpointorigin}{\pgfqpoint{6.400000in}{4.800000in}}%
\pgfusepath{use as bounding box, clip}%
\begin{pgfscope}%
\pgfsetbuttcap%
\pgfsetmiterjoin%
\definecolor{currentfill}{rgb}{1.000000,1.000000,1.000000}%
\pgfsetfillcolor{currentfill}%
\pgfsetlinewidth{0.000000pt}%
\definecolor{currentstroke}{rgb}{1.000000,1.000000,1.000000}%
\pgfsetstrokecolor{currentstroke}%
\pgfsetdash{}{0pt}%
\pgfpathmoveto{\pgfqpoint{0.000000in}{0.000000in}}%
\pgfpathlineto{\pgfqpoint{6.400000in}{0.000000in}}%
\pgfpathlineto{\pgfqpoint{6.400000in}{4.800000in}}%
\pgfpathlineto{\pgfqpoint{0.000000in}{4.800000in}}%
\pgfpathlineto{\pgfqpoint{0.000000in}{0.000000in}}%
\pgfpathclose%
\pgfusepath{fill}%
\end{pgfscope}%
\begin{pgfscope}%
\pgfsetbuttcap%
\pgfsetmiterjoin%
\definecolor{currentfill}{rgb}{0.917647,0.917647,0.949020}%
\pgfsetfillcolor{currentfill}%
\pgfsetlinewidth{0.000000pt}%
\definecolor{currentstroke}{rgb}{0.000000,0.000000,0.000000}%
\pgfsetstrokecolor{currentstroke}%
\pgfsetstrokeopacity{0.000000}%
\pgfsetdash{}{0pt}%
\pgfpathmoveto{\pgfqpoint{0.800000in}{0.528000in}}%
\pgfpathlineto{\pgfqpoint{5.760000in}{0.528000in}}%
\pgfpathlineto{\pgfqpoint{5.760000in}{4.224000in}}%
\pgfpathlineto{\pgfqpoint{0.800000in}{4.224000in}}%
\pgfpathlineto{\pgfqpoint{0.800000in}{0.528000in}}%
\pgfpathclose%
\pgfusepath{fill}%
\end{pgfscope}%
\begin{pgfscope}%
\pgfpathrectangle{\pgfqpoint{0.800000in}{0.528000in}}{\pgfqpoint{4.960000in}{3.696000in}}%
\pgfusepath{clip}%
\pgfsetroundcap%
\pgfsetroundjoin%
\pgfsetlinewidth{1.003750pt}%
\definecolor{currentstroke}{rgb}{1.000000,1.000000,1.000000}%
\pgfsetstrokecolor{currentstroke}%
\pgfsetdash{}{0pt}%
\pgfpathmoveto{\pgfqpoint{1.025455in}{0.528000in}}%
\pgfpathlineto{\pgfqpoint{1.025455in}{4.224000in}}%
\pgfusepath{stroke}%
\end{pgfscope}%
\begin{pgfscope}%
\definecolor{textcolor}{rgb}{0.150000,0.150000,0.150000}%
\pgfsetstrokecolor{textcolor}%
\pgfsetfillcolor{textcolor}%
\pgftext[x=1.025455in,y=0.396056in,,top]{\color{textcolor}{\sffamily\fontsize{11.000000}{13.200000}\selectfont\catcode`\^=\active\def^{\ifmmode\sp\else\^{}\fi}\catcode`\%=\active\def%{\%}00:00}}%
\end{pgfscope}%
\begin{pgfscope}%
\pgfpathrectangle{\pgfqpoint{0.800000in}{0.528000in}}{\pgfqpoint{4.960000in}{3.696000in}}%
\pgfusepath{clip}%
\pgfsetroundcap%
\pgfsetroundjoin%
\pgfsetlinewidth{1.003750pt}%
\definecolor{currentstroke}{rgb}{1.000000,1.000000,1.000000}%
\pgfsetstrokecolor{currentstroke}%
\pgfsetdash{}{0pt}%
\pgfpathmoveto{\pgfqpoint{1.589091in}{0.528000in}}%
\pgfpathlineto{\pgfqpoint{1.589091in}{4.224000in}}%
\pgfusepath{stroke}%
\end{pgfscope}%
\begin{pgfscope}%
\definecolor{textcolor}{rgb}{0.150000,0.150000,0.150000}%
\pgfsetstrokecolor{textcolor}%
\pgfsetfillcolor{textcolor}%
\pgftext[x=1.589091in,y=0.396056in,,top]{\color{textcolor}{\sffamily\fontsize{11.000000}{13.200000}\selectfont\catcode`\^=\active\def^{\ifmmode\sp\else\^{}\fi}\catcode`\%=\active\def%{\%}03:00}}%
\end{pgfscope}%
\begin{pgfscope}%
\pgfpathrectangle{\pgfqpoint{0.800000in}{0.528000in}}{\pgfqpoint{4.960000in}{3.696000in}}%
\pgfusepath{clip}%
\pgfsetroundcap%
\pgfsetroundjoin%
\pgfsetlinewidth{1.003750pt}%
\definecolor{currentstroke}{rgb}{1.000000,1.000000,1.000000}%
\pgfsetstrokecolor{currentstroke}%
\pgfsetdash{}{0pt}%
\pgfpathmoveto{\pgfqpoint{2.152727in}{0.528000in}}%
\pgfpathlineto{\pgfqpoint{2.152727in}{4.224000in}}%
\pgfusepath{stroke}%
\end{pgfscope}%
\begin{pgfscope}%
\definecolor{textcolor}{rgb}{0.150000,0.150000,0.150000}%
\pgfsetstrokecolor{textcolor}%
\pgfsetfillcolor{textcolor}%
\pgftext[x=2.152727in,y=0.396056in,,top]{\color{textcolor}{\sffamily\fontsize{11.000000}{13.200000}\selectfont\catcode`\^=\active\def^{\ifmmode\sp\else\^{}\fi}\catcode`\%=\active\def%{\%}06:00}}%
\end{pgfscope}%
\begin{pgfscope}%
\pgfpathrectangle{\pgfqpoint{0.800000in}{0.528000in}}{\pgfqpoint{4.960000in}{3.696000in}}%
\pgfusepath{clip}%
\pgfsetroundcap%
\pgfsetroundjoin%
\pgfsetlinewidth{1.003750pt}%
\definecolor{currentstroke}{rgb}{1.000000,1.000000,1.000000}%
\pgfsetstrokecolor{currentstroke}%
\pgfsetdash{}{0pt}%
\pgfpathmoveto{\pgfqpoint{2.716364in}{0.528000in}}%
\pgfpathlineto{\pgfqpoint{2.716364in}{4.224000in}}%
\pgfusepath{stroke}%
\end{pgfscope}%
\begin{pgfscope}%
\definecolor{textcolor}{rgb}{0.150000,0.150000,0.150000}%
\pgfsetstrokecolor{textcolor}%
\pgfsetfillcolor{textcolor}%
\pgftext[x=2.716364in,y=0.396056in,,top]{\color{textcolor}{\sffamily\fontsize{11.000000}{13.200000}\selectfont\catcode`\^=\active\def^{\ifmmode\sp\else\^{}\fi}\catcode`\%=\active\def%{\%}09:00}}%
\end{pgfscope}%
\begin{pgfscope}%
\pgfpathrectangle{\pgfqpoint{0.800000in}{0.528000in}}{\pgfqpoint{4.960000in}{3.696000in}}%
\pgfusepath{clip}%
\pgfsetroundcap%
\pgfsetroundjoin%
\pgfsetlinewidth{1.003750pt}%
\definecolor{currentstroke}{rgb}{1.000000,1.000000,1.000000}%
\pgfsetstrokecolor{currentstroke}%
\pgfsetdash{}{0pt}%
\pgfpathmoveto{\pgfqpoint{3.280000in}{0.528000in}}%
\pgfpathlineto{\pgfqpoint{3.280000in}{4.224000in}}%
\pgfusepath{stroke}%
\end{pgfscope}%
\begin{pgfscope}%
\definecolor{textcolor}{rgb}{0.150000,0.150000,0.150000}%
\pgfsetstrokecolor{textcolor}%
\pgfsetfillcolor{textcolor}%
\pgftext[x=3.280000in,y=0.396056in,,top]{\color{textcolor}{\sffamily\fontsize{11.000000}{13.200000}\selectfont\catcode`\^=\active\def^{\ifmmode\sp\else\^{}\fi}\catcode`\%=\active\def%{\%}12:00}}%
\end{pgfscope}%
\begin{pgfscope}%
\pgfpathrectangle{\pgfqpoint{0.800000in}{0.528000in}}{\pgfqpoint{4.960000in}{3.696000in}}%
\pgfusepath{clip}%
\pgfsetroundcap%
\pgfsetroundjoin%
\pgfsetlinewidth{1.003750pt}%
\definecolor{currentstroke}{rgb}{1.000000,1.000000,1.000000}%
\pgfsetstrokecolor{currentstroke}%
\pgfsetdash{}{0pt}%
\pgfpathmoveto{\pgfqpoint{3.843636in}{0.528000in}}%
\pgfpathlineto{\pgfqpoint{3.843636in}{4.224000in}}%
\pgfusepath{stroke}%
\end{pgfscope}%
\begin{pgfscope}%
\definecolor{textcolor}{rgb}{0.150000,0.150000,0.150000}%
\pgfsetstrokecolor{textcolor}%
\pgfsetfillcolor{textcolor}%
\pgftext[x=3.843636in,y=0.396056in,,top]{\color{textcolor}{\sffamily\fontsize{11.000000}{13.200000}\selectfont\catcode`\^=\active\def^{\ifmmode\sp\else\^{}\fi}\catcode`\%=\active\def%{\%}15:00}}%
\end{pgfscope}%
\begin{pgfscope}%
\pgfpathrectangle{\pgfqpoint{0.800000in}{0.528000in}}{\pgfqpoint{4.960000in}{3.696000in}}%
\pgfusepath{clip}%
\pgfsetroundcap%
\pgfsetroundjoin%
\pgfsetlinewidth{1.003750pt}%
\definecolor{currentstroke}{rgb}{1.000000,1.000000,1.000000}%
\pgfsetstrokecolor{currentstroke}%
\pgfsetdash{}{0pt}%
\pgfpathmoveto{\pgfqpoint{4.407273in}{0.528000in}}%
\pgfpathlineto{\pgfqpoint{4.407273in}{4.224000in}}%
\pgfusepath{stroke}%
\end{pgfscope}%
\begin{pgfscope}%
\definecolor{textcolor}{rgb}{0.150000,0.150000,0.150000}%
\pgfsetstrokecolor{textcolor}%
\pgfsetfillcolor{textcolor}%
\pgftext[x=4.407273in,y=0.396056in,,top]{\color{textcolor}{\sffamily\fontsize{11.000000}{13.200000}\selectfont\catcode`\^=\active\def^{\ifmmode\sp\else\^{}\fi}\catcode`\%=\active\def%{\%}18:00}}%
\end{pgfscope}%
\begin{pgfscope}%
\pgfpathrectangle{\pgfqpoint{0.800000in}{0.528000in}}{\pgfqpoint{4.960000in}{3.696000in}}%
\pgfusepath{clip}%
\pgfsetroundcap%
\pgfsetroundjoin%
\pgfsetlinewidth{1.003750pt}%
\definecolor{currentstroke}{rgb}{1.000000,1.000000,1.000000}%
\pgfsetstrokecolor{currentstroke}%
\pgfsetdash{}{0pt}%
\pgfpathmoveto{\pgfqpoint{4.970909in}{0.528000in}}%
\pgfpathlineto{\pgfqpoint{4.970909in}{4.224000in}}%
\pgfusepath{stroke}%
\end{pgfscope}%
\begin{pgfscope}%
\definecolor{textcolor}{rgb}{0.150000,0.150000,0.150000}%
\pgfsetstrokecolor{textcolor}%
\pgfsetfillcolor{textcolor}%
\pgftext[x=4.970909in,y=0.396056in,,top]{\color{textcolor}{\sffamily\fontsize{11.000000}{13.200000}\selectfont\catcode`\^=\active\def^{\ifmmode\sp\else\^{}\fi}\catcode`\%=\active\def%{\%}21:00}}%
\end{pgfscope}%
\begin{pgfscope}%
\pgfpathrectangle{\pgfqpoint{0.800000in}{0.528000in}}{\pgfqpoint{4.960000in}{3.696000in}}%
\pgfusepath{clip}%
\pgfsetroundcap%
\pgfsetroundjoin%
\pgfsetlinewidth{1.003750pt}%
\definecolor{currentstroke}{rgb}{1.000000,1.000000,1.000000}%
\pgfsetstrokecolor{currentstroke}%
\pgfsetdash{}{0pt}%
\pgfpathmoveto{\pgfqpoint{5.534545in}{0.528000in}}%
\pgfpathlineto{\pgfqpoint{5.534545in}{4.224000in}}%
\pgfusepath{stroke}%
\end{pgfscope}%
\begin{pgfscope}%
\definecolor{textcolor}{rgb}{0.150000,0.150000,0.150000}%
\pgfsetstrokecolor{textcolor}%
\pgfsetfillcolor{textcolor}%
\pgftext[x=5.534545in,y=0.396056in,,top]{\color{textcolor}{\sffamily\fontsize{11.000000}{13.200000}\selectfont\catcode`\^=\active\def^{\ifmmode\sp\else\^{}\fi}\catcode`\%=\active\def%{\%}24:00}}%
\end{pgfscope}%
\begin{pgfscope}%
\definecolor{textcolor}{rgb}{0.150000,0.150000,0.150000}%
\pgfsetstrokecolor{textcolor}%
\pgfsetfillcolor{textcolor}%
\pgftext[x=3.280000in,y=0.200777in,,top]{\color{textcolor}{\sffamily\fontsize{12.000000}{14.400000}\selectfont\catcode`\^=\active\def^{\ifmmode\sp\else\^{}\fi}\catcode`\%=\active\def%{\%}Time}}%
\end{pgfscope}%
\begin{pgfscope}%
\pgfpathrectangle{\pgfqpoint{0.800000in}{0.528000in}}{\pgfqpoint{4.960000in}{3.696000in}}%
\pgfusepath{clip}%
\pgfsetroundcap%
\pgfsetroundjoin%
\pgfsetlinewidth{1.003750pt}%
\definecolor{currentstroke}{rgb}{1.000000,1.000000,1.000000}%
\pgfsetstrokecolor{currentstroke}%
\pgfsetdash{}{0pt}%
\pgfpathmoveto{\pgfqpoint{0.800000in}{0.784658in}}%
\pgfpathlineto{\pgfqpoint{5.760000in}{0.784658in}}%
\pgfusepath{stroke}%
\end{pgfscope}%
\begin{pgfscope}%
\pgfpathrectangle{\pgfqpoint{0.800000in}{0.528000in}}{\pgfqpoint{4.960000in}{3.696000in}}%
\pgfusepath{clip}%
\pgfsetroundcap%
\pgfsetroundjoin%
\pgfsetlinewidth{1.003750pt}%
\definecolor{currentstroke}{rgb}{1.000000,1.000000,1.000000}%
\pgfsetstrokecolor{currentstroke}%
\pgfsetdash{}{0pt}%
\pgfpathmoveto{\pgfqpoint{0.800000in}{1.375205in}}%
\pgfpathlineto{\pgfqpoint{5.760000in}{1.375205in}}%
\pgfusepath{stroke}%
\end{pgfscope}%
\begin{pgfscope}%
\pgfpathrectangle{\pgfqpoint{0.800000in}{0.528000in}}{\pgfqpoint{4.960000in}{3.696000in}}%
\pgfusepath{clip}%
\pgfsetroundcap%
\pgfsetroundjoin%
\pgfsetlinewidth{1.003750pt}%
\definecolor{currentstroke}{rgb}{1.000000,1.000000,1.000000}%
\pgfsetstrokecolor{currentstroke}%
\pgfsetdash{}{0pt}%
\pgfpathmoveto{\pgfqpoint{0.800000in}{1.965751in}}%
\pgfpathlineto{\pgfqpoint{5.760000in}{1.965751in}}%
\pgfusepath{stroke}%
\end{pgfscope}%
\begin{pgfscope}%
\pgfpathrectangle{\pgfqpoint{0.800000in}{0.528000in}}{\pgfqpoint{4.960000in}{3.696000in}}%
\pgfusepath{clip}%
\pgfsetroundcap%
\pgfsetroundjoin%
\pgfsetlinewidth{1.003750pt}%
\definecolor{currentstroke}{rgb}{1.000000,1.000000,1.000000}%
\pgfsetstrokecolor{currentstroke}%
\pgfsetdash{}{0pt}%
\pgfpathmoveto{\pgfqpoint{0.800000in}{2.556297in}}%
\pgfpathlineto{\pgfqpoint{5.760000in}{2.556297in}}%
\pgfusepath{stroke}%
\end{pgfscope}%
\begin{pgfscope}%
\pgfpathrectangle{\pgfqpoint{0.800000in}{0.528000in}}{\pgfqpoint{4.960000in}{3.696000in}}%
\pgfusepath{clip}%
\pgfsetroundcap%
\pgfsetroundjoin%
\pgfsetlinewidth{1.003750pt}%
\definecolor{currentstroke}{rgb}{1.000000,1.000000,1.000000}%
\pgfsetstrokecolor{currentstroke}%
\pgfsetdash{}{0pt}%
\pgfpathmoveto{\pgfqpoint{0.800000in}{3.146844in}}%
\pgfpathlineto{\pgfqpoint{5.760000in}{3.146844in}}%
\pgfusepath{stroke}%
\end{pgfscope}%
\begin{pgfscope}%
\pgfpathrectangle{\pgfqpoint{0.800000in}{0.528000in}}{\pgfqpoint{4.960000in}{3.696000in}}%
\pgfusepath{clip}%
\pgfsetroundcap%
\pgfsetroundjoin%
\pgfsetlinewidth{1.003750pt}%
\definecolor{currentstroke}{rgb}{1.000000,1.000000,1.000000}%
\pgfsetstrokecolor{currentstroke}%
\pgfsetdash{}{0pt}%
\pgfpathmoveto{\pgfqpoint{0.800000in}{3.737390in}}%
\pgfpathlineto{\pgfqpoint{5.760000in}{3.737390in}}%
\pgfusepath{stroke}%
\end{pgfscope}%
\begin{pgfscope}%
\definecolor{textcolor}{rgb}{0.150000,0.150000,0.150000}%
\pgfsetstrokecolor{textcolor}%
\pgfsetfillcolor{textcolor}%
\pgftext[x=0.612500in,y=2.376000in,,bottom,rotate=90.000000]{\color{textcolor}{\sffamily\fontsize{12.000000}{14.400000}\selectfont\catcode`\^=\active\def^{\ifmmode\sp\else\^{}\fi}\catcode`\%=\active\def%{\%}Resources}}%
\end{pgfscope}%
\begin{pgfscope}%
\pgfpathrectangle{\pgfqpoint{0.800000in}{0.528000in}}{\pgfqpoint{4.960000in}{3.696000in}}%
\pgfusepath{clip}%
\pgfsetbuttcap%
\pgfsetroundjoin%
\definecolor{currentfill}{rgb}{0.172549,0.627451,0.172549}%
\pgfsetfillcolor{currentfill}%
\pgfsetfillopacity{0.300000}%
\pgfsetlinewidth{1.003750pt}%
\definecolor{currentstroke}{rgb}{0.172549,0.627451,0.172549}%
\pgfsetstrokecolor{currentstroke}%
\pgfsetstrokeopacity{0.300000}%
\pgfsetdash{}{0pt}%
\pgfpathmoveto{\pgfqpoint{1.034491in}{0.798290in}}%
\pgfpathlineto{\pgfqpoint{1.034491in}{0.784129in}}%
\pgfpathlineto{\pgfqpoint{1.043527in}{0.783552in}}%
\pgfpathlineto{\pgfqpoint{1.052563in}{0.782929in}}%
\pgfpathlineto{\pgfqpoint{1.061600in}{0.782265in}}%
\pgfpathlineto{\pgfqpoint{1.070636in}{0.781562in}}%
\pgfpathlineto{\pgfqpoint{1.079672in}{0.780823in}}%
\pgfpathlineto{\pgfqpoint{1.088708in}{0.780053in}}%
\pgfpathlineto{\pgfqpoint{1.097745in}{0.779253in}}%
\pgfpathlineto{\pgfqpoint{1.106781in}{0.778428in}}%
\pgfpathlineto{\pgfqpoint{1.115817in}{0.777581in}}%
\pgfpathlineto{\pgfqpoint{1.124853in}{0.776715in}}%
\pgfpathlineto{\pgfqpoint{1.133890in}{0.775833in}}%
\pgfpathlineto{\pgfqpoint{1.142926in}{0.774938in}}%
\pgfpathlineto{\pgfqpoint{1.151962in}{0.774034in}}%
\pgfpathlineto{\pgfqpoint{1.160998in}{0.773125in}}%
\pgfpathlineto{\pgfqpoint{1.170035in}{0.772212in}}%
\pgfpathlineto{\pgfqpoint{1.179071in}{0.771300in}}%
\pgfpathlineto{\pgfqpoint{1.188107in}{0.770392in}}%
\pgfpathlineto{\pgfqpoint{1.197143in}{0.769490in}}%
\pgfpathlineto{\pgfqpoint{1.206180in}{0.768599in}}%
\pgfpathlineto{\pgfqpoint{1.215216in}{0.767722in}}%
\pgfpathlineto{\pgfqpoint{1.224252in}{0.766862in}}%
\pgfpathlineto{\pgfqpoint{1.233288in}{0.766021in}}%
\pgfpathlineto{\pgfqpoint{1.242325in}{0.765204in}}%
\pgfpathlineto{\pgfqpoint{1.251361in}{0.764414in}}%
\pgfpathlineto{\pgfqpoint{1.260397in}{0.763653in}}%
\pgfpathlineto{\pgfqpoint{1.269433in}{0.762926in}}%
\pgfpathlineto{\pgfqpoint{1.278470in}{0.762234in}}%
\pgfpathlineto{\pgfqpoint{1.287506in}{0.761583in}}%
\pgfpathlineto{\pgfqpoint{1.296542in}{0.760975in}}%
\pgfpathlineto{\pgfqpoint{1.305578in}{0.760412in}}%
\pgfpathlineto{\pgfqpoint{1.314615in}{0.759900in}}%
\pgfpathlineto{\pgfqpoint{1.323651in}{0.759439in}}%
\pgfpathlineto{\pgfqpoint{1.332687in}{0.759035in}}%
\pgfpathlineto{\pgfqpoint{1.341723in}{0.758690in}}%
\pgfpathlineto{\pgfqpoint{1.350760in}{0.758408in}}%
\pgfpathlineto{\pgfqpoint{1.359796in}{0.758191in}}%
\pgfpathlineto{\pgfqpoint{1.368832in}{0.758044in}}%
\pgfpathlineto{\pgfqpoint{1.377868in}{0.757968in}}%
\pgfpathlineto{\pgfqpoint{1.386905in}{0.757969in}}%
\pgfpathlineto{\pgfqpoint{1.395941in}{0.758048in}}%
\pgfpathlineto{\pgfqpoint{1.404977in}{0.758209in}}%
\pgfpathlineto{\pgfqpoint{1.414013in}{0.758455in}}%
\pgfpathlineto{\pgfqpoint{1.423050in}{0.758790in}}%
\pgfpathlineto{\pgfqpoint{1.432086in}{0.759217in}}%
\pgfpathlineto{\pgfqpoint{1.441122in}{0.759739in}}%
\pgfpathlineto{\pgfqpoint{1.450158in}{0.760359in}}%
\pgfpathlineto{\pgfqpoint{1.459195in}{0.761081in}}%
\pgfpathlineto{\pgfqpoint{1.468231in}{0.761908in}}%
\pgfpathlineto{\pgfqpoint{1.477267in}{0.762843in}}%
\pgfpathlineto{\pgfqpoint{1.486304in}{0.763889in}}%
\pgfpathlineto{\pgfqpoint{1.495340in}{0.765050in}}%
\pgfpathlineto{\pgfqpoint{1.504376in}{0.766329in}}%
\pgfpathlineto{\pgfqpoint{1.513412in}{0.767730in}}%
\pgfpathlineto{\pgfqpoint{1.522449in}{0.769254in}}%
\pgfpathlineto{\pgfqpoint{1.531485in}{0.770907in}}%
\pgfpathlineto{\pgfqpoint{1.540521in}{0.772690in}}%
\pgfpathlineto{\pgfqpoint{1.549557in}{0.774608in}}%
\pgfpathlineto{\pgfqpoint{1.558594in}{0.776663in}}%
\pgfpathlineto{\pgfqpoint{1.567630in}{0.778859in}}%
\pgfpathlineto{\pgfqpoint{1.576666in}{0.781199in}}%
\pgfpathlineto{\pgfqpoint{1.585702in}{0.783687in}}%
\pgfpathlineto{\pgfqpoint{1.585702in}{0.786842in}}%
\pgfpathlineto{\pgfqpoint{1.585702in}{0.786842in}}%
\pgfpathlineto{\pgfqpoint{1.576666in}{0.792712in}}%
\pgfpathlineto{\pgfqpoint{1.567630in}{0.798641in}}%
\pgfpathlineto{\pgfqpoint{1.558594in}{0.804615in}}%
\pgfpathlineto{\pgfqpoint{1.549557in}{0.810623in}}%
\pgfpathlineto{\pgfqpoint{1.540521in}{0.816652in}}%
\pgfpathlineto{\pgfqpoint{1.531485in}{0.822689in}}%
\pgfpathlineto{\pgfqpoint{1.522449in}{0.828721in}}%
\pgfpathlineto{\pgfqpoint{1.513412in}{0.834736in}}%
\pgfpathlineto{\pgfqpoint{1.504376in}{0.840722in}}%
\pgfpathlineto{\pgfqpoint{1.495340in}{0.846665in}}%
\pgfpathlineto{\pgfqpoint{1.486304in}{0.852553in}}%
\pgfpathlineto{\pgfqpoint{1.477267in}{0.858374in}}%
\pgfpathlineto{\pgfqpoint{1.468231in}{0.864115in}}%
\pgfpathlineto{\pgfqpoint{1.459195in}{0.869763in}}%
\pgfpathlineto{\pgfqpoint{1.450158in}{0.875305in}}%
\pgfpathlineto{\pgfqpoint{1.441122in}{0.880730in}}%
\pgfpathlineto{\pgfqpoint{1.432086in}{0.886024in}}%
\pgfpathlineto{\pgfqpoint{1.423050in}{0.891174in}}%
\pgfpathlineto{\pgfqpoint{1.414013in}{0.896169in}}%
\pgfpathlineto{\pgfqpoint{1.404977in}{0.900996in}}%
\pgfpathlineto{\pgfqpoint{1.395941in}{0.905642in}}%
\pgfpathlineto{\pgfqpoint{1.386905in}{0.910094in}}%
\pgfpathlineto{\pgfqpoint{1.377868in}{0.914339in}}%
\pgfpathlineto{\pgfqpoint{1.368832in}{0.918366in}}%
\pgfpathlineto{\pgfqpoint{1.359796in}{0.922162in}}%
\pgfpathlineto{\pgfqpoint{1.350760in}{0.925713in}}%
\pgfpathlineto{\pgfqpoint{1.341723in}{0.929008in}}%
\pgfpathlineto{\pgfqpoint{1.332687in}{0.932033in}}%
\pgfpathlineto{\pgfqpoint{1.323651in}{0.934777in}}%
\pgfpathlineto{\pgfqpoint{1.314615in}{0.937226in}}%
\pgfpathlineto{\pgfqpoint{1.305578in}{0.939368in}}%
\pgfpathlineto{\pgfqpoint{1.296542in}{0.941190in}}%
\pgfpathlineto{\pgfqpoint{1.287506in}{0.942680in}}%
\pgfpathlineto{\pgfqpoint{1.278470in}{0.943826in}}%
\pgfpathlineto{\pgfqpoint{1.269433in}{0.944613in}}%
\pgfpathlineto{\pgfqpoint{1.260397in}{0.945031in}}%
\pgfpathlineto{\pgfqpoint{1.251361in}{0.945065in}}%
\pgfpathlineto{\pgfqpoint{1.242325in}{0.944704in}}%
\pgfpathlineto{\pgfqpoint{1.233288in}{0.943936in}}%
\pgfpathlineto{\pgfqpoint{1.224252in}{0.942747in}}%
\pgfpathlineto{\pgfqpoint{1.215216in}{0.941124in}}%
\pgfpathlineto{\pgfqpoint{1.206180in}{0.939056in}}%
\pgfpathlineto{\pgfqpoint{1.197143in}{0.936529in}}%
\pgfpathlineto{\pgfqpoint{1.188107in}{0.933532in}}%
\pgfpathlineto{\pgfqpoint{1.179071in}{0.930051in}}%
\pgfpathlineto{\pgfqpoint{1.170035in}{0.926073in}}%
\pgfpathlineto{\pgfqpoint{1.160998in}{0.921587in}}%
\pgfpathlineto{\pgfqpoint{1.151962in}{0.916579in}}%
\pgfpathlineto{\pgfqpoint{1.142926in}{0.911037in}}%
\pgfpathlineto{\pgfqpoint{1.133890in}{0.904949in}}%
\pgfpathlineto{\pgfqpoint{1.124853in}{0.898301in}}%
\pgfpathlineto{\pgfqpoint{1.115817in}{0.891081in}}%
\pgfpathlineto{\pgfqpoint{1.106781in}{0.883277in}}%
\pgfpathlineto{\pgfqpoint{1.097745in}{0.874876in}}%
\pgfpathlineto{\pgfqpoint{1.088708in}{0.865865in}}%
\pgfpathlineto{\pgfqpoint{1.079672in}{0.856232in}}%
\pgfpathlineto{\pgfqpoint{1.070636in}{0.845964in}}%
\pgfpathlineto{\pgfqpoint{1.061600in}{0.835048in}}%
\pgfpathlineto{\pgfqpoint{1.052563in}{0.823473in}}%
\pgfpathlineto{\pgfqpoint{1.043527in}{0.811224in}}%
\pgfpathlineto{\pgfqpoint{1.034491in}{0.798290in}}%
\pgfpathlineto{\pgfqpoint{1.034491in}{0.798290in}}%
\pgfpathclose%
\pgfusepath{stroke,fill}%
\end{pgfscope}%
\begin{pgfscope}%
\pgfpathrectangle{\pgfqpoint{0.800000in}{0.528000in}}{\pgfqpoint{4.960000in}{3.696000in}}%
\pgfusepath{clip}%
\pgfsetbuttcap%
\pgfsetroundjoin%
\definecolor{currentfill}{rgb}{0.172549,0.627451,0.172549}%
\pgfsetfillcolor{currentfill}%
\pgfsetfillopacity{0.300000}%
\pgfsetlinewidth{1.003750pt}%
\definecolor{currentstroke}{rgb}{0.172549,0.627451,0.172549}%
\pgfsetstrokecolor{currentstroke}%
\pgfsetstrokeopacity{0.300000}%
\pgfsetdash{}{0pt}%
\pgfpathmoveto{\pgfqpoint{2.426074in}{2.061485in}}%
\pgfpathlineto{\pgfqpoint{2.426074in}{2.053734in}}%
\pgfpathlineto{\pgfqpoint{2.435110in}{2.078830in}}%
\pgfpathlineto{\pgfqpoint{2.444146in}{2.103999in}}%
\pgfpathlineto{\pgfqpoint{2.453183in}{2.129234in}}%
\pgfpathlineto{\pgfqpoint{2.462219in}{2.154524in}}%
\pgfpathlineto{\pgfqpoint{2.471255in}{2.179860in}}%
\pgfpathlineto{\pgfqpoint{2.480291in}{2.205235in}}%
\pgfpathlineto{\pgfqpoint{2.489328in}{2.230638in}}%
\pgfpathlineto{\pgfqpoint{2.498364in}{2.256060in}}%
\pgfpathlineto{\pgfqpoint{2.507400in}{2.281494in}}%
\pgfpathlineto{\pgfqpoint{2.516437in}{2.306929in}}%
\pgfpathlineto{\pgfqpoint{2.525473in}{2.332356in}}%
\pgfpathlineto{\pgfqpoint{2.534509in}{2.357768in}}%
\pgfpathlineto{\pgfqpoint{2.543545in}{2.383154in}}%
\pgfpathlineto{\pgfqpoint{2.552582in}{2.408505in}}%
\pgfpathlineto{\pgfqpoint{2.561618in}{2.433814in}}%
\pgfpathlineto{\pgfqpoint{2.570654in}{2.459069in}}%
\pgfpathlineto{\pgfqpoint{2.579690in}{2.484264in}}%
\pgfpathlineto{\pgfqpoint{2.588727in}{2.509388in}}%
\pgfpathlineto{\pgfqpoint{2.597763in}{2.534432in}}%
\pgfpathlineto{\pgfqpoint{2.606799in}{2.559388in}}%
\pgfpathlineto{\pgfqpoint{2.615835in}{2.584247in}}%
\pgfpathlineto{\pgfqpoint{2.624872in}{2.608999in}}%
\pgfpathlineto{\pgfqpoint{2.633908in}{2.633636in}}%
\pgfpathlineto{\pgfqpoint{2.642944in}{2.658149in}}%
\pgfpathlineto{\pgfqpoint{2.651980in}{2.682527in}}%
\pgfpathlineto{\pgfqpoint{2.661017in}{2.706764in}}%
\pgfpathlineto{\pgfqpoint{2.670053in}{2.730849in}}%
\pgfpathlineto{\pgfqpoint{2.679089in}{2.754773in}}%
\pgfpathlineto{\pgfqpoint{2.688125in}{2.778528in}}%
\pgfpathlineto{\pgfqpoint{2.697162in}{2.802104in}}%
\pgfpathlineto{\pgfqpoint{2.706198in}{2.825493in}}%
\pgfpathlineto{\pgfqpoint{2.715234in}{2.848686in}}%
\pgfpathlineto{\pgfqpoint{2.724270in}{2.871672in}}%
\pgfpathlineto{\pgfqpoint{2.733307in}{2.894443in}}%
\pgfpathlineto{\pgfqpoint{2.742343in}{2.916985in}}%
\pgfpathlineto{\pgfqpoint{2.751379in}{2.939290in}}%
\pgfpathlineto{\pgfqpoint{2.760415in}{2.961345in}}%
\pgfpathlineto{\pgfqpoint{2.769452in}{2.983140in}}%
\pgfpathlineto{\pgfqpoint{2.778488in}{3.004663in}}%
\pgfpathlineto{\pgfqpoint{2.787524in}{3.025904in}}%
\pgfpathlineto{\pgfqpoint{2.796560in}{3.046851in}}%
\pgfpathlineto{\pgfqpoint{2.805597in}{3.067494in}}%
\pgfpathlineto{\pgfqpoint{2.814633in}{3.087821in}}%
\pgfpathlineto{\pgfqpoint{2.823669in}{3.107821in}}%
\pgfpathlineto{\pgfqpoint{2.832705in}{3.127484in}}%
\pgfpathlineto{\pgfqpoint{2.841742in}{3.146798in}}%
\pgfpathlineto{\pgfqpoint{2.850778in}{3.165753in}}%
\pgfpathlineto{\pgfqpoint{2.859814in}{3.184337in}}%
\pgfpathlineto{\pgfqpoint{2.868850in}{3.202539in}}%
\pgfpathlineto{\pgfqpoint{2.877887in}{3.220348in}}%
\pgfpathlineto{\pgfqpoint{2.886923in}{3.237753in}}%
\pgfpathlineto{\pgfqpoint{2.895959in}{3.254744in}}%
\pgfpathlineto{\pgfqpoint{2.904995in}{3.271308in}}%
\pgfpathlineto{\pgfqpoint{2.914032in}{3.287436in}}%
\pgfpathlineto{\pgfqpoint{2.923068in}{3.303115in}}%
\pgfpathlineto{\pgfqpoint{2.932104in}{3.318336in}}%
\pgfpathlineto{\pgfqpoint{2.941140in}{3.333087in}}%
\pgfpathlineto{\pgfqpoint{2.950177in}{3.347357in}}%
\pgfpathlineto{\pgfqpoint{2.959213in}{3.361134in}}%
\pgfpathlineto{\pgfqpoint{2.968249in}{3.374409in}}%
\pgfpathlineto{\pgfqpoint{2.977285in}{3.387169in}}%
\pgfpathlineto{\pgfqpoint{2.986322in}{3.399405in}}%
\pgfpathlineto{\pgfqpoint{2.995358in}{3.411104in}}%
\pgfpathlineto{\pgfqpoint{3.004394in}{3.422256in}}%
\pgfpathlineto{\pgfqpoint{3.013430in}{3.432849in}}%
\pgfpathlineto{\pgfqpoint{3.022467in}{3.442873in}}%
\pgfpathlineto{\pgfqpoint{3.031503in}{3.452317in}}%
\pgfpathlineto{\pgfqpoint{3.040539in}{3.461170in}}%
\pgfpathlineto{\pgfqpoint{3.049576in}{3.469421in}}%
\pgfpathlineto{\pgfqpoint{3.058612in}{3.477057in}}%
\pgfpathlineto{\pgfqpoint{3.067648in}{3.484070in}}%
\pgfpathlineto{\pgfqpoint{3.076684in}{3.490447in}}%
\pgfpathlineto{\pgfqpoint{3.085721in}{3.496178in}}%
\pgfpathlineto{\pgfqpoint{3.094757in}{3.501251in}}%
\pgfpathlineto{\pgfqpoint{3.103793in}{3.505655in}}%
\pgfpathlineto{\pgfqpoint{3.112829in}{3.509380in}}%
\pgfpathlineto{\pgfqpoint{3.121866in}{3.512414in}}%
\pgfpathlineto{\pgfqpoint{3.130902in}{3.514747in}}%
\pgfpathlineto{\pgfqpoint{3.139938in}{3.516367in}}%
\pgfpathlineto{\pgfqpoint{3.148974in}{3.517263in}}%
\pgfpathlineto{\pgfqpoint{3.158011in}{3.517425in}}%
\pgfpathlineto{\pgfqpoint{3.167047in}{3.516841in}}%
\pgfpathlineto{\pgfqpoint{3.176083in}{3.515500in}}%
\pgfpathlineto{\pgfqpoint{3.185119in}{3.513391in}}%
\pgfpathlineto{\pgfqpoint{3.194156in}{3.510503in}}%
\pgfpathlineto{\pgfqpoint{3.203192in}{3.506826in}}%
\pgfpathlineto{\pgfqpoint{3.212228in}{3.502347in}}%
\pgfpathlineto{\pgfqpoint{3.221264in}{3.497057in}}%
\pgfpathlineto{\pgfqpoint{3.230301in}{3.490944in}}%
\pgfpathlineto{\pgfqpoint{3.239337in}{3.483996in}}%
\pgfpathlineto{\pgfqpoint{3.248373in}{3.476204in}}%
\pgfpathlineto{\pgfqpoint{3.257409in}{3.467556in}}%
\pgfpathlineto{\pgfqpoint{3.266446in}{3.458040in}}%
\pgfpathlineto{\pgfqpoint{3.275482in}{3.447646in}}%
\pgfpathlineto{\pgfqpoint{3.275482in}{3.448205in}}%
\pgfpathlineto{\pgfqpoint{3.275482in}{3.448205in}}%
\pgfpathlineto{\pgfqpoint{3.266446in}{3.460061in}}%
\pgfpathlineto{\pgfqpoint{3.257409in}{3.471487in}}%
\pgfpathlineto{\pgfqpoint{3.248373in}{3.482477in}}%
\pgfpathlineto{\pgfqpoint{3.239337in}{3.493023in}}%
\pgfpathlineto{\pgfqpoint{3.230301in}{3.503121in}}%
\pgfpathlineto{\pgfqpoint{3.221264in}{3.512765in}}%
\pgfpathlineto{\pgfqpoint{3.212228in}{3.521948in}}%
\pgfpathlineto{\pgfqpoint{3.203192in}{3.530664in}}%
\pgfpathlineto{\pgfqpoint{3.194156in}{3.538909in}}%
\pgfpathlineto{\pgfqpoint{3.185119in}{3.546674in}}%
\pgfpathlineto{\pgfqpoint{3.176083in}{3.553956in}}%
\pgfpathlineto{\pgfqpoint{3.167047in}{3.560747in}}%
\pgfpathlineto{\pgfqpoint{3.158011in}{3.567042in}}%
\pgfpathlineto{\pgfqpoint{3.148974in}{3.572835in}}%
\pgfpathlineto{\pgfqpoint{3.139938in}{3.578120in}}%
\pgfpathlineto{\pgfqpoint{3.130902in}{3.582890in}}%
\pgfpathlineto{\pgfqpoint{3.121866in}{3.587141in}}%
\pgfpathlineto{\pgfqpoint{3.112829in}{3.590866in}}%
\pgfpathlineto{\pgfqpoint{3.103793in}{3.594058in}}%
\pgfpathlineto{\pgfqpoint{3.094757in}{3.596713in}}%
\pgfpathlineto{\pgfqpoint{3.085721in}{3.598824in}}%
\pgfpathlineto{\pgfqpoint{3.076684in}{3.600385in}}%
\pgfpathlineto{\pgfqpoint{3.067648in}{3.601391in}}%
\pgfpathlineto{\pgfqpoint{3.058612in}{3.601834in}}%
\pgfpathlineto{\pgfqpoint{3.049576in}{3.601710in}}%
\pgfpathlineto{\pgfqpoint{3.040539in}{3.601013in}}%
\pgfpathlineto{\pgfqpoint{3.031503in}{3.599735in}}%
\pgfpathlineto{\pgfqpoint{3.022467in}{3.597873in}}%
\pgfpathlineto{\pgfqpoint{3.013430in}{3.595418in}}%
\pgfpathlineto{\pgfqpoint{3.004394in}{3.592367in}}%
\pgfpathlineto{\pgfqpoint{2.995358in}{3.588712in}}%
\pgfpathlineto{\pgfqpoint{2.986322in}{3.584447in}}%
\pgfpathlineto{\pgfqpoint{2.977285in}{3.579568in}}%
\pgfpathlineto{\pgfqpoint{2.968249in}{3.574067in}}%
\pgfpathlineto{\pgfqpoint{2.959213in}{3.567938in}}%
\pgfpathlineto{\pgfqpoint{2.950177in}{3.561177in}}%
\pgfpathlineto{\pgfqpoint{2.941140in}{3.553777in}}%
\pgfpathlineto{\pgfqpoint{2.932104in}{3.545731in}}%
\pgfpathlineto{\pgfqpoint{2.923068in}{3.537035in}}%
\pgfpathlineto{\pgfqpoint{2.914032in}{3.527681in}}%
\pgfpathlineto{\pgfqpoint{2.904995in}{3.517665in}}%
\pgfpathlineto{\pgfqpoint{2.895959in}{3.506979in}}%
\pgfpathlineto{\pgfqpoint{2.886923in}{3.495619in}}%
\pgfpathlineto{\pgfqpoint{2.877887in}{3.483578in}}%
\pgfpathlineto{\pgfqpoint{2.868850in}{3.470850in}}%
\pgfpathlineto{\pgfqpoint{2.859814in}{3.457430in}}%
\pgfpathlineto{\pgfqpoint{2.850778in}{3.443311in}}%
\pgfpathlineto{\pgfqpoint{2.841742in}{3.428487in}}%
\pgfpathlineto{\pgfqpoint{2.832705in}{3.412953in}}%
\pgfpathlineto{\pgfqpoint{2.823669in}{3.396702in}}%
\pgfpathlineto{\pgfqpoint{2.814633in}{3.379729in}}%
\pgfpathlineto{\pgfqpoint{2.805597in}{3.362027in}}%
\pgfpathlineto{\pgfqpoint{2.796560in}{3.343591in}}%
\pgfpathlineto{\pgfqpoint{2.787524in}{3.324414in}}%
\pgfpathlineto{\pgfqpoint{2.778488in}{3.304492in}}%
\pgfpathlineto{\pgfqpoint{2.769452in}{3.283817in}}%
\pgfpathlineto{\pgfqpoint{2.760415in}{3.262383in}}%
\pgfpathlineto{\pgfqpoint{2.751379in}{3.240186in}}%
\pgfpathlineto{\pgfqpoint{2.742343in}{3.217219in}}%
\pgfpathlineto{\pgfqpoint{2.733307in}{3.193475in}}%
\pgfpathlineto{\pgfqpoint{2.724270in}{3.168950in}}%
\pgfpathlineto{\pgfqpoint{2.715234in}{3.143636in}}%
\pgfpathlineto{\pgfqpoint{2.706198in}{3.117536in}}%
\pgfpathlineto{\pgfqpoint{2.697162in}{3.090673in}}%
\pgfpathlineto{\pgfqpoint{2.688125in}{3.063074in}}%
\pgfpathlineto{\pgfqpoint{2.679089in}{3.034764in}}%
\pgfpathlineto{\pgfqpoint{2.670053in}{3.005770in}}%
\pgfpathlineto{\pgfqpoint{2.661017in}{2.976119in}}%
\pgfpathlineto{\pgfqpoint{2.651980in}{2.945837in}}%
\pgfpathlineto{\pgfqpoint{2.642944in}{2.914952in}}%
\pgfpathlineto{\pgfqpoint{2.633908in}{2.883489in}}%
\pgfpathlineto{\pgfqpoint{2.624872in}{2.851475in}}%
\pgfpathlineto{\pgfqpoint{2.615835in}{2.818936in}}%
\pgfpathlineto{\pgfqpoint{2.606799in}{2.785900in}}%
\pgfpathlineto{\pgfqpoint{2.597763in}{2.752392in}}%
\pgfpathlineto{\pgfqpoint{2.588727in}{2.718439in}}%
\pgfpathlineto{\pgfqpoint{2.579690in}{2.684068in}}%
\pgfpathlineto{\pgfqpoint{2.570654in}{2.649305in}}%
\pgfpathlineto{\pgfqpoint{2.561618in}{2.614178in}}%
\pgfpathlineto{\pgfqpoint{2.552582in}{2.578711in}}%
\pgfpathlineto{\pgfqpoint{2.543545in}{2.542932in}}%
\pgfpathlineto{\pgfqpoint{2.534509in}{2.506868in}}%
\pgfpathlineto{\pgfqpoint{2.525473in}{2.470545in}}%
\pgfpathlineto{\pgfqpoint{2.516437in}{2.433989in}}%
\pgfpathlineto{\pgfqpoint{2.507400in}{2.397228in}}%
\pgfpathlineto{\pgfqpoint{2.498364in}{2.360287in}}%
\pgfpathlineto{\pgfqpoint{2.489328in}{2.323193in}}%
\pgfpathlineto{\pgfqpoint{2.480291in}{2.285973in}}%
\pgfpathlineto{\pgfqpoint{2.471255in}{2.248652in}}%
\pgfpathlineto{\pgfqpoint{2.462219in}{2.211259in}}%
\pgfpathlineto{\pgfqpoint{2.453183in}{2.173819in}}%
\pgfpathlineto{\pgfqpoint{2.444146in}{2.136359in}}%
\pgfpathlineto{\pgfqpoint{2.435110in}{2.098905in}}%
\pgfpathlineto{\pgfqpoint{2.426074in}{2.061485in}}%
\pgfpathlineto{\pgfqpoint{2.426074in}{2.061485in}}%
\pgfpathclose%
\pgfusepath{stroke,fill}%
\end{pgfscope}%
\begin{pgfscope}%
\pgfpathrectangle{\pgfqpoint{0.800000in}{0.528000in}}{\pgfqpoint{4.960000in}{3.696000in}}%
\pgfusepath{clip}%
\pgfsetbuttcap%
\pgfsetroundjoin%
\definecolor{currentfill}{rgb}{0.172549,0.627451,0.172549}%
\pgfsetfillcolor{currentfill}%
\pgfsetfillopacity{0.300000}%
\pgfsetlinewidth{1.003750pt}%
\definecolor{currentstroke}{rgb}{0.172549,0.627451,0.172549}%
\pgfsetstrokecolor{currentstroke}%
\pgfsetstrokeopacity{0.300000}%
\pgfsetdash{}{0pt}%
\pgfpathmoveto{\pgfqpoint{3.320663in}{3.382953in}}%
\pgfpathlineto{\pgfqpoint{3.320663in}{3.382873in}}%
\pgfpathlineto{\pgfqpoint{3.329699in}{3.367601in}}%
\pgfpathlineto{\pgfqpoint{3.338736in}{3.351645in}}%
\pgfpathlineto{\pgfqpoint{3.347772in}{3.335046in}}%
\pgfpathlineto{\pgfqpoint{3.356808in}{3.317841in}}%
\pgfpathlineto{\pgfqpoint{3.365844in}{3.300069in}}%
\pgfpathlineto{\pgfqpoint{3.374881in}{3.281768in}}%
\pgfpathlineto{\pgfqpoint{3.383917in}{3.262977in}}%
\pgfpathlineto{\pgfqpoint{3.392953in}{3.243734in}}%
\pgfpathlineto{\pgfqpoint{3.401989in}{3.224078in}}%
\pgfpathlineto{\pgfqpoint{3.411026in}{3.204048in}}%
\pgfpathlineto{\pgfqpoint{3.420062in}{3.183682in}}%
\pgfpathlineto{\pgfqpoint{3.429098in}{3.163018in}}%
\pgfpathlineto{\pgfqpoint{3.438134in}{3.142095in}}%
\pgfpathlineto{\pgfqpoint{3.447171in}{3.120952in}}%
\pgfpathlineto{\pgfqpoint{3.456207in}{3.099627in}}%
\pgfpathlineto{\pgfqpoint{3.465243in}{3.078159in}}%
\pgfpathlineto{\pgfqpoint{3.474279in}{3.056586in}}%
\pgfpathlineto{\pgfqpoint{3.483316in}{3.034946in}}%
\pgfpathlineto{\pgfqpoint{3.492352in}{3.013279in}}%
\pgfpathlineto{\pgfqpoint{3.501388in}{2.991622in}}%
\pgfpathlineto{\pgfqpoint{3.510424in}{2.970015in}}%
\pgfpathlineto{\pgfqpoint{3.519461in}{2.948495in}}%
\pgfpathlineto{\pgfqpoint{3.528497in}{2.927102in}}%
\pgfpathlineto{\pgfqpoint{3.537533in}{2.905874in}}%
\pgfpathlineto{\pgfqpoint{3.546570in}{2.884849in}}%
\pgfpathlineto{\pgfqpoint{3.555606in}{2.864066in}}%
\pgfpathlineto{\pgfqpoint{3.564642in}{2.843563in}}%
\pgfpathlineto{\pgfqpoint{3.573678in}{2.823380in}}%
\pgfpathlineto{\pgfqpoint{3.582715in}{2.803554in}}%
\pgfpathlineto{\pgfqpoint{3.591751in}{2.784124in}}%
\pgfpathlineto{\pgfqpoint{3.600787in}{2.765128in}}%
\pgfpathlineto{\pgfqpoint{3.609823in}{2.746606in}}%
\pgfpathlineto{\pgfqpoint{3.618860in}{2.728595in}}%
\pgfpathlineto{\pgfqpoint{3.627896in}{2.711134in}}%
\pgfpathlineto{\pgfqpoint{3.636932in}{2.694262in}}%
\pgfpathlineto{\pgfqpoint{3.645968in}{2.678017in}}%
\pgfpathlineto{\pgfqpoint{3.655005in}{2.662438in}}%
\pgfpathlineto{\pgfqpoint{3.664041in}{2.647563in}}%
\pgfpathlineto{\pgfqpoint{3.673077in}{2.633431in}}%
\pgfpathlineto{\pgfqpoint{3.682113in}{2.620080in}}%
\pgfpathlineto{\pgfqpoint{3.691150in}{2.607549in}}%
\pgfpathlineto{\pgfqpoint{3.700186in}{2.595877in}}%
\pgfpathlineto{\pgfqpoint{3.709222in}{2.585101in}}%
\pgfpathlineto{\pgfqpoint{3.718258in}{2.575261in}}%
\pgfpathlineto{\pgfqpoint{3.727295in}{2.566395in}}%
\pgfpathlineto{\pgfqpoint{3.736331in}{2.558541in}}%
\pgfpathlineto{\pgfqpoint{3.745367in}{2.551738in}}%
\pgfpathlineto{\pgfqpoint{3.754403in}{2.546025in}}%
\pgfpathlineto{\pgfqpoint{3.763440in}{2.541439in}}%
\pgfpathlineto{\pgfqpoint{3.772476in}{2.538021in}}%
\pgfpathlineto{\pgfqpoint{3.781512in}{2.535807in}}%
\pgfpathlineto{\pgfqpoint{3.790548in}{2.534837in}}%
\pgfpathlineto{\pgfqpoint{3.799585in}{2.535150in}}%
\pgfpathlineto{\pgfqpoint{3.808621in}{2.536782in}}%
\pgfpathlineto{\pgfqpoint{3.817657in}{2.539775in}}%
\pgfpathlineto{\pgfqpoint{3.826693in}{2.544165in}}%
\pgfpathlineto{\pgfqpoint{3.835730in}{2.549991in}}%
\pgfpathlineto{\pgfqpoint{3.844766in}{2.557292in}}%
\pgfpathlineto{\pgfqpoint{3.853802in}{2.566086in}}%
\pgfpathlineto{\pgfqpoint{3.862838in}{2.576334in}}%
\pgfpathlineto{\pgfqpoint{3.871875in}{2.587988in}}%
\pgfpathlineto{\pgfqpoint{3.880911in}{2.601000in}}%
\pgfpathlineto{\pgfqpoint{3.889947in}{2.615322in}}%
\pgfpathlineto{\pgfqpoint{3.898983in}{2.630906in}}%
\pgfpathlineto{\pgfqpoint{3.908020in}{2.647703in}}%
\pgfpathlineto{\pgfqpoint{3.917056in}{2.665666in}}%
\pgfpathlineto{\pgfqpoint{3.926092in}{2.684745in}}%
\pgfpathlineto{\pgfqpoint{3.935128in}{2.704893in}}%
\pgfpathlineto{\pgfqpoint{3.944165in}{2.726062in}}%
\pgfpathlineto{\pgfqpoint{3.953201in}{2.748204in}}%
\pgfpathlineto{\pgfqpoint{3.962237in}{2.771270in}}%
\pgfpathlineto{\pgfqpoint{3.971273in}{2.795212in}}%
\pgfpathlineto{\pgfqpoint{3.980310in}{2.819982in}}%
\pgfpathlineto{\pgfqpoint{3.989346in}{2.845533in}}%
\pgfpathlineto{\pgfqpoint{3.998382in}{2.871814in}}%
\pgfpathlineto{\pgfqpoint{4.007418in}{2.898779in}}%
\pgfpathlineto{\pgfqpoint{4.016455in}{2.926380in}}%
\pgfpathlineto{\pgfqpoint{4.025491in}{2.954567in}}%
\pgfpathlineto{\pgfqpoint{4.034527in}{2.983294in}}%
\pgfpathlineto{\pgfqpoint{4.043563in}{3.012511in}}%
\pgfpathlineto{\pgfqpoint{4.052600in}{3.042171in}}%
\pgfpathlineto{\pgfqpoint{4.061636in}{3.072226in}}%
\pgfpathlineto{\pgfqpoint{4.070672in}{3.102626in}}%
\pgfpathlineto{\pgfqpoint{4.079709in}{3.133325in}}%
\pgfpathlineto{\pgfqpoint{4.088745in}{3.164273in}}%
\pgfpathlineto{\pgfqpoint{4.097781in}{3.195424in}}%
\pgfpathlineto{\pgfqpoint{4.106817in}{3.226727in}}%
\pgfpathlineto{\pgfqpoint{4.115854in}{3.258136in}}%
\pgfpathlineto{\pgfqpoint{4.124890in}{3.289603in}}%
\pgfpathlineto{\pgfqpoint{4.133926in}{3.321078in}}%
\pgfpathlineto{\pgfqpoint{4.142962in}{3.352514in}}%
\pgfpathlineto{\pgfqpoint{4.151999in}{3.383863in}}%
\pgfpathlineto{\pgfqpoint{4.161035in}{3.415076in}}%
\pgfpathlineto{\pgfqpoint{4.170071in}{3.446105in}}%
\pgfpathlineto{\pgfqpoint{4.179107in}{3.476903in}}%
\pgfpathlineto{\pgfqpoint{4.188144in}{3.507420in}}%
\pgfpathlineto{\pgfqpoint{4.197180in}{3.537610in}}%
\pgfpathlineto{\pgfqpoint{4.206216in}{3.567423in}}%
\pgfpathlineto{\pgfqpoint{4.215252in}{3.596811in}}%
\pgfpathlineto{\pgfqpoint{4.224289in}{3.625726in}}%
\pgfpathlineto{\pgfqpoint{4.233325in}{3.654121in}}%
\pgfpathlineto{\pgfqpoint{4.242361in}{3.681946in}}%
\pgfpathlineto{\pgfqpoint{4.251397in}{3.709154in}}%
\pgfpathlineto{\pgfqpoint{4.260434in}{3.735697in}}%
\pgfpathlineto{\pgfqpoint{4.269470in}{3.761526in}}%
\pgfpathlineto{\pgfqpoint{4.278506in}{3.786593in}}%
\pgfpathlineto{\pgfqpoint{4.287542in}{3.810850in}}%
\pgfpathlineto{\pgfqpoint{4.296579in}{3.834248in}}%
\pgfpathlineto{\pgfqpoint{4.305615in}{3.856741in}}%
\pgfpathlineto{\pgfqpoint{4.314651in}{3.878279in}}%
\pgfpathlineto{\pgfqpoint{4.323687in}{3.898814in}}%
\pgfpathlineto{\pgfqpoint{4.332724in}{3.918298in}}%
\pgfpathlineto{\pgfqpoint{4.341760in}{3.936683in}}%
\pgfpathlineto{\pgfqpoint{4.350796in}{3.953921in}}%
\pgfpathlineto{\pgfqpoint{4.359832in}{3.969963in}}%
\pgfpathlineto{\pgfqpoint{4.368869in}{3.984762in}}%
\pgfpathlineto{\pgfqpoint{4.377905in}{3.998269in}}%
\pgfpathlineto{\pgfqpoint{4.386941in}{4.010436in}}%
\pgfpathlineto{\pgfqpoint{4.395977in}{4.021215in}}%
\pgfpathlineto{\pgfqpoint{4.405014in}{4.030558in}}%
\pgfpathlineto{\pgfqpoint{4.405014in}{4.031766in}}%
\pgfpathlineto{\pgfqpoint{4.405014in}{4.031766in}}%
\pgfpathlineto{\pgfqpoint{4.395977in}{4.027377in}}%
\pgfpathlineto{\pgfqpoint{4.386941in}{4.021732in}}%
\pgfpathlineto{\pgfqpoint{4.377905in}{4.014872in}}%
\pgfpathlineto{\pgfqpoint{4.368869in}{4.006834in}}%
\pgfpathlineto{\pgfqpoint{4.359832in}{3.997658in}}%
\pgfpathlineto{\pgfqpoint{4.350796in}{3.987383in}}%
\pgfpathlineto{\pgfqpoint{4.341760in}{3.976046in}}%
\pgfpathlineto{\pgfqpoint{4.332724in}{3.963687in}}%
\pgfpathlineto{\pgfqpoint{4.323687in}{3.950345in}}%
\pgfpathlineto{\pgfqpoint{4.314651in}{3.936059in}}%
\pgfpathlineto{\pgfqpoint{4.305615in}{3.920866in}}%
\pgfpathlineto{\pgfqpoint{4.296579in}{3.904806in}}%
\pgfpathlineto{\pgfqpoint{4.287542in}{3.887918in}}%
\pgfpathlineto{\pgfqpoint{4.278506in}{3.870241in}}%
\pgfpathlineto{\pgfqpoint{4.269470in}{3.851812in}}%
\pgfpathlineto{\pgfqpoint{4.260434in}{3.832671in}}%
\pgfpathlineto{\pgfqpoint{4.251397in}{3.812857in}}%
\pgfpathlineto{\pgfqpoint{4.242361in}{3.792409in}}%
\pgfpathlineto{\pgfqpoint{4.233325in}{3.771365in}}%
\pgfpathlineto{\pgfqpoint{4.224289in}{3.749763in}}%
\pgfpathlineto{\pgfqpoint{4.215252in}{3.727643in}}%
\pgfpathlineto{\pgfqpoint{4.206216in}{3.705044in}}%
\pgfpathlineto{\pgfqpoint{4.197180in}{3.682004in}}%
\pgfpathlineto{\pgfqpoint{4.188144in}{3.658562in}}%
\pgfpathlineto{\pgfqpoint{4.179107in}{3.634756in}}%
\pgfpathlineto{\pgfqpoint{4.170071in}{3.610626in}}%
\pgfpathlineto{\pgfqpoint{4.161035in}{3.586210in}}%
\pgfpathlineto{\pgfqpoint{4.151999in}{3.561547in}}%
\pgfpathlineto{\pgfqpoint{4.142962in}{3.536676in}}%
\pgfpathlineto{\pgfqpoint{4.133926in}{3.511635in}}%
\pgfpathlineto{\pgfqpoint{4.124890in}{3.486464in}}%
\pgfpathlineto{\pgfqpoint{4.115854in}{3.461201in}}%
\pgfpathlineto{\pgfqpoint{4.106817in}{3.435884in}}%
\pgfpathlineto{\pgfqpoint{4.097781in}{3.410553in}}%
\pgfpathlineto{\pgfqpoint{4.088745in}{3.385246in}}%
\pgfpathlineto{\pgfqpoint{4.079709in}{3.360002in}}%
\pgfpathlineto{\pgfqpoint{4.070672in}{3.334860in}}%
\pgfpathlineto{\pgfqpoint{4.061636in}{3.309858in}}%
\pgfpathlineto{\pgfqpoint{4.052600in}{3.285036in}}%
\pgfpathlineto{\pgfqpoint{4.043563in}{3.260432in}}%
\pgfpathlineto{\pgfqpoint{4.034527in}{3.236084in}}%
\pgfpathlineto{\pgfqpoint{4.025491in}{3.212032in}}%
\pgfpathlineto{\pgfqpoint{4.016455in}{3.188315in}}%
\pgfpathlineto{\pgfqpoint{4.007418in}{3.164970in}}%
\pgfpathlineto{\pgfqpoint{3.998382in}{3.142037in}}%
\pgfpathlineto{\pgfqpoint{3.989346in}{3.119555in}}%
\pgfpathlineto{\pgfqpoint{3.980310in}{3.097562in}}%
\pgfpathlineto{\pgfqpoint{3.971273in}{3.076098in}}%
\pgfpathlineto{\pgfqpoint{3.962237in}{3.055200in}}%
\pgfpathlineto{\pgfqpoint{3.953201in}{3.034907in}}%
\pgfpathlineto{\pgfqpoint{3.944165in}{3.015259in}}%
\pgfpathlineto{\pgfqpoint{3.935128in}{2.996295in}}%
\pgfpathlineto{\pgfqpoint{3.926092in}{2.978052in}}%
\pgfpathlineto{\pgfqpoint{3.917056in}{2.960569in}}%
\pgfpathlineto{\pgfqpoint{3.908020in}{2.943886in}}%
\pgfpathlineto{\pgfqpoint{3.898983in}{2.928041in}}%
\pgfpathlineto{\pgfqpoint{3.889947in}{2.913074in}}%
\pgfpathlineto{\pgfqpoint{3.880911in}{2.899021in}}%
\pgfpathlineto{\pgfqpoint{3.871875in}{2.885923in}}%
\pgfpathlineto{\pgfqpoint{3.862838in}{2.873819in}}%
\pgfpathlineto{\pgfqpoint{3.853802in}{2.862746in}}%
\pgfpathlineto{\pgfqpoint{3.844766in}{2.852744in}}%
\pgfpathlineto{\pgfqpoint{3.835730in}{2.843844in}}%
\pgfpathlineto{\pgfqpoint{3.826693in}{2.836037in}}%
\pgfpathlineto{\pgfqpoint{3.817657in}{2.829298in}}%
\pgfpathlineto{\pgfqpoint{3.808621in}{2.823602in}}%
\pgfpathlineto{\pgfqpoint{3.799585in}{2.818925in}}%
\pgfpathlineto{\pgfqpoint{3.790548in}{2.815243in}}%
\pgfpathlineto{\pgfqpoint{3.781512in}{2.812530in}}%
\pgfpathlineto{\pgfqpoint{3.772476in}{2.810762in}}%
\pgfpathlineto{\pgfqpoint{3.763440in}{2.809914in}}%
\pgfpathlineto{\pgfqpoint{3.754403in}{2.809962in}}%
\pgfpathlineto{\pgfqpoint{3.745367in}{2.810882in}}%
\pgfpathlineto{\pgfqpoint{3.736331in}{2.812647in}}%
\pgfpathlineto{\pgfqpoint{3.727295in}{2.815235in}}%
\pgfpathlineto{\pgfqpoint{3.718258in}{2.818619in}}%
\pgfpathlineto{\pgfqpoint{3.709222in}{2.822777in}}%
\pgfpathlineto{\pgfqpoint{3.700186in}{2.827682in}}%
\pgfpathlineto{\pgfqpoint{3.691150in}{2.833310in}}%
\pgfpathlineto{\pgfqpoint{3.682113in}{2.839637in}}%
\pgfpathlineto{\pgfqpoint{3.673077in}{2.846639in}}%
\pgfpathlineto{\pgfqpoint{3.664041in}{2.854290in}}%
\pgfpathlineto{\pgfqpoint{3.655005in}{2.862565in}}%
\pgfpathlineto{\pgfqpoint{3.645968in}{2.871441in}}%
\pgfpathlineto{\pgfqpoint{3.636932in}{2.880892in}}%
\pgfpathlineto{\pgfqpoint{3.627896in}{2.890895in}}%
\pgfpathlineto{\pgfqpoint{3.618860in}{2.901423in}}%
\pgfpathlineto{\pgfqpoint{3.609823in}{2.912453in}}%
\pgfpathlineto{\pgfqpoint{3.600787in}{2.923961in}}%
\pgfpathlineto{\pgfqpoint{3.591751in}{2.935921in}}%
\pgfpathlineto{\pgfqpoint{3.582715in}{2.948308in}}%
\pgfpathlineto{\pgfqpoint{3.573678in}{2.961099in}}%
\pgfpathlineto{\pgfqpoint{3.564642in}{2.974268in}}%
\pgfpathlineto{\pgfqpoint{3.555606in}{2.987792in}}%
\pgfpathlineto{\pgfqpoint{3.546570in}{3.001644in}}%
\pgfpathlineto{\pgfqpoint{3.537533in}{3.015802in}}%
\pgfpathlineto{\pgfqpoint{3.528497in}{3.030239in}}%
\pgfpathlineto{\pgfqpoint{3.519461in}{3.044932in}}%
\pgfpathlineto{\pgfqpoint{3.510424in}{3.059855in}}%
\pgfpathlineto{\pgfqpoint{3.501388in}{3.074985in}}%
\pgfpathlineto{\pgfqpoint{3.492352in}{3.090296in}}%
\pgfpathlineto{\pgfqpoint{3.483316in}{3.105764in}}%
\pgfpathlineto{\pgfqpoint{3.474279in}{3.121365in}}%
\pgfpathlineto{\pgfqpoint{3.465243in}{3.137073in}}%
\pgfpathlineto{\pgfqpoint{3.456207in}{3.152864in}}%
\pgfpathlineto{\pgfqpoint{3.447171in}{3.168714in}}%
\pgfpathlineto{\pgfqpoint{3.438134in}{3.184597in}}%
\pgfpathlineto{\pgfqpoint{3.429098in}{3.200490in}}%
\pgfpathlineto{\pgfqpoint{3.420062in}{3.216367in}}%
\pgfpathlineto{\pgfqpoint{3.411026in}{3.232204in}}%
\pgfpathlineto{\pgfqpoint{3.401989in}{3.247977in}}%
\pgfpathlineto{\pgfqpoint{3.392953in}{3.263660in}}%
\pgfpathlineto{\pgfqpoint{3.383917in}{3.279229in}}%
\pgfpathlineto{\pgfqpoint{3.374881in}{3.294659in}}%
\pgfpathlineto{\pgfqpoint{3.365844in}{3.309927in}}%
\pgfpathlineto{\pgfqpoint{3.356808in}{3.325006in}}%
\pgfpathlineto{\pgfqpoint{3.347772in}{3.339873in}}%
\pgfpathlineto{\pgfqpoint{3.338736in}{3.354503in}}%
\pgfpathlineto{\pgfqpoint{3.329699in}{3.368871in}}%
\pgfpathlineto{\pgfqpoint{3.320663in}{3.382953in}}%
\pgfpathlineto{\pgfqpoint{3.320663in}{3.382953in}}%
\pgfpathclose%
\pgfusepath{stroke,fill}%
\end{pgfscope}%
\begin{pgfscope}%
\pgfpathrectangle{\pgfqpoint{0.800000in}{0.528000in}}{\pgfqpoint{4.960000in}{3.696000in}}%
\pgfusepath{clip}%
\pgfsetbuttcap%
\pgfsetroundjoin%
\definecolor{currentfill}{rgb}{0.172549,0.627451,0.172549}%
\pgfsetfillcolor{currentfill}%
\pgfsetfillopacity{0.300000}%
\pgfsetlinewidth{1.003750pt}%
\definecolor{currentstroke}{rgb}{0.172549,0.627451,0.172549}%
\pgfsetstrokecolor{currentstroke}%
\pgfsetstrokeopacity{0.300000}%
\pgfsetdash{}{0pt}%
\pgfpathmoveto{\pgfqpoint{4.974298in}{2.542341in}}%
\pgfpathlineto{\pgfqpoint{4.974298in}{2.540990in}}%
\pgfpathlineto{\pgfqpoint{4.983334in}{2.500128in}}%
\pgfpathlineto{\pgfqpoint{4.992370in}{2.459222in}}%
\pgfpathlineto{\pgfqpoint{5.001406in}{2.418294in}}%
\pgfpathlineto{\pgfqpoint{5.010443in}{2.377367in}}%
\pgfpathlineto{\pgfqpoint{5.019479in}{2.336463in}}%
\pgfpathlineto{\pgfqpoint{5.028515in}{2.295606in}}%
\pgfpathlineto{\pgfqpoint{5.037551in}{2.254819in}}%
\pgfpathlineto{\pgfqpoint{5.046588in}{2.214123in}}%
\pgfpathlineto{\pgfqpoint{5.055624in}{2.173542in}}%
\pgfpathlineto{\pgfqpoint{5.064660in}{2.133099in}}%
\pgfpathlineto{\pgfqpoint{5.073696in}{2.092817in}}%
\pgfpathlineto{\pgfqpoint{5.082733in}{2.052717in}}%
\pgfpathlineto{\pgfqpoint{5.091769in}{2.012824in}}%
\pgfpathlineto{\pgfqpoint{5.100805in}{1.973159in}}%
\pgfpathlineto{\pgfqpoint{5.109842in}{1.933746in}}%
\pgfpathlineto{\pgfqpoint{5.118878in}{1.894607in}}%
\pgfpathlineto{\pgfqpoint{5.127914in}{1.855765in}}%
\pgfpathlineto{\pgfqpoint{5.136950in}{1.817243in}}%
\pgfpathlineto{\pgfqpoint{5.145987in}{1.779063in}}%
\pgfpathlineto{\pgfqpoint{5.155023in}{1.741249in}}%
\pgfpathlineto{\pgfqpoint{5.164059in}{1.703823in}}%
\pgfpathlineto{\pgfqpoint{5.173095in}{1.666808in}}%
\pgfpathlineto{\pgfqpoint{5.182132in}{1.630227in}}%
\pgfpathlineto{\pgfqpoint{5.191168in}{1.594103in}}%
\pgfpathlineto{\pgfqpoint{5.200204in}{1.558457in}}%
\pgfpathlineto{\pgfqpoint{5.209240in}{1.523314in}}%
\pgfpathlineto{\pgfqpoint{5.218277in}{1.488695in}}%
\pgfpathlineto{\pgfqpoint{5.227313in}{1.454624in}}%
\pgfpathlineto{\pgfqpoint{5.236349in}{1.421124in}}%
\pgfpathlineto{\pgfqpoint{5.245385in}{1.388216in}}%
\pgfpathlineto{\pgfqpoint{5.254422in}{1.355925in}}%
\pgfpathlineto{\pgfqpoint{5.263458in}{1.324272in}}%
\pgfpathlineto{\pgfqpoint{5.272494in}{1.293281in}}%
\pgfpathlineto{\pgfqpoint{5.281530in}{1.262974in}}%
\pgfpathlineto{\pgfqpoint{5.290567in}{1.233374in}}%
\pgfpathlineto{\pgfqpoint{5.299603in}{1.204504in}}%
\pgfpathlineto{\pgfqpoint{5.308639in}{1.176386in}}%
\pgfpathlineto{\pgfqpoint{5.317675in}{1.149044in}}%
\pgfpathlineto{\pgfqpoint{5.326712in}{1.122500in}}%
\pgfpathlineto{\pgfqpoint{5.335748in}{1.096777in}}%
\pgfpathlineto{\pgfqpoint{5.344784in}{1.071897in}}%
\pgfpathlineto{\pgfqpoint{5.353820in}{1.047884in}}%
\pgfpathlineto{\pgfqpoint{5.362857in}{1.024760in}}%
\pgfpathlineto{\pgfqpoint{5.371893in}{1.002549in}}%
\pgfpathlineto{\pgfqpoint{5.380929in}{0.981272in}}%
\pgfpathlineto{\pgfqpoint{5.389965in}{0.960952in}}%
\pgfpathlineto{\pgfqpoint{5.399002in}{0.941613in}}%
\pgfpathlineto{\pgfqpoint{5.408038in}{0.923277in}}%
\pgfpathlineto{\pgfqpoint{5.417074in}{0.905967in}}%
\pgfpathlineto{\pgfqpoint{5.426110in}{0.889706in}}%
\pgfpathlineto{\pgfqpoint{5.435147in}{0.874516in}}%
\pgfpathlineto{\pgfqpoint{5.444183in}{0.860421in}}%
\pgfpathlineto{\pgfqpoint{5.453219in}{0.847442in}}%
\pgfpathlineto{\pgfqpoint{5.462255in}{0.835603in}}%
\pgfpathlineto{\pgfqpoint{5.471292in}{0.824927in}}%
\pgfpathlineto{\pgfqpoint{5.480328in}{0.815436in}}%
\pgfpathlineto{\pgfqpoint{5.489364in}{0.807153in}}%
\pgfpathlineto{\pgfqpoint{5.498400in}{0.800101in}}%
\pgfpathlineto{\pgfqpoint{5.507437in}{0.794303in}}%
\pgfpathlineto{\pgfqpoint{5.516473in}{0.789781in}}%
\pgfpathlineto{\pgfqpoint{5.525509in}{0.786559in}}%
\pgfpathlineto{\pgfqpoint{5.534545in}{0.784658in}}%
\pgfpathlineto{\pgfqpoint{5.534545in}{1.079931in}}%
\pgfpathlineto{\pgfqpoint{5.534545in}{1.079931in}}%
\pgfpathlineto{\pgfqpoint{5.525509in}{1.077155in}}%
\pgfpathlineto{\pgfqpoint{5.516473in}{1.075660in}}%
\pgfpathlineto{\pgfqpoint{5.507437in}{1.075427in}}%
\pgfpathlineto{\pgfqpoint{5.498400in}{1.076434in}}%
\pgfpathlineto{\pgfqpoint{5.489364in}{1.078660in}}%
\pgfpathlineto{\pgfqpoint{5.480328in}{1.082086in}}%
\pgfpathlineto{\pgfqpoint{5.471292in}{1.086689in}}%
\pgfpathlineto{\pgfqpoint{5.462255in}{1.092449in}}%
\pgfpathlineto{\pgfqpoint{5.453219in}{1.099345in}}%
\pgfpathlineto{\pgfqpoint{5.444183in}{1.107356in}}%
\pgfpathlineto{\pgfqpoint{5.435147in}{1.116461in}}%
\pgfpathlineto{\pgfqpoint{5.426110in}{1.126640in}}%
\pgfpathlineto{\pgfqpoint{5.417074in}{1.137871in}}%
\pgfpathlineto{\pgfqpoint{5.408038in}{1.150134in}}%
\pgfpathlineto{\pgfqpoint{5.399002in}{1.163407in}}%
\pgfpathlineto{\pgfqpoint{5.389965in}{1.177670in}}%
\pgfpathlineto{\pgfqpoint{5.380929in}{1.192902in}}%
\pgfpathlineto{\pgfqpoint{5.371893in}{1.209082in}}%
\pgfpathlineto{\pgfqpoint{5.362857in}{1.226189in}}%
\pgfpathlineto{\pgfqpoint{5.353820in}{1.244202in}}%
\pgfpathlineto{\pgfqpoint{5.344784in}{1.263100in}}%
\pgfpathlineto{\pgfqpoint{5.335748in}{1.282863in}}%
\pgfpathlineto{\pgfqpoint{5.326712in}{1.303469in}}%
\pgfpathlineto{\pgfqpoint{5.317675in}{1.324898in}}%
\pgfpathlineto{\pgfqpoint{5.308639in}{1.347129in}}%
\pgfpathlineto{\pgfqpoint{5.299603in}{1.370140in}}%
\pgfpathlineto{\pgfqpoint{5.290567in}{1.393911in}}%
\pgfpathlineto{\pgfqpoint{5.281530in}{1.418422in}}%
\pgfpathlineto{\pgfqpoint{5.272494in}{1.443650in}}%
\pgfpathlineto{\pgfqpoint{5.263458in}{1.469576in}}%
\pgfpathlineto{\pgfqpoint{5.254422in}{1.496178in}}%
\pgfpathlineto{\pgfqpoint{5.245385in}{1.523436in}}%
\pgfpathlineto{\pgfqpoint{5.236349in}{1.551328in}}%
\pgfpathlineto{\pgfqpoint{5.227313in}{1.579834in}}%
\pgfpathlineto{\pgfqpoint{5.218277in}{1.608933in}}%
\pgfpathlineto{\pgfqpoint{5.209240in}{1.638604in}}%
\pgfpathlineto{\pgfqpoint{5.200204in}{1.668825in}}%
\pgfpathlineto{\pgfqpoint{5.191168in}{1.699577in}}%
\pgfpathlineto{\pgfqpoint{5.182132in}{1.730838in}}%
\pgfpathlineto{\pgfqpoint{5.173095in}{1.762588in}}%
\pgfpathlineto{\pgfqpoint{5.164059in}{1.794805in}}%
\pgfpathlineto{\pgfqpoint{5.155023in}{1.827468in}}%
\pgfpathlineto{\pgfqpoint{5.145987in}{1.860557in}}%
\pgfpathlineto{\pgfqpoint{5.136950in}{1.894051in}}%
\pgfpathlineto{\pgfqpoint{5.127914in}{1.927929in}}%
\pgfpathlineto{\pgfqpoint{5.118878in}{1.962170in}}%
\pgfpathlineto{\pgfqpoint{5.109842in}{1.996752in}}%
\pgfpathlineto{\pgfqpoint{5.100805in}{2.031657in}}%
\pgfpathlineto{\pgfqpoint{5.091769in}{2.066861in}}%
\pgfpathlineto{\pgfqpoint{5.082733in}{2.102345in}}%
\pgfpathlineto{\pgfqpoint{5.073696in}{2.138087in}}%
\pgfpathlineto{\pgfqpoint{5.064660in}{2.174067in}}%
\pgfpathlineto{\pgfqpoint{5.055624in}{2.210264in}}%
\pgfpathlineto{\pgfqpoint{5.046588in}{2.246656in}}%
\pgfpathlineto{\pgfqpoint{5.037551in}{2.283224in}}%
\pgfpathlineto{\pgfqpoint{5.028515in}{2.319946in}}%
\pgfpathlineto{\pgfqpoint{5.019479in}{2.356801in}}%
\pgfpathlineto{\pgfqpoint{5.010443in}{2.393768in}}%
\pgfpathlineto{\pgfqpoint{5.001406in}{2.430826in}}%
\pgfpathlineto{\pgfqpoint{4.992370in}{2.467955in}}%
\pgfpathlineto{\pgfqpoint{4.983334in}{2.505134in}}%
\pgfpathlineto{\pgfqpoint{4.974298in}{2.542341in}}%
\pgfpathlineto{\pgfqpoint{4.974298in}{2.542341in}}%
\pgfpathclose%
\pgfusepath{stroke,fill}%
\end{pgfscope}%
\begin{pgfscope}%
\pgfpathrectangle{\pgfqpoint{0.800000in}{0.528000in}}{\pgfqpoint{4.960000in}{3.696000in}}%
\pgfusepath{clip}%
\pgfsetbuttcap%
\pgfsetroundjoin%
\definecolor{currentfill}{rgb}{0.839216,0.152941,0.156863}%
\pgfsetfillcolor{currentfill}%
\pgfsetfillopacity{0.300000}%
\pgfsetlinewidth{1.003750pt}%
\definecolor{currentstroke}{rgb}{0.839216,0.152941,0.156863}%
\pgfsetstrokecolor{currentstroke}%
\pgfsetstrokeopacity{0.300000}%
\pgfsetdash{}{0pt}%
\pgfpathmoveto{\pgfqpoint{1.594739in}{0.781043in}}%
\pgfpathlineto{\pgfqpoint{1.594739in}{0.786325in}}%
\pgfpathlineto{\pgfqpoint{1.603775in}{0.789116in}}%
\pgfpathlineto{\pgfqpoint{1.612811in}{0.792065in}}%
\pgfpathlineto{\pgfqpoint{1.621847in}{0.795174in}}%
\pgfpathlineto{\pgfqpoint{1.630884in}{0.798447in}}%
\pgfpathlineto{\pgfqpoint{1.639920in}{0.801887in}}%
\pgfpathlineto{\pgfqpoint{1.648956in}{0.805497in}}%
\pgfpathlineto{\pgfqpoint{1.657992in}{0.809280in}}%
\pgfpathlineto{\pgfqpoint{1.667029in}{0.813240in}}%
\pgfpathlineto{\pgfqpoint{1.676065in}{0.817380in}}%
\pgfpathlineto{\pgfqpoint{1.685101in}{0.821702in}}%
\pgfpathlineto{\pgfqpoint{1.694137in}{0.826212in}}%
\pgfpathlineto{\pgfqpoint{1.703174in}{0.830911in}}%
\pgfpathlineto{\pgfqpoint{1.712210in}{0.835803in}}%
\pgfpathlineto{\pgfqpoint{1.721246in}{0.840891in}}%
\pgfpathlineto{\pgfqpoint{1.730282in}{0.846179in}}%
\pgfpathlineto{\pgfqpoint{1.739319in}{0.851670in}}%
\pgfpathlineto{\pgfqpoint{1.748355in}{0.857366in}}%
\pgfpathlineto{\pgfqpoint{1.757391in}{0.863272in}}%
\pgfpathlineto{\pgfqpoint{1.766427in}{0.869390in}}%
\pgfpathlineto{\pgfqpoint{1.775464in}{0.875724in}}%
\pgfpathlineto{\pgfqpoint{1.784500in}{0.882278in}}%
\pgfpathlineto{\pgfqpoint{1.793536in}{0.889053in}}%
\pgfpathlineto{\pgfqpoint{1.802572in}{0.896054in}}%
\pgfpathlineto{\pgfqpoint{1.811609in}{0.903284in}}%
\pgfpathlineto{\pgfqpoint{1.820645in}{0.910746in}}%
\pgfpathlineto{\pgfqpoint{1.829681in}{0.918444in}}%
\pgfpathlineto{\pgfqpoint{1.838717in}{0.926380in}}%
\pgfpathlineto{\pgfqpoint{1.847754in}{0.934558in}}%
\pgfpathlineto{\pgfqpoint{1.856790in}{0.942981in}}%
\pgfpathlineto{\pgfqpoint{1.865826in}{0.951652in}}%
\pgfpathlineto{\pgfqpoint{1.874862in}{0.960576in}}%
\pgfpathlineto{\pgfqpoint{1.883899in}{0.969754in}}%
\pgfpathlineto{\pgfqpoint{1.892935in}{0.979190in}}%
\pgfpathlineto{\pgfqpoint{1.901971in}{0.988888in}}%
\pgfpathlineto{\pgfqpoint{1.911007in}{0.998851in}}%
\pgfpathlineto{\pgfqpoint{1.920044in}{1.009081in}}%
\pgfpathlineto{\pgfqpoint{1.929080in}{1.019583in}}%
\pgfpathlineto{\pgfqpoint{1.938116in}{1.030359in}}%
\pgfpathlineto{\pgfqpoint{1.947152in}{1.041413in}}%
\pgfpathlineto{\pgfqpoint{1.956189in}{1.052748in}}%
\pgfpathlineto{\pgfqpoint{1.965225in}{1.064368in}}%
\pgfpathlineto{\pgfqpoint{1.974261in}{1.076275in}}%
\pgfpathlineto{\pgfqpoint{1.983298in}{1.088472in}}%
\pgfpathlineto{\pgfqpoint{1.992334in}{1.100964in}}%
\pgfpathlineto{\pgfqpoint{2.001370in}{1.113753in}}%
\pgfpathlineto{\pgfqpoint{2.010406in}{1.126843in}}%
\pgfpathlineto{\pgfqpoint{2.019443in}{1.140237in}}%
\pgfpathlineto{\pgfqpoint{2.028479in}{1.153938in}}%
\pgfpathlineto{\pgfqpoint{2.037515in}{1.167949in}}%
\pgfpathlineto{\pgfqpoint{2.046551in}{1.182274in}}%
\pgfpathlineto{\pgfqpoint{2.055588in}{1.196915in}}%
\pgfpathlineto{\pgfqpoint{2.064624in}{1.211877in}}%
\pgfpathlineto{\pgfqpoint{2.073660in}{1.227163in}}%
\pgfpathlineto{\pgfqpoint{2.082696in}{1.242775in}}%
\pgfpathlineto{\pgfqpoint{2.091733in}{1.258717in}}%
\pgfpathlineto{\pgfqpoint{2.100769in}{1.274992in}}%
\pgfpathlineto{\pgfqpoint{2.109805in}{1.291604in}}%
\pgfpathlineto{\pgfqpoint{2.118841in}{1.308555in}}%
\pgfpathlineto{\pgfqpoint{2.127878in}{1.325850in}}%
\pgfpathlineto{\pgfqpoint{2.136914in}{1.343490in}}%
\pgfpathlineto{\pgfqpoint{2.145950in}{1.361481in}}%
\pgfpathlineto{\pgfqpoint{2.154986in}{1.379824in}}%
\pgfpathlineto{\pgfqpoint{2.164023in}{1.398519in}}%
\pgfpathlineto{\pgfqpoint{2.173059in}{1.417558in}}%
\pgfpathlineto{\pgfqpoint{2.182095in}{1.436932in}}%
\pgfpathlineto{\pgfqpoint{2.191131in}{1.456632in}}%
\pgfpathlineto{\pgfqpoint{2.200168in}{1.476649in}}%
\pgfpathlineto{\pgfqpoint{2.209204in}{1.496974in}}%
\pgfpathlineto{\pgfqpoint{2.218240in}{1.517598in}}%
\pgfpathlineto{\pgfqpoint{2.227276in}{1.538512in}}%
\pgfpathlineto{\pgfqpoint{2.236313in}{1.559706in}}%
\pgfpathlineto{\pgfqpoint{2.245349in}{1.581173in}}%
\pgfpathlineto{\pgfqpoint{2.254385in}{1.602903in}}%
\pgfpathlineto{\pgfqpoint{2.263421in}{1.624886in}}%
\pgfpathlineto{\pgfqpoint{2.272458in}{1.647114in}}%
\pgfpathlineto{\pgfqpoint{2.281494in}{1.669579in}}%
\pgfpathlineto{\pgfqpoint{2.290530in}{1.692270in}}%
\pgfpathlineto{\pgfqpoint{2.299566in}{1.715179in}}%
\pgfpathlineto{\pgfqpoint{2.308603in}{1.738296in}}%
\pgfpathlineto{\pgfqpoint{2.317639in}{1.761614in}}%
\pgfpathlineto{\pgfqpoint{2.326675in}{1.785122in}}%
\pgfpathlineto{\pgfqpoint{2.335711in}{1.808813in}}%
\pgfpathlineto{\pgfqpoint{2.344748in}{1.832676in}}%
\pgfpathlineto{\pgfqpoint{2.353784in}{1.856703in}}%
\pgfpathlineto{\pgfqpoint{2.362820in}{1.880884in}}%
\pgfpathlineto{\pgfqpoint{2.371856in}{1.905212in}}%
\pgfpathlineto{\pgfqpoint{2.380893in}{1.929676in}}%
\pgfpathlineto{\pgfqpoint{2.389929in}{1.954268in}}%
\pgfpathlineto{\pgfqpoint{2.398965in}{1.978979in}}%
\pgfpathlineto{\pgfqpoint{2.408001in}{2.003800in}}%
\pgfpathlineto{\pgfqpoint{2.417038in}{2.028721in}}%
\pgfpathlineto{\pgfqpoint{2.417038in}{2.024123in}}%
\pgfpathlineto{\pgfqpoint{2.417038in}{2.024123in}}%
\pgfpathlineto{\pgfqpoint{2.408001in}{1.986848in}}%
\pgfpathlineto{\pgfqpoint{2.398965in}{1.949685in}}%
\pgfpathlineto{\pgfqpoint{2.389929in}{1.912662in}}%
\pgfpathlineto{\pgfqpoint{2.380893in}{1.875804in}}%
\pgfpathlineto{\pgfqpoint{2.371856in}{1.839138in}}%
\pgfpathlineto{\pgfqpoint{2.362820in}{1.802691in}}%
\pgfpathlineto{\pgfqpoint{2.353784in}{1.766489in}}%
\pgfpathlineto{\pgfqpoint{2.344748in}{1.730559in}}%
\pgfpathlineto{\pgfqpoint{2.335711in}{1.694927in}}%
\pgfpathlineto{\pgfqpoint{2.326675in}{1.659621in}}%
\pgfpathlineto{\pgfqpoint{2.317639in}{1.624665in}}%
\pgfpathlineto{\pgfqpoint{2.308603in}{1.590088in}}%
\pgfpathlineto{\pgfqpoint{2.299566in}{1.555915in}}%
\pgfpathlineto{\pgfqpoint{2.290530in}{1.522173in}}%
\pgfpathlineto{\pgfqpoint{2.281494in}{1.488889in}}%
\pgfpathlineto{\pgfqpoint{2.272458in}{1.456089in}}%
\pgfpathlineto{\pgfqpoint{2.263421in}{1.423800in}}%
\pgfpathlineto{\pgfqpoint{2.254385in}{1.392047in}}%
\pgfpathlineto{\pgfqpoint{2.245349in}{1.360859in}}%
\pgfpathlineto{\pgfqpoint{2.236313in}{1.330261in}}%
\pgfpathlineto{\pgfqpoint{2.227276in}{1.300280in}}%
\pgfpathlineto{\pgfqpoint{2.218240in}{1.270942in}}%
\pgfpathlineto{\pgfqpoint{2.209204in}{1.242274in}}%
\pgfpathlineto{\pgfqpoint{2.200168in}{1.214303in}}%
\pgfpathlineto{\pgfqpoint{2.191131in}{1.187054in}}%
\pgfpathlineto{\pgfqpoint{2.182095in}{1.160555in}}%
\pgfpathlineto{\pgfqpoint{2.173059in}{1.134832in}}%
\pgfpathlineto{\pgfqpoint{2.164023in}{1.109912in}}%
\pgfpathlineto{\pgfqpoint{2.154986in}{1.085821in}}%
\pgfpathlineto{\pgfqpoint{2.145950in}{1.062583in}}%
\pgfpathlineto{\pgfqpoint{2.136914in}{1.040198in}}%
\pgfpathlineto{\pgfqpoint{2.127878in}{1.018653in}}%
\pgfpathlineto{\pgfqpoint{2.118841in}{0.997936in}}%
\pgfpathlineto{\pgfqpoint{2.109805in}{0.978034in}}%
\pgfpathlineto{\pgfqpoint{2.100769in}{0.958935in}}%
\pgfpathlineto{\pgfqpoint{2.091733in}{0.940625in}}%
\pgfpathlineto{\pgfqpoint{2.082696in}{0.923092in}}%
\pgfpathlineto{\pgfqpoint{2.073660in}{0.906324in}}%
\pgfpathlineto{\pgfqpoint{2.064624in}{0.890309in}}%
\pgfpathlineto{\pgfqpoint{2.055588in}{0.875032in}}%
\pgfpathlineto{\pgfqpoint{2.046551in}{0.860483in}}%
\pgfpathlineto{\pgfqpoint{2.037515in}{0.846647in}}%
\pgfpathlineto{\pgfqpoint{2.028479in}{0.833513in}}%
\pgfpathlineto{\pgfqpoint{2.019443in}{0.821069in}}%
\pgfpathlineto{\pgfqpoint{2.010406in}{0.809300in}}%
\pgfpathlineto{\pgfqpoint{2.001370in}{0.798196in}}%
\pgfpathlineto{\pgfqpoint{1.992334in}{0.787742in}}%
\pgfpathlineto{\pgfqpoint{1.983298in}{0.777927in}}%
\pgfpathlineto{\pgfqpoint{1.974261in}{0.768738in}}%
\pgfpathlineto{\pgfqpoint{1.965225in}{0.760163in}}%
\pgfpathlineto{\pgfqpoint{1.956189in}{0.752188in}}%
\pgfpathlineto{\pgfqpoint{1.947152in}{0.744801in}}%
\pgfpathlineto{\pgfqpoint{1.938116in}{0.737990in}}%
\pgfpathlineto{\pgfqpoint{1.929080in}{0.731741in}}%
\pgfpathlineto{\pgfqpoint{1.920044in}{0.726043in}}%
\pgfpathlineto{\pgfqpoint{1.911007in}{0.720883in}}%
\pgfpathlineto{\pgfqpoint{1.901971in}{0.716248in}}%
\pgfpathlineto{\pgfqpoint{1.892935in}{0.712125in}}%
\pgfpathlineto{\pgfqpoint{1.883899in}{0.708502in}}%
\pgfpathlineto{\pgfqpoint{1.874862in}{0.705366in}}%
\pgfpathlineto{\pgfqpoint{1.865826in}{0.702705in}}%
\pgfpathlineto{\pgfqpoint{1.856790in}{0.700506in}}%
\pgfpathlineto{\pgfqpoint{1.847754in}{0.698756in}}%
\pgfpathlineto{\pgfqpoint{1.838717in}{0.697444in}}%
\pgfpathlineto{\pgfqpoint{1.829681in}{0.696555in}}%
\pgfpathlineto{\pgfqpoint{1.820645in}{0.696078in}}%
\pgfpathlineto{\pgfqpoint{1.811609in}{0.696000in}}%
\pgfpathlineto{\pgfqpoint{1.802572in}{0.696308in}}%
\pgfpathlineto{\pgfqpoint{1.793536in}{0.696991in}}%
\pgfpathlineto{\pgfqpoint{1.784500in}{0.698034in}}%
\pgfpathlineto{\pgfqpoint{1.775464in}{0.699426in}}%
\pgfpathlineto{\pgfqpoint{1.766427in}{0.701154in}}%
\pgfpathlineto{\pgfqpoint{1.757391in}{0.703205in}}%
\pgfpathlineto{\pgfqpoint{1.748355in}{0.705567in}}%
\pgfpathlineto{\pgfqpoint{1.739319in}{0.708227in}}%
\pgfpathlineto{\pgfqpoint{1.730282in}{0.711172in}}%
\pgfpathlineto{\pgfqpoint{1.721246in}{0.714391in}}%
\pgfpathlineto{\pgfqpoint{1.712210in}{0.717869in}}%
\pgfpathlineto{\pgfqpoint{1.703174in}{0.721596in}}%
\pgfpathlineto{\pgfqpoint{1.694137in}{0.725557in}}%
\pgfpathlineto{\pgfqpoint{1.685101in}{0.729741in}}%
\pgfpathlineto{\pgfqpoint{1.676065in}{0.734135in}}%
\pgfpathlineto{\pgfqpoint{1.667029in}{0.738726in}}%
\pgfpathlineto{\pgfqpoint{1.657992in}{0.743502in}}%
\pgfpathlineto{\pgfqpoint{1.648956in}{0.748450in}}%
\pgfpathlineto{\pgfqpoint{1.639920in}{0.753557in}}%
\pgfpathlineto{\pgfqpoint{1.630884in}{0.758811in}}%
\pgfpathlineto{\pgfqpoint{1.621847in}{0.764199in}}%
\pgfpathlineto{\pgfqpoint{1.612811in}{0.769709in}}%
\pgfpathlineto{\pgfqpoint{1.603775in}{0.775328in}}%
\pgfpathlineto{\pgfqpoint{1.594739in}{0.781043in}}%
\pgfpathlineto{\pgfqpoint{1.594739in}{0.781043in}}%
\pgfpathclose%
\pgfusepath{stroke,fill}%
\end{pgfscope}%
\begin{pgfscope}%
\pgfpathrectangle{\pgfqpoint{0.800000in}{0.528000in}}{\pgfqpoint{4.960000in}{3.696000in}}%
\pgfusepath{clip}%
\pgfsetbuttcap%
\pgfsetroundjoin%
\definecolor{currentfill}{rgb}{0.839216,0.152941,0.156863}%
\pgfsetfillcolor{currentfill}%
\pgfsetfillopacity{0.300000}%
\pgfsetlinewidth{1.003750pt}%
\definecolor{currentstroke}{rgb}{0.839216,0.152941,0.156863}%
\pgfsetstrokecolor{currentstroke}%
\pgfsetstrokeopacity{0.300000}%
\pgfsetdash{}{0pt}%
\pgfpathmoveto{\pgfqpoint{3.284518in}{3.435924in}}%
\pgfpathlineto{\pgfqpoint{3.284518in}{3.436365in}}%
\pgfpathlineto{\pgfqpoint{3.293554in}{3.424209in}}%
\pgfpathlineto{\pgfqpoint{3.302591in}{3.411216in}}%
\pgfpathlineto{\pgfqpoint{3.311627in}{3.397424in}}%
\pgfpathlineto{\pgfqpoint{3.311627in}{3.396723in}}%
\pgfpathlineto{\pgfqpoint{3.311627in}{3.396723in}}%
\pgfpathlineto{\pgfqpoint{3.302591in}{3.410159in}}%
\pgfpathlineto{\pgfqpoint{3.293554in}{3.423234in}}%
\pgfpathlineto{\pgfqpoint{3.284518in}{3.435924in}}%
\pgfpathlineto{\pgfqpoint{3.284518in}{3.435924in}}%
\pgfpathclose%
\pgfusepath{stroke,fill}%
\end{pgfscope}%
\begin{pgfscope}%
\pgfpathrectangle{\pgfqpoint{0.800000in}{0.528000in}}{\pgfqpoint{4.960000in}{3.696000in}}%
\pgfusepath{clip}%
\pgfsetbuttcap%
\pgfsetroundjoin%
\definecolor{currentfill}{rgb}{0.839216,0.152941,0.156863}%
\pgfsetfillcolor{currentfill}%
\pgfsetfillopacity{0.300000}%
\pgfsetlinewidth{1.003750pt}%
\definecolor{currentstroke}{rgb}{0.839216,0.152941,0.156863}%
\pgfsetstrokecolor{currentstroke}%
\pgfsetstrokeopacity{0.300000}%
\pgfsetdash{}{0pt}%
\pgfpathmoveto{\pgfqpoint{4.414050in}{4.034867in}}%
\pgfpathlineto{\pgfqpoint{4.414050in}{4.038421in}}%
\pgfpathlineto{\pgfqpoint{4.423086in}{4.044804in}}%
\pgfpathlineto{\pgfqpoint{4.432122in}{4.049731in}}%
\pgfpathlineto{\pgfqpoint{4.441159in}{4.053224in}}%
\pgfpathlineto{\pgfqpoint{4.450195in}{4.055306in}}%
\pgfpathlineto{\pgfqpoint{4.459231in}{4.056000in}}%
\pgfpathlineto{\pgfqpoint{4.468267in}{4.055328in}}%
\pgfpathlineto{\pgfqpoint{4.477304in}{4.053313in}}%
\pgfpathlineto{\pgfqpoint{4.486340in}{4.049978in}}%
\pgfpathlineto{\pgfqpoint{4.495376in}{4.045346in}}%
\pgfpathlineto{\pgfqpoint{4.504412in}{4.039440in}}%
\pgfpathlineto{\pgfqpoint{4.513449in}{4.032281in}}%
\pgfpathlineto{\pgfqpoint{4.522485in}{4.023894in}}%
\pgfpathlineto{\pgfqpoint{4.531521in}{4.014301in}}%
\pgfpathlineto{\pgfqpoint{4.540557in}{4.003524in}}%
\pgfpathlineto{\pgfqpoint{4.549594in}{3.991586in}}%
\pgfpathlineto{\pgfqpoint{4.558630in}{3.978511in}}%
\pgfpathlineto{\pgfqpoint{4.567666in}{3.964321in}}%
\pgfpathlineto{\pgfqpoint{4.576702in}{3.949038in}}%
\pgfpathlineto{\pgfqpoint{4.585739in}{3.932686in}}%
\pgfpathlineto{\pgfqpoint{4.594775in}{3.915287in}}%
\pgfpathlineto{\pgfqpoint{4.603811in}{3.896864in}}%
\pgfpathlineto{\pgfqpoint{4.612848in}{3.877441in}}%
\pgfpathlineto{\pgfqpoint{4.621884in}{3.857038in}}%
\pgfpathlineto{\pgfqpoint{4.630920in}{3.835680in}}%
\pgfpathlineto{\pgfqpoint{4.639956in}{3.813389in}}%
\pgfpathlineto{\pgfqpoint{4.648993in}{3.790189in}}%
\pgfpathlineto{\pgfqpoint{4.658029in}{3.766100in}}%
\pgfpathlineto{\pgfqpoint{4.667065in}{3.741148in}}%
\pgfpathlineto{\pgfqpoint{4.676101in}{3.715353in}}%
\pgfpathlineto{\pgfqpoint{4.685138in}{3.688740in}}%
\pgfpathlineto{\pgfqpoint{4.694174in}{3.661330in}}%
\pgfpathlineto{\pgfqpoint{4.703210in}{3.633147in}}%
\pgfpathlineto{\pgfqpoint{4.712246in}{3.604214in}}%
\pgfpathlineto{\pgfqpoint{4.721283in}{3.574552in}}%
\pgfpathlineto{\pgfqpoint{4.730319in}{3.544186in}}%
\pgfpathlineto{\pgfqpoint{4.739355in}{3.513137in}}%
\pgfpathlineto{\pgfqpoint{4.748391in}{3.481428in}}%
\pgfpathlineto{\pgfqpoint{4.757428in}{3.449083in}}%
\pgfpathlineto{\pgfqpoint{4.766464in}{3.416124in}}%
\pgfpathlineto{\pgfqpoint{4.775500in}{3.382574in}}%
\pgfpathlineto{\pgfqpoint{4.784536in}{3.348455in}}%
\pgfpathlineto{\pgfqpoint{4.793573in}{3.313790in}}%
\pgfpathlineto{\pgfqpoint{4.802609in}{3.278603in}}%
\pgfpathlineto{\pgfqpoint{4.811645in}{3.242916in}}%
\pgfpathlineto{\pgfqpoint{4.820681in}{3.206751in}}%
\pgfpathlineto{\pgfqpoint{4.829718in}{3.170132in}}%
\pgfpathlineto{\pgfqpoint{4.838754in}{3.133081in}}%
\pgfpathlineto{\pgfqpoint{4.847790in}{3.095621in}}%
\pgfpathlineto{\pgfqpoint{4.856826in}{3.057774in}}%
\pgfpathlineto{\pgfqpoint{4.865863in}{3.019565in}}%
\pgfpathlineto{\pgfqpoint{4.874899in}{2.981014in}}%
\pgfpathlineto{\pgfqpoint{4.883935in}{2.942146in}}%
\pgfpathlineto{\pgfqpoint{4.892971in}{2.902983in}}%
\pgfpathlineto{\pgfqpoint{4.902008in}{2.863547in}}%
\pgfpathlineto{\pgfqpoint{4.911044in}{2.823862in}}%
\pgfpathlineto{\pgfqpoint{4.920080in}{2.783950in}}%
\pgfpathlineto{\pgfqpoint{4.929116in}{2.743834in}}%
\pgfpathlineto{\pgfqpoint{4.938153in}{2.703536in}}%
\pgfpathlineto{\pgfqpoint{4.947189in}{2.663080in}}%
\pgfpathlineto{\pgfqpoint{4.956225in}{2.622489in}}%
\pgfpathlineto{\pgfqpoint{4.965261in}{2.581785in}}%
\pgfpathlineto{\pgfqpoint{4.965261in}{2.579557in}}%
\pgfpathlineto{\pgfqpoint{4.965261in}{2.579557in}}%
\pgfpathlineto{\pgfqpoint{4.956225in}{2.616759in}}%
\pgfpathlineto{\pgfqpoint{4.947189in}{2.653927in}}%
\pgfpathlineto{\pgfqpoint{4.938153in}{2.691040in}}%
\pgfpathlineto{\pgfqpoint{4.929116in}{2.728078in}}%
\pgfpathlineto{\pgfqpoint{4.920080in}{2.765019in}}%
\pgfpathlineto{\pgfqpoint{4.911044in}{2.801843in}}%
\pgfpathlineto{\pgfqpoint{4.902008in}{2.838528in}}%
\pgfpathlineto{\pgfqpoint{4.892971in}{2.875053in}}%
\pgfpathlineto{\pgfqpoint{4.883935in}{2.911399in}}%
\pgfpathlineto{\pgfqpoint{4.874899in}{2.947544in}}%
\pgfpathlineto{\pgfqpoint{4.865863in}{2.983466in}}%
\pgfpathlineto{\pgfqpoint{4.856826in}{3.019146in}}%
\pgfpathlineto{\pgfqpoint{4.847790in}{3.054562in}}%
\pgfpathlineto{\pgfqpoint{4.838754in}{3.089693in}}%
\pgfpathlineto{\pgfqpoint{4.829718in}{3.124519in}}%
\pgfpathlineto{\pgfqpoint{4.820681in}{3.159018in}}%
\pgfpathlineto{\pgfqpoint{4.811645in}{3.193170in}}%
\pgfpathlineto{\pgfqpoint{4.802609in}{3.226954in}}%
\pgfpathlineto{\pgfqpoint{4.793573in}{3.260348in}}%
\pgfpathlineto{\pgfqpoint{4.784536in}{3.293333in}}%
\pgfpathlineto{\pgfqpoint{4.775500in}{3.325887in}}%
\pgfpathlineto{\pgfqpoint{4.766464in}{3.357989in}}%
\pgfpathlineto{\pgfqpoint{4.757428in}{3.389618in}}%
\pgfpathlineto{\pgfqpoint{4.748391in}{3.420754in}}%
\pgfpathlineto{\pgfqpoint{4.739355in}{3.451375in}}%
\pgfpathlineto{\pgfqpoint{4.730319in}{3.481461in}}%
\pgfpathlineto{\pgfqpoint{4.721283in}{3.510991in}}%
\pgfpathlineto{\pgfqpoint{4.712246in}{3.539943in}}%
\pgfpathlineto{\pgfqpoint{4.703210in}{3.568298in}}%
\pgfpathlineto{\pgfqpoint{4.694174in}{3.596033in}}%
\pgfpathlineto{\pgfqpoint{4.685138in}{3.623129in}}%
\pgfpathlineto{\pgfqpoint{4.676101in}{3.649564in}}%
\pgfpathlineto{\pgfqpoint{4.667065in}{3.675318in}}%
\pgfpathlineto{\pgfqpoint{4.658029in}{3.700369in}}%
\pgfpathlineto{\pgfqpoint{4.648993in}{3.724696in}}%
\pgfpathlineto{\pgfqpoint{4.639956in}{3.748280in}}%
\pgfpathlineto{\pgfqpoint{4.630920in}{3.771098in}}%
\pgfpathlineto{\pgfqpoint{4.621884in}{3.793130in}}%
\pgfpathlineto{\pgfqpoint{4.612848in}{3.814355in}}%
\pgfpathlineto{\pgfqpoint{4.603811in}{3.834752in}}%
\pgfpathlineto{\pgfqpoint{4.594775in}{3.854301in}}%
\pgfpathlineto{\pgfqpoint{4.585739in}{3.872980in}}%
\pgfpathlineto{\pgfqpoint{4.576702in}{3.890768in}}%
\pgfpathlineto{\pgfqpoint{4.567666in}{3.907646in}}%
\pgfpathlineto{\pgfqpoint{4.558630in}{3.923590in}}%
\pgfpathlineto{\pgfqpoint{4.549594in}{3.938582in}}%
\pgfpathlineto{\pgfqpoint{4.540557in}{3.952599in}}%
\pgfpathlineto{\pgfqpoint{4.531521in}{3.965622in}}%
\pgfpathlineto{\pgfqpoint{4.522485in}{3.977629in}}%
\pgfpathlineto{\pgfqpoint{4.513449in}{3.988599in}}%
\pgfpathlineto{\pgfqpoint{4.504412in}{3.998511in}}%
\pgfpathlineto{\pgfqpoint{4.495376in}{4.007345in}}%
\pgfpathlineto{\pgfqpoint{4.486340in}{4.015079in}}%
\pgfpathlineto{\pgfqpoint{4.477304in}{4.021693in}}%
\pgfpathlineto{\pgfqpoint{4.468267in}{4.027166in}}%
\pgfpathlineto{\pgfqpoint{4.459231in}{4.031476in}}%
\pgfpathlineto{\pgfqpoint{4.450195in}{4.034604in}}%
\pgfpathlineto{\pgfqpoint{4.441159in}{4.036528in}}%
\pgfpathlineto{\pgfqpoint{4.432122in}{4.037227in}}%
\pgfpathlineto{\pgfqpoint{4.423086in}{4.036680in}}%
\pgfpathlineto{\pgfqpoint{4.414050in}{4.034867in}}%
\pgfpathlineto{\pgfqpoint{4.414050in}{4.034867in}}%
\pgfpathclose%
\pgfusepath{stroke,fill}%
\end{pgfscope}%
\begin{pgfscope}%
\pgfpathrectangle{\pgfqpoint{0.800000in}{0.528000in}}{\pgfqpoint{4.960000in}{3.696000in}}%
\pgfusepath{clip}%
\pgfsetroundcap%
\pgfsetroundjoin%
\pgfsetlinewidth{1.505625pt}%
\definecolor{currentstroke}{rgb}{0.298039,0.447059,0.690196}%
\pgfsetstrokecolor{currentstroke}%
\pgfsetdash{}{0pt}%
\pgfpathmoveto{\pgfqpoint{1.025455in}{0.784658in}}%
\pgfpathlineto{\pgfqpoint{1.061600in}{0.782265in}}%
\pgfpathlineto{\pgfqpoint{1.097745in}{0.779253in}}%
\pgfpathlineto{\pgfqpoint{1.151962in}{0.774034in}}%
\pgfpathlineto{\pgfqpoint{1.242325in}{0.765204in}}%
\pgfpathlineto{\pgfqpoint{1.278470in}{0.762234in}}%
\pgfpathlineto{\pgfqpoint{1.314615in}{0.759900in}}%
\pgfpathlineto{\pgfqpoint{1.341723in}{0.758690in}}%
\pgfpathlineto{\pgfqpoint{1.368832in}{0.758044in}}%
\pgfpathlineto{\pgfqpoint{1.395941in}{0.758048in}}%
\pgfpathlineto{\pgfqpoint{1.423050in}{0.758790in}}%
\pgfpathlineto{\pgfqpoint{1.441122in}{0.759739in}}%
\pgfpathlineto{\pgfqpoint{1.459195in}{0.761081in}}%
\pgfpathlineto{\pgfqpoint{1.477267in}{0.762843in}}%
\pgfpathlineto{\pgfqpoint{1.495340in}{0.765050in}}%
\pgfpathlineto{\pgfqpoint{1.513412in}{0.767730in}}%
\pgfpathlineto{\pgfqpoint{1.531485in}{0.770907in}}%
\pgfpathlineto{\pgfqpoint{1.549557in}{0.774608in}}%
\pgfpathlineto{\pgfqpoint{1.567630in}{0.778859in}}%
\pgfpathlineto{\pgfqpoint{1.585702in}{0.783687in}}%
\pgfpathlineto{\pgfqpoint{1.603775in}{0.789116in}}%
\pgfpathlineto{\pgfqpoint{1.621847in}{0.795174in}}%
\pgfpathlineto{\pgfqpoint{1.639920in}{0.801887in}}%
\pgfpathlineto{\pgfqpoint{1.657992in}{0.809280in}}%
\pgfpathlineto{\pgfqpoint{1.676065in}{0.817380in}}%
\pgfpathlineto{\pgfqpoint{1.694137in}{0.826212in}}%
\pgfpathlineto{\pgfqpoint{1.712210in}{0.835803in}}%
\pgfpathlineto{\pgfqpoint{1.730282in}{0.846179in}}%
\pgfpathlineto{\pgfqpoint{1.748355in}{0.857366in}}%
\pgfpathlineto{\pgfqpoint{1.766427in}{0.869390in}}%
\pgfpathlineto{\pgfqpoint{1.784500in}{0.882278in}}%
\pgfpathlineto{\pgfqpoint{1.802572in}{0.896054in}}%
\pgfpathlineto{\pgfqpoint{1.820645in}{0.910746in}}%
\pgfpathlineto{\pgfqpoint{1.838717in}{0.926380in}}%
\pgfpathlineto{\pgfqpoint{1.856790in}{0.942981in}}%
\pgfpathlineto{\pgfqpoint{1.874862in}{0.960576in}}%
\pgfpathlineto{\pgfqpoint{1.892935in}{0.979190in}}%
\pgfpathlineto{\pgfqpoint{1.911007in}{0.998851in}}%
\pgfpathlineto{\pgfqpoint{1.929080in}{1.019583in}}%
\pgfpathlineto{\pgfqpoint{1.947152in}{1.041413in}}%
\pgfpathlineto{\pgfqpoint{1.965225in}{1.064368in}}%
\pgfpathlineto{\pgfqpoint{1.983298in}{1.088472in}}%
\pgfpathlineto{\pgfqpoint{2.001370in}{1.113753in}}%
\pgfpathlineto{\pgfqpoint{2.019443in}{1.140237in}}%
\pgfpathlineto{\pgfqpoint{2.037515in}{1.167949in}}%
\pgfpathlineto{\pgfqpoint{2.055588in}{1.196915in}}%
\pgfpathlineto{\pgfqpoint{2.073660in}{1.227163in}}%
\pgfpathlineto{\pgfqpoint{2.091733in}{1.258717in}}%
\pgfpathlineto{\pgfqpoint{2.109805in}{1.291604in}}%
\pgfpathlineto{\pgfqpoint{2.127878in}{1.325850in}}%
\pgfpathlineto{\pgfqpoint{2.145950in}{1.361481in}}%
\pgfpathlineto{\pgfqpoint{2.164023in}{1.398519in}}%
\pgfpathlineto{\pgfqpoint{2.182095in}{1.436932in}}%
\pgfpathlineto{\pgfqpoint{2.200168in}{1.476649in}}%
\pgfpathlineto{\pgfqpoint{2.218240in}{1.517598in}}%
\pgfpathlineto{\pgfqpoint{2.236313in}{1.559706in}}%
\pgfpathlineto{\pgfqpoint{2.254385in}{1.602903in}}%
\pgfpathlineto{\pgfqpoint{2.281494in}{1.669579in}}%
\pgfpathlineto{\pgfqpoint{2.308603in}{1.738296in}}%
\pgfpathlineto{\pgfqpoint{2.335711in}{1.808813in}}%
\pgfpathlineto{\pgfqpoint{2.362820in}{1.880884in}}%
\pgfpathlineto{\pgfqpoint{2.398965in}{1.978979in}}%
\pgfpathlineto{\pgfqpoint{2.435110in}{2.078830in}}%
\pgfpathlineto{\pgfqpoint{2.489328in}{2.230638in}}%
\pgfpathlineto{\pgfqpoint{2.579690in}{2.484264in}}%
\pgfpathlineto{\pgfqpoint{2.615835in}{2.584247in}}%
\pgfpathlineto{\pgfqpoint{2.651980in}{2.682527in}}%
\pgfpathlineto{\pgfqpoint{2.679089in}{2.754773in}}%
\pgfpathlineto{\pgfqpoint{2.706198in}{2.825493in}}%
\pgfpathlineto{\pgfqpoint{2.733307in}{2.894443in}}%
\pgfpathlineto{\pgfqpoint{2.760415in}{2.961345in}}%
\pgfpathlineto{\pgfqpoint{2.778488in}{3.004663in}}%
\pgfpathlineto{\pgfqpoint{2.796560in}{3.046851in}}%
\pgfpathlineto{\pgfqpoint{2.814633in}{3.087821in}}%
\pgfpathlineto{\pgfqpoint{2.832705in}{3.127484in}}%
\pgfpathlineto{\pgfqpoint{2.850778in}{3.165753in}}%
\pgfpathlineto{\pgfqpoint{2.868850in}{3.202539in}}%
\pgfpathlineto{\pgfqpoint{2.886923in}{3.237753in}}%
\pgfpathlineto{\pgfqpoint{2.904995in}{3.271308in}}%
\pgfpathlineto{\pgfqpoint{2.923068in}{3.303115in}}%
\pgfpathlineto{\pgfqpoint{2.941140in}{3.333087in}}%
\pgfpathlineto{\pgfqpoint{2.959213in}{3.361134in}}%
\pgfpathlineto{\pgfqpoint{2.968249in}{3.374409in}}%
\pgfpathlineto{\pgfqpoint{2.977285in}{3.387169in}}%
\pgfpathlineto{\pgfqpoint{2.986322in}{3.399405in}}%
\pgfpathlineto{\pgfqpoint{2.995358in}{3.411104in}}%
\pgfpathlineto{\pgfqpoint{3.004394in}{3.422256in}}%
\pgfpathlineto{\pgfqpoint{3.013430in}{3.432849in}}%
\pgfpathlineto{\pgfqpoint{3.022467in}{3.442873in}}%
\pgfpathlineto{\pgfqpoint{3.031503in}{3.452317in}}%
\pgfpathlineto{\pgfqpoint{3.040539in}{3.461170in}}%
\pgfpathlineto{\pgfqpoint{3.049576in}{3.469421in}}%
\pgfpathlineto{\pgfqpoint{3.058612in}{3.477057in}}%
\pgfpathlineto{\pgfqpoint{3.067648in}{3.484070in}}%
\pgfpathlineto{\pgfqpoint{3.076684in}{3.490447in}}%
\pgfpathlineto{\pgfqpoint{3.085721in}{3.496178in}}%
\pgfpathlineto{\pgfqpoint{3.094757in}{3.501251in}}%
\pgfpathlineto{\pgfqpoint{3.103793in}{3.505655in}}%
\pgfpathlineto{\pgfqpoint{3.112829in}{3.509380in}}%
\pgfpathlineto{\pgfqpoint{3.121866in}{3.512414in}}%
\pgfpathlineto{\pgfqpoint{3.130902in}{3.514747in}}%
\pgfpathlineto{\pgfqpoint{3.139938in}{3.516367in}}%
\pgfpathlineto{\pgfqpoint{3.148974in}{3.517263in}}%
\pgfpathlineto{\pgfqpoint{3.158011in}{3.517425in}}%
\pgfpathlineto{\pgfqpoint{3.167047in}{3.516841in}}%
\pgfpathlineto{\pgfqpoint{3.176083in}{3.515500in}}%
\pgfpathlineto{\pgfqpoint{3.185119in}{3.513391in}}%
\pgfpathlineto{\pgfqpoint{3.194156in}{3.510503in}}%
\pgfpathlineto{\pgfqpoint{3.203192in}{3.506826in}}%
\pgfpathlineto{\pgfqpoint{3.212228in}{3.502347in}}%
\pgfpathlineto{\pgfqpoint{3.221264in}{3.497057in}}%
\pgfpathlineto{\pgfqpoint{3.230301in}{3.490944in}}%
\pgfpathlineto{\pgfqpoint{3.239337in}{3.483996in}}%
\pgfpathlineto{\pgfqpoint{3.248373in}{3.476204in}}%
\pgfpathlineto{\pgfqpoint{3.257409in}{3.467556in}}%
\pgfpathlineto{\pgfqpoint{3.266446in}{3.458040in}}%
\pgfpathlineto{\pgfqpoint{3.275482in}{3.447646in}}%
\pgfpathlineto{\pgfqpoint{3.284518in}{3.436365in}}%
\pgfpathlineto{\pgfqpoint{3.293554in}{3.424209in}}%
\pgfpathlineto{\pgfqpoint{3.302591in}{3.411216in}}%
\pgfpathlineto{\pgfqpoint{3.311627in}{3.397424in}}%
\pgfpathlineto{\pgfqpoint{3.320663in}{3.382873in}}%
\pgfpathlineto{\pgfqpoint{3.329699in}{3.367601in}}%
\pgfpathlineto{\pgfqpoint{3.338736in}{3.351645in}}%
\pgfpathlineto{\pgfqpoint{3.347772in}{3.335046in}}%
\pgfpathlineto{\pgfqpoint{3.365844in}{3.300069in}}%
\pgfpathlineto{\pgfqpoint{3.383917in}{3.262977in}}%
\pgfpathlineto{\pgfqpoint{3.401989in}{3.224078in}}%
\pgfpathlineto{\pgfqpoint{3.420062in}{3.183682in}}%
\pgfpathlineto{\pgfqpoint{3.438134in}{3.142095in}}%
\pgfpathlineto{\pgfqpoint{3.465243in}{3.078159in}}%
\pgfpathlineto{\pgfqpoint{3.537533in}{2.905874in}}%
\pgfpathlineto{\pgfqpoint{3.555606in}{2.864066in}}%
\pgfpathlineto{\pgfqpoint{3.573678in}{2.823380in}}%
\pgfpathlineto{\pgfqpoint{3.591751in}{2.784124in}}%
\pgfpathlineto{\pgfqpoint{3.609823in}{2.746606in}}%
\pgfpathlineto{\pgfqpoint{3.627896in}{2.711134in}}%
\pgfpathlineto{\pgfqpoint{3.636932in}{2.694262in}}%
\pgfpathlineto{\pgfqpoint{3.645968in}{2.678017in}}%
\pgfpathlineto{\pgfqpoint{3.655005in}{2.662438in}}%
\pgfpathlineto{\pgfqpoint{3.664041in}{2.647563in}}%
\pgfpathlineto{\pgfqpoint{3.673077in}{2.633431in}}%
\pgfpathlineto{\pgfqpoint{3.682113in}{2.620080in}}%
\pgfpathlineto{\pgfqpoint{3.691150in}{2.607549in}}%
\pgfpathlineto{\pgfqpoint{3.700186in}{2.595877in}}%
\pgfpathlineto{\pgfqpoint{3.709222in}{2.585101in}}%
\pgfpathlineto{\pgfqpoint{3.718258in}{2.575261in}}%
\pgfpathlineto{\pgfqpoint{3.727295in}{2.566395in}}%
\pgfpathlineto{\pgfqpoint{3.736331in}{2.558541in}}%
\pgfpathlineto{\pgfqpoint{3.745367in}{2.551738in}}%
\pgfpathlineto{\pgfqpoint{3.754403in}{2.546025in}}%
\pgfpathlineto{\pgfqpoint{3.763440in}{2.541439in}}%
\pgfpathlineto{\pgfqpoint{3.772476in}{2.538021in}}%
\pgfpathlineto{\pgfqpoint{3.781512in}{2.535807in}}%
\pgfpathlineto{\pgfqpoint{3.790548in}{2.534837in}}%
\pgfpathlineto{\pgfqpoint{3.799585in}{2.535150in}}%
\pgfpathlineto{\pgfqpoint{3.808621in}{2.536782in}}%
\pgfpathlineto{\pgfqpoint{3.817657in}{2.539775in}}%
\pgfpathlineto{\pgfqpoint{3.826693in}{2.544165in}}%
\pgfpathlineto{\pgfqpoint{3.835730in}{2.549991in}}%
\pgfpathlineto{\pgfqpoint{3.844766in}{2.557292in}}%
\pgfpathlineto{\pgfqpoint{3.853802in}{2.566086in}}%
\pgfpathlineto{\pgfqpoint{3.862838in}{2.576334in}}%
\pgfpathlineto{\pgfqpoint{3.871875in}{2.587988in}}%
\pgfpathlineto{\pgfqpoint{3.880911in}{2.601000in}}%
\pgfpathlineto{\pgfqpoint{3.889947in}{2.615322in}}%
\pgfpathlineto{\pgfqpoint{3.898983in}{2.630906in}}%
\pgfpathlineto{\pgfqpoint{3.908020in}{2.647703in}}%
\pgfpathlineto{\pgfqpoint{3.917056in}{2.665666in}}%
\pgfpathlineto{\pgfqpoint{3.926092in}{2.684745in}}%
\pgfpathlineto{\pgfqpoint{3.935128in}{2.704893in}}%
\pgfpathlineto{\pgfqpoint{3.944165in}{2.726062in}}%
\pgfpathlineto{\pgfqpoint{3.953201in}{2.748204in}}%
\pgfpathlineto{\pgfqpoint{3.962237in}{2.771270in}}%
\pgfpathlineto{\pgfqpoint{3.971273in}{2.795212in}}%
\pgfpathlineto{\pgfqpoint{3.989346in}{2.845533in}}%
\pgfpathlineto{\pgfqpoint{4.007418in}{2.898779in}}%
\pgfpathlineto{\pgfqpoint{4.025491in}{2.954567in}}%
\pgfpathlineto{\pgfqpoint{4.043563in}{3.012511in}}%
\pgfpathlineto{\pgfqpoint{4.061636in}{3.072226in}}%
\pgfpathlineto{\pgfqpoint{4.088745in}{3.164273in}}%
\pgfpathlineto{\pgfqpoint{4.124890in}{3.289603in}}%
\pgfpathlineto{\pgfqpoint{4.170071in}{3.446105in}}%
\pgfpathlineto{\pgfqpoint{4.197180in}{3.537610in}}%
\pgfpathlineto{\pgfqpoint{4.215252in}{3.596811in}}%
\pgfpathlineto{\pgfqpoint{4.233325in}{3.654121in}}%
\pgfpathlineto{\pgfqpoint{4.251397in}{3.709154in}}%
\pgfpathlineto{\pgfqpoint{4.269470in}{3.761526in}}%
\pgfpathlineto{\pgfqpoint{4.287542in}{3.810850in}}%
\pgfpathlineto{\pgfqpoint{4.296579in}{3.834248in}}%
\pgfpathlineto{\pgfqpoint{4.305615in}{3.856741in}}%
\pgfpathlineto{\pgfqpoint{4.314651in}{3.878279in}}%
\pgfpathlineto{\pgfqpoint{4.323687in}{3.898814in}}%
\pgfpathlineto{\pgfqpoint{4.332724in}{3.918298in}}%
\pgfpathlineto{\pgfqpoint{4.341760in}{3.936683in}}%
\pgfpathlineto{\pgfqpoint{4.350796in}{3.953921in}}%
\pgfpathlineto{\pgfqpoint{4.359832in}{3.969963in}}%
\pgfpathlineto{\pgfqpoint{4.368869in}{3.984762in}}%
\pgfpathlineto{\pgfqpoint{4.377905in}{3.998269in}}%
\pgfpathlineto{\pgfqpoint{4.386941in}{4.010436in}}%
\pgfpathlineto{\pgfqpoint{4.395977in}{4.021215in}}%
\pgfpathlineto{\pgfqpoint{4.405014in}{4.030558in}}%
\pgfpathlineto{\pgfqpoint{4.414050in}{4.038421in}}%
\pgfpathlineto{\pgfqpoint{4.423086in}{4.044804in}}%
\pgfpathlineto{\pgfqpoint{4.432122in}{4.049731in}}%
\pgfpathlineto{\pgfqpoint{4.441159in}{4.053224in}}%
\pgfpathlineto{\pgfqpoint{4.450195in}{4.055306in}}%
\pgfpathlineto{\pgfqpoint{4.459231in}{4.056000in}}%
\pgfpathlineto{\pgfqpoint{4.468267in}{4.055328in}}%
\pgfpathlineto{\pgfqpoint{4.477304in}{4.053313in}}%
\pgfpathlineto{\pgfqpoint{4.486340in}{4.049978in}}%
\pgfpathlineto{\pgfqpoint{4.495376in}{4.045346in}}%
\pgfpathlineto{\pgfqpoint{4.504412in}{4.039440in}}%
\pgfpathlineto{\pgfqpoint{4.513449in}{4.032281in}}%
\pgfpathlineto{\pgfqpoint{4.522485in}{4.023894in}}%
\pgfpathlineto{\pgfqpoint{4.531521in}{4.014301in}}%
\pgfpathlineto{\pgfqpoint{4.540557in}{4.003524in}}%
\pgfpathlineto{\pgfqpoint{4.549594in}{3.991586in}}%
\pgfpathlineto{\pgfqpoint{4.558630in}{3.978511in}}%
\pgfpathlineto{\pgfqpoint{4.567666in}{3.964321in}}%
\pgfpathlineto{\pgfqpoint{4.576702in}{3.949038in}}%
\pgfpathlineto{\pgfqpoint{4.585739in}{3.932686in}}%
\pgfpathlineto{\pgfqpoint{4.594775in}{3.915287in}}%
\pgfpathlineto{\pgfqpoint{4.603811in}{3.896864in}}%
\pgfpathlineto{\pgfqpoint{4.612848in}{3.877441in}}%
\pgfpathlineto{\pgfqpoint{4.621884in}{3.857038in}}%
\pgfpathlineto{\pgfqpoint{4.630920in}{3.835680in}}%
\pgfpathlineto{\pgfqpoint{4.639956in}{3.813389in}}%
\pgfpathlineto{\pgfqpoint{4.648993in}{3.790189in}}%
\pgfpathlineto{\pgfqpoint{4.658029in}{3.766100in}}%
\pgfpathlineto{\pgfqpoint{4.667065in}{3.741148in}}%
\pgfpathlineto{\pgfqpoint{4.685138in}{3.688740in}}%
\pgfpathlineto{\pgfqpoint{4.703210in}{3.633147in}}%
\pgfpathlineto{\pgfqpoint{4.721283in}{3.574552in}}%
\pgfpathlineto{\pgfqpoint{4.739355in}{3.513137in}}%
\pgfpathlineto{\pgfqpoint{4.757428in}{3.449083in}}%
\pgfpathlineto{\pgfqpoint{4.775500in}{3.382574in}}%
\pgfpathlineto{\pgfqpoint{4.793573in}{3.313790in}}%
\pgfpathlineto{\pgfqpoint{4.811645in}{3.242916in}}%
\pgfpathlineto{\pgfqpoint{4.829718in}{3.170132in}}%
\pgfpathlineto{\pgfqpoint{4.847790in}{3.095621in}}%
\pgfpathlineto{\pgfqpoint{4.874899in}{2.981014in}}%
\pgfpathlineto{\pgfqpoint{4.902008in}{2.863547in}}%
\pgfpathlineto{\pgfqpoint{4.929116in}{2.743834in}}%
\pgfpathlineto{\pgfqpoint{4.965261in}{2.581785in}}%
\pgfpathlineto{\pgfqpoint{5.082733in}{2.052717in}}%
\pgfpathlineto{\pgfqpoint{5.109842in}{1.933746in}}%
\pgfpathlineto{\pgfqpoint{5.136950in}{1.817243in}}%
\pgfpathlineto{\pgfqpoint{5.164059in}{1.703823in}}%
\pgfpathlineto{\pgfqpoint{5.182132in}{1.630227in}}%
\pgfpathlineto{\pgfqpoint{5.200204in}{1.558457in}}%
\pgfpathlineto{\pgfqpoint{5.218277in}{1.488695in}}%
\pgfpathlineto{\pgfqpoint{5.236349in}{1.421124in}}%
\pgfpathlineto{\pgfqpoint{5.254422in}{1.355925in}}%
\pgfpathlineto{\pgfqpoint{5.272494in}{1.293281in}}%
\pgfpathlineto{\pgfqpoint{5.290567in}{1.233374in}}%
\pgfpathlineto{\pgfqpoint{5.308639in}{1.176386in}}%
\pgfpathlineto{\pgfqpoint{5.326712in}{1.122500in}}%
\pgfpathlineto{\pgfqpoint{5.335748in}{1.096777in}}%
\pgfpathlineto{\pgfqpoint{5.344784in}{1.071897in}}%
\pgfpathlineto{\pgfqpoint{5.353820in}{1.047884in}}%
\pgfpathlineto{\pgfqpoint{5.362857in}{1.024760in}}%
\pgfpathlineto{\pgfqpoint{5.371893in}{1.002549in}}%
\pgfpathlineto{\pgfqpoint{5.380929in}{0.981272in}}%
\pgfpathlineto{\pgfqpoint{5.389965in}{0.960952in}}%
\pgfpathlineto{\pgfqpoint{5.399002in}{0.941613in}}%
\pgfpathlineto{\pgfqpoint{5.408038in}{0.923277in}}%
\pgfpathlineto{\pgfqpoint{5.417074in}{0.905967in}}%
\pgfpathlineto{\pgfqpoint{5.426110in}{0.889706in}}%
\pgfpathlineto{\pgfqpoint{5.435147in}{0.874516in}}%
\pgfpathlineto{\pgfqpoint{5.444183in}{0.860421in}}%
\pgfpathlineto{\pgfqpoint{5.453219in}{0.847442in}}%
\pgfpathlineto{\pgfqpoint{5.462255in}{0.835603in}}%
\pgfpathlineto{\pgfqpoint{5.471292in}{0.824927in}}%
\pgfpathlineto{\pgfqpoint{5.480328in}{0.815436in}}%
\pgfpathlineto{\pgfqpoint{5.489364in}{0.807153in}}%
\pgfpathlineto{\pgfqpoint{5.498400in}{0.800101in}}%
\pgfpathlineto{\pgfqpoint{5.507437in}{0.794303in}}%
\pgfpathlineto{\pgfqpoint{5.516473in}{0.789781in}}%
\pgfpathlineto{\pgfqpoint{5.525509in}{0.786559in}}%
\pgfpathlineto{\pgfqpoint{5.534545in}{0.784658in}}%
\pgfpathlineto{\pgfqpoint{5.534545in}{0.784658in}}%
\pgfusepath{stroke}%
\end{pgfscope}%
\begin{pgfscope}%
\pgfpathrectangle{\pgfqpoint{0.800000in}{0.528000in}}{\pgfqpoint{4.960000in}{3.696000in}}%
\pgfusepath{clip}%
\pgfsetroundcap%
\pgfsetroundjoin%
\pgfsetlinewidth{1.505625pt}%
\definecolor{currentstroke}{rgb}{1.000000,0.647059,0.000000}%
\pgfsetstrokecolor{currentstroke}%
\pgfsetdash{}{0pt}%
\pgfpathmoveto{\pgfqpoint{1.025455in}{0.784658in}}%
\pgfpathlineto{\pgfqpoint{1.034491in}{0.798290in}}%
\pgfpathlineto{\pgfqpoint{1.043527in}{0.811224in}}%
\pgfpathlineto{\pgfqpoint{1.052563in}{0.823473in}}%
\pgfpathlineto{\pgfqpoint{1.061600in}{0.835048in}}%
\pgfpathlineto{\pgfqpoint{1.070636in}{0.845964in}}%
\pgfpathlineto{\pgfqpoint{1.079672in}{0.856232in}}%
\pgfpathlineto{\pgfqpoint{1.088708in}{0.865865in}}%
\pgfpathlineto{\pgfqpoint{1.097745in}{0.874876in}}%
\pgfpathlineto{\pgfqpoint{1.106781in}{0.883277in}}%
\pgfpathlineto{\pgfqpoint{1.115817in}{0.891081in}}%
\pgfpathlineto{\pgfqpoint{1.124853in}{0.898301in}}%
\pgfpathlineto{\pgfqpoint{1.133890in}{0.904949in}}%
\pgfpathlineto{\pgfqpoint{1.142926in}{0.911037in}}%
\pgfpathlineto{\pgfqpoint{1.151962in}{0.916579in}}%
\pgfpathlineto{\pgfqpoint{1.160998in}{0.921587in}}%
\pgfpathlineto{\pgfqpoint{1.170035in}{0.926073in}}%
\pgfpathlineto{\pgfqpoint{1.179071in}{0.930051in}}%
\pgfpathlineto{\pgfqpoint{1.188107in}{0.933532in}}%
\pgfpathlineto{\pgfqpoint{1.197143in}{0.936529in}}%
\pgfpathlineto{\pgfqpoint{1.206180in}{0.939056in}}%
\pgfpathlineto{\pgfqpoint{1.215216in}{0.941124in}}%
\pgfpathlineto{\pgfqpoint{1.224252in}{0.942747in}}%
\pgfpathlineto{\pgfqpoint{1.233288in}{0.943936in}}%
\pgfpathlineto{\pgfqpoint{1.242325in}{0.944704in}}%
\pgfpathlineto{\pgfqpoint{1.251361in}{0.945065in}}%
\pgfpathlineto{\pgfqpoint{1.260397in}{0.945031in}}%
\pgfpathlineto{\pgfqpoint{1.269433in}{0.944613in}}%
\pgfpathlineto{\pgfqpoint{1.278470in}{0.943826in}}%
\pgfpathlineto{\pgfqpoint{1.287506in}{0.942680in}}%
\pgfpathlineto{\pgfqpoint{1.296542in}{0.941190in}}%
\pgfpathlineto{\pgfqpoint{1.305578in}{0.939368in}}%
\pgfpathlineto{\pgfqpoint{1.314615in}{0.937226in}}%
\pgfpathlineto{\pgfqpoint{1.323651in}{0.934777in}}%
\pgfpathlineto{\pgfqpoint{1.341723in}{0.929008in}}%
\pgfpathlineto{\pgfqpoint{1.359796in}{0.922162in}}%
\pgfpathlineto{\pgfqpoint{1.377868in}{0.914339in}}%
\pgfpathlineto{\pgfqpoint{1.395941in}{0.905642in}}%
\pgfpathlineto{\pgfqpoint{1.414013in}{0.896169in}}%
\pgfpathlineto{\pgfqpoint{1.432086in}{0.886024in}}%
\pgfpathlineto{\pgfqpoint{1.450158in}{0.875305in}}%
\pgfpathlineto{\pgfqpoint{1.477267in}{0.858374in}}%
\pgfpathlineto{\pgfqpoint{1.504376in}{0.840722in}}%
\pgfpathlineto{\pgfqpoint{1.594739in}{0.781043in}}%
\pgfpathlineto{\pgfqpoint{1.621847in}{0.764199in}}%
\pgfpathlineto{\pgfqpoint{1.639920in}{0.753557in}}%
\pgfpathlineto{\pgfqpoint{1.657992in}{0.743502in}}%
\pgfpathlineto{\pgfqpoint{1.676065in}{0.734135in}}%
\pgfpathlineto{\pgfqpoint{1.694137in}{0.725557in}}%
\pgfpathlineto{\pgfqpoint{1.712210in}{0.717869in}}%
\pgfpathlineto{\pgfqpoint{1.730282in}{0.711172in}}%
\pgfpathlineto{\pgfqpoint{1.748355in}{0.705567in}}%
\pgfpathlineto{\pgfqpoint{1.757391in}{0.703205in}}%
\pgfpathlineto{\pgfqpoint{1.766427in}{0.701154in}}%
\pgfpathlineto{\pgfqpoint{1.775464in}{0.699426in}}%
\pgfpathlineto{\pgfqpoint{1.784500in}{0.698034in}}%
\pgfpathlineto{\pgfqpoint{1.793536in}{0.696991in}}%
\pgfpathlineto{\pgfqpoint{1.802572in}{0.696308in}}%
\pgfpathlineto{\pgfqpoint{1.811609in}{0.696000in}}%
\pgfpathlineto{\pgfqpoint{1.820645in}{0.696078in}}%
\pgfpathlineto{\pgfqpoint{1.829681in}{0.696555in}}%
\pgfpathlineto{\pgfqpoint{1.838717in}{0.697444in}}%
\pgfpathlineto{\pgfqpoint{1.847754in}{0.698756in}}%
\pgfpathlineto{\pgfqpoint{1.856790in}{0.700506in}}%
\pgfpathlineto{\pgfqpoint{1.865826in}{0.702705in}}%
\pgfpathlineto{\pgfqpoint{1.874862in}{0.705366in}}%
\pgfpathlineto{\pgfqpoint{1.883899in}{0.708502in}}%
\pgfpathlineto{\pgfqpoint{1.892935in}{0.712125in}}%
\pgfpathlineto{\pgfqpoint{1.901971in}{0.716248in}}%
\pgfpathlineto{\pgfqpoint{1.911007in}{0.720883in}}%
\pgfpathlineto{\pgfqpoint{1.920044in}{0.726043in}}%
\pgfpathlineto{\pgfqpoint{1.929080in}{0.731741in}}%
\pgfpathlineto{\pgfqpoint{1.938116in}{0.737990in}}%
\pgfpathlineto{\pgfqpoint{1.947152in}{0.744801in}}%
\pgfpathlineto{\pgfqpoint{1.956189in}{0.752188in}}%
\pgfpathlineto{\pgfqpoint{1.965225in}{0.760163in}}%
\pgfpathlineto{\pgfqpoint{1.974261in}{0.768738in}}%
\pgfpathlineto{\pgfqpoint{1.983298in}{0.777927in}}%
\pgfpathlineto{\pgfqpoint{1.992334in}{0.787742in}}%
\pgfpathlineto{\pgfqpoint{2.001370in}{0.798196in}}%
\pgfpathlineto{\pgfqpoint{2.010406in}{0.809300in}}%
\pgfpathlineto{\pgfqpoint{2.019443in}{0.821069in}}%
\pgfpathlineto{\pgfqpoint{2.028479in}{0.833513in}}%
\pgfpathlineto{\pgfqpoint{2.037515in}{0.846647in}}%
\pgfpathlineto{\pgfqpoint{2.046551in}{0.860483in}}%
\pgfpathlineto{\pgfqpoint{2.055588in}{0.875032in}}%
\pgfpathlineto{\pgfqpoint{2.064624in}{0.890309in}}%
\pgfpathlineto{\pgfqpoint{2.073660in}{0.906324in}}%
\pgfpathlineto{\pgfqpoint{2.082696in}{0.923092in}}%
\pgfpathlineto{\pgfqpoint{2.091733in}{0.940625in}}%
\pgfpathlineto{\pgfqpoint{2.100769in}{0.958935in}}%
\pgfpathlineto{\pgfqpoint{2.109805in}{0.978034in}}%
\pgfpathlineto{\pgfqpoint{2.118841in}{0.997936in}}%
\pgfpathlineto{\pgfqpoint{2.127878in}{1.018653in}}%
\pgfpathlineto{\pgfqpoint{2.136914in}{1.040198in}}%
\pgfpathlineto{\pgfqpoint{2.145950in}{1.062583in}}%
\pgfpathlineto{\pgfqpoint{2.154986in}{1.085821in}}%
\pgfpathlineto{\pgfqpoint{2.164023in}{1.109912in}}%
\pgfpathlineto{\pgfqpoint{2.182095in}{1.160555in}}%
\pgfpathlineto{\pgfqpoint{2.200168in}{1.214303in}}%
\pgfpathlineto{\pgfqpoint{2.218240in}{1.270942in}}%
\pgfpathlineto{\pgfqpoint{2.236313in}{1.330261in}}%
\pgfpathlineto{\pgfqpoint{2.254385in}{1.392047in}}%
\pgfpathlineto{\pgfqpoint{2.272458in}{1.456089in}}%
\pgfpathlineto{\pgfqpoint{2.290530in}{1.522173in}}%
\pgfpathlineto{\pgfqpoint{2.308603in}{1.590088in}}%
\pgfpathlineto{\pgfqpoint{2.326675in}{1.659621in}}%
\pgfpathlineto{\pgfqpoint{2.353784in}{1.766489in}}%
\pgfpathlineto{\pgfqpoint{2.380893in}{1.875804in}}%
\pgfpathlineto{\pgfqpoint{2.417038in}{2.024123in}}%
\pgfpathlineto{\pgfqpoint{2.516437in}{2.433989in}}%
\pgfpathlineto{\pgfqpoint{2.543545in}{2.542932in}}%
\pgfpathlineto{\pgfqpoint{2.570654in}{2.649305in}}%
\pgfpathlineto{\pgfqpoint{2.588727in}{2.718439in}}%
\pgfpathlineto{\pgfqpoint{2.606799in}{2.785900in}}%
\pgfpathlineto{\pgfqpoint{2.624872in}{2.851475in}}%
\pgfpathlineto{\pgfqpoint{2.642944in}{2.914952in}}%
\pgfpathlineto{\pgfqpoint{2.661017in}{2.976119in}}%
\pgfpathlineto{\pgfqpoint{2.679089in}{3.034764in}}%
\pgfpathlineto{\pgfqpoint{2.697162in}{3.090673in}}%
\pgfpathlineto{\pgfqpoint{2.715234in}{3.143636in}}%
\pgfpathlineto{\pgfqpoint{2.733307in}{3.193475in}}%
\pgfpathlineto{\pgfqpoint{2.751379in}{3.240186in}}%
\pgfpathlineto{\pgfqpoint{2.760415in}{3.262383in}}%
\pgfpathlineto{\pgfqpoint{2.769452in}{3.283817in}}%
\pgfpathlineto{\pgfqpoint{2.778488in}{3.304492in}}%
\pgfpathlineto{\pgfqpoint{2.787524in}{3.324414in}}%
\pgfpathlineto{\pgfqpoint{2.796560in}{3.343591in}}%
\pgfpathlineto{\pgfqpoint{2.805597in}{3.362027in}}%
\pgfpathlineto{\pgfqpoint{2.814633in}{3.379729in}}%
\pgfpathlineto{\pgfqpoint{2.823669in}{3.396702in}}%
\pgfpathlineto{\pgfqpoint{2.832705in}{3.412953in}}%
\pgfpathlineto{\pgfqpoint{2.841742in}{3.428487in}}%
\pgfpathlineto{\pgfqpoint{2.850778in}{3.443311in}}%
\pgfpathlineto{\pgfqpoint{2.859814in}{3.457430in}}%
\pgfpathlineto{\pgfqpoint{2.868850in}{3.470850in}}%
\pgfpathlineto{\pgfqpoint{2.877887in}{3.483578in}}%
\pgfpathlineto{\pgfqpoint{2.886923in}{3.495619in}}%
\pgfpathlineto{\pgfqpoint{2.895959in}{3.506979in}}%
\pgfpathlineto{\pgfqpoint{2.904995in}{3.517665in}}%
\pgfpathlineto{\pgfqpoint{2.914032in}{3.527681in}}%
\pgfpathlineto{\pgfqpoint{2.923068in}{3.537035in}}%
\pgfpathlineto{\pgfqpoint{2.932104in}{3.545731in}}%
\pgfpathlineto{\pgfqpoint{2.941140in}{3.553777in}}%
\pgfpathlineto{\pgfqpoint{2.950177in}{3.561177in}}%
\pgfpathlineto{\pgfqpoint{2.959213in}{3.567938in}}%
\pgfpathlineto{\pgfqpoint{2.968249in}{3.574067in}}%
\pgfpathlineto{\pgfqpoint{2.977285in}{3.579568in}}%
\pgfpathlineto{\pgfqpoint{2.986322in}{3.584447in}}%
\pgfpathlineto{\pgfqpoint{2.995358in}{3.588712in}}%
\pgfpathlineto{\pgfqpoint{3.004394in}{3.592367in}}%
\pgfpathlineto{\pgfqpoint{3.013430in}{3.595418in}}%
\pgfpathlineto{\pgfqpoint{3.022467in}{3.597873in}}%
\pgfpathlineto{\pgfqpoint{3.031503in}{3.599735in}}%
\pgfpathlineto{\pgfqpoint{3.040539in}{3.601013in}}%
\pgfpathlineto{\pgfqpoint{3.049576in}{3.601710in}}%
\pgfpathlineto{\pgfqpoint{3.058612in}{3.601834in}}%
\pgfpathlineto{\pgfqpoint{3.067648in}{3.601391in}}%
\pgfpathlineto{\pgfqpoint{3.076684in}{3.600385in}}%
\pgfpathlineto{\pgfqpoint{3.085721in}{3.598824in}}%
\pgfpathlineto{\pgfqpoint{3.094757in}{3.596713in}}%
\pgfpathlineto{\pgfqpoint{3.103793in}{3.594058in}}%
\pgfpathlineto{\pgfqpoint{3.112829in}{3.590866in}}%
\pgfpathlineto{\pgfqpoint{3.121866in}{3.587141in}}%
\pgfpathlineto{\pgfqpoint{3.130902in}{3.582890in}}%
\pgfpathlineto{\pgfqpoint{3.139938in}{3.578120in}}%
\pgfpathlineto{\pgfqpoint{3.148974in}{3.572835in}}%
\pgfpathlineto{\pgfqpoint{3.158011in}{3.567042in}}%
\pgfpathlineto{\pgfqpoint{3.167047in}{3.560747in}}%
\pgfpathlineto{\pgfqpoint{3.176083in}{3.553956in}}%
\pgfpathlineto{\pgfqpoint{3.185119in}{3.546674in}}%
\pgfpathlineto{\pgfqpoint{3.194156in}{3.538909in}}%
\pgfpathlineto{\pgfqpoint{3.203192in}{3.530664in}}%
\pgfpathlineto{\pgfqpoint{3.212228in}{3.521948in}}%
\pgfpathlineto{\pgfqpoint{3.221264in}{3.512765in}}%
\pgfpathlineto{\pgfqpoint{3.230301in}{3.503121in}}%
\pgfpathlineto{\pgfqpoint{3.239337in}{3.493023in}}%
\pgfpathlineto{\pgfqpoint{3.248373in}{3.482477in}}%
\pgfpathlineto{\pgfqpoint{3.266446in}{3.460061in}}%
\pgfpathlineto{\pgfqpoint{3.284518in}{3.435924in}}%
\pgfpathlineto{\pgfqpoint{3.302591in}{3.410159in}}%
\pgfpathlineto{\pgfqpoint{3.320663in}{3.382953in}}%
\pgfpathlineto{\pgfqpoint{3.338736in}{3.354503in}}%
\pgfpathlineto{\pgfqpoint{3.356808in}{3.325006in}}%
\pgfpathlineto{\pgfqpoint{3.383917in}{3.279229in}}%
\pgfpathlineto{\pgfqpoint{3.411026in}{3.232204in}}%
\pgfpathlineto{\pgfqpoint{3.483316in}{3.105764in}}%
\pgfpathlineto{\pgfqpoint{3.510424in}{3.059855in}}%
\pgfpathlineto{\pgfqpoint{3.528497in}{3.030239in}}%
\pgfpathlineto{\pgfqpoint{3.546570in}{3.001644in}}%
\pgfpathlineto{\pgfqpoint{3.564642in}{2.974268in}}%
\pgfpathlineto{\pgfqpoint{3.582715in}{2.948308in}}%
\pgfpathlineto{\pgfqpoint{3.600787in}{2.923961in}}%
\pgfpathlineto{\pgfqpoint{3.609823in}{2.912453in}}%
\pgfpathlineto{\pgfqpoint{3.618860in}{2.901423in}}%
\pgfpathlineto{\pgfqpoint{3.627896in}{2.890895in}}%
\pgfpathlineto{\pgfqpoint{3.636932in}{2.880892in}}%
\pgfpathlineto{\pgfqpoint{3.645968in}{2.871441in}}%
\pgfpathlineto{\pgfqpoint{3.655005in}{2.862565in}}%
\pgfpathlineto{\pgfqpoint{3.664041in}{2.854290in}}%
\pgfpathlineto{\pgfqpoint{3.673077in}{2.846639in}}%
\pgfpathlineto{\pgfqpoint{3.682113in}{2.839637in}}%
\pgfpathlineto{\pgfqpoint{3.691150in}{2.833310in}}%
\pgfpathlineto{\pgfqpoint{3.700186in}{2.827682in}}%
\pgfpathlineto{\pgfqpoint{3.709222in}{2.822777in}}%
\pgfpathlineto{\pgfqpoint{3.718258in}{2.818619in}}%
\pgfpathlineto{\pgfqpoint{3.727295in}{2.815235in}}%
\pgfpathlineto{\pgfqpoint{3.736331in}{2.812647in}}%
\pgfpathlineto{\pgfqpoint{3.745367in}{2.810882in}}%
\pgfpathlineto{\pgfqpoint{3.754403in}{2.809962in}}%
\pgfpathlineto{\pgfqpoint{3.763440in}{2.809914in}}%
\pgfpathlineto{\pgfqpoint{3.772476in}{2.810762in}}%
\pgfpathlineto{\pgfqpoint{3.781512in}{2.812530in}}%
\pgfpathlineto{\pgfqpoint{3.790548in}{2.815243in}}%
\pgfpathlineto{\pgfqpoint{3.799585in}{2.818925in}}%
\pgfpathlineto{\pgfqpoint{3.808621in}{2.823602in}}%
\pgfpathlineto{\pgfqpoint{3.817657in}{2.829298in}}%
\pgfpathlineto{\pgfqpoint{3.826693in}{2.836037in}}%
\pgfpathlineto{\pgfqpoint{3.835730in}{2.843844in}}%
\pgfpathlineto{\pgfqpoint{3.844766in}{2.852744in}}%
\pgfpathlineto{\pgfqpoint{3.853802in}{2.862746in}}%
\pgfpathlineto{\pgfqpoint{3.862838in}{2.873819in}}%
\pgfpathlineto{\pgfqpoint{3.871875in}{2.885923in}}%
\pgfpathlineto{\pgfqpoint{3.880911in}{2.899021in}}%
\pgfpathlineto{\pgfqpoint{3.889947in}{2.913074in}}%
\pgfpathlineto{\pgfqpoint{3.898983in}{2.928041in}}%
\pgfpathlineto{\pgfqpoint{3.908020in}{2.943886in}}%
\pgfpathlineto{\pgfqpoint{3.917056in}{2.960569in}}%
\pgfpathlineto{\pgfqpoint{3.926092in}{2.978052in}}%
\pgfpathlineto{\pgfqpoint{3.935128in}{2.996295in}}%
\pgfpathlineto{\pgfqpoint{3.944165in}{3.015259in}}%
\pgfpathlineto{\pgfqpoint{3.962237in}{3.055200in}}%
\pgfpathlineto{\pgfqpoint{3.980310in}{3.097562in}}%
\pgfpathlineto{\pgfqpoint{3.998382in}{3.142037in}}%
\pgfpathlineto{\pgfqpoint{4.016455in}{3.188315in}}%
\pgfpathlineto{\pgfqpoint{4.034527in}{3.236084in}}%
\pgfpathlineto{\pgfqpoint{4.061636in}{3.309858in}}%
\pgfpathlineto{\pgfqpoint{4.097781in}{3.410553in}}%
\pgfpathlineto{\pgfqpoint{4.142962in}{3.536676in}}%
\pgfpathlineto{\pgfqpoint{4.170071in}{3.610626in}}%
\pgfpathlineto{\pgfqpoint{4.188144in}{3.658562in}}%
\pgfpathlineto{\pgfqpoint{4.206216in}{3.705044in}}%
\pgfpathlineto{\pgfqpoint{4.224289in}{3.749763in}}%
\pgfpathlineto{\pgfqpoint{4.242361in}{3.792409in}}%
\pgfpathlineto{\pgfqpoint{4.260434in}{3.832671in}}%
\pgfpathlineto{\pgfqpoint{4.269470in}{3.851812in}}%
\pgfpathlineto{\pgfqpoint{4.278506in}{3.870241in}}%
\pgfpathlineto{\pgfqpoint{4.287542in}{3.887918in}}%
\pgfpathlineto{\pgfqpoint{4.296579in}{3.904806in}}%
\pgfpathlineto{\pgfqpoint{4.305615in}{3.920866in}}%
\pgfpathlineto{\pgfqpoint{4.314651in}{3.936059in}}%
\pgfpathlineto{\pgfqpoint{4.323687in}{3.950345in}}%
\pgfpathlineto{\pgfqpoint{4.332724in}{3.963687in}}%
\pgfpathlineto{\pgfqpoint{4.341760in}{3.976046in}}%
\pgfpathlineto{\pgfqpoint{4.350796in}{3.987383in}}%
\pgfpathlineto{\pgfqpoint{4.359832in}{3.997658in}}%
\pgfpathlineto{\pgfqpoint{4.368869in}{4.006834in}}%
\pgfpathlineto{\pgfqpoint{4.377905in}{4.014872in}}%
\pgfpathlineto{\pgfqpoint{4.386941in}{4.021732in}}%
\pgfpathlineto{\pgfqpoint{4.395977in}{4.027377in}}%
\pgfpathlineto{\pgfqpoint{4.405014in}{4.031766in}}%
\pgfpathlineto{\pgfqpoint{4.414050in}{4.034867in}}%
\pgfpathlineto{\pgfqpoint{4.423086in}{4.036680in}}%
\pgfpathlineto{\pgfqpoint{4.432122in}{4.037227in}}%
\pgfpathlineto{\pgfqpoint{4.441159in}{4.036528in}}%
\pgfpathlineto{\pgfqpoint{4.450195in}{4.034604in}}%
\pgfpathlineto{\pgfqpoint{4.459231in}{4.031476in}}%
\pgfpathlineto{\pgfqpoint{4.468267in}{4.027166in}}%
\pgfpathlineto{\pgfqpoint{4.477304in}{4.021693in}}%
\pgfpathlineto{\pgfqpoint{4.486340in}{4.015079in}}%
\pgfpathlineto{\pgfqpoint{4.495376in}{4.007345in}}%
\pgfpathlineto{\pgfqpoint{4.504412in}{3.998511in}}%
\pgfpathlineto{\pgfqpoint{4.513449in}{3.988599in}}%
\pgfpathlineto{\pgfqpoint{4.522485in}{3.977629in}}%
\pgfpathlineto{\pgfqpoint{4.531521in}{3.965622in}}%
\pgfpathlineto{\pgfqpoint{4.540557in}{3.952599in}}%
\pgfpathlineto{\pgfqpoint{4.549594in}{3.938582in}}%
\pgfpathlineto{\pgfqpoint{4.558630in}{3.923590in}}%
\pgfpathlineto{\pgfqpoint{4.567666in}{3.907646in}}%
\pgfpathlineto{\pgfqpoint{4.576702in}{3.890768in}}%
\pgfpathlineto{\pgfqpoint{4.585739in}{3.872980in}}%
\pgfpathlineto{\pgfqpoint{4.594775in}{3.854301in}}%
\pgfpathlineto{\pgfqpoint{4.603811in}{3.834752in}}%
\pgfpathlineto{\pgfqpoint{4.612848in}{3.814355in}}%
\pgfpathlineto{\pgfqpoint{4.621884in}{3.793130in}}%
\pgfpathlineto{\pgfqpoint{4.630920in}{3.771098in}}%
\pgfpathlineto{\pgfqpoint{4.639956in}{3.748280in}}%
\pgfpathlineto{\pgfqpoint{4.658029in}{3.700369in}}%
\pgfpathlineto{\pgfqpoint{4.676101in}{3.649564in}}%
\pgfpathlineto{\pgfqpoint{4.694174in}{3.596033in}}%
\pgfpathlineto{\pgfqpoint{4.712246in}{3.539943in}}%
\pgfpathlineto{\pgfqpoint{4.730319in}{3.481461in}}%
\pgfpathlineto{\pgfqpoint{4.748391in}{3.420754in}}%
\pgfpathlineto{\pgfqpoint{4.766464in}{3.357989in}}%
\pgfpathlineto{\pgfqpoint{4.784536in}{3.293333in}}%
\pgfpathlineto{\pgfqpoint{4.802609in}{3.226954in}}%
\pgfpathlineto{\pgfqpoint{4.820681in}{3.159018in}}%
\pgfpathlineto{\pgfqpoint{4.847790in}{3.054562in}}%
\pgfpathlineto{\pgfqpoint{4.874899in}{2.947544in}}%
\pgfpathlineto{\pgfqpoint{4.911044in}{2.801843in}}%
\pgfpathlineto{\pgfqpoint{4.956225in}{2.616759in}}%
\pgfpathlineto{\pgfqpoint{5.028515in}{2.319946in}}%
\pgfpathlineto{\pgfqpoint{5.064660in}{2.174067in}}%
\pgfpathlineto{\pgfqpoint{5.091769in}{2.066861in}}%
\pgfpathlineto{\pgfqpoint{5.118878in}{1.962170in}}%
\pgfpathlineto{\pgfqpoint{5.136950in}{1.894051in}}%
\pgfpathlineto{\pgfqpoint{5.155023in}{1.827468in}}%
\pgfpathlineto{\pgfqpoint{5.173095in}{1.762588in}}%
\pgfpathlineto{\pgfqpoint{5.191168in}{1.699577in}}%
\pgfpathlineto{\pgfqpoint{5.209240in}{1.638604in}}%
\pgfpathlineto{\pgfqpoint{5.227313in}{1.579834in}}%
\pgfpathlineto{\pgfqpoint{5.245385in}{1.523436in}}%
\pgfpathlineto{\pgfqpoint{5.263458in}{1.469576in}}%
\pgfpathlineto{\pgfqpoint{5.281530in}{1.418422in}}%
\pgfpathlineto{\pgfqpoint{5.299603in}{1.370140in}}%
\pgfpathlineto{\pgfqpoint{5.308639in}{1.347129in}}%
\pgfpathlineto{\pgfqpoint{5.317675in}{1.324898in}}%
\pgfpathlineto{\pgfqpoint{5.326712in}{1.303469in}}%
\pgfpathlineto{\pgfqpoint{5.335748in}{1.282863in}}%
\pgfpathlineto{\pgfqpoint{5.344784in}{1.263100in}}%
\pgfpathlineto{\pgfqpoint{5.353820in}{1.244202in}}%
\pgfpathlineto{\pgfqpoint{5.362857in}{1.226189in}}%
\pgfpathlineto{\pgfqpoint{5.371893in}{1.209082in}}%
\pgfpathlineto{\pgfqpoint{5.380929in}{1.192902in}}%
\pgfpathlineto{\pgfqpoint{5.389965in}{1.177670in}}%
\pgfpathlineto{\pgfqpoint{5.399002in}{1.163407in}}%
\pgfpathlineto{\pgfqpoint{5.408038in}{1.150134in}}%
\pgfpathlineto{\pgfqpoint{5.417074in}{1.137871in}}%
\pgfpathlineto{\pgfqpoint{5.426110in}{1.126640in}}%
\pgfpathlineto{\pgfqpoint{5.435147in}{1.116461in}}%
\pgfpathlineto{\pgfqpoint{5.444183in}{1.107356in}}%
\pgfpathlineto{\pgfqpoint{5.453219in}{1.099345in}}%
\pgfpathlineto{\pgfqpoint{5.462255in}{1.092449in}}%
\pgfpathlineto{\pgfqpoint{5.471292in}{1.086689in}}%
\pgfpathlineto{\pgfqpoint{5.480328in}{1.082086in}}%
\pgfpathlineto{\pgfqpoint{5.489364in}{1.078660in}}%
\pgfpathlineto{\pgfqpoint{5.498400in}{1.076434in}}%
\pgfpathlineto{\pgfqpoint{5.507437in}{1.075427in}}%
\pgfpathlineto{\pgfqpoint{5.516473in}{1.075660in}}%
\pgfpathlineto{\pgfqpoint{5.525509in}{1.077155in}}%
\pgfpathlineto{\pgfqpoint{5.534545in}{1.079931in}}%
\pgfpathlineto{\pgfqpoint{5.534545in}{1.079931in}}%
\pgfusepath{stroke}%
\end{pgfscope}%
\begin{pgfscope}%
\pgfsetrectcap%
\pgfsetmiterjoin%
\pgfsetlinewidth{1.254687pt}%
\definecolor{currentstroke}{rgb}{1.000000,1.000000,1.000000}%
\pgfsetstrokecolor{currentstroke}%
\pgfsetdash{}{0pt}%
\pgfpathmoveto{\pgfqpoint{0.800000in}{0.528000in}}%
\pgfpathlineto{\pgfqpoint{0.800000in}{4.224000in}}%
\pgfusepath{stroke}%
\end{pgfscope}%
\begin{pgfscope}%
\pgfsetrectcap%
\pgfsetmiterjoin%
\pgfsetlinewidth{1.254687pt}%
\definecolor{currentstroke}{rgb}{1.000000,1.000000,1.000000}%
\pgfsetstrokecolor{currentstroke}%
\pgfsetdash{}{0pt}%
\pgfpathmoveto{\pgfqpoint{5.760000in}{0.528000in}}%
\pgfpathlineto{\pgfqpoint{5.760000in}{4.224000in}}%
\pgfusepath{stroke}%
\end{pgfscope}%
\begin{pgfscope}%
\pgfsetrectcap%
\pgfsetmiterjoin%
\pgfsetlinewidth{1.254687pt}%
\definecolor{currentstroke}{rgb}{1.000000,1.000000,1.000000}%
\pgfsetstrokecolor{currentstroke}%
\pgfsetdash{}{0pt}%
\pgfpathmoveto{\pgfqpoint{0.800000in}{0.528000in}}%
\pgfpathlineto{\pgfqpoint{5.760000in}{0.528000in}}%
\pgfusepath{stroke}%
\end{pgfscope}%
\begin{pgfscope}%
\pgfsetrectcap%
\pgfsetmiterjoin%
\pgfsetlinewidth{1.254687pt}%
\definecolor{currentstroke}{rgb}{1.000000,1.000000,1.000000}%
\pgfsetstrokecolor{currentstroke}%
\pgfsetdash{}{0pt}%
\pgfpathmoveto{\pgfqpoint{0.800000in}{4.224000in}}%
\pgfpathlineto{\pgfqpoint{5.760000in}{4.224000in}}%
\pgfusepath{stroke}%
\end{pgfscope}%
\begin{pgfscope}%
\pgfsetbuttcap%
\pgfsetmiterjoin%
\definecolor{currentfill}{rgb}{0.917647,0.917647,0.949020}%
\pgfsetfillcolor{currentfill}%
\pgfsetfillopacity{0.800000}%
\pgfsetlinewidth{1.003750pt}%
\definecolor{currentstroke}{rgb}{0.800000,0.800000,0.800000}%
\pgfsetstrokecolor{currentstroke}%
\pgfsetstrokeopacity{0.800000}%
\pgfsetdash{}{0pt}%
\pgfpathmoveto{\pgfqpoint{0.906944in}{3.231960in}}%
\pgfpathlineto{\pgfqpoint{2.601868in}{3.231960in}}%
\pgfpathquadraticcurveto{\pgfqpoint{2.632423in}{3.231960in}}{\pgfqpoint{2.632423in}{3.262515in}}%
\pgfpathlineto{\pgfqpoint{2.632423in}{4.117056in}}%
\pgfpathquadraticcurveto{\pgfqpoint{2.632423in}{4.147611in}}{\pgfqpoint{2.601868in}{4.147611in}}%
\pgfpathlineto{\pgfqpoint{0.906944in}{4.147611in}}%
\pgfpathquadraticcurveto{\pgfqpoint{0.876389in}{4.147611in}}{\pgfqpoint{0.876389in}{4.117056in}}%
\pgfpathlineto{\pgfqpoint{0.876389in}{3.262515in}}%
\pgfpathquadraticcurveto{\pgfqpoint{0.876389in}{3.231960in}}{\pgfqpoint{0.906944in}{3.231960in}}%
\pgfpathlineto{\pgfqpoint{0.906944in}{3.231960in}}%
\pgfpathclose%
\pgfusepath{stroke,fill}%
\end{pgfscope}%
\begin{pgfscope}%
\pgfsetroundcap%
\pgfsetroundjoin%
\pgfsetlinewidth{1.505625pt}%
\definecolor{currentstroke}{rgb}{0.298039,0.447059,0.690196}%
\pgfsetstrokecolor{currentstroke}%
\pgfsetdash{}{0pt}%
\pgfpathmoveto{\pgfqpoint{0.937500in}{4.030611in}}%
\pgfpathlineto{\pgfqpoint{1.090278in}{4.030611in}}%
\pgfpathlineto{\pgfqpoint{1.243056in}{4.030611in}}%
\pgfusepath{stroke}%
\end{pgfscope}%
\begin{pgfscope}%
\definecolor{textcolor}{rgb}{0.150000,0.150000,0.150000}%
\pgfsetstrokecolor{textcolor}%
\pgfsetfillcolor{textcolor}%
\pgftext[x=1.365278in,y=3.977139in,left,base]{\color{textcolor}{\sffamily\fontsize{11.000000}{13.200000}\selectfont\catcode`\^=\active\def^{\ifmmode\sp\else\^{}\fi}\catcode`\%=\active\def%{\%}resource demand}}%
\end{pgfscope}%
\begin{pgfscope}%
\pgfsetroundcap%
\pgfsetroundjoin%
\pgfsetlinewidth{1.505625pt}%
\definecolor{currentstroke}{rgb}{1.000000,0.647059,0.000000}%
\pgfsetstrokecolor{currentstroke}%
\pgfsetdash{}{0pt}%
\pgfpathmoveto{\pgfqpoint{0.937500in}{3.814499in}}%
\pgfpathlineto{\pgfqpoint{1.090278in}{3.814499in}}%
\pgfpathlineto{\pgfqpoint{1.243056in}{3.814499in}}%
\pgfusepath{stroke}%
\end{pgfscope}%
\begin{pgfscope}%
\definecolor{textcolor}{rgb}{0.150000,0.150000,0.150000}%
\pgfsetstrokecolor{textcolor}%
\pgfsetfillcolor{textcolor}%
\pgftext[x=1.365278in,y=3.761027in,left,base]{\color{textcolor}{\sffamily\fontsize{11.000000}{13.200000}\selectfont\catcode`\^=\active\def^{\ifmmode\sp\else\^{}\fi}\catcode`\%=\active\def%{\%}resource supply}}%
\end{pgfscope}%
\begin{pgfscope}%
\pgfsetbuttcap%
\pgfsetmiterjoin%
\definecolor{currentfill}{rgb}{0.172549,0.627451,0.172549}%
\pgfsetfillcolor{currentfill}%
\pgfsetfillopacity{0.300000}%
\pgfsetlinewidth{1.003750pt}%
\definecolor{currentstroke}{rgb}{0.172549,0.627451,0.172549}%
\pgfsetstrokecolor{currentstroke}%
\pgfsetstrokeopacity{0.300000}%
\pgfsetdash{}{0pt}%
\pgfpathmoveto{\pgfqpoint{0.937500in}{3.543125in}}%
\pgfpathlineto{\pgfqpoint{1.243056in}{3.543125in}}%
\pgfpathlineto{\pgfqpoint{1.243056in}{3.650069in}}%
\pgfpathlineto{\pgfqpoint{0.937500in}{3.650069in}}%
\pgfpathlineto{\pgfqpoint{0.937500in}{3.543125in}}%
\pgfpathclose%
\pgfusepath{stroke,fill}%
\end{pgfscope}%
\begin{pgfscope}%
\definecolor{textcolor}{rgb}{0.150000,0.150000,0.150000}%
\pgfsetstrokecolor{textcolor}%
\pgfsetfillcolor{textcolor}%
\pgftext[x=1.365278in,y=3.543125in,left,base]{\color{textcolor}{\sffamily\fontsize{11.000000}{13.200000}\selectfont\catcode`\^=\active\def^{\ifmmode\sp\else\^{}\fi}\catcode`\%=\active\def%{\%}overprovisioning}}%
\end{pgfscope}%
\begin{pgfscope}%
\pgfsetbuttcap%
\pgfsetmiterjoin%
\definecolor{currentfill}{rgb}{0.839216,0.152941,0.156863}%
\pgfsetfillcolor{currentfill}%
\pgfsetfillopacity{0.300000}%
\pgfsetlinewidth{1.003750pt}%
\definecolor{currentstroke}{rgb}{0.839216,0.152941,0.156863}%
\pgfsetstrokecolor{currentstroke}%
\pgfsetstrokeopacity{0.300000}%
\pgfsetdash{}{0pt}%
\pgfpathmoveto{\pgfqpoint{0.937500in}{3.325223in}}%
\pgfpathlineto{\pgfqpoint{1.243056in}{3.325223in}}%
\pgfpathlineto{\pgfqpoint{1.243056in}{3.432167in}}%
\pgfpathlineto{\pgfqpoint{0.937500in}{3.432167in}}%
\pgfpathlineto{\pgfqpoint{0.937500in}{3.325223in}}%
\pgfpathclose%
\pgfusepath{stroke,fill}%
\end{pgfscope}%
\begin{pgfscope}%
\definecolor{textcolor}{rgb}{0.150000,0.150000,0.150000}%
\pgfsetstrokecolor{textcolor}%
\pgfsetfillcolor{textcolor}%
\pgftext[x=1.365278in,y=3.325223in,left,base]{\color{textcolor}{\sffamily\fontsize{11.000000}{13.200000}\selectfont\catcode`\^=\active\def^{\ifmmode\sp\else\^{}\fi}\catcode`\%=\active\def%{\%}underprovisioning}}%
\end{pgfscope}%
\end{pgfpicture}%
\makeatother%
\endgroup%

    \caption{Resource demand and supply for a website during a typical day.}
    \label{fig:elasticity-application-scaling}
\end{figure}

Elasticity has multiple properties which are interdependent: resource elasticity, cost elasticity and quality elasticity \cite{dustdarPrinciplesElasticProcesses2011}. These properties are discussed in the following sections.

\subsection{Resource Elasticity}

The resource dimension of elasticity is mistakenly often used synonymously with elasticity. Meanwhile, resource elasticity is defined as the degree to which a system is able to adapt to workload changes by claiming and releasing resources autonomously, such that the resource supply matches the current demand as closely as possible \cite{herbstElasticityCloudComputing2013}. Another way to think of this is ``on the fly'' adaptions to load variations \cite{al-dhuraibiElasticityCloudComputing2018}.

What makes this definition easily mistaken, is that it solely considers the aquired resources and not the consequently incurred costs or changing quality.

% The ability to acquire resources as you need them and release resources when you no longer need them \cite{ElasticityAWSWellArchitected}.

\subsection{Cost Elasticity}

Cost elasticity uses cost as its main factor for elasticity decisions. One of the most popular models that build upon cost elasticity is \textit{utility computing}, also known as the \textit{pay-as-you-go} pricing model.

Amazon Web Services uses this elasticity dimension in their EC2 Spot Instance\footnote{\url{https://aws.amazon.com/ec2/spot/}}. AWS provides its unused compute capacity at a large discount to its customers. But because these capacities are volatile, the prices are not fixed but are provided through a bidding process. The potential customer tells AWS their maximum price they are willing to pay. The customer can then run their instances as long as their bidding price is smaller than AWS's Spot Instance price.

\subsection{Quality Elasticity}

Similiar to the already discussed dimensions, quality elasticity is defined as letting software services adapt their mode of operastion to current operating conditions by providing results of varying output quality \cite{larssonQualityElasticityImprovedResource2019}. This means that when resource supply is low, the output quality also may be low. Likewise, if resource supply is sufficient, the output quality will also be high.

\section{Service Level Agreements and Service Level Objectives}

In order to deliver services up to a certain standard, agreements between the service provider, typically the cloud provider, and the service consumers are made - so called \textit{Service Level Agreements (SLA)} \cite{emeakarohaLowLevelMetrics2010d}. Contained inside these SLAs are \textit{Service Level Objectives (SLO)}, which are a ``commitment to maintain a particular state of the service in a given period'' \cite{kellerWSLAFrameworkSpecifying2003}.

SLOs are measurable values, e.g. an applications CPU usage or memory consumption, that have a specified operating target. In the case that this value is violated the supporting infrastructure of the application has to be either increased or decreased. This process of increasing or decreasing resources is called elasticity, which was further discussed in \cref{sec:elasticity}.

\section{Polaris Framework}
\label{sec:polaris}

The Polaris Framework\footnote{\url{https://polaris-slo-cloud.github.io/polaris-slo-framework/}} is a framework that provides a way to bring high-level SLOs to the cloud. It tries to solve the limitation that modern cloud cloud providers only offer rudimentary support for high-level SLOs and customers often need to manually map them to low-level metrics such as CPU usage or memory consumption \cite{pusztaiSLOScriptNovel2021}.

The authors of this framework introduce the concept of \textit{elasticity strategies}. A elasticity strategy is defined as any sequency of actions that adjust the amount of resources provisioned for a workload, their type or the workload configuration. The workload configuration adjustment is especially noteworthy, because workloads handled by Polaris can be affected in all three elasticity dimensions.

Another unique feature of Polaris is its object model, which allows for orchestrator independence. This is achieved by encapsulating all data that is transmitted to the orchestrator into a \texttt{ApiObject} type.

Decoupling SLOs from elasticity strategies is also a feature that Polaris provides. Tight coupling is a charactaristic that is even observed in industry standard scaling mechanisms such as Kubernetes' Horizontal Pod Autoscaler\footnote{\raggedright\url{https://kubernetes.io/docs/tasks/run-application/horizontal-pod-autoscale/}}. This autoscaler provides a CPU usage SLO which can only trigger horizontal elasticity, thus adding or removing CPU resources. To achieve this decoupling, Polaris utilizes an architecture that is depicted in \cref{fig:polaris-architecture}. This allows the controllers to focus on a single task, for example calculating SLO compliance. These individual components are then mapped using a SLO mapping type.

\begin{figure}
    \centering
    \incfig{polaris-architecture}
    \caption{Architecture of Polaris. Metrics controllers, elasticity strategy controllers and targets are decoupled and mapped using a SLO mapping.}
    \label{fig:polaris-architecture}
\end{figure}

\section{k8ssandra}
\label{sec:k8ssandra}

Cassandra is a popular wide-column store NoSQL database that was initially developed at Facebook and later integradet into the Apache Software Foundation\footnote{\url{https://cassandra.apache.org/_/cassandra-basics.html}\label{fn:cassandra-basics}}. Its main features include being easily horizontally scalable, being fully distributed and its schema-less data approach.

Being distributed means, that Cassandra is comprised of a set of nodes. Each nodes tasks and responsibilities are identical. Data is partitioned using a partition key and is replicated between nodes. How many times data is replicated is determined by the \textit{replication factor} or \(RF\). \(RF = 3\) would therefore mean that each piece of data must exist on 3 nodes.

Distributed data also comes with a certain cost. These drawbacks are formulated in the CAP theorem \cite{foxHarvestYieldScalable1999a}. CAP stands for consistency, availability and partition tolerance and the theorem states that databases which handle data in a distributed way can only provide two of these three guarantees. Cassandra, per default, is an AP database. The upside is, that this agreement is configurable on a per-query basis.

Queries can be made to any node. Cassandra does not have a main node that takes queries, instead any node that a client connects to takes over the role of coordinator for this specific query. This coordinator node then is responsible for querying other nodes for data in other partitions. This also implies that Cassandra uses peer-to-peer communication between its nodes. This architecture is also depicted in \cref{fig:cassandra-architecture}.

\begin{figure}
    \centering
    \incfig{cassandra-architecture}
    \caption{Architecture of a 5 node Cassandra Cluster. Dotted lines represent possible communication paths.}
    \label{fig:cassandra-architecture}
\end{figure}

Another powerful feature, which makes this database particular interesting for this thesis, is its capabilities to scale. If the partition key is chosen wisely and the database is therefore able to distribute data evenly between nodes, then doubling the amount of nodes also doubles the throughput \footref{fn:cassandra-basics}

K8ssandra\footnote{\url{https://k8ssandra.io}} (pronounced: ``Kate'' +  ``Sandra'') is a open-source cloud-native distribution of Cassandra. It includes several tools for providing a data API, backup/restore processes and automated database repairs. It also includes Kubernetes custom resource definitions (CRDs) to be able to easily deploy Cassandra databases.
