\chapter{Background}
\label{ch:background}

Over the last years and decades cloud computing emerged as a paradigm which allows customers to receive compute power in a pay-as-you-go and on-demand manner.

This chapter introduces some terminology and concepts that are used throughout this thesis. First the cloud computing concepts are defined and then the used framework is introduced.

\section{Elasticity in Cloud Computing}
\label{sec:elasticity}

Elasticity is one of the core concepts that solves a big problem of cloud computing: providing limited resources for potentially unlimited use. The solution is to scale workloads up and down as needed, to claim resources when bigger load is experienced and release resources when they are not needed, therefore making them available to other workloads.

The term elasticity in computing is conceptually similiar to the term in physics. Wikipedia, for example, defines elasticity as follows: ``In physics and materials science, elasticity is the ability of a body to resist a distorting influence and to return to its original size and shape when that influence or force is removed. Solid objects will deform when adequate loads are applied to them; if the material is elastic, the object will return to its initial shape and size after removal.''\footnote{\url{https://en.wikipedia.org/wiki/Elasticity_(physics)}}

The formula - which takes a more mathematical approach - of elasticity can be defined as \[ e(Y, X) = \diff{Y}{X} \frac{X}{Y}, \] where \(e(Y, X)\) is the elasticity of \(Y\) with respect to \(X\) \cite{dustdarPrinciplesElasticProcesses2011}.

To illustrate this imagine an application that serves some content to its customers. These customers typically interact with the application during the day. This means that the application experiences significantly less load during the night. Once people wake up in the morning the load rises until it peaks in the afternoon. Then the load falls again when people go to sleep in the evening. Using this example it can be seen in \cref{fig:elasticity-application-no-scaling} that during the night the resources of the application are overprovisioned and during the day the resoures are underprovisioned.

\begin{figure}
    \centering
    %% Creator: Matplotlib, PGF backend
%%
%% To include the figure in your LaTeX document, write
%%   \input{<filename>.pgf}
%%
%% Make sure the required packages are loaded in your preamble
%%   \usepackage{pgf}
%%
%% Also ensure that all the required font packages are loaded; for instance,
%% the lmodern package is sometimes necessary when using math font.
%%   \usepackage{lmodern}
%%
%% Figures using additional raster images can only be included by \input if
%% they are in the same directory as the main LaTeX file. For loading figures
%% from other directories you can use the `import` package
%%   \usepackage{import}
%%
%% and then include the figures with
%%   \import{<path to file>}{<filename>.pgf}
%%
%% Matplotlib used the following preamble
%%   \def\mathdefault#1{#1}
%%   \everymath=\expandafter{\the\everymath\displaystyle}
%%   
%%   \usepackage{fontspec}
%%   \setmainfont{DejaVuSerif.ttf}[Path=\detokenize{/Users/nkratky/private/polaris-elasticity-strategies/test/scripts/.venv/lib/python3.11/site-packages/matplotlib/mpl-data/fonts/ttf/}]
%%   \setsansfont{Arial.ttf}[Path=\detokenize{/System/Library/Fonts/Supplemental/}]
%%   \setmonofont{DejaVuSansMono.ttf}[Path=\detokenize{/Users/nkratky/private/polaris-elasticity-strategies/test/scripts/.venv/lib/python3.11/site-packages/matplotlib/mpl-data/fonts/ttf/}]
%%   \makeatletter\@ifpackageloaded{underscore}{}{\usepackage[strings]{underscore}}\makeatother
%%
\begingroup%
\makeatletter%
\begin{pgfpicture}%
\pgfpathrectangle{\pgfpointorigin}{\pgfqpoint{5.600000in}{4.500000in}}%
\pgfusepath{use as bounding box, clip}%
\begin{pgfscope}%
\pgfsetbuttcap%
\pgfsetmiterjoin%
\definecolor{currentfill}{rgb}{1.000000,1.000000,1.000000}%
\pgfsetfillcolor{currentfill}%
\pgfsetlinewidth{0.000000pt}%
\definecolor{currentstroke}{rgb}{1.000000,1.000000,1.000000}%
\pgfsetstrokecolor{currentstroke}%
\pgfsetdash{}{0pt}%
\pgfpathmoveto{\pgfqpoint{0.000000in}{0.000000in}}%
\pgfpathlineto{\pgfqpoint{5.600000in}{0.000000in}}%
\pgfpathlineto{\pgfqpoint{5.600000in}{4.500000in}}%
\pgfpathlineto{\pgfqpoint{0.000000in}{4.500000in}}%
\pgfpathlineto{\pgfqpoint{0.000000in}{0.000000in}}%
\pgfpathclose%
\pgfusepath{fill}%
\end{pgfscope}%
\begin{pgfscope}%
\pgfsetbuttcap%
\pgfsetmiterjoin%
\definecolor{currentfill}{rgb}{0.917647,0.917647,0.949020}%
\pgfsetfillcolor{currentfill}%
\pgfsetlinewidth{0.000000pt}%
\definecolor{currentstroke}{rgb}{0.000000,0.000000,0.000000}%
\pgfsetstrokecolor{currentstroke}%
\pgfsetstrokeopacity{0.000000}%
\pgfsetdash{}{0pt}%
\pgfpathmoveto{\pgfqpoint{0.700000in}{0.495000in}}%
\pgfpathlineto{\pgfqpoint{5.040000in}{0.495000in}}%
\pgfpathlineto{\pgfqpoint{5.040000in}{3.960000in}}%
\pgfpathlineto{\pgfqpoint{0.700000in}{3.960000in}}%
\pgfpathlineto{\pgfqpoint{0.700000in}{0.495000in}}%
\pgfpathclose%
\pgfusepath{fill}%
\end{pgfscope}%
\begin{pgfscope}%
\pgfpathrectangle{\pgfqpoint{0.700000in}{0.495000in}}{\pgfqpoint{4.340000in}{3.465000in}}%
\pgfusepath{clip}%
\pgfsetroundcap%
\pgfsetroundjoin%
\pgfsetlinewidth{1.003750pt}%
\definecolor{currentstroke}{rgb}{1.000000,1.000000,1.000000}%
\pgfsetstrokecolor{currentstroke}%
\pgfsetdash{}{0pt}%
\pgfpathmoveto{\pgfqpoint{0.897273in}{0.495000in}}%
\pgfpathlineto{\pgfqpoint{0.897273in}{3.960000in}}%
\pgfusepath{stroke}%
\end{pgfscope}%
\begin{pgfscope}%
\definecolor{textcolor}{rgb}{0.150000,0.150000,0.150000}%
\pgfsetstrokecolor{textcolor}%
\pgfsetfillcolor{textcolor}%
\pgftext[x=0.897273in,y=0.363056in,,top]{\color{textcolor}{\sffamily\fontsize{11.000000}{13.200000}\selectfont\catcode`\^=\active\def^{\ifmmode\sp\else\^{}\fi}\catcode`\%=\active\def%{\%}00:00}}%
\end{pgfscope}%
\begin{pgfscope}%
\pgfpathrectangle{\pgfqpoint{0.700000in}{0.495000in}}{\pgfqpoint{4.340000in}{3.465000in}}%
\pgfusepath{clip}%
\pgfsetroundcap%
\pgfsetroundjoin%
\pgfsetlinewidth{1.003750pt}%
\definecolor{currentstroke}{rgb}{1.000000,1.000000,1.000000}%
\pgfsetstrokecolor{currentstroke}%
\pgfsetdash{}{0pt}%
\pgfpathmoveto{\pgfqpoint{1.390455in}{0.495000in}}%
\pgfpathlineto{\pgfqpoint{1.390455in}{3.960000in}}%
\pgfusepath{stroke}%
\end{pgfscope}%
\begin{pgfscope}%
\definecolor{textcolor}{rgb}{0.150000,0.150000,0.150000}%
\pgfsetstrokecolor{textcolor}%
\pgfsetfillcolor{textcolor}%
\pgftext[x=1.390455in,y=0.363056in,,top]{\color{textcolor}{\sffamily\fontsize{11.000000}{13.200000}\selectfont\catcode`\^=\active\def^{\ifmmode\sp\else\^{}\fi}\catcode`\%=\active\def%{\%}03:00}}%
\end{pgfscope}%
\begin{pgfscope}%
\pgfpathrectangle{\pgfqpoint{0.700000in}{0.495000in}}{\pgfqpoint{4.340000in}{3.465000in}}%
\pgfusepath{clip}%
\pgfsetroundcap%
\pgfsetroundjoin%
\pgfsetlinewidth{1.003750pt}%
\definecolor{currentstroke}{rgb}{1.000000,1.000000,1.000000}%
\pgfsetstrokecolor{currentstroke}%
\pgfsetdash{}{0pt}%
\pgfpathmoveto{\pgfqpoint{1.883636in}{0.495000in}}%
\pgfpathlineto{\pgfqpoint{1.883636in}{3.960000in}}%
\pgfusepath{stroke}%
\end{pgfscope}%
\begin{pgfscope}%
\definecolor{textcolor}{rgb}{0.150000,0.150000,0.150000}%
\pgfsetstrokecolor{textcolor}%
\pgfsetfillcolor{textcolor}%
\pgftext[x=1.883636in,y=0.363056in,,top]{\color{textcolor}{\sffamily\fontsize{11.000000}{13.200000}\selectfont\catcode`\^=\active\def^{\ifmmode\sp\else\^{}\fi}\catcode`\%=\active\def%{\%}06:00}}%
\end{pgfscope}%
\begin{pgfscope}%
\pgfpathrectangle{\pgfqpoint{0.700000in}{0.495000in}}{\pgfqpoint{4.340000in}{3.465000in}}%
\pgfusepath{clip}%
\pgfsetroundcap%
\pgfsetroundjoin%
\pgfsetlinewidth{1.003750pt}%
\definecolor{currentstroke}{rgb}{1.000000,1.000000,1.000000}%
\pgfsetstrokecolor{currentstroke}%
\pgfsetdash{}{0pt}%
\pgfpathmoveto{\pgfqpoint{2.376818in}{0.495000in}}%
\pgfpathlineto{\pgfqpoint{2.376818in}{3.960000in}}%
\pgfusepath{stroke}%
\end{pgfscope}%
\begin{pgfscope}%
\definecolor{textcolor}{rgb}{0.150000,0.150000,0.150000}%
\pgfsetstrokecolor{textcolor}%
\pgfsetfillcolor{textcolor}%
\pgftext[x=2.376818in,y=0.363056in,,top]{\color{textcolor}{\sffamily\fontsize{11.000000}{13.200000}\selectfont\catcode`\^=\active\def^{\ifmmode\sp\else\^{}\fi}\catcode`\%=\active\def%{\%}09:00}}%
\end{pgfscope}%
\begin{pgfscope}%
\pgfpathrectangle{\pgfqpoint{0.700000in}{0.495000in}}{\pgfqpoint{4.340000in}{3.465000in}}%
\pgfusepath{clip}%
\pgfsetroundcap%
\pgfsetroundjoin%
\pgfsetlinewidth{1.003750pt}%
\definecolor{currentstroke}{rgb}{1.000000,1.000000,1.000000}%
\pgfsetstrokecolor{currentstroke}%
\pgfsetdash{}{0pt}%
\pgfpathmoveto{\pgfqpoint{2.870000in}{0.495000in}}%
\pgfpathlineto{\pgfqpoint{2.870000in}{3.960000in}}%
\pgfusepath{stroke}%
\end{pgfscope}%
\begin{pgfscope}%
\definecolor{textcolor}{rgb}{0.150000,0.150000,0.150000}%
\pgfsetstrokecolor{textcolor}%
\pgfsetfillcolor{textcolor}%
\pgftext[x=2.870000in,y=0.363056in,,top]{\color{textcolor}{\sffamily\fontsize{11.000000}{13.200000}\selectfont\catcode`\^=\active\def^{\ifmmode\sp\else\^{}\fi}\catcode`\%=\active\def%{\%}12:00}}%
\end{pgfscope}%
\begin{pgfscope}%
\pgfpathrectangle{\pgfqpoint{0.700000in}{0.495000in}}{\pgfqpoint{4.340000in}{3.465000in}}%
\pgfusepath{clip}%
\pgfsetroundcap%
\pgfsetroundjoin%
\pgfsetlinewidth{1.003750pt}%
\definecolor{currentstroke}{rgb}{1.000000,1.000000,1.000000}%
\pgfsetstrokecolor{currentstroke}%
\pgfsetdash{}{0pt}%
\pgfpathmoveto{\pgfqpoint{3.363182in}{0.495000in}}%
\pgfpathlineto{\pgfqpoint{3.363182in}{3.960000in}}%
\pgfusepath{stroke}%
\end{pgfscope}%
\begin{pgfscope}%
\definecolor{textcolor}{rgb}{0.150000,0.150000,0.150000}%
\pgfsetstrokecolor{textcolor}%
\pgfsetfillcolor{textcolor}%
\pgftext[x=3.363182in,y=0.363056in,,top]{\color{textcolor}{\sffamily\fontsize{11.000000}{13.200000}\selectfont\catcode`\^=\active\def^{\ifmmode\sp\else\^{}\fi}\catcode`\%=\active\def%{\%}15:00}}%
\end{pgfscope}%
\begin{pgfscope}%
\pgfpathrectangle{\pgfqpoint{0.700000in}{0.495000in}}{\pgfqpoint{4.340000in}{3.465000in}}%
\pgfusepath{clip}%
\pgfsetroundcap%
\pgfsetroundjoin%
\pgfsetlinewidth{1.003750pt}%
\definecolor{currentstroke}{rgb}{1.000000,1.000000,1.000000}%
\pgfsetstrokecolor{currentstroke}%
\pgfsetdash{}{0pt}%
\pgfpathmoveto{\pgfqpoint{3.856364in}{0.495000in}}%
\pgfpathlineto{\pgfqpoint{3.856364in}{3.960000in}}%
\pgfusepath{stroke}%
\end{pgfscope}%
\begin{pgfscope}%
\definecolor{textcolor}{rgb}{0.150000,0.150000,0.150000}%
\pgfsetstrokecolor{textcolor}%
\pgfsetfillcolor{textcolor}%
\pgftext[x=3.856364in,y=0.363056in,,top]{\color{textcolor}{\sffamily\fontsize{11.000000}{13.200000}\selectfont\catcode`\^=\active\def^{\ifmmode\sp\else\^{}\fi}\catcode`\%=\active\def%{\%}18:00}}%
\end{pgfscope}%
\begin{pgfscope}%
\pgfpathrectangle{\pgfqpoint{0.700000in}{0.495000in}}{\pgfqpoint{4.340000in}{3.465000in}}%
\pgfusepath{clip}%
\pgfsetroundcap%
\pgfsetroundjoin%
\pgfsetlinewidth{1.003750pt}%
\definecolor{currentstroke}{rgb}{1.000000,1.000000,1.000000}%
\pgfsetstrokecolor{currentstroke}%
\pgfsetdash{}{0pt}%
\pgfpathmoveto{\pgfqpoint{4.349545in}{0.495000in}}%
\pgfpathlineto{\pgfqpoint{4.349545in}{3.960000in}}%
\pgfusepath{stroke}%
\end{pgfscope}%
\begin{pgfscope}%
\definecolor{textcolor}{rgb}{0.150000,0.150000,0.150000}%
\pgfsetstrokecolor{textcolor}%
\pgfsetfillcolor{textcolor}%
\pgftext[x=4.349545in,y=0.363056in,,top]{\color{textcolor}{\sffamily\fontsize{11.000000}{13.200000}\selectfont\catcode`\^=\active\def^{\ifmmode\sp\else\^{}\fi}\catcode`\%=\active\def%{\%}21:00}}%
\end{pgfscope}%
\begin{pgfscope}%
\pgfpathrectangle{\pgfqpoint{0.700000in}{0.495000in}}{\pgfqpoint{4.340000in}{3.465000in}}%
\pgfusepath{clip}%
\pgfsetroundcap%
\pgfsetroundjoin%
\pgfsetlinewidth{1.003750pt}%
\definecolor{currentstroke}{rgb}{1.000000,1.000000,1.000000}%
\pgfsetstrokecolor{currentstroke}%
\pgfsetdash{}{0pt}%
\pgfpathmoveto{\pgfqpoint{4.842727in}{0.495000in}}%
\pgfpathlineto{\pgfqpoint{4.842727in}{3.960000in}}%
\pgfusepath{stroke}%
\end{pgfscope}%
\begin{pgfscope}%
\definecolor{textcolor}{rgb}{0.150000,0.150000,0.150000}%
\pgfsetstrokecolor{textcolor}%
\pgfsetfillcolor{textcolor}%
\pgftext[x=4.842727in,y=0.363056in,,top]{\color{textcolor}{\sffamily\fontsize{11.000000}{13.200000}\selectfont\catcode`\^=\active\def^{\ifmmode\sp\else\^{}\fi}\catcode`\%=\active\def%{\%}24:00}}%
\end{pgfscope}%
\begin{pgfscope}%
\definecolor{textcolor}{rgb}{0.150000,0.150000,0.150000}%
\pgfsetstrokecolor{textcolor}%
\pgfsetfillcolor{textcolor}%
\pgftext[x=2.870000in,y=0.167777in,,top]{\color{textcolor}{\sffamily\fontsize{12.000000}{14.400000}\selectfont\catcode`\^=\active\def^{\ifmmode\sp\else\^{}\fi}\catcode`\%=\active\def%{\%}Time}}%
\end{pgfscope}%
\begin{pgfscope}%
\pgfpathrectangle{\pgfqpoint{0.700000in}{0.495000in}}{\pgfqpoint{4.340000in}{3.465000in}}%
\pgfusepath{clip}%
\pgfsetroundcap%
\pgfsetroundjoin%
\pgfsetlinewidth{1.003750pt}%
\definecolor{currentstroke}{rgb}{1.000000,1.000000,1.000000}%
\pgfsetstrokecolor{currentstroke}%
\pgfsetdash{}{0pt}%
\pgfpathmoveto{\pgfqpoint{0.700000in}{0.677992in}}%
\pgfpathlineto{\pgfqpoint{5.040000in}{0.677992in}}%
\pgfusepath{stroke}%
\end{pgfscope}%
\begin{pgfscope}%
\pgfpathrectangle{\pgfqpoint{0.700000in}{0.495000in}}{\pgfqpoint{4.340000in}{3.465000in}}%
\pgfusepath{clip}%
\pgfsetroundcap%
\pgfsetroundjoin%
\pgfsetlinewidth{1.003750pt}%
\definecolor{currentstroke}{rgb}{1.000000,1.000000,1.000000}%
\pgfsetstrokecolor{currentstroke}%
\pgfsetdash{}{0pt}%
\pgfpathmoveto{\pgfqpoint{0.700000in}{1.242032in}}%
\pgfpathlineto{\pgfqpoint{5.040000in}{1.242032in}}%
\pgfusepath{stroke}%
\end{pgfscope}%
\begin{pgfscope}%
\pgfpathrectangle{\pgfqpoint{0.700000in}{0.495000in}}{\pgfqpoint{4.340000in}{3.465000in}}%
\pgfusepath{clip}%
\pgfsetroundcap%
\pgfsetroundjoin%
\pgfsetlinewidth{1.003750pt}%
\definecolor{currentstroke}{rgb}{1.000000,1.000000,1.000000}%
\pgfsetstrokecolor{currentstroke}%
\pgfsetdash{}{0pt}%
\pgfpathmoveto{\pgfqpoint{0.700000in}{1.806071in}}%
\pgfpathlineto{\pgfqpoint{5.040000in}{1.806071in}}%
\pgfusepath{stroke}%
\end{pgfscope}%
\begin{pgfscope}%
\pgfpathrectangle{\pgfqpoint{0.700000in}{0.495000in}}{\pgfqpoint{4.340000in}{3.465000in}}%
\pgfusepath{clip}%
\pgfsetroundcap%
\pgfsetroundjoin%
\pgfsetlinewidth{1.003750pt}%
\definecolor{currentstroke}{rgb}{1.000000,1.000000,1.000000}%
\pgfsetstrokecolor{currentstroke}%
\pgfsetdash{}{0pt}%
\pgfpathmoveto{\pgfqpoint{0.700000in}{2.370111in}}%
\pgfpathlineto{\pgfqpoint{5.040000in}{2.370111in}}%
\pgfusepath{stroke}%
\end{pgfscope}%
\begin{pgfscope}%
\pgfpathrectangle{\pgfqpoint{0.700000in}{0.495000in}}{\pgfqpoint{4.340000in}{3.465000in}}%
\pgfusepath{clip}%
\pgfsetroundcap%
\pgfsetroundjoin%
\pgfsetlinewidth{1.003750pt}%
\definecolor{currentstroke}{rgb}{1.000000,1.000000,1.000000}%
\pgfsetstrokecolor{currentstroke}%
\pgfsetdash{}{0pt}%
\pgfpathmoveto{\pgfqpoint{0.700000in}{2.934151in}}%
\pgfpathlineto{\pgfqpoint{5.040000in}{2.934151in}}%
\pgfusepath{stroke}%
\end{pgfscope}%
\begin{pgfscope}%
\pgfpathrectangle{\pgfqpoint{0.700000in}{0.495000in}}{\pgfqpoint{4.340000in}{3.465000in}}%
\pgfusepath{clip}%
\pgfsetroundcap%
\pgfsetroundjoin%
\pgfsetlinewidth{1.003750pt}%
\definecolor{currentstroke}{rgb}{1.000000,1.000000,1.000000}%
\pgfsetstrokecolor{currentstroke}%
\pgfsetdash{}{0pt}%
\pgfpathmoveto{\pgfqpoint{0.700000in}{3.498191in}}%
\pgfpathlineto{\pgfqpoint{5.040000in}{3.498191in}}%
\pgfusepath{stroke}%
\end{pgfscope}%
\begin{pgfscope}%
\definecolor{textcolor}{rgb}{0.150000,0.150000,0.150000}%
\pgfsetstrokecolor{textcolor}%
\pgfsetfillcolor{textcolor}%
\pgftext[x=0.512500in,y=2.227500in,,bottom,rotate=90.000000]{\color{textcolor}{\sffamily\fontsize{12.000000}{14.400000}\selectfont\catcode`\^=\active\def^{\ifmmode\sp\else\^{}\fi}\catcode`\%=\active\def%{\%}Resources}}%
\end{pgfscope}%
\begin{pgfscope}%
\pgfpathrectangle{\pgfqpoint{0.700000in}{0.495000in}}{\pgfqpoint{4.340000in}{3.465000in}}%
\pgfusepath{clip}%
\pgfsetbuttcap%
\pgfsetroundjoin%
\definecolor{currentfill}{rgb}{0.172549,0.627451,0.172549}%
\pgfsetfillcolor{currentfill}%
\pgfsetfillopacity{0.300000}%
\pgfsetlinewidth{1.003750pt}%
\definecolor{currentstroke}{rgb}{0.172549,0.627451,0.172549}%
\pgfsetstrokecolor{currentstroke}%
\pgfsetstrokeopacity{0.300000}%
\pgfsetdash{}{0pt}%
\pgfpathmoveto{\pgfqpoint{0.897273in}{2.088091in}}%
\pgfpathlineto{\pgfqpoint{0.897273in}{0.677992in}}%
\pgfpathlineto{\pgfqpoint{0.905179in}{0.677487in}}%
\pgfpathlineto{\pgfqpoint{0.913086in}{0.676935in}}%
\pgfpathlineto{\pgfqpoint{0.920993in}{0.676340in}}%
\pgfpathlineto{\pgfqpoint{0.928900in}{0.675706in}}%
\pgfpathlineto{\pgfqpoint{0.936806in}{0.675034in}}%
\pgfpathlineto{\pgfqpoint{0.944713in}{0.674329in}}%
\pgfpathlineto{\pgfqpoint{0.952620in}{0.673593in}}%
\pgfpathlineto{\pgfqpoint{0.960527in}{0.672830in}}%
\pgfpathlineto{\pgfqpoint{0.968433in}{0.672042in}}%
\pgfpathlineto{\pgfqpoint{0.976340in}{0.671233in}}%
\pgfpathlineto{\pgfqpoint{0.984247in}{0.670405in}}%
\pgfpathlineto{\pgfqpoint{0.992153in}{0.669563in}}%
\pgfpathlineto{\pgfqpoint{1.000060in}{0.668708in}}%
\pgfpathlineto{\pgfqpoint{1.007967in}{0.667845in}}%
\pgfpathlineto{\pgfqpoint{1.015874in}{0.666976in}}%
\pgfpathlineto{\pgfqpoint{1.023780in}{0.666104in}}%
\pgfpathlineto{\pgfqpoint{1.031687in}{0.665233in}}%
\pgfpathlineto{\pgfqpoint{1.039594in}{0.664366in}}%
\pgfpathlineto{\pgfqpoint{1.047500in}{0.663505in}}%
\pgfpathlineto{\pgfqpoint{1.055407in}{0.662654in}}%
\pgfpathlineto{\pgfqpoint{1.063314in}{0.661816in}}%
\pgfpathlineto{\pgfqpoint{1.071221in}{0.660994in}}%
\pgfpathlineto{\pgfqpoint{1.079127in}{0.660191in}}%
\pgfpathlineto{\pgfqpoint{1.087034in}{0.659411in}}%
\pgfpathlineto{\pgfqpoint{1.094941in}{0.658656in}}%
\pgfpathlineto{\pgfqpoint{1.102848in}{0.657930in}}%
\pgfpathlineto{\pgfqpoint{1.110754in}{0.657235in}}%
\pgfpathlineto{\pgfqpoint{1.118661in}{0.656575in}}%
\pgfpathlineto{\pgfqpoint{1.126568in}{0.655953in}}%
\pgfpathlineto{\pgfqpoint{1.134474in}{0.655371in}}%
\pgfpathlineto{\pgfqpoint{1.142381in}{0.654834in}}%
\pgfpathlineto{\pgfqpoint{1.150288in}{0.654345in}}%
\pgfpathlineto{\pgfqpoint{1.158195in}{0.653905in}}%
\pgfpathlineto{\pgfqpoint{1.166101in}{0.653519in}}%
\pgfpathlineto{\pgfqpoint{1.174008in}{0.653190in}}%
\pgfpathlineto{\pgfqpoint{1.181915in}{0.652920in}}%
\pgfpathlineto{\pgfqpoint{1.189821in}{0.652713in}}%
\pgfpathlineto{\pgfqpoint{1.197728in}{0.652572in}}%
\pgfpathlineto{\pgfqpoint{1.205635in}{0.652500in}}%
\pgfpathlineto{\pgfqpoint{1.213542in}{0.652500in}}%
\pgfpathlineto{\pgfqpoint{1.221448in}{0.652576in}}%
\pgfpathlineto{\pgfqpoint{1.229355in}{0.652729in}}%
\pgfpathlineto{\pgfqpoint{1.237262in}{0.652965in}}%
\pgfpathlineto{\pgfqpoint{1.245169in}{0.653285in}}%
\pgfpathlineto{\pgfqpoint{1.253075in}{0.653692in}}%
\pgfpathlineto{\pgfqpoint{1.260982in}{0.654191in}}%
\pgfpathlineto{\pgfqpoint{1.268889in}{0.654783in}}%
\pgfpathlineto{\pgfqpoint{1.276795in}{0.655473in}}%
\pgfpathlineto{\pgfqpoint{1.284702in}{0.656263in}}%
\pgfpathlineto{\pgfqpoint{1.292609in}{0.657156in}}%
\pgfpathlineto{\pgfqpoint{1.300516in}{0.658155in}}%
\pgfpathlineto{\pgfqpoint{1.308422in}{0.659264in}}%
\pgfpathlineto{\pgfqpoint{1.316329in}{0.660486in}}%
\pgfpathlineto{\pgfqpoint{1.324236in}{0.661823in}}%
\pgfpathlineto{\pgfqpoint{1.332142in}{0.663280in}}%
\pgfpathlineto{\pgfqpoint{1.340049in}{0.664858in}}%
\pgfpathlineto{\pgfqpoint{1.347956in}{0.666561in}}%
\pgfpathlineto{\pgfqpoint{1.355863in}{0.668393in}}%
\pgfpathlineto{\pgfqpoint{1.363769in}{0.670356in}}%
\pgfpathlineto{\pgfqpoint{1.371676in}{0.672453in}}%
\pgfpathlineto{\pgfqpoint{1.379583in}{0.674688in}}%
\pgfpathlineto{\pgfqpoint{1.387490in}{0.677064in}}%
\pgfpathlineto{\pgfqpoint{1.395396in}{0.679584in}}%
\pgfpathlineto{\pgfqpoint{1.403303in}{0.682250in}}%
\pgfpathlineto{\pgfqpoint{1.411210in}{0.685067in}}%
\pgfpathlineto{\pgfqpoint{1.419116in}{0.688036in}}%
\pgfpathlineto{\pgfqpoint{1.427023in}{0.691162in}}%
\pgfpathlineto{\pgfqpoint{1.434930in}{0.694447in}}%
\pgfpathlineto{\pgfqpoint{1.442837in}{0.697895in}}%
\pgfpathlineto{\pgfqpoint{1.450743in}{0.701508in}}%
\pgfpathlineto{\pgfqpoint{1.458650in}{0.705291in}}%
\pgfpathlineto{\pgfqpoint{1.466557in}{0.709245in}}%
\pgfpathlineto{\pgfqpoint{1.474463in}{0.713373in}}%
\pgfpathlineto{\pgfqpoint{1.482370in}{0.717680in}}%
\pgfpathlineto{\pgfqpoint{1.490277in}{0.722169in}}%
\pgfpathlineto{\pgfqpoint{1.498184in}{0.726841in}}%
\pgfpathlineto{\pgfqpoint{1.506090in}{0.731701in}}%
\pgfpathlineto{\pgfqpoint{1.513997in}{0.736751in}}%
\pgfpathlineto{\pgfqpoint{1.521904in}{0.741996in}}%
\pgfpathlineto{\pgfqpoint{1.529811in}{0.747436in}}%
\pgfpathlineto{\pgfqpoint{1.537717in}{0.753077in}}%
\pgfpathlineto{\pgfqpoint{1.545624in}{0.758921in}}%
\pgfpathlineto{\pgfqpoint{1.553531in}{0.764971in}}%
\pgfpathlineto{\pgfqpoint{1.561437in}{0.771230in}}%
\pgfpathlineto{\pgfqpoint{1.569344in}{0.777701in}}%
\pgfpathlineto{\pgfqpoint{1.577251in}{0.784388in}}%
\pgfpathlineto{\pgfqpoint{1.585158in}{0.791293in}}%
\pgfpathlineto{\pgfqpoint{1.593064in}{0.798421in}}%
\pgfpathlineto{\pgfqpoint{1.600971in}{0.805773in}}%
\pgfpathlineto{\pgfqpoint{1.608878in}{0.813352in}}%
\pgfpathlineto{\pgfqpoint{1.616784in}{0.821163in}}%
\pgfpathlineto{\pgfqpoint{1.624691in}{0.829208in}}%
\pgfpathlineto{\pgfqpoint{1.632598in}{0.837491in}}%
\pgfpathlineto{\pgfqpoint{1.640505in}{0.846013in}}%
\pgfpathlineto{\pgfqpoint{1.648411in}{0.854780in}}%
\pgfpathlineto{\pgfqpoint{1.656318in}{0.863792in}}%
\pgfpathlineto{\pgfqpoint{1.664225in}{0.873055in}}%
\pgfpathlineto{\pgfqpoint{1.672132in}{0.882570in}}%
\pgfpathlineto{\pgfqpoint{1.680038in}{0.892342in}}%
\pgfpathlineto{\pgfqpoint{1.687945in}{0.902372in}}%
\pgfpathlineto{\pgfqpoint{1.695852in}{0.912665in}}%
\pgfpathlineto{\pgfqpoint{1.703758in}{0.923222in}}%
\pgfpathlineto{\pgfqpoint{1.711665in}{0.934049in}}%
\pgfpathlineto{\pgfqpoint{1.719572in}{0.945147in}}%
\pgfpathlineto{\pgfqpoint{1.727479in}{0.956519in}}%
\pgfpathlineto{\pgfqpoint{1.735385in}{0.968169in}}%
\pgfpathlineto{\pgfqpoint{1.743292in}{0.980101in}}%
\pgfpathlineto{\pgfqpoint{1.751199in}{0.992316in}}%
\pgfpathlineto{\pgfqpoint{1.759105in}{1.004818in}}%
\pgfpathlineto{\pgfqpoint{1.767012in}{1.017610in}}%
\pgfpathlineto{\pgfqpoint{1.774919in}{1.030696in}}%
\pgfpathlineto{\pgfqpoint{1.782826in}{1.044079in}}%
\pgfpathlineto{\pgfqpoint{1.790732in}{1.057760in}}%
\pgfpathlineto{\pgfqpoint{1.798639in}{1.071745in}}%
\pgfpathlineto{\pgfqpoint{1.806546in}{1.086035in}}%
\pgfpathlineto{\pgfqpoint{1.814453in}{1.100635in}}%
\pgfpathlineto{\pgfqpoint{1.822359in}{1.115546in}}%
\pgfpathlineto{\pgfqpoint{1.830266in}{1.130772in}}%
\pgfpathlineto{\pgfqpoint{1.838173in}{1.146317in}}%
\pgfpathlineto{\pgfqpoint{1.846079in}{1.162183in}}%
\pgfpathlineto{\pgfqpoint{1.853986in}{1.178374in}}%
\pgfpathlineto{\pgfqpoint{1.861893in}{1.194892in}}%
\pgfpathlineto{\pgfqpoint{1.869800in}{1.211741in}}%
\pgfpathlineto{\pgfqpoint{1.877706in}{1.228924in}}%
\pgfpathlineto{\pgfqpoint{1.885613in}{1.246443in}}%
\pgfpathlineto{\pgfqpoint{1.893520in}{1.264299in}}%
\pgfpathlineto{\pgfqpoint{1.901426in}{1.282484in}}%
\pgfpathlineto{\pgfqpoint{1.909333in}{1.300988in}}%
\pgfpathlineto{\pgfqpoint{1.917240in}{1.319804in}}%
\pgfpathlineto{\pgfqpoint{1.925147in}{1.338923in}}%
\pgfpathlineto{\pgfqpoint{1.933053in}{1.358335in}}%
\pgfpathlineto{\pgfqpoint{1.940960in}{1.378034in}}%
\pgfpathlineto{\pgfqpoint{1.948867in}{1.398009in}}%
\pgfpathlineto{\pgfqpoint{1.956774in}{1.418252in}}%
\pgfpathlineto{\pgfqpoint{1.964680in}{1.438755in}}%
\pgfpathlineto{\pgfqpoint{1.972587in}{1.459509in}}%
\pgfpathlineto{\pgfqpoint{1.980494in}{1.480506in}}%
\pgfpathlineto{\pgfqpoint{1.988400in}{1.501737in}}%
\pgfpathlineto{\pgfqpoint{1.996307in}{1.523193in}}%
\pgfpathlineto{\pgfqpoint{2.004214in}{1.544865in}}%
\pgfpathlineto{\pgfqpoint{2.012121in}{1.566746in}}%
\pgfpathlineto{\pgfqpoint{2.020027in}{1.588826in}}%
\pgfpathlineto{\pgfqpoint{2.027934in}{1.611097in}}%
\pgfpathlineto{\pgfqpoint{2.035841in}{1.633550in}}%
\pgfpathlineto{\pgfqpoint{2.043747in}{1.656177in}}%
\pgfpathlineto{\pgfqpoint{2.051654in}{1.678969in}}%
\pgfpathlineto{\pgfqpoint{2.059561in}{1.701918in}}%
\pgfpathlineto{\pgfqpoint{2.067468in}{1.725014in}}%
\pgfpathlineto{\pgfqpoint{2.075374in}{1.748250in}}%
\pgfpathlineto{\pgfqpoint{2.083281in}{1.771616in}}%
\pgfpathlineto{\pgfqpoint{2.091188in}{1.795104in}}%
\pgfpathlineto{\pgfqpoint{2.099095in}{1.818706in}}%
\pgfpathlineto{\pgfqpoint{2.107001in}{1.842412in}}%
\pgfpathlineto{\pgfqpoint{2.114908in}{1.866215in}}%
\pgfpathlineto{\pgfqpoint{2.122815in}{1.890105in}}%
\pgfpathlineto{\pgfqpoint{2.130721in}{1.914075in}}%
\pgfpathlineto{\pgfqpoint{2.138628in}{1.938115in}}%
\pgfpathlineto{\pgfqpoint{2.146535in}{1.962216in}}%
\pgfpathlineto{\pgfqpoint{2.154442in}{1.986371in}}%
\pgfpathlineto{\pgfqpoint{2.162348in}{2.010571in}}%
\pgfpathlineto{\pgfqpoint{2.170255in}{2.034806in}}%
\pgfpathlineto{\pgfqpoint{2.178162in}{2.059069in}}%
\pgfpathlineto{\pgfqpoint{2.186069in}{2.083350in}}%
\pgfpathlineto{\pgfqpoint{2.186069in}{2.088091in}}%
\pgfpathlineto{\pgfqpoint{2.186069in}{2.088091in}}%
\pgfpathlineto{\pgfqpoint{2.178162in}{2.088091in}}%
\pgfpathlineto{\pgfqpoint{2.170255in}{2.088091in}}%
\pgfpathlineto{\pgfqpoint{2.162348in}{2.088091in}}%
\pgfpathlineto{\pgfqpoint{2.154442in}{2.088091in}}%
\pgfpathlineto{\pgfqpoint{2.146535in}{2.088091in}}%
\pgfpathlineto{\pgfqpoint{2.138628in}{2.088091in}}%
\pgfpathlineto{\pgfqpoint{2.130721in}{2.088091in}}%
\pgfpathlineto{\pgfqpoint{2.122815in}{2.088091in}}%
\pgfpathlineto{\pgfqpoint{2.114908in}{2.088091in}}%
\pgfpathlineto{\pgfqpoint{2.107001in}{2.088091in}}%
\pgfpathlineto{\pgfqpoint{2.099095in}{2.088091in}}%
\pgfpathlineto{\pgfqpoint{2.091188in}{2.088091in}}%
\pgfpathlineto{\pgfqpoint{2.083281in}{2.088091in}}%
\pgfpathlineto{\pgfqpoint{2.075374in}{2.088091in}}%
\pgfpathlineto{\pgfqpoint{2.067468in}{2.088091in}}%
\pgfpathlineto{\pgfqpoint{2.059561in}{2.088091in}}%
\pgfpathlineto{\pgfqpoint{2.051654in}{2.088091in}}%
\pgfpathlineto{\pgfqpoint{2.043747in}{2.088091in}}%
\pgfpathlineto{\pgfqpoint{2.035841in}{2.088091in}}%
\pgfpathlineto{\pgfqpoint{2.027934in}{2.088091in}}%
\pgfpathlineto{\pgfqpoint{2.020027in}{2.088091in}}%
\pgfpathlineto{\pgfqpoint{2.012121in}{2.088091in}}%
\pgfpathlineto{\pgfqpoint{2.004214in}{2.088091in}}%
\pgfpathlineto{\pgfqpoint{1.996307in}{2.088091in}}%
\pgfpathlineto{\pgfqpoint{1.988400in}{2.088091in}}%
\pgfpathlineto{\pgfqpoint{1.980494in}{2.088091in}}%
\pgfpathlineto{\pgfqpoint{1.972587in}{2.088091in}}%
\pgfpathlineto{\pgfqpoint{1.964680in}{2.088091in}}%
\pgfpathlineto{\pgfqpoint{1.956774in}{2.088091in}}%
\pgfpathlineto{\pgfqpoint{1.948867in}{2.088091in}}%
\pgfpathlineto{\pgfqpoint{1.940960in}{2.088091in}}%
\pgfpathlineto{\pgfqpoint{1.933053in}{2.088091in}}%
\pgfpathlineto{\pgfqpoint{1.925147in}{2.088091in}}%
\pgfpathlineto{\pgfqpoint{1.917240in}{2.088091in}}%
\pgfpathlineto{\pgfqpoint{1.909333in}{2.088091in}}%
\pgfpathlineto{\pgfqpoint{1.901426in}{2.088091in}}%
\pgfpathlineto{\pgfqpoint{1.893520in}{2.088091in}}%
\pgfpathlineto{\pgfqpoint{1.885613in}{2.088091in}}%
\pgfpathlineto{\pgfqpoint{1.877706in}{2.088091in}}%
\pgfpathlineto{\pgfqpoint{1.869800in}{2.088091in}}%
\pgfpathlineto{\pgfqpoint{1.861893in}{2.088091in}}%
\pgfpathlineto{\pgfqpoint{1.853986in}{2.088091in}}%
\pgfpathlineto{\pgfqpoint{1.846079in}{2.088091in}}%
\pgfpathlineto{\pgfqpoint{1.838173in}{2.088091in}}%
\pgfpathlineto{\pgfqpoint{1.830266in}{2.088091in}}%
\pgfpathlineto{\pgfqpoint{1.822359in}{2.088091in}}%
\pgfpathlineto{\pgfqpoint{1.814453in}{2.088091in}}%
\pgfpathlineto{\pgfqpoint{1.806546in}{2.088091in}}%
\pgfpathlineto{\pgfqpoint{1.798639in}{2.088091in}}%
\pgfpathlineto{\pgfqpoint{1.790732in}{2.088091in}}%
\pgfpathlineto{\pgfqpoint{1.782826in}{2.088091in}}%
\pgfpathlineto{\pgfqpoint{1.774919in}{2.088091in}}%
\pgfpathlineto{\pgfqpoint{1.767012in}{2.088091in}}%
\pgfpathlineto{\pgfqpoint{1.759105in}{2.088091in}}%
\pgfpathlineto{\pgfqpoint{1.751199in}{2.088091in}}%
\pgfpathlineto{\pgfqpoint{1.743292in}{2.088091in}}%
\pgfpathlineto{\pgfqpoint{1.735385in}{2.088091in}}%
\pgfpathlineto{\pgfqpoint{1.727479in}{2.088091in}}%
\pgfpathlineto{\pgfqpoint{1.719572in}{2.088091in}}%
\pgfpathlineto{\pgfqpoint{1.711665in}{2.088091in}}%
\pgfpathlineto{\pgfqpoint{1.703758in}{2.088091in}}%
\pgfpathlineto{\pgfqpoint{1.695852in}{2.088091in}}%
\pgfpathlineto{\pgfqpoint{1.687945in}{2.088091in}}%
\pgfpathlineto{\pgfqpoint{1.680038in}{2.088091in}}%
\pgfpathlineto{\pgfqpoint{1.672132in}{2.088091in}}%
\pgfpathlineto{\pgfqpoint{1.664225in}{2.088091in}}%
\pgfpathlineto{\pgfqpoint{1.656318in}{2.088091in}}%
\pgfpathlineto{\pgfqpoint{1.648411in}{2.088091in}}%
\pgfpathlineto{\pgfqpoint{1.640505in}{2.088091in}}%
\pgfpathlineto{\pgfqpoint{1.632598in}{2.088091in}}%
\pgfpathlineto{\pgfqpoint{1.624691in}{2.088091in}}%
\pgfpathlineto{\pgfqpoint{1.616784in}{2.088091in}}%
\pgfpathlineto{\pgfqpoint{1.608878in}{2.088091in}}%
\pgfpathlineto{\pgfqpoint{1.600971in}{2.088091in}}%
\pgfpathlineto{\pgfqpoint{1.593064in}{2.088091in}}%
\pgfpathlineto{\pgfqpoint{1.585158in}{2.088091in}}%
\pgfpathlineto{\pgfqpoint{1.577251in}{2.088091in}}%
\pgfpathlineto{\pgfqpoint{1.569344in}{2.088091in}}%
\pgfpathlineto{\pgfqpoint{1.561437in}{2.088091in}}%
\pgfpathlineto{\pgfqpoint{1.553531in}{2.088091in}}%
\pgfpathlineto{\pgfqpoint{1.545624in}{2.088091in}}%
\pgfpathlineto{\pgfqpoint{1.537717in}{2.088091in}}%
\pgfpathlineto{\pgfqpoint{1.529811in}{2.088091in}}%
\pgfpathlineto{\pgfqpoint{1.521904in}{2.088091in}}%
\pgfpathlineto{\pgfqpoint{1.513997in}{2.088091in}}%
\pgfpathlineto{\pgfqpoint{1.506090in}{2.088091in}}%
\pgfpathlineto{\pgfqpoint{1.498184in}{2.088091in}}%
\pgfpathlineto{\pgfqpoint{1.490277in}{2.088091in}}%
\pgfpathlineto{\pgfqpoint{1.482370in}{2.088091in}}%
\pgfpathlineto{\pgfqpoint{1.474463in}{2.088091in}}%
\pgfpathlineto{\pgfqpoint{1.466557in}{2.088091in}}%
\pgfpathlineto{\pgfqpoint{1.458650in}{2.088091in}}%
\pgfpathlineto{\pgfqpoint{1.450743in}{2.088091in}}%
\pgfpathlineto{\pgfqpoint{1.442837in}{2.088091in}}%
\pgfpathlineto{\pgfqpoint{1.434930in}{2.088091in}}%
\pgfpathlineto{\pgfqpoint{1.427023in}{2.088091in}}%
\pgfpathlineto{\pgfqpoint{1.419116in}{2.088091in}}%
\pgfpathlineto{\pgfqpoint{1.411210in}{2.088091in}}%
\pgfpathlineto{\pgfqpoint{1.403303in}{2.088091in}}%
\pgfpathlineto{\pgfqpoint{1.395396in}{2.088091in}}%
\pgfpathlineto{\pgfqpoint{1.387490in}{2.088091in}}%
\pgfpathlineto{\pgfqpoint{1.379583in}{2.088091in}}%
\pgfpathlineto{\pgfqpoint{1.371676in}{2.088091in}}%
\pgfpathlineto{\pgfqpoint{1.363769in}{2.088091in}}%
\pgfpathlineto{\pgfqpoint{1.355863in}{2.088091in}}%
\pgfpathlineto{\pgfqpoint{1.347956in}{2.088091in}}%
\pgfpathlineto{\pgfqpoint{1.340049in}{2.088091in}}%
\pgfpathlineto{\pgfqpoint{1.332142in}{2.088091in}}%
\pgfpathlineto{\pgfqpoint{1.324236in}{2.088091in}}%
\pgfpathlineto{\pgfqpoint{1.316329in}{2.088091in}}%
\pgfpathlineto{\pgfqpoint{1.308422in}{2.088091in}}%
\pgfpathlineto{\pgfqpoint{1.300516in}{2.088091in}}%
\pgfpathlineto{\pgfqpoint{1.292609in}{2.088091in}}%
\pgfpathlineto{\pgfqpoint{1.284702in}{2.088091in}}%
\pgfpathlineto{\pgfqpoint{1.276795in}{2.088091in}}%
\pgfpathlineto{\pgfqpoint{1.268889in}{2.088091in}}%
\pgfpathlineto{\pgfqpoint{1.260982in}{2.088091in}}%
\pgfpathlineto{\pgfqpoint{1.253075in}{2.088091in}}%
\pgfpathlineto{\pgfqpoint{1.245169in}{2.088091in}}%
\pgfpathlineto{\pgfqpoint{1.237262in}{2.088091in}}%
\pgfpathlineto{\pgfqpoint{1.229355in}{2.088091in}}%
\pgfpathlineto{\pgfqpoint{1.221448in}{2.088091in}}%
\pgfpathlineto{\pgfqpoint{1.213542in}{2.088091in}}%
\pgfpathlineto{\pgfqpoint{1.205635in}{2.088091in}}%
\pgfpathlineto{\pgfqpoint{1.197728in}{2.088091in}}%
\pgfpathlineto{\pgfqpoint{1.189821in}{2.088091in}}%
\pgfpathlineto{\pgfqpoint{1.181915in}{2.088091in}}%
\pgfpathlineto{\pgfqpoint{1.174008in}{2.088091in}}%
\pgfpathlineto{\pgfqpoint{1.166101in}{2.088091in}}%
\pgfpathlineto{\pgfqpoint{1.158195in}{2.088091in}}%
\pgfpathlineto{\pgfqpoint{1.150288in}{2.088091in}}%
\pgfpathlineto{\pgfqpoint{1.142381in}{2.088091in}}%
\pgfpathlineto{\pgfqpoint{1.134474in}{2.088091in}}%
\pgfpathlineto{\pgfqpoint{1.126568in}{2.088091in}}%
\pgfpathlineto{\pgfqpoint{1.118661in}{2.088091in}}%
\pgfpathlineto{\pgfqpoint{1.110754in}{2.088091in}}%
\pgfpathlineto{\pgfqpoint{1.102848in}{2.088091in}}%
\pgfpathlineto{\pgfqpoint{1.094941in}{2.088091in}}%
\pgfpathlineto{\pgfqpoint{1.087034in}{2.088091in}}%
\pgfpathlineto{\pgfqpoint{1.079127in}{2.088091in}}%
\pgfpathlineto{\pgfqpoint{1.071221in}{2.088091in}}%
\pgfpathlineto{\pgfqpoint{1.063314in}{2.088091in}}%
\pgfpathlineto{\pgfqpoint{1.055407in}{2.088091in}}%
\pgfpathlineto{\pgfqpoint{1.047500in}{2.088091in}}%
\pgfpathlineto{\pgfqpoint{1.039594in}{2.088091in}}%
\pgfpathlineto{\pgfqpoint{1.031687in}{2.088091in}}%
\pgfpathlineto{\pgfqpoint{1.023780in}{2.088091in}}%
\pgfpathlineto{\pgfqpoint{1.015874in}{2.088091in}}%
\pgfpathlineto{\pgfqpoint{1.007967in}{2.088091in}}%
\pgfpathlineto{\pgfqpoint{1.000060in}{2.088091in}}%
\pgfpathlineto{\pgfqpoint{0.992153in}{2.088091in}}%
\pgfpathlineto{\pgfqpoint{0.984247in}{2.088091in}}%
\pgfpathlineto{\pgfqpoint{0.976340in}{2.088091in}}%
\pgfpathlineto{\pgfqpoint{0.968433in}{2.088091in}}%
\pgfpathlineto{\pgfqpoint{0.960527in}{2.088091in}}%
\pgfpathlineto{\pgfqpoint{0.952620in}{2.088091in}}%
\pgfpathlineto{\pgfqpoint{0.944713in}{2.088091in}}%
\pgfpathlineto{\pgfqpoint{0.936806in}{2.088091in}}%
\pgfpathlineto{\pgfqpoint{0.928900in}{2.088091in}}%
\pgfpathlineto{\pgfqpoint{0.920993in}{2.088091in}}%
\pgfpathlineto{\pgfqpoint{0.913086in}{2.088091in}}%
\pgfpathlineto{\pgfqpoint{0.905179in}{2.088091in}}%
\pgfpathlineto{\pgfqpoint{0.897273in}{2.088091in}}%
\pgfpathlineto{\pgfqpoint{0.897273in}{2.088091in}}%
\pgfpathclose%
\pgfusepath{stroke,fill}%
\end{pgfscope}%
\begin{pgfscope}%
\pgfpathrectangle{\pgfqpoint{0.700000in}{0.495000in}}{\pgfqpoint{4.340000in}{3.465000in}}%
\pgfusepath{clip}%
\pgfsetbuttcap%
\pgfsetroundjoin%
\definecolor{currentfill}{rgb}{0.172549,0.627451,0.172549}%
\pgfsetfillcolor{currentfill}%
\pgfsetfillopacity{0.300000}%
\pgfsetlinewidth{1.003750pt}%
\definecolor{currentstroke}{rgb}{0.172549,0.627451,0.172549}%
\pgfsetstrokecolor{currentstroke}%
\pgfsetstrokeopacity{0.300000}%
\pgfsetdash{}{0pt}%
\pgfpathmoveto{\pgfqpoint{4.407858in}{2.088091in}}%
\pgfpathlineto{\pgfqpoint{4.407858in}{2.082164in}}%
\pgfpathlineto{\pgfqpoint{4.415764in}{2.043295in}}%
\pgfpathlineto{\pgfqpoint{4.423671in}{2.004536in}}%
\pgfpathlineto{\pgfqpoint{4.431578in}{1.965908in}}%
\pgfpathlineto{\pgfqpoint{4.439484in}{1.927434in}}%
\pgfpathlineto{\pgfqpoint{4.447391in}{1.889134in}}%
\pgfpathlineto{\pgfqpoint{4.455298in}{1.851031in}}%
\pgfpathlineto{\pgfqpoint{4.463205in}{1.813147in}}%
\pgfpathlineto{\pgfqpoint{4.471111in}{1.775503in}}%
\pgfpathlineto{\pgfqpoint{4.479018in}{1.738120in}}%
\pgfpathlineto{\pgfqpoint{4.486925in}{1.701022in}}%
\pgfpathlineto{\pgfqpoint{4.494831in}{1.664229in}}%
\pgfpathlineto{\pgfqpoint{4.502738in}{1.627763in}}%
\pgfpathlineto{\pgfqpoint{4.510645in}{1.591646in}}%
\pgfpathlineto{\pgfqpoint{4.518552in}{1.555900in}}%
\pgfpathlineto{\pgfqpoint{4.526458in}{1.520547in}}%
\pgfpathlineto{\pgfqpoint{4.534365in}{1.485608in}}%
\pgfpathlineto{\pgfqpoint{4.542272in}{1.451104in}}%
\pgfpathlineto{\pgfqpoint{4.550179in}{1.417059in}}%
\pgfpathlineto{\pgfqpoint{4.558085in}{1.383493in}}%
\pgfpathlineto{\pgfqpoint{4.565992in}{1.350428in}}%
\pgfpathlineto{\pgfqpoint{4.573899in}{1.317887in}}%
\pgfpathlineto{\pgfqpoint{4.581805in}{1.285890in}}%
\pgfpathlineto{\pgfqpoint{4.589712in}{1.254459in}}%
\pgfpathlineto{\pgfqpoint{4.597619in}{1.223617in}}%
\pgfpathlineto{\pgfqpoint{4.605526in}{1.193385in}}%
\pgfpathlineto{\pgfqpoint{4.613432in}{1.163785in}}%
\pgfpathlineto{\pgfqpoint{4.621339in}{1.134838in}}%
\pgfpathlineto{\pgfqpoint{4.629246in}{1.106567in}}%
\pgfpathlineto{\pgfqpoint{4.637152in}{1.078993in}}%
\pgfpathlineto{\pgfqpoint{4.645059in}{1.052137in}}%
\pgfpathlineto{\pgfqpoint{4.652966in}{1.026022in}}%
\pgfpathlineto{\pgfqpoint{4.660873in}{1.000669in}}%
\pgfpathlineto{\pgfqpoint{4.668779in}{0.976101in}}%
\pgfpathlineto{\pgfqpoint{4.676686in}{0.952338in}}%
\pgfpathlineto{\pgfqpoint{4.684593in}{0.929403in}}%
\pgfpathlineto{\pgfqpoint{4.692500in}{0.907317in}}%
\pgfpathlineto{\pgfqpoint{4.700406in}{0.886102in}}%
\pgfpathlineto{\pgfqpoint{4.708313in}{0.865780in}}%
\pgfpathlineto{\pgfqpoint{4.716220in}{0.846373in}}%
\pgfpathlineto{\pgfqpoint{4.724126in}{0.827902in}}%
\pgfpathlineto{\pgfqpoint{4.732033in}{0.810389in}}%
\pgfpathlineto{\pgfqpoint{4.739940in}{0.793856in}}%
\pgfpathlineto{\pgfqpoint{4.747847in}{0.778325in}}%
\pgfpathlineto{\pgfqpoint{4.755753in}{0.763817in}}%
\pgfpathlineto{\pgfqpoint{4.763660in}{0.750354in}}%
\pgfpathlineto{\pgfqpoint{4.771567in}{0.737958in}}%
\pgfpathlineto{\pgfqpoint{4.779473in}{0.726650in}}%
\pgfpathlineto{\pgfqpoint{4.787380in}{0.716453in}}%
\pgfpathlineto{\pgfqpoint{4.795287in}{0.707388in}}%
\pgfpathlineto{\pgfqpoint{4.803194in}{0.699477in}}%
\pgfpathlineto{\pgfqpoint{4.811100in}{0.692742in}}%
\pgfpathlineto{\pgfqpoint{4.819007in}{0.687204in}}%
\pgfpathlineto{\pgfqpoint{4.826914in}{0.682885in}}%
\pgfpathlineto{\pgfqpoint{4.834821in}{0.679807in}}%
\pgfpathlineto{\pgfqpoint{4.842727in}{0.677992in}}%
\pgfpathlineto{\pgfqpoint{4.842727in}{2.088091in}}%
\pgfpathlineto{\pgfqpoint{4.842727in}{2.088091in}}%
\pgfpathlineto{\pgfqpoint{4.834821in}{2.088091in}}%
\pgfpathlineto{\pgfqpoint{4.826914in}{2.088091in}}%
\pgfpathlineto{\pgfqpoint{4.819007in}{2.088091in}}%
\pgfpathlineto{\pgfqpoint{4.811100in}{2.088091in}}%
\pgfpathlineto{\pgfqpoint{4.803194in}{2.088091in}}%
\pgfpathlineto{\pgfqpoint{4.795287in}{2.088091in}}%
\pgfpathlineto{\pgfqpoint{4.787380in}{2.088091in}}%
\pgfpathlineto{\pgfqpoint{4.779473in}{2.088091in}}%
\pgfpathlineto{\pgfqpoint{4.771567in}{2.088091in}}%
\pgfpathlineto{\pgfqpoint{4.763660in}{2.088091in}}%
\pgfpathlineto{\pgfqpoint{4.755753in}{2.088091in}}%
\pgfpathlineto{\pgfqpoint{4.747847in}{2.088091in}}%
\pgfpathlineto{\pgfqpoint{4.739940in}{2.088091in}}%
\pgfpathlineto{\pgfqpoint{4.732033in}{2.088091in}}%
\pgfpathlineto{\pgfqpoint{4.724126in}{2.088091in}}%
\pgfpathlineto{\pgfqpoint{4.716220in}{2.088091in}}%
\pgfpathlineto{\pgfqpoint{4.708313in}{2.088091in}}%
\pgfpathlineto{\pgfqpoint{4.700406in}{2.088091in}}%
\pgfpathlineto{\pgfqpoint{4.692500in}{2.088091in}}%
\pgfpathlineto{\pgfqpoint{4.684593in}{2.088091in}}%
\pgfpathlineto{\pgfqpoint{4.676686in}{2.088091in}}%
\pgfpathlineto{\pgfqpoint{4.668779in}{2.088091in}}%
\pgfpathlineto{\pgfqpoint{4.660873in}{2.088091in}}%
\pgfpathlineto{\pgfqpoint{4.652966in}{2.088091in}}%
\pgfpathlineto{\pgfqpoint{4.645059in}{2.088091in}}%
\pgfpathlineto{\pgfqpoint{4.637152in}{2.088091in}}%
\pgfpathlineto{\pgfqpoint{4.629246in}{2.088091in}}%
\pgfpathlineto{\pgfqpoint{4.621339in}{2.088091in}}%
\pgfpathlineto{\pgfqpoint{4.613432in}{2.088091in}}%
\pgfpathlineto{\pgfqpoint{4.605526in}{2.088091in}}%
\pgfpathlineto{\pgfqpoint{4.597619in}{2.088091in}}%
\pgfpathlineto{\pgfqpoint{4.589712in}{2.088091in}}%
\pgfpathlineto{\pgfqpoint{4.581805in}{2.088091in}}%
\pgfpathlineto{\pgfqpoint{4.573899in}{2.088091in}}%
\pgfpathlineto{\pgfqpoint{4.565992in}{2.088091in}}%
\pgfpathlineto{\pgfqpoint{4.558085in}{2.088091in}}%
\pgfpathlineto{\pgfqpoint{4.550179in}{2.088091in}}%
\pgfpathlineto{\pgfqpoint{4.542272in}{2.088091in}}%
\pgfpathlineto{\pgfqpoint{4.534365in}{2.088091in}}%
\pgfpathlineto{\pgfqpoint{4.526458in}{2.088091in}}%
\pgfpathlineto{\pgfqpoint{4.518552in}{2.088091in}}%
\pgfpathlineto{\pgfqpoint{4.510645in}{2.088091in}}%
\pgfpathlineto{\pgfqpoint{4.502738in}{2.088091in}}%
\pgfpathlineto{\pgfqpoint{4.494831in}{2.088091in}}%
\pgfpathlineto{\pgfqpoint{4.486925in}{2.088091in}}%
\pgfpathlineto{\pgfqpoint{4.479018in}{2.088091in}}%
\pgfpathlineto{\pgfqpoint{4.471111in}{2.088091in}}%
\pgfpathlineto{\pgfqpoint{4.463205in}{2.088091in}}%
\pgfpathlineto{\pgfqpoint{4.455298in}{2.088091in}}%
\pgfpathlineto{\pgfqpoint{4.447391in}{2.088091in}}%
\pgfpathlineto{\pgfqpoint{4.439484in}{2.088091in}}%
\pgfpathlineto{\pgfqpoint{4.431578in}{2.088091in}}%
\pgfpathlineto{\pgfqpoint{4.423671in}{2.088091in}}%
\pgfpathlineto{\pgfqpoint{4.415764in}{2.088091in}}%
\pgfpathlineto{\pgfqpoint{4.407858in}{2.088091in}}%
\pgfpathlineto{\pgfqpoint{4.407858in}{2.088091in}}%
\pgfpathclose%
\pgfusepath{stroke,fill}%
\end{pgfscope}%
\begin{pgfscope}%
\pgfpathrectangle{\pgfqpoint{0.700000in}{0.495000in}}{\pgfqpoint{4.340000in}{3.465000in}}%
\pgfusepath{clip}%
\pgfsetbuttcap%
\pgfsetroundjoin%
\definecolor{currentfill}{rgb}{0.839216,0.152941,0.156863}%
\pgfsetfillcolor{currentfill}%
\pgfsetfillopacity{0.300000}%
\pgfsetlinewidth{1.003750pt}%
\definecolor{currentstroke}{rgb}{0.839216,0.152941,0.156863}%
\pgfsetstrokecolor{currentstroke}%
\pgfsetstrokeopacity{0.300000}%
\pgfsetdash{}{0pt}%
\pgfsys@defobject{currentmarker}{\pgfqpoint{2.193975in}{2.088091in}}{\pgfqpoint{4.399951in}{3.802500in}}{%
\pgfpathmoveto{\pgfqpoint{2.193975in}{2.088091in}}%
\pgfpathlineto{\pgfqpoint{2.193975in}{2.107642in}}%
\pgfpathlineto{\pgfqpoint{2.201882in}{2.131936in}}%
\pgfpathlineto{\pgfqpoint{2.209789in}{2.156222in}}%
\pgfpathlineto{\pgfqpoint{2.217695in}{2.180493in}}%
\pgfpathlineto{\pgfqpoint{2.225602in}{2.204739in}}%
\pgfpathlineto{\pgfqpoint{2.233509in}{2.228953in}}%
\pgfpathlineto{\pgfqpoint{2.241416in}{2.253125in}}%
\pgfpathlineto{\pgfqpoint{2.249322in}{2.277247in}}%
\pgfpathlineto{\pgfqpoint{2.257229in}{2.301311in}}%
\pgfpathlineto{\pgfqpoint{2.265136in}{2.325307in}}%
\pgfpathlineto{\pgfqpoint{2.273042in}{2.349228in}}%
\pgfpathlineto{\pgfqpoint{2.280949in}{2.373064in}}%
\pgfpathlineto{\pgfqpoint{2.288856in}{2.396807in}}%
\pgfpathlineto{\pgfqpoint{2.296763in}{2.420448in}}%
\pgfpathlineto{\pgfqpoint{2.304669in}{2.443979in}}%
\pgfpathlineto{\pgfqpoint{2.312576in}{2.467391in}}%
\pgfpathlineto{\pgfqpoint{2.320483in}{2.490676in}}%
\pgfpathlineto{\pgfqpoint{2.328390in}{2.513824in}}%
\pgfpathlineto{\pgfqpoint{2.336296in}{2.536828in}}%
\pgfpathlineto{\pgfqpoint{2.344203in}{2.559679in}}%
\pgfpathlineto{\pgfqpoint{2.352110in}{2.582367in}}%
\pgfpathlineto{\pgfqpoint{2.360016in}{2.604885in}}%
\pgfpathlineto{\pgfqpoint{2.367923in}{2.627224in}}%
\pgfpathlineto{\pgfqpoint{2.375830in}{2.649376in}}%
\pgfpathlineto{\pgfqpoint{2.383737in}{2.671331in}}%
\pgfpathlineto{\pgfqpoint{2.391643in}{2.693079in}}%
\pgfpathlineto{\pgfqpoint{2.399550in}{2.714610in}}%
\pgfpathlineto{\pgfqpoint{2.407457in}{2.735913in}}%
\pgfpathlineto{\pgfqpoint{2.415363in}{2.756979in}}%
\pgfpathlineto{\pgfqpoint{2.423270in}{2.777795in}}%
\pgfpathlineto{\pgfqpoint{2.431177in}{2.798352in}}%
\pgfpathlineto{\pgfqpoint{2.439084in}{2.818640in}}%
\pgfpathlineto{\pgfqpoint{2.446990in}{2.838647in}}%
\pgfpathlineto{\pgfqpoint{2.454897in}{2.858363in}}%
\pgfpathlineto{\pgfqpoint{2.462804in}{2.877778in}}%
\pgfpathlineto{\pgfqpoint{2.470711in}{2.896880in}}%
\pgfpathlineto{\pgfqpoint{2.478617in}{2.915661in}}%
\pgfpathlineto{\pgfqpoint{2.486524in}{2.934108in}}%
\pgfpathlineto{\pgfqpoint{2.494431in}{2.952211in}}%
\pgfpathlineto{\pgfqpoint{2.502337in}{2.969961in}}%
\pgfpathlineto{\pgfqpoint{2.510244in}{2.987346in}}%
\pgfpathlineto{\pgfqpoint{2.518151in}{3.004356in}}%
\pgfpathlineto{\pgfqpoint{2.526058in}{3.020980in}}%
\pgfpathlineto{\pgfqpoint{2.533964in}{3.037208in}}%
\pgfpathlineto{\pgfqpoint{2.541871in}{3.053029in}}%
\pgfpathlineto{\pgfqpoint{2.549778in}{3.068433in}}%
\pgfpathlineto{\pgfqpoint{2.557684in}{3.083409in}}%
\pgfpathlineto{\pgfqpoint{2.565591in}{3.097946in}}%
\pgfpathlineto{\pgfqpoint{2.573498in}{3.112035in}}%
\pgfpathlineto{\pgfqpoint{2.581405in}{3.125664in}}%
\pgfpathlineto{\pgfqpoint{2.589311in}{3.138823in}}%
\pgfpathlineto{\pgfqpoint{2.597218in}{3.151502in}}%
\pgfpathlineto{\pgfqpoint{2.605125in}{3.163690in}}%
\pgfpathlineto{\pgfqpoint{2.613032in}{3.175376in}}%
\pgfpathlineto{\pgfqpoint{2.620938in}{3.186550in}}%
\pgfpathlineto{\pgfqpoint{2.628845in}{3.197201in}}%
\pgfpathlineto{\pgfqpoint{2.636752in}{3.207319in}}%
\pgfpathlineto{\pgfqpoint{2.644658in}{3.216894in}}%
\pgfpathlineto{\pgfqpoint{2.652565in}{3.225914in}}%
\pgfpathlineto{\pgfqpoint{2.660472in}{3.234369in}}%
\pgfpathlineto{\pgfqpoint{2.668379in}{3.242249in}}%
\pgfpathlineto{\pgfqpoint{2.676285in}{3.249543in}}%
\pgfpathlineto{\pgfqpoint{2.684192in}{3.256241in}}%
\pgfpathlineto{\pgfqpoint{2.692099in}{3.262332in}}%
\pgfpathlineto{\pgfqpoint{2.700005in}{3.267805in}}%
\pgfpathlineto{\pgfqpoint{2.707912in}{3.272650in}}%
\pgfpathlineto{\pgfqpoint{2.715819in}{3.276857in}}%
\pgfpathlineto{\pgfqpoint{2.723726in}{3.280415in}}%
\pgfpathlineto{\pgfqpoint{2.731632in}{3.283313in}}%
\pgfpathlineto{\pgfqpoint{2.739539in}{3.285541in}}%
\pgfpathlineto{\pgfqpoint{2.747446in}{3.287088in}}%
\pgfpathlineto{\pgfqpoint{2.755353in}{3.287944in}}%
\pgfpathlineto{\pgfqpoint{2.763259in}{3.288099in}}%
\pgfpathlineto{\pgfqpoint{2.771166in}{3.287541in}}%
\pgfpathlineto{\pgfqpoint{2.779073in}{3.286260in}}%
\pgfpathlineto{\pgfqpoint{2.786979in}{3.284246in}}%
\pgfpathlineto{\pgfqpoint{2.794886in}{3.281488in}}%
\pgfpathlineto{\pgfqpoint{2.802793in}{3.277975in}}%
\pgfpathlineto{\pgfqpoint{2.810700in}{3.273698in}}%
\pgfpathlineto{\pgfqpoint{2.818606in}{3.268645in}}%
\pgfpathlineto{\pgfqpoint{2.826513in}{3.262806in}}%
\pgfpathlineto{\pgfqpoint{2.834420in}{3.256171in}}%
\pgfpathlineto{\pgfqpoint{2.842326in}{3.248728in}}%
\pgfpathlineto{\pgfqpoint{2.850233in}{3.240468in}}%
\pgfpathlineto{\pgfqpoint{2.858140in}{3.231379in}}%
\pgfpathlineto{\pgfqpoint{2.866047in}{3.221452in}}%
\pgfpathlineto{\pgfqpoint{2.873953in}{3.210677in}}%
\pgfpathlineto{\pgfqpoint{2.881860in}{3.199066in}}%
\pgfpathlineto{\pgfqpoint{2.889767in}{3.186657in}}%
\pgfpathlineto{\pgfqpoint{2.897674in}{3.173484in}}%
\pgfpathlineto{\pgfqpoint{2.905580in}{3.159586in}}%
\pgfpathlineto{\pgfqpoint{2.913487in}{3.144999in}}%
\pgfpathlineto{\pgfqpoint{2.921394in}{3.129760in}}%
\pgfpathlineto{\pgfqpoint{2.929300in}{3.113906in}}%
\pgfpathlineto{\pgfqpoint{2.937207in}{3.097473in}}%
\pgfpathlineto{\pgfqpoint{2.945114in}{3.080498in}}%
\pgfpathlineto{\pgfqpoint{2.953021in}{3.063019in}}%
\pgfpathlineto{\pgfqpoint{2.960927in}{3.045071in}}%
\pgfpathlineto{\pgfqpoint{2.968834in}{3.026692in}}%
\pgfpathlineto{\pgfqpoint{2.976741in}{3.007919in}}%
\pgfpathlineto{\pgfqpoint{2.984647in}{2.988788in}}%
\pgfpathlineto{\pgfqpoint{2.992554in}{2.969336in}}%
\pgfpathlineto{\pgfqpoint{3.000461in}{2.949599in}}%
\pgfpathlineto{\pgfqpoint{3.008368in}{2.929616in}}%
\pgfpathlineto{\pgfqpoint{3.016274in}{2.909422in}}%
\pgfpathlineto{\pgfqpoint{3.024181in}{2.889054in}}%
\pgfpathlineto{\pgfqpoint{3.032088in}{2.868549in}}%
\pgfpathlineto{\pgfqpoint{3.039995in}{2.847944in}}%
\pgfpathlineto{\pgfqpoint{3.047901in}{2.827276in}}%
\pgfpathlineto{\pgfqpoint{3.055808in}{2.806581in}}%
\pgfpathlineto{\pgfqpoint{3.063715in}{2.785897in}}%
\pgfpathlineto{\pgfqpoint{3.071621in}{2.765259in}}%
\pgfpathlineto{\pgfqpoint{3.079528in}{2.744706in}}%
\pgfpathlineto{\pgfqpoint{3.087435in}{2.724273in}}%
\pgfpathlineto{\pgfqpoint{3.095342in}{2.703997in}}%
\pgfpathlineto{\pgfqpoint{3.103248in}{2.683916in}}%
\pgfpathlineto{\pgfqpoint{3.111155in}{2.664066in}}%
\pgfpathlineto{\pgfqpoint{3.119062in}{2.644483in}}%
\pgfpathlineto{\pgfqpoint{3.126968in}{2.625206in}}%
\pgfpathlineto{\pgfqpoint{3.134875in}{2.606270in}}%
\pgfpathlineto{\pgfqpoint{3.142782in}{2.587712in}}%
\pgfpathlineto{\pgfqpoint{3.150689in}{2.569569in}}%
\pgfpathlineto{\pgfqpoint{3.158595in}{2.551878in}}%
\pgfpathlineto{\pgfqpoint{3.166502in}{2.534675in}}%
\pgfpathlineto{\pgfqpoint{3.174409in}{2.517998in}}%
\pgfpathlineto{\pgfqpoint{3.182316in}{2.501883in}}%
\pgfpathlineto{\pgfqpoint{3.190222in}{2.486368in}}%
\pgfpathlineto{\pgfqpoint{3.198129in}{2.471488in}}%
\pgfpathlineto{\pgfqpoint{3.206036in}{2.457281in}}%
\pgfpathlineto{\pgfqpoint{3.213942in}{2.443783in}}%
\pgfpathlineto{\pgfqpoint{3.221849in}{2.431031in}}%
\pgfpathlineto{\pgfqpoint{3.229756in}{2.419063in}}%
\pgfpathlineto{\pgfqpoint{3.237663in}{2.407914in}}%
\pgfpathlineto{\pgfqpoint{3.245569in}{2.397622in}}%
\pgfpathlineto{\pgfqpoint{3.253476in}{2.388224in}}%
\pgfpathlineto{\pgfqpoint{3.261383in}{2.379755in}}%
\pgfpathlineto{\pgfqpoint{3.269289in}{2.372254in}}%
\pgfpathlineto{\pgfqpoint{3.277196in}{2.365757in}}%
\pgfpathlineto{\pgfqpoint{3.285103in}{2.360300in}}%
\pgfpathlineto{\pgfqpoint{3.293010in}{2.355920in}}%
\pgfpathlineto{\pgfqpoint{3.300916in}{2.352655in}}%
\pgfpathlineto{\pgfqpoint{3.308823in}{2.350541in}}%
\pgfpathlineto{\pgfqpoint{3.316730in}{2.349614in}}%
\pgfpathlineto{\pgfqpoint{3.324637in}{2.349913in}}%
\pgfpathlineto{\pgfqpoint{3.332543in}{2.351472in}}%
\pgfpathlineto{\pgfqpoint{3.340450in}{2.354330in}}%
\pgfpathlineto{\pgfqpoint{3.348357in}{2.358523in}}%
\pgfpathlineto{\pgfqpoint{3.356263in}{2.364088in}}%
\pgfpathlineto{\pgfqpoint{3.364170in}{2.371061in}}%
\pgfpathlineto{\pgfqpoint{3.372077in}{2.379460in}}%
\pgfpathlineto{\pgfqpoint{3.379984in}{2.389248in}}%
\pgfpathlineto{\pgfqpoint{3.387890in}{2.400380in}}%
\pgfpathlineto{\pgfqpoint{3.395797in}{2.412808in}}%
\pgfpathlineto{\pgfqpoint{3.403704in}{2.426487in}}%
\pgfpathlineto{\pgfqpoint{3.411610in}{2.441371in}}%
\pgfpathlineto{\pgfqpoint{3.419517in}{2.457414in}}%
\pgfpathlineto{\pgfqpoint{3.427424in}{2.474571in}}%
\pgfpathlineto{\pgfqpoint{3.435331in}{2.492794in}}%
\pgfpathlineto{\pgfqpoint{3.443237in}{2.512038in}}%
\pgfpathlineto{\pgfqpoint{3.451144in}{2.532257in}}%
\pgfpathlineto{\pgfqpoint{3.459051in}{2.553404in}}%
\pgfpathlineto{\pgfqpoint{3.466958in}{2.575435in}}%
\pgfpathlineto{\pgfqpoint{3.474864in}{2.598303in}}%
\pgfpathlineto{\pgfqpoint{3.482771in}{2.621961in}}%
\pgfpathlineto{\pgfqpoint{3.490678in}{2.646364in}}%
\pgfpathlineto{\pgfqpoint{3.498584in}{2.671466in}}%
\pgfpathlineto{\pgfqpoint{3.506491in}{2.697221in}}%
\pgfpathlineto{\pgfqpoint{3.514398in}{2.723583in}}%
\pgfpathlineto{\pgfqpoint{3.522305in}{2.750505in}}%
\pgfpathlineto{\pgfqpoint{3.530211in}{2.777942in}}%
\pgfpathlineto{\pgfqpoint{3.538118in}{2.805848in}}%
\pgfpathlineto{\pgfqpoint{3.546025in}{2.834177in}}%
\pgfpathlineto{\pgfqpoint{3.553931in}{2.862882in}}%
\pgfpathlineto{\pgfqpoint{3.561838in}{2.891918in}}%
\pgfpathlineto{\pgfqpoint{3.569745in}{2.921239in}}%
\pgfpathlineto{\pgfqpoint{3.577652in}{2.950798in}}%
\pgfpathlineto{\pgfqpoint{3.585558in}{2.980550in}}%
\pgfpathlineto{\pgfqpoint{3.593465in}{3.010449in}}%
\pgfpathlineto{\pgfqpoint{3.601372in}{3.040448in}}%
\pgfpathlineto{\pgfqpoint{3.609279in}{3.070502in}}%
\pgfpathlineto{\pgfqpoint{3.617185in}{3.100565in}}%
\pgfpathlineto{\pgfqpoint{3.625092in}{3.130590in}}%
\pgfpathlineto{\pgfqpoint{3.632999in}{3.160532in}}%
\pgfpathlineto{\pgfqpoint{3.640905in}{3.190344in}}%
\pgfpathlineto{\pgfqpoint{3.648812in}{3.219980in}}%
\pgfpathlineto{\pgfqpoint{3.656719in}{3.249396in}}%
\pgfpathlineto{\pgfqpoint{3.664626in}{3.278543in}}%
\pgfpathlineto{\pgfqpoint{3.672532in}{3.307378in}}%
\pgfpathlineto{\pgfqpoint{3.680439in}{3.335852in}}%
\pgfpathlineto{\pgfqpoint{3.688346in}{3.363921in}}%
\pgfpathlineto{\pgfqpoint{3.696253in}{3.391539in}}%
\pgfpathlineto{\pgfqpoint{3.704159in}{3.418659in}}%
\pgfpathlineto{\pgfqpoint{3.712066in}{3.445235in}}%
\pgfpathlineto{\pgfqpoint{3.719973in}{3.471222in}}%
\pgfpathlineto{\pgfqpoint{3.727879in}{3.496573in}}%
\pgfpathlineto{\pgfqpoint{3.735786in}{3.521243in}}%
\pgfpathlineto{\pgfqpoint{3.743693in}{3.545185in}}%
\pgfpathlineto{\pgfqpoint{3.751600in}{3.568353in}}%
\pgfpathlineto{\pgfqpoint{3.759506in}{3.590702in}}%
\pgfpathlineto{\pgfqpoint{3.767413in}{3.612185in}}%
\pgfpathlineto{\pgfqpoint{3.775320in}{3.632756in}}%
\pgfpathlineto{\pgfqpoint{3.783226in}{3.652369in}}%
\pgfpathlineto{\pgfqpoint{3.791133in}{3.670979in}}%
\pgfpathlineto{\pgfqpoint{3.799040in}{3.688539in}}%
\pgfpathlineto{\pgfqpoint{3.806947in}{3.705003in}}%
\pgfpathlineto{\pgfqpoint{3.814853in}{3.720325in}}%
\pgfpathlineto{\pgfqpoint{3.822760in}{3.734460in}}%
\pgfpathlineto{\pgfqpoint{3.830667in}{3.747360in}}%
\pgfpathlineto{\pgfqpoint{3.838574in}{3.758981in}}%
\pgfpathlineto{\pgfqpoint{3.846480in}{3.769276in}}%
\pgfpathlineto{\pgfqpoint{3.854387in}{3.778200in}}%
\pgfpathlineto{\pgfqpoint{3.862294in}{3.785710in}}%
\pgfpathlineto{\pgfqpoint{3.870200in}{3.791807in}}%
\pgfpathlineto{\pgfqpoint{3.878107in}{3.796513in}}%
\pgfpathlineto{\pgfqpoint{3.886014in}{3.799849in}}%
\pgfpathlineto{\pgfqpoint{3.893921in}{3.801838in}}%
\pgfpathlineto{\pgfqpoint{3.901827in}{3.802500in}}%
\pgfpathlineto{\pgfqpoint{3.909734in}{3.801858in}}%
\pgfpathlineto{\pgfqpoint{3.917641in}{3.799934in}}%
\pgfpathlineto{\pgfqpoint{3.925547in}{3.796749in}}%
\pgfpathlineto{\pgfqpoint{3.933454in}{3.792324in}}%
\pgfpathlineto{\pgfqpoint{3.941361in}{3.786683in}}%
\pgfpathlineto{\pgfqpoint{3.949268in}{3.779846in}}%
\pgfpathlineto{\pgfqpoint{3.957174in}{3.771835in}}%
\pgfpathlineto{\pgfqpoint{3.965081in}{3.762672in}}%
\pgfpathlineto{\pgfqpoint{3.972988in}{3.752379in}}%
\pgfpathlineto{\pgfqpoint{3.980895in}{3.740977in}}%
\pgfpathlineto{\pgfqpoint{3.988801in}{3.728489in}}%
\pgfpathlineto{\pgfqpoint{3.996708in}{3.714936in}}%
\pgfpathlineto{\pgfqpoint{4.004615in}{3.700339in}}%
\pgfpathlineto{\pgfqpoint{4.012521in}{3.684721in}}%
\pgfpathlineto{\pgfqpoint{4.020428in}{3.668103in}}%
\pgfpathlineto{\pgfqpoint{4.028335in}{3.650507in}}%
\pgfpathlineto{\pgfqpoint{4.036242in}{3.631955in}}%
\pgfpathlineto{\pgfqpoint{4.044148in}{3.612469in}}%
\pgfpathlineto{\pgfqpoint{4.052055in}{3.592069in}}%
\pgfpathlineto{\pgfqpoint{4.059962in}{3.570779in}}%
\pgfpathlineto{\pgfqpoint{4.067868in}{3.548619in}}%
\pgfpathlineto{\pgfqpoint{4.075775in}{3.525612in}}%
\pgfpathlineto{\pgfqpoint{4.083682in}{3.501780in}}%
\pgfpathlineto{\pgfqpoint{4.091589in}{3.477143in}}%
\pgfpathlineto{\pgfqpoint{4.099495in}{3.451724in}}%
\pgfpathlineto{\pgfqpoint{4.107402in}{3.425545in}}%
\pgfpathlineto{\pgfqpoint{4.115309in}{3.398627in}}%
\pgfpathlineto{\pgfqpoint{4.123216in}{3.370992in}}%
\pgfpathlineto{\pgfqpoint{4.131122in}{3.342662in}}%
\pgfpathlineto{\pgfqpoint{4.139029in}{3.313658in}}%
\pgfpathlineto{\pgfqpoint{4.146936in}{3.284003in}}%
\pgfpathlineto{\pgfqpoint{4.154842in}{3.253718in}}%
\pgfpathlineto{\pgfqpoint{4.162749in}{3.222825in}}%
\pgfpathlineto{\pgfqpoint{4.170656in}{3.191345in}}%
\pgfpathlineto{\pgfqpoint{4.178563in}{3.159300in}}%
\pgfpathlineto{\pgfqpoint{4.186469in}{3.126713in}}%
\pgfpathlineto{\pgfqpoint{4.194376in}{3.093604in}}%
\pgfpathlineto{\pgfqpoint{4.202283in}{3.059996in}}%
\pgfpathlineto{\pgfqpoint{4.210189in}{3.025911in}}%
\pgfpathlineto{\pgfqpoint{4.218096in}{2.991369in}}%
\pgfpathlineto{\pgfqpoint{4.226003in}{2.956394in}}%
\pgfpathlineto{\pgfqpoint{4.233910in}{2.921006in}}%
\pgfpathlineto{\pgfqpoint{4.241816in}{2.885227in}}%
\pgfpathlineto{\pgfqpoint{4.249723in}{2.849080in}}%
\pgfpathlineto{\pgfqpoint{4.257630in}{2.812585in}}%
\pgfpathlineto{\pgfqpoint{4.265537in}{2.775765in}}%
\pgfpathlineto{\pgfqpoint{4.273443in}{2.738641in}}%
\pgfpathlineto{\pgfqpoint{4.281350in}{2.701236in}}%
\pgfpathlineto{\pgfqpoint{4.289257in}{2.663570in}}%
\pgfpathlineto{\pgfqpoint{4.297163in}{2.625666in}}%
\pgfpathlineto{\pgfqpoint{4.305070in}{2.587545in}}%
\pgfpathlineto{\pgfqpoint{4.312977in}{2.549230in}}%
\pgfpathlineto{\pgfqpoint{4.320884in}{2.510741in}}%
\pgfpathlineto{\pgfqpoint{4.328790in}{2.472101in}}%
\pgfpathlineto{\pgfqpoint{4.336697in}{2.433332in}}%
\pgfpathlineto{\pgfqpoint{4.344604in}{2.394455in}}%
\pgfpathlineto{\pgfqpoint{4.352510in}{2.355491in}}%
\pgfpathlineto{\pgfqpoint{4.360417in}{2.316463in}}%
\pgfpathlineto{\pgfqpoint{4.368324in}{2.277393in}}%
\pgfpathlineto{\pgfqpoint{4.376231in}{2.238302in}}%
\pgfpathlineto{\pgfqpoint{4.384137in}{2.199212in}}%
\pgfpathlineto{\pgfqpoint{4.392044in}{2.160144in}}%
\pgfpathlineto{\pgfqpoint{4.399951in}{2.121121in}}%
\pgfpathlineto{\pgfqpoint{4.399951in}{2.088091in}}%
\pgfpathlineto{\pgfqpoint{4.399951in}{2.088091in}}%
\pgfpathlineto{\pgfqpoint{4.392044in}{2.088091in}}%
\pgfpathlineto{\pgfqpoint{4.384137in}{2.088091in}}%
\pgfpathlineto{\pgfqpoint{4.376231in}{2.088091in}}%
\pgfpathlineto{\pgfqpoint{4.368324in}{2.088091in}}%
\pgfpathlineto{\pgfqpoint{4.360417in}{2.088091in}}%
\pgfpathlineto{\pgfqpoint{4.352510in}{2.088091in}}%
\pgfpathlineto{\pgfqpoint{4.344604in}{2.088091in}}%
\pgfpathlineto{\pgfqpoint{4.336697in}{2.088091in}}%
\pgfpathlineto{\pgfqpoint{4.328790in}{2.088091in}}%
\pgfpathlineto{\pgfqpoint{4.320884in}{2.088091in}}%
\pgfpathlineto{\pgfqpoint{4.312977in}{2.088091in}}%
\pgfpathlineto{\pgfqpoint{4.305070in}{2.088091in}}%
\pgfpathlineto{\pgfqpoint{4.297163in}{2.088091in}}%
\pgfpathlineto{\pgfqpoint{4.289257in}{2.088091in}}%
\pgfpathlineto{\pgfqpoint{4.281350in}{2.088091in}}%
\pgfpathlineto{\pgfqpoint{4.273443in}{2.088091in}}%
\pgfpathlineto{\pgfqpoint{4.265537in}{2.088091in}}%
\pgfpathlineto{\pgfqpoint{4.257630in}{2.088091in}}%
\pgfpathlineto{\pgfqpoint{4.249723in}{2.088091in}}%
\pgfpathlineto{\pgfqpoint{4.241816in}{2.088091in}}%
\pgfpathlineto{\pgfqpoint{4.233910in}{2.088091in}}%
\pgfpathlineto{\pgfqpoint{4.226003in}{2.088091in}}%
\pgfpathlineto{\pgfqpoint{4.218096in}{2.088091in}}%
\pgfpathlineto{\pgfqpoint{4.210189in}{2.088091in}}%
\pgfpathlineto{\pgfqpoint{4.202283in}{2.088091in}}%
\pgfpathlineto{\pgfqpoint{4.194376in}{2.088091in}}%
\pgfpathlineto{\pgfqpoint{4.186469in}{2.088091in}}%
\pgfpathlineto{\pgfqpoint{4.178563in}{2.088091in}}%
\pgfpathlineto{\pgfqpoint{4.170656in}{2.088091in}}%
\pgfpathlineto{\pgfqpoint{4.162749in}{2.088091in}}%
\pgfpathlineto{\pgfqpoint{4.154842in}{2.088091in}}%
\pgfpathlineto{\pgfqpoint{4.146936in}{2.088091in}}%
\pgfpathlineto{\pgfqpoint{4.139029in}{2.088091in}}%
\pgfpathlineto{\pgfqpoint{4.131122in}{2.088091in}}%
\pgfpathlineto{\pgfqpoint{4.123216in}{2.088091in}}%
\pgfpathlineto{\pgfqpoint{4.115309in}{2.088091in}}%
\pgfpathlineto{\pgfqpoint{4.107402in}{2.088091in}}%
\pgfpathlineto{\pgfqpoint{4.099495in}{2.088091in}}%
\pgfpathlineto{\pgfqpoint{4.091589in}{2.088091in}}%
\pgfpathlineto{\pgfqpoint{4.083682in}{2.088091in}}%
\pgfpathlineto{\pgfqpoint{4.075775in}{2.088091in}}%
\pgfpathlineto{\pgfqpoint{4.067868in}{2.088091in}}%
\pgfpathlineto{\pgfqpoint{4.059962in}{2.088091in}}%
\pgfpathlineto{\pgfqpoint{4.052055in}{2.088091in}}%
\pgfpathlineto{\pgfqpoint{4.044148in}{2.088091in}}%
\pgfpathlineto{\pgfqpoint{4.036242in}{2.088091in}}%
\pgfpathlineto{\pgfqpoint{4.028335in}{2.088091in}}%
\pgfpathlineto{\pgfqpoint{4.020428in}{2.088091in}}%
\pgfpathlineto{\pgfqpoint{4.012521in}{2.088091in}}%
\pgfpathlineto{\pgfqpoint{4.004615in}{2.088091in}}%
\pgfpathlineto{\pgfqpoint{3.996708in}{2.088091in}}%
\pgfpathlineto{\pgfqpoint{3.988801in}{2.088091in}}%
\pgfpathlineto{\pgfqpoint{3.980895in}{2.088091in}}%
\pgfpathlineto{\pgfqpoint{3.972988in}{2.088091in}}%
\pgfpathlineto{\pgfqpoint{3.965081in}{2.088091in}}%
\pgfpathlineto{\pgfqpoint{3.957174in}{2.088091in}}%
\pgfpathlineto{\pgfqpoint{3.949268in}{2.088091in}}%
\pgfpathlineto{\pgfqpoint{3.941361in}{2.088091in}}%
\pgfpathlineto{\pgfqpoint{3.933454in}{2.088091in}}%
\pgfpathlineto{\pgfqpoint{3.925547in}{2.088091in}}%
\pgfpathlineto{\pgfqpoint{3.917641in}{2.088091in}}%
\pgfpathlineto{\pgfqpoint{3.909734in}{2.088091in}}%
\pgfpathlineto{\pgfqpoint{3.901827in}{2.088091in}}%
\pgfpathlineto{\pgfqpoint{3.893921in}{2.088091in}}%
\pgfpathlineto{\pgfqpoint{3.886014in}{2.088091in}}%
\pgfpathlineto{\pgfqpoint{3.878107in}{2.088091in}}%
\pgfpathlineto{\pgfqpoint{3.870200in}{2.088091in}}%
\pgfpathlineto{\pgfqpoint{3.862294in}{2.088091in}}%
\pgfpathlineto{\pgfqpoint{3.854387in}{2.088091in}}%
\pgfpathlineto{\pgfqpoint{3.846480in}{2.088091in}}%
\pgfpathlineto{\pgfqpoint{3.838574in}{2.088091in}}%
\pgfpathlineto{\pgfqpoint{3.830667in}{2.088091in}}%
\pgfpathlineto{\pgfqpoint{3.822760in}{2.088091in}}%
\pgfpathlineto{\pgfqpoint{3.814853in}{2.088091in}}%
\pgfpathlineto{\pgfqpoint{3.806947in}{2.088091in}}%
\pgfpathlineto{\pgfqpoint{3.799040in}{2.088091in}}%
\pgfpathlineto{\pgfqpoint{3.791133in}{2.088091in}}%
\pgfpathlineto{\pgfqpoint{3.783226in}{2.088091in}}%
\pgfpathlineto{\pgfqpoint{3.775320in}{2.088091in}}%
\pgfpathlineto{\pgfqpoint{3.767413in}{2.088091in}}%
\pgfpathlineto{\pgfqpoint{3.759506in}{2.088091in}}%
\pgfpathlineto{\pgfqpoint{3.751600in}{2.088091in}}%
\pgfpathlineto{\pgfqpoint{3.743693in}{2.088091in}}%
\pgfpathlineto{\pgfqpoint{3.735786in}{2.088091in}}%
\pgfpathlineto{\pgfqpoint{3.727879in}{2.088091in}}%
\pgfpathlineto{\pgfqpoint{3.719973in}{2.088091in}}%
\pgfpathlineto{\pgfqpoint{3.712066in}{2.088091in}}%
\pgfpathlineto{\pgfqpoint{3.704159in}{2.088091in}}%
\pgfpathlineto{\pgfqpoint{3.696253in}{2.088091in}}%
\pgfpathlineto{\pgfqpoint{3.688346in}{2.088091in}}%
\pgfpathlineto{\pgfqpoint{3.680439in}{2.088091in}}%
\pgfpathlineto{\pgfqpoint{3.672532in}{2.088091in}}%
\pgfpathlineto{\pgfqpoint{3.664626in}{2.088091in}}%
\pgfpathlineto{\pgfqpoint{3.656719in}{2.088091in}}%
\pgfpathlineto{\pgfqpoint{3.648812in}{2.088091in}}%
\pgfpathlineto{\pgfqpoint{3.640905in}{2.088091in}}%
\pgfpathlineto{\pgfqpoint{3.632999in}{2.088091in}}%
\pgfpathlineto{\pgfqpoint{3.625092in}{2.088091in}}%
\pgfpathlineto{\pgfqpoint{3.617185in}{2.088091in}}%
\pgfpathlineto{\pgfqpoint{3.609279in}{2.088091in}}%
\pgfpathlineto{\pgfqpoint{3.601372in}{2.088091in}}%
\pgfpathlineto{\pgfqpoint{3.593465in}{2.088091in}}%
\pgfpathlineto{\pgfqpoint{3.585558in}{2.088091in}}%
\pgfpathlineto{\pgfqpoint{3.577652in}{2.088091in}}%
\pgfpathlineto{\pgfqpoint{3.569745in}{2.088091in}}%
\pgfpathlineto{\pgfqpoint{3.561838in}{2.088091in}}%
\pgfpathlineto{\pgfqpoint{3.553931in}{2.088091in}}%
\pgfpathlineto{\pgfqpoint{3.546025in}{2.088091in}}%
\pgfpathlineto{\pgfqpoint{3.538118in}{2.088091in}}%
\pgfpathlineto{\pgfqpoint{3.530211in}{2.088091in}}%
\pgfpathlineto{\pgfqpoint{3.522305in}{2.088091in}}%
\pgfpathlineto{\pgfqpoint{3.514398in}{2.088091in}}%
\pgfpathlineto{\pgfqpoint{3.506491in}{2.088091in}}%
\pgfpathlineto{\pgfqpoint{3.498584in}{2.088091in}}%
\pgfpathlineto{\pgfqpoint{3.490678in}{2.088091in}}%
\pgfpathlineto{\pgfqpoint{3.482771in}{2.088091in}}%
\pgfpathlineto{\pgfqpoint{3.474864in}{2.088091in}}%
\pgfpathlineto{\pgfqpoint{3.466958in}{2.088091in}}%
\pgfpathlineto{\pgfqpoint{3.459051in}{2.088091in}}%
\pgfpathlineto{\pgfqpoint{3.451144in}{2.088091in}}%
\pgfpathlineto{\pgfqpoint{3.443237in}{2.088091in}}%
\pgfpathlineto{\pgfqpoint{3.435331in}{2.088091in}}%
\pgfpathlineto{\pgfqpoint{3.427424in}{2.088091in}}%
\pgfpathlineto{\pgfqpoint{3.419517in}{2.088091in}}%
\pgfpathlineto{\pgfqpoint{3.411610in}{2.088091in}}%
\pgfpathlineto{\pgfqpoint{3.403704in}{2.088091in}}%
\pgfpathlineto{\pgfqpoint{3.395797in}{2.088091in}}%
\pgfpathlineto{\pgfqpoint{3.387890in}{2.088091in}}%
\pgfpathlineto{\pgfqpoint{3.379984in}{2.088091in}}%
\pgfpathlineto{\pgfqpoint{3.372077in}{2.088091in}}%
\pgfpathlineto{\pgfqpoint{3.364170in}{2.088091in}}%
\pgfpathlineto{\pgfqpoint{3.356263in}{2.088091in}}%
\pgfpathlineto{\pgfqpoint{3.348357in}{2.088091in}}%
\pgfpathlineto{\pgfqpoint{3.340450in}{2.088091in}}%
\pgfpathlineto{\pgfqpoint{3.332543in}{2.088091in}}%
\pgfpathlineto{\pgfqpoint{3.324637in}{2.088091in}}%
\pgfpathlineto{\pgfqpoint{3.316730in}{2.088091in}}%
\pgfpathlineto{\pgfqpoint{3.308823in}{2.088091in}}%
\pgfpathlineto{\pgfqpoint{3.300916in}{2.088091in}}%
\pgfpathlineto{\pgfqpoint{3.293010in}{2.088091in}}%
\pgfpathlineto{\pgfqpoint{3.285103in}{2.088091in}}%
\pgfpathlineto{\pgfqpoint{3.277196in}{2.088091in}}%
\pgfpathlineto{\pgfqpoint{3.269289in}{2.088091in}}%
\pgfpathlineto{\pgfqpoint{3.261383in}{2.088091in}}%
\pgfpathlineto{\pgfqpoint{3.253476in}{2.088091in}}%
\pgfpathlineto{\pgfqpoint{3.245569in}{2.088091in}}%
\pgfpathlineto{\pgfqpoint{3.237663in}{2.088091in}}%
\pgfpathlineto{\pgfqpoint{3.229756in}{2.088091in}}%
\pgfpathlineto{\pgfqpoint{3.221849in}{2.088091in}}%
\pgfpathlineto{\pgfqpoint{3.213942in}{2.088091in}}%
\pgfpathlineto{\pgfqpoint{3.206036in}{2.088091in}}%
\pgfpathlineto{\pgfqpoint{3.198129in}{2.088091in}}%
\pgfpathlineto{\pgfqpoint{3.190222in}{2.088091in}}%
\pgfpathlineto{\pgfqpoint{3.182316in}{2.088091in}}%
\pgfpathlineto{\pgfqpoint{3.174409in}{2.088091in}}%
\pgfpathlineto{\pgfqpoint{3.166502in}{2.088091in}}%
\pgfpathlineto{\pgfqpoint{3.158595in}{2.088091in}}%
\pgfpathlineto{\pgfqpoint{3.150689in}{2.088091in}}%
\pgfpathlineto{\pgfqpoint{3.142782in}{2.088091in}}%
\pgfpathlineto{\pgfqpoint{3.134875in}{2.088091in}}%
\pgfpathlineto{\pgfqpoint{3.126968in}{2.088091in}}%
\pgfpathlineto{\pgfqpoint{3.119062in}{2.088091in}}%
\pgfpathlineto{\pgfqpoint{3.111155in}{2.088091in}}%
\pgfpathlineto{\pgfqpoint{3.103248in}{2.088091in}}%
\pgfpathlineto{\pgfqpoint{3.095342in}{2.088091in}}%
\pgfpathlineto{\pgfqpoint{3.087435in}{2.088091in}}%
\pgfpathlineto{\pgfqpoint{3.079528in}{2.088091in}}%
\pgfpathlineto{\pgfqpoint{3.071621in}{2.088091in}}%
\pgfpathlineto{\pgfqpoint{3.063715in}{2.088091in}}%
\pgfpathlineto{\pgfqpoint{3.055808in}{2.088091in}}%
\pgfpathlineto{\pgfqpoint{3.047901in}{2.088091in}}%
\pgfpathlineto{\pgfqpoint{3.039995in}{2.088091in}}%
\pgfpathlineto{\pgfqpoint{3.032088in}{2.088091in}}%
\pgfpathlineto{\pgfqpoint{3.024181in}{2.088091in}}%
\pgfpathlineto{\pgfqpoint{3.016274in}{2.088091in}}%
\pgfpathlineto{\pgfqpoint{3.008368in}{2.088091in}}%
\pgfpathlineto{\pgfqpoint{3.000461in}{2.088091in}}%
\pgfpathlineto{\pgfqpoint{2.992554in}{2.088091in}}%
\pgfpathlineto{\pgfqpoint{2.984647in}{2.088091in}}%
\pgfpathlineto{\pgfqpoint{2.976741in}{2.088091in}}%
\pgfpathlineto{\pgfqpoint{2.968834in}{2.088091in}}%
\pgfpathlineto{\pgfqpoint{2.960927in}{2.088091in}}%
\pgfpathlineto{\pgfqpoint{2.953021in}{2.088091in}}%
\pgfpathlineto{\pgfqpoint{2.945114in}{2.088091in}}%
\pgfpathlineto{\pgfqpoint{2.937207in}{2.088091in}}%
\pgfpathlineto{\pgfqpoint{2.929300in}{2.088091in}}%
\pgfpathlineto{\pgfqpoint{2.921394in}{2.088091in}}%
\pgfpathlineto{\pgfqpoint{2.913487in}{2.088091in}}%
\pgfpathlineto{\pgfqpoint{2.905580in}{2.088091in}}%
\pgfpathlineto{\pgfqpoint{2.897674in}{2.088091in}}%
\pgfpathlineto{\pgfqpoint{2.889767in}{2.088091in}}%
\pgfpathlineto{\pgfqpoint{2.881860in}{2.088091in}}%
\pgfpathlineto{\pgfqpoint{2.873953in}{2.088091in}}%
\pgfpathlineto{\pgfqpoint{2.866047in}{2.088091in}}%
\pgfpathlineto{\pgfqpoint{2.858140in}{2.088091in}}%
\pgfpathlineto{\pgfqpoint{2.850233in}{2.088091in}}%
\pgfpathlineto{\pgfqpoint{2.842326in}{2.088091in}}%
\pgfpathlineto{\pgfqpoint{2.834420in}{2.088091in}}%
\pgfpathlineto{\pgfqpoint{2.826513in}{2.088091in}}%
\pgfpathlineto{\pgfqpoint{2.818606in}{2.088091in}}%
\pgfpathlineto{\pgfqpoint{2.810700in}{2.088091in}}%
\pgfpathlineto{\pgfqpoint{2.802793in}{2.088091in}}%
\pgfpathlineto{\pgfqpoint{2.794886in}{2.088091in}}%
\pgfpathlineto{\pgfqpoint{2.786979in}{2.088091in}}%
\pgfpathlineto{\pgfqpoint{2.779073in}{2.088091in}}%
\pgfpathlineto{\pgfqpoint{2.771166in}{2.088091in}}%
\pgfpathlineto{\pgfqpoint{2.763259in}{2.088091in}}%
\pgfpathlineto{\pgfqpoint{2.755353in}{2.088091in}}%
\pgfpathlineto{\pgfqpoint{2.747446in}{2.088091in}}%
\pgfpathlineto{\pgfqpoint{2.739539in}{2.088091in}}%
\pgfpathlineto{\pgfqpoint{2.731632in}{2.088091in}}%
\pgfpathlineto{\pgfqpoint{2.723726in}{2.088091in}}%
\pgfpathlineto{\pgfqpoint{2.715819in}{2.088091in}}%
\pgfpathlineto{\pgfqpoint{2.707912in}{2.088091in}}%
\pgfpathlineto{\pgfqpoint{2.700005in}{2.088091in}}%
\pgfpathlineto{\pgfqpoint{2.692099in}{2.088091in}}%
\pgfpathlineto{\pgfqpoint{2.684192in}{2.088091in}}%
\pgfpathlineto{\pgfqpoint{2.676285in}{2.088091in}}%
\pgfpathlineto{\pgfqpoint{2.668379in}{2.088091in}}%
\pgfpathlineto{\pgfqpoint{2.660472in}{2.088091in}}%
\pgfpathlineto{\pgfqpoint{2.652565in}{2.088091in}}%
\pgfpathlineto{\pgfqpoint{2.644658in}{2.088091in}}%
\pgfpathlineto{\pgfqpoint{2.636752in}{2.088091in}}%
\pgfpathlineto{\pgfqpoint{2.628845in}{2.088091in}}%
\pgfpathlineto{\pgfqpoint{2.620938in}{2.088091in}}%
\pgfpathlineto{\pgfqpoint{2.613032in}{2.088091in}}%
\pgfpathlineto{\pgfqpoint{2.605125in}{2.088091in}}%
\pgfpathlineto{\pgfqpoint{2.597218in}{2.088091in}}%
\pgfpathlineto{\pgfqpoint{2.589311in}{2.088091in}}%
\pgfpathlineto{\pgfqpoint{2.581405in}{2.088091in}}%
\pgfpathlineto{\pgfqpoint{2.573498in}{2.088091in}}%
\pgfpathlineto{\pgfqpoint{2.565591in}{2.088091in}}%
\pgfpathlineto{\pgfqpoint{2.557684in}{2.088091in}}%
\pgfpathlineto{\pgfqpoint{2.549778in}{2.088091in}}%
\pgfpathlineto{\pgfqpoint{2.541871in}{2.088091in}}%
\pgfpathlineto{\pgfqpoint{2.533964in}{2.088091in}}%
\pgfpathlineto{\pgfqpoint{2.526058in}{2.088091in}}%
\pgfpathlineto{\pgfqpoint{2.518151in}{2.088091in}}%
\pgfpathlineto{\pgfqpoint{2.510244in}{2.088091in}}%
\pgfpathlineto{\pgfqpoint{2.502337in}{2.088091in}}%
\pgfpathlineto{\pgfqpoint{2.494431in}{2.088091in}}%
\pgfpathlineto{\pgfqpoint{2.486524in}{2.088091in}}%
\pgfpathlineto{\pgfqpoint{2.478617in}{2.088091in}}%
\pgfpathlineto{\pgfqpoint{2.470711in}{2.088091in}}%
\pgfpathlineto{\pgfqpoint{2.462804in}{2.088091in}}%
\pgfpathlineto{\pgfqpoint{2.454897in}{2.088091in}}%
\pgfpathlineto{\pgfqpoint{2.446990in}{2.088091in}}%
\pgfpathlineto{\pgfqpoint{2.439084in}{2.088091in}}%
\pgfpathlineto{\pgfqpoint{2.431177in}{2.088091in}}%
\pgfpathlineto{\pgfqpoint{2.423270in}{2.088091in}}%
\pgfpathlineto{\pgfqpoint{2.415363in}{2.088091in}}%
\pgfpathlineto{\pgfqpoint{2.407457in}{2.088091in}}%
\pgfpathlineto{\pgfqpoint{2.399550in}{2.088091in}}%
\pgfpathlineto{\pgfqpoint{2.391643in}{2.088091in}}%
\pgfpathlineto{\pgfqpoint{2.383737in}{2.088091in}}%
\pgfpathlineto{\pgfqpoint{2.375830in}{2.088091in}}%
\pgfpathlineto{\pgfqpoint{2.367923in}{2.088091in}}%
\pgfpathlineto{\pgfqpoint{2.360016in}{2.088091in}}%
\pgfpathlineto{\pgfqpoint{2.352110in}{2.088091in}}%
\pgfpathlineto{\pgfqpoint{2.344203in}{2.088091in}}%
\pgfpathlineto{\pgfqpoint{2.336296in}{2.088091in}}%
\pgfpathlineto{\pgfqpoint{2.328390in}{2.088091in}}%
\pgfpathlineto{\pgfqpoint{2.320483in}{2.088091in}}%
\pgfpathlineto{\pgfqpoint{2.312576in}{2.088091in}}%
\pgfpathlineto{\pgfqpoint{2.304669in}{2.088091in}}%
\pgfpathlineto{\pgfqpoint{2.296763in}{2.088091in}}%
\pgfpathlineto{\pgfqpoint{2.288856in}{2.088091in}}%
\pgfpathlineto{\pgfqpoint{2.280949in}{2.088091in}}%
\pgfpathlineto{\pgfqpoint{2.273042in}{2.088091in}}%
\pgfpathlineto{\pgfqpoint{2.265136in}{2.088091in}}%
\pgfpathlineto{\pgfqpoint{2.257229in}{2.088091in}}%
\pgfpathlineto{\pgfqpoint{2.249322in}{2.088091in}}%
\pgfpathlineto{\pgfqpoint{2.241416in}{2.088091in}}%
\pgfpathlineto{\pgfqpoint{2.233509in}{2.088091in}}%
\pgfpathlineto{\pgfqpoint{2.225602in}{2.088091in}}%
\pgfpathlineto{\pgfqpoint{2.217695in}{2.088091in}}%
\pgfpathlineto{\pgfqpoint{2.209789in}{2.088091in}}%
\pgfpathlineto{\pgfqpoint{2.201882in}{2.088091in}}%
\pgfpathlineto{\pgfqpoint{2.193975in}{2.088091in}}%
\pgfpathlineto{\pgfqpoint{2.193975in}{2.088091in}}%
\pgfpathclose%
\pgfusepath{stroke,fill}%
}%
\begin{pgfscope}%
\pgfsys@transformshift{0.000000in}{0.000000in}%
\pgfsys@useobject{currentmarker}{}%
\end{pgfscope}%
\end{pgfscope}%
\begin{pgfscope}%
\pgfpathrectangle{\pgfqpoint{0.700000in}{0.495000in}}{\pgfqpoint{4.340000in}{3.465000in}}%
\pgfusepath{clip}%
\pgfsetroundcap%
\pgfsetroundjoin%
\pgfsetlinewidth{1.505625pt}%
\definecolor{currentstroke}{rgb}{0.298039,0.447059,0.690196}%
\pgfsetstrokecolor{currentstroke}%
\pgfsetdash{}{0pt}%
\pgfpathmoveto{\pgfqpoint{0.897273in}{0.677992in}}%
\pgfpathlineto{\pgfqpoint{0.928900in}{0.675706in}}%
\pgfpathlineto{\pgfqpoint{0.960527in}{0.672830in}}%
\pgfpathlineto{\pgfqpoint{1.007967in}{0.667845in}}%
\pgfpathlineto{\pgfqpoint{1.087034in}{0.659411in}}%
\pgfpathlineto{\pgfqpoint{1.118661in}{0.656575in}}%
\pgfpathlineto{\pgfqpoint{1.150288in}{0.654345in}}%
\pgfpathlineto{\pgfqpoint{1.174008in}{0.653190in}}%
\pgfpathlineto{\pgfqpoint{1.197728in}{0.652572in}}%
\pgfpathlineto{\pgfqpoint{1.221448in}{0.652576in}}%
\pgfpathlineto{\pgfqpoint{1.245169in}{0.653285in}}%
\pgfpathlineto{\pgfqpoint{1.268889in}{0.654783in}}%
\pgfpathlineto{\pgfqpoint{1.284702in}{0.656263in}}%
\pgfpathlineto{\pgfqpoint{1.300516in}{0.658155in}}%
\pgfpathlineto{\pgfqpoint{1.316329in}{0.660486in}}%
\pgfpathlineto{\pgfqpoint{1.332142in}{0.663280in}}%
\pgfpathlineto{\pgfqpoint{1.347956in}{0.666561in}}%
\pgfpathlineto{\pgfqpoint{1.363769in}{0.670356in}}%
\pgfpathlineto{\pgfqpoint{1.379583in}{0.674688in}}%
\pgfpathlineto{\pgfqpoint{1.395396in}{0.679584in}}%
\pgfpathlineto{\pgfqpoint{1.411210in}{0.685067in}}%
\pgfpathlineto{\pgfqpoint{1.427023in}{0.691162in}}%
\pgfpathlineto{\pgfqpoint{1.442837in}{0.697895in}}%
\pgfpathlineto{\pgfqpoint{1.458650in}{0.705291in}}%
\pgfpathlineto{\pgfqpoint{1.474463in}{0.713373in}}%
\pgfpathlineto{\pgfqpoint{1.490277in}{0.722169in}}%
\pgfpathlineto{\pgfqpoint{1.506090in}{0.731701in}}%
\pgfpathlineto{\pgfqpoint{1.521904in}{0.741996in}}%
\pgfpathlineto{\pgfqpoint{1.537717in}{0.753077in}}%
\pgfpathlineto{\pgfqpoint{1.553531in}{0.764971in}}%
\pgfpathlineto{\pgfqpoint{1.569344in}{0.777701in}}%
\pgfpathlineto{\pgfqpoint{1.585158in}{0.791293in}}%
\pgfpathlineto{\pgfqpoint{1.600971in}{0.805773in}}%
\pgfpathlineto{\pgfqpoint{1.616784in}{0.821163in}}%
\pgfpathlineto{\pgfqpoint{1.632598in}{0.837491in}}%
\pgfpathlineto{\pgfqpoint{1.648411in}{0.854780in}}%
\pgfpathlineto{\pgfqpoint{1.664225in}{0.873055in}}%
\pgfpathlineto{\pgfqpoint{1.680038in}{0.892342in}}%
\pgfpathlineto{\pgfqpoint{1.695852in}{0.912665in}}%
\pgfpathlineto{\pgfqpoint{1.711665in}{0.934049in}}%
\pgfpathlineto{\pgfqpoint{1.727479in}{0.956519in}}%
\pgfpathlineto{\pgfqpoint{1.743292in}{0.980101in}}%
\pgfpathlineto{\pgfqpoint{1.759105in}{1.004818in}}%
\pgfpathlineto{\pgfqpoint{1.774919in}{1.030696in}}%
\pgfpathlineto{\pgfqpoint{1.790732in}{1.057760in}}%
\pgfpathlineto{\pgfqpoint{1.806546in}{1.086035in}}%
\pgfpathlineto{\pgfqpoint{1.822359in}{1.115546in}}%
\pgfpathlineto{\pgfqpoint{1.838173in}{1.146317in}}%
\pgfpathlineto{\pgfqpoint{1.853986in}{1.178374in}}%
\pgfpathlineto{\pgfqpoint{1.869800in}{1.211741in}}%
\pgfpathlineto{\pgfqpoint{1.885613in}{1.246443in}}%
\pgfpathlineto{\pgfqpoint{1.901426in}{1.282484in}}%
\pgfpathlineto{\pgfqpoint{1.917240in}{1.319804in}}%
\pgfpathlineto{\pgfqpoint{1.933053in}{1.358335in}}%
\pgfpathlineto{\pgfqpoint{1.948867in}{1.398009in}}%
\pgfpathlineto{\pgfqpoint{1.964680in}{1.438755in}}%
\pgfpathlineto{\pgfqpoint{1.988400in}{1.501737in}}%
\pgfpathlineto{\pgfqpoint{2.012121in}{1.566746in}}%
\pgfpathlineto{\pgfqpoint{2.035841in}{1.633550in}}%
\pgfpathlineto{\pgfqpoint{2.059561in}{1.701918in}}%
\pgfpathlineto{\pgfqpoint{2.091188in}{1.795104in}}%
\pgfpathlineto{\pgfqpoint{2.122815in}{1.890105in}}%
\pgfpathlineto{\pgfqpoint{2.170255in}{2.034806in}}%
\pgfpathlineto{\pgfqpoint{2.265136in}{2.325307in}}%
\pgfpathlineto{\pgfqpoint{2.296763in}{2.420448in}}%
\pgfpathlineto{\pgfqpoint{2.328390in}{2.513824in}}%
\pgfpathlineto{\pgfqpoint{2.352110in}{2.582367in}}%
\pgfpathlineto{\pgfqpoint{2.375830in}{2.649376in}}%
\pgfpathlineto{\pgfqpoint{2.399550in}{2.714610in}}%
\pgfpathlineto{\pgfqpoint{2.423270in}{2.777795in}}%
\pgfpathlineto{\pgfqpoint{2.439084in}{2.818640in}}%
\pgfpathlineto{\pgfqpoint{2.454897in}{2.858363in}}%
\pgfpathlineto{\pgfqpoint{2.470711in}{2.896880in}}%
\pgfpathlineto{\pgfqpoint{2.486524in}{2.934108in}}%
\pgfpathlineto{\pgfqpoint{2.502337in}{2.969961in}}%
\pgfpathlineto{\pgfqpoint{2.518151in}{3.004356in}}%
\pgfpathlineto{\pgfqpoint{2.533964in}{3.037208in}}%
\pgfpathlineto{\pgfqpoint{2.549778in}{3.068433in}}%
\pgfpathlineto{\pgfqpoint{2.565591in}{3.097946in}}%
\pgfpathlineto{\pgfqpoint{2.581405in}{3.125664in}}%
\pgfpathlineto{\pgfqpoint{2.597218in}{3.151502in}}%
\pgfpathlineto{\pgfqpoint{2.613032in}{3.175376in}}%
\pgfpathlineto{\pgfqpoint{2.620938in}{3.186550in}}%
\pgfpathlineto{\pgfqpoint{2.628845in}{3.197201in}}%
\pgfpathlineto{\pgfqpoint{2.636752in}{3.207319in}}%
\pgfpathlineto{\pgfqpoint{2.644658in}{3.216894in}}%
\pgfpathlineto{\pgfqpoint{2.652565in}{3.225914in}}%
\pgfpathlineto{\pgfqpoint{2.660472in}{3.234369in}}%
\pgfpathlineto{\pgfqpoint{2.668379in}{3.242249in}}%
\pgfpathlineto{\pgfqpoint{2.676285in}{3.249543in}}%
\pgfpathlineto{\pgfqpoint{2.684192in}{3.256241in}}%
\pgfpathlineto{\pgfqpoint{2.692099in}{3.262332in}}%
\pgfpathlineto{\pgfqpoint{2.700005in}{3.267805in}}%
\pgfpathlineto{\pgfqpoint{2.707912in}{3.272650in}}%
\pgfpathlineto{\pgfqpoint{2.715819in}{3.276857in}}%
\pgfpathlineto{\pgfqpoint{2.723726in}{3.280415in}}%
\pgfpathlineto{\pgfqpoint{2.731632in}{3.283313in}}%
\pgfpathlineto{\pgfqpoint{2.739539in}{3.285541in}}%
\pgfpathlineto{\pgfqpoint{2.747446in}{3.287088in}}%
\pgfpathlineto{\pgfqpoint{2.755353in}{3.287944in}}%
\pgfpathlineto{\pgfqpoint{2.763259in}{3.288099in}}%
\pgfpathlineto{\pgfqpoint{2.771166in}{3.287541in}}%
\pgfpathlineto{\pgfqpoint{2.779073in}{3.286260in}}%
\pgfpathlineto{\pgfqpoint{2.786979in}{3.284246in}}%
\pgfpathlineto{\pgfqpoint{2.794886in}{3.281488in}}%
\pgfpathlineto{\pgfqpoint{2.802793in}{3.277975in}}%
\pgfpathlineto{\pgfqpoint{2.810700in}{3.273698in}}%
\pgfpathlineto{\pgfqpoint{2.818606in}{3.268645in}}%
\pgfpathlineto{\pgfqpoint{2.826513in}{3.262806in}}%
\pgfpathlineto{\pgfqpoint{2.834420in}{3.256171in}}%
\pgfpathlineto{\pgfqpoint{2.842326in}{3.248728in}}%
\pgfpathlineto{\pgfqpoint{2.850233in}{3.240468in}}%
\pgfpathlineto{\pgfqpoint{2.858140in}{3.231379in}}%
\pgfpathlineto{\pgfqpoint{2.866047in}{3.221452in}}%
\pgfpathlineto{\pgfqpoint{2.873953in}{3.210677in}}%
\pgfpathlineto{\pgfqpoint{2.881860in}{3.199066in}}%
\pgfpathlineto{\pgfqpoint{2.889767in}{3.186657in}}%
\pgfpathlineto{\pgfqpoint{2.897674in}{3.173484in}}%
\pgfpathlineto{\pgfqpoint{2.905580in}{3.159586in}}%
\pgfpathlineto{\pgfqpoint{2.913487in}{3.144999in}}%
\pgfpathlineto{\pgfqpoint{2.921394in}{3.129760in}}%
\pgfpathlineto{\pgfqpoint{2.937207in}{3.097473in}}%
\pgfpathlineto{\pgfqpoint{2.953021in}{3.063019in}}%
\pgfpathlineto{\pgfqpoint{2.968834in}{3.026692in}}%
\pgfpathlineto{\pgfqpoint{2.984647in}{2.988788in}}%
\pgfpathlineto{\pgfqpoint{3.000461in}{2.949599in}}%
\pgfpathlineto{\pgfqpoint{3.024181in}{2.889054in}}%
\pgfpathlineto{\pgfqpoint{3.071621in}{2.765259in}}%
\pgfpathlineto{\pgfqpoint{3.095342in}{2.703997in}}%
\pgfpathlineto{\pgfqpoint{3.119062in}{2.644483in}}%
\pgfpathlineto{\pgfqpoint{3.134875in}{2.606270in}}%
\pgfpathlineto{\pgfqpoint{3.150689in}{2.569569in}}%
\pgfpathlineto{\pgfqpoint{3.166502in}{2.534675in}}%
\pgfpathlineto{\pgfqpoint{3.182316in}{2.501883in}}%
\pgfpathlineto{\pgfqpoint{3.190222in}{2.486368in}}%
\pgfpathlineto{\pgfqpoint{3.198129in}{2.471488in}}%
\pgfpathlineto{\pgfqpoint{3.206036in}{2.457281in}}%
\pgfpathlineto{\pgfqpoint{3.213942in}{2.443783in}}%
\pgfpathlineto{\pgfqpoint{3.221849in}{2.431031in}}%
\pgfpathlineto{\pgfqpoint{3.229756in}{2.419063in}}%
\pgfpathlineto{\pgfqpoint{3.237663in}{2.407914in}}%
\pgfpathlineto{\pgfqpoint{3.245569in}{2.397622in}}%
\pgfpathlineto{\pgfqpoint{3.253476in}{2.388224in}}%
\pgfpathlineto{\pgfqpoint{3.261383in}{2.379755in}}%
\pgfpathlineto{\pgfqpoint{3.269289in}{2.372254in}}%
\pgfpathlineto{\pgfqpoint{3.277196in}{2.365757in}}%
\pgfpathlineto{\pgfqpoint{3.285103in}{2.360300in}}%
\pgfpathlineto{\pgfqpoint{3.293010in}{2.355920in}}%
\pgfpathlineto{\pgfqpoint{3.300916in}{2.352655in}}%
\pgfpathlineto{\pgfqpoint{3.308823in}{2.350541in}}%
\pgfpathlineto{\pgfqpoint{3.316730in}{2.349614in}}%
\pgfpathlineto{\pgfqpoint{3.324637in}{2.349913in}}%
\pgfpathlineto{\pgfqpoint{3.332543in}{2.351472in}}%
\pgfpathlineto{\pgfqpoint{3.340450in}{2.354330in}}%
\pgfpathlineto{\pgfqpoint{3.348357in}{2.358523in}}%
\pgfpathlineto{\pgfqpoint{3.356263in}{2.364088in}}%
\pgfpathlineto{\pgfqpoint{3.364170in}{2.371061in}}%
\pgfpathlineto{\pgfqpoint{3.372077in}{2.379460in}}%
\pgfpathlineto{\pgfqpoint{3.379984in}{2.389248in}}%
\pgfpathlineto{\pgfqpoint{3.387890in}{2.400380in}}%
\pgfpathlineto{\pgfqpoint{3.395797in}{2.412808in}}%
\pgfpathlineto{\pgfqpoint{3.403704in}{2.426487in}}%
\pgfpathlineto{\pgfqpoint{3.411610in}{2.441371in}}%
\pgfpathlineto{\pgfqpoint{3.419517in}{2.457414in}}%
\pgfpathlineto{\pgfqpoint{3.427424in}{2.474571in}}%
\pgfpathlineto{\pgfqpoint{3.435331in}{2.492794in}}%
\pgfpathlineto{\pgfqpoint{3.443237in}{2.512038in}}%
\pgfpathlineto{\pgfqpoint{3.451144in}{2.532257in}}%
\pgfpathlineto{\pgfqpoint{3.459051in}{2.553404in}}%
\pgfpathlineto{\pgfqpoint{3.466958in}{2.575435in}}%
\pgfpathlineto{\pgfqpoint{3.482771in}{2.621961in}}%
\pgfpathlineto{\pgfqpoint{3.498584in}{2.671466in}}%
\pgfpathlineto{\pgfqpoint{3.514398in}{2.723583in}}%
\pgfpathlineto{\pgfqpoint{3.530211in}{2.777942in}}%
\pgfpathlineto{\pgfqpoint{3.546025in}{2.834177in}}%
\pgfpathlineto{\pgfqpoint{3.569745in}{2.921239in}}%
\pgfpathlineto{\pgfqpoint{3.601372in}{3.040448in}}%
\pgfpathlineto{\pgfqpoint{3.648812in}{3.219980in}}%
\pgfpathlineto{\pgfqpoint{3.672532in}{3.307378in}}%
\pgfpathlineto{\pgfqpoint{3.688346in}{3.363921in}}%
\pgfpathlineto{\pgfqpoint{3.704159in}{3.418659in}}%
\pgfpathlineto{\pgfqpoint{3.719973in}{3.471222in}}%
\pgfpathlineto{\pgfqpoint{3.735786in}{3.521243in}}%
\pgfpathlineto{\pgfqpoint{3.751600in}{3.568353in}}%
\pgfpathlineto{\pgfqpoint{3.759506in}{3.590702in}}%
\pgfpathlineto{\pgfqpoint{3.767413in}{3.612185in}}%
\pgfpathlineto{\pgfqpoint{3.775320in}{3.632756in}}%
\pgfpathlineto{\pgfqpoint{3.783226in}{3.652369in}}%
\pgfpathlineto{\pgfqpoint{3.791133in}{3.670979in}}%
\pgfpathlineto{\pgfqpoint{3.799040in}{3.688539in}}%
\pgfpathlineto{\pgfqpoint{3.806947in}{3.705003in}}%
\pgfpathlineto{\pgfqpoint{3.814853in}{3.720325in}}%
\pgfpathlineto{\pgfqpoint{3.822760in}{3.734460in}}%
\pgfpathlineto{\pgfqpoint{3.830667in}{3.747360in}}%
\pgfpathlineto{\pgfqpoint{3.838574in}{3.758981in}}%
\pgfpathlineto{\pgfqpoint{3.846480in}{3.769276in}}%
\pgfpathlineto{\pgfqpoint{3.854387in}{3.778200in}}%
\pgfpathlineto{\pgfqpoint{3.862294in}{3.785710in}}%
\pgfpathlineto{\pgfqpoint{3.870200in}{3.791807in}}%
\pgfpathlineto{\pgfqpoint{3.878107in}{3.796513in}}%
\pgfpathlineto{\pgfqpoint{3.886014in}{3.799849in}}%
\pgfpathlineto{\pgfqpoint{3.893921in}{3.801838in}}%
\pgfpathlineto{\pgfqpoint{3.901827in}{3.802500in}}%
\pgfpathlineto{\pgfqpoint{3.909734in}{3.801858in}}%
\pgfpathlineto{\pgfqpoint{3.917641in}{3.799934in}}%
\pgfpathlineto{\pgfqpoint{3.925547in}{3.796749in}}%
\pgfpathlineto{\pgfqpoint{3.933454in}{3.792324in}}%
\pgfpathlineto{\pgfqpoint{3.941361in}{3.786683in}}%
\pgfpathlineto{\pgfqpoint{3.949268in}{3.779846in}}%
\pgfpathlineto{\pgfqpoint{3.957174in}{3.771835in}}%
\pgfpathlineto{\pgfqpoint{3.965081in}{3.762672in}}%
\pgfpathlineto{\pgfqpoint{3.972988in}{3.752379in}}%
\pgfpathlineto{\pgfqpoint{3.980895in}{3.740977in}}%
\pgfpathlineto{\pgfqpoint{3.988801in}{3.728489in}}%
\pgfpathlineto{\pgfqpoint{3.996708in}{3.714936in}}%
\pgfpathlineto{\pgfqpoint{4.004615in}{3.700339in}}%
\pgfpathlineto{\pgfqpoint{4.012521in}{3.684721in}}%
\pgfpathlineto{\pgfqpoint{4.020428in}{3.668103in}}%
\pgfpathlineto{\pgfqpoint{4.028335in}{3.650507in}}%
\pgfpathlineto{\pgfqpoint{4.036242in}{3.631955in}}%
\pgfpathlineto{\pgfqpoint{4.044148in}{3.612469in}}%
\pgfpathlineto{\pgfqpoint{4.052055in}{3.592069in}}%
\pgfpathlineto{\pgfqpoint{4.059962in}{3.570779in}}%
\pgfpathlineto{\pgfqpoint{4.067868in}{3.548619in}}%
\pgfpathlineto{\pgfqpoint{4.083682in}{3.501780in}}%
\pgfpathlineto{\pgfqpoint{4.099495in}{3.451724in}}%
\pgfpathlineto{\pgfqpoint{4.115309in}{3.398627in}}%
\pgfpathlineto{\pgfqpoint{4.131122in}{3.342662in}}%
\pgfpathlineto{\pgfqpoint{4.146936in}{3.284003in}}%
\pgfpathlineto{\pgfqpoint{4.162749in}{3.222825in}}%
\pgfpathlineto{\pgfqpoint{4.178563in}{3.159300in}}%
\pgfpathlineto{\pgfqpoint{4.194376in}{3.093604in}}%
\pgfpathlineto{\pgfqpoint{4.210189in}{3.025911in}}%
\pgfpathlineto{\pgfqpoint{4.226003in}{2.956394in}}%
\pgfpathlineto{\pgfqpoint{4.249723in}{2.849080in}}%
\pgfpathlineto{\pgfqpoint{4.273443in}{2.738641in}}%
\pgfpathlineto{\pgfqpoint{4.297163in}{2.625666in}}%
\pgfpathlineto{\pgfqpoint{4.328790in}{2.472101in}}%
\pgfpathlineto{\pgfqpoint{4.376231in}{2.238302in}}%
\pgfpathlineto{\pgfqpoint{4.431578in}{1.965908in}}%
\pgfpathlineto{\pgfqpoint{4.463205in}{1.813147in}}%
\pgfpathlineto{\pgfqpoint{4.486925in}{1.701022in}}%
\pgfpathlineto{\pgfqpoint{4.510645in}{1.591646in}}%
\pgfpathlineto{\pgfqpoint{4.534365in}{1.485608in}}%
\pgfpathlineto{\pgfqpoint{4.550179in}{1.417059in}}%
\pgfpathlineto{\pgfqpoint{4.565992in}{1.350428in}}%
\pgfpathlineto{\pgfqpoint{4.581805in}{1.285890in}}%
\pgfpathlineto{\pgfqpoint{4.597619in}{1.223617in}}%
\pgfpathlineto{\pgfqpoint{4.613432in}{1.163785in}}%
\pgfpathlineto{\pgfqpoint{4.629246in}{1.106567in}}%
\pgfpathlineto{\pgfqpoint{4.645059in}{1.052137in}}%
\pgfpathlineto{\pgfqpoint{4.660873in}{1.000669in}}%
\pgfpathlineto{\pgfqpoint{4.676686in}{0.952338in}}%
\pgfpathlineto{\pgfqpoint{4.692500in}{0.907317in}}%
\pgfpathlineto{\pgfqpoint{4.700406in}{0.886102in}}%
\pgfpathlineto{\pgfqpoint{4.708313in}{0.865780in}}%
\pgfpathlineto{\pgfqpoint{4.716220in}{0.846373in}}%
\pgfpathlineto{\pgfqpoint{4.724126in}{0.827902in}}%
\pgfpathlineto{\pgfqpoint{4.732033in}{0.810389in}}%
\pgfpathlineto{\pgfqpoint{4.739940in}{0.793856in}}%
\pgfpathlineto{\pgfqpoint{4.747847in}{0.778325in}}%
\pgfpathlineto{\pgfqpoint{4.755753in}{0.763817in}}%
\pgfpathlineto{\pgfqpoint{4.763660in}{0.750354in}}%
\pgfpathlineto{\pgfqpoint{4.771567in}{0.737958in}}%
\pgfpathlineto{\pgfqpoint{4.779473in}{0.726650in}}%
\pgfpathlineto{\pgfqpoint{4.787380in}{0.716453in}}%
\pgfpathlineto{\pgfqpoint{4.795287in}{0.707388in}}%
\pgfpathlineto{\pgfqpoint{4.803194in}{0.699477in}}%
\pgfpathlineto{\pgfqpoint{4.811100in}{0.692742in}}%
\pgfpathlineto{\pgfqpoint{4.819007in}{0.687204in}}%
\pgfpathlineto{\pgfqpoint{4.826914in}{0.682885in}}%
\pgfpathlineto{\pgfqpoint{4.834821in}{0.679807in}}%
\pgfpathlineto{\pgfqpoint{4.842727in}{0.677992in}}%
\pgfpathlineto{\pgfqpoint{4.842727in}{0.677992in}}%
\pgfusepath{stroke}%
\end{pgfscope}%
\begin{pgfscope}%
\pgfpathrectangle{\pgfqpoint{0.700000in}{0.495000in}}{\pgfqpoint{4.340000in}{3.465000in}}%
\pgfusepath{clip}%
\pgfsetroundcap%
\pgfsetroundjoin%
\pgfsetlinewidth{2.007500pt}%
\definecolor{currentstroke}{rgb}{1.000000,0.647059,0.000000}%
\pgfsetstrokecolor{currentstroke}%
\pgfsetdash{}{0pt}%
\pgfpathmoveto{\pgfqpoint{0.700000in}{2.088091in}}%
\pgfpathlineto{\pgfqpoint{5.040000in}{2.088091in}}%
\pgfusepath{stroke}%
\end{pgfscope}%
\begin{pgfscope}%
\pgfsetrectcap%
\pgfsetmiterjoin%
\pgfsetlinewidth{1.254687pt}%
\definecolor{currentstroke}{rgb}{1.000000,1.000000,1.000000}%
\pgfsetstrokecolor{currentstroke}%
\pgfsetdash{}{0pt}%
\pgfpathmoveto{\pgfqpoint{0.700000in}{0.495000in}}%
\pgfpathlineto{\pgfqpoint{0.700000in}{3.960000in}}%
\pgfusepath{stroke}%
\end{pgfscope}%
\begin{pgfscope}%
\pgfsetrectcap%
\pgfsetmiterjoin%
\pgfsetlinewidth{1.254687pt}%
\definecolor{currentstroke}{rgb}{1.000000,1.000000,1.000000}%
\pgfsetstrokecolor{currentstroke}%
\pgfsetdash{}{0pt}%
\pgfpathmoveto{\pgfqpoint{5.040000in}{0.495000in}}%
\pgfpathlineto{\pgfqpoint{5.040000in}{3.960000in}}%
\pgfusepath{stroke}%
\end{pgfscope}%
\begin{pgfscope}%
\pgfsetrectcap%
\pgfsetmiterjoin%
\pgfsetlinewidth{1.254687pt}%
\definecolor{currentstroke}{rgb}{1.000000,1.000000,1.000000}%
\pgfsetstrokecolor{currentstroke}%
\pgfsetdash{}{0pt}%
\pgfpathmoveto{\pgfqpoint{0.700000in}{0.495000in}}%
\pgfpathlineto{\pgfqpoint{5.040000in}{0.495000in}}%
\pgfusepath{stroke}%
\end{pgfscope}%
\begin{pgfscope}%
\pgfsetrectcap%
\pgfsetmiterjoin%
\pgfsetlinewidth{1.254687pt}%
\definecolor{currentstroke}{rgb}{1.000000,1.000000,1.000000}%
\pgfsetstrokecolor{currentstroke}%
\pgfsetdash{}{0pt}%
\pgfpathmoveto{\pgfqpoint{0.700000in}{3.960000in}}%
\pgfpathlineto{\pgfqpoint{5.040000in}{3.960000in}}%
\pgfusepath{stroke}%
\end{pgfscope}%
\begin{pgfscope}%
\pgfsetbuttcap%
\pgfsetmiterjoin%
\definecolor{currentfill}{rgb}{0.917647,0.917647,0.949020}%
\pgfsetfillcolor{currentfill}%
\pgfsetfillopacity{0.800000}%
\pgfsetlinewidth{1.003750pt}%
\definecolor{currentstroke}{rgb}{0.800000,0.800000,0.800000}%
\pgfsetstrokecolor{currentstroke}%
\pgfsetstrokeopacity{0.800000}%
\pgfsetdash{}{0pt}%
\pgfpathmoveto{\pgfqpoint{0.806944in}{2.967960in}}%
\pgfpathlineto{\pgfqpoint{2.501868in}{2.967960in}}%
\pgfpathquadraticcurveto{\pgfqpoint{2.532423in}{2.967960in}}{\pgfqpoint{2.532423in}{2.998515in}}%
\pgfpathlineto{\pgfqpoint{2.532423in}{3.853056in}}%
\pgfpathquadraticcurveto{\pgfqpoint{2.532423in}{3.883611in}}{\pgfqpoint{2.501868in}{3.883611in}}%
\pgfpathlineto{\pgfqpoint{0.806944in}{3.883611in}}%
\pgfpathquadraticcurveto{\pgfqpoint{0.776389in}{3.883611in}}{\pgfqpoint{0.776389in}{3.853056in}}%
\pgfpathlineto{\pgfqpoint{0.776389in}{2.998515in}}%
\pgfpathquadraticcurveto{\pgfqpoint{0.776389in}{2.967960in}}{\pgfqpoint{0.806944in}{2.967960in}}%
\pgfpathlineto{\pgfqpoint{0.806944in}{2.967960in}}%
\pgfpathclose%
\pgfusepath{stroke,fill}%
\end{pgfscope}%
\begin{pgfscope}%
\pgfsetroundcap%
\pgfsetroundjoin%
\pgfsetlinewidth{1.505625pt}%
\definecolor{currentstroke}{rgb}{0.298039,0.447059,0.690196}%
\pgfsetstrokecolor{currentstroke}%
\pgfsetdash{}{0pt}%
\pgfpathmoveto{\pgfqpoint{0.837500in}{3.766611in}}%
\pgfpathlineto{\pgfqpoint{0.990278in}{3.766611in}}%
\pgfpathlineto{\pgfqpoint{1.143056in}{3.766611in}}%
\pgfusepath{stroke}%
\end{pgfscope}%
\begin{pgfscope}%
\definecolor{textcolor}{rgb}{0.150000,0.150000,0.150000}%
\pgfsetstrokecolor{textcolor}%
\pgfsetfillcolor{textcolor}%
\pgftext[x=1.265278in,y=3.713139in,left,base]{\color{textcolor}{\sffamily\fontsize{11.000000}{13.200000}\selectfont\catcode`\^=\active\def^{\ifmmode\sp\else\^{}\fi}\catcode`\%=\active\def%{\%}resource demand}}%
\end{pgfscope}%
\begin{pgfscope}%
\pgfsetroundcap%
\pgfsetroundjoin%
\pgfsetlinewidth{2.007500pt}%
\definecolor{currentstroke}{rgb}{1.000000,0.647059,0.000000}%
\pgfsetstrokecolor{currentstroke}%
\pgfsetdash{}{0pt}%
\pgfpathmoveto{\pgfqpoint{0.837500in}{3.550499in}}%
\pgfpathlineto{\pgfqpoint{0.990278in}{3.550499in}}%
\pgfpathlineto{\pgfqpoint{1.143056in}{3.550499in}}%
\pgfusepath{stroke}%
\end{pgfscope}%
\begin{pgfscope}%
\definecolor{textcolor}{rgb}{0.150000,0.150000,0.150000}%
\pgfsetstrokecolor{textcolor}%
\pgfsetfillcolor{textcolor}%
\pgftext[x=1.265278in,y=3.497027in,left,base]{\color{textcolor}{\sffamily\fontsize{11.000000}{13.200000}\selectfont\catcode`\^=\active\def^{\ifmmode\sp\else\^{}\fi}\catcode`\%=\active\def%{\%}resource supply}}%
\end{pgfscope}%
\begin{pgfscope}%
\pgfsetbuttcap%
\pgfsetmiterjoin%
\definecolor{currentfill}{rgb}{0.172549,0.627451,0.172549}%
\pgfsetfillcolor{currentfill}%
\pgfsetfillopacity{0.300000}%
\pgfsetlinewidth{1.003750pt}%
\definecolor{currentstroke}{rgb}{0.172549,0.627451,0.172549}%
\pgfsetstrokecolor{currentstroke}%
\pgfsetstrokeopacity{0.300000}%
\pgfsetdash{}{0pt}%
\pgfpathmoveto{\pgfqpoint{0.837500in}{3.279125in}}%
\pgfpathlineto{\pgfqpoint{1.143056in}{3.279125in}}%
\pgfpathlineto{\pgfqpoint{1.143056in}{3.386069in}}%
\pgfpathlineto{\pgfqpoint{0.837500in}{3.386069in}}%
\pgfpathlineto{\pgfqpoint{0.837500in}{3.279125in}}%
\pgfpathclose%
\pgfusepath{stroke,fill}%
\end{pgfscope}%
\begin{pgfscope}%
\definecolor{textcolor}{rgb}{0.150000,0.150000,0.150000}%
\pgfsetstrokecolor{textcolor}%
\pgfsetfillcolor{textcolor}%
\pgftext[x=1.265278in,y=3.279125in,left,base]{\color{textcolor}{\sffamily\fontsize{11.000000}{13.200000}\selectfont\catcode`\^=\active\def^{\ifmmode\sp\else\^{}\fi}\catcode`\%=\active\def%{\%}overprovisioning}}%
\end{pgfscope}%
\begin{pgfscope}%
\pgfsetbuttcap%
\pgfsetmiterjoin%
\definecolor{currentfill}{rgb}{0.839216,0.152941,0.156863}%
\pgfsetfillcolor{currentfill}%
\pgfsetfillopacity{0.300000}%
\pgfsetlinewidth{1.003750pt}%
\definecolor{currentstroke}{rgb}{0.839216,0.152941,0.156863}%
\pgfsetstrokecolor{currentstroke}%
\pgfsetstrokeopacity{0.300000}%
\pgfsetdash{}{0pt}%
\pgfpathmoveto{\pgfqpoint{0.837500in}{3.061223in}}%
\pgfpathlineto{\pgfqpoint{1.143056in}{3.061223in}}%
\pgfpathlineto{\pgfqpoint{1.143056in}{3.168167in}}%
\pgfpathlineto{\pgfqpoint{0.837500in}{3.168167in}}%
\pgfpathlineto{\pgfqpoint{0.837500in}{3.061223in}}%
\pgfpathclose%
\pgfusepath{stroke,fill}%
\end{pgfscope}%
\begin{pgfscope}%
\definecolor{textcolor}{rgb}{0.150000,0.150000,0.150000}%
\pgfsetstrokecolor{textcolor}%
\pgfsetfillcolor{textcolor}%
\pgftext[x=1.265278in,y=3.061223in,left,base]{\color{textcolor}{\sffamily\fontsize{11.000000}{13.200000}\selectfont\catcode`\^=\active\def^{\ifmmode\sp\else\^{}\fi}\catcode`\%=\active\def%{\%}underprovisioning}}%
\end{pgfscope}%
\end{pgfpicture}%
\makeatother%
\endgroup%

    \caption{Resource demand and supply for a website during a typical day with no elastic processes.}
    \label{fig:elasticity-application-no-scaling}
\end{figure}

If the concept of elasticity is applied to this example, resources can be released during the night (so called \textit{scale-in}) and more resources can be claimed as they are needed during the day (so called \textit{scale-out}). This is illustrated in \cref{fig:elasticity-application-scaling}.

\begin{figure}
    \centering
    %% Creator: Matplotlib, PGF backend
%%
%% To include the figure in your LaTeX document, write
%%   \input{<filename>.pgf}
%%
%% Make sure the required packages are loaded in your preamble
%%   \usepackage{pgf}
%%
%% Also ensure that all the required font packages are loaded; for instance,
%% the lmodern package is sometimes necessary when using math font.
%%   \usepackage{lmodern}
%%
%% Figures using additional raster images can only be included by \input if
%% they are in the same directory as the main LaTeX file. For loading figures
%% from other directories you can use the `import` package
%%   \usepackage{import}
%%
%% and then include the figures with
%%   \import{<path to file>}{<filename>.pgf}
%%
%% Matplotlib used the following preamble
%%   \def\mathdefault#1{#1}
%%   \everymath=\expandafter{\the\everymath\displaystyle}
%%   
%%   \usepackage{fontspec}
%%   \setmainfont{DejaVuSerif.ttf}[Path=\detokenize{/Users/nkratky/private/polaris-elasticity-strategies/test/scripts/.venv/lib/python3.11/site-packages/matplotlib/mpl-data/fonts/ttf/}]
%%   \setsansfont{Arial.ttf}[Path=\detokenize{/System/Library/Fonts/Supplemental/}]
%%   \setmonofont{DejaVuSansMono.ttf}[Path=\detokenize{/Users/nkratky/private/polaris-elasticity-strategies/test/scripts/.venv/lib/python3.11/site-packages/matplotlib/mpl-data/fonts/ttf/}]
%%   \makeatletter\@ifpackageloaded{underscore}{}{\usepackage[strings]{underscore}}\makeatother
%%
\begingroup%
\makeatletter%
\begin{pgfpicture}%
\pgfpathrectangle{\pgfpointorigin}{\pgfqpoint{5.600000in}{4.500000in}}%
\pgfusepath{use as bounding box, clip}%
\begin{pgfscope}%
\pgfsetbuttcap%
\pgfsetmiterjoin%
\definecolor{currentfill}{rgb}{1.000000,1.000000,1.000000}%
\pgfsetfillcolor{currentfill}%
\pgfsetlinewidth{0.000000pt}%
\definecolor{currentstroke}{rgb}{1.000000,1.000000,1.000000}%
\pgfsetstrokecolor{currentstroke}%
\pgfsetdash{}{0pt}%
\pgfpathmoveto{\pgfqpoint{0.000000in}{0.000000in}}%
\pgfpathlineto{\pgfqpoint{5.600000in}{0.000000in}}%
\pgfpathlineto{\pgfqpoint{5.600000in}{4.500000in}}%
\pgfpathlineto{\pgfqpoint{0.000000in}{4.500000in}}%
\pgfpathlineto{\pgfqpoint{0.000000in}{0.000000in}}%
\pgfpathclose%
\pgfusepath{fill}%
\end{pgfscope}%
\begin{pgfscope}%
\pgfsetbuttcap%
\pgfsetmiterjoin%
\definecolor{currentfill}{rgb}{0.917647,0.917647,0.949020}%
\pgfsetfillcolor{currentfill}%
\pgfsetlinewidth{0.000000pt}%
\definecolor{currentstroke}{rgb}{0.000000,0.000000,0.000000}%
\pgfsetstrokecolor{currentstroke}%
\pgfsetstrokeopacity{0.000000}%
\pgfsetdash{}{0pt}%
\pgfpathmoveto{\pgfqpoint{0.700000in}{0.495000in}}%
\pgfpathlineto{\pgfqpoint{5.040000in}{0.495000in}}%
\pgfpathlineto{\pgfqpoint{5.040000in}{3.960000in}}%
\pgfpathlineto{\pgfqpoint{0.700000in}{3.960000in}}%
\pgfpathlineto{\pgfqpoint{0.700000in}{0.495000in}}%
\pgfpathclose%
\pgfusepath{fill}%
\end{pgfscope}%
\begin{pgfscope}%
\pgfpathrectangle{\pgfqpoint{0.700000in}{0.495000in}}{\pgfqpoint{4.340000in}{3.465000in}}%
\pgfusepath{clip}%
\pgfsetroundcap%
\pgfsetroundjoin%
\pgfsetlinewidth{1.003750pt}%
\definecolor{currentstroke}{rgb}{1.000000,1.000000,1.000000}%
\pgfsetstrokecolor{currentstroke}%
\pgfsetdash{}{0pt}%
\pgfpathmoveto{\pgfqpoint{0.897273in}{0.495000in}}%
\pgfpathlineto{\pgfqpoint{0.897273in}{3.960000in}}%
\pgfusepath{stroke}%
\end{pgfscope}%
\begin{pgfscope}%
\definecolor{textcolor}{rgb}{0.150000,0.150000,0.150000}%
\pgfsetstrokecolor{textcolor}%
\pgfsetfillcolor{textcolor}%
\pgftext[x=0.897273in,y=0.363056in,,top]{\color{textcolor}{\sffamily\fontsize{11.000000}{13.200000}\selectfont\catcode`\^=\active\def^{\ifmmode\sp\else\^{}\fi}\catcode`\%=\active\def%{\%}00:00}}%
\end{pgfscope}%
\begin{pgfscope}%
\pgfpathrectangle{\pgfqpoint{0.700000in}{0.495000in}}{\pgfqpoint{4.340000in}{3.465000in}}%
\pgfusepath{clip}%
\pgfsetroundcap%
\pgfsetroundjoin%
\pgfsetlinewidth{1.003750pt}%
\definecolor{currentstroke}{rgb}{1.000000,1.000000,1.000000}%
\pgfsetstrokecolor{currentstroke}%
\pgfsetdash{}{0pt}%
\pgfpathmoveto{\pgfqpoint{1.390455in}{0.495000in}}%
\pgfpathlineto{\pgfqpoint{1.390455in}{3.960000in}}%
\pgfusepath{stroke}%
\end{pgfscope}%
\begin{pgfscope}%
\definecolor{textcolor}{rgb}{0.150000,0.150000,0.150000}%
\pgfsetstrokecolor{textcolor}%
\pgfsetfillcolor{textcolor}%
\pgftext[x=1.390455in,y=0.363056in,,top]{\color{textcolor}{\sffamily\fontsize{11.000000}{13.200000}\selectfont\catcode`\^=\active\def^{\ifmmode\sp\else\^{}\fi}\catcode`\%=\active\def%{\%}03:00}}%
\end{pgfscope}%
\begin{pgfscope}%
\pgfpathrectangle{\pgfqpoint{0.700000in}{0.495000in}}{\pgfqpoint{4.340000in}{3.465000in}}%
\pgfusepath{clip}%
\pgfsetroundcap%
\pgfsetroundjoin%
\pgfsetlinewidth{1.003750pt}%
\definecolor{currentstroke}{rgb}{1.000000,1.000000,1.000000}%
\pgfsetstrokecolor{currentstroke}%
\pgfsetdash{}{0pt}%
\pgfpathmoveto{\pgfqpoint{1.883636in}{0.495000in}}%
\pgfpathlineto{\pgfqpoint{1.883636in}{3.960000in}}%
\pgfusepath{stroke}%
\end{pgfscope}%
\begin{pgfscope}%
\definecolor{textcolor}{rgb}{0.150000,0.150000,0.150000}%
\pgfsetstrokecolor{textcolor}%
\pgfsetfillcolor{textcolor}%
\pgftext[x=1.883636in,y=0.363056in,,top]{\color{textcolor}{\sffamily\fontsize{11.000000}{13.200000}\selectfont\catcode`\^=\active\def^{\ifmmode\sp\else\^{}\fi}\catcode`\%=\active\def%{\%}06:00}}%
\end{pgfscope}%
\begin{pgfscope}%
\pgfpathrectangle{\pgfqpoint{0.700000in}{0.495000in}}{\pgfqpoint{4.340000in}{3.465000in}}%
\pgfusepath{clip}%
\pgfsetroundcap%
\pgfsetroundjoin%
\pgfsetlinewidth{1.003750pt}%
\definecolor{currentstroke}{rgb}{1.000000,1.000000,1.000000}%
\pgfsetstrokecolor{currentstroke}%
\pgfsetdash{}{0pt}%
\pgfpathmoveto{\pgfqpoint{2.376818in}{0.495000in}}%
\pgfpathlineto{\pgfqpoint{2.376818in}{3.960000in}}%
\pgfusepath{stroke}%
\end{pgfscope}%
\begin{pgfscope}%
\definecolor{textcolor}{rgb}{0.150000,0.150000,0.150000}%
\pgfsetstrokecolor{textcolor}%
\pgfsetfillcolor{textcolor}%
\pgftext[x=2.376818in,y=0.363056in,,top]{\color{textcolor}{\sffamily\fontsize{11.000000}{13.200000}\selectfont\catcode`\^=\active\def^{\ifmmode\sp\else\^{}\fi}\catcode`\%=\active\def%{\%}09:00}}%
\end{pgfscope}%
\begin{pgfscope}%
\pgfpathrectangle{\pgfqpoint{0.700000in}{0.495000in}}{\pgfqpoint{4.340000in}{3.465000in}}%
\pgfusepath{clip}%
\pgfsetroundcap%
\pgfsetroundjoin%
\pgfsetlinewidth{1.003750pt}%
\definecolor{currentstroke}{rgb}{1.000000,1.000000,1.000000}%
\pgfsetstrokecolor{currentstroke}%
\pgfsetdash{}{0pt}%
\pgfpathmoveto{\pgfqpoint{2.870000in}{0.495000in}}%
\pgfpathlineto{\pgfqpoint{2.870000in}{3.960000in}}%
\pgfusepath{stroke}%
\end{pgfscope}%
\begin{pgfscope}%
\definecolor{textcolor}{rgb}{0.150000,0.150000,0.150000}%
\pgfsetstrokecolor{textcolor}%
\pgfsetfillcolor{textcolor}%
\pgftext[x=2.870000in,y=0.363056in,,top]{\color{textcolor}{\sffamily\fontsize{11.000000}{13.200000}\selectfont\catcode`\^=\active\def^{\ifmmode\sp\else\^{}\fi}\catcode`\%=\active\def%{\%}12:00}}%
\end{pgfscope}%
\begin{pgfscope}%
\pgfpathrectangle{\pgfqpoint{0.700000in}{0.495000in}}{\pgfqpoint{4.340000in}{3.465000in}}%
\pgfusepath{clip}%
\pgfsetroundcap%
\pgfsetroundjoin%
\pgfsetlinewidth{1.003750pt}%
\definecolor{currentstroke}{rgb}{1.000000,1.000000,1.000000}%
\pgfsetstrokecolor{currentstroke}%
\pgfsetdash{}{0pt}%
\pgfpathmoveto{\pgfqpoint{3.363182in}{0.495000in}}%
\pgfpathlineto{\pgfqpoint{3.363182in}{3.960000in}}%
\pgfusepath{stroke}%
\end{pgfscope}%
\begin{pgfscope}%
\definecolor{textcolor}{rgb}{0.150000,0.150000,0.150000}%
\pgfsetstrokecolor{textcolor}%
\pgfsetfillcolor{textcolor}%
\pgftext[x=3.363182in,y=0.363056in,,top]{\color{textcolor}{\sffamily\fontsize{11.000000}{13.200000}\selectfont\catcode`\^=\active\def^{\ifmmode\sp\else\^{}\fi}\catcode`\%=\active\def%{\%}15:00}}%
\end{pgfscope}%
\begin{pgfscope}%
\pgfpathrectangle{\pgfqpoint{0.700000in}{0.495000in}}{\pgfqpoint{4.340000in}{3.465000in}}%
\pgfusepath{clip}%
\pgfsetroundcap%
\pgfsetroundjoin%
\pgfsetlinewidth{1.003750pt}%
\definecolor{currentstroke}{rgb}{1.000000,1.000000,1.000000}%
\pgfsetstrokecolor{currentstroke}%
\pgfsetdash{}{0pt}%
\pgfpathmoveto{\pgfqpoint{3.856364in}{0.495000in}}%
\pgfpathlineto{\pgfqpoint{3.856364in}{3.960000in}}%
\pgfusepath{stroke}%
\end{pgfscope}%
\begin{pgfscope}%
\definecolor{textcolor}{rgb}{0.150000,0.150000,0.150000}%
\pgfsetstrokecolor{textcolor}%
\pgfsetfillcolor{textcolor}%
\pgftext[x=3.856364in,y=0.363056in,,top]{\color{textcolor}{\sffamily\fontsize{11.000000}{13.200000}\selectfont\catcode`\^=\active\def^{\ifmmode\sp\else\^{}\fi}\catcode`\%=\active\def%{\%}18:00}}%
\end{pgfscope}%
\begin{pgfscope}%
\pgfpathrectangle{\pgfqpoint{0.700000in}{0.495000in}}{\pgfqpoint{4.340000in}{3.465000in}}%
\pgfusepath{clip}%
\pgfsetroundcap%
\pgfsetroundjoin%
\pgfsetlinewidth{1.003750pt}%
\definecolor{currentstroke}{rgb}{1.000000,1.000000,1.000000}%
\pgfsetstrokecolor{currentstroke}%
\pgfsetdash{}{0pt}%
\pgfpathmoveto{\pgfqpoint{4.349545in}{0.495000in}}%
\pgfpathlineto{\pgfqpoint{4.349545in}{3.960000in}}%
\pgfusepath{stroke}%
\end{pgfscope}%
\begin{pgfscope}%
\definecolor{textcolor}{rgb}{0.150000,0.150000,0.150000}%
\pgfsetstrokecolor{textcolor}%
\pgfsetfillcolor{textcolor}%
\pgftext[x=4.349545in,y=0.363056in,,top]{\color{textcolor}{\sffamily\fontsize{11.000000}{13.200000}\selectfont\catcode`\^=\active\def^{\ifmmode\sp\else\^{}\fi}\catcode`\%=\active\def%{\%}21:00}}%
\end{pgfscope}%
\begin{pgfscope}%
\pgfpathrectangle{\pgfqpoint{0.700000in}{0.495000in}}{\pgfqpoint{4.340000in}{3.465000in}}%
\pgfusepath{clip}%
\pgfsetroundcap%
\pgfsetroundjoin%
\pgfsetlinewidth{1.003750pt}%
\definecolor{currentstroke}{rgb}{1.000000,1.000000,1.000000}%
\pgfsetstrokecolor{currentstroke}%
\pgfsetdash{}{0pt}%
\pgfpathmoveto{\pgfqpoint{4.842727in}{0.495000in}}%
\pgfpathlineto{\pgfqpoint{4.842727in}{3.960000in}}%
\pgfusepath{stroke}%
\end{pgfscope}%
\begin{pgfscope}%
\definecolor{textcolor}{rgb}{0.150000,0.150000,0.150000}%
\pgfsetstrokecolor{textcolor}%
\pgfsetfillcolor{textcolor}%
\pgftext[x=4.842727in,y=0.363056in,,top]{\color{textcolor}{\sffamily\fontsize{11.000000}{13.200000}\selectfont\catcode`\^=\active\def^{\ifmmode\sp\else\^{}\fi}\catcode`\%=\active\def%{\%}24:00}}%
\end{pgfscope}%
\begin{pgfscope}%
\definecolor{textcolor}{rgb}{0.150000,0.150000,0.150000}%
\pgfsetstrokecolor{textcolor}%
\pgfsetfillcolor{textcolor}%
\pgftext[x=2.870000in,y=0.167777in,,top]{\color{textcolor}{\sffamily\fontsize{12.000000}{14.400000}\selectfont\catcode`\^=\active\def^{\ifmmode\sp\else\^{}\fi}\catcode`\%=\active\def%{\%}Time}}%
\end{pgfscope}%
\begin{pgfscope}%
\pgfpathrectangle{\pgfqpoint{0.700000in}{0.495000in}}{\pgfqpoint{4.340000in}{3.465000in}}%
\pgfusepath{clip}%
\pgfsetroundcap%
\pgfsetroundjoin%
\pgfsetlinewidth{1.003750pt}%
\definecolor{currentstroke}{rgb}{1.000000,1.000000,1.000000}%
\pgfsetstrokecolor{currentstroke}%
\pgfsetdash{}{0pt}%
\pgfpathmoveto{\pgfqpoint{0.700000in}{0.735617in}}%
\pgfpathlineto{\pgfqpoint{5.040000in}{0.735617in}}%
\pgfusepath{stroke}%
\end{pgfscope}%
\begin{pgfscope}%
\pgfpathrectangle{\pgfqpoint{0.700000in}{0.495000in}}{\pgfqpoint{4.340000in}{3.465000in}}%
\pgfusepath{clip}%
\pgfsetroundcap%
\pgfsetroundjoin%
\pgfsetlinewidth{1.003750pt}%
\definecolor{currentstroke}{rgb}{1.000000,1.000000,1.000000}%
\pgfsetstrokecolor{currentstroke}%
\pgfsetdash{}{0pt}%
\pgfpathmoveto{\pgfqpoint{0.700000in}{1.289254in}}%
\pgfpathlineto{\pgfqpoint{5.040000in}{1.289254in}}%
\pgfusepath{stroke}%
\end{pgfscope}%
\begin{pgfscope}%
\pgfpathrectangle{\pgfqpoint{0.700000in}{0.495000in}}{\pgfqpoint{4.340000in}{3.465000in}}%
\pgfusepath{clip}%
\pgfsetroundcap%
\pgfsetroundjoin%
\pgfsetlinewidth{1.003750pt}%
\definecolor{currentstroke}{rgb}{1.000000,1.000000,1.000000}%
\pgfsetstrokecolor{currentstroke}%
\pgfsetdash{}{0pt}%
\pgfpathmoveto{\pgfqpoint{0.700000in}{1.842891in}}%
\pgfpathlineto{\pgfqpoint{5.040000in}{1.842891in}}%
\pgfusepath{stroke}%
\end{pgfscope}%
\begin{pgfscope}%
\pgfpathrectangle{\pgfqpoint{0.700000in}{0.495000in}}{\pgfqpoint{4.340000in}{3.465000in}}%
\pgfusepath{clip}%
\pgfsetroundcap%
\pgfsetroundjoin%
\pgfsetlinewidth{1.003750pt}%
\definecolor{currentstroke}{rgb}{1.000000,1.000000,1.000000}%
\pgfsetstrokecolor{currentstroke}%
\pgfsetdash{}{0pt}%
\pgfpathmoveto{\pgfqpoint{0.700000in}{2.396529in}}%
\pgfpathlineto{\pgfqpoint{5.040000in}{2.396529in}}%
\pgfusepath{stroke}%
\end{pgfscope}%
\begin{pgfscope}%
\pgfpathrectangle{\pgfqpoint{0.700000in}{0.495000in}}{\pgfqpoint{4.340000in}{3.465000in}}%
\pgfusepath{clip}%
\pgfsetroundcap%
\pgfsetroundjoin%
\pgfsetlinewidth{1.003750pt}%
\definecolor{currentstroke}{rgb}{1.000000,1.000000,1.000000}%
\pgfsetstrokecolor{currentstroke}%
\pgfsetdash{}{0pt}%
\pgfpathmoveto{\pgfqpoint{0.700000in}{2.950166in}}%
\pgfpathlineto{\pgfqpoint{5.040000in}{2.950166in}}%
\pgfusepath{stroke}%
\end{pgfscope}%
\begin{pgfscope}%
\pgfpathrectangle{\pgfqpoint{0.700000in}{0.495000in}}{\pgfqpoint{4.340000in}{3.465000in}}%
\pgfusepath{clip}%
\pgfsetroundcap%
\pgfsetroundjoin%
\pgfsetlinewidth{1.003750pt}%
\definecolor{currentstroke}{rgb}{1.000000,1.000000,1.000000}%
\pgfsetstrokecolor{currentstroke}%
\pgfsetdash{}{0pt}%
\pgfpathmoveto{\pgfqpoint{0.700000in}{3.503803in}}%
\pgfpathlineto{\pgfqpoint{5.040000in}{3.503803in}}%
\pgfusepath{stroke}%
\end{pgfscope}%
\begin{pgfscope}%
\definecolor{textcolor}{rgb}{0.150000,0.150000,0.150000}%
\pgfsetstrokecolor{textcolor}%
\pgfsetfillcolor{textcolor}%
\pgftext[x=0.512500in,y=2.227500in,,bottom,rotate=90.000000]{\color{textcolor}{\sffamily\fontsize{12.000000}{14.400000}\selectfont\catcode`\^=\active\def^{\ifmmode\sp\else\^{}\fi}\catcode`\%=\active\def%{\%}Resources}}%
\end{pgfscope}%
\begin{pgfscope}%
\pgfpathrectangle{\pgfqpoint{0.700000in}{0.495000in}}{\pgfqpoint{4.340000in}{3.465000in}}%
\pgfusepath{clip}%
\pgfsetbuttcap%
\pgfsetroundjoin%
\definecolor{currentfill}{rgb}{0.172549,0.627451,0.172549}%
\pgfsetfillcolor{currentfill}%
\pgfsetfillopacity{0.300000}%
\pgfsetlinewidth{1.003750pt}%
\definecolor{currentstroke}{rgb}{0.172549,0.627451,0.172549}%
\pgfsetstrokecolor{currentstroke}%
\pgfsetstrokeopacity{0.300000}%
\pgfsetdash{}{0pt}%
\pgfpathmoveto{\pgfqpoint{0.905179in}{0.748397in}}%
\pgfpathlineto{\pgfqpoint{0.905179in}{0.735121in}}%
\pgfpathlineto{\pgfqpoint{0.913086in}{0.734580in}}%
\pgfpathlineto{\pgfqpoint{0.920993in}{0.733996in}}%
\pgfpathlineto{\pgfqpoint{0.928900in}{0.733373in}}%
\pgfpathlineto{\pgfqpoint{0.936806in}{0.732714in}}%
\pgfpathlineto{\pgfqpoint{0.944713in}{0.732022in}}%
\pgfpathlineto{\pgfqpoint{0.952620in}{0.731299in}}%
\pgfpathlineto{\pgfqpoint{0.960527in}{0.730550in}}%
\pgfpathlineto{\pgfqpoint{0.968433in}{0.729777in}}%
\pgfpathlineto{\pgfqpoint{0.976340in}{0.728982in}}%
\pgfpathlineto{\pgfqpoint{0.984247in}{0.728170in}}%
\pgfpathlineto{\pgfqpoint{0.992153in}{0.727343in}}%
\pgfpathlineto{\pgfqpoint{1.000060in}{0.726505in}}%
\pgfpathlineto{\pgfqpoint{1.007967in}{0.725657in}}%
\pgfpathlineto{\pgfqpoint{1.015874in}{0.724804in}}%
\pgfpathlineto{\pgfqpoint{1.023780in}{0.723949in}}%
\pgfpathlineto{\pgfqpoint{1.031687in}{0.723094in}}%
\pgfpathlineto{\pgfqpoint{1.039594in}{0.722242in}}%
\pgfpathlineto{\pgfqpoint{1.047500in}{0.721397in}}%
\pgfpathlineto{\pgfqpoint{1.055407in}{0.720562in}}%
\pgfpathlineto{\pgfqpoint{1.063314in}{0.719739in}}%
\pgfpathlineto{\pgfqpoint{1.071221in}{0.718933in}}%
\pgfpathlineto{\pgfqpoint{1.079127in}{0.718145in}}%
\pgfpathlineto{\pgfqpoint{1.087034in}{0.717379in}}%
\pgfpathlineto{\pgfqpoint{1.094941in}{0.716638in}}%
\pgfpathlineto{\pgfqpoint{1.102848in}{0.715925in}}%
\pgfpathlineto{\pgfqpoint{1.110754in}{0.715243in}}%
\pgfpathlineto{\pgfqpoint{1.118661in}{0.714595in}}%
\pgfpathlineto{\pgfqpoint{1.126568in}{0.713984in}}%
\pgfpathlineto{\pgfqpoint{1.134474in}{0.713414in}}%
\pgfpathlineto{\pgfqpoint{1.142381in}{0.712887in}}%
\pgfpathlineto{\pgfqpoint{1.150288in}{0.712406in}}%
\pgfpathlineto{\pgfqpoint{1.158195in}{0.711974in}}%
\pgfpathlineto{\pgfqpoint{1.166101in}{0.711596in}}%
\pgfpathlineto{\pgfqpoint{1.174008in}{0.711272in}}%
\pgfpathlineto{\pgfqpoint{1.181915in}{0.711007in}}%
\pgfpathlineto{\pgfqpoint{1.189821in}{0.710804in}}%
\pgfpathlineto{\pgfqpoint{1.197728in}{0.710666in}}%
\pgfpathlineto{\pgfqpoint{1.205635in}{0.710595in}}%
\pgfpathlineto{\pgfqpoint{1.213542in}{0.710595in}}%
\pgfpathlineto{\pgfqpoint{1.221448in}{0.710670in}}%
\pgfpathlineto{\pgfqpoint{1.229355in}{0.710821in}}%
\pgfpathlineto{\pgfqpoint{1.237262in}{0.711052in}}%
\pgfpathlineto{\pgfqpoint{1.245169in}{0.711366in}}%
\pgfpathlineto{\pgfqpoint{1.253075in}{0.711766in}}%
\pgfpathlineto{\pgfqpoint{1.260982in}{0.712255in}}%
\pgfpathlineto{\pgfqpoint{1.268889in}{0.712837in}}%
\pgfpathlineto{\pgfqpoint{1.276795in}{0.713513in}}%
\pgfpathlineto{\pgfqpoint{1.284702in}{0.714289in}}%
\pgfpathlineto{\pgfqpoint{1.292609in}{0.715165in}}%
\pgfpathlineto{\pgfqpoint{1.300516in}{0.716146in}}%
\pgfpathlineto{\pgfqpoint{1.308422in}{0.717235in}}%
\pgfpathlineto{\pgfqpoint{1.316329in}{0.718434in}}%
\pgfpathlineto{\pgfqpoint{1.324236in}{0.719747in}}%
\pgfpathlineto{\pgfqpoint{1.332142in}{0.721176in}}%
\pgfpathlineto{\pgfqpoint{1.340049in}{0.722725in}}%
\pgfpathlineto{\pgfqpoint{1.347956in}{0.724397in}}%
\pgfpathlineto{\pgfqpoint{1.355863in}{0.726195in}}%
\pgfpathlineto{\pgfqpoint{1.363769in}{0.728122in}}%
\pgfpathlineto{\pgfqpoint{1.371676in}{0.730181in}}%
\pgfpathlineto{\pgfqpoint{1.379583in}{0.732374in}}%
\pgfpathlineto{\pgfqpoint{1.387490in}{0.734706in}}%
\pgfpathlineto{\pgfqpoint{1.387490in}{0.737664in}}%
\pgfpathlineto{\pgfqpoint{1.387490in}{0.737664in}}%
\pgfpathlineto{\pgfqpoint{1.379583in}{0.743167in}}%
\pgfpathlineto{\pgfqpoint{1.371676in}{0.748726in}}%
\pgfpathlineto{\pgfqpoint{1.363769in}{0.754327in}}%
\pgfpathlineto{\pgfqpoint{1.355863in}{0.759959in}}%
\pgfpathlineto{\pgfqpoint{1.347956in}{0.765611in}}%
\pgfpathlineto{\pgfqpoint{1.340049in}{0.771271in}}%
\pgfpathlineto{\pgfqpoint{1.332142in}{0.776926in}}%
\pgfpathlineto{\pgfqpoint{1.324236in}{0.782565in}}%
\pgfpathlineto{\pgfqpoint{1.316329in}{0.788177in}}%
\pgfpathlineto{\pgfqpoint{1.308422in}{0.793748in}}%
\pgfpathlineto{\pgfqpoint{1.300516in}{0.799269in}}%
\pgfpathlineto{\pgfqpoint{1.292609in}{0.804726in}}%
\pgfpathlineto{\pgfqpoint{1.284702in}{0.810108in}}%
\pgfpathlineto{\pgfqpoint{1.276795in}{0.815402in}}%
\pgfpathlineto{\pgfqpoint{1.268889in}{0.820599in}}%
\pgfpathlineto{\pgfqpoint{1.260982in}{0.825684in}}%
\pgfpathlineto{\pgfqpoint{1.253075in}{0.830647in}}%
\pgfpathlineto{\pgfqpoint{1.245169in}{0.835476in}}%
\pgfpathlineto{\pgfqpoint{1.237262in}{0.840159in}}%
\pgfpathlineto{\pgfqpoint{1.229355in}{0.844684in}}%
\pgfpathlineto{\pgfqpoint{1.221448in}{0.849039in}}%
\pgfpathlineto{\pgfqpoint{1.213542in}{0.853213in}}%
\pgfpathlineto{\pgfqpoint{1.205635in}{0.857193in}}%
\pgfpathlineto{\pgfqpoint{1.197728in}{0.860968in}}%
\pgfpathlineto{\pgfqpoint{1.189821in}{0.864527in}}%
\pgfpathlineto{\pgfqpoint{1.181915in}{0.867856in}}%
\pgfpathlineto{\pgfqpoint{1.174008in}{0.870945in}}%
\pgfpathlineto{\pgfqpoint{1.166101in}{0.873781in}}%
\pgfpathlineto{\pgfqpoint{1.158195in}{0.876353in}}%
\pgfpathlineto{\pgfqpoint{1.150288in}{0.878649in}}%
\pgfpathlineto{\pgfqpoint{1.142381in}{0.880658in}}%
\pgfpathlineto{\pgfqpoint{1.134474in}{0.882366in}}%
\pgfpathlineto{\pgfqpoint{1.126568in}{0.883763in}}%
\pgfpathlineto{\pgfqpoint{1.118661in}{0.884836in}}%
\pgfpathlineto{\pgfqpoint{1.110754in}{0.885575in}}%
\pgfpathlineto{\pgfqpoint{1.102848in}{0.885966in}}%
\pgfpathlineto{\pgfqpoint{1.094941in}{0.885999in}}%
\pgfpathlineto{\pgfqpoint{1.087034in}{0.885660in}}%
\pgfpathlineto{\pgfqpoint{1.079127in}{0.884940in}}%
\pgfpathlineto{\pgfqpoint{1.071221in}{0.883825in}}%
\pgfpathlineto{\pgfqpoint{1.063314in}{0.882304in}}%
\pgfpathlineto{\pgfqpoint{1.055407in}{0.880365in}}%
\pgfpathlineto{\pgfqpoint{1.047500in}{0.877996in}}%
\pgfpathlineto{\pgfqpoint{1.039594in}{0.875186in}}%
\pgfpathlineto{\pgfqpoint{1.031687in}{0.871922in}}%
\pgfpathlineto{\pgfqpoint{1.023780in}{0.868193in}}%
\pgfpathlineto{\pgfqpoint{1.015874in}{0.863988in}}%
\pgfpathlineto{\pgfqpoint{1.007967in}{0.859293in}}%
\pgfpathlineto{\pgfqpoint{1.000060in}{0.854097in}}%
\pgfpathlineto{\pgfqpoint{0.992153in}{0.848389in}}%
\pgfpathlineto{\pgfqpoint{0.984247in}{0.842157in}}%
\pgfpathlineto{\pgfqpoint{0.976340in}{0.835389in}}%
\pgfpathlineto{\pgfqpoint{0.968433in}{0.828073in}}%
\pgfpathlineto{\pgfqpoint{0.960527in}{0.820196in}}%
\pgfpathlineto{\pgfqpoint{0.952620in}{0.811749in}}%
\pgfpathlineto{\pgfqpoint{0.944713in}{0.802718in}}%
\pgfpathlineto{\pgfqpoint{0.936806in}{0.793091in}}%
\pgfpathlineto{\pgfqpoint{0.928900in}{0.782858in}}%
\pgfpathlineto{\pgfqpoint{0.920993in}{0.772005in}}%
\pgfpathlineto{\pgfqpoint{0.913086in}{0.760522in}}%
\pgfpathlineto{\pgfqpoint{0.905179in}{0.748397in}}%
\pgfpathlineto{\pgfqpoint{0.905179in}{0.748397in}}%
\pgfpathclose%
\pgfusepath{stroke,fill}%
\end{pgfscope}%
\begin{pgfscope}%
\pgfpathrectangle{\pgfqpoint{0.700000in}{0.495000in}}{\pgfqpoint{4.340000in}{3.465000in}}%
\pgfusepath{clip}%
\pgfsetbuttcap%
\pgfsetroundjoin%
\definecolor{currentfill}{rgb}{0.172549,0.627451,0.172549}%
\pgfsetfillcolor{currentfill}%
\pgfsetfillopacity{0.300000}%
\pgfsetlinewidth{1.003750pt}%
\definecolor{currentstroke}{rgb}{0.172549,0.627451,0.172549}%
\pgfsetstrokecolor{currentstroke}%
\pgfsetstrokeopacity{0.300000}%
\pgfsetdash{}{0pt}%
\pgfpathmoveto{\pgfqpoint{2.122815in}{1.932642in}}%
\pgfpathlineto{\pgfqpoint{2.122815in}{1.925376in}}%
\pgfpathlineto{\pgfqpoint{2.130721in}{1.948903in}}%
\pgfpathlineto{\pgfqpoint{2.138628in}{1.972499in}}%
\pgfpathlineto{\pgfqpoint{2.146535in}{1.996157in}}%
\pgfpathlineto{\pgfqpoint{2.154442in}{2.019866in}}%
\pgfpathlineto{\pgfqpoint{2.162348in}{2.043619in}}%
\pgfpathlineto{\pgfqpoint{2.170255in}{2.067407in}}%
\pgfpathlineto{\pgfqpoint{2.178162in}{2.091223in}}%
\pgfpathlineto{\pgfqpoint{2.186069in}{2.115057in}}%
\pgfpathlineto{\pgfqpoint{2.193975in}{2.138900in}}%
\pgfpathlineto{\pgfqpoint{2.201882in}{2.162746in}}%
\pgfpathlineto{\pgfqpoint{2.209789in}{2.186584in}}%
\pgfpathlineto{\pgfqpoint{2.217695in}{2.210407in}}%
\pgfpathlineto{\pgfqpoint{2.225602in}{2.234207in}}%
\pgfpathlineto{\pgfqpoint{2.233509in}{2.257974in}}%
\pgfpathlineto{\pgfqpoint{2.241416in}{2.281700in}}%
\pgfpathlineto{\pgfqpoint{2.249322in}{2.305377in}}%
\pgfpathlineto{\pgfqpoint{2.257229in}{2.328997in}}%
\pgfpathlineto{\pgfqpoint{2.265136in}{2.352551in}}%
\pgfpathlineto{\pgfqpoint{2.273042in}{2.376030in}}%
\pgfpathlineto{\pgfqpoint{2.280949in}{2.399427in}}%
\pgfpathlineto{\pgfqpoint{2.288856in}{2.422732in}}%
\pgfpathlineto{\pgfqpoint{2.296763in}{2.445937in}}%
\pgfpathlineto{\pgfqpoint{2.304669in}{2.469034in}}%
\pgfpathlineto{\pgfqpoint{2.312576in}{2.492014in}}%
\pgfpathlineto{\pgfqpoint{2.320483in}{2.514869in}}%
\pgfpathlineto{\pgfqpoint{2.328390in}{2.537591in}}%
\pgfpathlineto{\pgfqpoint{2.336296in}{2.560171in}}%
\pgfpathlineto{\pgfqpoint{2.344203in}{2.582600in}}%
\pgfpathlineto{\pgfqpoint{2.352110in}{2.604870in}}%
\pgfpathlineto{\pgfqpoint{2.360016in}{2.626973in}}%
\pgfpathlineto{\pgfqpoint{2.367923in}{2.648900in}}%
\pgfpathlineto{\pgfqpoint{2.375830in}{2.670643in}}%
\pgfpathlineto{\pgfqpoint{2.383737in}{2.692193in}}%
\pgfpathlineto{\pgfqpoint{2.391643in}{2.713540in}}%
\pgfpathlineto{\pgfqpoint{2.399550in}{2.734674in}}%
\pgfpathlineto{\pgfqpoint{2.407457in}{2.755584in}}%
\pgfpathlineto{\pgfqpoint{2.415363in}{2.776261in}}%
\pgfpathlineto{\pgfqpoint{2.423270in}{2.796694in}}%
\pgfpathlineto{\pgfqpoint{2.431177in}{2.816872in}}%
\pgfpathlineto{\pgfqpoint{2.439084in}{2.836785in}}%
\pgfpathlineto{\pgfqpoint{2.446990in}{2.856423in}}%
\pgfpathlineto{\pgfqpoint{2.454897in}{2.875775in}}%
\pgfpathlineto{\pgfqpoint{2.462804in}{2.894832in}}%
\pgfpathlineto{\pgfqpoint{2.470711in}{2.913583in}}%
\pgfpathlineto{\pgfqpoint{2.478617in}{2.932016in}}%
\pgfpathlineto{\pgfqpoint{2.486524in}{2.950124in}}%
\pgfpathlineto{\pgfqpoint{2.494431in}{2.967893in}}%
\pgfpathlineto{\pgfqpoint{2.502337in}{2.985316in}}%
\pgfpathlineto{\pgfqpoint{2.510244in}{3.002380in}}%
\pgfpathlineto{\pgfqpoint{2.518151in}{3.019076in}}%
\pgfpathlineto{\pgfqpoint{2.526058in}{3.035393in}}%
\pgfpathlineto{\pgfqpoint{2.533964in}{3.051322in}}%
\pgfpathlineto{\pgfqpoint{2.541871in}{3.066851in}}%
\pgfpathlineto{\pgfqpoint{2.549778in}{3.081971in}}%
\pgfpathlineto{\pgfqpoint{2.557684in}{3.096671in}}%
\pgfpathlineto{\pgfqpoint{2.565591in}{3.110940in}}%
\pgfpathlineto{\pgfqpoint{2.573498in}{3.124769in}}%
\pgfpathlineto{\pgfqpoint{2.581405in}{3.138147in}}%
\pgfpathlineto{\pgfqpoint{2.589311in}{3.151063in}}%
\pgfpathlineto{\pgfqpoint{2.597218in}{3.163508in}}%
\pgfpathlineto{\pgfqpoint{2.605125in}{3.175471in}}%
\pgfpathlineto{\pgfqpoint{2.613032in}{3.186942in}}%
\pgfpathlineto{\pgfqpoint{2.620938in}{3.197910in}}%
\pgfpathlineto{\pgfqpoint{2.628845in}{3.208365in}}%
\pgfpathlineto{\pgfqpoint{2.636752in}{3.218296in}}%
\pgfpathlineto{\pgfqpoint{2.644658in}{3.227694in}}%
\pgfpathlineto{\pgfqpoint{2.652565in}{3.236548in}}%
\pgfpathlineto{\pgfqpoint{2.660472in}{3.244847in}}%
\pgfpathlineto{\pgfqpoint{2.668379in}{3.252582in}}%
\pgfpathlineto{\pgfqpoint{2.676285in}{3.259741in}}%
\pgfpathlineto{\pgfqpoint{2.684192in}{3.266316in}}%
\pgfpathlineto{\pgfqpoint{2.692099in}{3.272294in}}%
\pgfpathlineto{\pgfqpoint{2.700005in}{3.277666in}}%
\pgfpathlineto{\pgfqpoint{2.707912in}{3.282422in}}%
\pgfpathlineto{\pgfqpoint{2.715819in}{3.286552in}}%
\pgfpathlineto{\pgfqpoint{2.723726in}{3.290044in}}%
\pgfpathlineto{\pgfqpoint{2.731632in}{3.292888in}}%
\pgfpathlineto{\pgfqpoint{2.739539in}{3.295075in}}%
\pgfpathlineto{\pgfqpoint{2.747446in}{3.296594in}}%
\pgfpathlineto{\pgfqpoint{2.755353in}{3.297434in}}%
\pgfpathlineto{\pgfqpoint{2.763259in}{3.297586in}}%
\pgfpathlineto{\pgfqpoint{2.771166in}{3.297038in}}%
\pgfpathlineto{\pgfqpoint{2.779073in}{3.295781in}}%
\pgfpathlineto{\pgfqpoint{2.786979in}{3.293804in}}%
\pgfpathlineto{\pgfqpoint{2.794886in}{3.291097in}}%
\pgfpathlineto{\pgfqpoint{2.802793in}{3.287649in}}%
\pgfpathlineto{\pgfqpoint{2.810700in}{3.283451in}}%
\pgfpathlineto{\pgfqpoint{2.818606in}{3.278491in}}%
\pgfpathlineto{\pgfqpoint{2.826513in}{3.272760in}}%
\pgfpathlineto{\pgfqpoint{2.834420in}{3.266247in}}%
\pgfpathlineto{\pgfqpoint{2.842326in}{3.258941in}}%
\pgfpathlineto{\pgfqpoint{2.850233in}{3.250833in}}%
\pgfpathlineto{\pgfqpoint{2.858140in}{3.241913in}}%
\pgfpathlineto{\pgfqpoint{2.866047in}{3.232168in}}%
\pgfpathlineto{\pgfqpoint{2.866047in}{3.232692in}}%
\pgfpathlineto{\pgfqpoint{2.866047in}{3.232692in}}%
\pgfpathlineto{\pgfqpoint{2.858140in}{3.243808in}}%
\pgfpathlineto{\pgfqpoint{2.850233in}{3.254519in}}%
\pgfpathlineto{\pgfqpoint{2.842326in}{3.264822in}}%
\pgfpathlineto{\pgfqpoint{2.834420in}{3.274709in}}%
\pgfpathlineto{\pgfqpoint{2.826513in}{3.284176in}}%
\pgfpathlineto{\pgfqpoint{2.818606in}{3.293217in}}%
\pgfpathlineto{\pgfqpoint{2.810700in}{3.301826in}}%
\pgfpathlineto{\pgfqpoint{2.802793in}{3.309998in}}%
\pgfpathlineto{\pgfqpoint{2.794886in}{3.317727in}}%
\pgfpathlineto{\pgfqpoint{2.786979in}{3.325007in}}%
\pgfpathlineto{\pgfqpoint{2.779073in}{3.331834in}}%
\pgfpathlineto{\pgfqpoint{2.771166in}{3.338201in}}%
\pgfpathlineto{\pgfqpoint{2.763259in}{3.344102in}}%
\pgfpathlineto{\pgfqpoint{2.755353in}{3.349533in}}%
\pgfpathlineto{\pgfqpoint{2.747446in}{3.354487in}}%
\pgfpathlineto{\pgfqpoint{2.739539in}{3.358960in}}%
\pgfpathlineto{\pgfqpoint{2.731632in}{3.362945in}}%
\pgfpathlineto{\pgfqpoint{2.723726in}{3.366437in}}%
\pgfpathlineto{\pgfqpoint{2.715819in}{3.369430in}}%
\pgfpathlineto{\pgfqpoint{2.707912in}{3.371919in}}%
\pgfpathlineto{\pgfqpoint{2.700005in}{3.373898in}}%
\pgfpathlineto{\pgfqpoint{2.692099in}{3.375361in}}%
\pgfpathlineto{\pgfqpoint{2.684192in}{3.376304in}}%
\pgfpathlineto{\pgfqpoint{2.676285in}{3.376720in}}%
\pgfpathlineto{\pgfqpoint{2.668379in}{3.376603in}}%
\pgfpathlineto{\pgfqpoint{2.660472in}{3.375949in}}%
\pgfpathlineto{\pgfqpoint{2.652565in}{3.374752in}}%
\pgfpathlineto{\pgfqpoint{2.644658in}{3.373006in}}%
\pgfpathlineto{\pgfqpoint{2.636752in}{3.370705in}}%
\pgfpathlineto{\pgfqpoint{2.628845in}{3.367844in}}%
\pgfpathlineto{\pgfqpoint{2.620938in}{3.364417in}}%
\pgfpathlineto{\pgfqpoint{2.613032in}{3.360419in}}%
\pgfpathlineto{\pgfqpoint{2.605125in}{3.355845in}}%
\pgfpathlineto{\pgfqpoint{2.597218in}{3.350687in}}%
\pgfpathlineto{\pgfqpoint{2.589311in}{3.344942in}}%
\pgfpathlineto{\pgfqpoint{2.581405in}{3.338604in}}%
\pgfpathlineto{\pgfqpoint{2.573498in}{3.331666in}}%
\pgfpathlineto{\pgfqpoint{2.565591in}{3.324123in}}%
\pgfpathlineto{\pgfqpoint{2.557684in}{3.315970in}}%
\pgfpathlineto{\pgfqpoint{2.549778in}{3.307201in}}%
\pgfpathlineto{\pgfqpoint{2.541871in}{3.297811in}}%
\pgfpathlineto{\pgfqpoint{2.533964in}{3.287793in}}%
\pgfpathlineto{\pgfqpoint{2.526058in}{3.277143in}}%
\pgfpathlineto{\pgfqpoint{2.518151in}{3.265854in}}%
\pgfpathlineto{\pgfqpoint{2.510244in}{3.253922in}}%
\pgfpathlineto{\pgfqpoint{2.502337in}{3.241340in}}%
\pgfpathlineto{\pgfqpoint{2.494431in}{3.228104in}}%
\pgfpathlineto{\pgfqpoint{2.486524in}{3.214207in}}%
\pgfpathlineto{\pgfqpoint{2.478617in}{3.199643in}}%
\pgfpathlineto{\pgfqpoint{2.470711in}{3.184408in}}%
\pgfpathlineto{\pgfqpoint{2.462804in}{3.168496in}}%
\pgfpathlineto{\pgfqpoint{2.454897in}{3.151900in}}%
\pgfpathlineto{\pgfqpoint{2.446990in}{3.134616in}}%
\pgfpathlineto{\pgfqpoint{2.439084in}{3.116638in}}%
\pgfpathlineto{\pgfqpoint{2.431177in}{3.097961in}}%
\pgfpathlineto{\pgfqpoint{2.423270in}{3.078578in}}%
\pgfpathlineto{\pgfqpoint{2.415363in}{3.058484in}}%
\pgfpathlineto{\pgfqpoint{2.407457in}{3.037674in}}%
\pgfpathlineto{\pgfqpoint{2.399550in}{3.016143in}}%
\pgfpathlineto{\pgfqpoint{2.391643in}{2.993883in}}%
\pgfpathlineto{\pgfqpoint{2.383737in}{2.970890in}}%
\pgfpathlineto{\pgfqpoint{2.375830in}{2.947159in}}%
\pgfpathlineto{\pgfqpoint{2.367923in}{2.922690in}}%
\pgfpathlineto{\pgfqpoint{2.360016in}{2.897506in}}%
\pgfpathlineto{\pgfqpoint{2.352110in}{2.871632in}}%
\pgfpathlineto{\pgfqpoint{2.344203in}{2.845091in}}%
\pgfpathlineto{\pgfqpoint{2.336296in}{2.817909in}}%
\pgfpathlineto{\pgfqpoint{2.328390in}{2.790111in}}%
\pgfpathlineto{\pgfqpoint{2.320483in}{2.761723in}}%
\pgfpathlineto{\pgfqpoint{2.312576in}{2.732767in}}%
\pgfpathlineto{\pgfqpoint{2.304669in}{2.703271in}}%
\pgfpathlineto{\pgfqpoint{2.296763in}{2.673257in}}%
\pgfpathlineto{\pgfqpoint{2.288856in}{2.642752in}}%
\pgfpathlineto{\pgfqpoint{2.280949in}{2.611781in}}%
\pgfpathlineto{\pgfqpoint{2.273042in}{2.580367in}}%
\pgfpathlineto{\pgfqpoint{2.265136in}{2.548537in}}%
\pgfpathlineto{\pgfqpoint{2.257229in}{2.516314in}}%
\pgfpathlineto{\pgfqpoint{2.249322in}{2.483724in}}%
\pgfpathlineto{\pgfqpoint{2.241416in}{2.450792in}}%
\pgfpathlineto{\pgfqpoint{2.233509in}{2.417542in}}%
\pgfpathlineto{\pgfqpoint{2.225602in}{2.383999in}}%
\pgfpathlineto{\pgfqpoint{2.217695in}{2.350189in}}%
\pgfpathlineto{\pgfqpoint{2.209789in}{2.316136in}}%
\pgfpathlineto{\pgfqpoint{2.201882in}{2.281865in}}%
\pgfpathlineto{\pgfqpoint{2.193975in}{2.247401in}}%
\pgfpathlineto{\pgfqpoint{2.186069in}{2.212769in}}%
\pgfpathlineto{\pgfqpoint{2.178162in}{2.177993in}}%
\pgfpathlineto{\pgfqpoint{2.170255in}{2.143099in}}%
\pgfpathlineto{\pgfqpoint{2.162348in}{2.108112in}}%
\pgfpathlineto{\pgfqpoint{2.154442in}{2.073056in}}%
\pgfpathlineto{\pgfqpoint{2.146535in}{2.037956in}}%
\pgfpathlineto{\pgfqpoint{2.138628in}{2.002837in}}%
\pgfpathlineto{\pgfqpoint{2.130721in}{1.967724in}}%
\pgfpathlineto{\pgfqpoint{2.122815in}{1.932642in}}%
\pgfpathlineto{\pgfqpoint{2.122815in}{1.932642in}}%
\pgfpathclose%
\pgfusepath{stroke,fill}%
\end{pgfscope}%
\begin{pgfscope}%
\pgfpathrectangle{\pgfqpoint{0.700000in}{0.495000in}}{\pgfqpoint{4.340000in}{3.465000in}}%
\pgfusepath{clip}%
\pgfsetbuttcap%
\pgfsetroundjoin%
\definecolor{currentfill}{rgb}{0.172549,0.627451,0.172549}%
\pgfsetfillcolor{currentfill}%
\pgfsetfillopacity{0.300000}%
\pgfsetlinewidth{1.003750pt}%
\definecolor{currentstroke}{rgb}{0.172549,0.627451,0.172549}%
\pgfsetstrokecolor{currentstroke}%
\pgfsetstrokeopacity{0.300000}%
\pgfsetdash{}{0pt}%
\pgfpathmoveto{\pgfqpoint{2.905580in}{3.171518in}}%
\pgfpathlineto{\pgfqpoint{2.905580in}{3.171444in}}%
\pgfpathlineto{\pgfqpoint{2.913487in}{3.157126in}}%
\pgfpathlineto{\pgfqpoint{2.921394in}{3.142168in}}%
\pgfpathlineto{\pgfqpoint{2.929300in}{3.126606in}}%
\pgfpathlineto{\pgfqpoint{2.937207in}{3.110476in}}%
\pgfpathlineto{\pgfqpoint{2.945114in}{3.093814in}}%
\pgfpathlineto{\pgfqpoint{2.953021in}{3.076657in}}%
\pgfpathlineto{\pgfqpoint{2.960927in}{3.059041in}}%
\pgfpathlineto{\pgfqpoint{2.968834in}{3.041001in}}%
\pgfpathlineto{\pgfqpoint{2.976741in}{3.022573in}}%
\pgfpathlineto{\pgfqpoint{2.984647in}{3.003795in}}%
\pgfpathlineto{\pgfqpoint{2.992554in}{2.984702in}}%
\pgfpathlineto{\pgfqpoint{3.000461in}{2.965329in}}%
\pgfpathlineto{\pgfqpoint{3.008368in}{2.945714in}}%
\pgfpathlineto{\pgfqpoint{3.016274in}{2.925893in}}%
\pgfpathlineto{\pgfqpoint{3.024181in}{2.905900in}}%
\pgfpathlineto{\pgfqpoint{3.032088in}{2.885774in}}%
\pgfpathlineto{\pgfqpoint{3.039995in}{2.865549in}}%
\pgfpathlineto{\pgfqpoint{3.047901in}{2.845262in}}%
\pgfpathlineto{\pgfqpoint{3.055808in}{2.824949in}}%
\pgfpathlineto{\pgfqpoint{3.063715in}{2.804646in}}%
\pgfpathlineto{\pgfqpoint{3.071621in}{2.784389in}}%
\pgfpathlineto{\pgfqpoint{3.079528in}{2.764214in}}%
\pgfpathlineto{\pgfqpoint{3.087435in}{2.744158in}}%
\pgfpathlineto{\pgfqpoint{3.095342in}{2.724257in}}%
\pgfpathlineto{\pgfqpoint{3.103248in}{2.704546in}}%
\pgfpathlineto{\pgfqpoint{3.111155in}{2.685062in}}%
\pgfpathlineto{\pgfqpoint{3.119062in}{2.665841in}}%
\pgfpathlineto{\pgfqpoint{3.126968in}{2.646919in}}%
\pgfpathlineto{\pgfqpoint{3.134875in}{2.628332in}}%
\pgfpathlineto{\pgfqpoint{3.142782in}{2.610116in}}%
\pgfpathlineto{\pgfqpoint{3.150689in}{2.592308in}}%
\pgfpathlineto{\pgfqpoint{3.158595in}{2.574943in}}%
\pgfpathlineto{\pgfqpoint{3.166502in}{2.558058in}}%
\pgfpathlineto{\pgfqpoint{3.174409in}{2.541688in}}%
\pgfpathlineto{\pgfqpoint{3.182316in}{2.525871in}}%
\pgfpathlineto{\pgfqpoint{3.190222in}{2.510641in}}%
\pgfpathlineto{\pgfqpoint{3.198129in}{2.496036in}}%
\pgfpathlineto{\pgfqpoint{3.206036in}{2.482091in}}%
\pgfpathlineto{\pgfqpoint{3.213942in}{2.468842in}}%
\pgfpathlineto{\pgfqpoint{3.221849in}{2.456325in}}%
\pgfpathlineto{\pgfqpoint{3.229756in}{2.444578in}}%
\pgfpathlineto{\pgfqpoint{3.237663in}{2.433635in}}%
\pgfpathlineto{\pgfqpoint{3.245569in}{2.423532in}}%
\pgfpathlineto{\pgfqpoint{3.253476in}{2.414307in}}%
\pgfpathlineto{\pgfqpoint{3.261383in}{2.405995in}}%
\pgfpathlineto{\pgfqpoint{3.269289in}{2.398632in}}%
\pgfpathlineto{\pgfqpoint{3.277196in}{2.392254in}}%
\pgfpathlineto{\pgfqpoint{3.285103in}{2.386898in}}%
\pgfpathlineto{\pgfqpoint{3.293010in}{2.382599in}}%
\pgfpathlineto{\pgfqpoint{3.300916in}{2.379394in}}%
\pgfpathlineto{\pgfqpoint{3.308823in}{2.377319in}}%
\pgfpathlineto{\pgfqpoint{3.316730in}{2.376410in}}%
\pgfpathlineto{\pgfqpoint{3.324637in}{2.376703in}}%
\pgfpathlineto{\pgfqpoint{3.332543in}{2.378234in}}%
\pgfpathlineto{\pgfqpoint{3.340450in}{2.381039in}}%
\pgfpathlineto{\pgfqpoint{3.348357in}{2.385154in}}%
\pgfpathlineto{\pgfqpoint{3.356263in}{2.390616in}}%
\pgfpathlineto{\pgfqpoint{3.364170in}{2.397461in}}%
\pgfpathlineto{\pgfqpoint{3.372077in}{2.405705in}}%
\pgfpathlineto{\pgfqpoint{3.379984in}{2.415313in}}%
\pgfpathlineto{\pgfqpoint{3.387890in}{2.426239in}}%
\pgfpathlineto{\pgfqpoint{3.395797in}{2.438438in}}%
\pgfpathlineto{\pgfqpoint{3.403704in}{2.451865in}}%
\pgfpathlineto{\pgfqpoint{3.411610in}{2.466475in}}%
\pgfpathlineto{\pgfqpoint{3.419517in}{2.482222in}}%
\pgfpathlineto{\pgfqpoint{3.427424in}{2.499062in}}%
\pgfpathlineto{\pgfqpoint{3.435331in}{2.516948in}}%
\pgfpathlineto{\pgfqpoint{3.443237in}{2.535838in}}%
\pgfpathlineto{\pgfqpoint{3.451144in}{2.555684in}}%
\pgfpathlineto{\pgfqpoint{3.459051in}{2.576441in}}%
\pgfpathlineto{\pgfqpoint{3.466958in}{2.598066in}}%
\pgfpathlineto{\pgfqpoint{3.474864in}{2.620512in}}%
\pgfpathlineto{\pgfqpoint{3.482771in}{2.643734in}}%
\pgfpathlineto{\pgfqpoint{3.490678in}{2.667687in}}%
\pgfpathlineto{\pgfqpoint{3.498584in}{2.692326in}}%
\pgfpathlineto{\pgfqpoint{3.506491in}{2.717606in}}%
\pgfpathlineto{\pgfqpoint{3.514398in}{2.743481in}}%
\pgfpathlineto{\pgfqpoint{3.522305in}{2.769907in}}%
\pgfpathlineto{\pgfqpoint{3.530211in}{2.796838in}}%
\pgfpathlineto{\pgfqpoint{3.538118in}{2.824229in}}%
\pgfpathlineto{\pgfqpoint{3.546025in}{2.852036in}}%
\pgfpathlineto{\pgfqpoint{3.553931in}{2.880211in}}%
\pgfpathlineto{\pgfqpoint{3.561838in}{2.908712in}}%
\pgfpathlineto{\pgfqpoint{3.569745in}{2.937492in}}%
\pgfpathlineto{\pgfqpoint{3.577652in}{2.966506in}}%
\pgfpathlineto{\pgfqpoint{3.585558in}{2.995710in}}%
\pgfpathlineto{\pgfqpoint{3.593465in}{3.025057in}}%
\pgfpathlineto{\pgfqpoint{3.601372in}{3.054503in}}%
\pgfpathlineto{\pgfqpoint{3.609279in}{3.084002in}}%
\pgfpathlineto{\pgfqpoint{3.617185in}{3.113511in}}%
\pgfpathlineto{\pgfqpoint{3.625092in}{3.142982in}}%
\pgfpathlineto{\pgfqpoint{3.632999in}{3.172371in}}%
\pgfpathlineto{\pgfqpoint{3.640905in}{3.201634in}}%
\pgfpathlineto{\pgfqpoint{3.648812in}{3.230724in}}%
\pgfpathlineto{\pgfqpoint{3.656719in}{3.259597in}}%
\pgfpathlineto{\pgfqpoint{3.664626in}{3.288207in}}%
\pgfpathlineto{\pgfqpoint{3.672532in}{3.316509in}}%
\pgfpathlineto{\pgfqpoint{3.680439in}{3.344459in}}%
\pgfpathlineto{\pgfqpoint{3.688346in}{3.372010in}}%
\pgfpathlineto{\pgfqpoint{3.696253in}{3.399118in}}%
\pgfpathlineto{\pgfqpoint{3.704159in}{3.425738in}}%
\pgfpathlineto{\pgfqpoint{3.712066in}{3.451824in}}%
\pgfpathlineto{\pgfqpoint{3.719973in}{3.477332in}}%
\pgfpathlineto{\pgfqpoint{3.727879in}{3.502216in}}%
\pgfpathlineto{\pgfqpoint{3.735786in}{3.526430in}}%
\pgfpathlineto{\pgfqpoint{3.743693in}{3.549931in}}%
\pgfpathlineto{\pgfqpoint{3.751600in}{3.572671in}}%
\pgfpathlineto{\pgfqpoint{3.759506in}{3.594608in}}%
\pgfpathlineto{\pgfqpoint{3.767413in}{3.615695in}}%
\pgfpathlineto{\pgfqpoint{3.775320in}{3.635886in}}%
\pgfpathlineto{\pgfqpoint{3.783226in}{3.655138in}}%
\pgfpathlineto{\pgfqpoint{3.791133in}{3.673404in}}%
\pgfpathlineto{\pgfqpoint{3.799040in}{3.690640in}}%
\pgfpathlineto{\pgfqpoint{3.806947in}{3.706801in}}%
\pgfpathlineto{\pgfqpoint{3.814853in}{3.721841in}}%
\pgfpathlineto{\pgfqpoint{3.822760in}{3.735715in}}%
\pgfpathlineto{\pgfqpoint{3.830667in}{3.748377in}}%
\pgfpathlineto{\pgfqpoint{3.838574in}{3.759784in}}%
\pgfpathlineto{\pgfqpoint{3.846480in}{3.769889in}}%
\pgfpathlineto{\pgfqpoint{3.854387in}{3.778648in}}%
\pgfpathlineto{\pgfqpoint{3.854387in}{3.779781in}}%
\pgfpathlineto{\pgfqpoint{3.854387in}{3.779781in}}%
\pgfpathlineto{\pgfqpoint{3.846480in}{3.775666in}}%
\pgfpathlineto{\pgfqpoint{3.838574in}{3.770374in}}%
\pgfpathlineto{\pgfqpoint{3.830667in}{3.763942in}}%
\pgfpathlineto{\pgfqpoint{3.822760in}{3.756407in}}%
\pgfpathlineto{\pgfqpoint{3.814853in}{3.747805in}}%
\pgfpathlineto{\pgfqpoint{3.806947in}{3.738171in}}%
\pgfpathlineto{\pgfqpoint{3.799040in}{3.727543in}}%
\pgfpathlineto{\pgfqpoint{3.791133in}{3.715957in}}%
\pgfpathlineto{\pgfqpoint{3.783226in}{3.703449in}}%
\pgfpathlineto{\pgfqpoint{3.775320in}{3.690055in}}%
\pgfpathlineto{\pgfqpoint{3.767413in}{3.675812in}}%
\pgfpathlineto{\pgfqpoint{3.759506in}{3.660756in}}%
\pgfpathlineto{\pgfqpoint{3.751600in}{3.644923in}}%
\pgfpathlineto{\pgfqpoint{3.743693in}{3.628351in}}%
\pgfpathlineto{\pgfqpoint{3.735786in}{3.611074in}}%
\pgfpathlineto{\pgfqpoint{3.727879in}{3.593129in}}%
\pgfpathlineto{\pgfqpoint{3.719973in}{3.574554in}}%
\pgfpathlineto{\pgfqpoint{3.712066in}{3.555383in}}%
\pgfpathlineto{\pgfqpoint{3.704159in}{3.535654in}}%
\pgfpathlineto{\pgfqpoint{3.696253in}{3.515403in}}%
\pgfpathlineto{\pgfqpoint{3.688346in}{3.494666in}}%
\pgfpathlineto{\pgfqpoint{3.680439in}{3.473479in}}%
\pgfpathlineto{\pgfqpoint{3.672532in}{3.451878in}}%
\pgfpathlineto{\pgfqpoint{3.664626in}{3.429901in}}%
\pgfpathlineto{\pgfqpoint{3.656719in}{3.407584in}}%
\pgfpathlineto{\pgfqpoint{3.648812in}{3.384962in}}%
\pgfpathlineto{\pgfqpoint{3.640905in}{3.362072in}}%
\pgfpathlineto{\pgfqpoint{3.632999in}{3.338951in}}%
\pgfpathlineto{\pgfqpoint{3.625092in}{3.315634in}}%
\pgfpathlineto{\pgfqpoint{3.617185in}{3.292158in}}%
\pgfpathlineto{\pgfqpoint{3.609279in}{3.268560in}}%
\pgfpathlineto{\pgfqpoint{3.601372in}{3.244876in}}%
\pgfpathlineto{\pgfqpoint{3.593465in}{3.221141in}}%
\pgfpathlineto{\pgfqpoint{3.585558in}{3.197393in}}%
\pgfpathlineto{\pgfqpoint{3.577652in}{3.173668in}}%
\pgfpathlineto{\pgfqpoint{3.569745in}{3.150002in}}%
\pgfpathlineto{\pgfqpoint{3.561838in}{3.126431in}}%
\pgfpathlineto{\pgfqpoint{3.553931in}{3.102992in}}%
\pgfpathlineto{\pgfqpoint{3.546025in}{3.079721in}}%
\pgfpathlineto{\pgfqpoint{3.538118in}{3.056655in}}%
\pgfpathlineto{\pgfqpoint{3.530211in}{3.033829in}}%
\pgfpathlineto{\pgfqpoint{3.522305in}{3.011280in}}%
\pgfpathlineto{\pgfqpoint{3.514398in}{2.989045in}}%
\pgfpathlineto{\pgfqpoint{3.506491in}{2.967159in}}%
\pgfpathlineto{\pgfqpoint{3.498584in}{2.945660in}}%
\pgfpathlineto{\pgfqpoint{3.490678in}{2.924583in}}%
\pgfpathlineto{\pgfqpoint{3.482771in}{2.903965in}}%
\pgfpathlineto{\pgfqpoint{3.474864in}{2.883841in}}%
\pgfpathlineto{\pgfqpoint{3.466958in}{2.864250in}}%
\pgfpathlineto{\pgfqpoint{3.459051in}{2.845226in}}%
\pgfpathlineto{\pgfqpoint{3.451144in}{2.826806in}}%
\pgfpathlineto{\pgfqpoint{3.443237in}{2.809026in}}%
\pgfpathlineto{\pgfqpoint{3.435331in}{2.791923in}}%
\pgfpathlineto{\pgfqpoint{3.427424in}{2.775534in}}%
\pgfpathlineto{\pgfqpoint{3.419517in}{2.759893in}}%
\pgfpathlineto{\pgfqpoint{3.411610in}{2.745039in}}%
\pgfpathlineto{\pgfqpoint{3.403704in}{2.731006in}}%
\pgfpathlineto{\pgfqpoint{3.395797in}{2.717832in}}%
\pgfpathlineto{\pgfqpoint{3.387890in}{2.705553in}}%
\pgfpathlineto{\pgfqpoint{3.379984in}{2.694205in}}%
\pgfpathlineto{\pgfqpoint{3.372077in}{2.683824in}}%
\pgfpathlineto{\pgfqpoint{3.364170in}{2.674447in}}%
\pgfpathlineto{\pgfqpoint{3.356263in}{2.666103in}}%
\pgfpathlineto{\pgfqpoint{3.348357in}{2.658784in}}%
\pgfpathlineto{\pgfqpoint{3.340450in}{2.652466in}}%
\pgfpathlineto{\pgfqpoint{3.332543in}{2.647127in}}%
\pgfpathlineto{\pgfqpoint{3.324637in}{2.642742in}}%
\pgfpathlineto{\pgfqpoint{3.316730in}{2.639290in}}%
\pgfpathlineto{\pgfqpoint{3.308823in}{2.636747in}}%
\pgfpathlineto{\pgfqpoint{3.300916in}{2.635089in}}%
\pgfpathlineto{\pgfqpoint{3.293010in}{2.634295in}}%
\pgfpathlineto{\pgfqpoint{3.285103in}{2.634340in}}%
\pgfpathlineto{\pgfqpoint{3.277196in}{2.635201in}}%
\pgfpathlineto{\pgfqpoint{3.269289in}{2.636857in}}%
\pgfpathlineto{\pgfqpoint{3.261383in}{2.639282in}}%
\pgfpathlineto{\pgfqpoint{3.253476in}{2.642456in}}%
\pgfpathlineto{\pgfqpoint{3.245569in}{2.646353in}}%
\pgfpathlineto{\pgfqpoint{3.237663in}{2.650952in}}%
\pgfpathlineto{\pgfqpoint{3.229756in}{2.656228in}}%
\pgfpathlineto{\pgfqpoint{3.221849in}{2.662160in}}%
\pgfpathlineto{\pgfqpoint{3.213942in}{2.668724in}}%
\pgfpathlineto{\pgfqpoint{3.206036in}{2.675896in}}%
\pgfpathlineto{\pgfqpoint{3.198129in}{2.683655in}}%
\pgfpathlineto{\pgfqpoint{3.190222in}{2.691976in}}%
\pgfpathlineto{\pgfqpoint{3.182316in}{2.700836in}}%
\pgfpathlineto{\pgfqpoint{3.174409in}{2.710214in}}%
\pgfpathlineto{\pgfqpoint{3.166502in}{2.720084in}}%
\pgfpathlineto{\pgfqpoint{3.158595in}{2.730425in}}%
\pgfpathlineto{\pgfqpoint{3.150689in}{2.741213in}}%
\pgfpathlineto{\pgfqpoint{3.142782in}{2.752426in}}%
\pgfpathlineto{\pgfqpoint{3.134875in}{2.764039in}}%
\pgfpathlineto{\pgfqpoint{3.126968in}{2.776030in}}%
\pgfpathlineto{\pgfqpoint{3.119062in}{2.788377in}}%
\pgfpathlineto{\pgfqpoint{3.111155in}{2.801055in}}%
\pgfpathlineto{\pgfqpoint{3.103248in}{2.814041in}}%
\pgfpathlineto{\pgfqpoint{3.095342in}{2.827314in}}%
\pgfpathlineto{\pgfqpoint{3.087435in}{2.840849in}}%
\pgfpathlineto{\pgfqpoint{3.079528in}{2.854623in}}%
\pgfpathlineto{\pgfqpoint{3.071621in}{2.868614in}}%
\pgfpathlineto{\pgfqpoint{3.063715in}{2.882798in}}%
\pgfpathlineto{\pgfqpoint{3.055808in}{2.897153in}}%
\pgfpathlineto{\pgfqpoint{3.047901in}{2.911654in}}%
\pgfpathlineto{\pgfqpoint{3.039995in}{2.926280in}}%
\pgfpathlineto{\pgfqpoint{3.032088in}{2.941006in}}%
\pgfpathlineto{\pgfqpoint{3.024181in}{2.955810in}}%
\pgfpathlineto{\pgfqpoint{3.016274in}{2.970669in}}%
\pgfpathlineto{\pgfqpoint{3.008368in}{2.985560in}}%
\pgfpathlineto{\pgfqpoint{3.000461in}{3.000459in}}%
\pgfpathlineto{\pgfqpoint{2.992554in}{3.015344in}}%
\pgfpathlineto{\pgfqpoint{2.984647in}{3.030192in}}%
\pgfpathlineto{\pgfqpoint{2.976741in}{3.044978in}}%
\pgfpathlineto{\pgfqpoint{2.968834in}{3.059681in}}%
\pgfpathlineto{\pgfqpoint{2.960927in}{3.074277in}}%
\pgfpathlineto{\pgfqpoint{2.953021in}{3.088743in}}%
\pgfpathlineto{\pgfqpoint{2.945114in}{3.103056in}}%
\pgfpathlineto{\pgfqpoint{2.937207in}{3.117193in}}%
\pgfpathlineto{\pgfqpoint{2.929300in}{3.131131in}}%
\pgfpathlineto{\pgfqpoint{2.921394in}{3.144846in}}%
\pgfpathlineto{\pgfqpoint{2.913487in}{3.158316in}}%
\pgfpathlineto{\pgfqpoint{2.905580in}{3.171518in}}%
\pgfpathlineto{\pgfqpoint{2.905580in}{3.171518in}}%
\pgfpathclose%
\pgfusepath{stroke,fill}%
\end{pgfscope}%
\begin{pgfscope}%
\pgfpathrectangle{\pgfqpoint{0.700000in}{0.495000in}}{\pgfqpoint{4.340000in}{3.465000in}}%
\pgfusepath{clip}%
\pgfsetbuttcap%
\pgfsetroundjoin%
\definecolor{currentfill}{rgb}{0.172549,0.627451,0.172549}%
\pgfsetfillcolor{currentfill}%
\pgfsetfillopacity{0.300000}%
\pgfsetlinewidth{1.003750pt}%
\definecolor{currentstroke}{rgb}{0.172549,0.627451,0.172549}%
\pgfsetstrokecolor{currentstroke}%
\pgfsetstrokeopacity{0.300000}%
\pgfsetdash{}{0pt}%
\pgfpathmoveto{\pgfqpoint{4.352510in}{2.383445in}}%
\pgfpathlineto{\pgfqpoint{4.352510in}{2.382178in}}%
\pgfpathlineto{\pgfqpoint{4.360417in}{2.343870in}}%
\pgfpathlineto{\pgfqpoint{4.368324in}{2.305521in}}%
\pgfpathlineto{\pgfqpoint{4.376231in}{2.267150in}}%
\pgfpathlineto{\pgfqpoint{4.384137in}{2.228781in}}%
\pgfpathlineto{\pgfqpoint{4.392044in}{2.190434in}}%
\pgfpathlineto{\pgfqpoint{4.399951in}{2.152131in}}%
\pgfpathlineto{\pgfqpoint{4.407858in}{2.113892in}}%
\pgfpathlineto{\pgfqpoint{4.415764in}{2.075740in}}%
\pgfpathlineto{\pgfqpoint{4.423671in}{2.037696in}}%
\pgfpathlineto{\pgfqpoint{4.431578in}{1.999781in}}%
\pgfpathlineto{\pgfqpoint{4.439484in}{1.962016in}}%
\pgfpathlineto{\pgfqpoint{4.447391in}{1.924422in}}%
\pgfpathlineto{\pgfqpoint{4.455298in}{1.887022in}}%
\pgfpathlineto{\pgfqpoint{4.463205in}{1.849836in}}%
\pgfpathlineto{\pgfqpoint{4.471111in}{1.812887in}}%
\pgfpathlineto{\pgfqpoint{4.479018in}{1.776194in}}%
\pgfpathlineto{\pgfqpoint{4.486925in}{1.739779in}}%
\pgfpathlineto{\pgfqpoint{4.494831in}{1.703665in}}%
\pgfpathlineto{\pgfqpoint{4.502738in}{1.667872in}}%
\pgfpathlineto{\pgfqpoint{4.510645in}{1.632421in}}%
\pgfpathlineto{\pgfqpoint{4.518552in}{1.597334in}}%
\pgfpathlineto{\pgfqpoint{4.526458in}{1.562633in}}%
\pgfpathlineto{\pgfqpoint{4.534365in}{1.528338in}}%
\pgfpathlineto{\pgfqpoint{4.542272in}{1.494471in}}%
\pgfpathlineto{\pgfqpoint{4.550179in}{1.461054in}}%
\pgfpathlineto{\pgfqpoint{4.558085in}{1.428107in}}%
\pgfpathlineto{\pgfqpoint{4.565992in}{1.395652in}}%
\pgfpathlineto{\pgfqpoint{4.573899in}{1.363710in}}%
\pgfpathlineto{\pgfqpoint{4.581805in}{1.332303in}}%
\pgfpathlineto{\pgfqpoint{4.589712in}{1.301453in}}%
\pgfpathlineto{\pgfqpoint{4.597619in}{1.271179in}}%
\pgfpathlineto{\pgfqpoint{4.605526in}{1.241505in}}%
\pgfpathlineto{\pgfqpoint{4.613432in}{1.212451in}}%
\pgfpathlineto{\pgfqpoint{4.621339in}{1.184038in}}%
\pgfpathlineto{\pgfqpoint{4.629246in}{1.156288in}}%
\pgfpathlineto{\pgfqpoint{4.637152in}{1.129222in}}%
\pgfpathlineto{\pgfqpoint{4.645059in}{1.102862in}}%
\pgfpathlineto{\pgfqpoint{4.652966in}{1.077229in}}%
\pgfpathlineto{\pgfqpoint{4.660873in}{1.052343in}}%
\pgfpathlineto{\pgfqpoint{4.668779in}{1.028228in}}%
\pgfpathlineto{\pgfqpoint{4.676686in}{1.004903in}}%
\pgfpathlineto{\pgfqpoint{4.684593in}{0.982391in}}%
\pgfpathlineto{\pgfqpoint{4.692500in}{0.960713in}}%
\pgfpathlineto{\pgfqpoint{4.700406in}{0.939889in}}%
\pgfpathlineto{\pgfqpoint{4.708313in}{0.919942in}}%
\pgfpathlineto{\pgfqpoint{4.716220in}{0.900893in}}%
\pgfpathlineto{\pgfqpoint{4.724126in}{0.882762in}}%
\pgfpathlineto{\pgfqpoint{4.732033in}{0.865573in}}%
\pgfpathlineto{\pgfqpoint{4.739940in}{0.849345in}}%
\pgfpathlineto{\pgfqpoint{4.747847in}{0.834100in}}%
\pgfpathlineto{\pgfqpoint{4.755753in}{0.819859in}}%
\pgfpathlineto{\pgfqpoint{4.763660in}{0.806644in}}%
\pgfpathlineto{\pgfqpoint{4.771567in}{0.794477in}}%
\pgfpathlineto{\pgfqpoint{4.779473in}{0.783378in}}%
\pgfpathlineto{\pgfqpoint{4.787380in}{0.773369in}}%
\pgfpathlineto{\pgfqpoint{4.795287in}{0.764471in}}%
\pgfpathlineto{\pgfqpoint{4.803194in}{0.756706in}}%
\pgfpathlineto{\pgfqpoint{4.811100in}{0.750095in}}%
\pgfpathlineto{\pgfqpoint{4.819007in}{0.744659in}}%
\pgfpathlineto{\pgfqpoint{4.826914in}{0.740420in}}%
\pgfpathlineto{\pgfqpoint{4.834821in}{0.737399in}}%
\pgfpathlineto{\pgfqpoint{4.842727in}{0.735617in}}%
\pgfpathlineto{\pgfqpoint{4.842727in}{1.012436in}}%
\pgfpathlineto{\pgfqpoint{4.842727in}{1.012436in}}%
\pgfpathlineto{\pgfqpoint{4.834821in}{1.009832in}}%
\pgfpathlineto{\pgfqpoint{4.826914in}{1.008431in}}%
\pgfpathlineto{\pgfqpoint{4.819007in}{1.008212in}}%
\pgfpathlineto{\pgfqpoint{4.811100in}{1.009157in}}%
\pgfpathlineto{\pgfqpoint{4.803194in}{1.011244in}}%
\pgfpathlineto{\pgfqpoint{4.795287in}{1.014455in}}%
\pgfpathlineto{\pgfqpoint{4.787380in}{1.018771in}}%
\pgfpathlineto{\pgfqpoint{4.779473in}{1.024171in}}%
\pgfpathlineto{\pgfqpoint{4.771567in}{1.030636in}}%
\pgfpathlineto{\pgfqpoint{4.763660in}{1.038146in}}%
\pgfpathlineto{\pgfqpoint{4.755753in}{1.046683in}}%
\pgfpathlineto{\pgfqpoint{4.747847in}{1.056225in}}%
\pgfpathlineto{\pgfqpoint{4.739940in}{1.066754in}}%
\pgfpathlineto{\pgfqpoint{4.732033in}{1.078250in}}%
\pgfpathlineto{\pgfqpoint{4.724126in}{1.090694in}}%
\pgfpathlineto{\pgfqpoint{4.716220in}{1.104066in}}%
\pgfpathlineto{\pgfqpoint{4.708313in}{1.118346in}}%
\pgfpathlineto{\pgfqpoint{4.700406in}{1.133514in}}%
\pgfpathlineto{\pgfqpoint{4.692500in}{1.149552in}}%
\pgfpathlineto{\pgfqpoint{4.684593in}{1.166439in}}%
\pgfpathlineto{\pgfqpoint{4.676686in}{1.184156in}}%
\pgfpathlineto{\pgfqpoint{4.668779in}{1.202684in}}%
\pgfpathlineto{\pgfqpoint{4.660873in}{1.222002in}}%
\pgfpathlineto{\pgfqpoint{4.652966in}{1.242092in}}%
\pgfpathlineto{\pgfqpoint{4.645059in}{1.262933in}}%
\pgfpathlineto{\pgfqpoint{4.637152in}{1.284506in}}%
\pgfpathlineto{\pgfqpoint{4.629246in}{1.306792in}}%
\pgfpathlineto{\pgfqpoint{4.621339in}{1.329770in}}%
\pgfpathlineto{\pgfqpoint{4.613432in}{1.353422in}}%
\pgfpathlineto{\pgfqpoint{4.605526in}{1.377728in}}%
\pgfpathlineto{\pgfqpoint{4.597619in}{1.402667in}}%
\pgfpathlineto{\pgfqpoint{4.589712in}{1.428221in}}%
\pgfpathlineto{\pgfqpoint{4.581805in}{1.454370in}}%
\pgfpathlineto{\pgfqpoint{4.573899in}{1.481095in}}%
\pgfpathlineto{\pgfqpoint{4.565992in}{1.508375in}}%
\pgfpathlineto{\pgfqpoint{4.558085in}{1.536191in}}%
\pgfpathlineto{\pgfqpoint{4.550179in}{1.564524in}}%
\pgfpathlineto{\pgfqpoint{4.542272in}{1.593354in}}%
\pgfpathlineto{\pgfqpoint{4.534365in}{1.622661in}}%
\pgfpathlineto{\pgfqpoint{4.526458in}{1.652426in}}%
\pgfpathlineto{\pgfqpoint{4.518552in}{1.682629in}}%
\pgfpathlineto{\pgfqpoint{4.510645in}{1.713251in}}%
\pgfpathlineto{\pgfqpoint{4.502738in}{1.744272in}}%
\pgfpathlineto{\pgfqpoint{4.494831in}{1.775673in}}%
\pgfpathlineto{\pgfqpoint{4.486925in}{1.807433in}}%
\pgfpathlineto{\pgfqpoint{4.479018in}{1.839534in}}%
\pgfpathlineto{\pgfqpoint{4.471111in}{1.871955in}}%
\pgfpathlineto{\pgfqpoint{4.463205in}{1.904678in}}%
\pgfpathlineto{\pgfqpoint{4.455298in}{1.937682in}}%
\pgfpathlineto{\pgfqpoint{4.447391in}{1.970948in}}%
\pgfpathlineto{\pgfqpoint{4.439484in}{2.004457in}}%
\pgfpathlineto{\pgfqpoint{4.431578in}{2.038188in}}%
\pgfpathlineto{\pgfqpoint{4.423671in}{2.072122in}}%
\pgfpathlineto{\pgfqpoint{4.415764in}{2.106240in}}%
\pgfpathlineto{\pgfqpoint{4.407858in}{2.140523in}}%
\pgfpathlineto{\pgfqpoint{4.399951in}{2.174949in}}%
\pgfpathlineto{\pgfqpoint{4.392044in}{2.209501in}}%
\pgfpathlineto{\pgfqpoint{4.384137in}{2.244157in}}%
\pgfpathlineto{\pgfqpoint{4.376231in}{2.278900in}}%
\pgfpathlineto{\pgfqpoint{4.368324in}{2.313708in}}%
\pgfpathlineto{\pgfqpoint{4.360417in}{2.348563in}}%
\pgfpathlineto{\pgfqpoint{4.352510in}{2.383445in}}%
\pgfpathlineto{\pgfqpoint{4.352510in}{2.383445in}}%
\pgfpathclose%
\pgfusepath{stroke,fill}%
\end{pgfscope}%
\begin{pgfscope}%
\pgfpathrectangle{\pgfqpoint{0.700000in}{0.495000in}}{\pgfqpoint{4.340000in}{3.465000in}}%
\pgfusepath{clip}%
\pgfsetbuttcap%
\pgfsetroundjoin%
\definecolor{currentfill}{rgb}{0.839216,0.152941,0.156863}%
\pgfsetfillcolor{currentfill}%
\pgfsetfillopacity{0.300000}%
\pgfsetlinewidth{1.003750pt}%
\definecolor{currentstroke}{rgb}{0.839216,0.152941,0.156863}%
\pgfsetstrokecolor{currentstroke}%
\pgfsetstrokeopacity{0.300000}%
\pgfsetdash{}{0pt}%
\pgfpathmoveto{\pgfqpoint{1.395396in}{0.732228in}}%
\pgfpathlineto{\pgfqpoint{1.395396in}{0.737179in}}%
\pgfpathlineto{\pgfqpoint{1.403303in}{0.739797in}}%
\pgfpathlineto{\pgfqpoint{1.411210in}{0.742561in}}%
\pgfpathlineto{\pgfqpoint{1.419116in}{0.745476in}}%
\pgfpathlineto{\pgfqpoint{1.427023in}{0.748544in}}%
\pgfpathlineto{\pgfqpoint{1.434930in}{0.751769in}}%
\pgfpathlineto{\pgfqpoint{1.442837in}{0.755153in}}%
\pgfpathlineto{\pgfqpoint{1.450743in}{0.758700in}}%
\pgfpathlineto{\pgfqpoint{1.458650in}{0.762412in}}%
\pgfpathlineto{\pgfqpoint{1.466557in}{0.766293in}}%
\pgfpathlineto{\pgfqpoint{1.474463in}{0.770346in}}%
\pgfpathlineto{\pgfqpoint{1.482370in}{0.774574in}}%
\pgfpathlineto{\pgfqpoint{1.490277in}{0.778979in}}%
\pgfpathlineto{\pgfqpoint{1.498184in}{0.783565in}}%
\pgfpathlineto{\pgfqpoint{1.506090in}{0.788336in}}%
\pgfpathlineto{\pgfqpoint{1.513997in}{0.793293in}}%
\pgfpathlineto{\pgfqpoint{1.521904in}{0.798440in}}%
\pgfpathlineto{\pgfqpoint{1.529811in}{0.803781in}}%
\pgfpathlineto{\pgfqpoint{1.537717in}{0.809317in}}%
\pgfpathlineto{\pgfqpoint{1.545624in}{0.815053in}}%
\pgfpathlineto{\pgfqpoint{1.553531in}{0.820992in}}%
\pgfpathlineto{\pgfqpoint{1.561437in}{0.827135in}}%
\pgfpathlineto{\pgfqpoint{1.569344in}{0.833487in}}%
\pgfpathlineto{\pgfqpoint{1.577251in}{0.840051in}}%
\pgfpathlineto{\pgfqpoint{1.585158in}{0.846829in}}%
\pgfpathlineto{\pgfqpoint{1.593064in}{0.853825in}}%
\pgfpathlineto{\pgfqpoint{1.600971in}{0.861041in}}%
\pgfpathlineto{\pgfqpoint{1.608878in}{0.868481in}}%
\pgfpathlineto{\pgfqpoint{1.616784in}{0.876148in}}%
\pgfpathlineto{\pgfqpoint{1.624691in}{0.884045in}}%
\pgfpathlineto{\pgfqpoint{1.632598in}{0.892174in}}%
\pgfpathlineto{\pgfqpoint{1.640505in}{0.900540in}}%
\pgfpathlineto{\pgfqpoint{1.648411in}{0.909144in}}%
\pgfpathlineto{\pgfqpoint{1.656318in}{0.917991in}}%
\pgfpathlineto{\pgfqpoint{1.664225in}{0.927083in}}%
\pgfpathlineto{\pgfqpoint{1.672132in}{0.936422in}}%
\pgfpathlineto{\pgfqpoint{1.680038in}{0.946014in}}%
\pgfpathlineto{\pgfqpoint{1.687945in}{0.955859in}}%
\pgfpathlineto{\pgfqpoint{1.695852in}{0.965962in}}%
\pgfpathlineto{\pgfqpoint{1.703758in}{0.976325in}}%
\pgfpathlineto{\pgfqpoint{1.711665in}{0.986952in}}%
\pgfpathlineto{\pgfqpoint{1.719572in}{0.997845in}}%
\pgfpathlineto{\pgfqpoint{1.727479in}{1.009007in}}%
\pgfpathlineto{\pgfqpoint{1.735385in}{1.020443in}}%
\pgfpathlineto{\pgfqpoint{1.743292in}{1.032154in}}%
\pgfpathlineto{\pgfqpoint{1.751199in}{1.044144in}}%
\pgfpathlineto{\pgfqpoint{1.759105in}{1.056416in}}%
\pgfpathlineto{\pgfqpoint{1.767012in}{1.068972in}}%
\pgfpathlineto{\pgfqpoint{1.774919in}{1.081817in}}%
\pgfpathlineto{\pgfqpoint{1.782826in}{1.094952in}}%
\pgfpathlineto{\pgfqpoint{1.790732in}{1.108382in}}%
\pgfpathlineto{\pgfqpoint{1.798639in}{1.122108in}}%
\pgfpathlineto{\pgfqpoint{1.806546in}{1.136135in}}%
\pgfpathlineto{\pgfqpoint{1.814453in}{1.150465in}}%
\pgfpathlineto{\pgfqpoint{1.822359in}{1.165101in}}%
\pgfpathlineto{\pgfqpoint{1.830266in}{1.180047in}}%
\pgfpathlineto{\pgfqpoint{1.838173in}{1.195305in}}%
\pgfpathlineto{\pgfqpoint{1.846079in}{1.210878in}}%
\pgfpathlineto{\pgfqpoint{1.853986in}{1.226770in}}%
\pgfpathlineto{\pgfqpoint{1.861893in}{1.242984in}}%
\pgfpathlineto{\pgfqpoint{1.869800in}{1.259522in}}%
\pgfpathlineto{\pgfqpoint{1.877706in}{1.276388in}}%
\pgfpathlineto{\pgfqpoint{1.885613in}{1.293585in}}%
\pgfpathlineto{\pgfqpoint{1.893520in}{1.311111in}}%
\pgfpathlineto{\pgfqpoint{1.901426in}{1.328960in}}%
\pgfpathlineto{\pgfqpoint{1.909333in}{1.347124in}}%
\pgfpathlineto{\pgfqpoint{1.917240in}{1.365592in}}%
\pgfpathlineto{\pgfqpoint{1.925147in}{1.384358in}}%
\pgfpathlineto{\pgfqpoint{1.933053in}{1.403413in}}%
\pgfpathlineto{\pgfqpoint{1.940960in}{1.422748in}}%
\pgfpathlineto{\pgfqpoint{1.948867in}{1.442355in}}%
\pgfpathlineto{\pgfqpoint{1.956774in}{1.462225in}}%
\pgfpathlineto{\pgfqpoint{1.964680in}{1.482350in}}%
\pgfpathlineto{\pgfqpoint{1.972587in}{1.502721in}}%
\pgfpathlineto{\pgfqpoint{1.980494in}{1.523331in}}%
\pgfpathlineto{\pgfqpoint{1.988400in}{1.544170in}}%
\pgfpathlineto{\pgfqpoint{1.996307in}{1.565230in}}%
\pgfpathlineto{\pgfqpoint{2.004214in}{1.586503in}}%
\pgfpathlineto{\pgfqpoint{2.012121in}{1.607980in}}%
\pgfpathlineto{\pgfqpoint{2.020027in}{1.629653in}}%
\pgfpathlineto{\pgfqpoint{2.027934in}{1.651513in}}%
\pgfpathlineto{\pgfqpoint{2.035841in}{1.673552in}}%
\pgfpathlineto{\pgfqpoint{2.043747in}{1.695762in}}%
\pgfpathlineto{\pgfqpoint{2.051654in}{1.718134in}}%
\pgfpathlineto{\pgfqpoint{2.059561in}{1.740659in}}%
\pgfpathlineto{\pgfqpoint{2.067468in}{1.763329in}}%
\pgfpathlineto{\pgfqpoint{2.075374in}{1.786136in}}%
\pgfpathlineto{\pgfqpoint{2.083281in}{1.809071in}}%
\pgfpathlineto{\pgfqpoint{2.091188in}{1.832126in}}%
\pgfpathlineto{\pgfqpoint{2.099095in}{1.855293in}}%
\pgfpathlineto{\pgfqpoint{2.107001in}{1.878562in}}%
\pgfpathlineto{\pgfqpoint{2.114908in}{1.901926in}}%
\pgfpathlineto{\pgfqpoint{2.114908in}{1.897616in}}%
\pgfpathlineto{\pgfqpoint{2.114908in}{1.897616in}}%
\pgfpathlineto{\pgfqpoint{2.107001in}{1.862670in}}%
\pgfpathlineto{\pgfqpoint{2.099095in}{1.827830in}}%
\pgfpathlineto{\pgfqpoint{2.091188in}{1.793120in}}%
\pgfpathlineto{\pgfqpoint{2.083281in}{1.758566in}}%
\pgfpathlineto{\pgfqpoint{2.075374in}{1.724192in}}%
\pgfpathlineto{\pgfqpoint{2.067468in}{1.690023in}}%
\pgfpathlineto{\pgfqpoint{2.059561in}{1.656083in}}%
\pgfpathlineto{\pgfqpoint{2.051654in}{1.622399in}}%
\pgfpathlineto{\pgfqpoint{2.043747in}{1.588994in}}%
\pgfpathlineto{\pgfqpoint{2.035841in}{1.555894in}}%
\pgfpathlineto{\pgfqpoint{2.027934in}{1.523124in}}%
\pgfpathlineto{\pgfqpoint{2.020027in}{1.490707in}}%
\pgfpathlineto{\pgfqpoint{2.012121in}{1.458670in}}%
\pgfpathlineto{\pgfqpoint{2.004214in}{1.427037in}}%
\pgfpathlineto{\pgfqpoint{1.996307in}{1.395833in}}%
\pgfpathlineto{\pgfqpoint{1.988400in}{1.365083in}}%
\pgfpathlineto{\pgfqpoint{1.980494in}{1.334812in}}%
\pgfpathlineto{\pgfqpoint{1.972587in}{1.305044in}}%
\pgfpathlineto{\pgfqpoint{1.964680in}{1.275805in}}%
\pgfpathlineto{\pgfqpoint{1.956774in}{1.247120in}}%
\pgfpathlineto{\pgfqpoint{1.948867in}{1.219012in}}%
\pgfpathlineto{\pgfqpoint{1.940960in}{1.191508in}}%
\pgfpathlineto{\pgfqpoint{1.933053in}{1.164632in}}%
\pgfpathlineto{\pgfqpoint{1.925147in}{1.138409in}}%
\pgfpathlineto{\pgfqpoint{1.917240in}{1.112863in}}%
\pgfpathlineto{\pgfqpoint{1.909333in}{1.088021in}}%
\pgfpathlineto{\pgfqpoint{1.901426in}{1.063905in}}%
\pgfpathlineto{\pgfqpoint{1.893520in}{1.040543in}}%
\pgfpathlineto{\pgfqpoint{1.885613in}{1.017958in}}%
\pgfpathlineto{\pgfqpoint{1.877706in}{0.996172in}}%
\pgfpathlineto{\pgfqpoint{1.869800in}{0.975186in}}%
\pgfpathlineto{\pgfqpoint{1.861893in}{0.954988in}}%
\pgfpathlineto{\pgfqpoint{1.853986in}{0.935565in}}%
\pgfpathlineto{\pgfqpoint{1.846079in}{0.916907in}}%
\pgfpathlineto{\pgfqpoint{1.838173in}{0.899001in}}%
\pgfpathlineto{\pgfqpoint{1.830266in}{0.881836in}}%
\pgfpathlineto{\pgfqpoint{1.822359in}{0.865399in}}%
\pgfpathlineto{\pgfqpoint{1.814453in}{0.849679in}}%
\pgfpathlineto{\pgfqpoint{1.806546in}{0.834664in}}%
\pgfpathlineto{\pgfqpoint{1.798639in}{0.820343in}}%
\pgfpathlineto{\pgfqpoint{1.790732in}{0.806702in}}%
\pgfpathlineto{\pgfqpoint{1.782826in}{0.793732in}}%
\pgfpathlineto{\pgfqpoint{1.774919in}{0.781419in}}%
\pgfpathlineto{\pgfqpoint{1.767012in}{0.769752in}}%
\pgfpathlineto{\pgfqpoint{1.759105in}{0.758719in}}%
\pgfpathlineto{\pgfqpoint{1.751199in}{0.748308in}}%
\pgfpathlineto{\pgfqpoint{1.743292in}{0.738508in}}%
\pgfpathlineto{\pgfqpoint{1.735385in}{0.729307in}}%
\pgfpathlineto{\pgfqpoint{1.727479in}{0.720692in}}%
\pgfpathlineto{\pgfqpoint{1.719572in}{0.712652in}}%
\pgfpathlineto{\pgfqpoint{1.711665in}{0.705176in}}%
\pgfpathlineto{\pgfqpoint{1.703758in}{0.698251in}}%
\pgfpathlineto{\pgfqpoint{1.695852in}{0.691865in}}%
\pgfpathlineto{\pgfqpoint{1.687945in}{0.686007in}}%
\pgfpathlineto{\pgfqpoint{1.680038in}{0.680666in}}%
\pgfpathlineto{\pgfqpoint{1.672132in}{0.675828in}}%
\pgfpathlineto{\pgfqpoint{1.664225in}{0.671482in}}%
\pgfpathlineto{\pgfqpoint{1.656318in}{0.667617in}}%
\pgfpathlineto{\pgfqpoint{1.648411in}{0.664220in}}%
\pgfpathlineto{\pgfqpoint{1.640505in}{0.661281in}}%
\pgfpathlineto{\pgfqpoint{1.632598in}{0.658786in}}%
\pgfpathlineto{\pgfqpoint{1.624691in}{0.656724in}}%
\pgfpathlineto{\pgfqpoint{1.616784in}{0.655084in}}%
\pgfpathlineto{\pgfqpoint{1.608878in}{0.653853in}}%
\pgfpathlineto{\pgfqpoint{1.600971in}{0.653020in}}%
\pgfpathlineto{\pgfqpoint{1.593064in}{0.652573in}}%
\pgfpathlineto{\pgfqpoint{1.585158in}{0.652500in}}%
\pgfpathlineto{\pgfqpoint{1.577251in}{0.652789in}}%
\pgfpathlineto{\pgfqpoint{1.569344in}{0.653429in}}%
\pgfpathlineto{\pgfqpoint{1.561437in}{0.654407in}}%
\pgfpathlineto{\pgfqpoint{1.553531in}{0.655712in}}%
\pgfpathlineto{\pgfqpoint{1.545624in}{0.657331in}}%
\pgfpathlineto{\pgfqpoint{1.537717in}{0.659254in}}%
\pgfpathlineto{\pgfqpoint{1.529811in}{0.661469in}}%
\pgfpathlineto{\pgfqpoint{1.521904in}{0.663963in}}%
\pgfpathlineto{\pgfqpoint{1.513997in}{0.666724in}}%
\pgfpathlineto{\pgfqpoint{1.506090in}{0.669741in}}%
\pgfpathlineto{\pgfqpoint{1.498184in}{0.673003in}}%
\pgfpathlineto{\pgfqpoint{1.490277in}{0.676496in}}%
\pgfpathlineto{\pgfqpoint{1.482370in}{0.680210in}}%
\pgfpathlineto{\pgfqpoint{1.474463in}{0.684132in}}%
\pgfpathlineto{\pgfqpoint{1.466557in}{0.688252in}}%
\pgfpathlineto{\pgfqpoint{1.458650in}{0.692556in}}%
\pgfpathlineto{\pgfqpoint{1.450743in}{0.697033in}}%
\pgfpathlineto{\pgfqpoint{1.442837in}{0.701672in}}%
\pgfpathlineto{\pgfqpoint{1.434930in}{0.706460in}}%
\pgfpathlineto{\pgfqpoint{1.427023in}{0.711385in}}%
\pgfpathlineto{\pgfqpoint{1.419116in}{0.716437in}}%
\pgfpathlineto{\pgfqpoint{1.411210in}{0.721602in}}%
\pgfpathlineto{\pgfqpoint{1.403303in}{0.726870in}}%
\pgfpathlineto{\pgfqpoint{1.395396in}{0.732228in}}%
\pgfpathlineto{\pgfqpoint{1.395396in}{0.732228in}}%
\pgfpathclose%
\pgfusepath{stroke,fill}%
\end{pgfscope}%
\begin{pgfscope}%
\pgfpathrectangle{\pgfqpoint{0.700000in}{0.495000in}}{\pgfqpoint{4.340000in}{3.465000in}}%
\pgfusepath{clip}%
\pgfsetbuttcap%
\pgfsetroundjoin%
\definecolor{currentfill}{rgb}{0.839216,0.152941,0.156863}%
\pgfsetfillcolor{currentfill}%
\pgfsetfillopacity{0.300000}%
\pgfsetlinewidth{1.003750pt}%
\definecolor{currentstroke}{rgb}{0.839216,0.152941,0.156863}%
\pgfsetstrokecolor{currentstroke}%
\pgfsetstrokeopacity{0.300000}%
\pgfsetdash{}{0pt}%
\pgfpathmoveto{\pgfqpoint{2.873953in}{3.221179in}}%
\pgfpathlineto{\pgfqpoint{2.873953in}{3.221592in}}%
\pgfpathlineto{\pgfqpoint{2.881860in}{3.210196in}}%
\pgfpathlineto{\pgfqpoint{2.889767in}{3.198015in}}%
\pgfpathlineto{\pgfqpoint{2.897674in}{3.185085in}}%
\pgfpathlineto{\pgfqpoint{2.897674in}{3.184428in}}%
\pgfpathlineto{\pgfqpoint{2.897674in}{3.184428in}}%
\pgfpathlineto{\pgfqpoint{2.889767in}{3.197024in}}%
\pgfpathlineto{\pgfqpoint{2.881860in}{3.209282in}}%
\pgfpathlineto{\pgfqpoint{2.873953in}{3.221179in}}%
\pgfpathlineto{\pgfqpoint{2.873953in}{3.221179in}}%
\pgfpathclose%
\pgfusepath{stroke,fill}%
\end{pgfscope}%
\begin{pgfscope}%
\pgfpathrectangle{\pgfqpoint{0.700000in}{0.495000in}}{\pgfqpoint{4.340000in}{3.465000in}}%
\pgfusepath{clip}%
\pgfsetbuttcap%
\pgfsetroundjoin%
\definecolor{currentfill}{rgb}{0.839216,0.152941,0.156863}%
\pgfsetfillcolor{currentfill}%
\pgfsetfillopacity{0.300000}%
\pgfsetlinewidth{1.003750pt}%
\definecolor{currentstroke}{rgb}{0.839216,0.152941,0.156863}%
\pgfsetstrokecolor{currentstroke}%
\pgfsetstrokeopacity{0.300000}%
\pgfsetdash{}{0pt}%
\pgfpathmoveto{\pgfqpoint{3.862294in}{3.782688in}}%
\pgfpathlineto{\pgfqpoint{3.862294in}{3.786019in}}%
\pgfpathlineto{\pgfqpoint{3.870200in}{3.792004in}}%
\pgfpathlineto{\pgfqpoint{3.878107in}{3.796623in}}%
\pgfpathlineto{\pgfqpoint{3.886014in}{3.799898in}}%
\pgfpathlineto{\pgfqpoint{3.893921in}{3.801850in}}%
\pgfpathlineto{\pgfqpoint{3.901827in}{3.802500in}}%
\pgfpathlineto{\pgfqpoint{3.909734in}{3.801870in}}%
\pgfpathlineto{\pgfqpoint{3.917641in}{3.799981in}}%
\pgfpathlineto{\pgfqpoint{3.925547in}{3.796855in}}%
\pgfpathlineto{\pgfqpoint{3.933454in}{3.792512in}}%
\pgfpathlineto{\pgfqpoint{3.941361in}{3.786975in}}%
\pgfpathlineto{\pgfqpoint{3.949268in}{3.780264in}}%
\pgfpathlineto{\pgfqpoint{3.957174in}{3.772401in}}%
\pgfpathlineto{\pgfqpoint{3.965081in}{3.763407in}}%
\pgfpathlineto{\pgfqpoint{3.972988in}{3.753303in}}%
\pgfpathlineto{\pgfqpoint{3.980895in}{3.742112in}}%
\pgfpathlineto{\pgfqpoint{3.988801in}{3.729854in}}%
\pgfpathlineto{\pgfqpoint{3.996708in}{3.716551in}}%
\pgfpathlineto{\pgfqpoint{4.004615in}{3.702223in}}%
\pgfpathlineto{\pgfqpoint{4.012521in}{3.686893in}}%
\pgfpathlineto{\pgfqpoint{4.020428in}{3.670582in}}%
\pgfpathlineto{\pgfqpoint{4.028335in}{3.653310in}}%
\pgfpathlineto{\pgfqpoint{4.036242in}{3.635100in}}%
\pgfpathlineto{\pgfqpoint{4.044148in}{3.615973in}}%
\pgfpathlineto{\pgfqpoint{4.052055in}{3.595950in}}%
\pgfpathlineto{\pgfqpoint{4.059962in}{3.575053in}}%
\pgfpathlineto{\pgfqpoint{4.067868in}{3.553302in}}%
\pgfpathlineto{\pgfqpoint{4.075775in}{3.530719in}}%
\pgfpathlineto{\pgfqpoint{4.083682in}{3.507326in}}%
\pgfpathlineto{\pgfqpoint{4.091589in}{3.483144in}}%
\pgfpathlineto{\pgfqpoint{4.099495in}{3.458194in}}%
\pgfpathlineto{\pgfqpoint{4.107402in}{3.432497in}}%
\pgfpathlineto{\pgfqpoint{4.115309in}{3.406076in}}%
\pgfpathlineto{\pgfqpoint{4.123216in}{3.378950in}}%
\pgfpathlineto{\pgfqpoint{4.131122in}{3.351143in}}%
\pgfpathlineto{\pgfqpoint{4.139029in}{3.322674in}}%
\pgfpathlineto{\pgfqpoint{4.146936in}{3.293566in}}%
\pgfpathlineto{\pgfqpoint{4.154842in}{3.263839in}}%
\pgfpathlineto{\pgfqpoint{4.162749in}{3.233515in}}%
\pgfpathlineto{\pgfqpoint{4.170656in}{3.202616in}}%
\pgfpathlineto{\pgfqpoint{4.178563in}{3.171163in}}%
\pgfpathlineto{\pgfqpoint{4.186469in}{3.139176in}}%
\pgfpathlineto{\pgfqpoint{4.194376in}{3.106679in}}%
\pgfpathlineto{\pgfqpoint{4.202283in}{3.073690in}}%
\pgfpathlineto{\pgfqpoint{4.210189in}{3.040233in}}%
\pgfpathlineto{\pgfqpoint{4.218096in}{3.006329in}}%
\pgfpathlineto{\pgfqpoint{4.226003in}{2.971998in}}%
\pgfpathlineto{\pgfqpoint{4.233910in}{2.937263in}}%
\pgfpathlineto{\pgfqpoint{4.241816in}{2.902144in}}%
\pgfpathlineto{\pgfqpoint{4.249723in}{2.866663in}}%
\pgfpathlineto{\pgfqpoint{4.257630in}{2.830842in}}%
\pgfpathlineto{\pgfqpoint{4.265537in}{2.794701in}}%
\pgfpathlineto{\pgfqpoint{4.273443in}{2.758262in}}%
\pgfpathlineto{\pgfqpoint{4.281350in}{2.721546in}}%
\pgfpathlineto{\pgfqpoint{4.289257in}{2.684575in}}%
\pgfpathlineto{\pgfqpoint{4.297163in}{2.647370in}}%
\pgfpathlineto{\pgfqpoint{4.305070in}{2.609953in}}%
\pgfpathlineto{\pgfqpoint{4.312977in}{2.572344in}}%
\pgfpathlineto{\pgfqpoint{4.320884in}{2.534565in}}%
\pgfpathlineto{\pgfqpoint{4.328790in}{2.496638in}}%
\pgfpathlineto{\pgfqpoint{4.336697in}{2.458583in}}%
\pgfpathlineto{\pgfqpoint{4.344604in}{2.420423in}}%
\pgfpathlineto{\pgfqpoint{4.344604in}{2.418334in}}%
\pgfpathlineto{\pgfqpoint{4.344604in}{2.418334in}}%
\pgfpathlineto{\pgfqpoint{4.336697in}{2.453211in}}%
\pgfpathlineto{\pgfqpoint{4.328790in}{2.488057in}}%
\pgfpathlineto{\pgfqpoint{4.320884in}{2.522850in}}%
\pgfpathlineto{\pgfqpoint{4.312977in}{2.557573in}}%
\pgfpathlineto{\pgfqpoint{4.305070in}{2.592205in}}%
\pgfpathlineto{\pgfqpoint{4.297163in}{2.626727in}}%
\pgfpathlineto{\pgfqpoint{4.289257in}{2.661120in}}%
\pgfpathlineto{\pgfqpoint{4.281350in}{2.695363in}}%
\pgfpathlineto{\pgfqpoint{4.273443in}{2.729437in}}%
\pgfpathlineto{\pgfqpoint{4.265537in}{2.763322in}}%
\pgfpathlineto{\pgfqpoint{4.257630in}{2.797000in}}%
\pgfpathlineto{\pgfqpoint{4.249723in}{2.830449in}}%
\pgfpathlineto{\pgfqpoint{4.241816in}{2.863652in}}%
\pgfpathlineto{\pgfqpoint{4.233910in}{2.896587in}}%
\pgfpathlineto{\pgfqpoint{4.226003in}{2.929236in}}%
\pgfpathlineto{\pgfqpoint{4.218096in}{2.961579in}}%
\pgfpathlineto{\pgfqpoint{4.210189in}{2.993597in}}%
\pgfpathlineto{\pgfqpoint{4.202283in}{3.025269in}}%
\pgfpathlineto{\pgfqpoint{4.194376in}{3.056577in}}%
\pgfpathlineto{\pgfqpoint{4.186469in}{3.087500in}}%
\pgfpathlineto{\pgfqpoint{4.178563in}{3.118019in}}%
\pgfpathlineto{\pgfqpoint{4.170656in}{3.148115in}}%
\pgfpathlineto{\pgfqpoint{4.162749in}{3.177767in}}%
\pgfpathlineto{\pgfqpoint{4.154842in}{3.206957in}}%
\pgfpathlineto{\pgfqpoint{4.146936in}{3.235664in}}%
\pgfpathlineto{\pgfqpoint{4.139029in}{3.263870in}}%
\pgfpathlineto{\pgfqpoint{4.131122in}{3.291554in}}%
\pgfpathlineto{\pgfqpoint{4.123216in}{3.318697in}}%
\pgfpathlineto{\pgfqpoint{4.115309in}{3.345279in}}%
\pgfpathlineto{\pgfqpoint{4.107402in}{3.371281in}}%
\pgfpathlineto{\pgfqpoint{4.099495in}{3.396684in}}%
\pgfpathlineto{\pgfqpoint{4.091589in}{3.421466in}}%
\pgfpathlineto{\pgfqpoint{4.083682in}{3.445610in}}%
\pgfpathlineto{\pgfqpoint{4.075775in}{3.469096in}}%
\pgfpathlineto{\pgfqpoint{4.067868in}{3.491903in}}%
\pgfpathlineto{\pgfqpoint{4.059962in}{3.514012in}}%
\pgfpathlineto{\pgfqpoint{4.052055in}{3.535404in}}%
\pgfpathlineto{\pgfqpoint{4.044148in}{3.556059in}}%
\pgfpathlineto{\pgfqpoint{4.036242in}{3.575958in}}%
\pgfpathlineto{\pgfqpoint{4.028335in}{3.595080in}}%
\pgfpathlineto{\pgfqpoint{4.020428in}{3.613407in}}%
\pgfpathlineto{\pgfqpoint{4.012521in}{3.630919in}}%
\pgfpathlineto{\pgfqpoint{4.004615in}{3.647595in}}%
\pgfpathlineto{\pgfqpoint{3.996708in}{3.663418in}}%
\pgfpathlineto{\pgfqpoint{3.988801in}{3.678366in}}%
\pgfpathlineto{\pgfqpoint{3.980895in}{3.692421in}}%
\pgfpathlineto{\pgfqpoint{3.972988in}{3.705562in}}%
\pgfpathlineto{\pgfqpoint{3.965081in}{3.717771in}}%
\pgfpathlineto{\pgfqpoint{3.957174in}{3.729027in}}%
\pgfpathlineto{\pgfqpoint{3.949268in}{3.739311in}}%
\pgfpathlineto{\pgfqpoint{3.941361in}{3.748604in}}%
\pgfpathlineto{\pgfqpoint{3.933454in}{3.756886in}}%
\pgfpathlineto{\pgfqpoint{3.925547in}{3.764136in}}%
\pgfpathlineto{\pgfqpoint{3.917641in}{3.770337in}}%
\pgfpathlineto{\pgfqpoint{3.909734in}{3.775468in}}%
\pgfpathlineto{\pgfqpoint{3.901827in}{3.779509in}}%
\pgfpathlineto{\pgfqpoint{3.893921in}{3.782441in}}%
\pgfpathlineto{\pgfqpoint{3.886014in}{3.784245in}}%
\pgfpathlineto{\pgfqpoint{3.878107in}{3.784900in}}%
\pgfpathlineto{\pgfqpoint{3.870200in}{3.784388in}}%
\pgfpathlineto{\pgfqpoint{3.862294in}{3.782688in}}%
\pgfpathlineto{\pgfqpoint{3.862294in}{3.782688in}}%
\pgfpathclose%
\pgfusepath{stroke,fill}%
\end{pgfscope}%
\begin{pgfscope}%
\pgfpathrectangle{\pgfqpoint{0.700000in}{0.495000in}}{\pgfqpoint{4.340000in}{3.465000in}}%
\pgfusepath{clip}%
\pgfsetroundcap%
\pgfsetroundjoin%
\pgfsetlinewidth{1.505625pt}%
\definecolor{currentstroke}{rgb}{0.298039,0.447059,0.690196}%
\pgfsetstrokecolor{currentstroke}%
\pgfsetdash{}{0pt}%
\pgfpathmoveto{\pgfqpoint{0.897273in}{0.735617in}}%
\pgfpathlineto{\pgfqpoint{0.928900in}{0.733373in}}%
\pgfpathlineto{\pgfqpoint{0.960527in}{0.730550in}}%
\pgfpathlineto{\pgfqpoint{1.007967in}{0.725657in}}%
\pgfpathlineto{\pgfqpoint{1.087034in}{0.717379in}}%
\pgfpathlineto{\pgfqpoint{1.118661in}{0.714595in}}%
\pgfpathlineto{\pgfqpoint{1.150288in}{0.712406in}}%
\pgfpathlineto{\pgfqpoint{1.174008in}{0.711272in}}%
\pgfpathlineto{\pgfqpoint{1.197728in}{0.710666in}}%
\pgfpathlineto{\pgfqpoint{1.221448in}{0.710670in}}%
\pgfpathlineto{\pgfqpoint{1.245169in}{0.711366in}}%
\pgfpathlineto{\pgfqpoint{1.268889in}{0.712837in}}%
\pgfpathlineto{\pgfqpoint{1.284702in}{0.714289in}}%
\pgfpathlineto{\pgfqpoint{1.300516in}{0.716146in}}%
\pgfpathlineto{\pgfqpoint{1.316329in}{0.718434in}}%
\pgfpathlineto{\pgfqpoint{1.332142in}{0.721176in}}%
\pgfpathlineto{\pgfqpoint{1.347956in}{0.724397in}}%
\pgfpathlineto{\pgfqpoint{1.363769in}{0.728122in}}%
\pgfpathlineto{\pgfqpoint{1.379583in}{0.732374in}}%
\pgfpathlineto{\pgfqpoint{1.395396in}{0.737179in}}%
\pgfpathlineto{\pgfqpoint{1.411210in}{0.742561in}}%
\pgfpathlineto{\pgfqpoint{1.427023in}{0.748544in}}%
\pgfpathlineto{\pgfqpoint{1.442837in}{0.755153in}}%
\pgfpathlineto{\pgfqpoint{1.458650in}{0.762412in}}%
\pgfpathlineto{\pgfqpoint{1.474463in}{0.770346in}}%
\pgfpathlineto{\pgfqpoint{1.490277in}{0.778979in}}%
\pgfpathlineto{\pgfqpoint{1.506090in}{0.788336in}}%
\pgfpathlineto{\pgfqpoint{1.521904in}{0.798440in}}%
\pgfpathlineto{\pgfqpoint{1.537717in}{0.809317in}}%
\pgfpathlineto{\pgfqpoint{1.553531in}{0.820992in}}%
\pgfpathlineto{\pgfqpoint{1.569344in}{0.833487in}}%
\pgfpathlineto{\pgfqpoint{1.585158in}{0.846829in}}%
\pgfpathlineto{\pgfqpoint{1.600971in}{0.861041in}}%
\pgfpathlineto{\pgfqpoint{1.616784in}{0.876148in}}%
\pgfpathlineto{\pgfqpoint{1.632598in}{0.892174in}}%
\pgfpathlineto{\pgfqpoint{1.648411in}{0.909144in}}%
\pgfpathlineto{\pgfqpoint{1.664225in}{0.927083in}}%
\pgfpathlineto{\pgfqpoint{1.680038in}{0.946014in}}%
\pgfpathlineto{\pgfqpoint{1.695852in}{0.965962in}}%
\pgfpathlineto{\pgfqpoint{1.711665in}{0.986952in}}%
\pgfpathlineto{\pgfqpoint{1.727479in}{1.009007in}}%
\pgfpathlineto{\pgfqpoint{1.743292in}{1.032154in}}%
\pgfpathlineto{\pgfqpoint{1.759105in}{1.056416in}}%
\pgfpathlineto{\pgfqpoint{1.774919in}{1.081817in}}%
\pgfpathlineto{\pgfqpoint{1.790732in}{1.108382in}}%
\pgfpathlineto{\pgfqpoint{1.806546in}{1.136135in}}%
\pgfpathlineto{\pgfqpoint{1.822359in}{1.165101in}}%
\pgfpathlineto{\pgfqpoint{1.838173in}{1.195305in}}%
\pgfpathlineto{\pgfqpoint{1.853986in}{1.226770in}}%
\pgfpathlineto{\pgfqpoint{1.869800in}{1.259522in}}%
\pgfpathlineto{\pgfqpoint{1.885613in}{1.293585in}}%
\pgfpathlineto{\pgfqpoint{1.901426in}{1.328960in}}%
\pgfpathlineto{\pgfqpoint{1.917240in}{1.365592in}}%
\pgfpathlineto{\pgfqpoint{1.933053in}{1.403413in}}%
\pgfpathlineto{\pgfqpoint{1.948867in}{1.442355in}}%
\pgfpathlineto{\pgfqpoint{1.964680in}{1.482350in}}%
\pgfpathlineto{\pgfqpoint{1.988400in}{1.544170in}}%
\pgfpathlineto{\pgfqpoint{2.012121in}{1.607980in}}%
\pgfpathlineto{\pgfqpoint{2.035841in}{1.673552in}}%
\pgfpathlineto{\pgfqpoint{2.059561in}{1.740659in}}%
\pgfpathlineto{\pgfqpoint{2.091188in}{1.832126in}}%
\pgfpathlineto{\pgfqpoint{2.122815in}{1.925376in}}%
\pgfpathlineto{\pgfqpoint{2.170255in}{2.067407in}}%
\pgfpathlineto{\pgfqpoint{2.265136in}{2.352551in}}%
\pgfpathlineto{\pgfqpoint{2.296763in}{2.445937in}}%
\pgfpathlineto{\pgfqpoint{2.328390in}{2.537591in}}%
\pgfpathlineto{\pgfqpoint{2.352110in}{2.604870in}}%
\pgfpathlineto{\pgfqpoint{2.375830in}{2.670643in}}%
\pgfpathlineto{\pgfqpoint{2.399550in}{2.734674in}}%
\pgfpathlineto{\pgfqpoint{2.423270in}{2.796694in}}%
\pgfpathlineto{\pgfqpoint{2.439084in}{2.836785in}}%
\pgfpathlineto{\pgfqpoint{2.454897in}{2.875775in}}%
\pgfpathlineto{\pgfqpoint{2.470711in}{2.913583in}}%
\pgfpathlineto{\pgfqpoint{2.486524in}{2.950124in}}%
\pgfpathlineto{\pgfqpoint{2.502337in}{2.985316in}}%
\pgfpathlineto{\pgfqpoint{2.518151in}{3.019076in}}%
\pgfpathlineto{\pgfqpoint{2.533964in}{3.051322in}}%
\pgfpathlineto{\pgfqpoint{2.549778in}{3.081971in}}%
\pgfpathlineto{\pgfqpoint{2.565591in}{3.110940in}}%
\pgfpathlineto{\pgfqpoint{2.581405in}{3.138147in}}%
\pgfpathlineto{\pgfqpoint{2.597218in}{3.163508in}}%
\pgfpathlineto{\pgfqpoint{2.613032in}{3.186942in}}%
\pgfpathlineto{\pgfqpoint{2.620938in}{3.197910in}}%
\pgfpathlineto{\pgfqpoint{2.628845in}{3.208365in}}%
\pgfpathlineto{\pgfqpoint{2.636752in}{3.218296in}}%
\pgfpathlineto{\pgfqpoint{2.644658in}{3.227694in}}%
\pgfpathlineto{\pgfqpoint{2.652565in}{3.236548in}}%
\pgfpathlineto{\pgfqpoint{2.660472in}{3.244847in}}%
\pgfpathlineto{\pgfqpoint{2.668379in}{3.252582in}}%
\pgfpathlineto{\pgfqpoint{2.676285in}{3.259741in}}%
\pgfpathlineto{\pgfqpoint{2.684192in}{3.266316in}}%
\pgfpathlineto{\pgfqpoint{2.692099in}{3.272294in}}%
\pgfpathlineto{\pgfqpoint{2.700005in}{3.277666in}}%
\pgfpathlineto{\pgfqpoint{2.707912in}{3.282422in}}%
\pgfpathlineto{\pgfqpoint{2.715819in}{3.286552in}}%
\pgfpathlineto{\pgfqpoint{2.723726in}{3.290044in}}%
\pgfpathlineto{\pgfqpoint{2.731632in}{3.292888in}}%
\pgfpathlineto{\pgfqpoint{2.739539in}{3.295075in}}%
\pgfpathlineto{\pgfqpoint{2.747446in}{3.296594in}}%
\pgfpathlineto{\pgfqpoint{2.755353in}{3.297434in}}%
\pgfpathlineto{\pgfqpoint{2.763259in}{3.297586in}}%
\pgfpathlineto{\pgfqpoint{2.771166in}{3.297038in}}%
\pgfpathlineto{\pgfqpoint{2.779073in}{3.295781in}}%
\pgfpathlineto{\pgfqpoint{2.786979in}{3.293804in}}%
\pgfpathlineto{\pgfqpoint{2.794886in}{3.291097in}}%
\pgfpathlineto{\pgfqpoint{2.802793in}{3.287649in}}%
\pgfpathlineto{\pgfqpoint{2.810700in}{3.283451in}}%
\pgfpathlineto{\pgfqpoint{2.818606in}{3.278491in}}%
\pgfpathlineto{\pgfqpoint{2.826513in}{3.272760in}}%
\pgfpathlineto{\pgfqpoint{2.834420in}{3.266247in}}%
\pgfpathlineto{\pgfqpoint{2.842326in}{3.258941in}}%
\pgfpathlineto{\pgfqpoint{2.850233in}{3.250833in}}%
\pgfpathlineto{\pgfqpoint{2.858140in}{3.241913in}}%
\pgfpathlineto{\pgfqpoint{2.866047in}{3.232168in}}%
\pgfpathlineto{\pgfqpoint{2.873953in}{3.221592in}}%
\pgfpathlineto{\pgfqpoint{2.881860in}{3.210196in}}%
\pgfpathlineto{\pgfqpoint{2.889767in}{3.198015in}}%
\pgfpathlineto{\pgfqpoint{2.897674in}{3.185085in}}%
\pgfpathlineto{\pgfqpoint{2.905580in}{3.171444in}}%
\pgfpathlineto{\pgfqpoint{2.913487in}{3.157126in}}%
\pgfpathlineto{\pgfqpoint{2.921394in}{3.142168in}}%
\pgfpathlineto{\pgfqpoint{2.937207in}{3.110476in}}%
\pgfpathlineto{\pgfqpoint{2.953021in}{3.076657in}}%
\pgfpathlineto{\pgfqpoint{2.968834in}{3.041001in}}%
\pgfpathlineto{\pgfqpoint{2.984647in}{3.003795in}}%
\pgfpathlineto{\pgfqpoint{3.000461in}{2.965329in}}%
\pgfpathlineto{\pgfqpoint{3.024181in}{2.905900in}}%
\pgfpathlineto{\pgfqpoint{3.071621in}{2.784389in}}%
\pgfpathlineto{\pgfqpoint{3.095342in}{2.724257in}}%
\pgfpathlineto{\pgfqpoint{3.119062in}{2.665841in}}%
\pgfpathlineto{\pgfqpoint{3.134875in}{2.628332in}}%
\pgfpathlineto{\pgfqpoint{3.150689in}{2.592308in}}%
\pgfpathlineto{\pgfqpoint{3.166502in}{2.558058in}}%
\pgfpathlineto{\pgfqpoint{3.182316in}{2.525871in}}%
\pgfpathlineto{\pgfqpoint{3.190222in}{2.510641in}}%
\pgfpathlineto{\pgfqpoint{3.198129in}{2.496036in}}%
\pgfpathlineto{\pgfqpoint{3.206036in}{2.482091in}}%
\pgfpathlineto{\pgfqpoint{3.213942in}{2.468842in}}%
\pgfpathlineto{\pgfqpoint{3.221849in}{2.456325in}}%
\pgfpathlineto{\pgfqpoint{3.229756in}{2.444578in}}%
\pgfpathlineto{\pgfqpoint{3.237663in}{2.433635in}}%
\pgfpathlineto{\pgfqpoint{3.245569in}{2.423532in}}%
\pgfpathlineto{\pgfqpoint{3.253476in}{2.414307in}}%
\pgfpathlineto{\pgfqpoint{3.261383in}{2.405995in}}%
\pgfpathlineto{\pgfqpoint{3.269289in}{2.398632in}}%
\pgfpathlineto{\pgfqpoint{3.277196in}{2.392254in}}%
\pgfpathlineto{\pgfqpoint{3.285103in}{2.386898in}}%
\pgfpathlineto{\pgfqpoint{3.293010in}{2.382599in}}%
\pgfpathlineto{\pgfqpoint{3.300916in}{2.379394in}}%
\pgfpathlineto{\pgfqpoint{3.308823in}{2.377319in}}%
\pgfpathlineto{\pgfqpoint{3.316730in}{2.376410in}}%
\pgfpathlineto{\pgfqpoint{3.324637in}{2.376703in}}%
\pgfpathlineto{\pgfqpoint{3.332543in}{2.378234in}}%
\pgfpathlineto{\pgfqpoint{3.340450in}{2.381039in}}%
\pgfpathlineto{\pgfqpoint{3.348357in}{2.385154in}}%
\pgfpathlineto{\pgfqpoint{3.356263in}{2.390616in}}%
\pgfpathlineto{\pgfqpoint{3.364170in}{2.397461in}}%
\pgfpathlineto{\pgfqpoint{3.372077in}{2.405705in}}%
\pgfpathlineto{\pgfqpoint{3.379984in}{2.415313in}}%
\pgfpathlineto{\pgfqpoint{3.387890in}{2.426239in}}%
\pgfpathlineto{\pgfqpoint{3.395797in}{2.438438in}}%
\pgfpathlineto{\pgfqpoint{3.403704in}{2.451865in}}%
\pgfpathlineto{\pgfqpoint{3.411610in}{2.466475in}}%
\pgfpathlineto{\pgfqpoint{3.419517in}{2.482222in}}%
\pgfpathlineto{\pgfqpoint{3.427424in}{2.499062in}}%
\pgfpathlineto{\pgfqpoint{3.435331in}{2.516948in}}%
\pgfpathlineto{\pgfqpoint{3.443237in}{2.535838in}}%
\pgfpathlineto{\pgfqpoint{3.451144in}{2.555684in}}%
\pgfpathlineto{\pgfqpoint{3.459051in}{2.576441in}}%
\pgfpathlineto{\pgfqpoint{3.466958in}{2.598066in}}%
\pgfpathlineto{\pgfqpoint{3.482771in}{2.643734in}}%
\pgfpathlineto{\pgfqpoint{3.498584in}{2.692326in}}%
\pgfpathlineto{\pgfqpoint{3.514398in}{2.743481in}}%
\pgfpathlineto{\pgfqpoint{3.530211in}{2.796838in}}%
\pgfpathlineto{\pgfqpoint{3.546025in}{2.852036in}}%
\pgfpathlineto{\pgfqpoint{3.569745in}{2.937492in}}%
\pgfpathlineto{\pgfqpoint{3.601372in}{3.054503in}}%
\pgfpathlineto{\pgfqpoint{3.648812in}{3.230724in}}%
\pgfpathlineto{\pgfqpoint{3.672532in}{3.316509in}}%
\pgfpathlineto{\pgfqpoint{3.688346in}{3.372010in}}%
\pgfpathlineto{\pgfqpoint{3.704159in}{3.425738in}}%
\pgfpathlineto{\pgfqpoint{3.719973in}{3.477332in}}%
\pgfpathlineto{\pgfqpoint{3.735786in}{3.526430in}}%
\pgfpathlineto{\pgfqpoint{3.751600in}{3.572671in}}%
\pgfpathlineto{\pgfqpoint{3.759506in}{3.594608in}}%
\pgfpathlineto{\pgfqpoint{3.767413in}{3.615695in}}%
\pgfpathlineto{\pgfqpoint{3.775320in}{3.635886in}}%
\pgfpathlineto{\pgfqpoint{3.783226in}{3.655138in}}%
\pgfpathlineto{\pgfqpoint{3.791133in}{3.673404in}}%
\pgfpathlineto{\pgfqpoint{3.799040in}{3.690640in}}%
\pgfpathlineto{\pgfqpoint{3.806947in}{3.706801in}}%
\pgfpathlineto{\pgfqpoint{3.814853in}{3.721841in}}%
\pgfpathlineto{\pgfqpoint{3.822760in}{3.735715in}}%
\pgfpathlineto{\pgfqpoint{3.830667in}{3.748377in}}%
\pgfpathlineto{\pgfqpoint{3.838574in}{3.759784in}}%
\pgfpathlineto{\pgfqpoint{3.846480in}{3.769889in}}%
\pgfpathlineto{\pgfqpoint{3.854387in}{3.778648in}}%
\pgfpathlineto{\pgfqpoint{3.862294in}{3.786019in}}%
\pgfpathlineto{\pgfqpoint{3.870200in}{3.792004in}}%
\pgfpathlineto{\pgfqpoint{3.878107in}{3.796623in}}%
\pgfpathlineto{\pgfqpoint{3.886014in}{3.799898in}}%
\pgfpathlineto{\pgfqpoint{3.893921in}{3.801850in}}%
\pgfpathlineto{\pgfqpoint{3.901827in}{3.802500in}}%
\pgfpathlineto{\pgfqpoint{3.909734in}{3.801870in}}%
\pgfpathlineto{\pgfqpoint{3.917641in}{3.799981in}}%
\pgfpathlineto{\pgfqpoint{3.925547in}{3.796855in}}%
\pgfpathlineto{\pgfqpoint{3.933454in}{3.792512in}}%
\pgfpathlineto{\pgfqpoint{3.941361in}{3.786975in}}%
\pgfpathlineto{\pgfqpoint{3.949268in}{3.780264in}}%
\pgfpathlineto{\pgfqpoint{3.957174in}{3.772401in}}%
\pgfpathlineto{\pgfqpoint{3.965081in}{3.763407in}}%
\pgfpathlineto{\pgfqpoint{3.972988in}{3.753303in}}%
\pgfpathlineto{\pgfqpoint{3.980895in}{3.742112in}}%
\pgfpathlineto{\pgfqpoint{3.988801in}{3.729854in}}%
\pgfpathlineto{\pgfqpoint{3.996708in}{3.716551in}}%
\pgfpathlineto{\pgfqpoint{4.004615in}{3.702223in}}%
\pgfpathlineto{\pgfqpoint{4.012521in}{3.686893in}}%
\pgfpathlineto{\pgfqpoint{4.020428in}{3.670582in}}%
\pgfpathlineto{\pgfqpoint{4.028335in}{3.653310in}}%
\pgfpathlineto{\pgfqpoint{4.036242in}{3.635100in}}%
\pgfpathlineto{\pgfqpoint{4.044148in}{3.615973in}}%
\pgfpathlineto{\pgfqpoint{4.052055in}{3.595950in}}%
\pgfpathlineto{\pgfqpoint{4.059962in}{3.575053in}}%
\pgfpathlineto{\pgfqpoint{4.067868in}{3.553302in}}%
\pgfpathlineto{\pgfqpoint{4.083682in}{3.507326in}}%
\pgfpathlineto{\pgfqpoint{4.099495in}{3.458194in}}%
\pgfpathlineto{\pgfqpoint{4.115309in}{3.406076in}}%
\pgfpathlineto{\pgfqpoint{4.131122in}{3.351143in}}%
\pgfpathlineto{\pgfqpoint{4.146936in}{3.293566in}}%
\pgfpathlineto{\pgfqpoint{4.162749in}{3.233515in}}%
\pgfpathlineto{\pgfqpoint{4.178563in}{3.171163in}}%
\pgfpathlineto{\pgfqpoint{4.194376in}{3.106679in}}%
\pgfpathlineto{\pgfqpoint{4.210189in}{3.040233in}}%
\pgfpathlineto{\pgfqpoint{4.226003in}{2.971998in}}%
\pgfpathlineto{\pgfqpoint{4.249723in}{2.866663in}}%
\pgfpathlineto{\pgfqpoint{4.273443in}{2.758262in}}%
\pgfpathlineto{\pgfqpoint{4.297163in}{2.647370in}}%
\pgfpathlineto{\pgfqpoint{4.328790in}{2.496638in}}%
\pgfpathlineto{\pgfqpoint{4.376231in}{2.267150in}}%
\pgfpathlineto{\pgfqpoint{4.431578in}{1.999781in}}%
\pgfpathlineto{\pgfqpoint{4.463205in}{1.849836in}}%
\pgfpathlineto{\pgfqpoint{4.486925in}{1.739779in}}%
\pgfpathlineto{\pgfqpoint{4.510645in}{1.632421in}}%
\pgfpathlineto{\pgfqpoint{4.534365in}{1.528338in}}%
\pgfpathlineto{\pgfqpoint{4.550179in}{1.461054in}}%
\pgfpathlineto{\pgfqpoint{4.565992in}{1.395652in}}%
\pgfpathlineto{\pgfqpoint{4.581805in}{1.332303in}}%
\pgfpathlineto{\pgfqpoint{4.597619in}{1.271179in}}%
\pgfpathlineto{\pgfqpoint{4.613432in}{1.212451in}}%
\pgfpathlineto{\pgfqpoint{4.629246in}{1.156288in}}%
\pgfpathlineto{\pgfqpoint{4.645059in}{1.102862in}}%
\pgfpathlineto{\pgfqpoint{4.660873in}{1.052343in}}%
\pgfpathlineto{\pgfqpoint{4.676686in}{1.004903in}}%
\pgfpathlineto{\pgfqpoint{4.692500in}{0.960713in}}%
\pgfpathlineto{\pgfqpoint{4.700406in}{0.939889in}}%
\pgfpathlineto{\pgfqpoint{4.708313in}{0.919942in}}%
\pgfpathlineto{\pgfqpoint{4.716220in}{0.900893in}}%
\pgfpathlineto{\pgfqpoint{4.724126in}{0.882762in}}%
\pgfpathlineto{\pgfqpoint{4.732033in}{0.865573in}}%
\pgfpathlineto{\pgfqpoint{4.739940in}{0.849345in}}%
\pgfpathlineto{\pgfqpoint{4.747847in}{0.834100in}}%
\pgfpathlineto{\pgfqpoint{4.755753in}{0.819859in}}%
\pgfpathlineto{\pgfqpoint{4.763660in}{0.806644in}}%
\pgfpathlineto{\pgfqpoint{4.771567in}{0.794477in}}%
\pgfpathlineto{\pgfqpoint{4.779473in}{0.783378in}}%
\pgfpathlineto{\pgfqpoint{4.787380in}{0.773369in}}%
\pgfpathlineto{\pgfqpoint{4.795287in}{0.764471in}}%
\pgfpathlineto{\pgfqpoint{4.803194in}{0.756706in}}%
\pgfpathlineto{\pgfqpoint{4.811100in}{0.750095in}}%
\pgfpathlineto{\pgfqpoint{4.819007in}{0.744659in}}%
\pgfpathlineto{\pgfqpoint{4.826914in}{0.740420in}}%
\pgfpathlineto{\pgfqpoint{4.834821in}{0.737399in}}%
\pgfpathlineto{\pgfqpoint{4.842727in}{0.735617in}}%
\pgfpathlineto{\pgfqpoint{4.842727in}{0.735617in}}%
\pgfusepath{stroke}%
\end{pgfscope}%
\begin{pgfscope}%
\pgfpathrectangle{\pgfqpoint{0.700000in}{0.495000in}}{\pgfqpoint{4.340000in}{3.465000in}}%
\pgfusepath{clip}%
\pgfsetroundcap%
\pgfsetroundjoin%
\pgfsetlinewidth{1.505625pt}%
\definecolor{currentstroke}{rgb}{1.000000,0.647059,0.000000}%
\pgfsetstrokecolor{currentstroke}%
\pgfsetdash{}{0pt}%
\pgfpathmoveto{\pgfqpoint{0.897273in}{0.735617in}}%
\pgfpathlineto{\pgfqpoint{0.905179in}{0.748397in}}%
\pgfpathlineto{\pgfqpoint{0.913086in}{0.760522in}}%
\pgfpathlineto{\pgfqpoint{0.920993in}{0.772005in}}%
\pgfpathlineto{\pgfqpoint{0.928900in}{0.782858in}}%
\pgfpathlineto{\pgfqpoint{0.936806in}{0.793091in}}%
\pgfpathlineto{\pgfqpoint{0.944713in}{0.802718in}}%
\pgfpathlineto{\pgfqpoint{0.952620in}{0.811749in}}%
\pgfpathlineto{\pgfqpoint{0.960527in}{0.820196in}}%
\pgfpathlineto{\pgfqpoint{0.968433in}{0.828073in}}%
\pgfpathlineto{\pgfqpoint{0.976340in}{0.835389in}}%
\pgfpathlineto{\pgfqpoint{0.984247in}{0.842157in}}%
\pgfpathlineto{\pgfqpoint{0.992153in}{0.848389in}}%
\pgfpathlineto{\pgfqpoint{1.000060in}{0.854097in}}%
\pgfpathlineto{\pgfqpoint{1.007967in}{0.859293in}}%
\pgfpathlineto{\pgfqpoint{1.015874in}{0.863988in}}%
\pgfpathlineto{\pgfqpoint{1.023780in}{0.868193in}}%
\pgfpathlineto{\pgfqpoint{1.031687in}{0.871922in}}%
\pgfpathlineto{\pgfqpoint{1.039594in}{0.875186in}}%
\pgfpathlineto{\pgfqpoint{1.047500in}{0.877996in}}%
\pgfpathlineto{\pgfqpoint{1.055407in}{0.880365in}}%
\pgfpathlineto{\pgfqpoint{1.063314in}{0.882304in}}%
\pgfpathlineto{\pgfqpoint{1.071221in}{0.883825in}}%
\pgfpathlineto{\pgfqpoint{1.079127in}{0.884940in}}%
\pgfpathlineto{\pgfqpoint{1.087034in}{0.885660in}}%
\pgfpathlineto{\pgfqpoint{1.094941in}{0.885999in}}%
\pgfpathlineto{\pgfqpoint{1.102848in}{0.885966in}}%
\pgfpathlineto{\pgfqpoint{1.110754in}{0.885575in}}%
\pgfpathlineto{\pgfqpoint{1.118661in}{0.884836in}}%
\pgfpathlineto{\pgfqpoint{1.126568in}{0.883763in}}%
\pgfpathlineto{\pgfqpoint{1.134474in}{0.882366in}}%
\pgfpathlineto{\pgfqpoint{1.142381in}{0.880658in}}%
\pgfpathlineto{\pgfqpoint{1.150288in}{0.878649in}}%
\pgfpathlineto{\pgfqpoint{1.166101in}{0.873781in}}%
\pgfpathlineto{\pgfqpoint{1.181915in}{0.867856in}}%
\pgfpathlineto{\pgfqpoint{1.197728in}{0.860968in}}%
\pgfpathlineto{\pgfqpoint{1.213542in}{0.853213in}}%
\pgfpathlineto{\pgfqpoint{1.229355in}{0.844684in}}%
\pgfpathlineto{\pgfqpoint{1.245169in}{0.835476in}}%
\pgfpathlineto{\pgfqpoint{1.260982in}{0.825684in}}%
\pgfpathlineto{\pgfqpoint{1.284702in}{0.810108in}}%
\pgfpathlineto{\pgfqpoint{1.308422in}{0.793748in}}%
\pgfpathlineto{\pgfqpoint{1.347956in}{0.765611in}}%
\pgfpathlineto{\pgfqpoint{1.387490in}{0.737664in}}%
\pgfpathlineto{\pgfqpoint{1.411210in}{0.721602in}}%
\pgfpathlineto{\pgfqpoint{1.427023in}{0.711385in}}%
\pgfpathlineto{\pgfqpoint{1.442837in}{0.701672in}}%
\pgfpathlineto{\pgfqpoint{1.458650in}{0.692556in}}%
\pgfpathlineto{\pgfqpoint{1.474463in}{0.684132in}}%
\pgfpathlineto{\pgfqpoint{1.490277in}{0.676496in}}%
\pgfpathlineto{\pgfqpoint{1.506090in}{0.669741in}}%
\pgfpathlineto{\pgfqpoint{1.521904in}{0.663963in}}%
\pgfpathlineto{\pgfqpoint{1.537717in}{0.659254in}}%
\pgfpathlineto{\pgfqpoint{1.545624in}{0.657331in}}%
\pgfpathlineto{\pgfqpoint{1.553531in}{0.655712in}}%
\pgfpathlineto{\pgfqpoint{1.561437in}{0.654407in}}%
\pgfpathlineto{\pgfqpoint{1.569344in}{0.653429in}}%
\pgfpathlineto{\pgfqpoint{1.577251in}{0.652789in}}%
\pgfpathlineto{\pgfqpoint{1.585158in}{0.652500in}}%
\pgfpathlineto{\pgfqpoint{1.593064in}{0.652573in}}%
\pgfpathlineto{\pgfqpoint{1.600971in}{0.653020in}}%
\pgfpathlineto{\pgfqpoint{1.608878in}{0.653853in}}%
\pgfpathlineto{\pgfqpoint{1.616784in}{0.655084in}}%
\pgfpathlineto{\pgfqpoint{1.624691in}{0.656724in}}%
\pgfpathlineto{\pgfqpoint{1.632598in}{0.658786in}}%
\pgfpathlineto{\pgfqpoint{1.640505in}{0.661281in}}%
\pgfpathlineto{\pgfqpoint{1.648411in}{0.664220in}}%
\pgfpathlineto{\pgfqpoint{1.656318in}{0.667617in}}%
\pgfpathlineto{\pgfqpoint{1.664225in}{0.671482in}}%
\pgfpathlineto{\pgfqpoint{1.672132in}{0.675828in}}%
\pgfpathlineto{\pgfqpoint{1.680038in}{0.680666in}}%
\pgfpathlineto{\pgfqpoint{1.687945in}{0.686007in}}%
\pgfpathlineto{\pgfqpoint{1.695852in}{0.691865in}}%
\pgfpathlineto{\pgfqpoint{1.703758in}{0.698251in}}%
\pgfpathlineto{\pgfqpoint{1.711665in}{0.705176in}}%
\pgfpathlineto{\pgfqpoint{1.719572in}{0.712652in}}%
\pgfpathlineto{\pgfqpoint{1.727479in}{0.720692in}}%
\pgfpathlineto{\pgfqpoint{1.735385in}{0.729307in}}%
\pgfpathlineto{\pgfqpoint{1.743292in}{0.738508in}}%
\pgfpathlineto{\pgfqpoint{1.751199in}{0.748308in}}%
\pgfpathlineto{\pgfqpoint{1.759105in}{0.758719in}}%
\pgfpathlineto{\pgfqpoint{1.767012in}{0.769752in}}%
\pgfpathlineto{\pgfqpoint{1.774919in}{0.781419in}}%
\pgfpathlineto{\pgfqpoint{1.782826in}{0.793732in}}%
\pgfpathlineto{\pgfqpoint{1.790732in}{0.806702in}}%
\pgfpathlineto{\pgfqpoint{1.798639in}{0.820343in}}%
\pgfpathlineto{\pgfqpoint{1.806546in}{0.834664in}}%
\pgfpathlineto{\pgfqpoint{1.814453in}{0.849679in}}%
\pgfpathlineto{\pgfqpoint{1.822359in}{0.865399in}}%
\pgfpathlineto{\pgfqpoint{1.830266in}{0.881836in}}%
\pgfpathlineto{\pgfqpoint{1.838173in}{0.899001in}}%
\pgfpathlineto{\pgfqpoint{1.846079in}{0.916907in}}%
\pgfpathlineto{\pgfqpoint{1.853986in}{0.935565in}}%
\pgfpathlineto{\pgfqpoint{1.861893in}{0.954988in}}%
\pgfpathlineto{\pgfqpoint{1.869800in}{0.975186in}}%
\pgfpathlineto{\pgfqpoint{1.877706in}{0.996172in}}%
\pgfpathlineto{\pgfqpoint{1.893520in}{1.040543in}}%
\pgfpathlineto{\pgfqpoint{1.909333in}{1.088021in}}%
\pgfpathlineto{\pgfqpoint{1.925147in}{1.138409in}}%
\pgfpathlineto{\pgfqpoint{1.940960in}{1.191508in}}%
\pgfpathlineto{\pgfqpoint{1.956774in}{1.247120in}}%
\pgfpathlineto{\pgfqpoint{1.972587in}{1.305044in}}%
\pgfpathlineto{\pgfqpoint{1.988400in}{1.365083in}}%
\pgfpathlineto{\pgfqpoint{2.004214in}{1.427037in}}%
\pgfpathlineto{\pgfqpoint{2.020027in}{1.490707in}}%
\pgfpathlineto{\pgfqpoint{2.043747in}{1.588994in}}%
\pgfpathlineto{\pgfqpoint{2.067468in}{1.690023in}}%
\pgfpathlineto{\pgfqpoint{2.099095in}{1.827830in}}%
\pgfpathlineto{\pgfqpoint{2.146535in}{2.037956in}}%
\pgfpathlineto{\pgfqpoint{2.193975in}{2.247401in}}%
\pgfpathlineto{\pgfqpoint{2.217695in}{2.350189in}}%
\pgfpathlineto{\pgfqpoint{2.241416in}{2.450792in}}%
\pgfpathlineto{\pgfqpoint{2.265136in}{2.548537in}}%
\pgfpathlineto{\pgfqpoint{2.280949in}{2.611781in}}%
\pgfpathlineto{\pgfqpoint{2.296763in}{2.673257in}}%
\pgfpathlineto{\pgfqpoint{2.312576in}{2.732767in}}%
\pgfpathlineto{\pgfqpoint{2.328390in}{2.790111in}}%
\pgfpathlineto{\pgfqpoint{2.344203in}{2.845091in}}%
\pgfpathlineto{\pgfqpoint{2.360016in}{2.897506in}}%
\pgfpathlineto{\pgfqpoint{2.375830in}{2.947159in}}%
\pgfpathlineto{\pgfqpoint{2.391643in}{2.993883in}}%
\pgfpathlineto{\pgfqpoint{2.407457in}{3.037674in}}%
\pgfpathlineto{\pgfqpoint{2.423270in}{3.078578in}}%
\pgfpathlineto{\pgfqpoint{2.439084in}{3.116638in}}%
\pgfpathlineto{\pgfqpoint{2.446990in}{3.134616in}}%
\pgfpathlineto{\pgfqpoint{2.454897in}{3.151900in}}%
\pgfpathlineto{\pgfqpoint{2.462804in}{3.168496in}}%
\pgfpathlineto{\pgfqpoint{2.470711in}{3.184408in}}%
\pgfpathlineto{\pgfqpoint{2.478617in}{3.199643in}}%
\pgfpathlineto{\pgfqpoint{2.486524in}{3.214207in}}%
\pgfpathlineto{\pgfqpoint{2.494431in}{3.228104in}}%
\pgfpathlineto{\pgfqpoint{2.502337in}{3.241340in}}%
\pgfpathlineto{\pgfqpoint{2.510244in}{3.253922in}}%
\pgfpathlineto{\pgfqpoint{2.518151in}{3.265854in}}%
\pgfpathlineto{\pgfqpoint{2.526058in}{3.277143in}}%
\pgfpathlineto{\pgfqpoint{2.533964in}{3.287793in}}%
\pgfpathlineto{\pgfqpoint{2.541871in}{3.297811in}}%
\pgfpathlineto{\pgfqpoint{2.549778in}{3.307201in}}%
\pgfpathlineto{\pgfqpoint{2.557684in}{3.315970in}}%
\pgfpathlineto{\pgfqpoint{2.565591in}{3.324123in}}%
\pgfpathlineto{\pgfqpoint{2.573498in}{3.331666in}}%
\pgfpathlineto{\pgfqpoint{2.581405in}{3.338604in}}%
\pgfpathlineto{\pgfqpoint{2.589311in}{3.344942in}}%
\pgfpathlineto{\pgfqpoint{2.597218in}{3.350687in}}%
\pgfpathlineto{\pgfqpoint{2.605125in}{3.355845in}}%
\pgfpathlineto{\pgfqpoint{2.613032in}{3.360419in}}%
\pgfpathlineto{\pgfqpoint{2.620938in}{3.364417in}}%
\pgfpathlineto{\pgfqpoint{2.628845in}{3.367844in}}%
\pgfpathlineto{\pgfqpoint{2.636752in}{3.370705in}}%
\pgfpathlineto{\pgfqpoint{2.644658in}{3.373006in}}%
\pgfpathlineto{\pgfqpoint{2.652565in}{3.374752in}}%
\pgfpathlineto{\pgfqpoint{2.660472in}{3.375949in}}%
\pgfpathlineto{\pgfqpoint{2.668379in}{3.376603in}}%
\pgfpathlineto{\pgfqpoint{2.676285in}{3.376720in}}%
\pgfpathlineto{\pgfqpoint{2.684192in}{3.376304in}}%
\pgfpathlineto{\pgfqpoint{2.692099in}{3.375361in}}%
\pgfpathlineto{\pgfqpoint{2.700005in}{3.373898in}}%
\pgfpathlineto{\pgfqpoint{2.707912in}{3.371919in}}%
\pgfpathlineto{\pgfqpoint{2.715819in}{3.369430in}}%
\pgfpathlineto{\pgfqpoint{2.723726in}{3.366437in}}%
\pgfpathlineto{\pgfqpoint{2.731632in}{3.362945in}}%
\pgfpathlineto{\pgfqpoint{2.739539in}{3.358960in}}%
\pgfpathlineto{\pgfqpoint{2.747446in}{3.354487in}}%
\pgfpathlineto{\pgfqpoint{2.755353in}{3.349533in}}%
\pgfpathlineto{\pgfqpoint{2.763259in}{3.344102in}}%
\pgfpathlineto{\pgfqpoint{2.771166in}{3.338201in}}%
\pgfpathlineto{\pgfqpoint{2.779073in}{3.331834in}}%
\pgfpathlineto{\pgfqpoint{2.786979in}{3.325007in}}%
\pgfpathlineto{\pgfqpoint{2.794886in}{3.317727in}}%
\pgfpathlineto{\pgfqpoint{2.802793in}{3.309998in}}%
\pgfpathlineto{\pgfqpoint{2.810700in}{3.301826in}}%
\pgfpathlineto{\pgfqpoint{2.818606in}{3.293217in}}%
\pgfpathlineto{\pgfqpoint{2.826513in}{3.284176in}}%
\pgfpathlineto{\pgfqpoint{2.842326in}{3.264822in}}%
\pgfpathlineto{\pgfqpoint{2.858140in}{3.243808in}}%
\pgfpathlineto{\pgfqpoint{2.873953in}{3.221179in}}%
\pgfpathlineto{\pgfqpoint{2.889767in}{3.197024in}}%
\pgfpathlineto{\pgfqpoint{2.905580in}{3.171518in}}%
\pgfpathlineto{\pgfqpoint{2.921394in}{3.144846in}}%
\pgfpathlineto{\pgfqpoint{2.937207in}{3.117193in}}%
\pgfpathlineto{\pgfqpoint{2.960927in}{3.074277in}}%
\pgfpathlineto{\pgfqpoint{2.992554in}{3.015344in}}%
\pgfpathlineto{\pgfqpoint{3.047901in}{2.911654in}}%
\pgfpathlineto{\pgfqpoint{3.071621in}{2.868614in}}%
\pgfpathlineto{\pgfqpoint{3.087435in}{2.840849in}}%
\pgfpathlineto{\pgfqpoint{3.103248in}{2.814041in}}%
\pgfpathlineto{\pgfqpoint{3.119062in}{2.788377in}}%
\pgfpathlineto{\pgfqpoint{3.134875in}{2.764039in}}%
\pgfpathlineto{\pgfqpoint{3.150689in}{2.741213in}}%
\pgfpathlineto{\pgfqpoint{3.166502in}{2.720084in}}%
\pgfpathlineto{\pgfqpoint{3.174409in}{2.710214in}}%
\pgfpathlineto{\pgfqpoint{3.182316in}{2.700836in}}%
\pgfpathlineto{\pgfqpoint{3.190222in}{2.691976in}}%
\pgfpathlineto{\pgfqpoint{3.198129in}{2.683655in}}%
\pgfpathlineto{\pgfqpoint{3.206036in}{2.675896in}}%
\pgfpathlineto{\pgfqpoint{3.213942in}{2.668724in}}%
\pgfpathlineto{\pgfqpoint{3.221849in}{2.662160in}}%
\pgfpathlineto{\pgfqpoint{3.229756in}{2.656228in}}%
\pgfpathlineto{\pgfqpoint{3.237663in}{2.650952in}}%
\pgfpathlineto{\pgfqpoint{3.245569in}{2.646353in}}%
\pgfpathlineto{\pgfqpoint{3.253476in}{2.642456in}}%
\pgfpathlineto{\pgfqpoint{3.261383in}{2.639282in}}%
\pgfpathlineto{\pgfqpoint{3.269289in}{2.636857in}}%
\pgfpathlineto{\pgfqpoint{3.277196in}{2.635201in}}%
\pgfpathlineto{\pgfqpoint{3.285103in}{2.634340in}}%
\pgfpathlineto{\pgfqpoint{3.293010in}{2.634295in}}%
\pgfpathlineto{\pgfqpoint{3.300916in}{2.635089in}}%
\pgfpathlineto{\pgfqpoint{3.308823in}{2.636747in}}%
\pgfpathlineto{\pgfqpoint{3.316730in}{2.639290in}}%
\pgfpathlineto{\pgfqpoint{3.324637in}{2.642742in}}%
\pgfpathlineto{\pgfqpoint{3.332543in}{2.647127in}}%
\pgfpathlineto{\pgfqpoint{3.340450in}{2.652466in}}%
\pgfpathlineto{\pgfqpoint{3.348357in}{2.658784in}}%
\pgfpathlineto{\pgfqpoint{3.356263in}{2.666103in}}%
\pgfpathlineto{\pgfqpoint{3.364170in}{2.674447in}}%
\pgfpathlineto{\pgfqpoint{3.372077in}{2.683824in}}%
\pgfpathlineto{\pgfqpoint{3.379984in}{2.694205in}}%
\pgfpathlineto{\pgfqpoint{3.387890in}{2.705553in}}%
\pgfpathlineto{\pgfqpoint{3.395797in}{2.717832in}}%
\pgfpathlineto{\pgfqpoint{3.403704in}{2.731006in}}%
\pgfpathlineto{\pgfqpoint{3.411610in}{2.745039in}}%
\pgfpathlineto{\pgfqpoint{3.419517in}{2.759893in}}%
\pgfpathlineto{\pgfqpoint{3.427424in}{2.775534in}}%
\pgfpathlineto{\pgfqpoint{3.435331in}{2.791923in}}%
\pgfpathlineto{\pgfqpoint{3.443237in}{2.809026in}}%
\pgfpathlineto{\pgfqpoint{3.459051in}{2.845226in}}%
\pgfpathlineto{\pgfqpoint{3.474864in}{2.883841in}}%
\pgfpathlineto{\pgfqpoint{3.490678in}{2.924583in}}%
\pgfpathlineto{\pgfqpoint{3.506491in}{2.967159in}}%
\pgfpathlineto{\pgfqpoint{3.522305in}{3.011280in}}%
\pgfpathlineto{\pgfqpoint{3.546025in}{3.079721in}}%
\pgfpathlineto{\pgfqpoint{3.577652in}{3.173668in}}%
\pgfpathlineto{\pgfqpoint{3.632999in}{3.338951in}}%
\pgfpathlineto{\pgfqpoint{3.656719in}{3.407584in}}%
\pgfpathlineto{\pgfqpoint{3.672532in}{3.451878in}}%
\pgfpathlineto{\pgfqpoint{3.688346in}{3.494666in}}%
\pgfpathlineto{\pgfqpoint{3.704159in}{3.535654in}}%
\pgfpathlineto{\pgfqpoint{3.719973in}{3.574554in}}%
\pgfpathlineto{\pgfqpoint{3.735786in}{3.611074in}}%
\pgfpathlineto{\pgfqpoint{3.743693in}{3.628351in}}%
\pgfpathlineto{\pgfqpoint{3.751600in}{3.644923in}}%
\pgfpathlineto{\pgfqpoint{3.759506in}{3.660756in}}%
\pgfpathlineto{\pgfqpoint{3.767413in}{3.675812in}}%
\pgfpathlineto{\pgfqpoint{3.775320in}{3.690055in}}%
\pgfpathlineto{\pgfqpoint{3.783226in}{3.703449in}}%
\pgfpathlineto{\pgfqpoint{3.791133in}{3.715957in}}%
\pgfpathlineto{\pgfqpoint{3.799040in}{3.727543in}}%
\pgfpathlineto{\pgfqpoint{3.806947in}{3.738171in}}%
\pgfpathlineto{\pgfqpoint{3.814853in}{3.747805in}}%
\pgfpathlineto{\pgfqpoint{3.822760in}{3.756407in}}%
\pgfpathlineto{\pgfqpoint{3.830667in}{3.763942in}}%
\pgfpathlineto{\pgfqpoint{3.838574in}{3.770374in}}%
\pgfpathlineto{\pgfqpoint{3.846480in}{3.775666in}}%
\pgfpathlineto{\pgfqpoint{3.854387in}{3.779781in}}%
\pgfpathlineto{\pgfqpoint{3.862294in}{3.782688in}}%
\pgfpathlineto{\pgfqpoint{3.870200in}{3.784388in}}%
\pgfpathlineto{\pgfqpoint{3.878107in}{3.784900in}}%
\pgfpathlineto{\pgfqpoint{3.886014in}{3.784245in}}%
\pgfpathlineto{\pgfqpoint{3.893921in}{3.782441in}}%
\pgfpathlineto{\pgfqpoint{3.901827in}{3.779509in}}%
\pgfpathlineto{\pgfqpoint{3.909734in}{3.775468in}}%
\pgfpathlineto{\pgfqpoint{3.917641in}{3.770337in}}%
\pgfpathlineto{\pgfqpoint{3.925547in}{3.764136in}}%
\pgfpathlineto{\pgfqpoint{3.933454in}{3.756886in}}%
\pgfpathlineto{\pgfqpoint{3.941361in}{3.748604in}}%
\pgfpathlineto{\pgfqpoint{3.949268in}{3.739311in}}%
\pgfpathlineto{\pgfqpoint{3.957174in}{3.729027in}}%
\pgfpathlineto{\pgfqpoint{3.965081in}{3.717771in}}%
\pgfpathlineto{\pgfqpoint{3.972988in}{3.705562in}}%
\pgfpathlineto{\pgfqpoint{3.980895in}{3.692421in}}%
\pgfpathlineto{\pgfqpoint{3.988801in}{3.678366in}}%
\pgfpathlineto{\pgfqpoint{3.996708in}{3.663418in}}%
\pgfpathlineto{\pgfqpoint{4.004615in}{3.647595in}}%
\pgfpathlineto{\pgfqpoint{4.012521in}{3.630919in}}%
\pgfpathlineto{\pgfqpoint{4.020428in}{3.613407in}}%
\pgfpathlineto{\pgfqpoint{4.028335in}{3.595080in}}%
\pgfpathlineto{\pgfqpoint{4.036242in}{3.575958in}}%
\pgfpathlineto{\pgfqpoint{4.044148in}{3.556059in}}%
\pgfpathlineto{\pgfqpoint{4.059962in}{3.514012in}}%
\pgfpathlineto{\pgfqpoint{4.075775in}{3.469096in}}%
\pgfpathlineto{\pgfqpoint{4.091589in}{3.421466in}}%
\pgfpathlineto{\pgfqpoint{4.107402in}{3.371281in}}%
\pgfpathlineto{\pgfqpoint{4.123216in}{3.318697in}}%
\pgfpathlineto{\pgfqpoint{4.139029in}{3.263870in}}%
\pgfpathlineto{\pgfqpoint{4.154842in}{3.206957in}}%
\pgfpathlineto{\pgfqpoint{4.170656in}{3.148115in}}%
\pgfpathlineto{\pgfqpoint{4.186469in}{3.087500in}}%
\pgfpathlineto{\pgfqpoint{4.202283in}{3.025269in}}%
\pgfpathlineto{\pgfqpoint{4.226003in}{2.929236in}}%
\pgfpathlineto{\pgfqpoint{4.249723in}{2.830449in}}%
\pgfpathlineto{\pgfqpoint{4.273443in}{2.729437in}}%
\pgfpathlineto{\pgfqpoint{4.305070in}{2.592205in}}%
\pgfpathlineto{\pgfqpoint{4.360417in}{2.348563in}}%
\pgfpathlineto{\pgfqpoint{4.407858in}{2.140523in}}%
\pgfpathlineto{\pgfqpoint{4.439484in}{2.004457in}}%
\pgfpathlineto{\pgfqpoint{4.463205in}{1.904678in}}%
\pgfpathlineto{\pgfqpoint{4.486925in}{1.807433in}}%
\pgfpathlineto{\pgfqpoint{4.510645in}{1.713251in}}%
\pgfpathlineto{\pgfqpoint{4.526458in}{1.652426in}}%
\pgfpathlineto{\pgfqpoint{4.542272in}{1.593354in}}%
\pgfpathlineto{\pgfqpoint{4.558085in}{1.536191in}}%
\pgfpathlineto{\pgfqpoint{4.573899in}{1.481095in}}%
\pgfpathlineto{\pgfqpoint{4.589712in}{1.428221in}}%
\pgfpathlineto{\pgfqpoint{4.605526in}{1.377728in}}%
\pgfpathlineto{\pgfqpoint{4.621339in}{1.329770in}}%
\pgfpathlineto{\pgfqpoint{4.637152in}{1.284506in}}%
\pgfpathlineto{\pgfqpoint{4.652966in}{1.242092in}}%
\pgfpathlineto{\pgfqpoint{4.660873in}{1.222002in}}%
\pgfpathlineto{\pgfqpoint{4.668779in}{1.202684in}}%
\pgfpathlineto{\pgfqpoint{4.676686in}{1.184156in}}%
\pgfpathlineto{\pgfqpoint{4.684593in}{1.166439in}}%
\pgfpathlineto{\pgfqpoint{4.692500in}{1.149552in}}%
\pgfpathlineto{\pgfqpoint{4.700406in}{1.133514in}}%
\pgfpathlineto{\pgfqpoint{4.708313in}{1.118346in}}%
\pgfpathlineto{\pgfqpoint{4.716220in}{1.104066in}}%
\pgfpathlineto{\pgfqpoint{4.724126in}{1.090694in}}%
\pgfpathlineto{\pgfqpoint{4.732033in}{1.078250in}}%
\pgfpathlineto{\pgfqpoint{4.739940in}{1.066754in}}%
\pgfpathlineto{\pgfqpoint{4.747847in}{1.056225in}}%
\pgfpathlineto{\pgfqpoint{4.755753in}{1.046683in}}%
\pgfpathlineto{\pgfqpoint{4.763660in}{1.038146in}}%
\pgfpathlineto{\pgfqpoint{4.771567in}{1.030636in}}%
\pgfpathlineto{\pgfqpoint{4.779473in}{1.024171in}}%
\pgfpathlineto{\pgfqpoint{4.787380in}{1.018771in}}%
\pgfpathlineto{\pgfqpoint{4.795287in}{1.014455in}}%
\pgfpathlineto{\pgfqpoint{4.803194in}{1.011244in}}%
\pgfpathlineto{\pgfqpoint{4.811100in}{1.009157in}}%
\pgfpathlineto{\pgfqpoint{4.819007in}{1.008212in}}%
\pgfpathlineto{\pgfqpoint{4.826914in}{1.008431in}}%
\pgfpathlineto{\pgfqpoint{4.834821in}{1.009832in}}%
\pgfpathlineto{\pgfqpoint{4.842727in}{1.012436in}}%
\pgfpathlineto{\pgfqpoint{4.842727in}{1.012436in}}%
\pgfusepath{stroke}%
\end{pgfscope}%
\begin{pgfscope}%
\pgfsetrectcap%
\pgfsetmiterjoin%
\pgfsetlinewidth{1.254687pt}%
\definecolor{currentstroke}{rgb}{1.000000,1.000000,1.000000}%
\pgfsetstrokecolor{currentstroke}%
\pgfsetdash{}{0pt}%
\pgfpathmoveto{\pgfqpoint{0.700000in}{0.495000in}}%
\pgfpathlineto{\pgfqpoint{0.700000in}{3.960000in}}%
\pgfusepath{stroke}%
\end{pgfscope}%
\begin{pgfscope}%
\pgfsetrectcap%
\pgfsetmiterjoin%
\pgfsetlinewidth{1.254687pt}%
\definecolor{currentstroke}{rgb}{1.000000,1.000000,1.000000}%
\pgfsetstrokecolor{currentstroke}%
\pgfsetdash{}{0pt}%
\pgfpathmoveto{\pgfqpoint{5.040000in}{0.495000in}}%
\pgfpathlineto{\pgfqpoint{5.040000in}{3.960000in}}%
\pgfusepath{stroke}%
\end{pgfscope}%
\begin{pgfscope}%
\pgfsetrectcap%
\pgfsetmiterjoin%
\pgfsetlinewidth{1.254687pt}%
\definecolor{currentstroke}{rgb}{1.000000,1.000000,1.000000}%
\pgfsetstrokecolor{currentstroke}%
\pgfsetdash{}{0pt}%
\pgfpathmoveto{\pgfqpoint{0.700000in}{0.495000in}}%
\pgfpathlineto{\pgfqpoint{5.040000in}{0.495000in}}%
\pgfusepath{stroke}%
\end{pgfscope}%
\begin{pgfscope}%
\pgfsetrectcap%
\pgfsetmiterjoin%
\pgfsetlinewidth{1.254687pt}%
\definecolor{currentstroke}{rgb}{1.000000,1.000000,1.000000}%
\pgfsetstrokecolor{currentstroke}%
\pgfsetdash{}{0pt}%
\pgfpathmoveto{\pgfqpoint{0.700000in}{3.960000in}}%
\pgfpathlineto{\pgfqpoint{5.040000in}{3.960000in}}%
\pgfusepath{stroke}%
\end{pgfscope}%
\begin{pgfscope}%
\pgfsetbuttcap%
\pgfsetmiterjoin%
\definecolor{currentfill}{rgb}{0.917647,0.917647,0.949020}%
\pgfsetfillcolor{currentfill}%
\pgfsetfillopacity{0.800000}%
\pgfsetlinewidth{1.003750pt}%
\definecolor{currentstroke}{rgb}{0.800000,0.800000,0.800000}%
\pgfsetstrokecolor{currentstroke}%
\pgfsetstrokeopacity{0.800000}%
\pgfsetdash{}{0pt}%
\pgfpathmoveto{\pgfqpoint{0.806944in}{2.967960in}}%
\pgfpathlineto{\pgfqpoint{2.501868in}{2.967960in}}%
\pgfpathquadraticcurveto{\pgfqpoint{2.532423in}{2.967960in}}{\pgfqpoint{2.532423in}{2.998515in}}%
\pgfpathlineto{\pgfqpoint{2.532423in}{3.853056in}}%
\pgfpathquadraticcurveto{\pgfqpoint{2.532423in}{3.883611in}}{\pgfqpoint{2.501868in}{3.883611in}}%
\pgfpathlineto{\pgfqpoint{0.806944in}{3.883611in}}%
\pgfpathquadraticcurveto{\pgfqpoint{0.776389in}{3.883611in}}{\pgfqpoint{0.776389in}{3.853056in}}%
\pgfpathlineto{\pgfqpoint{0.776389in}{2.998515in}}%
\pgfpathquadraticcurveto{\pgfqpoint{0.776389in}{2.967960in}}{\pgfqpoint{0.806944in}{2.967960in}}%
\pgfpathlineto{\pgfqpoint{0.806944in}{2.967960in}}%
\pgfpathclose%
\pgfusepath{stroke,fill}%
\end{pgfscope}%
\begin{pgfscope}%
\pgfsetroundcap%
\pgfsetroundjoin%
\pgfsetlinewidth{1.505625pt}%
\definecolor{currentstroke}{rgb}{0.298039,0.447059,0.690196}%
\pgfsetstrokecolor{currentstroke}%
\pgfsetdash{}{0pt}%
\pgfpathmoveto{\pgfqpoint{0.837500in}{3.766611in}}%
\pgfpathlineto{\pgfqpoint{0.990278in}{3.766611in}}%
\pgfpathlineto{\pgfqpoint{1.143056in}{3.766611in}}%
\pgfusepath{stroke}%
\end{pgfscope}%
\begin{pgfscope}%
\definecolor{textcolor}{rgb}{0.150000,0.150000,0.150000}%
\pgfsetstrokecolor{textcolor}%
\pgfsetfillcolor{textcolor}%
\pgftext[x=1.265278in,y=3.713139in,left,base]{\color{textcolor}{\sffamily\fontsize{11.000000}{13.200000}\selectfont\catcode`\^=\active\def^{\ifmmode\sp\else\^{}\fi}\catcode`\%=\active\def%{\%}resource demand}}%
\end{pgfscope}%
\begin{pgfscope}%
\pgfsetroundcap%
\pgfsetroundjoin%
\pgfsetlinewidth{1.505625pt}%
\definecolor{currentstroke}{rgb}{1.000000,0.647059,0.000000}%
\pgfsetstrokecolor{currentstroke}%
\pgfsetdash{}{0pt}%
\pgfpathmoveto{\pgfqpoint{0.837500in}{3.550499in}}%
\pgfpathlineto{\pgfqpoint{0.990278in}{3.550499in}}%
\pgfpathlineto{\pgfqpoint{1.143056in}{3.550499in}}%
\pgfusepath{stroke}%
\end{pgfscope}%
\begin{pgfscope}%
\definecolor{textcolor}{rgb}{0.150000,0.150000,0.150000}%
\pgfsetstrokecolor{textcolor}%
\pgfsetfillcolor{textcolor}%
\pgftext[x=1.265278in,y=3.497027in,left,base]{\color{textcolor}{\sffamily\fontsize{11.000000}{13.200000}\selectfont\catcode`\^=\active\def^{\ifmmode\sp\else\^{}\fi}\catcode`\%=\active\def%{\%}resource supply}}%
\end{pgfscope}%
\begin{pgfscope}%
\pgfsetbuttcap%
\pgfsetmiterjoin%
\definecolor{currentfill}{rgb}{0.172549,0.627451,0.172549}%
\pgfsetfillcolor{currentfill}%
\pgfsetfillopacity{0.300000}%
\pgfsetlinewidth{1.003750pt}%
\definecolor{currentstroke}{rgb}{0.172549,0.627451,0.172549}%
\pgfsetstrokecolor{currentstroke}%
\pgfsetstrokeopacity{0.300000}%
\pgfsetdash{}{0pt}%
\pgfpathmoveto{\pgfqpoint{0.837500in}{3.279125in}}%
\pgfpathlineto{\pgfqpoint{1.143056in}{3.279125in}}%
\pgfpathlineto{\pgfqpoint{1.143056in}{3.386069in}}%
\pgfpathlineto{\pgfqpoint{0.837500in}{3.386069in}}%
\pgfpathlineto{\pgfqpoint{0.837500in}{3.279125in}}%
\pgfpathclose%
\pgfusepath{stroke,fill}%
\end{pgfscope}%
\begin{pgfscope}%
\definecolor{textcolor}{rgb}{0.150000,0.150000,0.150000}%
\pgfsetstrokecolor{textcolor}%
\pgfsetfillcolor{textcolor}%
\pgftext[x=1.265278in,y=3.279125in,left,base]{\color{textcolor}{\sffamily\fontsize{11.000000}{13.200000}\selectfont\catcode`\^=\active\def^{\ifmmode\sp\else\^{}\fi}\catcode`\%=\active\def%{\%}overprovisioning}}%
\end{pgfscope}%
\begin{pgfscope}%
\pgfsetbuttcap%
\pgfsetmiterjoin%
\definecolor{currentfill}{rgb}{0.839216,0.152941,0.156863}%
\pgfsetfillcolor{currentfill}%
\pgfsetfillopacity{0.300000}%
\pgfsetlinewidth{1.003750pt}%
\definecolor{currentstroke}{rgb}{0.839216,0.152941,0.156863}%
\pgfsetstrokecolor{currentstroke}%
\pgfsetstrokeopacity{0.300000}%
\pgfsetdash{}{0pt}%
\pgfpathmoveto{\pgfqpoint{0.837500in}{3.061223in}}%
\pgfpathlineto{\pgfqpoint{1.143056in}{3.061223in}}%
\pgfpathlineto{\pgfqpoint{1.143056in}{3.168167in}}%
\pgfpathlineto{\pgfqpoint{0.837500in}{3.168167in}}%
\pgfpathlineto{\pgfqpoint{0.837500in}{3.061223in}}%
\pgfpathclose%
\pgfusepath{stroke,fill}%
\end{pgfscope}%
\begin{pgfscope}%
\definecolor{textcolor}{rgb}{0.150000,0.150000,0.150000}%
\pgfsetstrokecolor{textcolor}%
\pgfsetfillcolor{textcolor}%
\pgftext[x=1.265278in,y=3.061223in,left,base]{\color{textcolor}{\sffamily\fontsize{11.000000}{13.200000}\selectfont\catcode`\^=\active\def^{\ifmmode\sp\else\^{}\fi}\catcode`\%=\active\def%{\%}underprovisioning}}%
\end{pgfscope}%
\end{pgfpicture}%
\makeatother%
\endgroup%

    \caption{Resource demand and supply for a website during a typical day with elastic processes.}
    \label{fig:elasticity-application-scaling}
\end{figure}

Elasticity has multiple properties which are interdependent: resource elasticity, cost elasticity and quality elasticity \cite{dustdarPrinciplesElasticProcesses2011}. These properties are discussed in the following sections.

\subsection{Resource Elasticity}

The resource dimension of elasticity is mistakenly often used synonymously with elasticity. Meanwhile, resource elasticity is defined as the degree to which a system is able to adapt to workload changes by claiming and releasing resources autonomously, such that the resource supply matches the current demand as closely as possible \cite{herbstElasticityCloudComputing2013}. Another way to think of this is ``on the fly'' adaptions to load variations \cite{al-dhuraibiElasticityCloudComputing2018}.

What makes this definition easily mistaken, is that it solely considers the aquired resources and not the consequently incurred costs or changing quality.

% The ability to acquire resources as you need them and release resources when you no longer need them \cite{ElasticityAWSWellArchitected}.

\subsection{Cost Elasticity}

Cost elasticity uses cost as its main factor for elasticity decisions. One of the most popular models that build upon cost elasticity is \textit{utility computing}, also known as the \textit{pay-as-you-go} pricing model.

Amazon Web Services uses this elasticity dimension in their EC2 Spot Instance\footnote{\url{https://aws.amazon.com/ec2/spot/}}. AWS provides its unused compute capacity at a large discount to its customers. But because these capacities are volatile, the prices are not fixed but are provided through a bidding process. The potential customer tells AWS their maximum price they are willing to pay. The customer can then run their instances as long as their bidding price is smaller than AWS's Spot Instance price.

\subsection{Quality Elasticity}

Similiar to the already discussed dimensions, quality elasticity is defined as letting software services adapt their mode of operastion to current operating conditions by providing results of varying output quality \cite{larssonQualityElasticityImprovedResource2019}. This means that when resource supply is low, the output quality also may be low. Likewise, if resource supply is sufficient, the output quality will also be high.

\section{Service Level Agreements and Service Level Objectives}

In order to deliver services up to a certain standard, agreements between the service provider, typically the cloud provider, and the service consumers are made - so called \textit{Service Level Agreements (SLA)} \cite{emeakarohaLowLevelMetrics2010d}. Contained inside these SLAs are \textit{Service Level Objectives (SLO)}, which are a ``commitment to maintain a particular state of the service in a given period'' \cite{kellerWSLAFrameworkSpecifying2003}.

SLOs are measurable values, e.g. an applications CPU usage or memory consumption, that have a specified operating target. In the case that this value is violated the supporting infrastructure of the application has to be either increased or decreased. This process of increasing or decreasing resources is called elasticity, which was further discussed in \cref{sec:elasticity}.

\section{Polaris SLO Framework}
\label{sec:polaris}

The Polaris SLO Framework\footnote{\url{https://polaris-slo-cloud.github.io/polaris-slo-framework/}} is a framework that provides a way to bring high-level SLOs to the cloud. It tries to solve the limitation that modern cloud cloud providers only offer rudimentary support for high-level SLOs and customers often need to manually map them to low-level metrics such as CPU usage or memory consumption \cite{pusztaiSLOScriptNovel2021}.

The authors of this framework introduce the concept of \textit{elasticity strategies}. A elasticity strategy is defined as any sequency of actions that adjust the amount of resources provisioned for a workload, their type or the workload configuration. The workload configuration adjustment is especially noteworthy, because workloads handled by Polaris can be affected in all three elasticity dimensions.

Another unique feature of Polaris is its object model, which allows for orchestrator independence. This is achieved by encapsulating all data that is transmitted to the orchestrator into a \texttt{ApiObject} type.

Decoupling SLOs from elasticity strategies is also a feature that Polaris provides. Tight coupling is a charactaristic that is even observed in industry standard scaling mechanisms such as Kubernetes' Horizontal Pod Autoscaler\footnote{\raggedright\url{https://kubernetes.io/docs/tasks/run-application/horizontal-pod-autoscale/}}. This autoscaler provides a CPU usage SLO which can only trigger horizontal elasticity, thus adding or removing CPU resources. To achieve this decoupling, Polaris utilizes an architecture that is depicted in \cref{fig:polaris-architecture}. This allows the controllers to focus on a single task, for example calculating SLO compliance. These individual components are then mapped using a SLO mapping type.

\begin{figure}
    \centering
    \incfig{polaris-architecture}
    \caption{Architecture of the Polaris SLO framework. Metrics controllers, elasticity strategy controllers and targets are decoupled and mapped using a SLO mapping.}
    \label{fig:polaris-architecture}
\end{figure}

\section{k8ssandra}
\label{sec:k8ssandra}

Cassandra is a popular wide-column store NoSQL database that was initially developed at Facebook and later integradet into the Apache Software Foundation\footnote{\url{https://cassandra.apache.org/_/cassandra-basics.html}\label{fn:cassandra-basics}}. Its main features include being easily horizontally scalable, being fully distributed and its schema-less data approach.

Being distributed means, that Cassandra is comprised of a set of nodes. Each nodes tasks and responsibilities are identical. Data is partitioned using a partition key and is replicated between nodes. How many times data is replicated is determined by the \textit{replication factor} or \(RF\). \(RF = 3\) would therefore mean that each piece of data must exist on 3 nodes.

Distributed data also comes with a certain cost. These drawbacks are formulated in the CAP theorem \cite{foxHarvestYieldScalable1999a}. CAP stands for consistency, availability and partition tolerance and the theorem states that databases which handle data in a distributed way can only provide two of these three guarantees. Cassandra, per default, is an AP database. This agreement, however, is configurable on a per-query basis. This means, that whatever consistency level is configured, it represents the minimum amount of nodes that must acknowledge a operation back to the query coordinator node to consider this operation successful.

Queries can be made to any node. Cassandra does not have a main node that takes queries, instead any node that a client connects to takes over the role of coordinator for this specific query. This coordinator node then is responsible for querying other nodes for data in other partitions. This also implies that Cassandra uses peer-to-peer communication between its nodes. This architecture is also depicted in \cref{fig:cassandra-architecture}.

\begin{figure}
    \centering
    \incfig{cassandra-architecture}
    \caption{Architecture of a 5 node Cassandra Cluster. Dotted lines represent possible communication paths.}
    \label{fig:cassandra-architecture}
\end{figure}

Another powerful feature, which makes this database particular interesting for this thesis, is its capabilities to scale. If the partition key is chosen wisely and the database is therefore able to distribute data evenly between nodes, then doubling the amount of nodes also doubles the throughput \footref{fn:cassandra-basics}

k8ssandra\footnote{\url{https://k8ssandra.io}} (pronounced: ``Kate'' +  ``Sandra'') is a open-source cloud-native distribution of Cassandra that is specifically made to run on Kubernetes. It includes several tools for providing a data API, backup/restore processes and automated database repairs. It also includes Kubernetes custom resource definitions (CRDs) to be able to easily deploy Cassandra databases. It also allows easy integration in existing observability and monitoring stacks such as the \texttt{kube-prometheus-stack}\footnote{\raggedright\url{https://artifacthub.io/packages/helm/prometheus-community/kube-prometheus-stack}}.
