\chapter{Related Work}
\label{ch:related-work}

\section{Elasticity and Scalability in Cloud Computing}

\entryneedsurl{thekubernetesauthorsHorizontalPodAutoscaling}
\entryneedsurl{thekubernetesauthorsVerticalPodAutoscaler}
\entryneedsurl{thekubernetesauthorsMultidimensionalPodAutoscaler}

Kubernetes does allow for horizontal and vertical elasticity using their Horizontal and Vertical Pod Autoscalers\cite{thekubernetesauthorsHorizontalPodAutoscaling,thekubernetesauthorsVerticalPodAutoscaler}. While they are commonly used solutions, the used SLO is still tightly coupled to the elasticity action, which the Polaris SLO framework tries to prevent. Using the HPA and VPA in conjunction also brings its own problems, which led to the proposal of a Multidimensional Pod Autoscaler \cite{thekubernetesauthorsMultidimensionalPodAutoscaler}. This autoscaler tries to mitigate timing issues that arise when the HPA and VPA are configured to optimize the same target value, such as CPU usage. Because the MPA still does not allow high-level SLOs and compatible elasticity strategies are also limited, this autoscaler is no option for databases such as K8ssandra.

\citeauthor{laubisCloudAdoptionFineGrained2016} \cite{laubisCloudAdoptionFineGrained2016} recognize the potential of diagonal scalability and introduce a fine-grained pricing model that allows cloud providers to increase prices while staying competitive. They however do not apply their findings regarding diagonal scalability to an application. \citeauthor{quAutoscalingWebApplications2017} \cite{quAutoscalingWebApplications2017} introduce a survey that analyzes the challenges that come with elasticity. They identify diagonal, or as the authors call it, hybrid, elasticity as an opportunity for improvement, but no further possible research directions are given. \citeauthor{al-dhuraibiElasticityCloudComputing2018} \cite{al-dhuraibiElasticityCloudComputing2018} review different approaches for elasticity in cloud computing. Even if rather innovative approaches, such as proactive scaling using artificial intelligence and machine learing, are discussed, a hybrid scaling method such as diagonal scaling is not further considered.

\section{Advanced Elasticity and Scalability of Databases}

\citeauthor{baakindAutomaticScalingCassandra2013} \cite{baakindAutomaticScalingCassandra2013} implements an automatic scaling solution that is Cassandra specific. It adds and removes nodes to the cluster while minimizing performance and usage impacts. This work however does not cover vertical elasticity. \citeauthor{miyokawaElasticityImprovementCassandra2016a} \cite{miyokawaElasticityImprovementCassandra2016a} present research regarding node join times and propose methods to reduce these. Although a very interesting approach, they again only focus on horizontal scale-out and do not include vertical elasticity. \citeauthor{cockroftBenchmarkingCassandraScalability2011} \cite{cockroftBenchmarkingCassandraScalability2011} also focus on horizontal scalability of Cassandra. They managed to achieve linear scalability, but used AWS EC2 instances with a lot more resources than were available for this thesis. Again, vertical elasticity was not considered.

\citeauthor{seyboldElasticityScalableDatabases2016} \cite{seyboldElasticityScalableDatabases2016} evaluate scaling and elasticity features of NoSQL database systems, including Cassandra. They perform different benchmarks on static clusters as well as during scale-out scenarios. They conclude that Cassandra, in contrast to other NoSQL systems such as Couchbase and MongoDB, benefits from large cluster sizes in general while overload situations are not resolved in a favorable way through scale-out. \citeauthor{kuhlenkampBenchmarkingScalabilityElasticity2014} \cite{kuhlenkampBenchmarkingScalabilityElasticity2014} also perform different benchmarking tests of distributed database systems. They not only focus on horizontal scalability but also perform tests for vertical scalability. They conclude that while it is possible to scale Cassandra linearly, it largely depends on the EC2 instance type. However, they do not perform tests with combined horizontal and vertical scaling.
