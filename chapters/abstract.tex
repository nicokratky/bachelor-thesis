\begin{kurzfassung}

Cloud Computing hat in den letzten Jahren immens an Popularität gewonnen. Eigenschaften wie Elastizität und Pay-as-you-go-Preismodelle haben die Kunden dazu veranlasst, ihre Workload-Bereitstellungsmodelle zu überdenken. Um die Leistungserwartungen zu definieren, verwenden Cloud-Service-Anbieter Service Level Objectives (SLOs). Die meisten SLOs basieren auf Low-Level-Metriken und sind eng an eine bestimmte Elastizitätsstrategie gekoppelt, wie z. B. horizontale Skalierung. Das Polaris SLO löst diese Probleme durch die Einführung von High-Level SLOs, die lose an Elastizitätsstrategien gekoppelt sind. In dieser Arbeit wird eine diagonale Elastizitätsstrategie für Apache Cassandra vorgestellt. Diagonale Elastizität ist definiert als eine Kombination aus vertikaler und horizontaler Elastizität. Es werden die verschiedenen Komponenten vorgestellt, die zur Erreichung dieses Ziels notwendig sind. Schließlich wird das Ergebnis bewertet und mit der alleinigen Verwendung von vertikaler oder horizontaler Elastizität verglichen. Diese Auswertung zeigt, dass es mit diagonaler Elastizität für Cassandra möglich ist, die Ressourceneffizienz durch vertikale Skalierung zu erhöhen und gleichzeitig den Durchsatz zu steigern, indem dem Cluster bei hohem Bedarf weitere Knoten hinzugefügt werden.

\end{kurzfassung}

\begin{abstract}

Cloud computing has risen immensly in popularity over the recent years. Properties such as elasticity and pay-as-you-go pricing models have motivated customers to reconsider their workload deployment models. To define performance expectations, cloud service providers use Service Level Objectives (SLOs). Most SLOs rely on low-level metrics and are tightly coupled to a specific scaling strategy, such as horizontal scaling. The Polaris SLO tackles these issues by introducing high-level SLOs that are loosely coupled to elasticity (scaling) strategies. This thesis presents a diagonal elasticity strategy for Apache Cassandra. Diagonal elasticity is defined as a combination of vertical and horizontal elasticity. The different components that are necessary to achieve this are introduced. Finally the result is evaluated and compared to using vertical or horizontal elasticity alone. This evaluation shows that using diagonal elasticity for Cassandra it is possible to increase resource efficiency by scaling vertically while also increasing throughput by adding nodes to the cluster when demand is high.

\end{abstract}
