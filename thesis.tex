% Copyright (C) 2014-2023 by Thomas Auzinger <thomas@auzinger.name>

\documentclass[draft,final]{vutinfth} % Remove option 'final' to obtain debug information.

% Load packages to allow in- and output of non-ASCII characters.
\usepackage{lmodern}        % Use an extension of the original Computer Modern font to minimize the use of bitmapped letters.
\usepackage[T1]{fontenc}    % Determines font encoding of the output. Font packages have to be included before this line.
\usepackage[utf8]{inputenc} % Determines encoding of the input. All input files have to use UTF8 encoding.

% Extended LaTeX functionality is enables by including packages with \usepackage{...}.
\usepackage{amsmath}    % Extended typesetting of mathematical expression.
\usepackage{amssymb}    % Provides a multitude of mathematical symbols.
\usepackage{mathtools}  % Further extensions of mathematical typesetting.
\usepackage{microtype}  % Small-scale typographic enhancements.
\usepackage[inline]{enumitem} % User control over the layout of lists (itemize, enumerate, description).
\usepackage{multirow}   % Allows table elements to span several rows.
\usepackage{booktabs}   % Improves the typesetting of tables.
\usepackage{subcaption} % Allows the use of subfigures and enables their referencing.
\usepackage[ruled,linesnumbered,algochapter]{algorithm2e} % Enables the writing of pseudo code.
\usepackage[usenames,dvipsnames,table]{xcolor} % Allows the definition and use of colors. This package has to be included before tikz.
\usepackage{nag}       % Issues warnings when best practices in writing LaTeX documents are violated.
\usepackage{todonotes} % Provides tooltip-like todo notes.
\usepackage{hyperref}  % Enables hyperlinking in the electronic document version. This package has to be included second to last.
\usepackage[noabbrev,nameinlink]{cleveref}

\usepackage{silence} % Silence compilation warnings
\WarningFilter{glossaries}{Overriding \printglossary}
\WarningFilter{glossaries}{Overriding `theglossary'}

\usepackage[acronym,toc]{glossaries} % Enables the generation of glossaries and lists of acronyms. This package has to be included last.

\usepackage{esdiff}

\newcommand{\authorname}{Nico Kratky}
\newcommand{\thesistitle}{Enabling k8ssandra for Diagonal Elasticity Using the Polaris SLO Framework}
\newcommand{\gerthesistitle}{Diagonale Elastizität für k8ssandra mithilfe des Polaris SLO Frameworks}

% Set PDF document properties
\hypersetup{
    pdfpagelayout   = TwoPageRight,           % How the document is shown in PDF viewers (optional).
    linkbordercolor = {Melon},                % The color of the borders of boxes around hyperlinks (optional).
    pdfauthor       = {\authorname},          % The author's name in the document properties (optional).
    pdftitle        = {\thesistitle},         % The document's title in the document properties (optional).
    pdfsubject      = {Subject},              % The document's subject in the document properties (optional).
    pdfkeywords     = {a, list, of, keywords} % The document's keywords in the document properties (optional).
}

\setpnumwidth{2.5em}        % Avoid overfull hboxes in the table of contents (see memoir manual).
\setsecnumdepth{subsection} % Enumerate subsections.

\nonzeroparskip             % Create space between paragraphs (optional).
\setlength{\parindent}{0pt} % Remove paragraph indentation (optional).

\makeindex      % Use an optional index.
\makeglossaries % Use an optional glossary.
%\glstocfalse   % Remove the glossaries from the table of contents.

% Set persons with 4 arguments:
%  {title before name}{name}{title after name}{gender}
%  where both titles are optional (i.e. can be given as empty brackets {}).
\setauthor{}{\authorname}{}{male}
\setadvisor{Assistant Prof. Dipl.-Ing. Dr.techn.}{Stefan Nastic}{BSc}{male}

% For bachelor and master theses:
\setfirstassistant{Projektass. Dipl.-Ing.}{Thomas Werner Pusztai}{}{male}

% Required data.
\setregnumber{11909858}
\setdate{01}{01}{2024} % Set date with 3 arguments: {day}{month}{year}.
\settitle{\thesistitle}{\thesistitle} % Sets English and German version of the title (both can be English or German). If your title contains commas, enclose it with additional curvy brackets (i.e., {{your title}}) or define it as a macro as done with \thesistitle.
\setsubtitle{}{} % Sets English and German version of the subtitle (both can be English or German).

% Select the thesis type: bachelor / master / doctor.
\setthesis{bachelor}

% Set english and german name of curriculum
\setcurriculum{Software \& Information Engineering}{Software \& Information Engineering}

\begin{document}

\frontmatter % Switches to roman numbering.

\addtitlepage{naustrian} % German title page.
\addtitlepage{english} % English title page.

\addstatementpage

\begin{acknowledgements*}

First of all, I would like to thank my supervisors Thomas Pusztai and Stefan Nastic for advising my work. Thank you for guiding me through this process.

Moreover, I would like to thank my parents and entire family for always supporting me, no matter what.

\end{acknowledgements*}


\begin{kurzfassung}
\todo{Kurzfassung hinzufügen}
\end{kurzfassung}

\begin{abstract}
\todo{Add abstract}
\end{abstract}


\selectlanguage{english}

% Add a table of contents (toc).
\tableofcontents

% Switch to arabic numbering and start the enumeration of chapters in the table of content.
\mainmatter

\chapter{Introduction}
\label{ch:introduction}

The cloud computing paradigm emerged in the recent decade and provides ``ubiquitous, convenient, on-demand network access to a shared pool of configurable resources that can be rapidly provisioned and released with minimal management effort or service provider interaction'' \cite{mellNISTDefinitionCloud2011a}. These properties together with a pay-per-use principle motivated many customer to adopt this technology. Cloud computing can be differentiated in three basic service models:

\begin{enumerate}
    \item \textbf{Infrastructure as a Service (IaaS)}. This model enables customers to provision processing and storage infrastructure to run arbitrary software. While the consumer has control over both application and operating system, they are not responsible for controling and maintaining the underlying infrastructure. A well known example is Amazon Elastic Cloud Compute (EC2)\footnote{\url{https://aws.amazon.com/ec2/}}.

    \item \textbf{Platform as a Service (PaaS)}. The customer is able to deploy applications to provided hosting infrastructure. Control is given only to the deployed application and possibly single configuration settings. The underlying infrastructure is solely controlled by the provider. Notable products include the Google App Engine\footnote{\url{https://cloud.google.com/appengine}}.

    \item \textbf{Software as a Service (SaaS)}. This model allows users to use a provided application that runs on cloud infrastructure. Users do not control the application nor the underlying infrastructure. A suitable example would include Astra DB by DataStax\footnote{\url{https://www.datastax.com/products/datastax-astra}}.
\end{enumerate}

\section{Problem Statement}
\label{sec:problem-statement}
% what is the objective?
%   * implementation of combination of vertical and horizontal elasticity strategy
% why do we want diagonal elasticity?

The main goal of this thesis is to implement a elasticity strategy that combines vertical scaling with horizontal scaling. This elasticity strategy should be implementend using the Polaris SLO framework. This to be implemented strategy will be from now on called ``diagonal elasticity strategy''.

As of now, common automatic scaling mechanisms include horizontal and vertical scaling. Kubernetes,for example, has solutions to both. The Horizontal Pod Autoscaler\footnote{\raggedright\url{https://kubernetes.io/docs/tasks/run-application/horizontal-pod-autoscale/}} updates the amount of deployed pods to match current demand. Likewise, the Vertical Pod Autoscaler\footnote{\raggedright\url{https://github.com/kubernetes/autoscaler/tree/master/vertical-pod-autoscaler}} tries to set resource requests and limits based on current usage. Both of these are however limited to their single dimension. For some applications it can be definitely advantageous to be scaled both horizontally and vertically.

\section{Structure of the Thesis}
\label{sec:structure}

\begin{itemize}
    \item \Cref{ch:background} introduces concepts and terminology used throughout this thesis. It discusses the concepts of elasticity in cloud computing and introduces the framework that is used to implement elasticity strategies.

    \item \Cref{ch:implementation} presents the implementation details of the project. It first shows the used metric, then introduces the different service level objectives and finally discussed the elasticity strategies.

    \item \Cref{ch:evaluation} evaluates the implemented elasticity strategies. This is done by running stress tests in different scenarios.

    \item \Cref{ch:conclusion} concludes this thesis. It discussed limitations and provides an outlook into possible future work.
\end{itemize}


\chapter{Background}
\label{ch:background}

Over the last years and decades cloud computing emerged as a paradigm which allows customers to receive compute power in a pay-as-you-go and on-demand manner.

This chapter introduces some terminology and concepts that are used throughout this thesis. First the cloud computing concepts are defined and then the used framework is introduced.

\section{Elasticity in Cloud Computing}
\label{sec:elasticity}

Elasticity is one of the core concepts that solves a big problem of cloud computing: providing limited resources for potentially unlimited use. The solution is to scale workloads up and down as needed, to claim resources when bigger load is experienced and release resources when they are not needed, therefore making them available to other workloads.

The term elasticity in computing is conceptually similiar to the term in physics. Wikipedia, for example, defines elasticity as follows: ``In physics and materials science, elasticity is the ability of a body to resist a distorting influence and to return to its original size and shape when that influence or force is removed. Solid objects will deform when adequate loads are applied to them; if the material is elastic, the object will return to its initial shape and size after removal.''\footnote{\url{https://en.wikipedia.org/wiki/Elasticity_(physics)}}

The formula - which takes a more mathematical approach - of elasticity can be defined as \[ e(Y, X) = \diff{Y}{X} \frac{X}{Y}, \] where \(e(Y, X)\) is the elasticity of \(Y\) with respect to \(X\) \cite{dustdarPrinciplesElasticProcesses2011}.

To illustrate this imagine an application that serves some content to its customers. These customers typically interact with the application during the day. This means that the application experiences significantly less load during the night. Once people wake up in the morning the load rises until it peaks in the afternoon. Then the load falls again when people go to sleep in the evening. Using this example it can be seen in \cref{fig:elasticity-application-no-scaling} that during the night the resources of the application are overprovisioned and during the day the resoures are underprovisioned.

\begin{figure}
    \centering
    %% Creator: Matplotlib, PGF backend
%%
%% To include the figure in your LaTeX document, write
%%   \input{<filename>.pgf}
%%
%% Make sure the required packages are loaded in your preamble
%%   \usepackage{pgf}
%%
%% Also ensure that all the required font packages are loaded; for instance,
%% the lmodern package is sometimes necessary when using math font.
%%   \usepackage{lmodern}
%%
%% Figures using additional raster images can only be included by \input if
%% they are in the same directory as the main LaTeX file. For loading figures
%% from other directories you can use the `import` package
%%   \usepackage{import}
%%
%% and then include the figures with
%%   \import{<path to file>}{<filename>.pgf}
%%
%% Matplotlib used the following preamble
%%   \def\mathdefault#1{#1}
%%   \everymath=\expandafter{\the\everymath\displaystyle}
%%   
%%   \usepackage{fontspec}
%%   \setmainfont{DejaVuSerif.ttf}[Path=\detokenize{/Users/nkratky/private/polaris-elasticity-strategies/test/scripts/.venv/lib/python3.11/site-packages/matplotlib/mpl-data/fonts/ttf/}]
%%   \setsansfont{Arial.ttf}[Path=\detokenize{/System/Library/Fonts/Supplemental/}]
%%   \setmonofont{DejaVuSansMono.ttf}[Path=\detokenize{/Users/nkratky/private/polaris-elasticity-strategies/test/scripts/.venv/lib/python3.11/site-packages/matplotlib/mpl-data/fonts/ttf/}]
%%   \makeatletter\@ifpackageloaded{underscore}{}{\usepackage[strings]{underscore}}\makeatother
%%
\begingroup%
\makeatletter%
\begin{pgfpicture}%
\pgfpathrectangle{\pgfpointorigin}{\pgfqpoint{5.600000in}{4.500000in}}%
\pgfusepath{use as bounding box, clip}%
\begin{pgfscope}%
\pgfsetbuttcap%
\pgfsetmiterjoin%
\definecolor{currentfill}{rgb}{1.000000,1.000000,1.000000}%
\pgfsetfillcolor{currentfill}%
\pgfsetlinewidth{0.000000pt}%
\definecolor{currentstroke}{rgb}{1.000000,1.000000,1.000000}%
\pgfsetstrokecolor{currentstroke}%
\pgfsetdash{}{0pt}%
\pgfpathmoveto{\pgfqpoint{0.000000in}{0.000000in}}%
\pgfpathlineto{\pgfqpoint{5.600000in}{0.000000in}}%
\pgfpathlineto{\pgfqpoint{5.600000in}{4.500000in}}%
\pgfpathlineto{\pgfqpoint{0.000000in}{4.500000in}}%
\pgfpathlineto{\pgfqpoint{0.000000in}{0.000000in}}%
\pgfpathclose%
\pgfusepath{fill}%
\end{pgfscope}%
\begin{pgfscope}%
\pgfsetbuttcap%
\pgfsetmiterjoin%
\definecolor{currentfill}{rgb}{0.917647,0.917647,0.949020}%
\pgfsetfillcolor{currentfill}%
\pgfsetlinewidth{0.000000pt}%
\definecolor{currentstroke}{rgb}{0.000000,0.000000,0.000000}%
\pgfsetstrokecolor{currentstroke}%
\pgfsetstrokeopacity{0.000000}%
\pgfsetdash{}{0pt}%
\pgfpathmoveto{\pgfqpoint{0.700000in}{0.495000in}}%
\pgfpathlineto{\pgfqpoint{5.040000in}{0.495000in}}%
\pgfpathlineto{\pgfqpoint{5.040000in}{3.960000in}}%
\pgfpathlineto{\pgfqpoint{0.700000in}{3.960000in}}%
\pgfpathlineto{\pgfqpoint{0.700000in}{0.495000in}}%
\pgfpathclose%
\pgfusepath{fill}%
\end{pgfscope}%
\begin{pgfscope}%
\pgfpathrectangle{\pgfqpoint{0.700000in}{0.495000in}}{\pgfqpoint{4.340000in}{3.465000in}}%
\pgfusepath{clip}%
\pgfsetroundcap%
\pgfsetroundjoin%
\pgfsetlinewidth{1.003750pt}%
\definecolor{currentstroke}{rgb}{1.000000,1.000000,1.000000}%
\pgfsetstrokecolor{currentstroke}%
\pgfsetdash{}{0pt}%
\pgfpathmoveto{\pgfqpoint{0.897273in}{0.495000in}}%
\pgfpathlineto{\pgfqpoint{0.897273in}{3.960000in}}%
\pgfusepath{stroke}%
\end{pgfscope}%
\begin{pgfscope}%
\definecolor{textcolor}{rgb}{0.150000,0.150000,0.150000}%
\pgfsetstrokecolor{textcolor}%
\pgfsetfillcolor{textcolor}%
\pgftext[x=0.897273in,y=0.363056in,,top]{\color{textcolor}{\sffamily\fontsize{11.000000}{13.200000}\selectfont\catcode`\^=\active\def^{\ifmmode\sp\else\^{}\fi}\catcode`\%=\active\def%{\%}00:00}}%
\end{pgfscope}%
\begin{pgfscope}%
\pgfpathrectangle{\pgfqpoint{0.700000in}{0.495000in}}{\pgfqpoint{4.340000in}{3.465000in}}%
\pgfusepath{clip}%
\pgfsetroundcap%
\pgfsetroundjoin%
\pgfsetlinewidth{1.003750pt}%
\definecolor{currentstroke}{rgb}{1.000000,1.000000,1.000000}%
\pgfsetstrokecolor{currentstroke}%
\pgfsetdash{}{0pt}%
\pgfpathmoveto{\pgfqpoint{1.390455in}{0.495000in}}%
\pgfpathlineto{\pgfqpoint{1.390455in}{3.960000in}}%
\pgfusepath{stroke}%
\end{pgfscope}%
\begin{pgfscope}%
\definecolor{textcolor}{rgb}{0.150000,0.150000,0.150000}%
\pgfsetstrokecolor{textcolor}%
\pgfsetfillcolor{textcolor}%
\pgftext[x=1.390455in,y=0.363056in,,top]{\color{textcolor}{\sffamily\fontsize{11.000000}{13.200000}\selectfont\catcode`\^=\active\def^{\ifmmode\sp\else\^{}\fi}\catcode`\%=\active\def%{\%}03:00}}%
\end{pgfscope}%
\begin{pgfscope}%
\pgfpathrectangle{\pgfqpoint{0.700000in}{0.495000in}}{\pgfqpoint{4.340000in}{3.465000in}}%
\pgfusepath{clip}%
\pgfsetroundcap%
\pgfsetroundjoin%
\pgfsetlinewidth{1.003750pt}%
\definecolor{currentstroke}{rgb}{1.000000,1.000000,1.000000}%
\pgfsetstrokecolor{currentstroke}%
\pgfsetdash{}{0pt}%
\pgfpathmoveto{\pgfqpoint{1.883636in}{0.495000in}}%
\pgfpathlineto{\pgfqpoint{1.883636in}{3.960000in}}%
\pgfusepath{stroke}%
\end{pgfscope}%
\begin{pgfscope}%
\definecolor{textcolor}{rgb}{0.150000,0.150000,0.150000}%
\pgfsetstrokecolor{textcolor}%
\pgfsetfillcolor{textcolor}%
\pgftext[x=1.883636in,y=0.363056in,,top]{\color{textcolor}{\sffamily\fontsize{11.000000}{13.200000}\selectfont\catcode`\^=\active\def^{\ifmmode\sp\else\^{}\fi}\catcode`\%=\active\def%{\%}06:00}}%
\end{pgfscope}%
\begin{pgfscope}%
\pgfpathrectangle{\pgfqpoint{0.700000in}{0.495000in}}{\pgfqpoint{4.340000in}{3.465000in}}%
\pgfusepath{clip}%
\pgfsetroundcap%
\pgfsetroundjoin%
\pgfsetlinewidth{1.003750pt}%
\definecolor{currentstroke}{rgb}{1.000000,1.000000,1.000000}%
\pgfsetstrokecolor{currentstroke}%
\pgfsetdash{}{0pt}%
\pgfpathmoveto{\pgfqpoint{2.376818in}{0.495000in}}%
\pgfpathlineto{\pgfqpoint{2.376818in}{3.960000in}}%
\pgfusepath{stroke}%
\end{pgfscope}%
\begin{pgfscope}%
\definecolor{textcolor}{rgb}{0.150000,0.150000,0.150000}%
\pgfsetstrokecolor{textcolor}%
\pgfsetfillcolor{textcolor}%
\pgftext[x=2.376818in,y=0.363056in,,top]{\color{textcolor}{\sffamily\fontsize{11.000000}{13.200000}\selectfont\catcode`\^=\active\def^{\ifmmode\sp\else\^{}\fi}\catcode`\%=\active\def%{\%}09:00}}%
\end{pgfscope}%
\begin{pgfscope}%
\pgfpathrectangle{\pgfqpoint{0.700000in}{0.495000in}}{\pgfqpoint{4.340000in}{3.465000in}}%
\pgfusepath{clip}%
\pgfsetroundcap%
\pgfsetroundjoin%
\pgfsetlinewidth{1.003750pt}%
\definecolor{currentstroke}{rgb}{1.000000,1.000000,1.000000}%
\pgfsetstrokecolor{currentstroke}%
\pgfsetdash{}{0pt}%
\pgfpathmoveto{\pgfqpoint{2.870000in}{0.495000in}}%
\pgfpathlineto{\pgfqpoint{2.870000in}{3.960000in}}%
\pgfusepath{stroke}%
\end{pgfscope}%
\begin{pgfscope}%
\definecolor{textcolor}{rgb}{0.150000,0.150000,0.150000}%
\pgfsetstrokecolor{textcolor}%
\pgfsetfillcolor{textcolor}%
\pgftext[x=2.870000in,y=0.363056in,,top]{\color{textcolor}{\sffamily\fontsize{11.000000}{13.200000}\selectfont\catcode`\^=\active\def^{\ifmmode\sp\else\^{}\fi}\catcode`\%=\active\def%{\%}12:00}}%
\end{pgfscope}%
\begin{pgfscope}%
\pgfpathrectangle{\pgfqpoint{0.700000in}{0.495000in}}{\pgfqpoint{4.340000in}{3.465000in}}%
\pgfusepath{clip}%
\pgfsetroundcap%
\pgfsetroundjoin%
\pgfsetlinewidth{1.003750pt}%
\definecolor{currentstroke}{rgb}{1.000000,1.000000,1.000000}%
\pgfsetstrokecolor{currentstroke}%
\pgfsetdash{}{0pt}%
\pgfpathmoveto{\pgfqpoint{3.363182in}{0.495000in}}%
\pgfpathlineto{\pgfqpoint{3.363182in}{3.960000in}}%
\pgfusepath{stroke}%
\end{pgfscope}%
\begin{pgfscope}%
\definecolor{textcolor}{rgb}{0.150000,0.150000,0.150000}%
\pgfsetstrokecolor{textcolor}%
\pgfsetfillcolor{textcolor}%
\pgftext[x=3.363182in,y=0.363056in,,top]{\color{textcolor}{\sffamily\fontsize{11.000000}{13.200000}\selectfont\catcode`\^=\active\def^{\ifmmode\sp\else\^{}\fi}\catcode`\%=\active\def%{\%}15:00}}%
\end{pgfscope}%
\begin{pgfscope}%
\pgfpathrectangle{\pgfqpoint{0.700000in}{0.495000in}}{\pgfqpoint{4.340000in}{3.465000in}}%
\pgfusepath{clip}%
\pgfsetroundcap%
\pgfsetroundjoin%
\pgfsetlinewidth{1.003750pt}%
\definecolor{currentstroke}{rgb}{1.000000,1.000000,1.000000}%
\pgfsetstrokecolor{currentstroke}%
\pgfsetdash{}{0pt}%
\pgfpathmoveto{\pgfqpoint{3.856364in}{0.495000in}}%
\pgfpathlineto{\pgfqpoint{3.856364in}{3.960000in}}%
\pgfusepath{stroke}%
\end{pgfscope}%
\begin{pgfscope}%
\definecolor{textcolor}{rgb}{0.150000,0.150000,0.150000}%
\pgfsetstrokecolor{textcolor}%
\pgfsetfillcolor{textcolor}%
\pgftext[x=3.856364in,y=0.363056in,,top]{\color{textcolor}{\sffamily\fontsize{11.000000}{13.200000}\selectfont\catcode`\^=\active\def^{\ifmmode\sp\else\^{}\fi}\catcode`\%=\active\def%{\%}18:00}}%
\end{pgfscope}%
\begin{pgfscope}%
\pgfpathrectangle{\pgfqpoint{0.700000in}{0.495000in}}{\pgfqpoint{4.340000in}{3.465000in}}%
\pgfusepath{clip}%
\pgfsetroundcap%
\pgfsetroundjoin%
\pgfsetlinewidth{1.003750pt}%
\definecolor{currentstroke}{rgb}{1.000000,1.000000,1.000000}%
\pgfsetstrokecolor{currentstroke}%
\pgfsetdash{}{0pt}%
\pgfpathmoveto{\pgfqpoint{4.349545in}{0.495000in}}%
\pgfpathlineto{\pgfqpoint{4.349545in}{3.960000in}}%
\pgfusepath{stroke}%
\end{pgfscope}%
\begin{pgfscope}%
\definecolor{textcolor}{rgb}{0.150000,0.150000,0.150000}%
\pgfsetstrokecolor{textcolor}%
\pgfsetfillcolor{textcolor}%
\pgftext[x=4.349545in,y=0.363056in,,top]{\color{textcolor}{\sffamily\fontsize{11.000000}{13.200000}\selectfont\catcode`\^=\active\def^{\ifmmode\sp\else\^{}\fi}\catcode`\%=\active\def%{\%}21:00}}%
\end{pgfscope}%
\begin{pgfscope}%
\pgfpathrectangle{\pgfqpoint{0.700000in}{0.495000in}}{\pgfqpoint{4.340000in}{3.465000in}}%
\pgfusepath{clip}%
\pgfsetroundcap%
\pgfsetroundjoin%
\pgfsetlinewidth{1.003750pt}%
\definecolor{currentstroke}{rgb}{1.000000,1.000000,1.000000}%
\pgfsetstrokecolor{currentstroke}%
\pgfsetdash{}{0pt}%
\pgfpathmoveto{\pgfqpoint{4.842727in}{0.495000in}}%
\pgfpathlineto{\pgfqpoint{4.842727in}{3.960000in}}%
\pgfusepath{stroke}%
\end{pgfscope}%
\begin{pgfscope}%
\definecolor{textcolor}{rgb}{0.150000,0.150000,0.150000}%
\pgfsetstrokecolor{textcolor}%
\pgfsetfillcolor{textcolor}%
\pgftext[x=4.842727in,y=0.363056in,,top]{\color{textcolor}{\sffamily\fontsize{11.000000}{13.200000}\selectfont\catcode`\^=\active\def^{\ifmmode\sp\else\^{}\fi}\catcode`\%=\active\def%{\%}24:00}}%
\end{pgfscope}%
\begin{pgfscope}%
\definecolor{textcolor}{rgb}{0.150000,0.150000,0.150000}%
\pgfsetstrokecolor{textcolor}%
\pgfsetfillcolor{textcolor}%
\pgftext[x=2.870000in,y=0.167777in,,top]{\color{textcolor}{\sffamily\fontsize{12.000000}{14.400000}\selectfont\catcode`\^=\active\def^{\ifmmode\sp\else\^{}\fi}\catcode`\%=\active\def%{\%}Time}}%
\end{pgfscope}%
\begin{pgfscope}%
\pgfpathrectangle{\pgfqpoint{0.700000in}{0.495000in}}{\pgfqpoint{4.340000in}{3.465000in}}%
\pgfusepath{clip}%
\pgfsetroundcap%
\pgfsetroundjoin%
\pgfsetlinewidth{1.003750pt}%
\definecolor{currentstroke}{rgb}{1.000000,1.000000,1.000000}%
\pgfsetstrokecolor{currentstroke}%
\pgfsetdash{}{0pt}%
\pgfpathmoveto{\pgfqpoint{0.700000in}{0.677992in}}%
\pgfpathlineto{\pgfqpoint{5.040000in}{0.677992in}}%
\pgfusepath{stroke}%
\end{pgfscope}%
\begin{pgfscope}%
\pgfpathrectangle{\pgfqpoint{0.700000in}{0.495000in}}{\pgfqpoint{4.340000in}{3.465000in}}%
\pgfusepath{clip}%
\pgfsetroundcap%
\pgfsetroundjoin%
\pgfsetlinewidth{1.003750pt}%
\definecolor{currentstroke}{rgb}{1.000000,1.000000,1.000000}%
\pgfsetstrokecolor{currentstroke}%
\pgfsetdash{}{0pt}%
\pgfpathmoveto{\pgfqpoint{0.700000in}{1.242032in}}%
\pgfpathlineto{\pgfqpoint{5.040000in}{1.242032in}}%
\pgfusepath{stroke}%
\end{pgfscope}%
\begin{pgfscope}%
\pgfpathrectangle{\pgfqpoint{0.700000in}{0.495000in}}{\pgfqpoint{4.340000in}{3.465000in}}%
\pgfusepath{clip}%
\pgfsetroundcap%
\pgfsetroundjoin%
\pgfsetlinewidth{1.003750pt}%
\definecolor{currentstroke}{rgb}{1.000000,1.000000,1.000000}%
\pgfsetstrokecolor{currentstroke}%
\pgfsetdash{}{0pt}%
\pgfpathmoveto{\pgfqpoint{0.700000in}{1.806071in}}%
\pgfpathlineto{\pgfqpoint{5.040000in}{1.806071in}}%
\pgfusepath{stroke}%
\end{pgfscope}%
\begin{pgfscope}%
\pgfpathrectangle{\pgfqpoint{0.700000in}{0.495000in}}{\pgfqpoint{4.340000in}{3.465000in}}%
\pgfusepath{clip}%
\pgfsetroundcap%
\pgfsetroundjoin%
\pgfsetlinewidth{1.003750pt}%
\definecolor{currentstroke}{rgb}{1.000000,1.000000,1.000000}%
\pgfsetstrokecolor{currentstroke}%
\pgfsetdash{}{0pt}%
\pgfpathmoveto{\pgfqpoint{0.700000in}{2.370111in}}%
\pgfpathlineto{\pgfqpoint{5.040000in}{2.370111in}}%
\pgfusepath{stroke}%
\end{pgfscope}%
\begin{pgfscope}%
\pgfpathrectangle{\pgfqpoint{0.700000in}{0.495000in}}{\pgfqpoint{4.340000in}{3.465000in}}%
\pgfusepath{clip}%
\pgfsetroundcap%
\pgfsetroundjoin%
\pgfsetlinewidth{1.003750pt}%
\definecolor{currentstroke}{rgb}{1.000000,1.000000,1.000000}%
\pgfsetstrokecolor{currentstroke}%
\pgfsetdash{}{0pt}%
\pgfpathmoveto{\pgfqpoint{0.700000in}{2.934151in}}%
\pgfpathlineto{\pgfqpoint{5.040000in}{2.934151in}}%
\pgfusepath{stroke}%
\end{pgfscope}%
\begin{pgfscope}%
\pgfpathrectangle{\pgfqpoint{0.700000in}{0.495000in}}{\pgfqpoint{4.340000in}{3.465000in}}%
\pgfusepath{clip}%
\pgfsetroundcap%
\pgfsetroundjoin%
\pgfsetlinewidth{1.003750pt}%
\definecolor{currentstroke}{rgb}{1.000000,1.000000,1.000000}%
\pgfsetstrokecolor{currentstroke}%
\pgfsetdash{}{0pt}%
\pgfpathmoveto{\pgfqpoint{0.700000in}{3.498191in}}%
\pgfpathlineto{\pgfqpoint{5.040000in}{3.498191in}}%
\pgfusepath{stroke}%
\end{pgfscope}%
\begin{pgfscope}%
\definecolor{textcolor}{rgb}{0.150000,0.150000,0.150000}%
\pgfsetstrokecolor{textcolor}%
\pgfsetfillcolor{textcolor}%
\pgftext[x=0.512500in,y=2.227500in,,bottom,rotate=90.000000]{\color{textcolor}{\sffamily\fontsize{12.000000}{14.400000}\selectfont\catcode`\^=\active\def^{\ifmmode\sp\else\^{}\fi}\catcode`\%=\active\def%{\%}Resources}}%
\end{pgfscope}%
\begin{pgfscope}%
\pgfpathrectangle{\pgfqpoint{0.700000in}{0.495000in}}{\pgfqpoint{4.340000in}{3.465000in}}%
\pgfusepath{clip}%
\pgfsetbuttcap%
\pgfsetroundjoin%
\definecolor{currentfill}{rgb}{0.172549,0.627451,0.172549}%
\pgfsetfillcolor{currentfill}%
\pgfsetfillopacity{0.300000}%
\pgfsetlinewidth{1.003750pt}%
\definecolor{currentstroke}{rgb}{0.172549,0.627451,0.172549}%
\pgfsetstrokecolor{currentstroke}%
\pgfsetstrokeopacity{0.300000}%
\pgfsetdash{}{0pt}%
\pgfpathmoveto{\pgfqpoint{0.897273in}{2.088091in}}%
\pgfpathlineto{\pgfqpoint{0.897273in}{0.677992in}}%
\pgfpathlineto{\pgfqpoint{0.905179in}{0.677487in}}%
\pgfpathlineto{\pgfqpoint{0.913086in}{0.676935in}}%
\pgfpathlineto{\pgfqpoint{0.920993in}{0.676340in}}%
\pgfpathlineto{\pgfqpoint{0.928900in}{0.675706in}}%
\pgfpathlineto{\pgfqpoint{0.936806in}{0.675034in}}%
\pgfpathlineto{\pgfqpoint{0.944713in}{0.674329in}}%
\pgfpathlineto{\pgfqpoint{0.952620in}{0.673593in}}%
\pgfpathlineto{\pgfqpoint{0.960527in}{0.672830in}}%
\pgfpathlineto{\pgfqpoint{0.968433in}{0.672042in}}%
\pgfpathlineto{\pgfqpoint{0.976340in}{0.671233in}}%
\pgfpathlineto{\pgfqpoint{0.984247in}{0.670405in}}%
\pgfpathlineto{\pgfqpoint{0.992153in}{0.669563in}}%
\pgfpathlineto{\pgfqpoint{1.000060in}{0.668708in}}%
\pgfpathlineto{\pgfqpoint{1.007967in}{0.667845in}}%
\pgfpathlineto{\pgfqpoint{1.015874in}{0.666976in}}%
\pgfpathlineto{\pgfqpoint{1.023780in}{0.666104in}}%
\pgfpathlineto{\pgfqpoint{1.031687in}{0.665233in}}%
\pgfpathlineto{\pgfqpoint{1.039594in}{0.664366in}}%
\pgfpathlineto{\pgfqpoint{1.047500in}{0.663505in}}%
\pgfpathlineto{\pgfqpoint{1.055407in}{0.662654in}}%
\pgfpathlineto{\pgfqpoint{1.063314in}{0.661816in}}%
\pgfpathlineto{\pgfqpoint{1.071221in}{0.660994in}}%
\pgfpathlineto{\pgfqpoint{1.079127in}{0.660191in}}%
\pgfpathlineto{\pgfqpoint{1.087034in}{0.659411in}}%
\pgfpathlineto{\pgfqpoint{1.094941in}{0.658656in}}%
\pgfpathlineto{\pgfqpoint{1.102848in}{0.657930in}}%
\pgfpathlineto{\pgfqpoint{1.110754in}{0.657235in}}%
\pgfpathlineto{\pgfqpoint{1.118661in}{0.656575in}}%
\pgfpathlineto{\pgfqpoint{1.126568in}{0.655953in}}%
\pgfpathlineto{\pgfqpoint{1.134474in}{0.655371in}}%
\pgfpathlineto{\pgfqpoint{1.142381in}{0.654834in}}%
\pgfpathlineto{\pgfqpoint{1.150288in}{0.654345in}}%
\pgfpathlineto{\pgfqpoint{1.158195in}{0.653905in}}%
\pgfpathlineto{\pgfqpoint{1.166101in}{0.653519in}}%
\pgfpathlineto{\pgfqpoint{1.174008in}{0.653190in}}%
\pgfpathlineto{\pgfqpoint{1.181915in}{0.652920in}}%
\pgfpathlineto{\pgfqpoint{1.189821in}{0.652713in}}%
\pgfpathlineto{\pgfqpoint{1.197728in}{0.652572in}}%
\pgfpathlineto{\pgfqpoint{1.205635in}{0.652500in}}%
\pgfpathlineto{\pgfqpoint{1.213542in}{0.652500in}}%
\pgfpathlineto{\pgfqpoint{1.221448in}{0.652576in}}%
\pgfpathlineto{\pgfqpoint{1.229355in}{0.652729in}}%
\pgfpathlineto{\pgfqpoint{1.237262in}{0.652965in}}%
\pgfpathlineto{\pgfqpoint{1.245169in}{0.653285in}}%
\pgfpathlineto{\pgfqpoint{1.253075in}{0.653692in}}%
\pgfpathlineto{\pgfqpoint{1.260982in}{0.654191in}}%
\pgfpathlineto{\pgfqpoint{1.268889in}{0.654783in}}%
\pgfpathlineto{\pgfqpoint{1.276795in}{0.655473in}}%
\pgfpathlineto{\pgfqpoint{1.284702in}{0.656263in}}%
\pgfpathlineto{\pgfqpoint{1.292609in}{0.657156in}}%
\pgfpathlineto{\pgfqpoint{1.300516in}{0.658155in}}%
\pgfpathlineto{\pgfqpoint{1.308422in}{0.659264in}}%
\pgfpathlineto{\pgfqpoint{1.316329in}{0.660486in}}%
\pgfpathlineto{\pgfqpoint{1.324236in}{0.661823in}}%
\pgfpathlineto{\pgfqpoint{1.332142in}{0.663280in}}%
\pgfpathlineto{\pgfqpoint{1.340049in}{0.664858in}}%
\pgfpathlineto{\pgfqpoint{1.347956in}{0.666561in}}%
\pgfpathlineto{\pgfqpoint{1.355863in}{0.668393in}}%
\pgfpathlineto{\pgfqpoint{1.363769in}{0.670356in}}%
\pgfpathlineto{\pgfqpoint{1.371676in}{0.672453in}}%
\pgfpathlineto{\pgfqpoint{1.379583in}{0.674688in}}%
\pgfpathlineto{\pgfqpoint{1.387490in}{0.677064in}}%
\pgfpathlineto{\pgfqpoint{1.395396in}{0.679584in}}%
\pgfpathlineto{\pgfqpoint{1.403303in}{0.682250in}}%
\pgfpathlineto{\pgfqpoint{1.411210in}{0.685067in}}%
\pgfpathlineto{\pgfqpoint{1.419116in}{0.688036in}}%
\pgfpathlineto{\pgfqpoint{1.427023in}{0.691162in}}%
\pgfpathlineto{\pgfqpoint{1.434930in}{0.694447in}}%
\pgfpathlineto{\pgfqpoint{1.442837in}{0.697895in}}%
\pgfpathlineto{\pgfqpoint{1.450743in}{0.701508in}}%
\pgfpathlineto{\pgfqpoint{1.458650in}{0.705291in}}%
\pgfpathlineto{\pgfqpoint{1.466557in}{0.709245in}}%
\pgfpathlineto{\pgfqpoint{1.474463in}{0.713373in}}%
\pgfpathlineto{\pgfqpoint{1.482370in}{0.717680in}}%
\pgfpathlineto{\pgfqpoint{1.490277in}{0.722169in}}%
\pgfpathlineto{\pgfqpoint{1.498184in}{0.726841in}}%
\pgfpathlineto{\pgfqpoint{1.506090in}{0.731701in}}%
\pgfpathlineto{\pgfqpoint{1.513997in}{0.736751in}}%
\pgfpathlineto{\pgfqpoint{1.521904in}{0.741996in}}%
\pgfpathlineto{\pgfqpoint{1.529811in}{0.747436in}}%
\pgfpathlineto{\pgfqpoint{1.537717in}{0.753077in}}%
\pgfpathlineto{\pgfqpoint{1.545624in}{0.758921in}}%
\pgfpathlineto{\pgfqpoint{1.553531in}{0.764971in}}%
\pgfpathlineto{\pgfqpoint{1.561437in}{0.771230in}}%
\pgfpathlineto{\pgfqpoint{1.569344in}{0.777701in}}%
\pgfpathlineto{\pgfqpoint{1.577251in}{0.784388in}}%
\pgfpathlineto{\pgfqpoint{1.585158in}{0.791293in}}%
\pgfpathlineto{\pgfqpoint{1.593064in}{0.798421in}}%
\pgfpathlineto{\pgfqpoint{1.600971in}{0.805773in}}%
\pgfpathlineto{\pgfqpoint{1.608878in}{0.813352in}}%
\pgfpathlineto{\pgfqpoint{1.616784in}{0.821163in}}%
\pgfpathlineto{\pgfqpoint{1.624691in}{0.829208in}}%
\pgfpathlineto{\pgfqpoint{1.632598in}{0.837491in}}%
\pgfpathlineto{\pgfqpoint{1.640505in}{0.846013in}}%
\pgfpathlineto{\pgfqpoint{1.648411in}{0.854780in}}%
\pgfpathlineto{\pgfqpoint{1.656318in}{0.863792in}}%
\pgfpathlineto{\pgfqpoint{1.664225in}{0.873055in}}%
\pgfpathlineto{\pgfqpoint{1.672132in}{0.882570in}}%
\pgfpathlineto{\pgfqpoint{1.680038in}{0.892342in}}%
\pgfpathlineto{\pgfqpoint{1.687945in}{0.902372in}}%
\pgfpathlineto{\pgfqpoint{1.695852in}{0.912665in}}%
\pgfpathlineto{\pgfqpoint{1.703758in}{0.923222in}}%
\pgfpathlineto{\pgfqpoint{1.711665in}{0.934049in}}%
\pgfpathlineto{\pgfqpoint{1.719572in}{0.945147in}}%
\pgfpathlineto{\pgfqpoint{1.727479in}{0.956519in}}%
\pgfpathlineto{\pgfqpoint{1.735385in}{0.968169in}}%
\pgfpathlineto{\pgfqpoint{1.743292in}{0.980101in}}%
\pgfpathlineto{\pgfqpoint{1.751199in}{0.992316in}}%
\pgfpathlineto{\pgfqpoint{1.759105in}{1.004818in}}%
\pgfpathlineto{\pgfqpoint{1.767012in}{1.017610in}}%
\pgfpathlineto{\pgfqpoint{1.774919in}{1.030696in}}%
\pgfpathlineto{\pgfqpoint{1.782826in}{1.044079in}}%
\pgfpathlineto{\pgfqpoint{1.790732in}{1.057760in}}%
\pgfpathlineto{\pgfqpoint{1.798639in}{1.071745in}}%
\pgfpathlineto{\pgfqpoint{1.806546in}{1.086035in}}%
\pgfpathlineto{\pgfqpoint{1.814453in}{1.100635in}}%
\pgfpathlineto{\pgfqpoint{1.822359in}{1.115546in}}%
\pgfpathlineto{\pgfqpoint{1.830266in}{1.130772in}}%
\pgfpathlineto{\pgfqpoint{1.838173in}{1.146317in}}%
\pgfpathlineto{\pgfqpoint{1.846079in}{1.162183in}}%
\pgfpathlineto{\pgfqpoint{1.853986in}{1.178374in}}%
\pgfpathlineto{\pgfqpoint{1.861893in}{1.194892in}}%
\pgfpathlineto{\pgfqpoint{1.869800in}{1.211741in}}%
\pgfpathlineto{\pgfqpoint{1.877706in}{1.228924in}}%
\pgfpathlineto{\pgfqpoint{1.885613in}{1.246443in}}%
\pgfpathlineto{\pgfqpoint{1.893520in}{1.264299in}}%
\pgfpathlineto{\pgfqpoint{1.901426in}{1.282484in}}%
\pgfpathlineto{\pgfqpoint{1.909333in}{1.300988in}}%
\pgfpathlineto{\pgfqpoint{1.917240in}{1.319804in}}%
\pgfpathlineto{\pgfqpoint{1.925147in}{1.338923in}}%
\pgfpathlineto{\pgfqpoint{1.933053in}{1.358335in}}%
\pgfpathlineto{\pgfqpoint{1.940960in}{1.378034in}}%
\pgfpathlineto{\pgfqpoint{1.948867in}{1.398009in}}%
\pgfpathlineto{\pgfqpoint{1.956774in}{1.418252in}}%
\pgfpathlineto{\pgfqpoint{1.964680in}{1.438755in}}%
\pgfpathlineto{\pgfqpoint{1.972587in}{1.459509in}}%
\pgfpathlineto{\pgfqpoint{1.980494in}{1.480506in}}%
\pgfpathlineto{\pgfqpoint{1.988400in}{1.501737in}}%
\pgfpathlineto{\pgfqpoint{1.996307in}{1.523193in}}%
\pgfpathlineto{\pgfqpoint{2.004214in}{1.544865in}}%
\pgfpathlineto{\pgfqpoint{2.012121in}{1.566746in}}%
\pgfpathlineto{\pgfqpoint{2.020027in}{1.588826in}}%
\pgfpathlineto{\pgfqpoint{2.027934in}{1.611097in}}%
\pgfpathlineto{\pgfqpoint{2.035841in}{1.633550in}}%
\pgfpathlineto{\pgfqpoint{2.043747in}{1.656177in}}%
\pgfpathlineto{\pgfqpoint{2.051654in}{1.678969in}}%
\pgfpathlineto{\pgfqpoint{2.059561in}{1.701918in}}%
\pgfpathlineto{\pgfqpoint{2.067468in}{1.725014in}}%
\pgfpathlineto{\pgfqpoint{2.075374in}{1.748250in}}%
\pgfpathlineto{\pgfqpoint{2.083281in}{1.771616in}}%
\pgfpathlineto{\pgfqpoint{2.091188in}{1.795104in}}%
\pgfpathlineto{\pgfqpoint{2.099095in}{1.818706in}}%
\pgfpathlineto{\pgfqpoint{2.107001in}{1.842412in}}%
\pgfpathlineto{\pgfqpoint{2.114908in}{1.866215in}}%
\pgfpathlineto{\pgfqpoint{2.122815in}{1.890105in}}%
\pgfpathlineto{\pgfqpoint{2.130721in}{1.914075in}}%
\pgfpathlineto{\pgfqpoint{2.138628in}{1.938115in}}%
\pgfpathlineto{\pgfqpoint{2.146535in}{1.962216in}}%
\pgfpathlineto{\pgfqpoint{2.154442in}{1.986371in}}%
\pgfpathlineto{\pgfqpoint{2.162348in}{2.010571in}}%
\pgfpathlineto{\pgfqpoint{2.170255in}{2.034806in}}%
\pgfpathlineto{\pgfqpoint{2.178162in}{2.059069in}}%
\pgfpathlineto{\pgfqpoint{2.186069in}{2.083350in}}%
\pgfpathlineto{\pgfqpoint{2.186069in}{2.088091in}}%
\pgfpathlineto{\pgfqpoint{2.186069in}{2.088091in}}%
\pgfpathlineto{\pgfqpoint{2.178162in}{2.088091in}}%
\pgfpathlineto{\pgfqpoint{2.170255in}{2.088091in}}%
\pgfpathlineto{\pgfqpoint{2.162348in}{2.088091in}}%
\pgfpathlineto{\pgfqpoint{2.154442in}{2.088091in}}%
\pgfpathlineto{\pgfqpoint{2.146535in}{2.088091in}}%
\pgfpathlineto{\pgfqpoint{2.138628in}{2.088091in}}%
\pgfpathlineto{\pgfqpoint{2.130721in}{2.088091in}}%
\pgfpathlineto{\pgfqpoint{2.122815in}{2.088091in}}%
\pgfpathlineto{\pgfqpoint{2.114908in}{2.088091in}}%
\pgfpathlineto{\pgfqpoint{2.107001in}{2.088091in}}%
\pgfpathlineto{\pgfqpoint{2.099095in}{2.088091in}}%
\pgfpathlineto{\pgfqpoint{2.091188in}{2.088091in}}%
\pgfpathlineto{\pgfqpoint{2.083281in}{2.088091in}}%
\pgfpathlineto{\pgfqpoint{2.075374in}{2.088091in}}%
\pgfpathlineto{\pgfqpoint{2.067468in}{2.088091in}}%
\pgfpathlineto{\pgfqpoint{2.059561in}{2.088091in}}%
\pgfpathlineto{\pgfqpoint{2.051654in}{2.088091in}}%
\pgfpathlineto{\pgfqpoint{2.043747in}{2.088091in}}%
\pgfpathlineto{\pgfqpoint{2.035841in}{2.088091in}}%
\pgfpathlineto{\pgfqpoint{2.027934in}{2.088091in}}%
\pgfpathlineto{\pgfqpoint{2.020027in}{2.088091in}}%
\pgfpathlineto{\pgfqpoint{2.012121in}{2.088091in}}%
\pgfpathlineto{\pgfqpoint{2.004214in}{2.088091in}}%
\pgfpathlineto{\pgfqpoint{1.996307in}{2.088091in}}%
\pgfpathlineto{\pgfqpoint{1.988400in}{2.088091in}}%
\pgfpathlineto{\pgfqpoint{1.980494in}{2.088091in}}%
\pgfpathlineto{\pgfqpoint{1.972587in}{2.088091in}}%
\pgfpathlineto{\pgfqpoint{1.964680in}{2.088091in}}%
\pgfpathlineto{\pgfqpoint{1.956774in}{2.088091in}}%
\pgfpathlineto{\pgfqpoint{1.948867in}{2.088091in}}%
\pgfpathlineto{\pgfqpoint{1.940960in}{2.088091in}}%
\pgfpathlineto{\pgfqpoint{1.933053in}{2.088091in}}%
\pgfpathlineto{\pgfqpoint{1.925147in}{2.088091in}}%
\pgfpathlineto{\pgfqpoint{1.917240in}{2.088091in}}%
\pgfpathlineto{\pgfqpoint{1.909333in}{2.088091in}}%
\pgfpathlineto{\pgfqpoint{1.901426in}{2.088091in}}%
\pgfpathlineto{\pgfqpoint{1.893520in}{2.088091in}}%
\pgfpathlineto{\pgfqpoint{1.885613in}{2.088091in}}%
\pgfpathlineto{\pgfqpoint{1.877706in}{2.088091in}}%
\pgfpathlineto{\pgfqpoint{1.869800in}{2.088091in}}%
\pgfpathlineto{\pgfqpoint{1.861893in}{2.088091in}}%
\pgfpathlineto{\pgfqpoint{1.853986in}{2.088091in}}%
\pgfpathlineto{\pgfqpoint{1.846079in}{2.088091in}}%
\pgfpathlineto{\pgfqpoint{1.838173in}{2.088091in}}%
\pgfpathlineto{\pgfqpoint{1.830266in}{2.088091in}}%
\pgfpathlineto{\pgfqpoint{1.822359in}{2.088091in}}%
\pgfpathlineto{\pgfqpoint{1.814453in}{2.088091in}}%
\pgfpathlineto{\pgfqpoint{1.806546in}{2.088091in}}%
\pgfpathlineto{\pgfqpoint{1.798639in}{2.088091in}}%
\pgfpathlineto{\pgfqpoint{1.790732in}{2.088091in}}%
\pgfpathlineto{\pgfqpoint{1.782826in}{2.088091in}}%
\pgfpathlineto{\pgfqpoint{1.774919in}{2.088091in}}%
\pgfpathlineto{\pgfqpoint{1.767012in}{2.088091in}}%
\pgfpathlineto{\pgfqpoint{1.759105in}{2.088091in}}%
\pgfpathlineto{\pgfqpoint{1.751199in}{2.088091in}}%
\pgfpathlineto{\pgfqpoint{1.743292in}{2.088091in}}%
\pgfpathlineto{\pgfqpoint{1.735385in}{2.088091in}}%
\pgfpathlineto{\pgfqpoint{1.727479in}{2.088091in}}%
\pgfpathlineto{\pgfqpoint{1.719572in}{2.088091in}}%
\pgfpathlineto{\pgfqpoint{1.711665in}{2.088091in}}%
\pgfpathlineto{\pgfqpoint{1.703758in}{2.088091in}}%
\pgfpathlineto{\pgfqpoint{1.695852in}{2.088091in}}%
\pgfpathlineto{\pgfqpoint{1.687945in}{2.088091in}}%
\pgfpathlineto{\pgfqpoint{1.680038in}{2.088091in}}%
\pgfpathlineto{\pgfqpoint{1.672132in}{2.088091in}}%
\pgfpathlineto{\pgfqpoint{1.664225in}{2.088091in}}%
\pgfpathlineto{\pgfqpoint{1.656318in}{2.088091in}}%
\pgfpathlineto{\pgfqpoint{1.648411in}{2.088091in}}%
\pgfpathlineto{\pgfqpoint{1.640505in}{2.088091in}}%
\pgfpathlineto{\pgfqpoint{1.632598in}{2.088091in}}%
\pgfpathlineto{\pgfqpoint{1.624691in}{2.088091in}}%
\pgfpathlineto{\pgfqpoint{1.616784in}{2.088091in}}%
\pgfpathlineto{\pgfqpoint{1.608878in}{2.088091in}}%
\pgfpathlineto{\pgfqpoint{1.600971in}{2.088091in}}%
\pgfpathlineto{\pgfqpoint{1.593064in}{2.088091in}}%
\pgfpathlineto{\pgfqpoint{1.585158in}{2.088091in}}%
\pgfpathlineto{\pgfqpoint{1.577251in}{2.088091in}}%
\pgfpathlineto{\pgfqpoint{1.569344in}{2.088091in}}%
\pgfpathlineto{\pgfqpoint{1.561437in}{2.088091in}}%
\pgfpathlineto{\pgfqpoint{1.553531in}{2.088091in}}%
\pgfpathlineto{\pgfqpoint{1.545624in}{2.088091in}}%
\pgfpathlineto{\pgfqpoint{1.537717in}{2.088091in}}%
\pgfpathlineto{\pgfqpoint{1.529811in}{2.088091in}}%
\pgfpathlineto{\pgfqpoint{1.521904in}{2.088091in}}%
\pgfpathlineto{\pgfqpoint{1.513997in}{2.088091in}}%
\pgfpathlineto{\pgfqpoint{1.506090in}{2.088091in}}%
\pgfpathlineto{\pgfqpoint{1.498184in}{2.088091in}}%
\pgfpathlineto{\pgfqpoint{1.490277in}{2.088091in}}%
\pgfpathlineto{\pgfqpoint{1.482370in}{2.088091in}}%
\pgfpathlineto{\pgfqpoint{1.474463in}{2.088091in}}%
\pgfpathlineto{\pgfqpoint{1.466557in}{2.088091in}}%
\pgfpathlineto{\pgfqpoint{1.458650in}{2.088091in}}%
\pgfpathlineto{\pgfqpoint{1.450743in}{2.088091in}}%
\pgfpathlineto{\pgfqpoint{1.442837in}{2.088091in}}%
\pgfpathlineto{\pgfqpoint{1.434930in}{2.088091in}}%
\pgfpathlineto{\pgfqpoint{1.427023in}{2.088091in}}%
\pgfpathlineto{\pgfqpoint{1.419116in}{2.088091in}}%
\pgfpathlineto{\pgfqpoint{1.411210in}{2.088091in}}%
\pgfpathlineto{\pgfqpoint{1.403303in}{2.088091in}}%
\pgfpathlineto{\pgfqpoint{1.395396in}{2.088091in}}%
\pgfpathlineto{\pgfqpoint{1.387490in}{2.088091in}}%
\pgfpathlineto{\pgfqpoint{1.379583in}{2.088091in}}%
\pgfpathlineto{\pgfqpoint{1.371676in}{2.088091in}}%
\pgfpathlineto{\pgfqpoint{1.363769in}{2.088091in}}%
\pgfpathlineto{\pgfqpoint{1.355863in}{2.088091in}}%
\pgfpathlineto{\pgfqpoint{1.347956in}{2.088091in}}%
\pgfpathlineto{\pgfqpoint{1.340049in}{2.088091in}}%
\pgfpathlineto{\pgfqpoint{1.332142in}{2.088091in}}%
\pgfpathlineto{\pgfqpoint{1.324236in}{2.088091in}}%
\pgfpathlineto{\pgfqpoint{1.316329in}{2.088091in}}%
\pgfpathlineto{\pgfqpoint{1.308422in}{2.088091in}}%
\pgfpathlineto{\pgfqpoint{1.300516in}{2.088091in}}%
\pgfpathlineto{\pgfqpoint{1.292609in}{2.088091in}}%
\pgfpathlineto{\pgfqpoint{1.284702in}{2.088091in}}%
\pgfpathlineto{\pgfqpoint{1.276795in}{2.088091in}}%
\pgfpathlineto{\pgfqpoint{1.268889in}{2.088091in}}%
\pgfpathlineto{\pgfqpoint{1.260982in}{2.088091in}}%
\pgfpathlineto{\pgfqpoint{1.253075in}{2.088091in}}%
\pgfpathlineto{\pgfqpoint{1.245169in}{2.088091in}}%
\pgfpathlineto{\pgfqpoint{1.237262in}{2.088091in}}%
\pgfpathlineto{\pgfqpoint{1.229355in}{2.088091in}}%
\pgfpathlineto{\pgfqpoint{1.221448in}{2.088091in}}%
\pgfpathlineto{\pgfqpoint{1.213542in}{2.088091in}}%
\pgfpathlineto{\pgfqpoint{1.205635in}{2.088091in}}%
\pgfpathlineto{\pgfqpoint{1.197728in}{2.088091in}}%
\pgfpathlineto{\pgfqpoint{1.189821in}{2.088091in}}%
\pgfpathlineto{\pgfqpoint{1.181915in}{2.088091in}}%
\pgfpathlineto{\pgfqpoint{1.174008in}{2.088091in}}%
\pgfpathlineto{\pgfqpoint{1.166101in}{2.088091in}}%
\pgfpathlineto{\pgfqpoint{1.158195in}{2.088091in}}%
\pgfpathlineto{\pgfqpoint{1.150288in}{2.088091in}}%
\pgfpathlineto{\pgfqpoint{1.142381in}{2.088091in}}%
\pgfpathlineto{\pgfqpoint{1.134474in}{2.088091in}}%
\pgfpathlineto{\pgfqpoint{1.126568in}{2.088091in}}%
\pgfpathlineto{\pgfqpoint{1.118661in}{2.088091in}}%
\pgfpathlineto{\pgfqpoint{1.110754in}{2.088091in}}%
\pgfpathlineto{\pgfqpoint{1.102848in}{2.088091in}}%
\pgfpathlineto{\pgfqpoint{1.094941in}{2.088091in}}%
\pgfpathlineto{\pgfqpoint{1.087034in}{2.088091in}}%
\pgfpathlineto{\pgfqpoint{1.079127in}{2.088091in}}%
\pgfpathlineto{\pgfqpoint{1.071221in}{2.088091in}}%
\pgfpathlineto{\pgfqpoint{1.063314in}{2.088091in}}%
\pgfpathlineto{\pgfqpoint{1.055407in}{2.088091in}}%
\pgfpathlineto{\pgfqpoint{1.047500in}{2.088091in}}%
\pgfpathlineto{\pgfqpoint{1.039594in}{2.088091in}}%
\pgfpathlineto{\pgfqpoint{1.031687in}{2.088091in}}%
\pgfpathlineto{\pgfqpoint{1.023780in}{2.088091in}}%
\pgfpathlineto{\pgfqpoint{1.015874in}{2.088091in}}%
\pgfpathlineto{\pgfqpoint{1.007967in}{2.088091in}}%
\pgfpathlineto{\pgfqpoint{1.000060in}{2.088091in}}%
\pgfpathlineto{\pgfqpoint{0.992153in}{2.088091in}}%
\pgfpathlineto{\pgfqpoint{0.984247in}{2.088091in}}%
\pgfpathlineto{\pgfqpoint{0.976340in}{2.088091in}}%
\pgfpathlineto{\pgfqpoint{0.968433in}{2.088091in}}%
\pgfpathlineto{\pgfqpoint{0.960527in}{2.088091in}}%
\pgfpathlineto{\pgfqpoint{0.952620in}{2.088091in}}%
\pgfpathlineto{\pgfqpoint{0.944713in}{2.088091in}}%
\pgfpathlineto{\pgfqpoint{0.936806in}{2.088091in}}%
\pgfpathlineto{\pgfqpoint{0.928900in}{2.088091in}}%
\pgfpathlineto{\pgfqpoint{0.920993in}{2.088091in}}%
\pgfpathlineto{\pgfqpoint{0.913086in}{2.088091in}}%
\pgfpathlineto{\pgfqpoint{0.905179in}{2.088091in}}%
\pgfpathlineto{\pgfqpoint{0.897273in}{2.088091in}}%
\pgfpathlineto{\pgfqpoint{0.897273in}{2.088091in}}%
\pgfpathclose%
\pgfusepath{stroke,fill}%
\end{pgfscope}%
\begin{pgfscope}%
\pgfpathrectangle{\pgfqpoint{0.700000in}{0.495000in}}{\pgfqpoint{4.340000in}{3.465000in}}%
\pgfusepath{clip}%
\pgfsetbuttcap%
\pgfsetroundjoin%
\definecolor{currentfill}{rgb}{0.172549,0.627451,0.172549}%
\pgfsetfillcolor{currentfill}%
\pgfsetfillopacity{0.300000}%
\pgfsetlinewidth{1.003750pt}%
\definecolor{currentstroke}{rgb}{0.172549,0.627451,0.172549}%
\pgfsetstrokecolor{currentstroke}%
\pgfsetstrokeopacity{0.300000}%
\pgfsetdash{}{0pt}%
\pgfpathmoveto{\pgfqpoint{4.407858in}{2.088091in}}%
\pgfpathlineto{\pgfqpoint{4.407858in}{2.082164in}}%
\pgfpathlineto{\pgfqpoint{4.415764in}{2.043295in}}%
\pgfpathlineto{\pgfqpoint{4.423671in}{2.004536in}}%
\pgfpathlineto{\pgfqpoint{4.431578in}{1.965908in}}%
\pgfpathlineto{\pgfqpoint{4.439484in}{1.927434in}}%
\pgfpathlineto{\pgfqpoint{4.447391in}{1.889134in}}%
\pgfpathlineto{\pgfqpoint{4.455298in}{1.851031in}}%
\pgfpathlineto{\pgfqpoint{4.463205in}{1.813147in}}%
\pgfpathlineto{\pgfqpoint{4.471111in}{1.775503in}}%
\pgfpathlineto{\pgfqpoint{4.479018in}{1.738120in}}%
\pgfpathlineto{\pgfqpoint{4.486925in}{1.701022in}}%
\pgfpathlineto{\pgfqpoint{4.494831in}{1.664229in}}%
\pgfpathlineto{\pgfqpoint{4.502738in}{1.627763in}}%
\pgfpathlineto{\pgfqpoint{4.510645in}{1.591646in}}%
\pgfpathlineto{\pgfqpoint{4.518552in}{1.555900in}}%
\pgfpathlineto{\pgfqpoint{4.526458in}{1.520547in}}%
\pgfpathlineto{\pgfqpoint{4.534365in}{1.485608in}}%
\pgfpathlineto{\pgfqpoint{4.542272in}{1.451104in}}%
\pgfpathlineto{\pgfqpoint{4.550179in}{1.417059in}}%
\pgfpathlineto{\pgfqpoint{4.558085in}{1.383493in}}%
\pgfpathlineto{\pgfqpoint{4.565992in}{1.350428in}}%
\pgfpathlineto{\pgfqpoint{4.573899in}{1.317887in}}%
\pgfpathlineto{\pgfqpoint{4.581805in}{1.285890in}}%
\pgfpathlineto{\pgfqpoint{4.589712in}{1.254459in}}%
\pgfpathlineto{\pgfqpoint{4.597619in}{1.223617in}}%
\pgfpathlineto{\pgfqpoint{4.605526in}{1.193385in}}%
\pgfpathlineto{\pgfqpoint{4.613432in}{1.163785in}}%
\pgfpathlineto{\pgfqpoint{4.621339in}{1.134838in}}%
\pgfpathlineto{\pgfqpoint{4.629246in}{1.106567in}}%
\pgfpathlineto{\pgfqpoint{4.637152in}{1.078993in}}%
\pgfpathlineto{\pgfqpoint{4.645059in}{1.052137in}}%
\pgfpathlineto{\pgfqpoint{4.652966in}{1.026022in}}%
\pgfpathlineto{\pgfqpoint{4.660873in}{1.000669in}}%
\pgfpathlineto{\pgfqpoint{4.668779in}{0.976101in}}%
\pgfpathlineto{\pgfqpoint{4.676686in}{0.952338in}}%
\pgfpathlineto{\pgfqpoint{4.684593in}{0.929403in}}%
\pgfpathlineto{\pgfqpoint{4.692500in}{0.907317in}}%
\pgfpathlineto{\pgfqpoint{4.700406in}{0.886102in}}%
\pgfpathlineto{\pgfqpoint{4.708313in}{0.865780in}}%
\pgfpathlineto{\pgfqpoint{4.716220in}{0.846373in}}%
\pgfpathlineto{\pgfqpoint{4.724126in}{0.827902in}}%
\pgfpathlineto{\pgfqpoint{4.732033in}{0.810389in}}%
\pgfpathlineto{\pgfqpoint{4.739940in}{0.793856in}}%
\pgfpathlineto{\pgfqpoint{4.747847in}{0.778325in}}%
\pgfpathlineto{\pgfqpoint{4.755753in}{0.763817in}}%
\pgfpathlineto{\pgfqpoint{4.763660in}{0.750354in}}%
\pgfpathlineto{\pgfqpoint{4.771567in}{0.737958in}}%
\pgfpathlineto{\pgfqpoint{4.779473in}{0.726650in}}%
\pgfpathlineto{\pgfqpoint{4.787380in}{0.716453in}}%
\pgfpathlineto{\pgfqpoint{4.795287in}{0.707388in}}%
\pgfpathlineto{\pgfqpoint{4.803194in}{0.699477in}}%
\pgfpathlineto{\pgfqpoint{4.811100in}{0.692742in}}%
\pgfpathlineto{\pgfqpoint{4.819007in}{0.687204in}}%
\pgfpathlineto{\pgfqpoint{4.826914in}{0.682885in}}%
\pgfpathlineto{\pgfqpoint{4.834821in}{0.679807in}}%
\pgfpathlineto{\pgfqpoint{4.842727in}{0.677992in}}%
\pgfpathlineto{\pgfqpoint{4.842727in}{2.088091in}}%
\pgfpathlineto{\pgfqpoint{4.842727in}{2.088091in}}%
\pgfpathlineto{\pgfqpoint{4.834821in}{2.088091in}}%
\pgfpathlineto{\pgfqpoint{4.826914in}{2.088091in}}%
\pgfpathlineto{\pgfqpoint{4.819007in}{2.088091in}}%
\pgfpathlineto{\pgfqpoint{4.811100in}{2.088091in}}%
\pgfpathlineto{\pgfqpoint{4.803194in}{2.088091in}}%
\pgfpathlineto{\pgfqpoint{4.795287in}{2.088091in}}%
\pgfpathlineto{\pgfqpoint{4.787380in}{2.088091in}}%
\pgfpathlineto{\pgfqpoint{4.779473in}{2.088091in}}%
\pgfpathlineto{\pgfqpoint{4.771567in}{2.088091in}}%
\pgfpathlineto{\pgfqpoint{4.763660in}{2.088091in}}%
\pgfpathlineto{\pgfqpoint{4.755753in}{2.088091in}}%
\pgfpathlineto{\pgfqpoint{4.747847in}{2.088091in}}%
\pgfpathlineto{\pgfqpoint{4.739940in}{2.088091in}}%
\pgfpathlineto{\pgfqpoint{4.732033in}{2.088091in}}%
\pgfpathlineto{\pgfqpoint{4.724126in}{2.088091in}}%
\pgfpathlineto{\pgfqpoint{4.716220in}{2.088091in}}%
\pgfpathlineto{\pgfqpoint{4.708313in}{2.088091in}}%
\pgfpathlineto{\pgfqpoint{4.700406in}{2.088091in}}%
\pgfpathlineto{\pgfqpoint{4.692500in}{2.088091in}}%
\pgfpathlineto{\pgfqpoint{4.684593in}{2.088091in}}%
\pgfpathlineto{\pgfqpoint{4.676686in}{2.088091in}}%
\pgfpathlineto{\pgfqpoint{4.668779in}{2.088091in}}%
\pgfpathlineto{\pgfqpoint{4.660873in}{2.088091in}}%
\pgfpathlineto{\pgfqpoint{4.652966in}{2.088091in}}%
\pgfpathlineto{\pgfqpoint{4.645059in}{2.088091in}}%
\pgfpathlineto{\pgfqpoint{4.637152in}{2.088091in}}%
\pgfpathlineto{\pgfqpoint{4.629246in}{2.088091in}}%
\pgfpathlineto{\pgfqpoint{4.621339in}{2.088091in}}%
\pgfpathlineto{\pgfqpoint{4.613432in}{2.088091in}}%
\pgfpathlineto{\pgfqpoint{4.605526in}{2.088091in}}%
\pgfpathlineto{\pgfqpoint{4.597619in}{2.088091in}}%
\pgfpathlineto{\pgfqpoint{4.589712in}{2.088091in}}%
\pgfpathlineto{\pgfqpoint{4.581805in}{2.088091in}}%
\pgfpathlineto{\pgfqpoint{4.573899in}{2.088091in}}%
\pgfpathlineto{\pgfqpoint{4.565992in}{2.088091in}}%
\pgfpathlineto{\pgfqpoint{4.558085in}{2.088091in}}%
\pgfpathlineto{\pgfqpoint{4.550179in}{2.088091in}}%
\pgfpathlineto{\pgfqpoint{4.542272in}{2.088091in}}%
\pgfpathlineto{\pgfqpoint{4.534365in}{2.088091in}}%
\pgfpathlineto{\pgfqpoint{4.526458in}{2.088091in}}%
\pgfpathlineto{\pgfqpoint{4.518552in}{2.088091in}}%
\pgfpathlineto{\pgfqpoint{4.510645in}{2.088091in}}%
\pgfpathlineto{\pgfqpoint{4.502738in}{2.088091in}}%
\pgfpathlineto{\pgfqpoint{4.494831in}{2.088091in}}%
\pgfpathlineto{\pgfqpoint{4.486925in}{2.088091in}}%
\pgfpathlineto{\pgfqpoint{4.479018in}{2.088091in}}%
\pgfpathlineto{\pgfqpoint{4.471111in}{2.088091in}}%
\pgfpathlineto{\pgfqpoint{4.463205in}{2.088091in}}%
\pgfpathlineto{\pgfqpoint{4.455298in}{2.088091in}}%
\pgfpathlineto{\pgfqpoint{4.447391in}{2.088091in}}%
\pgfpathlineto{\pgfqpoint{4.439484in}{2.088091in}}%
\pgfpathlineto{\pgfqpoint{4.431578in}{2.088091in}}%
\pgfpathlineto{\pgfqpoint{4.423671in}{2.088091in}}%
\pgfpathlineto{\pgfqpoint{4.415764in}{2.088091in}}%
\pgfpathlineto{\pgfqpoint{4.407858in}{2.088091in}}%
\pgfpathlineto{\pgfqpoint{4.407858in}{2.088091in}}%
\pgfpathclose%
\pgfusepath{stroke,fill}%
\end{pgfscope}%
\begin{pgfscope}%
\pgfpathrectangle{\pgfqpoint{0.700000in}{0.495000in}}{\pgfqpoint{4.340000in}{3.465000in}}%
\pgfusepath{clip}%
\pgfsetbuttcap%
\pgfsetroundjoin%
\definecolor{currentfill}{rgb}{0.839216,0.152941,0.156863}%
\pgfsetfillcolor{currentfill}%
\pgfsetfillopacity{0.300000}%
\pgfsetlinewidth{1.003750pt}%
\definecolor{currentstroke}{rgb}{0.839216,0.152941,0.156863}%
\pgfsetstrokecolor{currentstroke}%
\pgfsetstrokeopacity{0.300000}%
\pgfsetdash{}{0pt}%
\pgfsys@defobject{currentmarker}{\pgfqpoint{2.193975in}{2.088091in}}{\pgfqpoint{4.399951in}{3.802500in}}{%
\pgfpathmoveto{\pgfqpoint{2.193975in}{2.088091in}}%
\pgfpathlineto{\pgfqpoint{2.193975in}{2.107642in}}%
\pgfpathlineto{\pgfqpoint{2.201882in}{2.131936in}}%
\pgfpathlineto{\pgfqpoint{2.209789in}{2.156222in}}%
\pgfpathlineto{\pgfqpoint{2.217695in}{2.180493in}}%
\pgfpathlineto{\pgfqpoint{2.225602in}{2.204739in}}%
\pgfpathlineto{\pgfqpoint{2.233509in}{2.228953in}}%
\pgfpathlineto{\pgfqpoint{2.241416in}{2.253125in}}%
\pgfpathlineto{\pgfqpoint{2.249322in}{2.277247in}}%
\pgfpathlineto{\pgfqpoint{2.257229in}{2.301311in}}%
\pgfpathlineto{\pgfqpoint{2.265136in}{2.325307in}}%
\pgfpathlineto{\pgfqpoint{2.273042in}{2.349228in}}%
\pgfpathlineto{\pgfqpoint{2.280949in}{2.373064in}}%
\pgfpathlineto{\pgfqpoint{2.288856in}{2.396807in}}%
\pgfpathlineto{\pgfqpoint{2.296763in}{2.420448in}}%
\pgfpathlineto{\pgfqpoint{2.304669in}{2.443979in}}%
\pgfpathlineto{\pgfqpoint{2.312576in}{2.467391in}}%
\pgfpathlineto{\pgfqpoint{2.320483in}{2.490676in}}%
\pgfpathlineto{\pgfqpoint{2.328390in}{2.513824in}}%
\pgfpathlineto{\pgfqpoint{2.336296in}{2.536828in}}%
\pgfpathlineto{\pgfqpoint{2.344203in}{2.559679in}}%
\pgfpathlineto{\pgfqpoint{2.352110in}{2.582367in}}%
\pgfpathlineto{\pgfqpoint{2.360016in}{2.604885in}}%
\pgfpathlineto{\pgfqpoint{2.367923in}{2.627224in}}%
\pgfpathlineto{\pgfqpoint{2.375830in}{2.649376in}}%
\pgfpathlineto{\pgfqpoint{2.383737in}{2.671331in}}%
\pgfpathlineto{\pgfqpoint{2.391643in}{2.693079in}}%
\pgfpathlineto{\pgfqpoint{2.399550in}{2.714610in}}%
\pgfpathlineto{\pgfqpoint{2.407457in}{2.735913in}}%
\pgfpathlineto{\pgfqpoint{2.415363in}{2.756979in}}%
\pgfpathlineto{\pgfqpoint{2.423270in}{2.777795in}}%
\pgfpathlineto{\pgfqpoint{2.431177in}{2.798352in}}%
\pgfpathlineto{\pgfqpoint{2.439084in}{2.818640in}}%
\pgfpathlineto{\pgfqpoint{2.446990in}{2.838647in}}%
\pgfpathlineto{\pgfqpoint{2.454897in}{2.858363in}}%
\pgfpathlineto{\pgfqpoint{2.462804in}{2.877778in}}%
\pgfpathlineto{\pgfqpoint{2.470711in}{2.896880in}}%
\pgfpathlineto{\pgfqpoint{2.478617in}{2.915661in}}%
\pgfpathlineto{\pgfqpoint{2.486524in}{2.934108in}}%
\pgfpathlineto{\pgfqpoint{2.494431in}{2.952211in}}%
\pgfpathlineto{\pgfqpoint{2.502337in}{2.969961in}}%
\pgfpathlineto{\pgfqpoint{2.510244in}{2.987346in}}%
\pgfpathlineto{\pgfqpoint{2.518151in}{3.004356in}}%
\pgfpathlineto{\pgfqpoint{2.526058in}{3.020980in}}%
\pgfpathlineto{\pgfqpoint{2.533964in}{3.037208in}}%
\pgfpathlineto{\pgfqpoint{2.541871in}{3.053029in}}%
\pgfpathlineto{\pgfqpoint{2.549778in}{3.068433in}}%
\pgfpathlineto{\pgfqpoint{2.557684in}{3.083409in}}%
\pgfpathlineto{\pgfqpoint{2.565591in}{3.097946in}}%
\pgfpathlineto{\pgfqpoint{2.573498in}{3.112035in}}%
\pgfpathlineto{\pgfqpoint{2.581405in}{3.125664in}}%
\pgfpathlineto{\pgfqpoint{2.589311in}{3.138823in}}%
\pgfpathlineto{\pgfqpoint{2.597218in}{3.151502in}}%
\pgfpathlineto{\pgfqpoint{2.605125in}{3.163690in}}%
\pgfpathlineto{\pgfqpoint{2.613032in}{3.175376in}}%
\pgfpathlineto{\pgfqpoint{2.620938in}{3.186550in}}%
\pgfpathlineto{\pgfqpoint{2.628845in}{3.197201in}}%
\pgfpathlineto{\pgfqpoint{2.636752in}{3.207319in}}%
\pgfpathlineto{\pgfqpoint{2.644658in}{3.216894in}}%
\pgfpathlineto{\pgfqpoint{2.652565in}{3.225914in}}%
\pgfpathlineto{\pgfqpoint{2.660472in}{3.234369in}}%
\pgfpathlineto{\pgfqpoint{2.668379in}{3.242249in}}%
\pgfpathlineto{\pgfqpoint{2.676285in}{3.249543in}}%
\pgfpathlineto{\pgfqpoint{2.684192in}{3.256241in}}%
\pgfpathlineto{\pgfqpoint{2.692099in}{3.262332in}}%
\pgfpathlineto{\pgfqpoint{2.700005in}{3.267805in}}%
\pgfpathlineto{\pgfqpoint{2.707912in}{3.272650in}}%
\pgfpathlineto{\pgfqpoint{2.715819in}{3.276857in}}%
\pgfpathlineto{\pgfqpoint{2.723726in}{3.280415in}}%
\pgfpathlineto{\pgfqpoint{2.731632in}{3.283313in}}%
\pgfpathlineto{\pgfqpoint{2.739539in}{3.285541in}}%
\pgfpathlineto{\pgfqpoint{2.747446in}{3.287088in}}%
\pgfpathlineto{\pgfqpoint{2.755353in}{3.287944in}}%
\pgfpathlineto{\pgfqpoint{2.763259in}{3.288099in}}%
\pgfpathlineto{\pgfqpoint{2.771166in}{3.287541in}}%
\pgfpathlineto{\pgfqpoint{2.779073in}{3.286260in}}%
\pgfpathlineto{\pgfqpoint{2.786979in}{3.284246in}}%
\pgfpathlineto{\pgfqpoint{2.794886in}{3.281488in}}%
\pgfpathlineto{\pgfqpoint{2.802793in}{3.277975in}}%
\pgfpathlineto{\pgfqpoint{2.810700in}{3.273698in}}%
\pgfpathlineto{\pgfqpoint{2.818606in}{3.268645in}}%
\pgfpathlineto{\pgfqpoint{2.826513in}{3.262806in}}%
\pgfpathlineto{\pgfqpoint{2.834420in}{3.256171in}}%
\pgfpathlineto{\pgfqpoint{2.842326in}{3.248728in}}%
\pgfpathlineto{\pgfqpoint{2.850233in}{3.240468in}}%
\pgfpathlineto{\pgfqpoint{2.858140in}{3.231379in}}%
\pgfpathlineto{\pgfqpoint{2.866047in}{3.221452in}}%
\pgfpathlineto{\pgfqpoint{2.873953in}{3.210677in}}%
\pgfpathlineto{\pgfqpoint{2.881860in}{3.199066in}}%
\pgfpathlineto{\pgfqpoint{2.889767in}{3.186657in}}%
\pgfpathlineto{\pgfqpoint{2.897674in}{3.173484in}}%
\pgfpathlineto{\pgfqpoint{2.905580in}{3.159586in}}%
\pgfpathlineto{\pgfqpoint{2.913487in}{3.144999in}}%
\pgfpathlineto{\pgfqpoint{2.921394in}{3.129760in}}%
\pgfpathlineto{\pgfqpoint{2.929300in}{3.113906in}}%
\pgfpathlineto{\pgfqpoint{2.937207in}{3.097473in}}%
\pgfpathlineto{\pgfqpoint{2.945114in}{3.080498in}}%
\pgfpathlineto{\pgfqpoint{2.953021in}{3.063019in}}%
\pgfpathlineto{\pgfqpoint{2.960927in}{3.045071in}}%
\pgfpathlineto{\pgfqpoint{2.968834in}{3.026692in}}%
\pgfpathlineto{\pgfqpoint{2.976741in}{3.007919in}}%
\pgfpathlineto{\pgfqpoint{2.984647in}{2.988788in}}%
\pgfpathlineto{\pgfqpoint{2.992554in}{2.969336in}}%
\pgfpathlineto{\pgfqpoint{3.000461in}{2.949599in}}%
\pgfpathlineto{\pgfqpoint{3.008368in}{2.929616in}}%
\pgfpathlineto{\pgfqpoint{3.016274in}{2.909422in}}%
\pgfpathlineto{\pgfqpoint{3.024181in}{2.889054in}}%
\pgfpathlineto{\pgfqpoint{3.032088in}{2.868549in}}%
\pgfpathlineto{\pgfqpoint{3.039995in}{2.847944in}}%
\pgfpathlineto{\pgfqpoint{3.047901in}{2.827276in}}%
\pgfpathlineto{\pgfqpoint{3.055808in}{2.806581in}}%
\pgfpathlineto{\pgfqpoint{3.063715in}{2.785897in}}%
\pgfpathlineto{\pgfqpoint{3.071621in}{2.765259in}}%
\pgfpathlineto{\pgfqpoint{3.079528in}{2.744706in}}%
\pgfpathlineto{\pgfqpoint{3.087435in}{2.724273in}}%
\pgfpathlineto{\pgfqpoint{3.095342in}{2.703997in}}%
\pgfpathlineto{\pgfqpoint{3.103248in}{2.683916in}}%
\pgfpathlineto{\pgfqpoint{3.111155in}{2.664066in}}%
\pgfpathlineto{\pgfqpoint{3.119062in}{2.644483in}}%
\pgfpathlineto{\pgfqpoint{3.126968in}{2.625206in}}%
\pgfpathlineto{\pgfqpoint{3.134875in}{2.606270in}}%
\pgfpathlineto{\pgfqpoint{3.142782in}{2.587712in}}%
\pgfpathlineto{\pgfqpoint{3.150689in}{2.569569in}}%
\pgfpathlineto{\pgfqpoint{3.158595in}{2.551878in}}%
\pgfpathlineto{\pgfqpoint{3.166502in}{2.534675in}}%
\pgfpathlineto{\pgfqpoint{3.174409in}{2.517998in}}%
\pgfpathlineto{\pgfqpoint{3.182316in}{2.501883in}}%
\pgfpathlineto{\pgfqpoint{3.190222in}{2.486368in}}%
\pgfpathlineto{\pgfqpoint{3.198129in}{2.471488in}}%
\pgfpathlineto{\pgfqpoint{3.206036in}{2.457281in}}%
\pgfpathlineto{\pgfqpoint{3.213942in}{2.443783in}}%
\pgfpathlineto{\pgfqpoint{3.221849in}{2.431031in}}%
\pgfpathlineto{\pgfqpoint{3.229756in}{2.419063in}}%
\pgfpathlineto{\pgfqpoint{3.237663in}{2.407914in}}%
\pgfpathlineto{\pgfqpoint{3.245569in}{2.397622in}}%
\pgfpathlineto{\pgfqpoint{3.253476in}{2.388224in}}%
\pgfpathlineto{\pgfqpoint{3.261383in}{2.379755in}}%
\pgfpathlineto{\pgfqpoint{3.269289in}{2.372254in}}%
\pgfpathlineto{\pgfqpoint{3.277196in}{2.365757in}}%
\pgfpathlineto{\pgfqpoint{3.285103in}{2.360300in}}%
\pgfpathlineto{\pgfqpoint{3.293010in}{2.355920in}}%
\pgfpathlineto{\pgfqpoint{3.300916in}{2.352655in}}%
\pgfpathlineto{\pgfqpoint{3.308823in}{2.350541in}}%
\pgfpathlineto{\pgfqpoint{3.316730in}{2.349614in}}%
\pgfpathlineto{\pgfqpoint{3.324637in}{2.349913in}}%
\pgfpathlineto{\pgfqpoint{3.332543in}{2.351472in}}%
\pgfpathlineto{\pgfqpoint{3.340450in}{2.354330in}}%
\pgfpathlineto{\pgfqpoint{3.348357in}{2.358523in}}%
\pgfpathlineto{\pgfqpoint{3.356263in}{2.364088in}}%
\pgfpathlineto{\pgfqpoint{3.364170in}{2.371061in}}%
\pgfpathlineto{\pgfqpoint{3.372077in}{2.379460in}}%
\pgfpathlineto{\pgfqpoint{3.379984in}{2.389248in}}%
\pgfpathlineto{\pgfqpoint{3.387890in}{2.400380in}}%
\pgfpathlineto{\pgfqpoint{3.395797in}{2.412808in}}%
\pgfpathlineto{\pgfqpoint{3.403704in}{2.426487in}}%
\pgfpathlineto{\pgfqpoint{3.411610in}{2.441371in}}%
\pgfpathlineto{\pgfqpoint{3.419517in}{2.457414in}}%
\pgfpathlineto{\pgfqpoint{3.427424in}{2.474571in}}%
\pgfpathlineto{\pgfqpoint{3.435331in}{2.492794in}}%
\pgfpathlineto{\pgfqpoint{3.443237in}{2.512038in}}%
\pgfpathlineto{\pgfqpoint{3.451144in}{2.532257in}}%
\pgfpathlineto{\pgfqpoint{3.459051in}{2.553404in}}%
\pgfpathlineto{\pgfqpoint{3.466958in}{2.575435in}}%
\pgfpathlineto{\pgfqpoint{3.474864in}{2.598303in}}%
\pgfpathlineto{\pgfqpoint{3.482771in}{2.621961in}}%
\pgfpathlineto{\pgfqpoint{3.490678in}{2.646364in}}%
\pgfpathlineto{\pgfqpoint{3.498584in}{2.671466in}}%
\pgfpathlineto{\pgfqpoint{3.506491in}{2.697221in}}%
\pgfpathlineto{\pgfqpoint{3.514398in}{2.723583in}}%
\pgfpathlineto{\pgfqpoint{3.522305in}{2.750505in}}%
\pgfpathlineto{\pgfqpoint{3.530211in}{2.777942in}}%
\pgfpathlineto{\pgfqpoint{3.538118in}{2.805848in}}%
\pgfpathlineto{\pgfqpoint{3.546025in}{2.834177in}}%
\pgfpathlineto{\pgfqpoint{3.553931in}{2.862882in}}%
\pgfpathlineto{\pgfqpoint{3.561838in}{2.891918in}}%
\pgfpathlineto{\pgfqpoint{3.569745in}{2.921239in}}%
\pgfpathlineto{\pgfqpoint{3.577652in}{2.950798in}}%
\pgfpathlineto{\pgfqpoint{3.585558in}{2.980550in}}%
\pgfpathlineto{\pgfqpoint{3.593465in}{3.010449in}}%
\pgfpathlineto{\pgfqpoint{3.601372in}{3.040448in}}%
\pgfpathlineto{\pgfqpoint{3.609279in}{3.070502in}}%
\pgfpathlineto{\pgfqpoint{3.617185in}{3.100565in}}%
\pgfpathlineto{\pgfqpoint{3.625092in}{3.130590in}}%
\pgfpathlineto{\pgfqpoint{3.632999in}{3.160532in}}%
\pgfpathlineto{\pgfqpoint{3.640905in}{3.190344in}}%
\pgfpathlineto{\pgfqpoint{3.648812in}{3.219980in}}%
\pgfpathlineto{\pgfqpoint{3.656719in}{3.249396in}}%
\pgfpathlineto{\pgfqpoint{3.664626in}{3.278543in}}%
\pgfpathlineto{\pgfqpoint{3.672532in}{3.307378in}}%
\pgfpathlineto{\pgfqpoint{3.680439in}{3.335852in}}%
\pgfpathlineto{\pgfqpoint{3.688346in}{3.363921in}}%
\pgfpathlineto{\pgfqpoint{3.696253in}{3.391539in}}%
\pgfpathlineto{\pgfqpoint{3.704159in}{3.418659in}}%
\pgfpathlineto{\pgfqpoint{3.712066in}{3.445235in}}%
\pgfpathlineto{\pgfqpoint{3.719973in}{3.471222in}}%
\pgfpathlineto{\pgfqpoint{3.727879in}{3.496573in}}%
\pgfpathlineto{\pgfqpoint{3.735786in}{3.521243in}}%
\pgfpathlineto{\pgfqpoint{3.743693in}{3.545185in}}%
\pgfpathlineto{\pgfqpoint{3.751600in}{3.568353in}}%
\pgfpathlineto{\pgfqpoint{3.759506in}{3.590702in}}%
\pgfpathlineto{\pgfqpoint{3.767413in}{3.612185in}}%
\pgfpathlineto{\pgfqpoint{3.775320in}{3.632756in}}%
\pgfpathlineto{\pgfqpoint{3.783226in}{3.652369in}}%
\pgfpathlineto{\pgfqpoint{3.791133in}{3.670979in}}%
\pgfpathlineto{\pgfqpoint{3.799040in}{3.688539in}}%
\pgfpathlineto{\pgfqpoint{3.806947in}{3.705003in}}%
\pgfpathlineto{\pgfqpoint{3.814853in}{3.720325in}}%
\pgfpathlineto{\pgfqpoint{3.822760in}{3.734460in}}%
\pgfpathlineto{\pgfqpoint{3.830667in}{3.747360in}}%
\pgfpathlineto{\pgfqpoint{3.838574in}{3.758981in}}%
\pgfpathlineto{\pgfqpoint{3.846480in}{3.769276in}}%
\pgfpathlineto{\pgfqpoint{3.854387in}{3.778200in}}%
\pgfpathlineto{\pgfqpoint{3.862294in}{3.785710in}}%
\pgfpathlineto{\pgfqpoint{3.870200in}{3.791807in}}%
\pgfpathlineto{\pgfqpoint{3.878107in}{3.796513in}}%
\pgfpathlineto{\pgfqpoint{3.886014in}{3.799849in}}%
\pgfpathlineto{\pgfqpoint{3.893921in}{3.801838in}}%
\pgfpathlineto{\pgfqpoint{3.901827in}{3.802500in}}%
\pgfpathlineto{\pgfqpoint{3.909734in}{3.801858in}}%
\pgfpathlineto{\pgfqpoint{3.917641in}{3.799934in}}%
\pgfpathlineto{\pgfqpoint{3.925547in}{3.796749in}}%
\pgfpathlineto{\pgfqpoint{3.933454in}{3.792324in}}%
\pgfpathlineto{\pgfqpoint{3.941361in}{3.786683in}}%
\pgfpathlineto{\pgfqpoint{3.949268in}{3.779846in}}%
\pgfpathlineto{\pgfqpoint{3.957174in}{3.771835in}}%
\pgfpathlineto{\pgfqpoint{3.965081in}{3.762672in}}%
\pgfpathlineto{\pgfqpoint{3.972988in}{3.752379in}}%
\pgfpathlineto{\pgfqpoint{3.980895in}{3.740977in}}%
\pgfpathlineto{\pgfqpoint{3.988801in}{3.728489in}}%
\pgfpathlineto{\pgfqpoint{3.996708in}{3.714936in}}%
\pgfpathlineto{\pgfqpoint{4.004615in}{3.700339in}}%
\pgfpathlineto{\pgfqpoint{4.012521in}{3.684721in}}%
\pgfpathlineto{\pgfqpoint{4.020428in}{3.668103in}}%
\pgfpathlineto{\pgfqpoint{4.028335in}{3.650507in}}%
\pgfpathlineto{\pgfqpoint{4.036242in}{3.631955in}}%
\pgfpathlineto{\pgfqpoint{4.044148in}{3.612469in}}%
\pgfpathlineto{\pgfqpoint{4.052055in}{3.592069in}}%
\pgfpathlineto{\pgfqpoint{4.059962in}{3.570779in}}%
\pgfpathlineto{\pgfqpoint{4.067868in}{3.548619in}}%
\pgfpathlineto{\pgfqpoint{4.075775in}{3.525612in}}%
\pgfpathlineto{\pgfqpoint{4.083682in}{3.501780in}}%
\pgfpathlineto{\pgfqpoint{4.091589in}{3.477143in}}%
\pgfpathlineto{\pgfqpoint{4.099495in}{3.451724in}}%
\pgfpathlineto{\pgfqpoint{4.107402in}{3.425545in}}%
\pgfpathlineto{\pgfqpoint{4.115309in}{3.398627in}}%
\pgfpathlineto{\pgfqpoint{4.123216in}{3.370992in}}%
\pgfpathlineto{\pgfqpoint{4.131122in}{3.342662in}}%
\pgfpathlineto{\pgfqpoint{4.139029in}{3.313658in}}%
\pgfpathlineto{\pgfqpoint{4.146936in}{3.284003in}}%
\pgfpathlineto{\pgfqpoint{4.154842in}{3.253718in}}%
\pgfpathlineto{\pgfqpoint{4.162749in}{3.222825in}}%
\pgfpathlineto{\pgfqpoint{4.170656in}{3.191345in}}%
\pgfpathlineto{\pgfqpoint{4.178563in}{3.159300in}}%
\pgfpathlineto{\pgfqpoint{4.186469in}{3.126713in}}%
\pgfpathlineto{\pgfqpoint{4.194376in}{3.093604in}}%
\pgfpathlineto{\pgfqpoint{4.202283in}{3.059996in}}%
\pgfpathlineto{\pgfqpoint{4.210189in}{3.025911in}}%
\pgfpathlineto{\pgfqpoint{4.218096in}{2.991369in}}%
\pgfpathlineto{\pgfqpoint{4.226003in}{2.956394in}}%
\pgfpathlineto{\pgfqpoint{4.233910in}{2.921006in}}%
\pgfpathlineto{\pgfqpoint{4.241816in}{2.885227in}}%
\pgfpathlineto{\pgfqpoint{4.249723in}{2.849080in}}%
\pgfpathlineto{\pgfqpoint{4.257630in}{2.812585in}}%
\pgfpathlineto{\pgfqpoint{4.265537in}{2.775765in}}%
\pgfpathlineto{\pgfqpoint{4.273443in}{2.738641in}}%
\pgfpathlineto{\pgfqpoint{4.281350in}{2.701236in}}%
\pgfpathlineto{\pgfqpoint{4.289257in}{2.663570in}}%
\pgfpathlineto{\pgfqpoint{4.297163in}{2.625666in}}%
\pgfpathlineto{\pgfqpoint{4.305070in}{2.587545in}}%
\pgfpathlineto{\pgfqpoint{4.312977in}{2.549230in}}%
\pgfpathlineto{\pgfqpoint{4.320884in}{2.510741in}}%
\pgfpathlineto{\pgfqpoint{4.328790in}{2.472101in}}%
\pgfpathlineto{\pgfqpoint{4.336697in}{2.433332in}}%
\pgfpathlineto{\pgfqpoint{4.344604in}{2.394455in}}%
\pgfpathlineto{\pgfqpoint{4.352510in}{2.355491in}}%
\pgfpathlineto{\pgfqpoint{4.360417in}{2.316463in}}%
\pgfpathlineto{\pgfqpoint{4.368324in}{2.277393in}}%
\pgfpathlineto{\pgfqpoint{4.376231in}{2.238302in}}%
\pgfpathlineto{\pgfqpoint{4.384137in}{2.199212in}}%
\pgfpathlineto{\pgfqpoint{4.392044in}{2.160144in}}%
\pgfpathlineto{\pgfqpoint{4.399951in}{2.121121in}}%
\pgfpathlineto{\pgfqpoint{4.399951in}{2.088091in}}%
\pgfpathlineto{\pgfqpoint{4.399951in}{2.088091in}}%
\pgfpathlineto{\pgfqpoint{4.392044in}{2.088091in}}%
\pgfpathlineto{\pgfqpoint{4.384137in}{2.088091in}}%
\pgfpathlineto{\pgfqpoint{4.376231in}{2.088091in}}%
\pgfpathlineto{\pgfqpoint{4.368324in}{2.088091in}}%
\pgfpathlineto{\pgfqpoint{4.360417in}{2.088091in}}%
\pgfpathlineto{\pgfqpoint{4.352510in}{2.088091in}}%
\pgfpathlineto{\pgfqpoint{4.344604in}{2.088091in}}%
\pgfpathlineto{\pgfqpoint{4.336697in}{2.088091in}}%
\pgfpathlineto{\pgfqpoint{4.328790in}{2.088091in}}%
\pgfpathlineto{\pgfqpoint{4.320884in}{2.088091in}}%
\pgfpathlineto{\pgfqpoint{4.312977in}{2.088091in}}%
\pgfpathlineto{\pgfqpoint{4.305070in}{2.088091in}}%
\pgfpathlineto{\pgfqpoint{4.297163in}{2.088091in}}%
\pgfpathlineto{\pgfqpoint{4.289257in}{2.088091in}}%
\pgfpathlineto{\pgfqpoint{4.281350in}{2.088091in}}%
\pgfpathlineto{\pgfqpoint{4.273443in}{2.088091in}}%
\pgfpathlineto{\pgfqpoint{4.265537in}{2.088091in}}%
\pgfpathlineto{\pgfqpoint{4.257630in}{2.088091in}}%
\pgfpathlineto{\pgfqpoint{4.249723in}{2.088091in}}%
\pgfpathlineto{\pgfqpoint{4.241816in}{2.088091in}}%
\pgfpathlineto{\pgfqpoint{4.233910in}{2.088091in}}%
\pgfpathlineto{\pgfqpoint{4.226003in}{2.088091in}}%
\pgfpathlineto{\pgfqpoint{4.218096in}{2.088091in}}%
\pgfpathlineto{\pgfqpoint{4.210189in}{2.088091in}}%
\pgfpathlineto{\pgfqpoint{4.202283in}{2.088091in}}%
\pgfpathlineto{\pgfqpoint{4.194376in}{2.088091in}}%
\pgfpathlineto{\pgfqpoint{4.186469in}{2.088091in}}%
\pgfpathlineto{\pgfqpoint{4.178563in}{2.088091in}}%
\pgfpathlineto{\pgfqpoint{4.170656in}{2.088091in}}%
\pgfpathlineto{\pgfqpoint{4.162749in}{2.088091in}}%
\pgfpathlineto{\pgfqpoint{4.154842in}{2.088091in}}%
\pgfpathlineto{\pgfqpoint{4.146936in}{2.088091in}}%
\pgfpathlineto{\pgfqpoint{4.139029in}{2.088091in}}%
\pgfpathlineto{\pgfqpoint{4.131122in}{2.088091in}}%
\pgfpathlineto{\pgfqpoint{4.123216in}{2.088091in}}%
\pgfpathlineto{\pgfqpoint{4.115309in}{2.088091in}}%
\pgfpathlineto{\pgfqpoint{4.107402in}{2.088091in}}%
\pgfpathlineto{\pgfqpoint{4.099495in}{2.088091in}}%
\pgfpathlineto{\pgfqpoint{4.091589in}{2.088091in}}%
\pgfpathlineto{\pgfqpoint{4.083682in}{2.088091in}}%
\pgfpathlineto{\pgfqpoint{4.075775in}{2.088091in}}%
\pgfpathlineto{\pgfqpoint{4.067868in}{2.088091in}}%
\pgfpathlineto{\pgfqpoint{4.059962in}{2.088091in}}%
\pgfpathlineto{\pgfqpoint{4.052055in}{2.088091in}}%
\pgfpathlineto{\pgfqpoint{4.044148in}{2.088091in}}%
\pgfpathlineto{\pgfqpoint{4.036242in}{2.088091in}}%
\pgfpathlineto{\pgfqpoint{4.028335in}{2.088091in}}%
\pgfpathlineto{\pgfqpoint{4.020428in}{2.088091in}}%
\pgfpathlineto{\pgfqpoint{4.012521in}{2.088091in}}%
\pgfpathlineto{\pgfqpoint{4.004615in}{2.088091in}}%
\pgfpathlineto{\pgfqpoint{3.996708in}{2.088091in}}%
\pgfpathlineto{\pgfqpoint{3.988801in}{2.088091in}}%
\pgfpathlineto{\pgfqpoint{3.980895in}{2.088091in}}%
\pgfpathlineto{\pgfqpoint{3.972988in}{2.088091in}}%
\pgfpathlineto{\pgfqpoint{3.965081in}{2.088091in}}%
\pgfpathlineto{\pgfqpoint{3.957174in}{2.088091in}}%
\pgfpathlineto{\pgfqpoint{3.949268in}{2.088091in}}%
\pgfpathlineto{\pgfqpoint{3.941361in}{2.088091in}}%
\pgfpathlineto{\pgfqpoint{3.933454in}{2.088091in}}%
\pgfpathlineto{\pgfqpoint{3.925547in}{2.088091in}}%
\pgfpathlineto{\pgfqpoint{3.917641in}{2.088091in}}%
\pgfpathlineto{\pgfqpoint{3.909734in}{2.088091in}}%
\pgfpathlineto{\pgfqpoint{3.901827in}{2.088091in}}%
\pgfpathlineto{\pgfqpoint{3.893921in}{2.088091in}}%
\pgfpathlineto{\pgfqpoint{3.886014in}{2.088091in}}%
\pgfpathlineto{\pgfqpoint{3.878107in}{2.088091in}}%
\pgfpathlineto{\pgfqpoint{3.870200in}{2.088091in}}%
\pgfpathlineto{\pgfqpoint{3.862294in}{2.088091in}}%
\pgfpathlineto{\pgfqpoint{3.854387in}{2.088091in}}%
\pgfpathlineto{\pgfqpoint{3.846480in}{2.088091in}}%
\pgfpathlineto{\pgfqpoint{3.838574in}{2.088091in}}%
\pgfpathlineto{\pgfqpoint{3.830667in}{2.088091in}}%
\pgfpathlineto{\pgfqpoint{3.822760in}{2.088091in}}%
\pgfpathlineto{\pgfqpoint{3.814853in}{2.088091in}}%
\pgfpathlineto{\pgfqpoint{3.806947in}{2.088091in}}%
\pgfpathlineto{\pgfqpoint{3.799040in}{2.088091in}}%
\pgfpathlineto{\pgfqpoint{3.791133in}{2.088091in}}%
\pgfpathlineto{\pgfqpoint{3.783226in}{2.088091in}}%
\pgfpathlineto{\pgfqpoint{3.775320in}{2.088091in}}%
\pgfpathlineto{\pgfqpoint{3.767413in}{2.088091in}}%
\pgfpathlineto{\pgfqpoint{3.759506in}{2.088091in}}%
\pgfpathlineto{\pgfqpoint{3.751600in}{2.088091in}}%
\pgfpathlineto{\pgfqpoint{3.743693in}{2.088091in}}%
\pgfpathlineto{\pgfqpoint{3.735786in}{2.088091in}}%
\pgfpathlineto{\pgfqpoint{3.727879in}{2.088091in}}%
\pgfpathlineto{\pgfqpoint{3.719973in}{2.088091in}}%
\pgfpathlineto{\pgfqpoint{3.712066in}{2.088091in}}%
\pgfpathlineto{\pgfqpoint{3.704159in}{2.088091in}}%
\pgfpathlineto{\pgfqpoint{3.696253in}{2.088091in}}%
\pgfpathlineto{\pgfqpoint{3.688346in}{2.088091in}}%
\pgfpathlineto{\pgfqpoint{3.680439in}{2.088091in}}%
\pgfpathlineto{\pgfqpoint{3.672532in}{2.088091in}}%
\pgfpathlineto{\pgfqpoint{3.664626in}{2.088091in}}%
\pgfpathlineto{\pgfqpoint{3.656719in}{2.088091in}}%
\pgfpathlineto{\pgfqpoint{3.648812in}{2.088091in}}%
\pgfpathlineto{\pgfqpoint{3.640905in}{2.088091in}}%
\pgfpathlineto{\pgfqpoint{3.632999in}{2.088091in}}%
\pgfpathlineto{\pgfqpoint{3.625092in}{2.088091in}}%
\pgfpathlineto{\pgfqpoint{3.617185in}{2.088091in}}%
\pgfpathlineto{\pgfqpoint{3.609279in}{2.088091in}}%
\pgfpathlineto{\pgfqpoint{3.601372in}{2.088091in}}%
\pgfpathlineto{\pgfqpoint{3.593465in}{2.088091in}}%
\pgfpathlineto{\pgfqpoint{3.585558in}{2.088091in}}%
\pgfpathlineto{\pgfqpoint{3.577652in}{2.088091in}}%
\pgfpathlineto{\pgfqpoint{3.569745in}{2.088091in}}%
\pgfpathlineto{\pgfqpoint{3.561838in}{2.088091in}}%
\pgfpathlineto{\pgfqpoint{3.553931in}{2.088091in}}%
\pgfpathlineto{\pgfqpoint{3.546025in}{2.088091in}}%
\pgfpathlineto{\pgfqpoint{3.538118in}{2.088091in}}%
\pgfpathlineto{\pgfqpoint{3.530211in}{2.088091in}}%
\pgfpathlineto{\pgfqpoint{3.522305in}{2.088091in}}%
\pgfpathlineto{\pgfqpoint{3.514398in}{2.088091in}}%
\pgfpathlineto{\pgfqpoint{3.506491in}{2.088091in}}%
\pgfpathlineto{\pgfqpoint{3.498584in}{2.088091in}}%
\pgfpathlineto{\pgfqpoint{3.490678in}{2.088091in}}%
\pgfpathlineto{\pgfqpoint{3.482771in}{2.088091in}}%
\pgfpathlineto{\pgfqpoint{3.474864in}{2.088091in}}%
\pgfpathlineto{\pgfqpoint{3.466958in}{2.088091in}}%
\pgfpathlineto{\pgfqpoint{3.459051in}{2.088091in}}%
\pgfpathlineto{\pgfqpoint{3.451144in}{2.088091in}}%
\pgfpathlineto{\pgfqpoint{3.443237in}{2.088091in}}%
\pgfpathlineto{\pgfqpoint{3.435331in}{2.088091in}}%
\pgfpathlineto{\pgfqpoint{3.427424in}{2.088091in}}%
\pgfpathlineto{\pgfqpoint{3.419517in}{2.088091in}}%
\pgfpathlineto{\pgfqpoint{3.411610in}{2.088091in}}%
\pgfpathlineto{\pgfqpoint{3.403704in}{2.088091in}}%
\pgfpathlineto{\pgfqpoint{3.395797in}{2.088091in}}%
\pgfpathlineto{\pgfqpoint{3.387890in}{2.088091in}}%
\pgfpathlineto{\pgfqpoint{3.379984in}{2.088091in}}%
\pgfpathlineto{\pgfqpoint{3.372077in}{2.088091in}}%
\pgfpathlineto{\pgfqpoint{3.364170in}{2.088091in}}%
\pgfpathlineto{\pgfqpoint{3.356263in}{2.088091in}}%
\pgfpathlineto{\pgfqpoint{3.348357in}{2.088091in}}%
\pgfpathlineto{\pgfqpoint{3.340450in}{2.088091in}}%
\pgfpathlineto{\pgfqpoint{3.332543in}{2.088091in}}%
\pgfpathlineto{\pgfqpoint{3.324637in}{2.088091in}}%
\pgfpathlineto{\pgfqpoint{3.316730in}{2.088091in}}%
\pgfpathlineto{\pgfqpoint{3.308823in}{2.088091in}}%
\pgfpathlineto{\pgfqpoint{3.300916in}{2.088091in}}%
\pgfpathlineto{\pgfqpoint{3.293010in}{2.088091in}}%
\pgfpathlineto{\pgfqpoint{3.285103in}{2.088091in}}%
\pgfpathlineto{\pgfqpoint{3.277196in}{2.088091in}}%
\pgfpathlineto{\pgfqpoint{3.269289in}{2.088091in}}%
\pgfpathlineto{\pgfqpoint{3.261383in}{2.088091in}}%
\pgfpathlineto{\pgfqpoint{3.253476in}{2.088091in}}%
\pgfpathlineto{\pgfqpoint{3.245569in}{2.088091in}}%
\pgfpathlineto{\pgfqpoint{3.237663in}{2.088091in}}%
\pgfpathlineto{\pgfqpoint{3.229756in}{2.088091in}}%
\pgfpathlineto{\pgfqpoint{3.221849in}{2.088091in}}%
\pgfpathlineto{\pgfqpoint{3.213942in}{2.088091in}}%
\pgfpathlineto{\pgfqpoint{3.206036in}{2.088091in}}%
\pgfpathlineto{\pgfqpoint{3.198129in}{2.088091in}}%
\pgfpathlineto{\pgfqpoint{3.190222in}{2.088091in}}%
\pgfpathlineto{\pgfqpoint{3.182316in}{2.088091in}}%
\pgfpathlineto{\pgfqpoint{3.174409in}{2.088091in}}%
\pgfpathlineto{\pgfqpoint{3.166502in}{2.088091in}}%
\pgfpathlineto{\pgfqpoint{3.158595in}{2.088091in}}%
\pgfpathlineto{\pgfqpoint{3.150689in}{2.088091in}}%
\pgfpathlineto{\pgfqpoint{3.142782in}{2.088091in}}%
\pgfpathlineto{\pgfqpoint{3.134875in}{2.088091in}}%
\pgfpathlineto{\pgfqpoint{3.126968in}{2.088091in}}%
\pgfpathlineto{\pgfqpoint{3.119062in}{2.088091in}}%
\pgfpathlineto{\pgfqpoint{3.111155in}{2.088091in}}%
\pgfpathlineto{\pgfqpoint{3.103248in}{2.088091in}}%
\pgfpathlineto{\pgfqpoint{3.095342in}{2.088091in}}%
\pgfpathlineto{\pgfqpoint{3.087435in}{2.088091in}}%
\pgfpathlineto{\pgfqpoint{3.079528in}{2.088091in}}%
\pgfpathlineto{\pgfqpoint{3.071621in}{2.088091in}}%
\pgfpathlineto{\pgfqpoint{3.063715in}{2.088091in}}%
\pgfpathlineto{\pgfqpoint{3.055808in}{2.088091in}}%
\pgfpathlineto{\pgfqpoint{3.047901in}{2.088091in}}%
\pgfpathlineto{\pgfqpoint{3.039995in}{2.088091in}}%
\pgfpathlineto{\pgfqpoint{3.032088in}{2.088091in}}%
\pgfpathlineto{\pgfqpoint{3.024181in}{2.088091in}}%
\pgfpathlineto{\pgfqpoint{3.016274in}{2.088091in}}%
\pgfpathlineto{\pgfqpoint{3.008368in}{2.088091in}}%
\pgfpathlineto{\pgfqpoint{3.000461in}{2.088091in}}%
\pgfpathlineto{\pgfqpoint{2.992554in}{2.088091in}}%
\pgfpathlineto{\pgfqpoint{2.984647in}{2.088091in}}%
\pgfpathlineto{\pgfqpoint{2.976741in}{2.088091in}}%
\pgfpathlineto{\pgfqpoint{2.968834in}{2.088091in}}%
\pgfpathlineto{\pgfqpoint{2.960927in}{2.088091in}}%
\pgfpathlineto{\pgfqpoint{2.953021in}{2.088091in}}%
\pgfpathlineto{\pgfqpoint{2.945114in}{2.088091in}}%
\pgfpathlineto{\pgfqpoint{2.937207in}{2.088091in}}%
\pgfpathlineto{\pgfqpoint{2.929300in}{2.088091in}}%
\pgfpathlineto{\pgfqpoint{2.921394in}{2.088091in}}%
\pgfpathlineto{\pgfqpoint{2.913487in}{2.088091in}}%
\pgfpathlineto{\pgfqpoint{2.905580in}{2.088091in}}%
\pgfpathlineto{\pgfqpoint{2.897674in}{2.088091in}}%
\pgfpathlineto{\pgfqpoint{2.889767in}{2.088091in}}%
\pgfpathlineto{\pgfqpoint{2.881860in}{2.088091in}}%
\pgfpathlineto{\pgfqpoint{2.873953in}{2.088091in}}%
\pgfpathlineto{\pgfqpoint{2.866047in}{2.088091in}}%
\pgfpathlineto{\pgfqpoint{2.858140in}{2.088091in}}%
\pgfpathlineto{\pgfqpoint{2.850233in}{2.088091in}}%
\pgfpathlineto{\pgfqpoint{2.842326in}{2.088091in}}%
\pgfpathlineto{\pgfqpoint{2.834420in}{2.088091in}}%
\pgfpathlineto{\pgfqpoint{2.826513in}{2.088091in}}%
\pgfpathlineto{\pgfqpoint{2.818606in}{2.088091in}}%
\pgfpathlineto{\pgfqpoint{2.810700in}{2.088091in}}%
\pgfpathlineto{\pgfqpoint{2.802793in}{2.088091in}}%
\pgfpathlineto{\pgfqpoint{2.794886in}{2.088091in}}%
\pgfpathlineto{\pgfqpoint{2.786979in}{2.088091in}}%
\pgfpathlineto{\pgfqpoint{2.779073in}{2.088091in}}%
\pgfpathlineto{\pgfqpoint{2.771166in}{2.088091in}}%
\pgfpathlineto{\pgfqpoint{2.763259in}{2.088091in}}%
\pgfpathlineto{\pgfqpoint{2.755353in}{2.088091in}}%
\pgfpathlineto{\pgfqpoint{2.747446in}{2.088091in}}%
\pgfpathlineto{\pgfqpoint{2.739539in}{2.088091in}}%
\pgfpathlineto{\pgfqpoint{2.731632in}{2.088091in}}%
\pgfpathlineto{\pgfqpoint{2.723726in}{2.088091in}}%
\pgfpathlineto{\pgfqpoint{2.715819in}{2.088091in}}%
\pgfpathlineto{\pgfqpoint{2.707912in}{2.088091in}}%
\pgfpathlineto{\pgfqpoint{2.700005in}{2.088091in}}%
\pgfpathlineto{\pgfqpoint{2.692099in}{2.088091in}}%
\pgfpathlineto{\pgfqpoint{2.684192in}{2.088091in}}%
\pgfpathlineto{\pgfqpoint{2.676285in}{2.088091in}}%
\pgfpathlineto{\pgfqpoint{2.668379in}{2.088091in}}%
\pgfpathlineto{\pgfqpoint{2.660472in}{2.088091in}}%
\pgfpathlineto{\pgfqpoint{2.652565in}{2.088091in}}%
\pgfpathlineto{\pgfqpoint{2.644658in}{2.088091in}}%
\pgfpathlineto{\pgfqpoint{2.636752in}{2.088091in}}%
\pgfpathlineto{\pgfqpoint{2.628845in}{2.088091in}}%
\pgfpathlineto{\pgfqpoint{2.620938in}{2.088091in}}%
\pgfpathlineto{\pgfqpoint{2.613032in}{2.088091in}}%
\pgfpathlineto{\pgfqpoint{2.605125in}{2.088091in}}%
\pgfpathlineto{\pgfqpoint{2.597218in}{2.088091in}}%
\pgfpathlineto{\pgfqpoint{2.589311in}{2.088091in}}%
\pgfpathlineto{\pgfqpoint{2.581405in}{2.088091in}}%
\pgfpathlineto{\pgfqpoint{2.573498in}{2.088091in}}%
\pgfpathlineto{\pgfqpoint{2.565591in}{2.088091in}}%
\pgfpathlineto{\pgfqpoint{2.557684in}{2.088091in}}%
\pgfpathlineto{\pgfqpoint{2.549778in}{2.088091in}}%
\pgfpathlineto{\pgfqpoint{2.541871in}{2.088091in}}%
\pgfpathlineto{\pgfqpoint{2.533964in}{2.088091in}}%
\pgfpathlineto{\pgfqpoint{2.526058in}{2.088091in}}%
\pgfpathlineto{\pgfqpoint{2.518151in}{2.088091in}}%
\pgfpathlineto{\pgfqpoint{2.510244in}{2.088091in}}%
\pgfpathlineto{\pgfqpoint{2.502337in}{2.088091in}}%
\pgfpathlineto{\pgfqpoint{2.494431in}{2.088091in}}%
\pgfpathlineto{\pgfqpoint{2.486524in}{2.088091in}}%
\pgfpathlineto{\pgfqpoint{2.478617in}{2.088091in}}%
\pgfpathlineto{\pgfqpoint{2.470711in}{2.088091in}}%
\pgfpathlineto{\pgfqpoint{2.462804in}{2.088091in}}%
\pgfpathlineto{\pgfqpoint{2.454897in}{2.088091in}}%
\pgfpathlineto{\pgfqpoint{2.446990in}{2.088091in}}%
\pgfpathlineto{\pgfqpoint{2.439084in}{2.088091in}}%
\pgfpathlineto{\pgfqpoint{2.431177in}{2.088091in}}%
\pgfpathlineto{\pgfqpoint{2.423270in}{2.088091in}}%
\pgfpathlineto{\pgfqpoint{2.415363in}{2.088091in}}%
\pgfpathlineto{\pgfqpoint{2.407457in}{2.088091in}}%
\pgfpathlineto{\pgfqpoint{2.399550in}{2.088091in}}%
\pgfpathlineto{\pgfqpoint{2.391643in}{2.088091in}}%
\pgfpathlineto{\pgfqpoint{2.383737in}{2.088091in}}%
\pgfpathlineto{\pgfqpoint{2.375830in}{2.088091in}}%
\pgfpathlineto{\pgfqpoint{2.367923in}{2.088091in}}%
\pgfpathlineto{\pgfqpoint{2.360016in}{2.088091in}}%
\pgfpathlineto{\pgfqpoint{2.352110in}{2.088091in}}%
\pgfpathlineto{\pgfqpoint{2.344203in}{2.088091in}}%
\pgfpathlineto{\pgfqpoint{2.336296in}{2.088091in}}%
\pgfpathlineto{\pgfqpoint{2.328390in}{2.088091in}}%
\pgfpathlineto{\pgfqpoint{2.320483in}{2.088091in}}%
\pgfpathlineto{\pgfqpoint{2.312576in}{2.088091in}}%
\pgfpathlineto{\pgfqpoint{2.304669in}{2.088091in}}%
\pgfpathlineto{\pgfqpoint{2.296763in}{2.088091in}}%
\pgfpathlineto{\pgfqpoint{2.288856in}{2.088091in}}%
\pgfpathlineto{\pgfqpoint{2.280949in}{2.088091in}}%
\pgfpathlineto{\pgfqpoint{2.273042in}{2.088091in}}%
\pgfpathlineto{\pgfqpoint{2.265136in}{2.088091in}}%
\pgfpathlineto{\pgfqpoint{2.257229in}{2.088091in}}%
\pgfpathlineto{\pgfqpoint{2.249322in}{2.088091in}}%
\pgfpathlineto{\pgfqpoint{2.241416in}{2.088091in}}%
\pgfpathlineto{\pgfqpoint{2.233509in}{2.088091in}}%
\pgfpathlineto{\pgfqpoint{2.225602in}{2.088091in}}%
\pgfpathlineto{\pgfqpoint{2.217695in}{2.088091in}}%
\pgfpathlineto{\pgfqpoint{2.209789in}{2.088091in}}%
\pgfpathlineto{\pgfqpoint{2.201882in}{2.088091in}}%
\pgfpathlineto{\pgfqpoint{2.193975in}{2.088091in}}%
\pgfpathlineto{\pgfqpoint{2.193975in}{2.088091in}}%
\pgfpathclose%
\pgfusepath{stroke,fill}%
}%
\begin{pgfscope}%
\pgfsys@transformshift{0.000000in}{0.000000in}%
\pgfsys@useobject{currentmarker}{}%
\end{pgfscope}%
\end{pgfscope}%
\begin{pgfscope}%
\pgfpathrectangle{\pgfqpoint{0.700000in}{0.495000in}}{\pgfqpoint{4.340000in}{3.465000in}}%
\pgfusepath{clip}%
\pgfsetroundcap%
\pgfsetroundjoin%
\pgfsetlinewidth{1.505625pt}%
\definecolor{currentstroke}{rgb}{0.298039,0.447059,0.690196}%
\pgfsetstrokecolor{currentstroke}%
\pgfsetdash{}{0pt}%
\pgfpathmoveto{\pgfqpoint{0.897273in}{0.677992in}}%
\pgfpathlineto{\pgfqpoint{0.928900in}{0.675706in}}%
\pgfpathlineto{\pgfqpoint{0.960527in}{0.672830in}}%
\pgfpathlineto{\pgfqpoint{1.007967in}{0.667845in}}%
\pgfpathlineto{\pgfqpoint{1.087034in}{0.659411in}}%
\pgfpathlineto{\pgfqpoint{1.118661in}{0.656575in}}%
\pgfpathlineto{\pgfqpoint{1.150288in}{0.654345in}}%
\pgfpathlineto{\pgfqpoint{1.174008in}{0.653190in}}%
\pgfpathlineto{\pgfqpoint{1.197728in}{0.652572in}}%
\pgfpathlineto{\pgfqpoint{1.221448in}{0.652576in}}%
\pgfpathlineto{\pgfqpoint{1.245169in}{0.653285in}}%
\pgfpathlineto{\pgfqpoint{1.268889in}{0.654783in}}%
\pgfpathlineto{\pgfqpoint{1.284702in}{0.656263in}}%
\pgfpathlineto{\pgfqpoint{1.300516in}{0.658155in}}%
\pgfpathlineto{\pgfqpoint{1.316329in}{0.660486in}}%
\pgfpathlineto{\pgfqpoint{1.332142in}{0.663280in}}%
\pgfpathlineto{\pgfqpoint{1.347956in}{0.666561in}}%
\pgfpathlineto{\pgfqpoint{1.363769in}{0.670356in}}%
\pgfpathlineto{\pgfqpoint{1.379583in}{0.674688in}}%
\pgfpathlineto{\pgfqpoint{1.395396in}{0.679584in}}%
\pgfpathlineto{\pgfqpoint{1.411210in}{0.685067in}}%
\pgfpathlineto{\pgfqpoint{1.427023in}{0.691162in}}%
\pgfpathlineto{\pgfqpoint{1.442837in}{0.697895in}}%
\pgfpathlineto{\pgfqpoint{1.458650in}{0.705291in}}%
\pgfpathlineto{\pgfqpoint{1.474463in}{0.713373in}}%
\pgfpathlineto{\pgfqpoint{1.490277in}{0.722169in}}%
\pgfpathlineto{\pgfqpoint{1.506090in}{0.731701in}}%
\pgfpathlineto{\pgfqpoint{1.521904in}{0.741996in}}%
\pgfpathlineto{\pgfqpoint{1.537717in}{0.753077in}}%
\pgfpathlineto{\pgfqpoint{1.553531in}{0.764971in}}%
\pgfpathlineto{\pgfqpoint{1.569344in}{0.777701in}}%
\pgfpathlineto{\pgfqpoint{1.585158in}{0.791293in}}%
\pgfpathlineto{\pgfqpoint{1.600971in}{0.805773in}}%
\pgfpathlineto{\pgfqpoint{1.616784in}{0.821163in}}%
\pgfpathlineto{\pgfqpoint{1.632598in}{0.837491in}}%
\pgfpathlineto{\pgfqpoint{1.648411in}{0.854780in}}%
\pgfpathlineto{\pgfqpoint{1.664225in}{0.873055in}}%
\pgfpathlineto{\pgfqpoint{1.680038in}{0.892342in}}%
\pgfpathlineto{\pgfqpoint{1.695852in}{0.912665in}}%
\pgfpathlineto{\pgfqpoint{1.711665in}{0.934049in}}%
\pgfpathlineto{\pgfqpoint{1.727479in}{0.956519in}}%
\pgfpathlineto{\pgfqpoint{1.743292in}{0.980101in}}%
\pgfpathlineto{\pgfqpoint{1.759105in}{1.004818in}}%
\pgfpathlineto{\pgfqpoint{1.774919in}{1.030696in}}%
\pgfpathlineto{\pgfqpoint{1.790732in}{1.057760in}}%
\pgfpathlineto{\pgfqpoint{1.806546in}{1.086035in}}%
\pgfpathlineto{\pgfqpoint{1.822359in}{1.115546in}}%
\pgfpathlineto{\pgfqpoint{1.838173in}{1.146317in}}%
\pgfpathlineto{\pgfqpoint{1.853986in}{1.178374in}}%
\pgfpathlineto{\pgfqpoint{1.869800in}{1.211741in}}%
\pgfpathlineto{\pgfqpoint{1.885613in}{1.246443in}}%
\pgfpathlineto{\pgfqpoint{1.901426in}{1.282484in}}%
\pgfpathlineto{\pgfqpoint{1.917240in}{1.319804in}}%
\pgfpathlineto{\pgfqpoint{1.933053in}{1.358335in}}%
\pgfpathlineto{\pgfqpoint{1.948867in}{1.398009in}}%
\pgfpathlineto{\pgfqpoint{1.964680in}{1.438755in}}%
\pgfpathlineto{\pgfqpoint{1.988400in}{1.501737in}}%
\pgfpathlineto{\pgfqpoint{2.012121in}{1.566746in}}%
\pgfpathlineto{\pgfqpoint{2.035841in}{1.633550in}}%
\pgfpathlineto{\pgfqpoint{2.059561in}{1.701918in}}%
\pgfpathlineto{\pgfqpoint{2.091188in}{1.795104in}}%
\pgfpathlineto{\pgfqpoint{2.122815in}{1.890105in}}%
\pgfpathlineto{\pgfqpoint{2.170255in}{2.034806in}}%
\pgfpathlineto{\pgfqpoint{2.265136in}{2.325307in}}%
\pgfpathlineto{\pgfqpoint{2.296763in}{2.420448in}}%
\pgfpathlineto{\pgfqpoint{2.328390in}{2.513824in}}%
\pgfpathlineto{\pgfqpoint{2.352110in}{2.582367in}}%
\pgfpathlineto{\pgfqpoint{2.375830in}{2.649376in}}%
\pgfpathlineto{\pgfqpoint{2.399550in}{2.714610in}}%
\pgfpathlineto{\pgfqpoint{2.423270in}{2.777795in}}%
\pgfpathlineto{\pgfqpoint{2.439084in}{2.818640in}}%
\pgfpathlineto{\pgfqpoint{2.454897in}{2.858363in}}%
\pgfpathlineto{\pgfqpoint{2.470711in}{2.896880in}}%
\pgfpathlineto{\pgfqpoint{2.486524in}{2.934108in}}%
\pgfpathlineto{\pgfqpoint{2.502337in}{2.969961in}}%
\pgfpathlineto{\pgfqpoint{2.518151in}{3.004356in}}%
\pgfpathlineto{\pgfqpoint{2.533964in}{3.037208in}}%
\pgfpathlineto{\pgfqpoint{2.549778in}{3.068433in}}%
\pgfpathlineto{\pgfqpoint{2.565591in}{3.097946in}}%
\pgfpathlineto{\pgfqpoint{2.581405in}{3.125664in}}%
\pgfpathlineto{\pgfqpoint{2.597218in}{3.151502in}}%
\pgfpathlineto{\pgfqpoint{2.613032in}{3.175376in}}%
\pgfpathlineto{\pgfqpoint{2.620938in}{3.186550in}}%
\pgfpathlineto{\pgfqpoint{2.628845in}{3.197201in}}%
\pgfpathlineto{\pgfqpoint{2.636752in}{3.207319in}}%
\pgfpathlineto{\pgfqpoint{2.644658in}{3.216894in}}%
\pgfpathlineto{\pgfqpoint{2.652565in}{3.225914in}}%
\pgfpathlineto{\pgfqpoint{2.660472in}{3.234369in}}%
\pgfpathlineto{\pgfqpoint{2.668379in}{3.242249in}}%
\pgfpathlineto{\pgfqpoint{2.676285in}{3.249543in}}%
\pgfpathlineto{\pgfqpoint{2.684192in}{3.256241in}}%
\pgfpathlineto{\pgfqpoint{2.692099in}{3.262332in}}%
\pgfpathlineto{\pgfqpoint{2.700005in}{3.267805in}}%
\pgfpathlineto{\pgfqpoint{2.707912in}{3.272650in}}%
\pgfpathlineto{\pgfqpoint{2.715819in}{3.276857in}}%
\pgfpathlineto{\pgfqpoint{2.723726in}{3.280415in}}%
\pgfpathlineto{\pgfqpoint{2.731632in}{3.283313in}}%
\pgfpathlineto{\pgfqpoint{2.739539in}{3.285541in}}%
\pgfpathlineto{\pgfqpoint{2.747446in}{3.287088in}}%
\pgfpathlineto{\pgfqpoint{2.755353in}{3.287944in}}%
\pgfpathlineto{\pgfqpoint{2.763259in}{3.288099in}}%
\pgfpathlineto{\pgfqpoint{2.771166in}{3.287541in}}%
\pgfpathlineto{\pgfqpoint{2.779073in}{3.286260in}}%
\pgfpathlineto{\pgfqpoint{2.786979in}{3.284246in}}%
\pgfpathlineto{\pgfqpoint{2.794886in}{3.281488in}}%
\pgfpathlineto{\pgfqpoint{2.802793in}{3.277975in}}%
\pgfpathlineto{\pgfqpoint{2.810700in}{3.273698in}}%
\pgfpathlineto{\pgfqpoint{2.818606in}{3.268645in}}%
\pgfpathlineto{\pgfqpoint{2.826513in}{3.262806in}}%
\pgfpathlineto{\pgfqpoint{2.834420in}{3.256171in}}%
\pgfpathlineto{\pgfqpoint{2.842326in}{3.248728in}}%
\pgfpathlineto{\pgfqpoint{2.850233in}{3.240468in}}%
\pgfpathlineto{\pgfqpoint{2.858140in}{3.231379in}}%
\pgfpathlineto{\pgfqpoint{2.866047in}{3.221452in}}%
\pgfpathlineto{\pgfqpoint{2.873953in}{3.210677in}}%
\pgfpathlineto{\pgfqpoint{2.881860in}{3.199066in}}%
\pgfpathlineto{\pgfqpoint{2.889767in}{3.186657in}}%
\pgfpathlineto{\pgfqpoint{2.897674in}{3.173484in}}%
\pgfpathlineto{\pgfqpoint{2.905580in}{3.159586in}}%
\pgfpathlineto{\pgfqpoint{2.913487in}{3.144999in}}%
\pgfpathlineto{\pgfqpoint{2.921394in}{3.129760in}}%
\pgfpathlineto{\pgfqpoint{2.937207in}{3.097473in}}%
\pgfpathlineto{\pgfqpoint{2.953021in}{3.063019in}}%
\pgfpathlineto{\pgfqpoint{2.968834in}{3.026692in}}%
\pgfpathlineto{\pgfqpoint{2.984647in}{2.988788in}}%
\pgfpathlineto{\pgfqpoint{3.000461in}{2.949599in}}%
\pgfpathlineto{\pgfqpoint{3.024181in}{2.889054in}}%
\pgfpathlineto{\pgfqpoint{3.071621in}{2.765259in}}%
\pgfpathlineto{\pgfqpoint{3.095342in}{2.703997in}}%
\pgfpathlineto{\pgfqpoint{3.119062in}{2.644483in}}%
\pgfpathlineto{\pgfqpoint{3.134875in}{2.606270in}}%
\pgfpathlineto{\pgfqpoint{3.150689in}{2.569569in}}%
\pgfpathlineto{\pgfqpoint{3.166502in}{2.534675in}}%
\pgfpathlineto{\pgfqpoint{3.182316in}{2.501883in}}%
\pgfpathlineto{\pgfqpoint{3.190222in}{2.486368in}}%
\pgfpathlineto{\pgfqpoint{3.198129in}{2.471488in}}%
\pgfpathlineto{\pgfqpoint{3.206036in}{2.457281in}}%
\pgfpathlineto{\pgfqpoint{3.213942in}{2.443783in}}%
\pgfpathlineto{\pgfqpoint{3.221849in}{2.431031in}}%
\pgfpathlineto{\pgfqpoint{3.229756in}{2.419063in}}%
\pgfpathlineto{\pgfqpoint{3.237663in}{2.407914in}}%
\pgfpathlineto{\pgfqpoint{3.245569in}{2.397622in}}%
\pgfpathlineto{\pgfqpoint{3.253476in}{2.388224in}}%
\pgfpathlineto{\pgfqpoint{3.261383in}{2.379755in}}%
\pgfpathlineto{\pgfqpoint{3.269289in}{2.372254in}}%
\pgfpathlineto{\pgfqpoint{3.277196in}{2.365757in}}%
\pgfpathlineto{\pgfqpoint{3.285103in}{2.360300in}}%
\pgfpathlineto{\pgfqpoint{3.293010in}{2.355920in}}%
\pgfpathlineto{\pgfqpoint{3.300916in}{2.352655in}}%
\pgfpathlineto{\pgfqpoint{3.308823in}{2.350541in}}%
\pgfpathlineto{\pgfqpoint{3.316730in}{2.349614in}}%
\pgfpathlineto{\pgfqpoint{3.324637in}{2.349913in}}%
\pgfpathlineto{\pgfqpoint{3.332543in}{2.351472in}}%
\pgfpathlineto{\pgfqpoint{3.340450in}{2.354330in}}%
\pgfpathlineto{\pgfqpoint{3.348357in}{2.358523in}}%
\pgfpathlineto{\pgfqpoint{3.356263in}{2.364088in}}%
\pgfpathlineto{\pgfqpoint{3.364170in}{2.371061in}}%
\pgfpathlineto{\pgfqpoint{3.372077in}{2.379460in}}%
\pgfpathlineto{\pgfqpoint{3.379984in}{2.389248in}}%
\pgfpathlineto{\pgfqpoint{3.387890in}{2.400380in}}%
\pgfpathlineto{\pgfqpoint{3.395797in}{2.412808in}}%
\pgfpathlineto{\pgfqpoint{3.403704in}{2.426487in}}%
\pgfpathlineto{\pgfqpoint{3.411610in}{2.441371in}}%
\pgfpathlineto{\pgfqpoint{3.419517in}{2.457414in}}%
\pgfpathlineto{\pgfqpoint{3.427424in}{2.474571in}}%
\pgfpathlineto{\pgfqpoint{3.435331in}{2.492794in}}%
\pgfpathlineto{\pgfqpoint{3.443237in}{2.512038in}}%
\pgfpathlineto{\pgfqpoint{3.451144in}{2.532257in}}%
\pgfpathlineto{\pgfqpoint{3.459051in}{2.553404in}}%
\pgfpathlineto{\pgfqpoint{3.466958in}{2.575435in}}%
\pgfpathlineto{\pgfqpoint{3.482771in}{2.621961in}}%
\pgfpathlineto{\pgfqpoint{3.498584in}{2.671466in}}%
\pgfpathlineto{\pgfqpoint{3.514398in}{2.723583in}}%
\pgfpathlineto{\pgfqpoint{3.530211in}{2.777942in}}%
\pgfpathlineto{\pgfqpoint{3.546025in}{2.834177in}}%
\pgfpathlineto{\pgfqpoint{3.569745in}{2.921239in}}%
\pgfpathlineto{\pgfqpoint{3.601372in}{3.040448in}}%
\pgfpathlineto{\pgfqpoint{3.648812in}{3.219980in}}%
\pgfpathlineto{\pgfqpoint{3.672532in}{3.307378in}}%
\pgfpathlineto{\pgfqpoint{3.688346in}{3.363921in}}%
\pgfpathlineto{\pgfqpoint{3.704159in}{3.418659in}}%
\pgfpathlineto{\pgfqpoint{3.719973in}{3.471222in}}%
\pgfpathlineto{\pgfqpoint{3.735786in}{3.521243in}}%
\pgfpathlineto{\pgfqpoint{3.751600in}{3.568353in}}%
\pgfpathlineto{\pgfqpoint{3.759506in}{3.590702in}}%
\pgfpathlineto{\pgfqpoint{3.767413in}{3.612185in}}%
\pgfpathlineto{\pgfqpoint{3.775320in}{3.632756in}}%
\pgfpathlineto{\pgfqpoint{3.783226in}{3.652369in}}%
\pgfpathlineto{\pgfqpoint{3.791133in}{3.670979in}}%
\pgfpathlineto{\pgfqpoint{3.799040in}{3.688539in}}%
\pgfpathlineto{\pgfqpoint{3.806947in}{3.705003in}}%
\pgfpathlineto{\pgfqpoint{3.814853in}{3.720325in}}%
\pgfpathlineto{\pgfqpoint{3.822760in}{3.734460in}}%
\pgfpathlineto{\pgfqpoint{3.830667in}{3.747360in}}%
\pgfpathlineto{\pgfqpoint{3.838574in}{3.758981in}}%
\pgfpathlineto{\pgfqpoint{3.846480in}{3.769276in}}%
\pgfpathlineto{\pgfqpoint{3.854387in}{3.778200in}}%
\pgfpathlineto{\pgfqpoint{3.862294in}{3.785710in}}%
\pgfpathlineto{\pgfqpoint{3.870200in}{3.791807in}}%
\pgfpathlineto{\pgfqpoint{3.878107in}{3.796513in}}%
\pgfpathlineto{\pgfqpoint{3.886014in}{3.799849in}}%
\pgfpathlineto{\pgfqpoint{3.893921in}{3.801838in}}%
\pgfpathlineto{\pgfqpoint{3.901827in}{3.802500in}}%
\pgfpathlineto{\pgfqpoint{3.909734in}{3.801858in}}%
\pgfpathlineto{\pgfqpoint{3.917641in}{3.799934in}}%
\pgfpathlineto{\pgfqpoint{3.925547in}{3.796749in}}%
\pgfpathlineto{\pgfqpoint{3.933454in}{3.792324in}}%
\pgfpathlineto{\pgfqpoint{3.941361in}{3.786683in}}%
\pgfpathlineto{\pgfqpoint{3.949268in}{3.779846in}}%
\pgfpathlineto{\pgfqpoint{3.957174in}{3.771835in}}%
\pgfpathlineto{\pgfqpoint{3.965081in}{3.762672in}}%
\pgfpathlineto{\pgfqpoint{3.972988in}{3.752379in}}%
\pgfpathlineto{\pgfqpoint{3.980895in}{3.740977in}}%
\pgfpathlineto{\pgfqpoint{3.988801in}{3.728489in}}%
\pgfpathlineto{\pgfqpoint{3.996708in}{3.714936in}}%
\pgfpathlineto{\pgfqpoint{4.004615in}{3.700339in}}%
\pgfpathlineto{\pgfqpoint{4.012521in}{3.684721in}}%
\pgfpathlineto{\pgfqpoint{4.020428in}{3.668103in}}%
\pgfpathlineto{\pgfqpoint{4.028335in}{3.650507in}}%
\pgfpathlineto{\pgfqpoint{4.036242in}{3.631955in}}%
\pgfpathlineto{\pgfqpoint{4.044148in}{3.612469in}}%
\pgfpathlineto{\pgfqpoint{4.052055in}{3.592069in}}%
\pgfpathlineto{\pgfqpoint{4.059962in}{3.570779in}}%
\pgfpathlineto{\pgfqpoint{4.067868in}{3.548619in}}%
\pgfpathlineto{\pgfqpoint{4.083682in}{3.501780in}}%
\pgfpathlineto{\pgfqpoint{4.099495in}{3.451724in}}%
\pgfpathlineto{\pgfqpoint{4.115309in}{3.398627in}}%
\pgfpathlineto{\pgfqpoint{4.131122in}{3.342662in}}%
\pgfpathlineto{\pgfqpoint{4.146936in}{3.284003in}}%
\pgfpathlineto{\pgfqpoint{4.162749in}{3.222825in}}%
\pgfpathlineto{\pgfqpoint{4.178563in}{3.159300in}}%
\pgfpathlineto{\pgfqpoint{4.194376in}{3.093604in}}%
\pgfpathlineto{\pgfqpoint{4.210189in}{3.025911in}}%
\pgfpathlineto{\pgfqpoint{4.226003in}{2.956394in}}%
\pgfpathlineto{\pgfqpoint{4.249723in}{2.849080in}}%
\pgfpathlineto{\pgfqpoint{4.273443in}{2.738641in}}%
\pgfpathlineto{\pgfqpoint{4.297163in}{2.625666in}}%
\pgfpathlineto{\pgfqpoint{4.328790in}{2.472101in}}%
\pgfpathlineto{\pgfqpoint{4.376231in}{2.238302in}}%
\pgfpathlineto{\pgfqpoint{4.431578in}{1.965908in}}%
\pgfpathlineto{\pgfqpoint{4.463205in}{1.813147in}}%
\pgfpathlineto{\pgfqpoint{4.486925in}{1.701022in}}%
\pgfpathlineto{\pgfqpoint{4.510645in}{1.591646in}}%
\pgfpathlineto{\pgfqpoint{4.534365in}{1.485608in}}%
\pgfpathlineto{\pgfqpoint{4.550179in}{1.417059in}}%
\pgfpathlineto{\pgfqpoint{4.565992in}{1.350428in}}%
\pgfpathlineto{\pgfqpoint{4.581805in}{1.285890in}}%
\pgfpathlineto{\pgfqpoint{4.597619in}{1.223617in}}%
\pgfpathlineto{\pgfqpoint{4.613432in}{1.163785in}}%
\pgfpathlineto{\pgfqpoint{4.629246in}{1.106567in}}%
\pgfpathlineto{\pgfqpoint{4.645059in}{1.052137in}}%
\pgfpathlineto{\pgfqpoint{4.660873in}{1.000669in}}%
\pgfpathlineto{\pgfqpoint{4.676686in}{0.952338in}}%
\pgfpathlineto{\pgfqpoint{4.692500in}{0.907317in}}%
\pgfpathlineto{\pgfqpoint{4.700406in}{0.886102in}}%
\pgfpathlineto{\pgfqpoint{4.708313in}{0.865780in}}%
\pgfpathlineto{\pgfqpoint{4.716220in}{0.846373in}}%
\pgfpathlineto{\pgfqpoint{4.724126in}{0.827902in}}%
\pgfpathlineto{\pgfqpoint{4.732033in}{0.810389in}}%
\pgfpathlineto{\pgfqpoint{4.739940in}{0.793856in}}%
\pgfpathlineto{\pgfqpoint{4.747847in}{0.778325in}}%
\pgfpathlineto{\pgfqpoint{4.755753in}{0.763817in}}%
\pgfpathlineto{\pgfqpoint{4.763660in}{0.750354in}}%
\pgfpathlineto{\pgfqpoint{4.771567in}{0.737958in}}%
\pgfpathlineto{\pgfqpoint{4.779473in}{0.726650in}}%
\pgfpathlineto{\pgfqpoint{4.787380in}{0.716453in}}%
\pgfpathlineto{\pgfqpoint{4.795287in}{0.707388in}}%
\pgfpathlineto{\pgfqpoint{4.803194in}{0.699477in}}%
\pgfpathlineto{\pgfqpoint{4.811100in}{0.692742in}}%
\pgfpathlineto{\pgfqpoint{4.819007in}{0.687204in}}%
\pgfpathlineto{\pgfqpoint{4.826914in}{0.682885in}}%
\pgfpathlineto{\pgfqpoint{4.834821in}{0.679807in}}%
\pgfpathlineto{\pgfqpoint{4.842727in}{0.677992in}}%
\pgfpathlineto{\pgfqpoint{4.842727in}{0.677992in}}%
\pgfusepath{stroke}%
\end{pgfscope}%
\begin{pgfscope}%
\pgfpathrectangle{\pgfqpoint{0.700000in}{0.495000in}}{\pgfqpoint{4.340000in}{3.465000in}}%
\pgfusepath{clip}%
\pgfsetroundcap%
\pgfsetroundjoin%
\pgfsetlinewidth{2.007500pt}%
\definecolor{currentstroke}{rgb}{1.000000,0.647059,0.000000}%
\pgfsetstrokecolor{currentstroke}%
\pgfsetdash{}{0pt}%
\pgfpathmoveto{\pgfqpoint{0.700000in}{2.088091in}}%
\pgfpathlineto{\pgfqpoint{5.040000in}{2.088091in}}%
\pgfusepath{stroke}%
\end{pgfscope}%
\begin{pgfscope}%
\pgfsetrectcap%
\pgfsetmiterjoin%
\pgfsetlinewidth{1.254687pt}%
\definecolor{currentstroke}{rgb}{1.000000,1.000000,1.000000}%
\pgfsetstrokecolor{currentstroke}%
\pgfsetdash{}{0pt}%
\pgfpathmoveto{\pgfqpoint{0.700000in}{0.495000in}}%
\pgfpathlineto{\pgfqpoint{0.700000in}{3.960000in}}%
\pgfusepath{stroke}%
\end{pgfscope}%
\begin{pgfscope}%
\pgfsetrectcap%
\pgfsetmiterjoin%
\pgfsetlinewidth{1.254687pt}%
\definecolor{currentstroke}{rgb}{1.000000,1.000000,1.000000}%
\pgfsetstrokecolor{currentstroke}%
\pgfsetdash{}{0pt}%
\pgfpathmoveto{\pgfqpoint{5.040000in}{0.495000in}}%
\pgfpathlineto{\pgfqpoint{5.040000in}{3.960000in}}%
\pgfusepath{stroke}%
\end{pgfscope}%
\begin{pgfscope}%
\pgfsetrectcap%
\pgfsetmiterjoin%
\pgfsetlinewidth{1.254687pt}%
\definecolor{currentstroke}{rgb}{1.000000,1.000000,1.000000}%
\pgfsetstrokecolor{currentstroke}%
\pgfsetdash{}{0pt}%
\pgfpathmoveto{\pgfqpoint{0.700000in}{0.495000in}}%
\pgfpathlineto{\pgfqpoint{5.040000in}{0.495000in}}%
\pgfusepath{stroke}%
\end{pgfscope}%
\begin{pgfscope}%
\pgfsetrectcap%
\pgfsetmiterjoin%
\pgfsetlinewidth{1.254687pt}%
\definecolor{currentstroke}{rgb}{1.000000,1.000000,1.000000}%
\pgfsetstrokecolor{currentstroke}%
\pgfsetdash{}{0pt}%
\pgfpathmoveto{\pgfqpoint{0.700000in}{3.960000in}}%
\pgfpathlineto{\pgfqpoint{5.040000in}{3.960000in}}%
\pgfusepath{stroke}%
\end{pgfscope}%
\begin{pgfscope}%
\pgfsetbuttcap%
\pgfsetmiterjoin%
\definecolor{currentfill}{rgb}{0.917647,0.917647,0.949020}%
\pgfsetfillcolor{currentfill}%
\pgfsetfillopacity{0.800000}%
\pgfsetlinewidth{1.003750pt}%
\definecolor{currentstroke}{rgb}{0.800000,0.800000,0.800000}%
\pgfsetstrokecolor{currentstroke}%
\pgfsetstrokeopacity{0.800000}%
\pgfsetdash{}{0pt}%
\pgfpathmoveto{\pgfqpoint{0.806944in}{2.967960in}}%
\pgfpathlineto{\pgfqpoint{2.501868in}{2.967960in}}%
\pgfpathquadraticcurveto{\pgfqpoint{2.532423in}{2.967960in}}{\pgfqpoint{2.532423in}{2.998515in}}%
\pgfpathlineto{\pgfqpoint{2.532423in}{3.853056in}}%
\pgfpathquadraticcurveto{\pgfqpoint{2.532423in}{3.883611in}}{\pgfqpoint{2.501868in}{3.883611in}}%
\pgfpathlineto{\pgfqpoint{0.806944in}{3.883611in}}%
\pgfpathquadraticcurveto{\pgfqpoint{0.776389in}{3.883611in}}{\pgfqpoint{0.776389in}{3.853056in}}%
\pgfpathlineto{\pgfqpoint{0.776389in}{2.998515in}}%
\pgfpathquadraticcurveto{\pgfqpoint{0.776389in}{2.967960in}}{\pgfqpoint{0.806944in}{2.967960in}}%
\pgfpathlineto{\pgfqpoint{0.806944in}{2.967960in}}%
\pgfpathclose%
\pgfusepath{stroke,fill}%
\end{pgfscope}%
\begin{pgfscope}%
\pgfsetroundcap%
\pgfsetroundjoin%
\pgfsetlinewidth{1.505625pt}%
\definecolor{currentstroke}{rgb}{0.298039,0.447059,0.690196}%
\pgfsetstrokecolor{currentstroke}%
\pgfsetdash{}{0pt}%
\pgfpathmoveto{\pgfqpoint{0.837500in}{3.766611in}}%
\pgfpathlineto{\pgfqpoint{0.990278in}{3.766611in}}%
\pgfpathlineto{\pgfqpoint{1.143056in}{3.766611in}}%
\pgfusepath{stroke}%
\end{pgfscope}%
\begin{pgfscope}%
\definecolor{textcolor}{rgb}{0.150000,0.150000,0.150000}%
\pgfsetstrokecolor{textcolor}%
\pgfsetfillcolor{textcolor}%
\pgftext[x=1.265278in,y=3.713139in,left,base]{\color{textcolor}{\sffamily\fontsize{11.000000}{13.200000}\selectfont\catcode`\^=\active\def^{\ifmmode\sp\else\^{}\fi}\catcode`\%=\active\def%{\%}resource demand}}%
\end{pgfscope}%
\begin{pgfscope}%
\pgfsetroundcap%
\pgfsetroundjoin%
\pgfsetlinewidth{2.007500pt}%
\definecolor{currentstroke}{rgb}{1.000000,0.647059,0.000000}%
\pgfsetstrokecolor{currentstroke}%
\pgfsetdash{}{0pt}%
\pgfpathmoveto{\pgfqpoint{0.837500in}{3.550499in}}%
\pgfpathlineto{\pgfqpoint{0.990278in}{3.550499in}}%
\pgfpathlineto{\pgfqpoint{1.143056in}{3.550499in}}%
\pgfusepath{stroke}%
\end{pgfscope}%
\begin{pgfscope}%
\definecolor{textcolor}{rgb}{0.150000,0.150000,0.150000}%
\pgfsetstrokecolor{textcolor}%
\pgfsetfillcolor{textcolor}%
\pgftext[x=1.265278in,y=3.497027in,left,base]{\color{textcolor}{\sffamily\fontsize{11.000000}{13.200000}\selectfont\catcode`\^=\active\def^{\ifmmode\sp\else\^{}\fi}\catcode`\%=\active\def%{\%}resource supply}}%
\end{pgfscope}%
\begin{pgfscope}%
\pgfsetbuttcap%
\pgfsetmiterjoin%
\definecolor{currentfill}{rgb}{0.172549,0.627451,0.172549}%
\pgfsetfillcolor{currentfill}%
\pgfsetfillopacity{0.300000}%
\pgfsetlinewidth{1.003750pt}%
\definecolor{currentstroke}{rgb}{0.172549,0.627451,0.172549}%
\pgfsetstrokecolor{currentstroke}%
\pgfsetstrokeopacity{0.300000}%
\pgfsetdash{}{0pt}%
\pgfpathmoveto{\pgfqpoint{0.837500in}{3.279125in}}%
\pgfpathlineto{\pgfqpoint{1.143056in}{3.279125in}}%
\pgfpathlineto{\pgfqpoint{1.143056in}{3.386069in}}%
\pgfpathlineto{\pgfqpoint{0.837500in}{3.386069in}}%
\pgfpathlineto{\pgfqpoint{0.837500in}{3.279125in}}%
\pgfpathclose%
\pgfusepath{stroke,fill}%
\end{pgfscope}%
\begin{pgfscope}%
\definecolor{textcolor}{rgb}{0.150000,0.150000,0.150000}%
\pgfsetstrokecolor{textcolor}%
\pgfsetfillcolor{textcolor}%
\pgftext[x=1.265278in,y=3.279125in,left,base]{\color{textcolor}{\sffamily\fontsize{11.000000}{13.200000}\selectfont\catcode`\^=\active\def^{\ifmmode\sp\else\^{}\fi}\catcode`\%=\active\def%{\%}overprovisioning}}%
\end{pgfscope}%
\begin{pgfscope}%
\pgfsetbuttcap%
\pgfsetmiterjoin%
\definecolor{currentfill}{rgb}{0.839216,0.152941,0.156863}%
\pgfsetfillcolor{currentfill}%
\pgfsetfillopacity{0.300000}%
\pgfsetlinewidth{1.003750pt}%
\definecolor{currentstroke}{rgb}{0.839216,0.152941,0.156863}%
\pgfsetstrokecolor{currentstroke}%
\pgfsetstrokeopacity{0.300000}%
\pgfsetdash{}{0pt}%
\pgfpathmoveto{\pgfqpoint{0.837500in}{3.061223in}}%
\pgfpathlineto{\pgfqpoint{1.143056in}{3.061223in}}%
\pgfpathlineto{\pgfqpoint{1.143056in}{3.168167in}}%
\pgfpathlineto{\pgfqpoint{0.837500in}{3.168167in}}%
\pgfpathlineto{\pgfqpoint{0.837500in}{3.061223in}}%
\pgfpathclose%
\pgfusepath{stroke,fill}%
\end{pgfscope}%
\begin{pgfscope}%
\definecolor{textcolor}{rgb}{0.150000,0.150000,0.150000}%
\pgfsetstrokecolor{textcolor}%
\pgfsetfillcolor{textcolor}%
\pgftext[x=1.265278in,y=3.061223in,left,base]{\color{textcolor}{\sffamily\fontsize{11.000000}{13.200000}\selectfont\catcode`\^=\active\def^{\ifmmode\sp\else\^{}\fi}\catcode`\%=\active\def%{\%}underprovisioning}}%
\end{pgfscope}%
\end{pgfpicture}%
\makeatother%
\endgroup%

    \caption{Resource demand and supply for a website during a typical day.}
    \label{fig:elasticity-application-no-scaling}
\end{figure}

If the concept of elasticity is applied to this example, resources can be released during the night (so called \textit{scale-in}) and more resources can be claimed as they are needed during the day (so called \textit{scale-out}). This is illustrated in \cref{fig:elasticity-application-scaling}.

\begin{figure}
    \centering
    %% Creator: Matplotlib, PGF backend
%%
%% To include the figure in your LaTeX document, write
%%   \input{<filename>.pgf}
%%
%% Make sure the required packages are loaded in your preamble
%%   \usepackage{pgf}
%%
%% Also ensure that all the required font packages are loaded; for instance,
%% the lmodern package is sometimes necessary when using math font.
%%   \usepackage{lmodern}
%%
%% Figures using additional raster images can only be included by \input if
%% they are in the same directory as the main LaTeX file. For loading figures
%% from other directories you can use the `import` package
%%   \usepackage{import}
%%
%% and then include the figures with
%%   \import{<path to file>}{<filename>.pgf}
%%
%% Matplotlib used the following preamble
%%   \def\mathdefault#1{#1}
%%   \everymath=\expandafter{\the\everymath\displaystyle}
%%   
%%   \usepackage{fontspec}
%%   \setmainfont{DejaVuSerif.ttf}[Path=\detokenize{/Users/nkratky/private/polaris-elasticity-strategies/test/scripts/.venv/lib/python3.11/site-packages/matplotlib/mpl-data/fonts/ttf/}]
%%   \setsansfont{Arial.ttf}[Path=\detokenize{/System/Library/Fonts/Supplemental/}]
%%   \setmonofont{DejaVuSansMono.ttf}[Path=\detokenize{/Users/nkratky/private/polaris-elasticity-strategies/test/scripts/.venv/lib/python3.11/site-packages/matplotlib/mpl-data/fonts/ttf/}]
%%   \makeatletter\@ifpackageloaded{underscore}{}{\usepackage[strings]{underscore}}\makeatother
%%
\begingroup%
\makeatletter%
\begin{pgfpicture}%
\pgfpathrectangle{\pgfpointorigin}{\pgfqpoint{6.400000in}{4.800000in}}%
\pgfusepath{use as bounding box, clip}%
\begin{pgfscope}%
\pgfsetbuttcap%
\pgfsetmiterjoin%
\definecolor{currentfill}{rgb}{1.000000,1.000000,1.000000}%
\pgfsetfillcolor{currentfill}%
\pgfsetlinewidth{0.000000pt}%
\definecolor{currentstroke}{rgb}{1.000000,1.000000,1.000000}%
\pgfsetstrokecolor{currentstroke}%
\pgfsetdash{}{0pt}%
\pgfpathmoveto{\pgfqpoint{0.000000in}{0.000000in}}%
\pgfpathlineto{\pgfqpoint{6.400000in}{0.000000in}}%
\pgfpathlineto{\pgfqpoint{6.400000in}{4.800000in}}%
\pgfpathlineto{\pgfqpoint{0.000000in}{4.800000in}}%
\pgfpathlineto{\pgfqpoint{0.000000in}{0.000000in}}%
\pgfpathclose%
\pgfusepath{fill}%
\end{pgfscope}%
\begin{pgfscope}%
\pgfsetbuttcap%
\pgfsetmiterjoin%
\definecolor{currentfill}{rgb}{0.917647,0.917647,0.949020}%
\pgfsetfillcolor{currentfill}%
\pgfsetlinewidth{0.000000pt}%
\definecolor{currentstroke}{rgb}{0.000000,0.000000,0.000000}%
\pgfsetstrokecolor{currentstroke}%
\pgfsetstrokeopacity{0.000000}%
\pgfsetdash{}{0pt}%
\pgfpathmoveto{\pgfqpoint{0.800000in}{0.528000in}}%
\pgfpathlineto{\pgfqpoint{5.760000in}{0.528000in}}%
\pgfpathlineto{\pgfqpoint{5.760000in}{4.224000in}}%
\pgfpathlineto{\pgfqpoint{0.800000in}{4.224000in}}%
\pgfpathlineto{\pgfqpoint{0.800000in}{0.528000in}}%
\pgfpathclose%
\pgfusepath{fill}%
\end{pgfscope}%
\begin{pgfscope}%
\pgfpathrectangle{\pgfqpoint{0.800000in}{0.528000in}}{\pgfqpoint{4.960000in}{3.696000in}}%
\pgfusepath{clip}%
\pgfsetroundcap%
\pgfsetroundjoin%
\pgfsetlinewidth{1.003750pt}%
\definecolor{currentstroke}{rgb}{1.000000,1.000000,1.000000}%
\pgfsetstrokecolor{currentstroke}%
\pgfsetdash{}{0pt}%
\pgfpathmoveto{\pgfqpoint{1.025455in}{0.528000in}}%
\pgfpathlineto{\pgfqpoint{1.025455in}{4.224000in}}%
\pgfusepath{stroke}%
\end{pgfscope}%
\begin{pgfscope}%
\definecolor{textcolor}{rgb}{0.150000,0.150000,0.150000}%
\pgfsetstrokecolor{textcolor}%
\pgfsetfillcolor{textcolor}%
\pgftext[x=1.025455in,y=0.396056in,,top]{\color{textcolor}{\sffamily\fontsize{11.000000}{13.200000}\selectfont\catcode`\^=\active\def^{\ifmmode\sp\else\^{}\fi}\catcode`\%=\active\def%{\%}00:00}}%
\end{pgfscope}%
\begin{pgfscope}%
\pgfpathrectangle{\pgfqpoint{0.800000in}{0.528000in}}{\pgfqpoint{4.960000in}{3.696000in}}%
\pgfusepath{clip}%
\pgfsetroundcap%
\pgfsetroundjoin%
\pgfsetlinewidth{1.003750pt}%
\definecolor{currentstroke}{rgb}{1.000000,1.000000,1.000000}%
\pgfsetstrokecolor{currentstroke}%
\pgfsetdash{}{0pt}%
\pgfpathmoveto{\pgfqpoint{1.589091in}{0.528000in}}%
\pgfpathlineto{\pgfqpoint{1.589091in}{4.224000in}}%
\pgfusepath{stroke}%
\end{pgfscope}%
\begin{pgfscope}%
\definecolor{textcolor}{rgb}{0.150000,0.150000,0.150000}%
\pgfsetstrokecolor{textcolor}%
\pgfsetfillcolor{textcolor}%
\pgftext[x=1.589091in,y=0.396056in,,top]{\color{textcolor}{\sffamily\fontsize{11.000000}{13.200000}\selectfont\catcode`\^=\active\def^{\ifmmode\sp\else\^{}\fi}\catcode`\%=\active\def%{\%}03:00}}%
\end{pgfscope}%
\begin{pgfscope}%
\pgfpathrectangle{\pgfqpoint{0.800000in}{0.528000in}}{\pgfqpoint{4.960000in}{3.696000in}}%
\pgfusepath{clip}%
\pgfsetroundcap%
\pgfsetroundjoin%
\pgfsetlinewidth{1.003750pt}%
\definecolor{currentstroke}{rgb}{1.000000,1.000000,1.000000}%
\pgfsetstrokecolor{currentstroke}%
\pgfsetdash{}{0pt}%
\pgfpathmoveto{\pgfqpoint{2.152727in}{0.528000in}}%
\pgfpathlineto{\pgfqpoint{2.152727in}{4.224000in}}%
\pgfusepath{stroke}%
\end{pgfscope}%
\begin{pgfscope}%
\definecolor{textcolor}{rgb}{0.150000,0.150000,0.150000}%
\pgfsetstrokecolor{textcolor}%
\pgfsetfillcolor{textcolor}%
\pgftext[x=2.152727in,y=0.396056in,,top]{\color{textcolor}{\sffamily\fontsize{11.000000}{13.200000}\selectfont\catcode`\^=\active\def^{\ifmmode\sp\else\^{}\fi}\catcode`\%=\active\def%{\%}06:00}}%
\end{pgfscope}%
\begin{pgfscope}%
\pgfpathrectangle{\pgfqpoint{0.800000in}{0.528000in}}{\pgfqpoint{4.960000in}{3.696000in}}%
\pgfusepath{clip}%
\pgfsetroundcap%
\pgfsetroundjoin%
\pgfsetlinewidth{1.003750pt}%
\definecolor{currentstroke}{rgb}{1.000000,1.000000,1.000000}%
\pgfsetstrokecolor{currentstroke}%
\pgfsetdash{}{0pt}%
\pgfpathmoveto{\pgfqpoint{2.716364in}{0.528000in}}%
\pgfpathlineto{\pgfqpoint{2.716364in}{4.224000in}}%
\pgfusepath{stroke}%
\end{pgfscope}%
\begin{pgfscope}%
\definecolor{textcolor}{rgb}{0.150000,0.150000,0.150000}%
\pgfsetstrokecolor{textcolor}%
\pgfsetfillcolor{textcolor}%
\pgftext[x=2.716364in,y=0.396056in,,top]{\color{textcolor}{\sffamily\fontsize{11.000000}{13.200000}\selectfont\catcode`\^=\active\def^{\ifmmode\sp\else\^{}\fi}\catcode`\%=\active\def%{\%}09:00}}%
\end{pgfscope}%
\begin{pgfscope}%
\pgfpathrectangle{\pgfqpoint{0.800000in}{0.528000in}}{\pgfqpoint{4.960000in}{3.696000in}}%
\pgfusepath{clip}%
\pgfsetroundcap%
\pgfsetroundjoin%
\pgfsetlinewidth{1.003750pt}%
\definecolor{currentstroke}{rgb}{1.000000,1.000000,1.000000}%
\pgfsetstrokecolor{currentstroke}%
\pgfsetdash{}{0pt}%
\pgfpathmoveto{\pgfqpoint{3.280000in}{0.528000in}}%
\pgfpathlineto{\pgfqpoint{3.280000in}{4.224000in}}%
\pgfusepath{stroke}%
\end{pgfscope}%
\begin{pgfscope}%
\definecolor{textcolor}{rgb}{0.150000,0.150000,0.150000}%
\pgfsetstrokecolor{textcolor}%
\pgfsetfillcolor{textcolor}%
\pgftext[x=3.280000in,y=0.396056in,,top]{\color{textcolor}{\sffamily\fontsize{11.000000}{13.200000}\selectfont\catcode`\^=\active\def^{\ifmmode\sp\else\^{}\fi}\catcode`\%=\active\def%{\%}12:00}}%
\end{pgfscope}%
\begin{pgfscope}%
\pgfpathrectangle{\pgfqpoint{0.800000in}{0.528000in}}{\pgfqpoint{4.960000in}{3.696000in}}%
\pgfusepath{clip}%
\pgfsetroundcap%
\pgfsetroundjoin%
\pgfsetlinewidth{1.003750pt}%
\definecolor{currentstroke}{rgb}{1.000000,1.000000,1.000000}%
\pgfsetstrokecolor{currentstroke}%
\pgfsetdash{}{0pt}%
\pgfpathmoveto{\pgfqpoint{3.843636in}{0.528000in}}%
\pgfpathlineto{\pgfqpoint{3.843636in}{4.224000in}}%
\pgfusepath{stroke}%
\end{pgfscope}%
\begin{pgfscope}%
\definecolor{textcolor}{rgb}{0.150000,0.150000,0.150000}%
\pgfsetstrokecolor{textcolor}%
\pgfsetfillcolor{textcolor}%
\pgftext[x=3.843636in,y=0.396056in,,top]{\color{textcolor}{\sffamily\fontsize{11.000000}{13.200000}\selectfont\catcode`\^=\active\def^{\ifmmode\sp\else\^{}\fi}\catcode`\%=\active\def%{\%}15:00}}%
\end{pgfscope}%
\begin{pgfscope}%
\pgfpathrectangle{\pgfqpoint{0.800000in}{0.528000in}}{\pgfqpoint{4.960000in}{3.696000in}}%
\pgfusepath{clip}%
\pgfsetroundcap%
\pgfsetroundjoin%
\pgfsetlinewidth{1.003750pt}%
\definecolor{currentstroke}{rgb}{1.000000,1.000000,1.000000}%
\pgfsetstrokecolor{currentstroke}%
\pgfsetdash{}{0pt}%
\pgfpathmoveto{\pgfqpoint{4.407273in}{0.528000in}}%
\pgfpathlineto{\pgfqpoint{4.407273in}{4.224000in}}%
\pgfusepath{stroke}%
\end{pgfscope}%
\begin{pgfscope}%
\definecolor{textcolor}{rgb}{0.150000,0.150000,0.150000}%
\pgfsetstrokecolor{textcolor}%
\pgfsetfillcolor{textcolor}%
\pgftext[x=4.407273in,y=0.396056in,,top]{\color{textcolor}{\sffamily\fontsize{11.000000}{13.200000}\selectfont\catcode`\^=\active\def^{\ifmmode\sp\else\^{}\fi}\catcode`\%=\active\def%{\%}18:00}}%
\end{pgfscope}%
\begin{pgfscope}%
\pgfpathrectangle{\pgfqpoint{0.800000in}{0.528000in}}{\pgfqpoint{4.960000in}{3.696000in}}%
\pgfusepath{clip}%
\pgfsetroundcap%
\pgfsetroundjoin%
\pgfsetlinewidth{1.003750pt}%
\definecolor{currentstroke}{rgb}{1.000000,1.000000,1.000000}%
\pgfsetstrokecolor{currentstroke}%
\pgfsetdash{}{0pt}%
\pgfpathmoveto{\pgfqpoint{4.970909in}{0.528000in}}%
\pgfpathlineto{\pgfqpoint{4.970909in}{4.224000in}}%
\pgfusepath{stroke}%
\end{pgfscope}%
\begin{pgfscope}%
\definecolor{textcolor}{rgb}{0.150000,0.150000,0.150000}%
\pgfsetstrokecolor{textcolor}%
\pgfsetfillcolor{textcolor}%
\pgftext[x=4.970909in,y=0.396056in,,top]{\color{textcolor}{\sffamily\fontsize{11.000000}{13.200000}\selectfont\catcode`\^=\active\def^{\ifmmode\sp\else\^{}\fi}\catcode`\%=\active\def%{\%}21:00}}%
\end{pgfscope}%
\begin{pgfscope}%
\pgfpathrectangle{\pgfqpoint{0.800000in}{0.528000in}}{\pgfqpoint{4.960000in}{3.696000in}}%
\pgfusepath{clip}%
\pgfsetroundcap%
\pgfsetroundjoin%
\pgfsetlinewidth{1.003750pt}%
\definecolor{currentstroke}{rgb}{1.000000,1.000000,1.000000}%
\pgfsetstrokecolor{currentstroke}%
\pgfsetdash{}{0pt}%
\pgfpathmoveto{\pgfqpoint{5.534545in}{0.528000in}}%
\pgfpathlineto{\pgfqpoint{5.534545in}{4.224000in}}%
\pgfusepath{stroke}%
\end{pgfscope}%
\begin{pgfscope}%
\definecolor{textcolor}{rgb}{0.150000,0.150000,0.150000}%
\pgfsetstrokecolor{textcolor}%
\pgfsetfillcolor{textcolor}%
\pgftext[x=5.534545in,y=0.396056in,,top]{\color{textcolor}{\sffamily\fontsize{11.000000}{13.200000}\selectfont\catcode`\^=\active\def^{\ifmmode\sp\else\^{}\fi}\catcode`\%=\active\def%{\%}24:00}}%
\end{pgfscope}%
\begin{pgfscope}%
\definecolor{textcolor}{rgb}{0.150000,0.150000,0.150000}%
\pgfsetstrokecolor{textcolor}%
\pgfsetfillcolor{textcolor}%
\pgftext[x=3.280000in,y=0.200777in,,top]{\color{textcolor}{\sffamily\fontsize{12.000000}{14.400000}\selectfont\catcode`\^=\active\def^{\ifmmode\sp\else\^{}\fi}\catcode`\%=\active\def%{\%}Time}}%
\end{pgfscope}%
\begin{pgfscope}%
\pgfpathrectangle{\pgfqpoint{0.800000in}{0.528000in}}{\pgfqpoint{4.960000in}{3.696000in}}%
\pgfusepath{clip}%
\pgfsetroundcap%
\pgfsetroundjoin%
\pgfsetlinewidth{1.003750pt}%
\definecolor{currentstroke}{rgb}{1.000000,1.000000,1.000000}%
\pgfsetstrokecolor{currentstroke}%
\pgfsetdash{}{0pt}%
\pgfpathmoveto{\pgfqpoint{0.800000in}{0.784658in}}%
\pgfpathlineto{\pgfqpoint{5.760000in}{0.784658in}}%
\pgfusepath{stroke}%
\end{pgfscope}%
\begin{pgfscope}%
\pgfpathrectangle{\pgfqpoint{0.800000in}{0.528000in}}{\pgfqpoint{4.960000in}{3.696000in}}%
\pgfusepath{clip}%
\pgfsetroundcap%
\pgfsetroundjoin%
\pgfsetlinewidth{1.003750pt}%
\definecolor{currentstroke}{rgb}{1.000000,1.000000,1.000000}%
\pgfsetstrokecolor{currentstroke}%
\pgfsetdash{}{0pt}%
\pgfpathmoveto{\pgfqpoint{0.800000in}{1.375205in}}%
\pgfpathlineto{\pgfqpoint{5.760000in}{1.375205in}}%
\pgfusepath{stroke}%
\end{pgfscope}%
\begin{pgfscope}%
\pgfpathrectangle{\pgfqpoint{0.800000in}{0.528000in}}{\pgfqpoint{4.960000in}{3.696000in}}%
\pgfusepath{clip}%
\pgfsetroundcap%
\pgfsetroundjoin%
\pgfsetlinewidth{1.003750pt}%
\definecolor{currentstroke}{rgb}{1.000000,1.000000,1.000000}%
\pgfsetstrokecolor{currentstroke}%
\pgfsetdash{}{0pt}%
\pgfpathmoveto{\pgfqpoint{0.800000in}{1.965751in}}%
\pgfpathlineto{\pgfqpoint{5.760000in}{1.965751in}}%
\pgfusepath{stroke}%
\end{pgfscope}%
\begin{pgfscope}%
\pgfpathrectangle{\pgfqpoint{0.800000in}{0.528000in}}{\pgfqpoint{4.960000in}{3.696000in}}%
\pgfusepath{clip}%
\pgfsetroundcap%
\pgfsetroundjoin%
\pgfsetlinewidth{1.003750pt}%
\definecolor{currentstroke}{rgb}{1.000000,1.000000,1.000000}%
\pgfsetstrokecolor{currentstroke}%
\pgfsetdash{}{0pt}%
\pgfpathmoveto{\pgfqpoint{0.800000in}{2.556297in}}%
\pgfpathlineto{\pgfqpoint{5.760000in}{2.556297in}}%
\pgfusepath{stroke}%
\end{pgfscope}%
\begin{pgfscope}%
\pgfpathrectangle{\pgfqpoint{0.800000in}{0.528000in}}{\pgfqpoint{4.960000in}{3.696000in}}%
\pgfusepath{clip}%
\pgfsetroundcap%
\pgfsetroundjoin%
\pgfsetlinewidth{1.003750pt}%
\definecolor{currentstroke}{rgb}{1.000000,1.000000,1.000000}%
\pgfsetstrokecolor{currentstroke}%
\pgfsetdash{}{0pt}%
\pgfpathmoveto{\pgfqpoint{0.800000in}{3.146844in}}%
\pgfpathlineto{\pgfqpoint{5.760000in}{3.146844in}}%
\pgfusepath{stroke}%
\end{pgfscope}%
\begin{pgfscope}%
\pgfpathrectangle{\pgfqpoint{0.800000in}{0.528000in}}{\pgfqpoint{4.960000in}{3.696000in}}%
\pgfusepath{clip}%
\pgfsetroundcap%
\pgfsetroundjoin%
\pgfsetlinewidth{1.003750pt}%
\definecolor{currentstroke}{rgb}{1.000000,1.000000,1.000000}%
\pgfsetstrokecolor{currentstroke}%
\pgfsetdash{}{0pt}%
\pgfpathmoveto{\pgfqpoint{0.800000in}{3.737390in}}%
\pgfpathlineto{\pgfqpoint{5.760000in}{3.737390in}}%
\pgfusepath{stroke}%
\end{pgfscope}%
\begin{pgfscope}%
\definecolor{textcolor}{rgb}{0.150000,0.150000,0.150000}%
\pgfsetstrokecolor{textcolor}%
\pgfsetfillcolor{textcolor}%
\pgftext[x=0.612500in,y=2.376000in,,bottom,rotate=90.000000]{\color{textcolor}{\sffamily\fontsize{12.000000}{14.400000}\selectfont\catcode`\^=\active\def^{\ifmmode\sp\else\^{}\fi}\catcode`\%=\active\def%{\%}Resources}}%
\end{pgfscope}%
\begin{pgfscope}%
\pgfpathrectangle{\pgfqpoint{0.800000in}{0.528000in}}{\pgfqpoint{4.960000in}{3.696000in}}%
\pgfusepath{clip}%
\pgfsetbuttcap%
\pgfsetroundjoin%
\definecolor{currentfill}{rgb}{0.172549,0.627451,0.172549}%
\pgfsetfillcolor{currentfill}%
\pgfsetfillopacity{0.300000}%
\pgfsetlinewidth{1.003750pt}%
\definecolor{currentstroke}{rgb}{0.172549,0.627451,0.172549}%
\pgfsetstrokecolor{currentstroke}%
\pgfsetstrokeopacity{0.300000}%
\pgfsetdash{}{0pt}%
\pgfpathmoveto{\pgfqpoint{1.034491in}{0.798290in}}%
\pgfpathlineto{\pgfqpoint{1.034491in}{0.784129in}}%
\pgfpathlineto{\pgfqpoint{1.043527in}{0.783552in}}%
\pgfpathlineto{\pgfqpoint{1.052563in}{0.782929in}}%
\pgfpathlineto{\pgfqpoint{1.061600in}{0.782265in}}%
\pgfpathlineto{\pgfqpoint{1.070636in}{0.781562in}}%
\pgfpathlineto{\pgfqpoint{1.079672in}{0.780823in}}%
\pgfpathlineto{\pgfqpoint{1.088708in}{0.780053in}}%
\pgfpathlineto{\pgfqpoint{1.097745in}{0.779253in}}%
\pgfpathlineto{\pgfqpoint{1.106781in}{0.778428in}}%
\pgfpathlineto{\pgfqpoint{1.115817in}{0.777581in}}%
\pgfpathlineto{\pgfqpoint{1.124853in}{0.776715in}}%
\pgfpathlineto{\pgfqpoint{1.133890in}{0.775833in}}%
\pgfpathlineto{\pgfqpoint{1.142926in}{0.774938in}}%
\pgfpathlineto{\pgfqpoint{1.151962in}{0.774034in}}%
\pgfpathlineto{\pgfqpoint{1.160998in}{0.773125in}}%
\pgfpathlineto{\pgfqpoint{1.170035in}{0.772212in}}%
\pgfpathlineto{\pgfqpoint{1.179071in}{0.771300in}}%
\pgfpathlineto{\pgfqpoint{1.188107in}{0.770392in}}%
\pgfpathlineto{\pgfqpoint{1.197143in}{0.769490in}}%
\pgfpathlineto{\pgfqpoint{1.206180in}{0.768599in}}%
\pgfpathlineto{\pgfqpoint{1.215216in}{0.767722in}}%
\pgfpathlineto{\pgfqpoint{1.224252in}{0.766862in}}%
\pgfpathlineto{\pgfqpoint{1.233288in}{0.766021in}}%
\pgfpathlineto{\pgfqpoint{1.242325in}{0.765204in}}%
\pgfpathlineto{\pgfqpoint{1.251361in}{0.764414in}}%
\pgfpathlineto{\pgfqpoint{1.260397in}{0.763653in}}%
\pgfpathlineto{\pgfqpoint{1.269433in}{0.762926in}}%
\pgfpathlineto{\pgfqpoint{1.278470in}{0.762234in}}%
\pgfpathlineto{\pgfqpoint{1.287506in}{0.761583in}}%
\pgfpathlineto{\pgfqpoint{1.296542in}{0.760975in}}%
\pgfpathlineto{\pgfqpoint{1.305578in}{0.760412in}}%
\pgfpathlineto{\pgfqpoint{1.314615in}{0.759900in}}%
\pgfpathlineto{\pgfqpoint{1.323651in}{0.759439in}}%
\pgfpathlineto{\pgfqpoint{1.332687in}{0.759035in}}%
\pgfpathlineto{\pgfqpoint{1.341723in}{0.758690in}}%
\pgfpathlineto{\pgfqpoint{1.350760in}{0.758408in}}%
\pgfpathlineto{\pgfqpoint{1.359796in}{0.758191in}}%
\pgfpathlineto{\pgfqpoint{1.368832in}{0.758044in}}%
\pgfpathlineto{\pgfqpoint{1.377868in}{0.757968in}}%
\pgfpathlineto{\pgfqpoint{1.386905in}{0.757969in}}%
\pgfpathlineto{\pgfqpoint{1.395941in}{0.758048in}}%
\pgfpathlineto{\pgfqpoint{1.404977in}{0.758209in}}%
\pgfpathlineto{\pgfqpoint{1.414013in}{0.758455in}}%
\pgfpathlineto{\pgfqpoint{1.423050in}{0.758790in}}%
\pgfpathlineto{\pgfqpoint{1.432086in}{0.759217in}}%
\pgfpathlineto{\pgfqpoint{1.441122in}{0.759739in}}%
\pgfpathlineto{\pgfqpoint{1.450158in}{0.760359in}}%
\pgfpathlineto{\pgfqpoint{1.459195in}{0.761081in}}%
\pgfpathlineto{\pgfqpoint{1.468231in}{0.761908in}}%
\pgfpathlineto{\pgfqpoint{1.477267in}{0.762843in}}%
\pgfpathlineto{\pgfqpoint{1.486304in}{0.763889in}}%
\pgfpathlineto{\pgfqpoint{1.495340in}{0.765050in}}%
\pgfpathlineto{\pgfqpoint{1.504376in}{0.766329in}}%
\pgfpathlineto{\pgfqpoint{1.513412in}{0.767730in}}%
\pgfpathlineto{\pgfqpoint{1.522449in}{0.769254in}}%
\pgfpathlineto{\pgfqpoint{1.531485in}{0.770907in}}%
\pgfpathlineto{\pgfqpoint{1.540521in}{0.772690in}}%
\pgfpathlineto{\pgfqpoint{1.549557in}{0.774608in}}%
\pgfpathlineto{\pgfqpoint{1.558594in}{0.776663in}}%
\pgfpathlineto{\pgfqpoint{1.567630in}{0.778859in}}%
\pgfpathlineto{\pgfqpoint{1.576666in}{0.781199in}}%
\pgfpathlineto{\pgfqpoint{1.585702in}{0.783687in}}%
\pgfpathlineto{\pgfqpoint{1.585702in}{0.786842in}}%
\pgfpathlineto{\pgfqpoint{1.585702in}{0.786842in}}%
\pgfpathlineto{\pgfqpoint{1.576666in}{0.792712in}}%
\pgfpathlineto{\pgfqpoint{1.567630in}{0.798641in}}%
\pgfpathlineto{\pgfqpoint{1.558594in}{0.804615in}}%
\pgfpathlineto{\pgfqpoint{1.549557in}{0.810623in}}%
\pgfpathlineto{\pgfqpoint{1.540521in}{0.816652in}}%
\pgfpathlineto{\pgfqpoint{1.531485in}{0.822689in}}%
\pgfpathlineto{\pgfqpoint{1.522449in}{0.828721in}}%
\pgfpathlineto{\pgfqpoint{1.513412in}{0.834736in}}%
\pgfpathlineto{\pgfqpoint{1.504376in}{0.840722in}}%
\pgfpathlineto{\pgfqpoint{1.495340in}{0.846665in}}%
\pgfpathlineto{\pgfqpoint{1.486304in}{0.852553in}}%
\pgfpathlineto{\pgfqpoint{1.477267in}{0.858374in}}%
\pgfpathlineto{\pgfqpoint{1.468231in}{0.864115in}}%
\pgfpathlineto{\pgfqpoint{1.459195in}{0.869763in}}%
\pgfpathlineto{\pgfqpoint{1.450158in}{0.875305in}}%
\pgfpathlineto{\pgfqpoint{1.441122in}{0.880730in}}%
\pgfpathlineto{\pgfqpoint{1.432086in}{0.886024in}}%
\pgfpathlineto{\pgfqpoint{1.423050in}{0.891174in}}%
\pgfpathlineto{\pgfqpoint{1.414013in}{0.896169in}}%
\pgfpathlineto{\pgfqpoint{1.404977in}{0.900996in}}%
\pgfpathlineto{\pgfqpoint{1.395941in}{0.905642in}}%
\pgfpathlineto{\pgfqpoint{1.386905in}{0.910094in}}%
\pgfpathlineto{\pgfqpoint{1.377868in}{0.914339in}}%
\pgfpathlineto{\pgfqpoint{1.368832in}{0.918366in}}%
\pgfpathlineto{\pgfqpoint{1.359796in}{0.922162in}}%
\pgfpathlineto{\pgfqpoint{1.350760in}{0.925713in}}%
\pgfpathlineto{\pgfqpoint{1.341723in}{0.929008in}}%
\pgfpathlineto{\pgfqpoint{1.332687in}{0.932033in}}%
\pgfpathlineto{\pgfqpoint{1.323651in}{0.934777in}}%
\pgfpathlineto{\pgfqpoint{1.314615in}{0.937226in}}%
\pgfpathlineto{\pgfqpoint{1.305578in}{0.939368in}}%
\pgfpathlineto{\pgfqpoint{1.296542in}{0.941190in}}%
\pgfpathlineto{\pgfqpoint{1.287506in}{0.942680in}}%
\pgfpathlineto{\pgfqpoint{1.278470in}{0.943826in}}%
\pgfpathlineto{\pgfqpoint{1.269433in}{0.944613in}}%
\pgfpathlineto{\pgfqpoint{1.260397in}{0.945031in}}%
\pgfpathlineto{\pgfqpoint{1.251361in}{0.945065in}}%
\pgfpathlineto{\pgfqpoint{1.242325in}{0.944704in}}%
\pgfpathlineto{\pgfqpoint{1.233288in}{0.943936in}}%
\pgfpathlineto{\pgfqpoint{1.224252in}{0.942747in}}%
\pgfpathlineto{\pgfqpoint{1.215216in}{0.941124in}}%
\pgfpathlineto{\pgfqpoint{1.206180in}{0.939056in}}%
\pgfpathlineto{\pgfqpoint{1.197143in}{0.936529in}}%
\pgfpathlineto{\pgfqpoint{1.188107in}{0.933532in}}%
\pgfpathlineto{\pgfqpoint{1.179071in}{0.930051in}}%
\pgfpathlineto{\pgfqpoint{1.170035in}{0.926073in}}%
\pgfpathlineto{\pgfqpoint{1.160998in}{0.921587in}}%
\pgfpathlineto{\pgfqpoint{1.151962in}{0.916579in}}%
\pgfpathlineto{\pgfqpoint{1.142926in}{0.911037in}}%
\pgfpathlineto{\pgfqpoint{1.133890in}{0.904949in}}%
\pgfpathlineto{\pgfqpoint{1.124853in}{0.898301in}}%
\pgfpathlineto{\pgfqpoint{1.115817in}{0.891081in}}%
\pgfpathlineto{\pgfqpoint{1.106781in}{0.883277in}}%
\pgfpathlineto{\pgfqpoint{1.097745in}{0.874876in}}%
\pgfpathlineto{\pgfqpoint{1.088708in}{0.865865in}}%
\pgfpathlineto{\pgfqpoint{1.079672in}{0.856232in}}%
\pgfpathlineto{\pgfqpoint{1.070636in}{0.845964in}}%
\pgfpathlineto{\pgfqpoint{1.061600in}{0.835048in}}%
\pgfpathlineto{\pgfqpoint{1.052563in}{0.823473in}}%
\pgfpathlineto{\pgfqpoint{1.043527in}{0.811224in}}%
\pgfpathlineto{\pgfqpoint{1.034491in}{0.798290in}}%
\pgfpathlineto{\pgfqpoint{1.034491in}{0.798290in}}%
\pgfpathclose%
\pgfusepath{stroke,fill}%
\end{pgfscope}%
\begin{pgfscope}%
\pgfpathrectangle{\pgfqpoint{0.800000in}{0.528000in}}{\pgfqpoint{4.960000in}{3.696000in}}%
\pgfusepath{clip}%
\pgfsetbuttcap%
\pgfsetroundjoin%
\definecolor{currentfill}{rgb}{0.172549,0.627451,0.172549}%
\pgfsetfillcolor{currentfill}%
\pgfsetfillopacity{0.300000}%
\pgfsetlinewidth{1.003750pt}%
\definecolor{currentstroke}{rgb}{0.172549,0.627451,0.172549}%
\pgfsetstrokecolor{currentstroke}%
\pgfsetstrokeopacity{0.300000}%
\pgfsetdash{}{0pt}%
\pgfpathmoveto{\pgfqpoint{2.426074in}{2.061485in}}%
\pgfpathlineto{\pgfqpoint{2.426074in}{2.053734in}}%
\pgfpathlineto{\pgfqpoint{2.435110in}{2.078830in}}%
\pgfpathlineto{\pgfqpoint{2.444146in}{2.103999in}}%
\pgfpathlineto{\pgfqpoint{2.453183in}{2.129234in}}%
\pgfpathlineto{\pgfqpoint{2.462219in}{2.154524in}}%
\pgfpathlineto{\pgfqpoint{2.471255in}{2.179860in}}%
\pgfpathlineto{\pgfqpoint{2.480291in}{2.205235in}}%
\pgfpathlineto{\pgfqpoint{2.489328in}{2.230638in}}%
\pgfpathlineto{\pgfqpoint{2.498364in}{2.256060in}}%
\pgfpathlineto{\pgfqpoint{2.507400in}{2.281494in}}%
\pgfpathlineto{\pgfqpoint{2.516437in}{2.306929in}}%
\pgfpathlineto{\pgfqpoint{2.525473in}{2.332356in}}%
\pgfpathlineto{\pgfqpoint{2.534509in}{2.357768in}}%
\pgfpathlineto{\pgfqpoint{2.543545in}{2.383154in}}%
\pgfpathlineto{\pgfqpoint{2.552582in}{2.408505in}}%
\pgfpathlineto{\pgfqpoint{2.561618in}{2.433814in}}%
\pgfpathlineto{\pgfqpoint{2.570654in}{2.459069in}}%
\pgfpathlineto{\pgfqpoint{2.579690in}{2.484264in}}%
\pgfpathlineto{\pgfqpoint{2.588727in}{2.509388in}}%
\pgfpathlineto{\pgfqpoint{2.597763in}{2.534432in}}%
\pgfpathlineto{\pgfqpoint{2.606799in}{2.559388in}}%
\pgfpathlineto{\pgfqpoint{2.615835in}{2.584247in}}%
\pgfpathlineto{\pgfqpoint{2.624872in}{2.608999in}}%
\pgfpathlineto{\pgfqpoint{2.633908in}{2.633636in}}%
\pgfpathlineto{\pgfqpoint{2.642944in}{2.658149in}}%
\pgfpathlineto{\pgfqpoint{2.651980in}{2.682527in}}%
\pgfpathlineto{\pgfqpoint{2.661017in}{2.706764in}}%
\pgfpathlineto{\pgfqpoint{2.670053in}{2.730849in}}%
\pgfpathlineto{\pgfqpoint{2.679089in}{2.754773in}}%
\pgfpathlineto{\pgfqpoint{2.688125in}{2.778528in}}%
\pgfpathlineto{\pgfqpoint{2.697162in}{2.802104in}}%
\pgfpathlineto{\pgfqpoint{2.706198in}{2.825493in}}%
\pgfpathlineto{\pgfqpoint{2.715234in}{2.848686in}}%
\pgfpathlineto{\pgfqpoint{2.724270in}{2.871672in}}%
\pgfpathlineto{\pgfqpoint{2.733307in}{2.894443in}}%
\pgfpathlineto{\pgfqpoint{2.742343in}{2.916985in}}%
\pgfpathlineto{\pgfqpoint{2.751379in}{2.939290in}}%
\pgfpathlineto{\pgfqpoint{2.760415in}{2.961345in}}%
\pgfpathlineto{\pgfqpoint{2.769452in}{2.983140in}}%
\pgfpathlineto{\pgfqpoint{2.778488in}{3.004663in}}%
\pgfpathlineto{\pgfqpoint{2.787524in}{3.025904in}}%
\pgfpathlineto{\pgfqpoint{2.796560in}{3.046851in}}%
\pgfpathlineto{\pgfqpoint{2.805597in}{3.067494in}}%
\pgfpathlineto{\pgfqpoint{2.814633in}{3.087821in}}%
\pgfpathlineto{\pgfqpoint{2.823669in}{3.107821in}}%
\pgfpathlineto{\pgfqpoint{2.832705in}{3.127484in}}%
\pgfpathlineto{\pgfqpoint{2.841742in}{3.146798in}}%
\pgfpathlineto{\pgfqpoint{2.850778in}{3.165753in}}%
\pgfpathlineto{\pgfqpoint{2.859814in}{3.184337in}}%
\pgfpathlineto{\pgfqpoint{2.868850in}{3.202539in}}%
\pgfpathlineto{\pgfqpoint{2.877887in}{3.220348in}}%
\pgfpathlineto{\pgfqpoint{2.886923in}{3.237753in}}%
\pgfpathlineto{\pgfqpoint{2.895959in}{3.254744in}}%
\pgfpathlineto{\pgfqpoint{2.904995in}{3.271308in}}%
\pgfpathlineto{\pgfqpoint{2.914032in}{3.287436in}}%
\pgfpathlineto{\pgfqpoint{2.923068in}{3.303115in}}%
\pgfpathlineto{\pgfqpoint{2.932104in}{3.318336in}}%
\pgfpathlineto{\pgfqpoint{2.941140in}{3.333087in}}%
\pgfpathlineto{\pgfqpoint{2.950177in}{3.347357in}}%
\pgfpathlineto{\pgfqpoint{2.959213in}{3.361134in}}%
\pgfpathlineto{\pgfqpoint{2.968249in}{3.374409in}}%
\pgfpathlineto{\pgfqpoint{2.977285in}{3.387169in}}%
\pgfpathlineto{\pgfqpoint{2.986322in}{3.399405in}}%
\pgfpathlineto{\pgfqpoint{2.995358in}{3.411104in}}%
\pgfpathlineto{\pgfqpoint{3.004394in}{3.422256in}}%
\pgfpathlineto{\pgfqpoint{3.013430in}{3.432849in}}%
\pgfpathlineto{\pgfqpoint{3.022467in}{3.442873in}}%
\pgfpathlineto{\pgfqpoint{3.031503in}{3.452317in}}%
\pgfpathlineto{\pgfqpoint{3.040539in}{3.461170in}}%
\pgfpathlineto{\pgfqpoint{3.049576in}{3.469421in}}%
\pgfpathlineto{\pgfqpoint{3.058612in}{3.477057in}}%
\pgfpathlineto{\pgfqpoint{3.067648in}{3.484070in}}%
\pgfpathlineto{\pgfqpoint{3.076684in}{3.490447in}}%
\pgfpathlineto{\pgfqpoint{3.085721in}{3.496178in}}%
\pgfpathlineto{\pgfqpoint{3.094757in}{3.501251in}}%
\pgfpathlineto{\pgfqpoint{3.103793in}{3.505655in}}%
\pgfpathlineto{\pgfqpoint{3.112829in}{3.509380in}}%
\pgfpathlineto{\pgfqpoint{3.121866in}{3.512414in}}%
\pgfpathlineto{\pgfqpoint{3.130902in}{3.514747in}}%
\pgfpathlineto{\pgfqpoint{3.139938in}{3.516367in}}%
\pgfpathlineto{\pgfqpoint{3.148974in}{3.517263in}}%
\pgfpathlineto{\pgfqpoint{3.158011in}{3.517425in}}%
\pgfpathlineto{\pgfqpoint{3.167047in}{3.516841in}}%
\pgfpathlineto{\pgfqpoint{3.176083in}{3.515500in}}%
\pgfpathlineto{\pgfqpoint{3.185119in}{3.513391in}}%
\pgfpathlineto{\pgfqpoint{3.194156in}{3.510503in}}%
\pgfpathlineto{\pgfqpoint{3.203192in}{3.506826in}}%
\pgfpathlineto{\pgfqpoint{3.212228in}{3.502347in}}%
\pgfpathlineto{\pgfqpoint{3.221264in}{3.497057in}}%
\pgfpathlineto{\pgfqpoint{3.230301in}{3.490944in}}%
\pgfpathlineto{\pgfqpoint{3.239337in}{3.483996in}}%
\pgfpathlineto{\pgfqpoint{3.248373in}{3.476204in}}%
\pgfpathlineto{\pgfqpoint{3.257409in}{3.467556in}}%
\pgfpathlineto{\pgfqpoint{3.266446in}{3.458040in}}%
\pgfpathlineto{\pgfqpoint{3.275482in}{3.447646in}}%
\pgfpathlineto{\pgfqpoint{3.275482in}{3.448205in}}%
\pgfpathlineto{\pgfqpoint{3.275482in}{3.448205in}}%
\pgfpathlineto{\pgfqpoint{3.266446in}{3.460061in}}%
\pgfpathlineto{\pgfqpoint{3.257409in}{3.471487in}}%
\pgfpathlineto{\pgfqpoint{3.248373in}{3.482477in}}%
\pgfpathlineto{\pgfqpoint{3.239337in}{3.493023in}}%
\pgfpathlineto{\pgfqpoint{3.230301in}{3.503121in}}%
\pgfpathlineto{\pgfqpoint{3.221264in}{3.512765in}}%
\pgfpathlineto{\pgfqpoint{3.212228in}{3.521948in}}%
\pgfpathlineto{\pgfqpoint{3.203192in}{3.530664in}}%
\pgfpathlineto{\pgfqpoint{3.194156in}{3.538909in}}%
\pgfpathlineto{\pgfqpoint{3.185119in}{3.546674in}}%
\pgfpathlineto{\pgfqpoint{3.176083in}{3.553956in}}%
\pgfpathlineto{\pgfqpoint{3.167047in}{3.560747in}}%
\pgfpathlineto{\pgfqpoint{3.158011in}{3.567042in}}%
\pgfpathlineto{\pgfqpoint{3.148974in}{3.572835in}}%
\pgfpathlineto{\pgfqpoint{3.139938in}{3.578120in}}%
\pgfpathlineto{\pgfqpoint{3.130902in}{3.582890in}}%
\pgfpathlineto{\pgfqpoint{3.121866in}{3.587141in}}%
\pgfpathlineto{\pgfqpoint{3.112829in}{3.590866in}}%
\pgfpathlineto{\pgfqpoint{3.103793in}{3.594058in}}%
\pgfpathlineto{\pgfqpoint{3.094757in}{3.596713in}}%
\pgfpathlineto{\pgfqpoint{3.085721in}{3.598824in}}%
\pgfpathlineto{\pgfqpoint{3.076684in}{3.600385in}}%
\pgfpathlineto{\pgfqpoint{3.067648in}{3.601391in}}%
\pgfpathlineto{\pgfqpoint{3.058612in}{3.601834in}}%
\pgfpathlineto{\pgfqpoint{3.049576in}{3.601710in}}%
\pgfpathlineto{\pgfqpoint{3.040539in}{3.601013in}}%
\pgfpathlineto{\pgfqpoint{3.031503in}{3.599735in}}%
\pgfpathlineto{\pgfqpoint{3.022467in}{3.597873in}}%
\pgfpathlineto{\pgfqpoint{3.013430in}{3.595418in}}%
\pgfpathlineto{\pgfqpoint{3.004394in}{3.592367in}}%
\pgfpathlineto{\pgfqpoint{2.995358in}{3.588712in}}%
\pgfpathlineto{\pgfqpoint{2.986322in}{3.584447in}}%
\pgfpathlineto{\pgfqpoint{2.977285in}{3.579568in}}%
\pgfpathlineto{\pgfqpoint{2.968249in}{3.574067in}}%
\pgfpathlineto{\pgfqpoint{2.959213in}{3.567938in}}%
\pgfpathlineto{\pgfqpoint{2.950177in}{3.561177in}}%
\pgfpathlineto{\pgfqpoint{2.941140in}{3.553777in}}%
\pgfpathlineto{\pgfqpoint{2.932104in}{3.545731in}}%
\pgfpathlineto{\pgfqpoint{2.923068in}{3.537035in}}%
\pgfpathlineto{\pgfqpoint{2.914032in}{3.527681in}}%
\pgfpathlineto{\pgfqpoint{2.904995in}{3.517665in}}%
\pgfpathlineto{\pgfqpoint{2.895959in}{3.506979in}}%
\pgfpathlineto{\pgfqpoint{2.886923in}{3.495619in}}%
\pgfpathlineto{\pgfqpoint{2.877887in}{3.483578in}}%
\pgfpathlineto{\pgfqpoint{2.868850in}{3.470850in}}%
\pgfpathlineto{\pgfqpoint{2.859814in}{3.457430in}}%
\pgfpathlineto{\pgfqpoint{2.850778in}{3.443311in}}%
\pgfpathlineto{\pgfqpoint{2.841742in}{3.428487in}}%
\pgfpathlineto{\pgfqpoint{2.832705in}{3.412953in}}%
\pgfpathlineto{\pgfqpoint{2.823669in}{3.396702in}}%
\pgfpathlineto{\pgfqpoint{2.814633in}{3.379729in}}%
\pgfpathlineto{\pgfqpoint{2.805597in}{3.362027in}}%
\pgfpathlineto{\pgfqpoint{2.796560in}{3.343591in}}%
\pgfpathlineto{\pgfqpoint{2.787524in}{3.324414in}}%
\pgfpathlineto{\pgfqpoint{2.778488in}{3.304492in}}%
\pgfpathlineto{\pgfqpoint{2.769452in}{3.283817in}}%
\pgfpathlineto{\pgfqpoint{2.760415in}{3.262383in}}%
\pgfpathlineto{\pgfqpoint{2.751379in}{3.240186in}}%
\pgfpathlineto{\pgfqpoint{2.742343in}{3.217219in}}%
\pgfpathlineto{\pgfqpoint{2.733307in}{3.193475in}}%
\pgfpathlineto{\pgfqpoint{2.724270in}{3.168950in}}%
\pgfpathlineto{\pgfqpoint{2.715234in}{3.143636in}}%
\pgfpathlineto{\pgfqpoint{2.706198in}{3.117536in}}%
\pgfpathlineto{\pgfqpoint{2.697162in}{3.090673in}}%
\pgfpathlineto{\pgfqpoint{2.688125in}{3.063074in}}%
\pgfpathlineto{\pgfqpoint{2.679089in}{3.034764in}}%
\pgfpathlineto{\pgfqpoint{2.670053in}{3.005770in}}%
\pgfpathlineto{\pgfqpoint{2.661017in}{2.976119in}}%
\pgfpathlineto{\pgfqpoint{2.651980in}{2.945837in}}%
\pgfpathlineto{\pgfqpoint{2.642944in}{2.914952in}}%
\pgfpathlineto{\pgfqpoint{2.633908in}{2.883489in}}%
\pgfpathlineto{\pgfqpoint{2.624872in}{2.851475in}}%
\pgfpathlineto{\pgfqpoint{2.615835in}{2.818936in}}%
\pgfpathlineto{\pgfqpoint{2.606799in}{2.785900in}}%
\pgfpathlineto{\pgfqpoint{2.597763in}{2.752392in}}%
\pgfpathlineto{\pgfqpoint{2.588727in}{2.718439in}}%
\pgfpathlineto{\pgfqpoint{2.579690in}{2.684068in}}%
\pgfpathlineto{\pgfqpoint{2.570654in}{2.649305in}}%
\pgfpathlineto{\pgfqpoint{2.561618in}{2.614178in}}%
\pgfpathlineto{\pgfqpoint{2.552582in}{2.578711in}}%
\pgfpathlineto{\pgfqpoint{2.543545in}{2.542932in}}%
\pgfpathlineto{\pgfqpoint{2.534509in}{2.506868in}}%
\pgfpathlineto{\pgfqpoint{2.525473in}{2.470545in}}%
\pgfpathlineto{\pgfqpoint{2.516437in}{2.433989in}}%
\pgfpathlineto{\pgfqpoint{2.507400in}{2.397228in}}%
\pgfpathlineto{\pgfqpoint{2.498364in}{2.360287in}}%
\pgfpathlineto{\pgfqpoint{2.489328in}{2.323193in}}%
\pgfpathlineto{\pgfqpoint{2.480291in}{2.285973in}}%
\pgfpathlineto{\pgfqpoint{2.471255in}{2.248652in}}%
\pgfpathlineto{\pgfqpoint{2.462219in}{2.211259in}}%
\pgfpathlineto{\pgfqpoint{2.453183in}{2.173819in}}%
\pgfpathlineto{\pgfqpoint{2.444146in}{2.136359in}}%
\pgfpathlineto{\pgfqpoint{2.435110in}{2.098905in}}%
\pgfpathlineto{\pgfqpoint{2.426074in}{2.061485in}}%
\pgfpathlineto{\pgfqpoint{2.426074in}{2.061485in}}%
\pgfpathclose%
\pgfusepath{stroke,fill}%
\end{pgfscope}%
\begin{pgfscope}%
\pgfpathrectangle{\pgfqpoint{0.800000in}{0.528000in}}{\pgfqpoint{4.960000in}{3.696000in}}%
\pgfusepath{clip}%
\pgfsetbuttcap%
\pgfsetroundjoin%
\definecolor{currentfill}{rgb}{0.172549,0.627451,0.172549}%
\pgfsetfillcolor{currentfill}%
\pgfsetfillopacity{0.300000}%
\pgfsetlinewidth{1.003750pt}%
\definecolor{currentstroke}{rgb}{0.172549,0.627451,0.172549}%
\pgfsetstrokecolor{currentstroke}%
\pgfsetstrokeopacity{0.300000}%
\pgfsetdash{}{0pt}%
\pgfpathmoveto{\pgfqpoint{3.320663in}{3.382953in}}%
\pgfpathlineto{\pgfqpoint{3.320663in}{3.382873in}}%
\pgfpathlineto{\pgfqpoint{3.329699in}{3.367601in}}%
\pgfpathlineto{\pgfqpoint{3.338736in}{3.351645in}}%
\pgfpathlineto{\pgfqpoint{3.347772in}{3.335046in}}%
\pgfpathlineto{\pgfqpoint{3.356808in}{3.317841in}}%
\pgfpathlineto{\pgfqpoint{3.365844in}{3.300069in}}%
\pgfpathlineto{\pgfqpoint{3.374881in}{3.281768in}}%
\pgfpathlineto{\pgfqpoint{3.383917in}{3.262977in}}%
\pgfpathlineto{\pgfqpoint{3.392953in}{3.243734in}}%
\pgfpathlineto{\pgfqpoint{3.401989in}{3.224078in}}%
\pgfpathlineto{\pgfqpoint{3.411026in}{3.204048in}}%
\pgfpathlineto{\pgfqpoint{3.420062in}{3.183682in}}%
\pgfpathlineto{\pgfqpoint{3.429098in}{3.163018in}}%
\pgfpathlineto{\pgfqpoint{3.438134in}{3.142095in}}%
\pgfpathlineto{\pgfqpoint{3.447171in}{3.120952in}}%
\pgfpathlineto{\pgfqpoint{3.456207in}{3.099627in}}%
\pgfpathlineto{\pgfqpoint{3.465243in}{3.078159in}}%
\pgfpathlineto{\pgfqpoint{3.474279in}{3.056586in}}%
\pgfpathlineto{\pgfqpoint{3.483316in}{3.034946in}}%
\pgfpathlineto{\pgfqpoint{3.492352in}{3.013279in}}%
\pgfpathlineto{\pgfqpoint{3.501388in}{2.991622in}}%
\pgfpathlineto{\pgfqpoint{3.510424in}{2.970015in}}%
\pgfpathlineto{\pgfqpoint{3.519461in}{2.948495in}}%
\pgfpathlineto{\pgfqpoint{3.528497in}{2.927102in}}%
\pgfpathlineto{\pgfqpoint{3.537533in}{2.905874in}}%
\pgfpathlineto{\pgfqpoint{3.546570in}{2.884849in}}%
\pgfpathlineto{\pgfqpoint{3.555606in}{2.864066in}}%
\pgfpathlineto{\pgfqpoint{3.564642in}{2.843563in}}%
\pgfpathlineto{\pgfqpoint{3.573678in}{2.823380in}}%
\pgfpathlineto{\pgfqpoint{3.582715in}{2.803554in}}%
\pgfpathlineto{\pgfqpoint{3.591751in}{2.784124in}}%
\pgfpathlineto{\pgfqpoint{3.600787in}{2.765128in}}%
\pgfpathlineto{\pgfqpoint{3.609823in}{2.746606in}}%
\pgfpathlineto{\pgfqpoint{3.618860in}{2.728595in}}%
\pgfpathlineto{\pgfqpoint{3.627896in}{2.711134in}}%
\pgfpathlineto{\pgfqpoint{3.636932in}{2.694262in}}%
\pgfpathlineto{\pgfqpoint{3.645968in}{2.678017in}}%
\pgfpathlineto{\pgfqpoint{3.655005in}{2.662438in}}%
\pgfpathlineto{\pgfqpoint{3.664041in}{2.647563in}}%
\pgfpathlineto{\pgfqpoint{3.673077in}{2.633431in}}%
\pgfpathlineto{\pgfqpoint{3.682113in}{2.620080in}}%
\pgfpathlineto{\pgfqpoint{3.691150in}{2.607549in}}%
\pgfpathlineto{\pgfqpoint{3.700186in}{2.595877in}}%
\pgfpathlineto{\pgfqpoint{3.709222in}{2.585101in}}%
\pgfpathlineto{\pgfqpoint{3.718258in}{2.575261in}}%
\pgfpathlineto{\pgfqpoint{3.727295in}{2.566395in}}%
\pgfpathlineto{\pgfqpoint{3.736331in}{2.558541in}}%
\pgfpathlineto{\pgfqpoint{3.745367in}{2.551738in}}%
\pgfpathlineto{\pgfqpoint{3.754403in}{2.546025in}}%
\pgfpathlineto{\pgfqpoint{3.763440in}{2.541439in}}%
\pgfpathlineto{\pgfqpoint{3.772476in}{2.538021in}}%
\pgfpathlineto{\pgfqpoint{3.781512in}{2.535807in}}%
\pgfpathlineto{\pgfqpoint{3.790548in}{2.534837in}}%
\pgfpathlineto{\pgfqpoint{3.799585in}{2.535150in}}%
\pgfpathlineto{\pgfqpoint{3.808621in}{2.536782in}}%
\pgfpathlineto{\pgfqpoint{3.817657in}{2.539775in}}%
\pgfpathlineto{\pgfqpoint{3.826693in}{2.544165in}}%
\pgfpathlineto{\pgfqpoint{3.835730in}{2.549991in}}%
\pgfpathlineto{\pgfqpoint{3.844766in}{2.557292in}}%
\pgfpathlineto{\pgfqpoint{3.853802in}{2.566086in}}%
\pgfpathlineto{\pgfqpoint{3.862838in}{2.576334in}}%
\pgfpathlineto{\pgfqpoint{3.871875in}{2.587988in}}%
\pgfpathlineto{\pgfqpoint{3.880911in}{2.601000in}}%
\pgfpathlineto{\pgfqpoint{3.889947in}{2.615322in}}%
\pgfpathlineto{\pgfqpoint{3.898983in}{2.630906in}}%
\pgfpathlineto{\pgfqpoint{3.908020in}{2.647703in}}%
\pgfpathlineto{\pgfqpoint{3.917056in}{2.665666in}}%
\pgfpathlineto{\pgfqpoint{3.926092in}{2.684745in}}%
\pgfpathlineto{\pgfqpoint{3.935128in}{2.704893in}}%
\pgfpathlineto{\pgfqpoint{3.944165in}{2.726062in}}%
\pgfpathlineto{\pgfqpoint{3.953201in}{2.748204in}}%
\pgfpathlineto{\pgfqpoint{3.962237in}{2.771270in}}%
\pgfpathlineto{\pgfqpoint{3.971273in}{2.795212in}}%
\pgfpathlineto{\pgfqpoint{3.980310in}{2.819982in}}%
\pgfpathlineto{\pgfqpoint{3.989346in}{2.845533in}}%
\pgfpathlineto{\pgfqpoint{3.998382in}{2.871814in}}%
\pgfpathlineto{\pgfqpoint{4.007418in}{2.898779in}}%
\pgfpathlineto{\pgfqpoint{4.016455in}{2.926380in}}%
\pgfpathlineto{\pgfqpoint{4.025491in}{2.954567in}}%
\pgfpathlineto{\pgfqpoint{4.034527in}{2.983294in}}%
\pgfpathlineto{\pgfqpoint{4.043563in}{3.012511in}}%
\pgfpathlineto{\pgfqpoint{4.052600in}{3.042171in}}%
\pgfpathlineto{\pgfqpoint{4.061636in}{3.072226in}}%
\pgfpathlineto{\pgfqpoint{4.070672in}{3.102626in}}%
\pgfpathlineto{\pgfqpoint{4.079709in}{3.133325in}}%
\pgfpathlineto{\pgfqpoint{4.088745in}{3.164273in}}%
\pgfpathlineto{\pgfqpoint{4.097781in}{3.195424in}}%
\pgfpathlineto{\pgfqpoint{4.106817in}{3.226727in}}%
\pgfpathlineto{\pgfqpoint{4.115854in}{3.258136in}}%
\pgfpathlineto{\pgfqpoint{4.124890in}{3.289603in}}%
\pgfpathlineto{\pgfqpoint{4.133926in}{3.321078in}}%
\pgfpathlineto{\pgfqpoint{4.142962in}{3.352514in}}%
\pgfpathlineto{\pgfqpoint{4.151999in}{3.383863in}}%
\pgfpathlineto{\pgfqpoint{4.161035in}{3.415076in}}%
\pgfpathlineto{\pgfqpoint{4.170071in}{3.446105in}}%
\pgfpathlineto{\pgfqpoint{4.179107in}{3.476903in}}%
\pgfpathlineto{\pgfqpoint{4.188144in}{3.507420in}}%
\pgfpathlineto{\pgfqpoint{4.197180in}{3.537610in}}%
\pgfpathlineto{\pgfqpoint{4.206216in}{3.567423in}}%
\pgfpathlineto{\pgfqpoint{4.215252in}{3.596811in}}%
\pgfpathlineto{\pgfqpoint{4.224289in}{3.625726in}}%
\pgfpathlineto{\pgfqpoint{4.233325in}{3.654121in}}%
\pgfpathlineto{\pgfqpoint{4.242361in}{3.681946in}}%
\pgfpathlineto{\pgfqpoint{4.251397in}{3.709154in}}%
\pgfpathlineto{\pgfqpoint{4.260434in}{3.735697in}}%
\pgfpathlineto{\pgfqpoint{4.269470in}{3.761526in}}%
\pgfpathlineto{\pgfqpoint{4.278506in}{3.786593in}}%
\pgfpathlineto{\pgfqpoint{4.287542in}{3.810850in}}%
\pgfpathlineto{\pgfqpoint{4.296579in}{3.834248in}}%
\pgfpathlineto{\pgfqpoint{4.305615in}{3.856741in}}%
\pgfpathlineto{\pgfqpoint{4.314651in}{3.878279in}}%
\pgfpathlineto{\pgfqpoint{4.323687in}{3.898814in}}%
\pgfpathlineto{\pgfqpoint{4.332724in}{3.918298in}}%
\pgfpathlineto{\pgfqpoint{4.341760in}{3.936683in}}%
\pgfpathlineto{\pgfqpoint{4.350796in}{3.953921in}}%
\pgfpathlineto{\pgfqpoint{4.359832in}{3.969963in}}%
\pgfpathlineto{\pgfqpoint{4.368869in}{3.984762in}}%
\pgfpathlineto{\pgfqpoint{4.377905in}{3.998269in}}%
\pgfpathlineto{\pgfqpoint{4.386941in}{4.010436in}}%
\pgfpathlineto{\pgfqpoint{4.395977in}{4.021215in}}%
\pgfpathlineto{\pgfqpoint{4.405014in}{4.030558in}}%
\pgfpathlineto{\pgfqpoint{4.405014in}{4.031766in}}%
\pgfpathlineto{\pgfqpoint{4.405014in}{4.031766in}}%
\pgfpathlineto{\pgfqpoint{4.395977in}{4.027377in}}%
\pgfpathlineto{\pgfqpoint{4.386941in}{4.021732in}}%
\pgfpathlineto{\pgfqpoint{4.377905in}{4.014872in}}%
\pgfpathlineto{\pgfqpoint{4.368869in}{4.006834in}}%
\pgfpathlineto{\pgfqpoint{4.359832in}{3.997658in}}%
\pgfpathlineto{\pgfqpoint{4.350796in}{3.987383in}}%
\pgfpathlineto{\pgfqpoint{4.341760in}{3.976046in}}%
\pgfpathlineto{\pgfqpoint{4.332724in}{3.963687in}}%
\pgfpathlineto{\pgfqpoint{4.323687in}{3.950345in}}%
\pgfpathlineto{\pgfqpoint{4.314651in}{3.936059in}}%
\pgfpathlineto{\pgfqpoint{4.305615in}{3.920866in}}%
\pgfpathlineto{\pgfqpoint{4.296579in}{3.904806in}}%
\pgfpathlineto{\pgfqpoint{4.287542in}{3.887918in}}%
\pgfpathlineto{\pgfqpoint{4.278506in}{3.870241in}}%
\pgfpathlineto{\pgfqpoint{4.269470in}{3.851812in}}%
\pgfpathlineto{\pgfqpoint{4.260434in}{3.832671in}}%
\pgfpathlineto{\pgfqpoint{4.251397in}{3.812857in}}%
\pgfpathlineto{\pgfqpoint{4.242361in}{3.792409in}}%
\pgfpathlineto{\pgfqpoint{4.233325in}{3.771365in}}%
\pgfpathlineto{\pgfqpoint{4.224289in}{3.749763in}}%
\pgfpathlineto{\pgfqpoint{4.215252in}{3.727643in}}%
\pgfpathlineto{\pgfqpoint{4.206216in}{3.705044in}}%
\pgfpathlineto{\pgfqpoint{4.197180in}{3.682004in}}%
\pgfpathlineto{\pgfqpoint{4.188144in}{3.658562in}}%
\pgfpathlineto{\pgfqpoint{4.179107in}{3.634756in}}%
\pgfpathlineto{\pgfqpoint{4.170071in}{3.610626in}}%
\pgfpathlineto{\pgfqpoint{4.161035in}{3.586210in}}%
\pgfpathlineto{\pgfqpoint{4.151999in}{3.561547in}}%
\pgfpathlineto{\pgfqpoint{4.142962in}{3.536676in}}%
\pgfpathlineto{\pgfqpoint{4.133926in}{3.511635in}}%
\pgfpathlineto{\pgfqpoint{4.124890in}{3.486464in}}%
\pgfpathlineto{\pgfqpoint{4.115854in}{3.461201in}}%
\pgfpathlineto{\pgfqpoint{4.106817in}{3.435884in}}%
\pgfpathlineto{\pgfqpoint{4.097781in}{3.410553in}}%
\pgfpathlineto{\pgfqpoint{4.088745in}{3.385246in}}%
\pgfpathlineto{\pgfqpoint{4.079709in}{3.360002in}}%
\pgfpathlineto{\pgfqpoint{4.070672in}{3.334860in}}%
\pgfpathlineto{\pgfqpoint{4.061636in}{3.309858in}}%
\pgfpathlineto{\pgfqpoint{4.052600in}{3.285036in}}%
\pgfpathlineto{\pgfqpoint{4.043563in}{3.260432in}}%
\pgfpathlineto{\pgfqpoint{4.034527in}{3.236084in}}%
\pgfpathlineto{\pgfqpoint{4.025491in}{3.212032in}}%
\pgfpathlineto{\pgfqpoint{4.016455in}{3.188315in}}%
\pgfpathlineto{\pgfqpoint{4.007418in}{3.164970in}}%
\pgfpathlineto{\pgfqpoint{3.998382in}{3.142037in}}%
\pgfpathlineto{\pgfqpoint{3.989346in}{3.119555in}}%
\pgfpathlineto{\pgfqpoint{3.980310in}{3.097562in}}%
\pgfpathlineto{\pgfqpoint{3.971273in}{3.076098in}}%
\pgfpathlineto{\pgfqpoint{3.962237in}{3.055200in}}%
\pgfpathlineto{\pgfqpoint{3.953201in}{3.034907in}}%
\pgfpathlineto{\pgfqpoint{3.944165in}{3.015259in}}%
\pgfpathlineto{\pgfqpoint{3.935128in}{2.996295in}}%
\pgfpathlineto{\pgfqpoint{3.926092in}{2.978052in}}%
\pgfpathlineto{\pgfqpoint{3.917056in}{2.960569in}}%
\pgfpathlineto{\pgfqpoint{3.908020in}{2.943886in}}%
\pgfpathlineto{\pgfqpoint{3.898983in}{2.928041in}}%
\pgfpathlineto{\pgfqpoint{3.889947in}{2.913074in}}%
\pgfpathlineto{\pgfqpoint{3.880911in}{2.899021in}}%
\pgfpathlineto{\pgfqpoint{3.871875in}{2.885923in}}%
\pgfpathlineto{\pgfqpoint{3.862838in}{2.873819in}}%
\pgfpathlineto{\pgfqpoint{3.853802in}{2.862746in}}%
\pgfpathlineto{\pgfqpoint{3.844766in}{2.852744in}}%
\pgfpathlineto{\pgfqpoint{3.835730in}{2.843844in}}%
\pgfpathlineto{\pgfqpoint{3.826693in}{2.836037in}}%
\pgfpathlineto{\pgfqpoint{3.817657in}{2.829298in}}%
\pgfpathlineto{\pgfqpoint{3.808621in}{2.823602in}}%
\pgfpathlineto{\pgfqpoint{3.799585in}{2.818925in}}%
\pgfpathlineto{\pgfqpoint{3.790548in}{2.815243in}}%
\pgfpathlineto{\pgfqpoint{3.781512in}{2.812530in}}%
\pgfpathlineto{\pgfqpoint{3.772476in}{2.810762in}}%
\pgfpathlineto{\pgfqpoint{3.763440in}{2.809914in}}%
\pgfpathlineto{\pgfqpoint{3.754403in}{2.809962in}}%
\pgfpathlineto{\pgfqpoint{3.745367in}{2.810882in}}%
\pgfpathlineto{\pgfqpoint{3.736331in}{2.812647in}}%
\pgfpathlineto{\pgfqpoint{3.727295in}{2.815235in}}%
\pgfpathlineto{\pgfqpoint{3.718258in}{2.818619in}}%
\pgfpathlineto{\pgfqpoint{3.709222in}{2.822777in}}%
\pgfpathlineto{\pgfqpoint{3.700186in}{2.827682in}}%
\pgfpathlineto{\pgfqpoint{3.691150in}{2.833310in}}%
\pgfpathlineto{\pgfqpoint{3.682113in}{2.839637in}}%
\pgfpathlineto{\pgfqpoint{3.673077in}{2.846639in}}%
\pgfpathlineto{\pgfqpoint{3.664041in}{2.854290in}}%
\pgfpathlineto{\pgfqpoint{3.655005in}{2.862565in}}%
\pgfpathlineto{\pgfqpoint{3.645968in}{2.871441in}}%
\pgfpathlineto{\pgfqpoint{3.636932in}{2.880892in}}%
\pgfpathlineto{\pgfqpoint{3.627896in}{2.890895in}}%
\pgfpathlineto{\pgfqpoint{3.618860in}{2.901423in}}%
\pgfpathlineto{\pgfqpoint{3.609823in}{2.912453in}}%
\pgfpathlineto{\pgfqpoint{3.600787in}{2.923961in}}%
\pgfpathlineto{\pgfqpoint{3.591751in}{2.935921in}}%
\pgfpathlineto{\pgfqpoint{3.582715in}{2.948308in}}%
\pgfpathlineto{\pgfqpoint{3.573678in}{2.961099in}}%
\pgfpathlineto{\pgfqpoint{3.564642in}{2.974268in}}%
\pgfpathlineto{\pgfqpoint{3.555606in}{2.987792in}}%
\pgfpathlineto{\pgfqpoint{3.546570in}{3.001644in}}%
\pgfpathlineto{\pgfqpoint{3.537533in}{3.015802in}}%
\pgfpathlineto{\pgfqpoint{3.528497in}{3.030239in}}%
\pgfpathlineto{\pgfqpoint{3.519461in}{3.044932in}}%
\pgfpathlineto{\pgfqpoint{3.510424in}{3.059855in}}%
\pgfpathlineto{\pgfqpoint{3.501388in}{3.074985in}}%
\pgfpathlineto{\pgfqpoint{3.492352in}{3.090296in}}%
\pgfpathlineto{\pgfqpoint{3.483316in}{3.105764in}}%
\pgfpathlineto{\pgfqpoint{3.474279in}{3.121365in}}%
\pgfpathlineto{\pgfqpoint{3.465243in}{3.137073in}}%
\pgfpathlineto{\pgfqpoint{3.456207in}{3.152864in}}%
\pgfpathlineto{\pgfqpoint{3.447171in}{3.168714in}}%
\pgfpathlineto{\pgfqpoint{3.438134in}{3.184597in}}%
\pgfpathlineto{\pgfqpoint{3.429098in}{3.200490in}}%
\pgfpathlineto{\pgfqpoint{3.420062in}{3.216367in}}%
\pgfpathlineto{\pgfqpoint{3.411026in}{3.232204in}}%
\pgfpathlineto{\pgfqpoint{3.401989in}{3.247977in}}%
\pgfpathlineto{\pgfqpoint{3.392953in}{3.263660in}}%
\pgfpathlineto{\pgfqpoint{3.383917in}{3.279229in}}%
\pgfpathlineto{\pgfqpoint{3.374881in}{3.294659in}}%
\pgfpathlineto{\pgfqpoint{3.365844in}{3.309927in}}%
\pgfpathlineto{\pgfqpoint{3.356808in}{3.325006in}}%
\pgfpathlineto{\pgfqpoint{3.347772in}{3.339873in}}%
\pgfpathlineto{\pgfqpoint{3.338736in}{3.354503in}}%
\pgfpathlineto{\pgfqpoint{3.329699in}{3.368871in}}%
\pgfpathlineto{\pgfqpoint{3.320663in}{3.382953in}}%
\pgfpathlineto{\pgfqpoint{3.320663in}{3.382953in}}%
\pgfpathclose%
\pgfusepath{stroke,fill}%
\end{pgfscope}%
\begin{pgfscope}%
\pgfpathrectangle{\pgfqpoint{0.800000in}{0.528000in}}{\pgfqpoint{4.960000in}{3.696000in}}%
\pgfusepath{clip}%
\pgfsetbuttcap%
\pgfsetroundjoin%
\definecolor{currentfill}{rgb}{0.172549,0.627451,0.172549}%
\pgfsetfillcolor{currentfill}%
\pgfsetfillopacity{0.300000}%
\pgfsetlinewidth{1.003750pt}%
\definecolor{currentstroke}{rgb}{0.172549,0.627451,0.172549}%
\pgfsetstrokecolor{currentstroke}%
\pgfsetstrokeopacity{0.300000}%
\pgfsetdash{}{0pt}%
\pgfpathmoveto{\pgfqpoint{4.974298in}{2.542341in}}%
\pgfpathlineto{\pgfqpoint{4.974298in}{2.540990in}}%
\pgfpathlineto{\pgfqpoint{4.983334in}{2.500128in}}%
\pgfpathlineto{\pgfqpoint{4.992370in}{2.459222in}}%
\pgfpathlineto{\pgfqpoint{5.001406in}{2.418294in}}%
\pgfpathlineto{\pgfqpoint{5.010443in}{2.377367in}}%
\pgfpathlineto{\pgfqpoint{5.019479in}{2.336463in}}%
\pgfpathlineto{\pgfqpoint{5.028515in}{2.295606in}}%
\pgfpathlineto{\pgfqpoint{5.037551in}{2.254819in}}%
\pgfpathlineto{\pgfqpoint{5.046588in}{2.214123in}}%
\pgfpathlineto{\pgfqpoint{5.055624in}{2.173542in}}%
\pgfpathlineto{\pgfqpoint{5.064660in}{2.133099in}}%
\pgfpathlineto{\pgfqpoint{5.073696in}{2.092817in}}%
\pgfpathlineto{\pgfqpoint{5.082733in}{2.052717in}}%
\pgfpathlineto{\pgfqpoint{5.091769in}{2.012824in}}%
\pgfpathlineto{\pgfqpoint{5.100805in}{1.973159in}}%
\pgfpathlineto{\pgfqpoint{5.109842in}{1.933746in}}%
\pgfpathlineto{\pgfqpoint{5.118878in}{1.894607in}}%
\pgfpathlineto{\pgfqpoint{5.127914in}{1.855765in}}%
\pgfpathlineto{\pgfqpoint{5.136950in}{1.817243in}}%
\pgfpathlineto{\pgfqpoint{5.145987in}{1.779063in}}%
\pgfpathlineto{\pgfqpoint{5.155023in}{1.741249in}}%
\pgfpathlineto{\pgfqpoint{5.164059in}{1.703823in}}%
\pgfpathlineto{\pgfqpoint{5.173095in}{1.666808in}}%
\pgfpathlineto{\pgfqpoint{5.182132in}{1.630227in}}%
\pgfpathlineto{\pgfqpoint{5.191168in}{1.594103in}}%
\pgfpathlineto{\pgfqpoint{5.200204in}{1.558457in}}%
\pgfpathlineto{\pgfqpoint{5.209240in}{1.523314in}}%
\pgfpathlineto{\pgfqpoint{5.218277in}{1.488695in}}%
\pgfpathlineto{\pgfqpoint{5.227313in}{1.454624in}}%
\pgfpathlineto{\pgfqpoint{5.236349in}{1.421124in}}%
\pgfpathlineto{\pgfqpoint{5.245385in}{1.388216in}}%
\pgfpathlineto{\pgfqpoint{5.254422in}{1.355925in}}%
\pgfpathlineto{\pgfqpoint{5.263458in}{1.324272in}}%
\pgfpathlineto{\pgfqpoint{5.272494in}{1.293281in}}%
\pgfpathlineto{\pgfqpoint{5.281530in}{1.262974in}}%
\pgfpathlineto{\pgfqpoint{5.290567in}{1.233374in}}%
\pgfpathlineto{\pgfqpoint{5.299603in}{1.204504in}}%
\pgfpathlineto{\pgfqpoint{5.308639in}{1.176386in}}%
\pgfpathlineto{\pgfqpoint{5.317675in}{1.149044in}}%
\pgfpathlineto{\pgfqpoint{5.326712in}{1.122500in}}%
\pgfpathlineto{\pgfqpoint{5.335748in}{1.096777in}}%
\pgfpathlineto{\pgfqpoint{5.344784in}{1.071897in}}%
\pgfpathlineto{\pgfqpoint{5.353820in}{1.047884in}}%
\pgfpathlineto{\pgfqpoint{5.362857in}{1.024760in}}%
\pgfpathlineto{\pgfqpoint{5.371893in}{1.002549in}}%
\pgfpathlineto{\pgfqpoint{5.380929in}{0.981272in}}%
\pgfpathlineto{\pgfqpoint{5.389965in}{0.960952in}}%
\pgfpathlineto{\pgfqpoint{5.399002in}{0.941613in}}%
\pgfpathlineto{\pgfqpoint{5.408038in}{0.923277in}}%
\pgfpathlineto{\pgfqpoint{5.417074in}{0.905967in}}%
\pgfpathlineto{\pgfqpoint{5.426110in}{0.889706in}}%
\pgfpathlineto{\pgfqpoint{5.435147in}{0.874516in}}%
\pgfpathlineto{\pgfqpoint{5.444183in}{0.860421in}}%
\pgfpathlineto{\pgfqpoint{5.453219in}{0.847442in}}%
\pgfpathlineto{\pgfqpoint{5.462255in}{0.835603in}}%
\pgfpathlineto{\pgfqpoint{5.471292in}{0.824927in}}%
\pgfpathlineto{\pgfqpoint{5.480328in}{0.815436in}}%
\pgfpathlineto{\pgfqpoint{5.489364in}{0.807153in}}%
\pgfpathlineto{\pgfqpoint{5.498400in}{0.800101in}}%
\pgfpathlineto{\pgfqpoint{5.507437in}{0.794303in}}%
\pgfpathlineto{\pgfqpoint{5.516473in}{0.789781in}}%
\pgfpathlineto{\pgfqpoint{5.525509in}{0.786559in}}%
\pgfpathlineto{\pgfqpoint{5.534545in}{0.784658in}}%
\pgfpathlineto{\pgfqpoint{5.534545in}{1.079931in}}%
\pgfpathlineto{\pgfqpoint{5.534545in}{1.079931in}}%
\pgfpathlineto{\pgfqpoint{5.525509in}{1.077155in}}%
\pgfpathlineto{\pgfqpoint{5.516473in}{1.075660in}}%
\pgfpathlineto{\pgfqpoint{5.507437in}{1.075427in}}%
\pgfpathlineto{\pgfqpoint{5.498400in}{1.076434in}}%
\pgfpathlineto{\pgfqpoint{5.489364in}{1.078660in}}%
\pgfpathlineto{\pgfqpoint{5.480328in}{1.082086in}}%
\pgfpathlineto{\pgfqpoint{5.471292in}{1.086689in}}%
\pgfpathlineto{\pgfqpoint{5.462255in}{1.092449in}}%
\pgfpathlineto{\pgfqpoint{5.453219in}{1.099345in}}%
\pgfpathlineto{\pgfqpoint{5.444183in}{1.107356in}}%
\pgfpathlineto{\pgfqpoint{5.435147in}{1.116461in}}%
\pgfpathlineto{\pgfqpoint{5.426110in}{1.126640in}}%
\pgfpathlineto{\pgfqpoint{5.417074in}{1.137871in}}%
\pgfpathlineto{\pgfqpoint{5.408038in}{1.150134in}}%
\pgfpathlineto{\pgfqpoint{5.399002in}{1.163407in}}%
\pgfpathlineto{\pgfqpoint{5.389965in}{1.177670in}}%
\pgfpathlineto{\pgfqpoint{5.380929in}{1.192902in}}%
\pgfpathlineto{\pgfqpoint{5.371893in}{1.209082in}}%
\pgfpathlineto{\pgfqpoint{5.362857in}{1.226189in}}%
\pgfpathlineto{\pgfqpoint{5.353820in}{1.244202in}}%
\pgfpathlineto{\pgfqpoint{5.344784in}{1.263100in}}%
\pgfpathlineto{\pgfqpoint{5.335748in}{1.282863in}}%
\pgfpathlineto{\pgfqpoint{5.326712in}{1.303469in}}%
\pgfpathlineto{\pgfqpoint{5.317675in}{1.324898in}}%
\pgfpathlineto{\pgfqpoint{5.308639in}{1.347129in}}%
\pgfpathlineto{\pgfqpoint{5.299603in}{1.370140in}}%
\pgfpathlineto{\pgfqpoint{5.290567in}{1.393911in}}%
\pgfpathlineto{\pgfqpoint{5.281530in}{1.418422in}}%
\pgfpathlineto{\pgfqpoint{5.272494in}{1.443650in}}%
\pgfpathlineto{\pgfqpoint{5.263458in}{1.469576in}}%
\pgfpathlineto{\pgfqpoint{5.254422in}{1.496178in}}%
\pgfpathlineto{\pgfqpoint{5.245385in}{1.523436in}}%
\pgfpathlineto{\pgfqpoint{5.236349in}{1.551328in}}%
\pgfpathlineto{\pgfqpoint{5.227313in}{1.579834in}}%
\pgfpathlineto{\pgfqpoint{5.218277in}{1.608933in}}%
\pgfpathlineto{\pgfqpoint{5.209240in}{1.638604in}}%
\pgfpathlineto{\pgfqpoint{5.200204in}{1.668825in}}%
\pgfpathlineto{\pgfqpoint{5.191168in}{1.699577in}}%
\pgfpathlineto{\pgfqpoint{5.182132in}{1.730838in}}%
\pgfpathlineto{\pgfqpoint{5.173095in}{1.762588in}}%
\pgfpathlineto{\pgfqpoint{5.164059in}{1.794805in}}%
\pgfpathlineto{\pgfqpoint{5.155023in}{1.827468in}}%
\pgfpathlineto{\pgfqpoint{5.145987in}{1.860557in}}%
\pgfpathlineto{\pgfqpoint{5.136950in}{1.894051in}}%
\pgfpathlineto{\pgfqpoint{5.127914in}{1.927929in}}%
\pgfpathlineto{\pgfqpoint{5.118878in}{1.962170in}}%
\pgfpathlineto{\pgfqpoint{5.109842in}{1.996752in}}%
\pgfpathlineto{\pgfqpoint{5.100805in}{2.031657in}}%
\pgfpathlineto{\pgfqpoint{5.091769in}{2.066861in}}%
\pgfpathlineto{\pgfqpoint{5.082733in}{2.102345in}}%
\pgfpathlineto{\pgfqpoint{5.073696in}{2.138087in}}%
\pgfpathlineto{\pgfqpoint{5.064660in}{2.174067in}}%
\pgfpathlineto{\pgfqpoint{5.055624in}{2.210264in}}%
\pgfpathlineto{\pgfqpoint{5.046588in}{2.246656in}}%
\pgfpathlineto{\pgfqpoint{5.037551in}{2.283224in}}%
\pgfpathlineto{\pgfqpoint{5.028515in}{2.319946in}}%
\pgfpathlineto{\pgfqpoint{5.019479in}{2.356801in}}%
\pgfpathlineto{\pgfqpoint{5.010443in}{2.393768in}}%
\pgfpathlineto{\pgfqpoint{5.001406in}{2.430826in}}%
\pgfpathlineto{\pgfqpoint{4.992370in}{2.467955in}}%
\pgfpathlineto{\pgfqpoint{4.983334in}{2.505134in}}%
\pgfpathlineto{\pgfqpoint{4.974298in}{2.542341in}}%
\pgfpathlineto{\pgfqpoint{4.974298in}{2.542341in}}%
\pgfpathclose%
\pgfusepath{stroke,fill}%
\end{pgfscope}%
\begin{pgfscope}%
\pgfpathrectangle{\pgfqpoint{0.800000in}{0.528000in}}{\pgfqpoint{4.960000in}{3.696000in}}%
\pgfusepath{clip}%
\pgfsetbuttcap%
\pgfsetroundjoin%
\definecolor{currentfill}{rgb}{0.839216,0.152941,0.156863}%
\pgfsetfillcolor{currentfill}%
\pgfsetfillopacity{0.300000}%
\pgfsetlinewidth{1.003750pt}%
\definecolor{currentstroke}{rgb}{0.839216,0.152941,0.156863}%
\pgfsetstrokecolor{currentstroke}%
\pgfsetstrokeopacity{0.300000}%
\pgfsetdash{}{0pt}%
\pgfpathmoveto{\pgfqpoint{1.594739in}{0.781043in}}%
\pgfpathlineto{\pgfqpoint{1.594739in}{0.786325in}}%
\pgfpathlineto{\pgfqpoint{1.603775in}{0.789116in}}%
\pgfpathlineto{\pgfqpoint{1.612811in}{0.792065in}}%
\pgfpathlineto{\pgfqpoint{1.621847in}{0.795174in}}%
\pgfpathlineto{\pgfqpoint{1.630884in}{0.798447in}}%
\pgfpathlineto{\pgfqpoint{1.639920in}{0.801887in}}%
\pgfpathlineto{\pgfqpoint{1.648956in}{0.805497in}}%
\pgfpathlineto{\pgfqpoint{1.657992in}{0.809280in}}%
\pgfpathlineto{\pgfqpoint{1.667029in}{0.813240in}}%
\pgfpathlineto{\pgfqpoint{1.676065in}{0.817380in}}%
\pgfpathlineto{\pgfqpoint{1.685101in}{0.821702in}}%
\pgfpathlineto{\pgfqpoint{1.694137in}{0.826212in}}%
\pgfpathlineto{\pgfqpoint{1.703174in}{0.830911in}}%
\pgfpathlineto{\pgfqpoint{1.712210in}{0.835803in}}%
\pgfpathlineto{\pgfqpoint{1.721246in}{0.840891in}}%
\pgfpathlineto{\pgfqpoint{1.730282in}{0.846179in}}%
\pgfpathlineto{\pgfqpoint{1.739319in}{0.851670in}}%
\pgfpathlineto{\pgfqpoint{1.748355in}{0.857366in}}%
\pgfpathlineto{\pgfqpoint{1.757391in}{0.863272in}}%
\pgfpathlineto{\pgfqpoint{1.766427in}{0.869390in}}%
\pgfpathlineto{\pgfqpoint{1.775464in}{0.875724in}}%
\pgfpathlineto{\pgfqpoint{1.784500in}{0.882278in}}%
\pgfpathlineto{\pgfqpoint{1.793536in}{0.889053in}}%
\pgfpathlineto{\pgfqpoint{1.802572in}{0.896054in}}%
\pgfpathlineto{\pgfqpoint{1.811609in}{0.903284in}}%
\pgfpathlineto{\pgfqpoint{1.820645in}{0.910746in}}%
\pgfpathlineto{\pgfqpoint{1.829681in}{0.918444in}}%
\pgfpathlineto{\pgfqpoint{1.838717in}{0.926380in}}%
\pgfpathlineto{\pgfqpoint{1.847754in}{0.934558in}}%
\pgfpathlineto{\pgfqpoint{1.856790in}{0.942981in}}%
\pgfpathlineto{\pgfqpoint{1.865826in}{0.951652in}}%
\pgfpathlineto{\pgfqpoint{1.874862in}{0.960576in}}%
\pgfpathlineto{\pgfqpoint{1.883899in}{0.969754in}}%
\pgfpathlineto{\pgfqpoint{1.892935in}{0.979190in}}%
\pgfpathlineto{\pgfqpoint{1.901971in}{0.988888in}}%
\pgfpathlineto{\pgfqpoint{1.911007in}{0.998851in}}%
\pgfpathlineto{\pgfqpoint{1.920044in}{1.009081in}}%
\pgfpathlineto{\pgfqpoint{1.929080in}{1.019583in}}%
\pgfpathlineto{\pgfqpoint{1.938116in}{1.030359in}}%
\pgfpathlineto{\pgfqpoint{1.947152in}{1.041413in}}%
\pgfpathlineto{\pgfqpoint{1.956189in}{1.052748in}}%
\pgfpathlineto{\pgfqpoint{1.965225in}{1.064368in}}%
\pgfpathlineto{\pgfqpoint{1.974261in}{1.076275in}}%
\pgfpathlineto{\pgfqpoint{1.983298in}{1.088472in}}%
\pgfpathlineto{\pgfqpoint{1.992334in}{1.100964in}}%
\pgfpathlineto{\pgfqpoint{2.001370in}{1.113753in}}%
\pgfpathlineto{\pgfqpoint{2.010406in}{1.126843in}}%
\pgfpathlineto{\pgfqpoint{2.019443in}{1.140237in}}%
\pgfpathlineto{\pgfqpoint{2.028479in}{1.153938in}}%
\pgfpathlineto{\pgfqpoint{2.037515in}{1.167949in}}%
\pgfpathlineto{\pgfqpoint{2.046551in}{1.182274in}}%
\pgfpathlineto{\pgfqpoint{2.055588in}{1.196915in}}%
\pgfpathlineto{\pgfqpoint{2.064624in}{1.211877in}}%
\pgfpathlineto{\pgfqpoint{2.073660in}{1.227163in}}%
\pgfpathlineto{\pgfqpoint{2.082696in}{1.242775in}}%
\pgfpathlineto{\pgfqpoint{2.091733in}{1.258717in}}%
\pgfpathlineto{\pgfqpoint{2.100769in}{1.274992in}}%
\pgfpathlineto{\pgfqpoint{2.109805in}{1.291604in}}%
\pgfpathlineto{\pgfqpoint{2.118841in}{1.308555in}}%
\pgfpathlineto{\pgfqpoint{2.127878in}{1.325850in}}%
\pgfpathlineto{\pgfqpoint{2.136914in}{1.343490in}}%
\pgfpathlineto{\pgfqpoint{2.145950in}{1.361481in}}%
\pgfpathlineto{\pgfqpoint{2.154986in}{1.379824in}}%
\pgfpathlineto{\pgfqpoint{2.164023in}{1.398519in}}%
\pgfpathlineto{\pgfqpoint{2.173059in}{1.417558in}}%
\pgfpathlineto{\pgfqpoint{2.182095in}{1.436932in}}%
\pgfpathlineto{\pgfqpoint{2.191131in}{1.456632in}}%
\pgfpathlineto{\pgfqpoint{2.200168in}{1.476649in}}%
\pgfpathlineto{\pgfqpoint{2.209204in}{1.496974in}}%
\pgfpathlineto{\pgfqpoint{2.218240in}{1.517598in}}%
\pgfpathlineto{\pgfqpoint{2.227276in}{1.538512in}}%
\pgfpathlineto{\pgfqpoint{2.236313in}{1.559706in}}%
\pgfpathlineto{\pgfqpoint{2.245349in}{1.581173in}}%
\pgfpathlineto{\pgfqpoint{2.254385in}{1.602903in}}%
\pgfpathlineto{\pgfqpoint{2.263421in}{1.624886in}}%
\pgfpathlineto{\pgfqpoint{2.272458in}{1.647114in}}%
\pgfpathlineto{\pgfqpoint{2.281494in}{1.669579in}}%
\pgfpathlineto{\pgfqpoint{2.290530in}{1.692270in}}%
\pgfpathlineto{\pgfqpoint{2.299566in}{1.715179in}}%
\pgfpathlineto{\pgfqpoint{2.308603in}{1.738296in}}%
\pgfpathlineto{\pgfqpoint{2.317639in}{1.761614in}}%
\pgfpathlineto{\pgfqpoint{2.326675in}{1.785122in}}%
\pgfpathlineto{\pgfqpoint{2.335711in}{1.808813in}}%
\pgfpathlineto{\pgfqpoint{2.344748in}{1.832676in}}%
\pgfpathlineto{\pgfqpoint{2.353784in}{1.856703in}}%
\pgfpathlineto{\pgfqpoint{2.362820in}{1.880884in}}%
\pgfpathlineto{\pgfqpoint{2.371856in}{1.905212in}}%
\pgfpathlineto{\pgfqpoint{2.380893in}{1.929676in}}%
\pgfpathlineto{\pgfqpoint{2.389929in}{1.954268in}}%
\pgfpathlineto{\pgfqpoint{2.398965in}{1.978979in}}%
\pgfpathlineto{\pgfqpoint{2.408001in}{2.003800in}}%
\pgfpathlineto{\pgfqpoint{2.417038in}{2.028721in}}%
\pgfpathlineto{\pgfqpoint{2.417038in}{2.024123in}}%
\pgfpathlineto{\pgfqpoint{2.417038in}{2.024123in}}%
\pgfpathlineto{\pgfqpoint{2.408001in}{1.986848in}}%
\pgfpathlineto{\pgfqpoint{2.398965in}{1.949685in}}%
\pgfpathlineto{\pgfqpoint{2.389929in}{1.912662in}}%
\pgfpathlineto{\pgfqpoint{2.380893in}{1.875804in}}%
\pgfpathlineto{\pgfqpoint{2.371856in}{1.839138in}}%
\pgfpathlineto{\pgfqpoint{2.362820in}{1.802691in}}%
\pgfpathlineto{\pgfqpoint{2.353784in}{1.766489in}}%
\pgfpathlineto{\pgfqpoint{2.344748in}{1.730559in}}%
\pgfpathlineto{\pgfqpoint{2.335711in}{1.694927in}}%
\pgfpathlineto{\pgfqpoint{2.326675in}{1.659621in}}%
\pgfpathlineto{\pgfqpoint{2.317639in}{1.624665in}}%
\pgfpathlineto{\pgfqpoint{2.308603in}{1.590088in}}%
\pgfpathlineto{\pgfqpoint{2.299566in}{1.555915in}}%
\pgfpathlineto{\pgfqpoint{2.290530in}{1.522173in}}%
\pgfpathlineto{\pgfqpoint{2.281494in}{1.488889in}}%
\pgfpathlineto{\pgfqpoint{2.272458in}{1.456089in}}%
\pgfpathlineto{\pgfqpoint{2.263421in}{1.423800in}}%
\pgfpathlineto{\pgfqpoint{2.254385in}{1.392047in}}%
\pgfpathlineto{\pgfqpoint{2.245349in}{1.360859in}}%
\pgfpathlineto{\pgfqpoint{2.236313in}{1.330261in}}%
\pgfpathlineto{\pgfqpoint{2.227276in}{1.300280in}}%
\pgfpathlineto{\pgfqpoint{2.218240in}{1.270942in}}%
\pgfpathlineto{\pgfqpoint{2.209204in}{1.242274in}}%
\pgfpathlineto{\pgfqpoint{2.200168in}{1.214303in}}%
\pgfpathlineto{\pgfqpoint{2.191131in}{1.187054in}}%
\pgfpathlineto{\pgfqpoint{2.182095in}{1.160555in}}%
\pgfpathlineto{\pgfqpoint{2.173059in}{1.134832in}}%
\pgfpathlineto{\pgfqpoint{2.164023in}{1.109912in}}%
\pgfpathlineto{\pgfqpoint{2.154986in}{1.085821in}}%
\pgfpathlineto{\pgfqpoint{2.145950in}{1.062583in}}%
\pgfpathlineto{\pgfqpoint{2.136914in}{1.040198in}}%
\pgfpathlineto{\pgfqpoint{2.127878in}{1.018653in}}%
\pgfpathlineto{\pgfqpoint{2.118841in}{0.997936in}}%
\pgfpathlineto{\pgfqpoint{2.109805in}{0.978034in}}%
\pgfpathlineto{\pgfqpoint{2.100769in}{0.958935in}}%
\pgfpathlineto{\pgfqpoint{2.091733in}{0.940625in}}%
\pgfpathlineto{\pgfqpoint{2.082696in}{0.923092in}}%
\pgfpathlineto{\pgfqpoint{2.073660in}{0.906324in}}%
\pgfpathlineto{\pgfqpoint{2.064624in}{0.890309in}}%
\pgfpathlineto{\pgfqpoint{2.055588in}{0.875032in}}%
\pgfpathlineto{\pgfqpoint{2.046551in}{0.860483in}}%
\pgfpathlineto{\pgfqpoint{2.037515in}{0.846647in}}%
\pgfpathlineto{\pgfqpoint{2.028479in}{0.833513in}}%
\pgfpathlineto{\pgfqpoint{2.019443in}{0.821069in}}%
\pgfpathlineto{\pgfqpoint{2.010406in}{0.809300in}}%
\pgfpathlineto{\pgfqpoint{2.001370in}{0.798196in}}%
\pgfpathlineto{\pgfqpoint{1.992334in}{0.787742in}}%
\pgfpathlineto{\pgfqpoint{1.983298in}{0.777927in}}%
\pgfpathlineto{\pgfqpoint{1.974261in}{0.768738in}}%
\pgfpathlineto{\pgfqpoint{1.965225in}{0.760163in}}%
\pgfpathlineto{\pgfqpoint{1.956189in}{0.752188in}}%
\pgfpathlineto{\pgfqpoint{1.947152in}{0.744801in}}%
\pgfpathlineto{\pgfqpoint{1.938116in}{0.737990in}}%
\pgfpathlineto{\pgfqpoint{1.929080in}{0.731741in}}%
\pgfpathlineto{\pgfqpoint{1.920044in}{0.726043in}}%
\pgfpathlineto{\pgfqpoint{1.911007in}{0.720883in}}%
\pgfpathlineto{\pgfqpoint{1.901971in}{0.716248in}}%
\pgfpathlineto{\pgfqpoint{1.892935in}{0.712125in}}%
\pgfpathlineto{\pgfqpoint{1.883899in}{0.708502in}}%
\pgfpathlineto{\pgfqpoint{1.874862in}{0.705366in}}%
\pgfpathlineto{\pgfqpoint{1.865826in}{0.702705in}}%
\pgfpathlineto{\pgfqpoint{1.856790in}{0.700506in}}%
\pgfpathlineto{\pgfqpoint{1.847754in}{0.698756in}}%
\pgfpathlineto{\pgfqpoint{1.838717in}{0.697444in}}%
\pgfpathlineto{\pgfqpoint{1.829681in}{0.696555in}}%
\pgfpathlineto{\pgfqpoint{1.820645in}{0.696078in}}%
\pgfpathlineto{\pgfqpoint{1.811609in}{0.696000in}}%
\pgfpathlineto{\pgfqpoint{1.802572in}{0.696308in}}%
\pgfpathlineto{\pgfqpoint{1.793536in}{0.696991in}}%
\pgfpathlineto{\pgfqpoint{1.784500in}{0.698034in}}%
\pgfpathlineto{\pgfqpoint{1.775464in}{0.699426in}}%
\pgfpathlineto{\pgfqpoint{1.766427in}{0.701154in}}%
\pgfpathlineto{\pgfqpoint{1.757391in}{0.703205in}}%
\pgfpathlineto{\pgfqpoint{1.748355in}{0.705567in}}%
\pgfpathlineto{\pgfqpoint{1.739319in}{0.708227in}}%
\pgfpathlineto{\pgfqpoint{1.730282in}{0.711172in}}%
\pgfpathlineto{\pgfqpoint{1.721246in}{0.714391in}}%
\pgfpathlineto{\pgfqpoint{1.712210in}{0.717869in}}%
\pgfpathlineto{\pgfqpoint{1.703174in}{0.721596in}}%
\pgfpathlineto{\pgfqpoint{1.694137in}{0.725557in}}%
\pgfpathlineto{\pgfqpoint{1.685101in}{0.729741in}}%
\pgfpathlineto{\pgfqpoint{1.676065in}{0.734135in}}%
\pgfpathlineto{\pgfqpoint{1.667029in}{0.738726in}}%
\pgfpathlineto{\pgfqpoint{1.657992in}{0.743502in}}%
\pgfpathlineto{\pgfqpoint{1.648956in}{0.748450in}}%
\pgfpathlineto{\pgfqpoint{1.639920in}{0.753557in}}%
\pgfpathlineto{\pgfqpoint{1.630884in}{0.758811in}}%
\pgfpathlineto{\pgfqpoint{1.621847in}{0.764199in}}%
\pgfpathlineto{\pgfqpoint{1.612811in}{0.769709in}}%
\pgfpathlineto{\pgfqpoint{1.603775in}{0.775328in}}%
\pgfpathlineto{\pgfqpoint{1.594739in}{0.781043in}}%
\pgfpathlineto{\pgfqpoint{1.594739in}{0.781043in}}%
\pgfpathclose%
\pgfusepath{stroke,fill}%
\end{pgfscope}%
\begin{pgfscope}%
\pgfpathrectangle{\pgfqpoint{0.800000in}{0.528000in}}{\pgfqpoint{4.960000in}{3.696000in}}%
\pgfusepath{clip}%
\pgfsetbuttcap%
\pgfsetroundjoin%
\definecolor{currentfill}{rgb}{0.839216,0.152941,0.156863}%
\pgfsetfillcolor{currentfill}%
\pgfsetfillopacity{0.300000}%
\pgfsetlinewidth{1.003750pt}%
\definecolor{currentstroke}{rgb}{0.839216,0.152941,0.156863}%
\pgfsetstrokecolor{currentstroke}%
\pgfsetstrokeopacity{0.300000}%
\pgfsetdash{}{0pt}%
\pgfpathmoveto{\pgfqpoint{3.284518in}{3.435924in}}%
\pgfpathlineto{\pgfqpoint{3.284518in}{3.436365in}}%
\pgfpathlineto{\pgfqpoint{3.293554in}{3.424209in}}%
\pgfpathlineto{\pgfqpoint{3.302591in}{3.411216in}}%
\pgfpathlineto{\pgfqpoint{3.311627in}{3.397424in}}%
\pgfpathlineto{\pgfqpoint{3.311627in}{3.396723in}}%
\pgfpathlineto{\pgfqpoint{3.311627in}{3.396723in}}%
\pgfpathlineto{\pgfqpoint{3.302591in}{3.410159in}}%
\pgfpathlineto{\pgfqpoint{3.293554in}{3.423234in}}%
\pgfpathlineto{\pgfqpoint{3.284518in}{3.435924in}}%
\pgfpathlineto{\pgfqpoint{3.284518in}{3.435924in}}%
\pgfpathclose%
\pgfusepath{stroke,fill}%
\end{pgfscope}%
\begin{pgfscope}%
\pgfpathrectangle{\pgfqpoint{0.800000in}{0.528000in}}{\pgfqpoint{4.960000in}{3.696000in}}%
\pgfusepath{clip}%
\pgfsetbuttcap%
\pgfsetroundjoin%
\definecolor{currentfill}{rgb}{0.839216,0.152941,0.156863}%
\pgfsetfillcolor{currentfill}%
\pgfsetfillopacity{0.300000}%
\pgfsetlinewidth{1.003750pt}%
\definecolor{currentstroke}{rgb}{0.839216,0.152941,0.156863}%
\pgfsetstrokecolor{currentstroke}%
\pgfsetstrokeopacity{0.300000}%
\pgfsetdash{}{0pt}%
\pgfpathmoveto{\pgfqpoint{4.414050in}{4.034867in}}%
\pgfpathlineto{\pgfqpoint{4.414050in}{4.038421in}}%
\pgfpathlineto{\pgfqpoint{4.423086in}{4.044804in}}%
\pgfpathlineto{\pgfqpoint{4.432122in}{4.049731in}}%
\pgfpathlineto{\pgfqpoint{4.441159in}{4.053224in}}%
\pgfpathlineto{\pgfqpoint{4.450195in}{4.055306in}}%
\pgfpathlineto{\pgfqpoint{4.459231in}{4.056000in}}%
\pgfpathlineto{\pgfqpoint{4.468267in}{4.055328in}}%
\pgfpathlineto{\pgfqpoint{4.477304in}{4.053313in}}%
\pgfpathlineto{\pgfqpoint{4.486340in}{4.049978in}}%
\pgfpathlineto{\pgfqpoint{4.495376in}{4.045346in}}%
\pgfpathlineto{\pgfqpoint{4.504412in}{4.039440in}}%
\pgfpathlineto{\pgfqpoint{4.513449in}{4.032281in}}%
\pgfpathlineto{\pgfqpoint{4.522485in}{4.023894in}}%
\pgfpathlineto{\pgfqpoint{4.531521in}{4.014301in}}%
\pgfpathlineto{\pgfqpoint{4.540557in}{4.003524in}}%
\pgfpathlineto{\pgfqpoint{4.549594in}{3.991586in}}%
\pgfpathlineto{\pgfqpoint{4.558630in}{3.978511in}}%
\pgfpathlineto{\pgfqpoint{4.567666in}{3.964321in}}%
\pgfpathlineto{\pgfqpoint{4.576702in}{3.949038in}}%
\pgfpathlineto{\pgfqpoint{4.585739in}{3.932686in}}%
\pgfpathlineto{\pgfqpoint{4.594775in}{3.915287in}}%
\pgfpathlineto{\pgfqpoint{4.603811in}{3.896864in}}%
\pgfpathlineto{\pgfqpoint{4.612848in}{3.877441in}}%
\pgfpathlineto{\pgfqpoint{4.621884in}{3.857038in}}%
\pgfpathlineto{\pgfqpoint{4.630920in}{3.835680in}}%
\pgfpathlineto{\pgfqpoint{4.639956in}{3.813389in}}%
\pgfpathlineto{\pgfqpoint{4.648993in}{3.790189in}}%
\pgfpathlineto{\pgfqpoint{4.658029in}{3.766100in}}%
\pgfpathlineto{\pgfqpoint{4.667065in}{3.741148in}}%
\pgfpathlineto{\pgfqpoint{4.676101in}{3.715353in}}%
\pgfpathlineto{\pgfqpoint{4.685138in}{3.688740in}}%
\pgfpathlineto{\pgfqpoint{4.694174in}{3.661330in}}%
\pgfpathlineto{\pgfqpoint{4.703210in}{3.633147in}}%
\pgfpathlineto{\pgfqpoint{4.712246in}{3.604214in}}%
\pgfpathlineto{\pgfqpoint{4.721283in}{3.574552in}}%
\pgfpathlineto{\pgfqpoint{4.730319in}{3.544186in}}%
\pgfpathlineto{\pgfqpoint{4.739355in}{3.513137in}}%
\pgfpathlineto{\pgfqpoint{4.748391in}{3.481428in}}%
\pgfpathlineto{\pgfqpoint{4.757428in}{3.449083in}}%
\pgfpathlineto{\pgfqpoint{4.766464in}{3.416124in}}%
\pgfpathlineto{\pgfqpoint{4.775500in}{3.382574in}}%
\pgfpathlineto{\pgfqpoint{4.784536in}{3.348455in}}%
\pgfpathlineto{\pgfqpoint{4.793573in}{3.313790in}}%
\pgfpathlineto{\pgfqpoint{4.802609in}{3.278603in}}%
\pgfpathlineto{\pgfqpoint{4.811645in}{3.242916in}}%
\pgfpathlineto{\pgfqpoint{4.820681in}{3.206751in}}%
\pgfpathlineto{\pgfqpoint{4.829718in}{3.170132in}}%
\pgfpathlineto{\pgfqpoint{4.838754in}{3.133081in}}%
\pgfpathlineto{\pgfqpoint{4.847790in}{3.095621in}}%
\pgfpathlineto{\pgfqpoint{4.856826in}{3.057774in}}%
\pgfpathlineto{\pgfqpoint{4.865863in}{3.019565in}}%
\pgfpathlineto{\pgfqpoint{4.874899in}{2.981014in}}%
\pgfpathlineto{\pgfqpoint{4.883935in}{2.942146in}}%
\pgfpathlineto{\pgfqpoint{4.892971in}{2.902983in}}%
\pgfpathlineto{\pgfqpoint{4.902008in}{2.863547in}}%
\pgfpathlineto{\pgfqpoint{4.911044in}{2.823862in}}%
\pgfpathlineto{\pgfqpoint{4.920080in}{2.783950in}}%
\pgfpathlineto{\pgfqpoint{4.929116in}{2.743834in}}%
\pgfpathlineto{\pgfqpoint{4.938153in}{2.703536in}}%
\pgfpathlineto{\pgfqpoint{4.947189in}{2.663080in}}%
\pgfpathlineto{\pgfqpoint{4.956225in}{2.622489in}}%
\pgfpathlineto{\pgfqpoint{4.965261in}{2.581785in}}%
\pgfpathlineto{\pgfqpoint{4.965261in}{2.579557in}}%
\pgfpathlineto{\pgfqpoint{4.965261in}{2.579557in}}%
\pgfpathlineto{\pgfqpoint{4.956225in}{2.616759in}}%
\pgfpathlineto{\pgfqpoint{4.947189in}{2.653927in}}%
\pgfpathlineto{\pgfqpoint{4.938153in}{2.691040in}}%
\pgfpathlineto{\pgfqpoint{4.929116in}{2.728078in}}%
\pgfpathlineto{\pgfqpoint{4.920080in}{2.765019in}}%
\pgfpathlineto{\pgfqpoint{4.911044in}{2.801843in}}%
\pgfpathlineto{\pgfqpoint{4.902008in}{2.838528in}}%
\pgfpathlineto{\pgfqpoint{4.892971in}{2.875053in}}%
\pgfpathlineto{\pgfqpoint{4.883935in}{2.911399in}}%
\pgfpathlineto{\pgfqpoint{4.874899in}{2.947544in}}%
\pgfpathlineto{\pgfqpoint{4.865863in}{2.983466in}}%
\pgfpathlineto{\pgfqpoint{4.856826in}{3.019146in}}%
\pgfpathlineto{\pgfqpoint{4.847790in}{3.054562in}}%
\pgfpathlineto{\pgfqpoint{4.838754in}{3.089693in}}%
\pgfpathlineto{\pgfqpoint{4.829718in}{3.124519in}}%
\pgfpathlineto{\pgfqpoint{4.820681in}{3.159018in}}%
\pgfpathlineto{\pgfqpoint{4.811645in}{3.193170in}}%
\pgfpathlineto{\pgfqpoint{4.802609in}{3.226954in}}%
\pgfpathlineto{\pgfqpoint{4.793573in}{3.260348in}}%
\pgfpathlineto{\pgfqpoint{4.784536in}{3.293333in}}%
\pgfpathlineto{\pgfqpoint{4.775500in}{3.325887in}}%
\pgfpathlineto{\pgfqpoint{4.766464in}{3.357989in}}%
\pgfpathlineto{\pgfqpoint{4.757428in}{3.389618in}}%
\pgfpathlineto{\pgfqpoint{4.748391in}{3.420754in}}%
\pgfpathlineto{\pgfqpoint{4.739355in}{3.451375in}}%
\pgfpathlineto{\pgfqpoint{4.730319in}{3.481461in}}%
\pgfpathlineto{\pgfqpoint{4.721283in}{3.510991in}}%
\pgfpathlineto{\pgfqpoint{4.712246in}{3.539943in}}%
\pgfpathlineto{\pgfqpoint{4.703210in}{3.568298in}}%
\pgfpathlineto{\pgfqpoint{4.694174in}{3.596033in}}%
\pgfpathlineto{\pgfqpoint{4.685138in}{3.623129in}}%
\pgfpathlineto{\pgfqpoint{4.676101in}{3.649564in}}%
\pgfpathlineto{\pgfqpoint{4.667065in}{3.675318in}}%
\pgfpathlineto{\pgfqpoint{4.658029in}{3.700369in}}%
\pgfpathlineto{\pgfqpoint{4.648993in}{3.724696in}}%
\pgfpathlineto{\pgfqpoint{4.639956in}{3.748280in}}%
\pgfpathlineto{\pgfqpoint{4.630920in}{3.771098in}}%
\pgfpathlineto{\pgfqpoint{4.621884in}{3.793130in}}%
\pgfpathlineto{\pgfqpoint{4.612848in}{3.814355in}}%
\pgfpathlineto{\pgfqpoint{4.603811in}{3.834752in}}%
\pgfpathlineto{\pgfqpoint{4.594775in}{3.854301in}}%
\pgfpathlineto{\pgfqpoint{4.585739in}{3.872980in}}%
\pgfpathlineto{\pgfqpoint{4.576702in}{3.890768in}}%
\pgfpathlineto{\pgfqpoint{4.567666in}{3.907646in}}%
\pgfpathlineto{\pgfqpoint{4.558630in}{3.923590in}}%
\pgfpathlineto{\pgfqpoint{4.549594in}{3.938582in}}%
\pgfpathlineto{\pgfqpoint{4.540557in}{3.952599in}}%
\pgfpathlineto{\pgfqpoint{4.531521in}{3.965622in}}%
\pgfpathlineto{\pgfqpoint{4.522485in}{3.977629in}}%
\pgfpathlineto{\pgfqpoint{4.513449in}{3.988599in}}%
\pgfpathlineto{\pgfqpoint{4.504412in}{3.998511in}}%
\pgfpathlineto{\pgfqpoint{4.495376in}{4.007345in}}%
\pgfpathlineto{\pgfqpoint{4.486340in}{4.015079in}}%
\pgfpathlineto{\pgfqpoint{4.477304in}{4.021693in}}%
\pgfpathlineto{\pgfqpoint{4.468267in}{4.027166in}}%
\pgfpathlineto{\pgfqpoint{4.459231in}{4.031476in}}%
\pgfpathlineto{\pgfqpoint{4.450195in}{4.034604in}}%
\pgfpathlineto{\pgfqpoint{4.441159in}{4.036528in}}%
\pgfpathlineto{\pgfqpoint{4.432122in}{4.037227in}}%
\pgfpathlineto{\pgfqpoint{4.423086in}{4.036680in}}%
\pgfpathlineto{\pgfqpoint{4.414050in}{4.034867in}}%
\pgfpathlineto{\pgfqpoint{4.414050in}{4.034867in}}%
\pgfpathclose%
\pgfusepath{stroke,fill}%
\end{pgfscope}%
\begin{pgfscope}%
\pgfpathrectangle{\pgfqpoint{0.800000in}{0.528000in}}{\pgfqpoint{4.960000in}{3.696000in}}%
\pgfusepath{clip}%
\pgfsetroundcap%
\pgfsetroundjoin%
\pgfsetlinewidth{1.505625pt}%
\definecolor{currentstroke}{rgb}{0.298039,0.447059,0.690196}%
\pgfsetstrokecolor{currentstroke}%
\pgfsetdash{}{0pt}%
\pgfpathmoveto{\pgfqpoint{1.025455in}{0.784658in}}%
\pgfpathlineto{\pgfqpoint{1.061600in}{0.782265in}}%
\pgfpathlineto{\pgfqpoint{1.097745in}{0.779253in}}%
\pgfpathlineto{\pgfqpoint{1.151962in}{0.774034in}}%
\pgfpathlineto{\pgfqpoint{1.242325in}{0.765204in}}%
\pgfpathlineto{\pgfqpoint{1.278470in}{0.762234in}}%
\pgfpathlineto{\pgfqpoint{1.314615in}{0.759900in}}%
\pgfpathlineto{\pgfqpoint{1.341723in}{0.758690in}}%
\pgfpathlineto{\pgfqpoint{1.368832in}{0.758044in}}%
\pgfpathlineto{\pgfqpoint{1.395941in}{0.758048in}}%
\pgfpathlineto{\pgfqpoint{1.423050in}{0.758790in}}%
\pgfpathlineto{\pgfqpoint{1.441122in}{0.759739in}}%
\pgfpathlineto{\pgfqpoint{1.459195in}{0.761081in}}%
\pgfpathlineto{\pgfqpoint{1.477267in}{0.762843in}}%
\pgfpathlineto{\pgfqpoint{1.495340in}{0.765050in}}%
\pgfpathlineto{\pgfqpoint{1.513412in}{0.767730in}}%
\pgfpathlineto{\pgfqpoint{1.531485in}{0.770907in}}%
\pgfpathlineto{\pgfqpoint{1.549557in}{0.774608in}}%
\pgfpathlineto{\pgfqpoint{1.567630in}{0.778859in}}%
\pgfpathlineto{\pgfqpoint{1.585702in}{0.783687in}}%
\pgfpathlineto{\pgfqpoint{1.603775in}{0.789116in}}%
\pgfpathlineto{\pgfqpoint{1.621847in}{0.795174in}}%
\pgfpathlineto{\pgfqpoint{1.639920in}{0.801887in}}%
\pgfpathlineto{\pgfqpoint{1.657992in}{0.809280in}}%
\pgfpathlineto{\pgfqpoint{1.676065in}{0.817380in}}%
\pgfpathlineto{\pgfqpoint{1.694137in}{0.826212in}}%
\pgfpathlineto{\pgfqpoint{1.712210in}{0.835803in}}%
\pgfpathlineto{\pgfqpoint{1.730282in}{0.846179in}}%
\pgfpathlineto{\pgfqpoint{1.748355in}{0.857366in}}%
\pgfpathlineto{\pgfqpoint{1.766427in}{0.869390in}}%
\pgfpathlineto{\pgfqpoint{1.784500in}{0.882278in}}%
\pgfpathlineto{\pgfqpoint{1.802572in}{0.896054in}}%
\pgfpathlineto{\pgfqpoint{1.820645in}{0.910746in}}%
\pgfpathlineto{\pgfqpoint{1.838717in}{0.926380in}}%
\pgfpathlineto{\pgfqpoint{1.856790in}{0.942981in}}%
\pgfpathlineto{\pgfqpoint{1.874862in}{0.960576in}}%
\pgfpathlineto{\pgfqpoint{1.892935in}{0.979190in}}%
\pgfpathlineto{\pgfqpoint{1.911007in}{0.998851in}}%
\pgfpathlineto{\pgfqpoint{1.929080in}{1.019583in}}%
\pgfpathlineto{\pgfqpoint{1.947152in}{1.041413in}}%
\pgfpathlineto{\pgfqpoint{1.965225in}{1.064368in}}%
\pgfpathlineto{\pgfqpoint{1.983298in}{1.088472in}}%
\pgfpathlineto{\pgfqpoint{2.001370in}{1.113753in}}%
\pgfpathlineto{\pgfqpoint{2.019443in}{1.140237in}}%
\pgfpathlineto{\pgfqpoint{2.037515in}{1.167949in}}%
\pgfpathlineto{\pgfqpoint{2.055588in}{1.196915in}}%
\pgfpathlineto{\pgfqpoint{2.073660in}{1.227163in}}%
\pgfpathlineto{\pgfqpoint{2.091733in}{1.258717in}}%
\pgfpathlineto{\pgfqpoint{2.109805in}{1.291604in}}%
\pgfpathlineto{\pgfqpoint{2.127878in}{1.325850in}}%
\pgfpathlineto{\pgfqpoint{2.145950in}{1.361481in}}%
\pgfpathlineto{\pgfqpoint{2.164023in}{1.398519in}}%
\pgfpathlineto{\pgfqpoint{2.182095in}{1.436932in}}%
\pgfpathlineto{\pgfqpoint{2.200168in}{1.476649in}}%
\pgfpathlineto{\pgfqpoint{2.218240in}{1.517598in}}%
\pgfpathlineto{\pgfqpoint{2.236313in}{1.559706in}}%
\pgfpathlineto{\pgfqpoint{2.254385in}{1.602903in}}%
\pgfpathlineto{\pgfqpoint{2.281494in}{1.669579in}}%
\pgfpathlineto{\pgfqpoint{2.308603in}{1.738296in}}%
\pgfpathlineto{\pgfqpoint{2.335711in}{1.808813in}}%
\pgfpathlineto{\pgfqpoint{2.362820in}{1.880884in}}%
\pgfpathlineto{\pgfqpoint{2.398965in}{1.978979in}}%
\pgfpathlineto{\pgfqpoint{2.435110in}{2.078830in}}%
\pgfpathlineto{\pgfqpoint{2.489328in}{2.230638in}}%
\pgfpathlineto{\pgfqpoint{2.579690in}{2.484264in}}%
\pgfpathlineto{\pgfqpoint{2.615835in}{2.584247in}}%
\pgfpathlineto{\pgfqpoint{2.651980in}{2.682527in}}%
\pgfpathlineto{\pgfqpoint{2.679089in}{2.754773in}}%
\pgfpathlineto{\pgfqpoint{2.706198in}{2.825493in}}%
\pgfpathlineto{\pgfqpoint{2.733307in}{2.894443in}}%
\pgfpathlineto{\pgfqpoint{2.760415in}{2.961345in}}%
\pgfpathlineto{\pgfqpoint{2.778488in}{3.004663in}}%
\pgfpathlineto{\pgfqpoint{2.796560in}{3.046851in}}%
\pgfpathlineto{\pgfqpoint{2.814633in}{3.087821in}}%
\pgfpathlineto{\pgfqpoint{2.832705in}{3.127484in}}%
\pgfpathlineto{\pgfqpoint{2.850778in}{3.165753in}}%
\pgfpathlineto{\pgfqpoint{2.868850in}{3.202539in}}%
\pgfpathlineto{\pgfqpoint{2.886923in}{3.237753in}}%
\pgfpathlineto{\pgfqpoint{2.904995in}{3.271308in}}%
\pgfpathlineto{\pgfqpoint{2.923068in}{3.303115in}}%
\pgfpathlineto{\pgfqpoint{2.941140in}{3.333087in}}%
\pgfpathlineto{\pgfqpoint{2.959213in}{3.361134in}}%
\pgfpathlineto{\pgfqpoint{2.968249in}{3.374409in}}%
\pgfpathlineto{\pgfqpoint{2.977285in}{3.387169in}}%
\pgfpathlineto{\pgfqpoint{2.986322in}{3.399405in}}%
\pgfpathlineto{\pgfqpoint{2.995358in}{3.411104in}}%
\pgfpathlineto{\pgfqpoint{3.004394in}{3.422256in}}%
\pgfpathlineto{\pgfqpoint{3.013430in}{3.432849in}}%
\pgfpathlineto{\pgfqpoint{3.022467in}{3.442873in}}%
\pgfpathlineto{\pgfqpoint{3.031503in}{3.452317in}}%
\pgfpathlineto{\pgfqpoint{3.040539in}{3.461170in}}%
\pgfpathlineto{\pgfqpoint{3.049576in}{3.469421in}}%
\pgfpathlineto{\pgfqpoint{3.058612in}{3.477057in}}%
\pgfpathlineto{\pgfqpoint{3.067648in}{3.484070in}}%
\pgfpathlineto{\pgfqpoint{3.076684in}{3.490447in}}%
\pgfpathlineto{\pgfqpoint{3.085721in}{3.496178in}}%
\pgfpathlineto{\pgfqpoint{3.094757in}{3.501251in}}%
\pgfpathlineto{\pgfqpoint{3.103793in}{3.505655in}}%
\pgfpathlineto{\pgfqpoint{3.112829in}{3.509380in}}%
\pgfpathlineto{\pgfqpoint{3.121866in}{3.512414in}}%
\pgfpathlineto{\pgfqpoint{3.130902in}{3.514747in}}%
\pgfpathlineto{\pgfqpoint{3.139938in}{3.516367in}}%
\pgfpathlineto{\pgfqpoint{3.148974in}{3.517263in}}%
\pgfpathlineto{\pgfqpoint{3.158011in}{3.517425in}}%
\pgfpathlineto{\pgfqpoint{3.167047in}{3.516841in}}%
\pgfpathlineto{\pgfqpoint{3.176083in}{3.515500in}}%
\pgfpathlineto{\pgfqpoint{3.185119in}{3.513391in}}%
\pgfpathlineto{\pgfqpoint{3.194156in}{3.510503in}}%
\pgfpathlineto{\pgfqpoint{3.203192in}{3.506826in}}%
\pgfpathlineto{\pgfqpoint{3.212228in}{3.502347in}}%
\pgfpathlineto{\pgfqpoint{3.221264in}{3.497057in}}%
\pgfpathlineto{\pgfqpoint{3.230301in}{3.490944in}}%
\pgfpathlineto{\pgfqpoint{3.239337in}{3.483996in}}%
\pgfpathlineto{\pgfqpoint{3.248373in}{3.476204in}}%
\pgfpathlineto{\pgfqpoint{3.257409in}{3.467556in}}%
\pgfpathlineto{\pgfqpoint{3.266446in}{3.458040in}}%
\pgfpathlineto{\pgfqpoint{3.275482in}{3.447646in}}%
\pgfpathlineto{\pgfqpoint{3.284518in}{3.436365in}}%
\pgfpathlineto{\pgfqpoint{3.293554in}{3.424209in}}%
\pgfpathlineto{\pgfqpoint{3.302591in}{3.411216in}}%
\pgfpathlineto{\pgfqpoint{3.311627in}{3.397424in}}%
\pgfpathlineto{\pgfqpoint{3.320663in}{3.382873in}}%
\pgfpathlineto{\pgfqpoint{3.329699in}{3.367601in}}%
\pgfpathlineto{\pgfqpoint{3.338736in}{3.351645in}}%
\pgfpathlineto{\pgfqpoint{3.347772in}{3.335046in}}%
\pgfpathlineto{\pgfqpoint{3.365844in}{3.300069in}}%
\pgfpathlineto{\pgfqpoint{3.383917in}{3.262977in}}%
\pgfpathlineto{\pgfqpoint{3.401989in}{3.224078in}}%
\pgfpathlineto{\pgfqpoint{3.420062in}{3.183682in}}%
\pgfpathlineto{\pgfqpoint{3.438134in}{3.142095in}}%
\pgfpathlineto{\pgfqpoint{3.465243in}{3.078159in}}%
\pgfpathlineto{\pgfqpoint{3.537533in}{2.905874in}}%
\pgfpathlineto{\pgfqpoint{3.555606in}{2.864066in}}%
\pgfpathlineto{\pgfqpoint{3.573678in}{2.823380in}}%
\pgfpathlineto{\pgfqpoint{3.591751in}{2.784124in}}%
\pgfpathlineto{\pgfqpoint{3.609823in}{2.746606in}}%
\pgfpathlineto{\pgfqpoint{3.627896in}{2.711134in}}%
\pgfpathlineto{\pgfqpoint{3.636932in}{2.694262in}}%
\pgfpathlineto{\pgfqpoint{3.645968in}{2.678017in}}%
\pgfpathlineto{\pgfqpoint{3.655005in}{2.662438in}}%
\pgfpathlineto{\pgfqpoint{3.664041in}{2.647563in}}%
\pgfpathlineto{\pgfqpoint{3.673077in}{2.633431in}}%
\pgfpathlineto{\pgfqpoint{3.682113in}{2.620080in}}%
\pgfpathlineto{\pgfqpoint{3.691150in}{2.607549in}}%
\pgfpathlineto{\pgfqpoint{3.700186in}{2.595877in}}%
\pgfpathlineto{\pgfqpoint{3.709222in}{2.585101in}}%
\pgfpathlineto{\pgfqpoint{3.718258in}{2.575261in}}%
\pgfpathlineto{\pgfqpoint{3.727295in}{2.566395in}}%
\pgfpathlineto{\pgfqpoint{3.736331in}{2.558541in}}%
\pgfpathlineto{\pgfqpoint{3.745367in}{2.551738in}}%
\pgfpathlineto{\pgfqpoint{3.754403in}{2.546025in}}%
\pgfpathlineto{\pgfqpoint{3.763440in}{2.541439in}}%
\pgfpathlineto{\pgfqpoint{3.772476in}{2.538021in}}%
\pgfpathlineto{\pgfqpoint{3.781512in}{2.535807in}}%
\pgfpathlineto{\pgfqpoint{3.790548in}{2.534837in}}%
\pgfpathlineto{\pgfqpoint{3.799585in}{2.535150in}}%
\pgfpathlineto{\pgfqpoint{3.808621in}{2.536782in}}%
\pgfpathlineto{\pgfqpoint{3.817657in}{2.539775in}}%
\pgfpathlineto{\pgfqpoint{3.826693in}{2.544165in}}%
\pgfpathlineto{\pgfqpoint{3.835730in}{2.549991in}}%
\pgfpathlineto{\pgfqpoint{3.844766in}{2.557292in}}%
\pgfpathlineto{\pgfqpoint{3.853802in}{2.566086in}}%
\pgfpathlineto{\pgfqpoint{3.862838in}{2.576334in}}%
\pgfpathlineto{\pgfqpoint{3.871875in}{2.587988in}}%
\pgfpathlineto{\pgfqpoint{3.880911in}{2.601000in}}%
\pgfpathlineto{\pgfqpoint{3.889947in}{2.615322in}}%
\pgfpathlineto{\pgfqpoint{3.898983in}{2.630906in}}%
\pgfpathlineto{\pgfqpoint{3.908020in}{2.647703in}}%
\pgfpathlineto{\pgfqpoint{3.917056in}{2.665666in}}%
\pgfpathlineto{\pgfqpoint{3.926092in}{2.684745in}}%
\pgfpathlineto{\pgfqpoint{3.935128in}{2.704893in}}%
\pgfpathlineto{\pgfqpoint{3.944165in}{2.726062in}}%
\pgfpathlineto{\pgfqpoint{3.953201in}{2.748204in}}%
\pgfpathlineto{\pgfqpoint{3.962237in}{2.771270in}}%
\pgfpathlineto{\pgfqpoint{3.971273in}{2.795212in}}%
\pgfpathlineto{\pgfqpoint{3.989346in}{2.845533in}}%
\pgfpathlineto{\pgfqpoint{4.007418in}{2.898779in}}%
\pgfpathlineto{\pgfqpoint{4.025491in}{2.954567in}}%
\pgfpathlineto{\pgfqpoint{4.043563in}{3.012511in}}%
\pgfpathlineto{\pgfqpoint{4.061636in}{3.072226in}}%
\pgfpathlineto{\pgfqpoint{4.088745in}{3.164273in}}%
\pgfpathlineto{\pgfqpoint{4.124890in}{3.289603in}}%
\pgfpathlineto{\pgfqpoint{4.170071in}{3.446105in}}%
\pgfpathlineto{\pgfqpoint{4.197180in}{3.537610in}}%
\pgfpathlineto{\pgfqpoint{4.215252in}{3.596811in}}%
\pgfpathlineto{\pgfqpoint{4.233325in}{3.654121in}}%
\pgfpathlineto{\pgfqpoint{4.251397in}{3.709154in}}%
\pgfpathlineto{\pgfqpoint{4.269470in}{3.761526in}}%
\pgfpathlineto{\pgfqpoint{4.287542in}{3.810850in}}%
\pgfpathlineto{\pgfqpoint{4.296579in}{3.834248in}}%
\pgfpathlineto{\pgfqpoint{4.305615in}{3.856741in}}%
\pgfpathlineto{\pgfqpoint{4.314651in}{3.878279in}}%
\pgfpathlineto{\pgfqpoint{4.323687in}{3.898814in}}%
\pgfpathlineto{\pgfqpoint{4.332724in}{3.918298in}}%
\pgfpathlineto{\pgfqpoint{4.341760in}{3.936683in}}%
\pgfpathlineto{\pgfqpoint{4.350796in}{3.953921in}}%
\pgfpathlineto{\pgfqpoint{4.359832in}{3.969963in}}%
\pgfpathlineto{\pgfqpoint{4.368869in}{3.984762in}}%
\pgfpathlineto{\pgfqpoint{4.377905in}{3.998269in}}%
\pgfpathlineto{\pgfqpoint{4.386941in}{4.010436in}}%
\pgfpathlineto{\pgfqpoint{4.395977in}{4.021215in}}%
\pgfpathlineto{\pgfqpoint{4.405014in}{4.030558in}}%
\pgfpathlineto{\pgfqpoint{4.414050in}{4.038421in}}%
\pgfpathlineto{\pgfqpoint{4.423086in}{4.044804in}}%
\pgfpathlineto{\pgfqpoint{4.432122in}{4.049731in}}%
\pgfpathlineto{\pgfqpoint{4.441159in}{4.053224in}}%
\pgfpathlineto{\pgfqpoint{4.450195in}{4.055306in}}%
\pgfpathlineto{\pgfqpoint{4.459231in}{4.056000in}}%
\pgfpathlineto{\pgfqpoint{4.468267in}{4.055328in}}%
\pgfpathlineto{\pgfqpoint{4.477304in}{4.053313in}}%
\pgfpathlineto{\pgfqpoint{4.486340in}{4.049978in}}%
\pgfpathlineto{\pgfqpoint{4.495376in}{4.045346in}}%
\pgfpathlineto{\pgfqpoint{4.504412in}{4.039440in}}%
\pgfpathlineto{\pgfqpoint{4.513449in}{4.032281in}}%
\pgfpathlineto{\pgfqpoint{4.522485in}{4.023894in}}%
\pgfpathlineto{\pgfqpoint{4.531521in}{4.014301in}}%
\pgfpathlineto{\pgfqpoint{4.540557in}{4.003524in}}%
\pgfpathlineto{\pgfqpoint{4.549594in}{3.991586in}}%
\pgfpathlineto{\pgfqpoint{4.558630in}{3.978511in}}%
\pgfpathlineto{\pgfqpoint{4.567666in}{3.964321in}}%
\pgfpathlineto{\pgfqpoint{4.576702in}{3.949038in}}%
\pgfpathlineto{\pgfqpoint{4.585739in}{3.932686in}}%
\pgfpathlineto{\pgfqpoint{4.594775in}{3.915287in}}%
\pgfpathlineto{\pgfqpoint{4.603811in}{3.896864in}}%
\pgfpathlineto{\pgfqpoint{4.612848in}{3.877441in}}%
\pgfpathlineto{\pgfqpoint{4.621884in}{3.857038in}}%
\pgfpathlineto{\pgfqpoint{4.630920in}{3.835680in}}%
\pgfpathlineto{\pgfqpoint{4.639956in}{3.813389in}}%
\pgfpathlineto{\pgfqpoint{4.648993in}{3.790189in}}%
\pgfpathlineto{\pgfqpoint{4.658029in}{3.766100in}}%
\pgfpathlineto{\pgfqpoint{4.667065in}{3.741148in}}%
\pgfpathlineto{\pgfqpoint{4.685138in}{3.688740in}}%
\pgfpathlineto{\pgfqpoint{4.703210in}{3.633147in}}%
\pgfpathlineto{\pgfqpoint{4.721283in}{3.574552in}}%
\pgfpathlineto{\pgfqpoint{4.739355in}{3.513137in}}%
\pgfpathlineto{\pgfqpoint{4.757428in}{3.449083in}}%
\pgfpathlineto{\pgfqpoint{4.775500in}{3.382574in}}%
\pgfpathlineto{\pgfqpoint{4.793573in}{3.313790in}}%
\pgfpathlineto{\pgfqpoint{4.811645in}{3.242916in}}%
\pgfpathlineto{\pgfqpoint{4.829718in}{3.170132in}}%
\pgfpathlineto{\pgfqpoint{4.847790in}{3.095621in}}%
\pgfpathlineto{\pgfqpoint{4.874899in}{2.981014in}}%
\pgfpathlineto{\pgfqpoint{4.902008in}{2.863547in}}%
\pgfpathlineto{\pgfqpoint{4.929116in}{2.743834in}}%
\pgfpathlineto{\pgfqpoint{4.965261in}{2.581785in}}%
\pgfpathlineto{\pgfqpoint{5.082733in}{2.052717in}}%
\pgfpathlineto{\pgfqpoint{5.109842in}{1.933746in}}%
\pgfpathlineto{\pgfqpoint{5.136950in}{1.817243in}}%
\pgfpathlineto{\pgfqpoint{5.164059in}{1.703823in}}%
\pgfpathlineto{\pgfqpoint{5.182132in}{1.630227in}}%
\pgfpathlineto{\pgfqpoint{5.200204in}{1.558457in}}%
\pgfpathlineto{\pgfqpoint{5.218277in}{1.488695in}}%
\pgfpathlineto{\pgfqpoint{5.236349in}{1.421124in}}%
\pgfpathlineto{\pgfqpoint{5.254422in}{1.355925in}}%
\pgfpathlineto{\pgfqpoint{5.272494in}{1.293281in}}%
\pgfpathlineto{\pgfqpoint{5.290567in}{1.233374in}}%
\pgfpathlineto{\pgfqpoint{5.308639in}{1.176386in}}%
\pgfpathlineto{\pgfqpoint{5.326712in}{1.122500in}}%
\pgfpathlineto{\pgfqpoint{5.335748in}{1.096777in}}%
\pgfpathlineto{\pgfqpoint{5.344784in}{1.071897in}}%
\pgfpathlineto{\pgfqpoint{5.353820in}{1.047884in}}%
\pgfpathlineto{\pgfqpoint{5.362857in}{1.024760in}}%
\pgfpathlineto{\pgfqpoint{5.371893in}{1.002549in}}%
\pgfpathlineto{\pgfqpoint{5.380929in}{0.981272in}}%
\pgfpathlineto{\pgfqpoint{5.389965in}{0.960952in}}%
\pgfpathlineto{\pgfqpoint{5.399002in}{0.941613in}}%
\pgfpathlineto{\pgfqpoint{5.408038in}{0.923277in}}%
\pgfpathlineto{\pgfqpoint{5.417074in}{0.905967in}}%
\pgfpathlineto{\pgfqpoint{5.426110in}{0.889706in}}%
\pgfpathlineto{\pgfqpoint{5.435147in}{0.874516in}}%
\pgfpathlineto{\pgfqpoint{5.444183in}{0.860421in}}%
\pgfpathlineto{\pgfqpoint{5.453219in}{0.847442in}}%
\pgfpathlineto{\pgfqpoint{5.462255in}{0.835603in}}%
\pgfpathlineto{\pgfqpoint{5.471292in}{0.824927in}}%
\pgfpathlineto{\pgfqpoint{5.480328in}{0.815436in}}%
\pgfpathlineto{\pgfqpoint{5.489364in}{0.807153in}}%
\pgfpathlineto{\pgfqpoint{5.498400in}{0.800101in}}%
\pgfpathlineto{\pgfqpoint{5.507437in}{0.794303in}}%
\pgfpathlineto{\pgfqpoint{5.516473in}{0.789781in}}%
\pgfpathlineto{\pgfqpoint{5.525509in}{0.786559in}}%
\pgfpathlineto{\pgfqpoint{5.534545in}{0.784658in}}%
\pgfpathlineto{\pgfqpoint{5.534545in}{0.784658in}}%
\pgfusepath{stroke}%
\end{pgfscope}%
\begin{pgfscope}%
\pgfpathrectangle{\pgfqpoint{0.800000in}{0.528000in}}{\pgfqpoint{4.960000in}{3.696000in}}%
\pgfusepath{clip}%
\pgfsetroundcap%
\pgfsetroundjoin%
\pgfsetlinewidth{1.505625pt}%
\definecolor{currentstroke}{rgb}{1.000000,0.647059,0.000000}%
\pgfsetstrokecolor{currentstroke}%
\pgfsetdash{}{0pt}%
\pgfpathmoveto{\pgfqpoint{1.025455in}{0.784658in}}%
\pgfpathlineto{\pgfqpoint{1.034491in}{0.798290in}}%
\pgfpathlineto{\pgfqpoint{1.043527in}{0.811224in}}%
\pgfpathlineto{\pgfqpoint{1.052563in}{0.823473in}}%
\pgfpathlineto{\pgfqpoint{1.061600in}{0.835048in}}%
\pgfpathlineto{\pgfqpoint{1.070636in}{0.845964in}}%
\pgfpathlineto{\pgfqpoint{1.079672in}{0.856232in}}%
\pgfpathlineto{\pgfqpoint{1.088708in}{0.865865in}}%
\pgfpathlineto{\pgfqpoint{1.097745in}{0.874876in}}%
\pgfpathlineto{\pgfqpoint{1.106781in}{0.883277in}}%
\pgfpathlineto{\pgfqpoint{1.115817in}{0.891081in}}%
\pgfpathlineto{\pgfqpoint{1.124853in}{0.898301in}}%
\pgfpathlineto{\pgfqpoint{1.133890in}{0.904949in}}%
\pgfpathlineto{\pgfqpoint{1.142926in}{0.911037in}}%
\pgfpathlineto{\pgfqpoint{1.151962in}{0.916579in}}%
\pgfpathlineto{\pgfqpoint{1.160998in}{0.921587in}}%
\pgfpathlineto{\pgfqpoint{1.170035in}{0.926073in}}%
\pgfpathlineto{\pgfqpoint{1.179071in}{0.930051in}}%
\pgfpathlineto{\pgfqpoint{1.188107in}{0.933532in}}%
\pgfpathlineto{\pgfqpoint{1.197143in}{0.936529in}}%
\pgfpathlineto{\pgfqpoint{1.206180in}{0.939056in}}%
\pgfpathlineto{\pgfqpoint{1.215216in}{0.941124in}}%
\pgfpathlineto{\pgfqpoint{1.224252in}{0.942747in}}%
\pgfpathlineto{\pgfqpoint{1.233288in}{0.943936in}}%
\pgfpathlineto{\pgfqpoint{1.242325in}{0.944704in}}%
\pgfpathlineto{\pgfqpoint{1.251361in}{0.945065in}}%
\pgfpathlineto{\pgfqpoint{1.260397in}{0.945031in}}%
\pgfpathlineto{\pgfqpoint{1.269433in}{0.944613in}}%
\pgfpathlineto{\pgfqpoint{1.278470in}{0.943826in}}%
\pgfpathlineto{\pgfqpoint{1.287506in}{0.942680in}}%
\pgfpathlineto{\pgfqpoint{1.296542in}{0.941190in}}%
\pgfpathlineto{\pgfqpoint{1.305578in}{0.939368in}}%
\pgfpathlineto{\pgfqpoint{1.314615in}{0.937226in}}%
\pgfpathlineto{\pgfqpoint{1.323651in}{0.934777in}}%
\pgfpathlineto{\pgfqpoint{1.341723in}{0.929008in}}%
\pgfpathlineto{\pgfqpoint{1.359796in}{0.922162in}}%
\pgfpathlineto{\pgfqpoint{1.377868in}{0.914339in}}%
\pgfpathlineto{\pgfqpoint{1.395941in}{0.905642in}}%
\pgfpathlineto{\pgfqpoint{1.414013in}{0.896169in}}%
\pgfpathlineto{\pgfqpoint{1.432086in}{0.886024in}}%
\pgfpathlineto{\pgfqpoint{1.450158in}{0.875305in}}%
\pgfpathlineto{\pgfqpoint{1.477267in}{0.858374in}}%
\pgfpathlineto{\pgfqpoint{1.504376in}{0.840722in}}%
\pgfpathlineto{\pgfqpoint{1.594739in}{0.781043in}}%
\pgfpathlineto{\pgfqpoint{1.621847in}{0.764199in}}%
\pgfpathlineto{\pgfqpoint{1.639920in}{0.753557in}}%
\pgfpathlineto{\pgfqpoint{1.657992in}{0.743502in}}%
\pgfpathlineto{\pgfqpoint{1.676065in}{0.734135in}}%
\pgfpathlineto{\pgfqpoint{1.694137in}{0.725557in}}%
\pgfpathlineto{\pgfqpoint{1.712210in}{0.717869in}}%
\pgfpathlineto{\pgfqpoint{1.730282in}{0.711172in}}%
\pgfpathlineto{\pgfqpoint{1.748355in}{0.705567in}}%
\pgfpathlineto{\pgfqpoint{1.757391in}{0.703205in}}%
\pgfpathlineto{\pgfqpoint{1.766427in}{0.701154in}}%
\pgfpathlineto{\pgfqpoint{1.775464in}{0.699426in}}%
\pgfpathlineto{\pgfqpoint{1.784500in}{0.698034in}}%
\pgfpathlineto{\pgfqpoint{1.793536in}{0.696991in}}%
\pgfpathlineto{\pgfqpoint{1.802572in}{0.696308in}}%
\pgfpathlineto{\pgfqpoint{1.811609in}{0.696000in}}%
\pgfpathlineto{\pgfqpoint{1.820645in}{0.696078in}}%
\pgfpathlineto{\pgfqpoint{1.829681in}{0.696555in}}%
\pgfpathlineto{\pgfqpoint{1.838717in}{0.697444in}}%
\pgfpathlineto{\pgfqpoint{1.847754in}{0.698756in}}%
\pgfpathlineto{\pgfqpoint{1.856790in}{0.700506in}}%
\pgfpathlineto{\pgfqpoint{1.865826in}{0.702705in}}%
\pgfpathlineto{\pgfqpoint{1.874862in}{0.705366in}}%
\pgfpathlineto{\pgfqpoint{1.883899in}{0.708502in}}%
\pgfpathlineto{\pgfqpoint{1.892935in}{0.712125in}}%
\pgfpathlineto{\pgfqpoint{1.901971in}{0.716248in}}%
\pgfpathlineto{\pgfqpoint{1.911007in}{0.720883in}}%
\pgfpathlineto{\pgfqpoint{1.920044in}{0.726043in}}%
\pgfpathlineto{\pgfqpoint{1.929080in}{0.731741in}}%
\pgfpathlineto{\pgfqpoint{1.938116in}{0.737990in}}%
\pgfpathlineto{\pgfqpoint{1.947152in}{0.744801in}}%
\pgfpathlineto{\pgfqpoint{1.956189in}{0.752188in}}%
\pgfpathlineto{\pgfqpoint{1.965225in}{0.760163in}}%
\pgfpathlineto{\pgfqpoint{1.974261in}{0.768738in}}%
\pgfpathlineto{\pgfqpoint{1.983298in}{0.777927in}}%
\pgfpathlineto{\pgfqpoint{1.992334in}{0.787742in}}%
\pgfpathlineto{\pgfqpoint{2.001370in}{0.798196in}}%
\pgfpathlineto{\pgfqpoint{2.010406in}{0.809300in}}%
\pgfpathlineto{\pgfqpoint{2.019443in}{0.821069in}}%
\pgfpathlineto{\pgfqpoint{2.028479in}{0.833513in}}%
\pgfpathlineto{\pgfqpoint{2.037515in}{0.846647in}}%
\pgfpathlineto{\pgfqpoint{2.046551in}{0.860483in}}%
\pgfpathlineto{\pgfqpoint{2.055588in}{0.875032in}}%
\pgfpathlineto{\pgfqpoint{2.064624in}{0.890309in}}%
\pgfpathlineto{\pgfqpoint{2.073660in}{0.906324in}}%
\pgfpathlineto{\pgfqpoint{2.082696in}{0.923092in}}%
\pgfpathlineto{\pgfqpoint{2.091733in}{0.940625in}}%
\pgfpathlineto{\pgfqpoint{2.100769in}{0.958935in}}%
\pgfpathlineto{\pgfqpoint{2.109805in}{0.978034in}}%
\pgfpathlineto{\pgfqpoint{2.118841in}{0.997936in}}%
\pgfpathlineto{\pgfqpoint{2.127878in}{1.018653in}}%
\pgfpathlineto{\pgfqpoint{2.136914in}{1.040198in}}%
\pgfpathlineto{\pgfqpoint{2.145950in}{1.062583in}}%
\pgfpathlineto{\pgfqpoint{2.154986in}{1.085821in}}%
\pgfpathlineto{\pgfqpoint{2.164023in}{1.109912in}}%
\pgfpathlineto{\pgfqpoint{2.182095in}{1.160555in}}%
\pgfpathlineto{\pgfqpoint{2.200168in}{1.214303in}}%
\pgfpathlineto{\pgfqpoint{2.218240in}{1.270942in}}%
\pgfpathlineto{\pgfqpoint{2.236313in}{1.330261in}}%
\pgfpathlineto{\pgfqpoint{2.254385in}{1.392047in}}%
\pgfpathlineto{\pgfqpoint{2.272458in}{1.456089in}}%
\pgfpathlineto{\pgfqpoint{2.290530in}{1.522173in}}%
\pgfpathlineto{\pgfqpoint{2.308603in}{1.590088in}}%
\pgfpathlineto{\pgfqpoint{2.326675in}{1.659621in}}%
\pgfpathlineto{\pgfqpoint{2.353784in}{1.766489in}}%
\pgfpathlineto{\pgfqpoint{2.380893in}{1.875804in}}%
\pgfpathlineto{\pgfqpoint{2.417038in}{2.024123in}}%
\pgfpathlineto{\pgfqpoint{2.516437in}{2.433989in}}%
\pgfpathlineto{\pgfqpoint{2.543545in}{2.542932in}}%
\pgfpathlineto{\pgfqpoint{2.570654in}{2.649305in}}%
\pgfpathlineto{\pgfqpoint{2.588727in}{2.718439in}}%
\pgfpathlineto{\pgfqpoint{2.606799in}{2.785900in}}%
\pgfpathlineto{\pgfqpoint{2.624872in}{2.851475in}}%
\pgfpathlineto{\pgfqpoint{2.642944in}{2.914952in}}%
\pgfpathlineto{\pgfqpoint{2.661017in}{2.976119in}}%
\pgfpathlineto{\pgfqpoint{2.679089in}{3.034764in}}%
\pgfpathlineto{\pgfqpoint{2.697162in}{3.090673in}}%
\pgfpathlineto{\pgfqpoint{2.715234in}{3.143636in}}%
\pgfpathlineto{\pgfqpoint{2.733307in}{3.193475in}}%
\pgfpathlineto{\pgfqpoint{2.751379in}{3.240186in}}%
\pgfpathlineto{\pgfqpoint{2.760415in}{3.262383in}}%
\pgfpathlineto{\pgfqpoint{2.769452in}{3.283817in}}%
\pgfpathlineto{\pgfqpoint{2.778488in}{3.304492in}}%
\pgfpathlineto{\pgfqpoint{2.787524in}{3.324414in}}%
\pgfpathlineto{\pgfqpoint{2.796560in}{3.343591in}}%
\pgfpathlineto{\pgfqpoint{2.805597in}{3.362027in}}%
\pgfpathlineto{\pgfqpoint{2.814633in}{3.379729in}}%
\pgfpathlineto{\pgfqpoint{2.823669in}{3.396702in}}%
\pgfpathlineto{\pgfqpoint{2.832705in}{3.412953in}}%
\pgfpathlineto{\pgfqpoint{2.841742in}{3.428487in}}%
\pgfpathlineto{\pgfqpoint{2.850778in}{3.443311in}}%
\pgfpathlineto{\pgfqpoint{2.859814in}{3.457430in}}%
\pgfpathlineto{\pgfqpoint{2.868850in}{3.470850in}}%
\pgfpathlineto{\pgfqpoint{2.877887in}{3.483578in}}%
\pgfpathlineto{\pgfqpoint{2.886923in}{3.495619in}}%
\pgfpathlineto{\pgfqpoint{2.895959in}{3.506979in}}%
\pgfpathlineto{\pgfqpoint{2.904995in}{3.517665in}}%
\pgfpathlineto{\pgfqpoint{2.914032in}{3.527681in}}%
\pgfpathlineto{\pgfqpoint{2.923068in}{3.537035in}}%
\pgfpathlineto{\pgfqpoint{2.932104in}{3.545731in}}%
\pgfpathlineto{\pgfqpoint{2.941140in}{3.553777in}}%
\pgfpathlineto{\pgfqpoint{2.950177in}{3.561177in}}%
\pgfpathlineto{\pgfqpoint{2.959213in}{3.567938in}}%
\pgfpathlineto{\pgfqpoint{2.968249in}{3.574067in}}%
\pgfpathlineto{\pgfqpoint{2.977285in}{3.579568in}}%
\pgfpathlineto{\pgfqpoint{2.986322in}{3.584447in}}%
\pgfpathlineto{\pgfqpoint{2.995358in}{3.588712in}}%
\pgfpathlineto{\pgfqpoint{3.004394in}{3.592367in}}%
\pgfpathlineto{\pgfqpoint{3.013430in}{3.595418in}}%
\pgfpathlineto{\pgfqpoint{3.022467in}{3.597873in}}%
\pgfpathlineto{\pgfqpoint{3.031503in}{3.599735in}}%
\pgfpathlineto{\pgfqpoint{3.040539in}{3.601013in}}%
\pgfpathlineto{\pgfqpoint{3.049576in}{3.601710in}}%
\pgfpathlineto{\pgfqpoint{3.058612in}{3.601834in}}%
\pgfpathlineto{\pgfqpoint{3.067648in}{3.601391in}}%
\pgfpathlineto{\pgfqpoint{3.076684in}{3.600385in}}%
\pgfpathlineto{\pgfqpoint{3.085721in}{3.598824in}}%
\pgfpathlineto{\pgfqpoint{3.094757in}{3.596713in}}%
\pgfpathlineto{\pgfqpoint{3.103793in}{3.594058in}}%
\pgfpathlineto{\pgfqpoint{3.112829in}{3.590866in}}%
\pgfpathlineto{\pgfqpoint{3.121866in}{3.587141in}}%
\pgfpathlineto{\pgfqpoint{3.130902in}{3.582890in}}%
\pgfpathlineto{\pgfqpoint{3.139938in}{3.578120in}}%
\pgfpathlineto{\pgfqpoint{3.148974in}{3.572835in}}%
\pgfpathlineto{\pgfqpoint{3.158011in}{3.567042in}}%
\pgfpathlineto{\pgfqpoint{3.167047in}{3.560747in}}%
\pgfpathlineto{\pgfqpoint{3.176083in}{3.553956in}}%
\pgfpathlineto{\pgfqpoint{3.185119in}{3.546674in}}%
\pgfpathlineto{\pgfqpoint{3.194156in}{3.538909in}}%
\pgfpathlineto{\pgfqpoint{3.203192in}{3.530664in}}%
\pgfpathlineto{\pgfqpoint{3.212228in}{3.521948in}}%
\pgfpathlineto{\pgfqpoint{3.221264in}{3.512765in}}%
\pgfpathlineto{\pgfqpoint{3.230301in}{3.503121in}}%
\pgfpathlineto{\pgfqpoint{3.239337in}{3.493023in}}%
\pgfpathlineto{\pgfqpoint{3.248373in}{3.482477in}}%
\pgfpathlineto{\pgfqpoint{3.266446in}{3.460061in}}%
\pgfpathlineto{\pgfqpoint{3.284518in}{3.435924in}}%
\pgfpathlineto{\pgfqpoint{3.302591in}{3.410159in}}%
\pgfpathlineto{\pgfqpoint{3.320663in}{3.382953in}}%
\pgfpathlineto{\pgfqpoint{3.338736in}{3.354503in}}%
\pgfpathlineto{\pgfqpoint{3.356808in}{3.325006in}}%
\pgfpathlineto{\pgfqpoint{3.383917in}{3.279229in}}%
\pgfpathlineto{\pgfqpoint{3.411026in}{3.232204in}}%
\pgfpathlineto{\pgfqpoint{3.483316in}{3.105764in}}%
\pgfpathlineto{\pgfqpoint{3.510424in}{3.059855in}}%
\pgfpathlineto{\pgfqpoint{3.528497in}{3.030239in}}%
\pgfpathlineto{\pgfqpoint{3.546570in}{3.001644in}}%
\pgfpathlineto{\pgfqpoint{3.564642in}{2.974268in}}%
\pgfpathlineto{\pgfqpoint{3.582715in}{2.948308in}}%
\pgfpathlineto{\pgfqpoint{3.600787in}{2.923961in}}%
\pgfpathlineto{\pgfqpoint{3.609823in}{2.912453in}}%
\pgfpathlineto{\pgfqpoint{3.618860in}{2.901423in}}%
\pgfpathlineto{\pgfqpoint{3.627896in}{2.890895in}}%
\pgfpathlineto{\pgfqpoint{3.636932in}{2.880892in}}%
\pgfpathlineto{\pgfqpoint{3.645968in}{2.871441in}}%
\pgfpathlineto{\pgfqpoint{3.655005in}{2.862565in}}%
\pgfpathlineto{\pgfqpoint{3.664041in}{2.854290in}}%
\pgfpathlineto{\pgfqpoint{3.673077in}{2.846639in}}%
\pgfpathlineto{\pgfqpoint{3.682113in}{2.839637in}}%
\pgfpathlineto{\pgfqpoint{3.691150in}{2.833310in}}%
\pgfpathlineto{\pgfqpoint{3.700186in}{2.827682in}}%
\pgfpathlineto{\pgfqpoint{3.709222in}{2.822777in}}%
\pgfpathlineto{\pgfqpoint{3.718258in}{2.818619in}}%
\pgfpathlineto{\pgfqpoint{3.727295in}{2.815235in}}%
\pgfpathlineto{\pgfqpoint{3.736331in}{2.812647in}}%
\pgfpathlineto{\pgfqpoint{3.745367in}{2.810882in}}%
\pgfpathlineto{\pgfqpoint{3.754403in}{2.809962in}}%
\pgfpathlineto{\pgfqpoint{3.763440in}{2.809914in}}%
\pgfpathlineto{\pgfqpoint{3.772476in}{2.810762in}}%
\pgfpathlineto{\pgfqpoint{3.781512in}{2.812530in}}%
\pgfpathlineto{\pgfqpoint{3.790548in}{2.815243in}}%
\pgfpathlineto{\pgfqpoint{3.799585in}{2.818925in}}%
\pgfpathlineto{\pgfqpoint{3.808621in}{2.823602in}}%
\pgfpathlineto{\pgfqpoint{3.817657in}{2.829298in}}%
\pgfpathlineto{\pgfqpoint{3.826693in}{2.836037in}}%
\pgfpathlineto{\pgfqpoint{3.835730in}{2.843844in}}%
\pgfpathlineto{\pgfqpoint{3.844766in}{2.852744in}}%
\pgfpathlineto{\pgfqpoint{3.853802in}{2.862746in}}%
\pgfpathlineto{\pgfqpoint{3.862838in}{2.873819in}}%
\pgfpathlineto{\pgfqpoint{3.871875in}{2.885923in}}%
\pgfpathlineto{\pgfqpoint{3.880911in}{2.899021in}}%
\pgfpathlineto{\pgfqpoint{3.889947in}{2.913074in}}%
\pgfpathlineto{\pgfqpoint{3.898983in}{2.928041in}}%
\pgfpathlineto{\pgfqpoint{3.908020in}{2.943886in}}%
\pgfpathlineto{\pgfqpoint{3.917056in}{2.960569in}}%
\pgfpathlineto{\pgfqpoint{3.926092in}{2.978052in}}%
\pgfpathlineto{\pgfqpoint{3.935128in}{2.996295in}}%
\pgfpathlineto{\pgfqpoint{3.944165in}{3.015259in}}%
\pgfpathlineto{\pgfqpoint{3.962237in}{3.055200in}}%
\pgfpathlineto{\pgfqpoint{3.980310in}{3.097562in}}%
\pgfpathlineto{\pgfqpoint{3.998382in}{3.142037in}}%
\pgfpathlineto{\pgfqpoint{4.016455in}{3.188315in}}%
\pgfpathlineto{\pgfqpoint{4.034527in}{3.236084in}}%
\pgfpathlineto{\pgfqpoint{4.061636in}{3.309858in}}%
\pgfpathlineto{\pgfqpoint{4.097781in}{3.410553in}}%
\pgfpathlineto{\pgfqpoint{4.142962in}{3.536676in}}%
\pgfpathlineto{\pgfqpoint{4.170071in}{3.610626in}}%
\pgfpathlineto{\pgfqpoint{4.188144in}{3.658562in}}%
\pgfpathlineto{\pgfqpoint{4.206216in}{3.705044in}}%
\pgfpathlineto{\pgfqpoint{4.224289in}{3.749763in}}%
\pgfpathlineto{\pgfqpoint{4.242361in}{3.792409in}}%
\pgfpathlineto{\pgfqpoint{4.260434in}{3.832671in}}%
\pgfpathlineto{\pgfqpoint{4.269470in}{3.851812in}}%
\pgfpathlineto{\pgfqpoint{4.278506in}{3.870241in}}%
\pgfpathlineto{\pgfqpoint{4.287542in}{3.887918in}}%
\pgfpathlineto{\pgfqpoint{4.296579in}{3.904806in}}%
\pgfpathlineto{\pgfqpoint{4.305615in}{3.920866in}}%
\pgfpathlineto{\pgfqpoint{4.314651in}{3.936059in}}%
\pgfpathlineto{\pgfqpoint{4.323687in}{3.950345in}}%
\pgfpathlineto{\pgfqpoint{4.332724in}{3.963687in}}%
\pgfpathlineto{\pgfqpoint{4.341760in}{3.976046in}}%
\pgfpathlineto{\pgfqpoint{4.350796in}{3.987383in}}%
\pgfpathlineto{\pgfqpoint{4.359832in}{3.997658in}}%
\pgfpathlineto{\pgfqpoint{4.368869in}{4.006834in}}%
\pgfpathlineto{\pgfqpoint{4.377905in}{4.014872in}}%
\pgfpathlineto{\pgfqpoint{4.386941in}{4.021732in}}%
\pgfpathlineto{\pgfqpoint{4.395977in}{4.027377in}}%
\pgfpathlineto{\pgfqpoint{4.405014in}{4.031766in}}%
\pgfpathlineto{\pgfqpoint{4.414050in}{4.034867in}}%
\pgfpathlineto{\pgfqpoint{4.423086in}{4.036680in}}%
\pgfpathlineto{\pgfqpoint{4.432122in}{4.037227in}}%
\pgfpathlineto{\pgfqpoint{4.441159in}{4.036528in}}%
\pgfpathlineto{\pgfqpoint{4.450195in}{4.034604in}}%
\pgfpathlineto{\pgfqpoint{4.459231in}{4.031476in}}%
\pgfpathlineto{\pgfqpoint{4.468267in}{4.027166in}}%
\pgfpathlineto{\pgfqpoint{4.477304in}{4.021693in}}%
\pgfpathlineto{\pgfqpoint{4.486340in}{4.015079in}}%
\pgfpathlineto{\pgfqpoint{4.495376in}{4.007345in}}%
\pgfpathlineto{\pgfqpoint{4.504412in}{3.998511in}}%
\pgfpathlineto{\pgfqpoint{4.513449in}{3.988599in}}%
\pgfpathlineto{\pgfqpoint{4.522485in}{3.977629in}}%
\pgfpathlineto{\pgfqpoint{4.531521in}{3.965622in}}%
\pgfpathlineto{\pgfqpoint{4.540557in}{3.952599in}}%
\pgfpathlineto{\pgfqpoint{4.549594in}{3.938582in}}%
\pgfpathlineto{\pgfqpoint{4.558630in}{3.923590in}}%
\pgfpathlineto{\pgfqpoint{4.567666in}{3.907646in}}%
\pgfpathlineto{\pgfqpoint{4.576702in}{3.890768in}}%
\pgfpathlineto{\pgfqpoint{4.585739in}{3.872980in}}%
\pgfpathlineto{\pgfqpoint{4.594775in}{3.854301in}}%
\pgfpathlineto{\pgfqpoint{4.603811in}{3.834752in}}%
\pgfpathlineto{\pgfqpoint{4.612848in}{3.814355in}}%
\pgfpathlineto{\pgfqpoint{4.621884in}{3.793130in}}%
\pgfpathlineto{\pgfqpoint{4.630920in}{3.771098in}}%
\pgfpathlineto{\pgfqpoint{4.639956in}{3.748280in}}%
\pgfpathlineto{\pgfqpoint{4.658029in}{3.700369in}}%
\pgfpathlineto{\pgfqpoint{4.676101in}{3.649564in}}%
\pgfpathlineto{\pgfqpoint{4.694174in}{3.596033in}}%
\pgfpathlineto{\pgfqpoint{4.712246in}{3.539943in}}%
\pgfpathlineto{\pgfqpoint{4.730319in}{3.481461in}}%
\pgfpathlineto{\pgfqpoint{4.748391in}{3.420754in}}%
\pgfpathlineto{\pgfqpoint{4.766464in}{3.357989in}}%
\pgfpathlineto{\pgfqpoint{4.784536in}{3.293333in}}%
\pgfpathlineto{\pgfqpoint{4.802609in}{3.226954in}}%
\pgfpathlineto{\pgfqpoint{4.820681in}{3.159018in}}%
\pgfpathlineto{\pgfqpoint{4.847790in}{3.054562in}}%
\pgfpathlineto{\pgfqpoint{4.874899in}{2.947544in}}%
\pgfpathlineto{\pgfqpoint{4.911044in}{2.801843in}}%
\pgfpathlineto{\pgfqpoint{4.956225in}{2.616759in}}%
\pgfpathlineto{\pgfqpoint{5.028515in}{2.319946in}}%
\pgfpathlineto{\pgfqpoint{5.064660in}{2.174067in}}%
\pgfpathlineto{\pgfqpoint{5.091769in}{2.066861in}}%
\pgfpathlineto{\pgfqpoint{5.118878in}{1.962170in}}%
\pgfpathlineto{\pgfqpoint{5.136950in}{1.894051in}}%
\pgfpathlineto{\pgfqpoint{5.155023in}{1.827468in}}%
\pgfpathlineto{\pgfqpoint{5.173095in}{1.762588in}}%
\pgfpathlineto{\pgfqpoint{5.191168in}{1.699577in}}%
\pgfpathlineto{\pgfqpoint{5.209240in}{1.638604in}}%
\pgfpathlineto{\pgfqpoint{5.227313in}{1.579834in}}%
\pgfpathlineto{\pgfqpoint{5.245385in}{1.523436in}}%
\pgfpathlineto{\pgfqpoint{5.263458in}{1.469576in}}%
\pgfpathlineto{\pgfqpoint{5.281530in}{1.418422in}}%
\pgfpathlineto{\pgfqpoint{5.299603in}{1.370140in}}%
\pgfpathlineto{\pgfqpoint{5.308639in}{1.347129in}}%
\pgfpathlineto{\pgfqpoint{5.317675in}{1.324898in}}%
\pgfpathlineto{\pgfqpoint{5.326712in}{1.303469in}}%
\pgfpathlineto{\pgfqpoint{5.335748in}{1.282863in}}%
\pgfpathlineto{\pgfqpoint{5.344784in}{1.263100in}}%
\pgfpathlineto{\pgfqpoint{5.353820in}{1.244202in}}%
\pgfpathlineto{\pgfqpoint{5.362857in}{1.226189in}}%
\pgfpathlineto{\pgfqpoint{5.371893in}{1.209082in}}%
\pgfpathlineto{\pgfqpoint{5.380929in}{1.192902in}}%
\pgfpathlineto{\pgfqpoint{5.389965in}{1.177670in}}%
\pgfpathlineto{\pgfqpoint{5.399002in}{1.163407in}}%
\pgfpathlineto{\pgfqpoint{5.408038in}{1.150134in}}%
\pgfpathlineto{\pgfqpoint{5.417074in}{1.137871in}}%
\pgfpathlineto{\pgfqpoint{5.426110in}{1.126640in}}%
\pgfpathlineto{\pgfqpoint{5.435147in}{1.116461in}}%
\pgfpathlineto{\pgfqpoint{5.444183in}{1.107356in}}%
\pgfpathlineto{\pgfqpoint{5.453219in}{1.099345in}}%
\pgfpathlineto{\pgfqpoint{5.462255in}{1.092449in}}%
\pgfpathlineto{\pgfqpoint{5.471292in}{1.086689in}}%
\pgfpathlineto{\pgfqpoint{5.480328in}{1.082086in}}%
\pgfpathlineto{\pgfqpoint{5.489364in}{1.078660in}}%
\pgfpathlineto{\pgfqpoint{5.498400in}{1.076434in}}%
\pgfpathlineto{\pgfqpoint{5.507437in}{1.075427in}}%
\pgfpathlineto{\pgfqpoint{5.516473in}{1.075660in}}%
\pgfpathlineto{\pgfqpoint{5.525509in}{1.077155in}}%
\pgfpathlineto{\pgfqpoint{5.534545in}{1.079931in}}%
\pgfpathlineto{\pgfqpoint{5.534545in}{1.079931in}}%
\pgfusepath{stroke}%
\end{pgfscope}%
\begin{pgfscope}%
\pgfsetrectcap%
\pgfsetmiterjoin%
\pgfsetlinewidth{1.254687pt}%
\definecolor{currentstroke}{rgb}{1.000000,1.000000,1.000000}%
\pgfsetstrokecolor{currentstroke}%
\pgfsetdash{}{0pt}%
\pgfpathmoveto{\pgfqpoint{0.800000in}{0.528000in}}%
\pgfpathlineto{\pgfqpoint{0.800000in}{4.224000in}}%
\pgfusepath{stroke}%
\end{pgfscope}%
\begin{pgfscope}%
\pgfsetrectcap%
\pgfsetmiterjoin%
\pgfsetlinewidth{1.254687pt}%
\definecolor{currentstroke}{rgb}{1.000000,1.000000,1.000000}%
\pgfsetstrokecolor{currentstroke}%
\pgfsetdash{}{0pt}%
\pgfpathmoveto{\pgfqpoint{5.760000in}{0.528000in}}%
\pgfpathlineto{\pgfqpoint{5.760000in}{4.224000in}}%
\pgfusepath{stroke}%
\end{pgfscope}%
\begin{pgfscope}%
\pgfsetrectcap%
\pgfsetmiterjoin%
\pgfsetlinewidth{1.254687pt}%
\definecolor{currentstroke}{rgb}{1.000000,1.000000,1.000000}%
\pgfsetstrokecolor{currentstroke}%
\pgfsetdash{}{0pt}%
\pgfpathmoveto{\pgfqpoint{0.800000in}{0.528000in}}%
\pgfpathlineto{\pgfqpoint{5.760000in}{0.528000in}}%
\pgfusepath{stroke}%
\end{pgfscope}%
\begin{pgfscope}%
\pgfsetrectcap%
\pgfsetmiterjoin%
\pgfsetlinewidth{1.254687pt}%
\definecolor{currentstroke}{rgb}{1.000000,1.000000,1.000000}%
\pgfsetstrokecolor{currentstroke}%
\pgfsetdash{}{0pt}%
\pgfpathmoveto{\pgfqpoint{0.800000in}{4.224000in}}%
\pgfpathlineto{\pgfqpoint{5.760000in}{4.224000in}}%
\pgfusepath{stroke}%
\end{pgfscope}%
\begin{pgfscope}%
\pgfsetbuttcap%
\pgfsetmiterjoin%
\definecolor{currentfill}{rgb}{0.917647,0.917647,0.949020}%
\pgfsetfillcolor{currentfill}%
\pgfsetfillopacity{0.800000}%
\pgfsetlinewidth{1.003750pt}%
\definecolor{currentstroke}{rgb}{0.800000,0.800000,0.800000}%
\pgfsetstrokecolor{currentstroke}%
\pgfsetstrokeopacity{0.800000}%
\pgfsetdash{}{0pt}%
\pgfpathmoveto{\pgfqpoint{0.906944in}{3.231960in}}%
\pgfpathlineto{\pgfqpoint{2.601868in}{3.231960in}}%
\pgfpathquadraticcurveto{\pgfqpoint{2.632423in}{3.231960in}}{\pgfqpoint{2.632423in}{3.262515in}}%
\pgfpathlineto{\pgfqpoint{2.632423in}{4.117056in}}%
\pgfpathquadraticcurveto{\pgfqpoint{2.632423in}{4.147611in}}{\pgfqpoint{2.601868in}{4.147611in}}%
\pgfpathlineto{\pgfqpoint{0.906944in}{4.147611in}}%
\pgfpathquadraticcurveto{\pgfqpoint{0.876389in}{4.147611in}}{\pgfqpoint{0.876389in}{4.117056in}}%
\pgfpathlineto{\pgfqpoint{0.876389in}{3.262515in}}%
\pgfpathquadraticcurveto{\pgfqpoint{0.876389in}{3.231960in}}{\pgfqpoint{0.906944in}{3.231960in}}%
\pgfpathlineto{\pgfqpoint{0.906944in}{3.231960in}}%
\pgfpathclose%
\pgfusepath{stroke,fill}%
\end{pgfscope}%
\begin{pgfscope}%
\pgfsetroundcap%
\pgfsetroundjoin%
\pgfsetlinewidth{1.505625pt}%
\definecolor{currentstroke}{rgb}{0.298039,0.447059,0.690196}%
\pgfsetstrokecolor{currentstroke}%
\pgfsetdash{}{0pt}%
\pgfpathmoveto{\pgfqpoint{0.937500in}{4.030611in}}%
\pgfpathlineto{\pgfqpoint{1.090278in}{4.030611in}}%
\pgfpathlineto{\pgfqpoint{1.243056in}{4.030611in}}%
\pgfusepath{stroke}%
\end{pgfscope}%
\begin{pgfscope}%
\definecolor{textcolor}{rgb}{0.150000,0.150000,0.150000}%
\pgfsetstrokecolor{textcolor}%
\pgfsetfillcolor{textcolor}%
\pgftext[x=1.365278in,y=3.977139in,left,base]{\color{textcolor}{\sffamily\fontsize{11.000000}{13.200000}\selectfont\catcode`\^=\active\def^{\ifmmode\sp\else\^{}\fi}\catcode`\%=\active\def%{\%}resource demand}}%
\end{pgfscope}%
\begin{pgfscope}%
\pgfsetroundcap%
\pgfsetroundjoin%
\pgfsetlinewidth{1.505625pt}%
\definecolor{currentstroke}{rgb}{1.000000,0.647059,0.000000}%
\pgfsetstrokecolor{currentstroke}%
\pgfsetdash{}{0pt}%
\pgfpathmoveto{\pgfqpoint{0.937500in}{3.814499in}}%
\pgfpathlineto{\pgfqpoint{1.090278in}{3.814499in}}%
\pgfpathlineto{\pgfqpoint{1.243056in}{3.814499in}}%
\pgfusepath{stroke}%
\end{pgfscope}%
\begin{pgfscope}%
\definecolor{textcolor}{rgb}{0.150000,0.150000,0.150000}%
\pgfsetstrokecolor{textcolor}%
\pgfsetfillcolor{textcolor}%
\pgftext[x=1.365278in,y=3.761027in,left,base]{\color{textcolor}{\sffamily\fontsize{11.000000}{13.200000}\selectfont\catcode`\^=\active\def^{\ifmmode\sp\else\^{}\fi}\catcode`\%=\active\def%{\%}resource supply}}%
\end{pgfscope}%
\begin{pgfscope}%
\pgfsetbuttcap%
\pgfsetmiterjoin%
\definecolor{currentfill}{rgb}{0.172549,0.627451,0.172549}%
\pgfsetfillcolor{currentfill}%
\pgfsetfillopacity{0.300000}%
\pgfsetlinewidth{1.003750pt}%
\definecolor{currentstroke}{rgb}{0.172549,0.627451,0.172549}%
\pgfsetstrokecolor{currentstroke}%
\pgfsetstrokeopacity{0.300000}%
\pgfsetdash{}{0pt}%
\pgfpathmoveto{\pgfqpoint{0.937500in}{3.543125in}}%
\pgfpathlineto{\pgfqpoint{1.243056in}{3.543125in}}%
\pgfpathlineto{\pgfqpoint{1.243056in}{3.650069in}}%
\pgfpathlineto{\pgfqpoint{0.937500in}{3.650069in}}%
\pgfpathlineto{\pgfqpoint{0.937500in}{3.543125in}}%
\pgfpathclose%
\pgfusepath{stroke,fill}%
\end{pgfscope}%
\begin{pgfscope}%
\definecolor{textcolor}{rgb}{0.150000,0.150000,0.150000}%
\pgfsetstrokecolor{textcolor}%
\pgfsetfillcolor{textcolor}%
\pgftext[x=1.365278in,y=3.543125in,left,base]{\color{textcolor}{\sffamily\fontsize{11.000000}{13.200000}\selectfont\catcode`\^=\active\def^{\ifmmode\sp\else\^{}\fi}\catcode`\%=\active\def%{\%}overprovisioning}}%
\end{pgfscope}%
\begin{pgfscope}%
\pgfsetbuttcap%
\pgfsetmiterjoin%
\definecolor{currentfill}{rgb}{0.839216,0.152941,0.156863}%
\pgfsetfillcolor{currentfill}%
\pgfsetfillopacity{0.300000}%
\pgfsetlinewidth{1.003750pt}%
\definecolor{currentstroke}{rgb}{0.839216,0.152941,0.156863}%
\pgfsetstrokecolor{currentstroke}%
\pgfsetstrokeopacity{0.300000}%
\pgfsetdash{}{0pt}%
\pgfpathmoveto{\pgfqpoint{0.937500in}{3.325223in}}%
\pgfpathlineto{\pgfqpoint{1.243056in}{3.325223in}}%
\pgfpathlineto{\pgfqpoint{1.243056in}{3.432167in}}%
\pgfpathlineto{\pgfqpoint{0.937500in}{3.432167in}}%
\pgfpathlineto{\pgfqpoint{0.937500in}{3.325223in}}%
\pgfpathclose%
\pgfusepath{stroke,fill}%
\end{pgfscope}%
\begin{pgfscope}%
\definecolor{textcolor}{rgb}{0.150000,0.150000,0.150000}%
\pgfsetstrokecolor{textcolor}%
\pgfsetfillcolor{textcolor}%
\pgftext[x=1.365278in,y=3.325223in,left,base]{\color{textcolor}{\sffamily\fontsize{11.000000}{13.200000}\selectfont\catcode`\^=\active\def^{\ifmmode\sp\else\^{}\fi}\catcode`\%=\active\def%{\%}underprovisioning}}%
\end{pgfscope}%
\end{pgfpicture}%
\makeatother%
\endgroup%

    \caption{Resource demand and supply for a website during a typical day.}
    \label{fig:elasticity-application-scaling}
\end{figure}

Elasticity has multiple properties which are interdependent: resource elasticity, cost elasticity and quality elasticity \cite{dustdarPrinciplesElasticProcesses2011}. These properties are discussed in the following sections.

\subsection{Resource Elasticity}

The resource dimension of elasticity is mistakenly often used synonymously with elasticity. Meanwhile, resource elasticity is defined as the degree to which a system is able to adapt to workload changes by claiming and releasing resources autonomously, such that the resource supply matches the current demand as closely as possible \cite{herbstElasticityCloudComputing2013}. Another way to think of this is ``on the fly'' adaptions to load variations \cite{al-dhuraibiElasticityCloudComputing2018}.

What makes this definition easily mistaken, is that it solely considers the aquired resources and not the consequently incurred costs or changing quality.

% The ability to acquire resources as you need them and release resources when you no longer need them \cite{ElasticityAWSWellArchitected}.

\subsection{Cost Elasticity}

Cost elasticity uses cost as its main factor for elasticity decisions. One of the most popular models that build upon cost elasticity is \textit{utility computing}, also known as the \textit{pay-as-you-go} pricing model.

Amazon Web Services uses this elasticity dimension in their EC2 Spot Instance\footnote{\url{https://aws.amazon.com/ec2/spot/}}. AWS provides its unused compute capacity at a large discount to its customers. But because these capacities are volatile, the prices are not fixed but are provided through a bidding process. The potential customer tells AWS their maximum price they are willing to pay. The customer can then run their instances as long as their bidding price is smaller than AWS's Spot Instance price.

\subsection{Quality Elasticity}

Similiar to the already discussed dimensions, quality elasticity is defined as letting software services adapt their mode of operastion to current operating conditions by providing results of varying output quality \cite{larssonQualityElasticityImprovedResource2019}. This means that when resource supply is low, the output quality also may be low. Likewise, if resource supply is sufficient, the output quality will also be high.

\section{Service Level Agreements and Service Level Objectives}

In order to deliver services up to a certain standard, agreements between the service provider, typically the cloud provider, and the service consumers are made - so called \textit{Service Level Agreements (SLA)} \cite{emeakarohaLowLevelMetrics2010d}. Contained inside these SLAs are \textit{Service Level Objectives (SLO)}, which are a ``commitment to maintain a particular state of the service in a given period'' \cite{kellerWSLAFrameworkSpecifying2003}.

SLOs are measurable values, e.g. an applications CPU usage or memory consumption, that have a specified operating target. In the case that this value is violated the supporting infrastructure of the application has to be either increased or decreased. This process of increasing or decreasing resources is called elasticity, which was further discussed in \cref{sec:elasticity}.

\section{Polaris SLO Framework}
\label{sec:polaris}

The Polaris SLO Framework\footnote{\url{https://polaris-slo-cloud.github.io/polaris-slo-framework/}} is a framework that provides a way to bring high-level SLOs to the cloud. It tries to solve the limitation that modern cloud cloud providers only offer rudimentary support for high-level SLOs and customers often need to manually map them to low-level metrics such as CPU usage or memory consumption \cite{pusztaiSLOScriptNovel2021}.

The authors of this framework introduce the concept of \textit{elasticity strategies}. A elasticity strategy is defined as any sequency of actions that adjust the amount of resources provisioned for a workload, their type or the workload configuration. The workload configuration adjustment is especially noteworthy, because workloads handled by Polaris can be affected in all three elasticity dimensions.

Another unique feature of Polaris is its object model, which allows for orchestrator independence. This is achieved by encapsulating all data that is transmitted to the orchestrator into a \texttt{ApiObject} type.

Decoupling SLOs from elasticity strategies is also a feature that Polaris provides. Tight coupling is a charactaristic that is even observed in industry standard scaling mechanisms such as Kubernetes' Horizontal Pod Autoscaler\footnote{\raggedright\url{https://kubernetes.io/docs/tasks/run-application/horizontal-pod-autoscale/}}. This autoscaler provides a CPU usage SLO which can only trigger horizontal elasticity, thus adding or removing CPU resources. To achieve this decoupling, Polaris utilizes an architecture that is depicted in \cref{fig:polaris-architecture}. This allows the controllers to focus on a single task, for example calculating SLO compliance. These individual components are then mapped using a SLO mapping type.

\begin{figure}
    \centering
    \incfig{polaris-architecture}
    \caption{Architecture of the Polaris SLO framework. Metrics controllers, elasticity strategy controllers and targets are decoupled and mapped using a SLO mapping.}
    \label{fig:polaris-architecture}
\end{figure}

\section{k8ssandra}
\label{sec:k8ssandra}

Cassandra is a popular wide-column store NoSQL database that was initially developed at Facebook and later integradet into the Apache Software Foundation\footnote{\url{https://cassandra.apache.org/_/cassandra-basics.html}\label{fn:cassandra-basics}}. Its main features include being easily horizontally scalable, being fully distributed and its schema-less data approach.

Being distributed means, that Cassandra is comprised of a set of nodes. Each nodes tasks and responsibilities are identical. Data is partitioned using a partition key and is replicated between nodes. How many times data is replicated is determined by the \textit{replication factor} or \(RF\). \(RF = 3\) would therefore mean that each piece of data must exist on 3 nodes.

Distributed data also comes with a certain cost. These drawbacks are formulated in the CAP theorem \cite{foxHarvestYieldScalable1999a}. CAP stands for consistency, availability and partition tolerance and the theorem states that databases which handle data in a distributed way can only provide two of these three guarantees. Cassandra, per default, is an AP database. The upside is, that this agreement is configurable on a per-query basis.

Queries can be made to any node. Cassandra does not have a main node that takes queries, instead any node that a client connects to takes over the role of coordinator for this specific query. This coordinator node then is responsible for querying other nodes for data in other partitions. This also implies that Cassandra uses peer-to-peer communication between its nodes. This architecture is also depicted in \cref{fig:cassandra-architecture}.

\begin{figure}
    \centering
    \incfig{cassandra-architecture}
    \caption{Architecture of a 5 node Cassandra Cluster. Dotted lines represent possible communication paths.}
    \label{fig:cassandra-architecture}
\end{figure}

Another powerful feature, which makes this database particular interesting for this thesis, is its capabilities to scale. If the partition key is chosen wisely and the database is therefore able to distribute data evenly between nodes, then doubling the amount of nodes also doubles the throughput \footref{fn:cassandra-basics}

K8ssandra\footnote{\url{https://k8ssandra.io}} (pronounced: ``Kate'' +  ``Sandra'') is a open-source cloud-native distribution of Cassandra. It includes several tools for providing a data API, backup/restore processes and automated database repairs. It also includes Kubernetes custom resource definitions (CRDs) to be able to easily deploy Cassandra databases.


\chapter{Kubernetes Cluster \& Testsetup}
\label{ch:cluster_testsetup}

\section{kind}
\label{sec:kind}

\section{microk8s}
\label{sec:microk8s}

\section{Infrastructure as Code}
\label{sec:iac}

\section{Architecture}
\label{sec:architecture}



\chapter{Implementation}
\label{ch:implementation}

This chapter discusses the implementation of the metrics controller, SLO controllers and the elasticity strategy controllers. These are all components of the Polaris architecture that was described in \cref{sec:polaris}.
\section{Metrics}
\label{sec:metrics}

In order to continously monitor the k8ssandra cluster, a custom metric is introduced. The Polaris SLO framework supports two kinds of metrics, raw metrics and composed metrics, with the new metric being of the latter type. A composed metric consists of a combination of raw metrics.

The new composed metrics includes three raw metrics: average CPU utilization, average memory utilization and average write utilization. All raw metrics are calculated using the Metrics Collector for Apache Cassandra\footnote{\url{https://docs.k8ssandra.io/components/metrics-collector/}} (MCAC), which is a component included with k8ssandra. The Metrics Collector for Apache Cassandra aggregates operating system level metrics alongside with Cassandra metrics. K8ssandra also provides preconfigured Grafana dashboards\footnote{\raggedright\url{https://docs.k8ssandra.io/tasks/monitor/prometheus-grafana/grafana-dashboards.yaml}}. The following metrics were heavily influenced by the metrics that were used in these dashboards.

\subsection{Average CPU Utilization}

The average CPU utilization metric expresses the CPU utilization averaged over the target k8ssandra cluster. This metric is used for vertical elasticity. \Cref{lst:avgCpuUtilization} shows the respective PromQL query.

\begin{lstlisting}[caption={PromQL query used for the average CPU utilisation metric},
                    captionpos=b,
                    label=lst:avgCpuUtilization,
                    float]
avg by (cluster) (
  1 - (
    sum by (cluster, dc, rack, instance) (
      rate(
        collectd_cpu_total{
          cluster="polaris-k8ssandra-cluster",
          type="idle"
        }[10m]
      )
    )
    /
    sum by (cluster, dc, rack, instance) (
      rate(
        collectd_cpu_total{cluster="polaris-k8ssandra-cluster"}[10m]
      )
    )
  )
)
\end{lstlisting}

\subsection{Average Memory Utilization}

Similarly to the average CPU utilization metric, the average memory utilization metric measures the average memory consumption of the target k8ssandra cluster. It is also aimed to be used by vertical elasticity strategies. \Cref{lst:avgMemoryUtilization} shows a trimmed down version of the PromQL query used by this metric.

\begin{lstlisting}[caption={PromQL query used for the average memory utilization metric},
                    captionpos=b,
                    label=lst:avgMemoryUtilization,
                    float]
max(
    sum by (pod) (
        container_memory_working_set_bytes{cluster="",namespace="k8ssandra"}
      * on (namespace, pod) group_left (workload, workload_type)
        namespace_workload_pod:kube_pod_owner:relabel{
            namespace="k8ssandra",
            workload="dc1-default-sts",
            workload_type="statefulset"
        }
    )
  /
    sum by (pod) (
        kube_pod_container_resource_limits{
            job="kube-state-metrics",
            namespace="k8ssandra",
            resource="memory"
        }
      * on (namespace, pod) group_left (workload, workload_type)
        namespace_workload_pod:kube_pod_owner:relabel{
            namespace="k8ssandra",
            workload="dc1-default-sts",
            workload_type="statefulset"
        }
    )
)
\end{lstlisting}

\subsection{Average Write Utilization}
\label{sec:metrics-average-write-utilization}

This metric measures the average write load that one k8ssandra node experiences. It is used for horizontal scaling, which means adding nodes to the cluster. The metric consists of two separate queries which are shown in \cref{lst:writeUtilization,lst:getNodeCount}. The first query gets the total write load of the cluster and the second query calculates the current amount of active nodes. The before mentioned provided Grafana dashboards offer multiple ways of getting the node count, with the one listed here being among the simplest.

\begin{lstlisting}[caption={PromQL query used to get the current write throughput},
                    captionpos=b,
                    label=lst:writeUtilization,
                    float]
sum by (cluster, request_type) (
  rate(
    mcac_client_request_latency_total{
        cluster="polaris-k8ssandra-cluster",
        request_type="write"
    }[5m]
  )
)
\end{lstlisting}

\begin{lstlisting}[caption={PromQL query used to get the amount of nodes in the k8ssandra cluster},
                    captionpos=b,
                    label=lst:getNodeCount,
                    float]
count(
  mcac_compaction_completed_tasks{cluster="polaris-k8ssandra-cluster"} >= 0
)
\end{lstlisting}

\section{SLO Controllers}
\label{sec:slos}

SLO controllers are used to configure and evaluate specific service level objectives. These evaluations are then used to configure the respective elasticity strategies.

As part of this thesis three SLOs and their corresponding controllers were implemented. Two of these are used for the vertical and horizontal elasticity strategies. The third one, called ``k8ssandra-efficiency'' is a combination of the other ones that is used for the diagonal elasticity strategy.

\subsection{Compliance Types}
\label{sec:compliance-types}

As both the vertical and diagonal elasticity strategies expect input types other than the generic \texttt{SloCompliance}, custom types have been created. This is necessary because the elasticity strategy controllers use this data to decide what dimension has to be scaled to what extend. For example, the diagonal elasticity strategy has three parameters that are adjustable: CPU, memory and node count. These values have to be passed from the SLO controller to the elasticity strategy controller.

\texttt{K8ssandraVerticalCompliance} is a type that is used, as the name suggests, for expressing vertical compliance. It contains two fields: \texttt{currCpuCompliancePercentage} and \texttt{currMemorySloCompliancePercentage}. Both these values indicate how much the target k8ssandra clusters current resource claims comply with the SLO.

\texttt{K8ssandraCompliance} is a type that includes both of the values from \texttt{K8ssandra\-Vertical\-Compliance} and additionally a field \texttt{curr\-Horizontal\-Slo\-Compliance\-Percentage}.

All of these values are given as percentages. Both of these types also have a field \texttt{tolerance}. By using all of these values it is possible to determine if scaling actions are required at any given time. 

\subsection{API Object}

To enable the Polaris SLO framework to interact with the k8ssandra CRD subtype of \texttt{ApiObject} was used. \texttt{ApiObject} is used for any object that should be added, read, changed or deleted by Polaris using the orchestrator's API.

Because of this use of a subtype the framework is also able to automatically transform fields. Kubernetes for example uses two separate fields for resources, requests and limits. Polaris on the other hand simply uses ``resources'' as orchestrator details are abstracted. This conversion from requests and limits to resources is handled by Polaris through annotating the respective fields with \texttt{PolarisType}.

\section{Elasticity Strategies}

The elasticity strategy controllers perform the actions that are required to scale the workload. All elasticity strategy controllers must implement the interface \texttt{Elasticity\-Strategy\-Controller} which requires the implementation of four methods: \texttt{check\-If\-Action\-Needed}, \texttt{execute}, \texttt{on\-Elasticity\-Strategy\-Deleted} and \texttt{on\-Destroy}, with the latter two being optional.

These elasticity strategy controllers are called with the appropriate \texttt{SloOutput} during the SLO control loop \cite{pusztaiNovelMiddlewareEfficiently2021a}.

As part of this thesis, three elasticity strategies for k8ssandra have been implemented. One each for vertical and horizontal elasticity and one that combines these two into a diagonal elasticity strategy.

\subsection{Vertical Elasticity Strategy}
\label{sec:vertical-elasticity}

The vertical elasticity strategy controller is a subtype of \texttt{Elasticity\-Strategy\-Controller}. It expects \texttt{K8ssandra\-Vertical\-Slo\-Compliance}, as described in \cref{sec:compliance-types}, as input. The controller uses the CPU and memory compliance value to scale the current resources accordingly. If the current CPU and memory compliance is the given tolerance range, no scaling is performed by the elasticity strategy controller.

\subsection{Horizontal Elasticity Strategy}
\label{sec:horizontal-elasticity}

The horizontal elasticity strategy controller is able to use the \texttt{Slo\-Compliance\-E\-las\-tic\-i\-ty\-Strategy\-Controller\-Base} as its supertype as it expects \texttt{Slo\-Compliance} as input. This reduces the amount of boilerplate code and therefore also complies with the ``Don't repeat yourself'' (DRY) principle. Again, the elasticity strategy controller performs a scaling action if the compliance is out of range of the set tolerance.

The here implemented version of horizontal scaling \textit{only} performs scale-out. The reason for this is that for scaling-in databases, special considerations have to be made. This is especially true for storage. When, for example, reducing the node count in a Cassandra cluster from 3 to 2, the amount of stored data stays the same, therefore it is possible that the stored data per node increases. This, however, was considered out of scope of this thesis.

\subsection{Diagonal Elasticity Strategy}
\label{sec:diagonal-elasticity}

% eigenen abstract ControllerBase die ElasticityStrategyController implementiert
% kombiniert im prinzip die beiden vorher besprochenen strategien

The third and last elasticity strategy controller combines the controller described in \cref{sec:vertical-elasticity,sec:horizontal-elasticity}.

Again, because this controller expects a different input that \texttt{SloCompliance}, \texttt{K8\-ssan\-dra\-Slo\-Compliance} it is not possible to use any of the provided controller bases. Therefore a custom controller base that expects this input has been implemented. The diagonal elasticity controller then is a subtype of this newly created controller base.

Due to a normalization process that takes place after the actual scaling, it is possible that even if the elasticity strategy is executed no update to the target is made. This is because there are certain limits that are set statically that have to be adhered to. CPU and memory have physical limits as there is no infite amount of resources that can be claimed by the target. Similarly, a lower boundary is also in place because even if the current utilization is very low, a minimum amount of resources is necessary to guarantee normal operation.


\chapter{Evaluation}
\label{ch:evaluation}

This chapter first introduces the setup that was used for evaluating the different elasticity strategies. Then the results of different tests are presented and discussed.

\section{Test Setup}
\label{sec:testsetup}

In order to test the different elasticity strategies a test environment has to be set up. It was decided to create three virtual machines (VM) that will form a Kubernetes cluster. Because of its ease of use microk8s was chosen as distribution\footnote{\url{https://microk8s.io/}}. All three virtual machines were assigned 10 vCPUs and 10GB of memory. One VM acts as the Kubernetes control plane while the other two join the cluster as worker nodes.

Everything that was deployed into the Kubernetes cluster was built using the infrastructure as code (IaC) tool HashiCorp Terraform\footnote{\url{https://www.terraform.io/}}. This enables rapid changes and reproducibility. Deployed resources include the kube-prometheus-stack\footnote{\raggedright\url{https://artifacthub.io/packages/helm/prometheus-community/kube-prometheus-stack}} for monitoring, the k8ssandra-operator\footnote{\url{https://docs.k8ssandra.io/components/k8ssandra-operator/}} for managing K8ssandra clusters and a definition for a K8ssandra cluster. Additionally, the Grafana dashboards mentioned in \Cref{sec:metrics} are also deployed using Terraform.

\Cref{lst:k8c} illustrates a minimal definition of a 3 node K8ssandra cluster. Each node has resource limits of 800 millicpu and 6000MB of memory and 3GiB storage space.

\begin{lstlisting}[caption={Minimal example of a K8ssandraCluster definition.},
                label=lst:k8c,
                captionpos=b,
                float]
apiVersion: k8ssandra.io/v1alpha1
kind: K8ssandraCluster
metadata:
  name: polaris-test-cluster
  namespace: k8ssandra
spec:
  cassandra:
    resources:
      limits:
        cpu: 800m
        memory: 6000M
    datacenters:
      - metadata:
          name: dc1
        size: 3
        storageConfig:
          cassandraDataVolumeClaimSpec:
            resources:
              requests:
                storage: 3Gi
\end{lstlisting}

\section{Benchmarks}

In the following sections, different test scenarios will be discussed. To let K8ssandra experience load, the built-in stress testing tool \texttt{cassandra-stress} was used\footnote{\raggedright\url{https://cassandra.apache.org/doc/stable/cassandra/tools/cassandra_stress.html}}. This tool is able to perform benchmarks and load-test Cassandra clusters and is part of the default Cassandra installation. Different operation modes, such as read-only, write-only or mixed, are available.

\Cref{sec:stress-testing} introduces the load testing tool and sets a baseline scenario. Based on these results, the following sections will add vertical, horizontal and, finally, diagonal elasticity in order to for increased throughput and resource efficiency. All of these different elasticity strategy tests build upon the same configuration that was used during the baseline tests.

\subsection{Stress Testing}
\label{sec:stress-testing}

To set a baseline, three different K8ssandra cluster setups, with one, two, and three nodes respectively, have been stress tested using \texttt{cassandra-stress}. All nodes were provisioned with limits of 2 CPUs and 6GB of memory. The amount of write requests that the tool will make is set to be 1,000,000, the exact call is listed in \Cref{lst:stress-1000000writes}. During these runs, no elasticity was involved.

\begin{lstlisting}[caption={Call of the \texttt{cassandra-stress} tool that triggers 1000000 writes.},
                    captionpos=b,
                    label=lst:stress-1000000writes,
                    float]
./cassandra-stress write n=1000000 -mode native cql3
\end{lstlisting}

The results of these tests are depicted in \Cref{fig:stress-1000000writes-1node,fig:stress-1000000writes-2node,,fig:stress-1000000writes-3node}. The write throughput increases with the amount of nodes, but not linearly. This, however, was to be expected as \texttt{cassandra-stress} does not partition data in a way that favours linear scalability. The average write throughputs of these different clusters can be seen in \Cref{tab:stress-1000000writes-ops}. It was not only shown by the Apache Software Foundation, the developers of Cassandra, but also by industry leading companies such as Netflix, that horizontal scaling allows K8ssandra to essentially scale its throughput linearly \cite{cockroftBenchmarkingCassandraScalability2011}.

\begin{table}[H]
\centering
\begin{tabular}{|l|l|l|}
\hline
\textbf{Cluster size} & \textbf{operations/s} & \textbf{Time to complete} \\ \hline
1                     & 12,514                 & 2m38s                     \\ \hline
2                     & 13,142                 & 1m57s                     \\ \hline
3                     & 14,318                 & 1m50s                     \\ \hline
\end{tabular}
\caption{Average write throughput for different K8ssandra clusters. With increasing cluster size the throughput also increases}
\label{tab:stress-1000000writes-ops}
\end{table}

\begin{figure}
    \centering
    %% Creator: Matplotlib, PGF backend
%%
%% To include the figure in your LaTeX document, write
%%   \input{<filename>.pgf}
%%
%% Make sure the required packages are loaded in your preamble
%%   \usepackage{pgf}
%%
%% Also ensure that all the required font packages are loaded; for instance,
%% the lmodern package is sometimes necessary when using math font.
%%   \usepackage{lmodern}
%%
%% Figures using additional raster images can only be included by \input if
%% they are in the same directory as the main LaTeX file. For loading figures
%% from other directories you can use the `import` package
%%   \usepackage{import}
%%
%% and then include the figures with
%%   \import{<path to file>}{<filename>.pgf}
%%
%% Matplotlib used the following preamble
%%   \def\mathdefault#1{#1}
%%   \everymath=\expandafter{\the\everymath\displaystyle}
%%   
%%   \usepackage{fontspec}
%%   \setmainfont{DejaVuSerif.ttf}[Path=\detokenize{/Users/nkratky/private/polaris-elasticity-strategies/test/scripts/.venv/lib/python3.11/site-packages/matplotlib/mpl-data/fonts/ttf/}]
%%   \setsansfont{Arial.ttf}[Path=\detokenize{/System/Library/Fonts/Supplemental/}]
%%   \setmonofont{DejaVuSansMono.ttf}[Path=\detokenize{/Users/nkratky/private/polaris-elasticity-strategies/test/scripts/.venv/lib/python3.11/site-packages/matplotlib/mpl-data/fonts/ttf/}]
%%   \makeatletter\@ifpackageloaded{underscore}{}{\usepackage[strings]{underscore}}\makeatother
%%
\begingroup%
\makeatletter%
\begin{pgfpicture}%
\pgfpathrectangle{\pgfpointorigin}{\pgfqpoint{5.600000in}{2.500000in}}%
\pgfusepath{use as bounding box, clip}%
\begin{pgfscope}%
\pgfsetbuttcap%
\pgfsetmiterjoin%
\definecolor{currentfill}{rgb}{1.000000,1.000000,1.000000}%
\pgfsetfillcolor{currentfill}%
\pgfsetlinewidth{0.000000pt}%
\definecolor{currentstroke}{rgb}{1.000000,1.000000,1.000000}%
\pgfsetstrokecolor{currentstroke}%
\pgfsetdash{}{0pt}%
\pgfpathmoveto{\pgfqpoint{0.000000in}{0.000000in}}%
\pgfpathlineto{\pgfqpoint{5.600000in}{0.000000in}}%
\pgfpathlineto{\pgfqpoint{5.600000in}{2.500000in}}%
\pgfpathlineto{\pgfqpoint{0.000000in}{2.500000in}}%
\pgfpathlineto{\pgfqpoint{0.000000in}{0.000000in}}%
\pgfpathclose%
\pgfusepath{fill}%
\end{pgfscope}%
\begin{pgfscope}%
\pgfsetbuttcap%
\pgfsetmiterjoin%
\definecolor{currentfill}{rgb}{0.917647,0.917647,0.949020}%
\pgfsetfillcolor{currentfill}%
\pgfsetlinewidth{0.000000pt}%
\definecolor{currentstroke}{rgb}{0.000000,0.000000,0.000000}%
\pgfsetstrokecolor{currentstroke}%
\pgfsetstrokeopacity{0.000000}%
\pgfsetdash{}{0pt}%
\pgfpathmoveto{\pgfqpoint{0.948751in}{0.663635in}}%
\pgfpathlineto{\pgfqpoint{5.420000in}{0.663635in}}%
\pgfpathlineto{\pgfqpoint{5.420000in}{2.320000in}}%
\pgfpathlineto{\pgfqpoint{0.948751in}{2.320000in}}%
\pgfpathlineto{\pgfqpoint{0.948751in}{0.663635in}}%
\pgfpathclose%
\pgfusepath{fill}%
\end{pgfscope}%
\begin{pgfscope}%
\pgfpathrectangle{\pgfqpoint{0.948751in}{0.663635in}}{\pgfqpoint{4.471249in}{1.656365in}}%
\pgfusepath{clip}%
\pgfsetroundcap%
\pgfsetroundjoin%
\pgfsetlinewidth{1.003750pt}%
\definecolor{currentstroke}{rgb}{1.000000,1.000000,1.000000}%
\pgfsetstrokecolor{currentstroke}%
\pgfsetdash{}{0pt}%
\pgfpathmoveto{\pgfqpoint{1.151990in}{0.663635in}}%
\pgfpathlineto{\pgfqpoint{1.151990in}{2.320000in}}%
\pgfusepath{stroke}%
\end{pgfscope}%
\begin{pgfscope}%
\definecolor{textcolor}{rgb}{0.150000,0.150000,0.150000}%
\pgfsetstrokecolor{textcolor}%
\pgfsetfillcolor{textcolor}%
\pgftext[x=1.151990in,y=0.531691in,,top]{\color{textcolor}{\sffamily\fontsize{11.000000}{13.200000}\selectfont\catcode`\^=\active\def^{\ifmmode\sp\else\^{}\fi}\catcode`\%=\active\def%{\%}0}}%
\end{pgfscope}%
\begin{pgfscope}%
\pgfpathrectangle{\pgfqpoint{0.948751in}{0.663635in}}{\pgfqpoint{4.471249in}{1.656365in}}%
\pgfusepath{clip}%
\pgfsetroundcap%
\pgfsetroundjoin%
\pgfsetlinewidth{1.003750pt}%
\definecolor{currentstroke}{rgb}{1.000000,1.000000,1.000000}%
\pgfsetstrokecolor{currentstroke}%
\pgfsetdash{}{0pt}%
\pgfpathmoveto{\pgfqpoint{1.829452in}{0.663635in}}%
\pgfpathlineto{\pgfqpoint{1.829452in}{2.320000in}}%
\pgfusepath{stroke}%
\end{pgfscope}%
\begin{pgfscope}%
\definecolor{textcolor}{rgb}{0.150000,0.150000,0.150000}%
\pgfsetstrokecolor{textcolor}%
\pgfsetfillcolor{textcolor}%
\pgftext[x=1.829452in,y=0.531691in,,top]{\color{textcolor}{\sffamily\fontsize{11.000000}{13.200000}\selectfont\catcode`\^=\active\def^{\ifmmode\sp\else\^{}\fi}\catcode`\%=\active\def%{\%}20}}%
\end{pgfscope}%
\begin{pgfscope}%
\pgfpathrectangle{\pgfqpoint{0.948751in}{0.663635in}}{\pgfqpoint{4.471249in}{1.656365in}}%
\pgfusepath{clip}%
\pgfsetroundcap%
\pgfsetroundjoin%
\pgfsetlinewidth{1.003750pt}%
\definecolor{currentstroke}{rgb}{1.000000,1.000000,1.000000}%
\pgfsetstrokecolor{currentstroke}%
\pgfsetdash{}{0pt}%
\pgfpathmoveto{\pgfqpoint{2.506914in}{0.663635in}}%
\pgfpathlineto{\pgfqpoint{2.506914in}{2.320000in}}%
\pgfusepath{stroke}%
\end{pgfscope}%
\begin{pgfscope}%
\definecolor{textcolor}{rgb}{0.150000,0.150000,0.150000}%
\pgfsetstrokecolor{textcolor}%
\pgfsetfillcolor{textcolor}%
\pgftext[x=2.506914in,y=0.531691in,,top]{\color{textcolor}{\sffamily\fontsize{11.000000}{13.200000}\selectfont\catcode`\^=\active\def^{\ifmmode\sp\else\^{}\fi}\catcode`\%=\active\def%{\%}40}}%
\end{pgfscope}%
\begin{pgfscope}%
\pgfpathrectangle{\pgfqpoint{0.948751in}{0.663635in}}{\pgfqpoint{4.471249in}{1.656365in}}%
\pgfusepath{clip}%
\pgfsetroundcap%
\pgfsetroundjoin%
\pgfsetlinewidth{1.003750pt}%
\definecolor{currentstroke}{rgb}{1.000000,1.000000,1.000000}%
\pgfsetstrokecolor{currentstroke}%
\pgfsetdash{}{0pt}%
\pgfpathmoveto{\pgfqpoint{3.184376in}{0.663635in}}%
\pgfpathlineto{\pgfqpoint{3.184376in}{2.320000in}}%
\pgfusepath{stroke}%
\end{pgfscope}%
\begin{pgfscope}%
\definecolor{textcolor}{rgb}{0.150000,0.150000,0.150000}%
\pgfsetstrokecolor{textcolor}%
\pgfsetfillcolor{textcolor}%
\pgftext[x=3.184376in,y=0.531691in,,top]{\color{textcolor}{\sffamily\fontsize{11.000000}{13.200000}\selectfont\catcode`\^=\active\def^{\ifmmode\sp\else\^{}\fi}\catcode`\%=\active\def%{\%}60}}%
\end{pgfscope}%
\begin{pgfscope}%
\pgfpathrectangle{\pgfqpoint{0.948751in}{0.663635in}}{\pgfqpoint{4.471249in}{1.656365in}}%
\pgfusepath{clip}%
\pgfsetroundcap%
\pgfsetroundjoin%
\pgfsetlinewidth{1.003750pt}%
\definecolor{currentstroke}{rgb}{1.000000,1.000000,1.000000}%
\pgfsetstrokecolor{currentstroke}%
\pgfsetdash{}{0pt}%
\pgfpathmoveto{\pgfqpoint{3.861838in}{0.663635in}}%
\pgfpathlineto{\pgfqpoint{3.861838in}{2.320000in}}%
\pgfusepath{stroke}%
\end{pgfscope}%
\begin{pgfscope}%
\definecolor{textcolor}{rgb}{0.150000,0.150000,0.150000}%
\pgfsetstrokecolor{textcolor}%
\pgfsetfillcolor{textcolor}%
\pgftext[x=3.861838in,y=0.531691in,,top]{\color{textcolor}{\sffamily\fontsize{11.000000}{13.200000}\selectfont\catcode`\^=\active\def^{\ifmmode\sp\else\^{}\fi}\catcode`\%=\active\def%{\%}80}}%
\end{pgfscope}%
\begin{pgfscope}%
\pgfpathrectangle{\pgfqpoint{0.948751in}{0.663635in}}{\pgfqpoint{4.471249in}{1.656365in}}%
\pgfusepath{clip}%
\pgfsetroundcap%
\pgfsetroundjoin%
\pgfsetlinewidth{1.003750pt}%
\definecolor{currentstroke}{rgb}{1.000000,1.000000,1.000000}%
\pgfsetstrokecolor{currentstroke}%
\pgfsetdash{}{0pt}%
\pgfpathmoveto{\pgfqpoint{4.539299in}{0.663635in}}%
\pgfpathlineto{\pgfqpoint{4.539299in}{2.320000in}}%
\pgfusepath{stroke}%
\end{pgfscope}%
\begin{pgfscope}%
\definecolor{textcolor}{rgb}{0.150000,0.150000,0.150000}%
\pgfsetstrokecolor{textcolor}%
\pgfsetfillcolor{textcolor}%
\pgftext[x=4.539299in,y=0.531691in,,top]{\color{textcolor}{\sffamily\fontsize{11.000000}{13.200000}\selectfont\catcode`\^=\active\def^{\ifmmode\sp\else\^{}\fi}\catcode`\%=\active\def%{\%}100}}%
\end{pgfscope}%
\begin{pgfscope}%
\pgfpathrectangle{\pgfqpoint{0.948751in}{0.663635in}}{\pgfqpoint{4.471249in}{1.656365in}}%
\pgfusepath{clip}%
\pgfsetroundcap%
\pgfsetroundjoin%
\pgfsetlinewidth{1.003750pt}%
\definecolor{currentstroke}{rgb}{1.000000,1.000000,1.000000}%
\pgfsetstrokecolor{currentstroke}%
\pgfsetdash{}{0pt}%
\pgfpathmoveto{\pgfqpoint{5.216761in}{0.663635in}}%
\pgfpathlineto{\pgfqpoint{5.216761in}{2.320000in}}%
\pgfusepath{stroke}%
\end{pgfscope}%
\begin{pgfscope}%
\definecolor{textcolor}{rgb}{0.150000,0.150000,0.150000}%
\pgfsetstrokecolor{textcolor}%
\pgfsetfillcolor{textcolor}%
\pgftext[x=5.216761in,y=0.531691in,,top]{\color{textcolor}{\sffamily\fontsize{11.000000}{13.200000}\selectfont\catcode`\^=\active\def^{\ifmmode\sp\else\^{}\fi}\catcode`\%=\active\def%{\%}120}}%
\end{pgfscope}%
\begin{pgfscope}%
\definecolor{textcolor}{rgb}{0.150000,0.150000,0.150000}%
\pgfsetstrokecolor{textcolor}%
\pgfsetfillcolor{textcolor}%
\pgftext[x=3.184376in,y=0.336413in,,top]{\color{textcolor}{\sffamily\fontsize{12.000000}{14.400000}\selectfont\catcode`\^=\active\def^{\ifmmode\sp\else\^{}\fi}\catcode`\%=\active\def%{\%}Time (s)}}%
\end{pgfscope}%
\begin{pgfscope}%
\pgfpathrectangle{\pgfqpoint{0.948751in}{0.663635in}}{\pgfqpoint{4.471249in}{1.656365in}}%
\pgfusepath{clip}%
\pgfsetroundcap%
\pgfsetroundjoin%
\pgfsetlinewidth{1.003750pt}%
\definecolor{currentstroke}{rgb}{1.000000,1.000000,1.000000}%
\pgfsetstrokecolor{currentstroke}%
\pgfsetdash{}{0pt}%
\pgfpathmoveto{\pgfqpoint{0.948751in}{0.738925in}}%
\pgfpathlineto{\pgfqpoint{5.420000in}{0.738925in}}%
\pgfusepath{stroke}%
\end{pgfscope}%
\begin{pgfscope}%
\definecolor{textcolor}{rgb}{0.150000,0.150000,0.150000}%
\pgfsetstrokecolor{textcolor}%
\pgfsetfillcolor{textcolor}%
\pgftext[x=0.731839in, y=0.684244in, left, base]{\color{textcolor}{\sffamily\fontsize{11.000000}{13.200000}\selectfont\catcode`\^=\active\def^{\ifmmode\sp\else\^{}\fi}\catcode`\%=\active\def%{\%}0}}%
\end{pgfscope}%
\begin{pgfscope}%
\pgfpathrectangle{\pgfqpoint{0.948751in}{0.663635in}}{\pgfqpoint{4.471249in}{1.656365in}}%
\pgfusepath{clip}%
\pgfsetroundcap%
\pgfsetroundjoin%
\pgfsetlinewidth{1.003750pt}%
\definecolor{currentstroke}{rgb}{1.000000,1.000000,1.000000}%
\pgfsetstrokecolor{currentstroke}%
\pgfsetdash{}{0pt}%
\pgfpathmoveto{\pgfqpoint{0.948751in}{1.339773in}}%
\pgfpathlineto{\pgfqpoint{5.420000in}{1.339773in}}%
\pgfusepath{stroke}%
\end{pgfscope}%
\begin{pgfscope}%
\definecolor{textcolor}{rgb}{0.150000,0.150000,0.150000}%
\pgfsetstrokecolor{textcolor}%
\pgfsetfillcolor{textcolor}%
\pgftext[x=0.391968in, y=1.285092in, left, base]{\color{textcolor}{\sffamily\fontsize{11.000000}{13.200000}\selectfont\catcode`\^=\active\def^{\ifmmode\sp\else\^{}\fi}\catcode`\%=\active\def%{\%}10000}}%
\end{pgfscope}%
\begin{pgfscope}%
\pgfpathrectangle{\pgfqpoint{0.948751in}{0.663635in}}{\pgfqpoint{4.471249in}{1.656365in}}%
\pgfusepath{clip}%
\pgfsetroundcap%
\pgfsetroundjoin%
\pgfsetlinewidth{1.003750pt}%
\definecolor{currentstroke}{rgb}{1.000000,1.000000,1.000000}%
\pgfsetstrokecolor{currentstroke}%
\pgfsetdash{}{0pt}%
\pgfpathmoveto{\pgfqpoint{0.948751in}{1.940621in}}%
\pgfpathlineto{\pgfqpoint{5.420000in}{1.940621in}}%
\pgfusepath{stroke}%
\end{pgfscope}%
\begin{pgfscope}%
\definecolor{textcolor}{rgb}{0.150000,0.150000,0.150000}%
\pgfsetstrokecolor{textcolor}%
\pgfsetfillcolor{textcolor}%
\pgftext[x=0.391968in, y=1.885941in, left, base]{\color{textcolor}{\sffamily\fontsize{11.000000}{13.200000}\selectfont\catcode`\^=\active\def^{\ifmmode\sp\else\^{}\fi}\catcode`\%=\active\def%{\%}20000}}%
\end{pgfscope}%
\begin{pgfscope}%
\definecolor{textcolor}{rgb}{0.150000,0.150000,0.150000}%
\pgfsetstrokecolor{textcolor}%
\pgfsetfillcolor{textcolor}%
\pgftext[x=0.336413in,y=1.491818in,,bottom,rotate=90.000000]{\color{textcolor}{\sffamily\fontsize{12.000000}{14.400000}\selectfont\catcode`\^=\active\def^{\ifmmode\sp\else\^{}\fi}\catcode`\%=\active\def%{\%}Writes (op/s)}}%
\end{pgfscope}%
\begin{pgfscope}%
\pgfpathrectangle{\pgfqpoint{0.948751in}{0.663635in}}{\pgfqpoint{4.471249in}{1.656365in}}%
\pgfusepath{clip}%
\pgfsetroundcap%
\pgfsetroundjoin%
\pgfsetlinewidth{1.505625pt}%
\definecolor{currentstroke}{rgb}{0.298039,0.447059,0.690196}%
\pgfsetstrokecolor{currentstroke}%
\pgfsetdash{}{0pt}%
\pgfpathmoveto{\pgfqpoint{1.151990in}{0.738925in}}%
\pgfpathlineto{\pgfqpoint{1.321355in}{0.738925in}}%
\pgfpathlineto{\pgfqpoint{1.490721in}{0.738925in}}%
\pgfpathlineto{\pgfqpoint{1.660086in}{1.222487in}}%
\pgfpathlineto{\pgfqpoint{1.829452in}{1.352691in}}%
\pgfpathlineto{\pgfqpoint{1.998817in}{1.482895in}}%
\pgfpathlineto{\pgfqpoint{2.168183in}{1.607451in}}%
\pgfpathlineto{\pgfqpoint{2.337548in}{1.694694in}}%
\pgfpathlineto{\pgfqpoint{2.506914in}{1.694694in}}%
\pgfpathlineto{\pgfqpoint{2.676279in}{1.765955in}}%
\pgfpathlineto{\pgfqpoint{2.845645in}{1.837275in}}%
\pgfpathlineto{\pgfqpoint{3.015010in}{1.837275in}}%
\pgfpathlineto{\pgfqpoint{3.184376in}{1.946450in}}%
\pgfpathlineto{\pgfqpoint{3.353741in}{2.055624in}}%
\pgfpathlineto{\pgfqpoint{3.523107in}{2.055624in}}%
\pgfpathlineto{\pgfqpoint{3.692472in}{2.150137in}}%
\pgfpathlineto{\pgfqpoint{3.861838in}{2.244711in}}%
\pgfpathlineto{\pgfqpoint{4.031203in}{2.244711in}}%
\pgfpathlineto{\pgfqpoint{4.200569in}{1.724496in}}%
\pgfpathlineto{\pgfqpoint{4.369934in}{1.204282in}}%
\pgfpathlineto{\pgfqpoint{4.539299in}{1.204282in}}%
\pgfpathlineto{\pgfqpoint{4.708665in}{0.971573in}}%
\pgfpathlineto{\pgfqpoint{4.878030in}{0.738925in}}%
\pgfpathlineto{\pgfqpoint{5.047396in}{0.738925in}}%
\pgfpathlineto{\pgfqpoint{5.216761in}{0.738925in}}%
\pgfusepath{stroke}%
\end{pgfscope}%
\begin{pgfscope}%
\pgfpathrectangle{\pgfqpoint{0.948751in}{0.663635in}}{\pgfqpoint{4.471249in}{1.656365in}}%
\pgfusepath{clip}%
\pgfsetroundcap%
\pgfsetroundjoin%
\pgfsetlinewidth{1.505625pt}%
\definecolor{currentstroke}{rgb}{1.000000,0.000000,0.000000}%
\pgfsetstrokecolor{currentstroke}%
\pgfsetdash{}{0pt}%
\pgfpathmoveto{\pgfqpoint{0.948751in}{1.469234in}}%
\pgfpathlineto{\pgfqpoint{5.420000in}{1.469234in}}%
\pgfusepath{stroke}%
\end{pgfscope}%
\begin{pgfscope}%
\pgfsetrectcap%
\pgfsetmiterjoin%
\pgfsetlinewidth{1.254687pt}%
\definecolor{currentstroke}{rgb}{1.000000,1.000000,1.000000}%
\pgfsetstrokecolor{currentstroke}%
\pgfsetdash{}{0pt}%
\pgfpathmoveto{\pgfqpoint{0.948751in}{0.663635in}}%
\pgfpathlineto{\pgfqpoint{0.948751in}{2.320000in}}%
\pgfusepath{stroke}%
\end{pgfscope}%
\begin{pgfscope}%
\pgfsetrectcap%
\pgfsetmiterjoin%
\pgfsetlinewidth{1.254687pt}%
\definecolor{currentstroke}{rgb}{1.000000,1.000000,1.000000}%
\pgfsetstrokecolor{currentstroke}%
\pgfsetdash{}{0pt}%
\pgfpathmoveto{\pgfqpoint{5.420000in}{0.663635in}}%
\pgfpathlineto{\pgfqpoint{5.420000in}{2.320000in}}%
\pgfusepath{stroke}%
\end{pgfscope}%
\begin{pgfscope}%
\pgfsetrectcap%
\pgfsetmiterjoin%
\pgfsetlinewidth{1.254687pt}%
\definecolor{currentstroke}{rgb}{1.000000,1.000000,1.000000}%
\pgfsetstrokecolor{currentstroke}%
\pgfsetdash{}{0pt}%
\pgfpathmoveto{\pgfqpoint{0.948751in}{0.663635in}}%
\pgfpathlineto{\pgfqpoint{5.420000in}{0.663635in}}%
\pgfusepath{stroke}%
\end{pgfscope}%
\begin{pgfscope}%
\pgfsetrectcap%
\pgfsetmiterjoin%
\pgfsetlinewidth{1.254687pt}%
\definecolor{currentstroke}{rgb}{1.000000,1.000000,1.000000}%
\pgfsetstrokecolor{currentstroke}%
\pgfsetdash{}{0pt}%
\pgfpathmoveto{\pgfqpoint{0.948751in}{2.320000in}}%
\pgfpathlineto{\pgfqpoint{5.420000in}{2.320000in}}%
\pgfusepath{stroke}%
\end{pgfscope}%
\begin{pgfscope}%
\pgfsetbuttcap%
\pgfsetmiterjoin%
\definecolor{currentfill}{rgb}{0.917647,0.917647,0.949020}%
\pgfsetfillcolor{currentfill}%
\pgfsetfillopacity{0.800000}%
\pgfsetlinewidth{1.003750pt}%
\definecolor{currentstroke}{rgb}{0.800000,0.800000,0.800000}%
\pgfsetstrokecolor{currentstroke}%
\pgfsetstrokeopacity{0.800000}%
\pgfsetdash{}{0pt}%
\pgfpathmoveto{\pgfqpoint{1.026529in}{2.072637in}}%
\pgfpathlineto{\pgfqpoint{3.179448in}{2.072637in}}%
\pgfpathquadraticcurveto{\pgfqpoint{3.201670in}{2.072637in}}{\pgfqpoint{3.201670in}{2.094859in}}%
\pgfpathlineto{\pgfqpoint{3.201670in}{2.242222in}}%
\pgfpathquadraticcurveto{\pgfqpoint{3.201670in}{2.264444in}}{\pgfqpoint{3.179448in}{2.264444in}}%
\pgfpathlineto{\pgfqpoint{1.026529in}{2.264444in}}%
\pgfpathquadraticcurveto{\pgfqpoint{1.004307in}{2.264444in}}{\pgfqpoint{1.004307in}{2.242222in}}%
\pgfpathlineto{\pgfqpoint{1.004307in}{2.094859in}}%
\pgfpathquadraticcurveto{\pgfqpoint{1.004307in}{2.072637in}}{\pgfqpoint{1.026529in}{2.072637in}}%
\pgfpathlineto{\pgfqpoint{1.026529in}{2.072637in}}%
\pgfpathclose%
\pgfusepath{stroke,fill}%
\end{pgfscope}%
\begin{pgfscope}%
\pgfsetroundcap%
\pgfsetroundjoin%
\pgfsetlinewidth{1.505625pt}%
\definecolor{currentstroke}{rgb}{1.000000,0.000000,0.000000}%
\pgfsetstrokecolor{currentstroke}%
\pgfsetdash{}{0pt}%
\pgfpathmoveto{\pgfqpoint{1.048751in}{2.179353in}}%
\pgfpathlineto{\pgfqpoint{1.159862in}{2.179353in}}%
\pgfpathlineto{\pgfqpoint{1.270973in}{2.179353in}}%
\pgfusepath{stroke}%
\end{pgfscope}%
\begin{pgfscope}%
\definecolor{textcolor}{rgb}{0.150000,0.150000,0.150000}%
\pgfsetstrokecolor{textcolor}%
\pgfsetfillcolor{textcolor}%
\pgftext[x=1.359862in,y=2.140464in,left,base]{\color{textcolor}{\sffamily\fontsize{8.000000}{9.600000}\selectfont\catcode`\^=\active\def^{\ifmmode\sp\else\^{}\fi}\catcode`\%=\active\def%{\%}average write operations per second}}%
\end{pgfscope}%
\end{pgfpicture}%
\makeatother%
\endgroup%

    \caption{Stress test of 1 node with 1000000 writes}
    \label{fig:stress-1000000writes-1node}
\end{figure}

\begin{figure}
    \centering
    %% Creator: Matplotlib, PGF backend
%%
%% To include the figure in your LaTeX document, write
%%   \input{<filename>.pgf}
%%
%% Make sure the required packages are loaded in your preamble
%%   \usepackage{pgf}
%%
%% Also ensure that all the required font packages are loaded; for instance,
%% the lmodern package is sometimes necessary when using math font.
%%   \usepackage{lmodern}
%%
%% Figures using additional raster images can only be included by \input if
%% they are in the same directory as the main LaTeX file. For loading figures
%% from other directories you can use the `import` package
%%   \usepackage{import}
%%
%% and then include the figures with
%%   \import{<path to file>}{<filename>.pgf}
%%
%% Matplotlib used the following preamble
%%   \def\mathdefault#1{#1}
%%   \everymath=\expandafter{\the\everymath\displaystyle}
%%   
%%   \usepackage{fontspec}
%%   \setmainfont{DejaVuSerif.ttf}[Path=\detokenize{/Users/nkratky/private/polaris-elasticity-strategies/test/scripts/.venv/lib/python3.11/site-packages/matplotlib/mpl-data/fonts/ttf/}]
%%   \setsansfont{Arial.ttf}[Path=\detokenize{/System/Library/Fonts/Supplemental/}]
%%   \setmonofont{DejaVuSansMono.ttf}[Path=\detokenize{/Users/nkratky/private/polaris-elasticity-strategies/test/scripts/.venv/lib/python3.11/site-packages/matplotlib/mpl-data/fonts/ttf/}]
%%   \makeatletter\@ifpackageloaded{underscore}{}{\usepackage[strings]{underscore}}\makeatother
%%
\begingroup%
\makeatletter%
\begin{pgfpicture}%
\pgfpathrectangle{\pgfpointorigin}{\pgfqpoint{5.600000in}{2.500000in}}%
\pgfusepath{use as bounding box, clip}%
\begin{pgfscope}%
\pgfsetbuttcap%
\pgfsetmiterjoin%
\definecolor{currentfill}{rgb}{1.000000,1.000000,1.000000}%
\pgfsetfillcolor{currentfill}%
\pgfsetlinewidth{0.000000pt}%
\definecolor{currentstroke}{rgb}{1.000000,1.000000,1.000000}%
\pgfsetstrokecolor{currentstroke}%
\pgfsetdash{}{0pt}%
\pgfpathmoveto{\pgfqpoint{0.000000in}{0.000000in}}%
\pgfpathlineto{\pgfqpoint{5.600000in}{0.000000in}}%
\pgfpathlineto{\pgfqpoint{5.600000in}{2.500000in}}%
\pgfpathlineto{\pgfqpoint{0.000000in}{2.500000in}}%
\pgfpathlineto{\pgfqpoint{0.000000in}{0.000000in}}%
\pgfpathclose%
\pgfusepath{fill}%
\end{pgfscope}%
\begin{pgfscope}%
\pgfsetbuttcap%
\pgfsetmiterjoin%
\definecolor{currentfill}{rgb}{0.917647,0.917647,0.949020}%
\pgfsetfillcolor{currentfill}%
\pgfsetlinewidth{0.000000pt}%
\definecolor{currentstroke}{rgb}{0.000000,0.000000,0.000000}%
\pgfsetstrokecolor{currentstroke}%
\pgfsetstrokeopacity{0.000000}%
\pgfsetdash{}{0pt}%
\pgfpathmoveto{\pgfqpoint{0.948751in}{0.663635in}}%
\pgfpathlineto{\pgfqpoint{5.343693in}{0.663635in}}%
\pgfpathlineto{\pgfqpoint{5.343693in}{2.320000in}}%
\pgfpathlineto{\pgfqpoint{0.948751in}{2.320000in}}%
\pgfpathlineto{\pgfqpoint{0.948751in}{0.663635in}}%
\pgfpathclose%
\pgfusepath{fill}%
\end{pgfscope}%
\begin{pgfscope}%
\pgfpathrectangle{\pgfqpoint{0.948751in}{0.663635in}}{\pgfqpoint{4.394942in}{1.656365in}}%
\pgfusepath{clip}%
\pgfsetroundcap%
\pgfsetroundjoin%
\pgfsetlinewidth{1.003750pt}%
\definecolor{currentstroke}{rgb}{1.000000,1.000000,1.000000}%
\pgfsetstrokecolor{currentstroke}%
\pgfsetdash{}{0pt}%
\pgfpathmoveto{\pgfqpoint{1.148521in}{0.663635in}}%
\pgfpathlineto{\pgfqpoint{1.148521in}{2.320000in}}%
\pgfusepath{stroke}%
\end{pgfscope}%
\begin{pgfscope}%
\definecolor{textcolor}{rgb}{0.150000,0.150000,0.150000}%
\pgfsetstrokecolor{textcolor}%
\pgfsetfillcolor{textcolor}%
\pgftext[x=1.148521in,y=0.531691in,,top]{\color{textcolor}{\sffamily\fontsize{11.000000}{13.200000}\selectfont\catcode`\^=\active\def^{\ifmmode\sp\else\^{}\fi}\catcode`\%=\active\def%{\%}0}}%
\end{pgfscope}%
\begin{pgfscope}%
\pgfpathrectangle{\pgfqpoint{0.948751in}{0.663635in}}{\pgfqpoint{4.394942in}{1.656365in}}%
\pgfusepath{clip}%
\pgfsetroundcap%
\pgfsetroundjoin%
\pgfsetlinewidth{1.003750pt}%
\definecolor{currentstroke}{rgb}{1.000000,1.000000,1.000000}%
\pgfsetstrokecolor{currentstroke}%
\pgfsetdash{}{0pt}%
\pgfpathmoveto{\pgfqpoint{1.740433in}{0.663635in}}%
\pgfpathlineto{\pgfqpoint{1.740433in}{2.320000in}}%
\pgfusepath{stroke}%
\end{pgfscope}%
\begin{pgfscope}%
\definecolor{textcolor}{rgb}{0.150000,0.150000,0.150000}%
\pgfsetstrokecolor{textcolor}%
\pgfsetfillcolor{textcolor}%
\pgftext[x=1.740433in,y=0.531691in,,top]{\color{textcolor}{\sffamily\fontsize{11.000000}{13.200000}\selectfont\catcode`\^=\active\def^{\ifmmode\sp\else\^{}\fi}\catcode`\%=\active\def%{\%}20}}%
\end{pgfscope}%
\begin{pgfscope}%
\pgfpathrectangle{\pgfqpoint{0.948751in}{0.663635in}}{\pgfqpoint{4.394942in}{1.656365in}}%
\pgfusepath{clip}%
\pgfsetroundcap%
\pgfsetroundjoin%
\pgfsetlinewidth{1.003750pt}%
\definecolor{currentstroke}{rgb}{1.000000,1.000000,1.000000}%
\pgfsetstrokecolor{currentstroke}%
\pgfsetdash{}{0pt}%
\pgfpathmoveto{\pgfqpoint{2.332344in}{0.663635in}}%
\pgfpathlineto{\pgfqpoint{2.332344in}{2.320000in}}%
\pgfusepath{stroke}%
\end{pgfscope}%
\begin{pgfscope}%
\definecolor{textcolor}{rgb}{0.150000,0.150000,0.150000}%
\pgfsetstrokecolor{textcolor}%
\pgfsetfillcolor{textcolor}%
\pgftext[x=2.332344in,y=0.531691in,,top]{\color{textcolor}{\sffamily\fontsize{11.000000}{13.200000}\selectfont\catcode`\^=\active\def^{\ifmmode\sp\else\^{}\fi}\catcode`\%=\active\def%{\%}40}}%
\end{pgfscope}%
\begin{pgfscope}%
\pgfpathrectangle{\pgfqpoint{0.948751in}{0.663635in}}{\pgfqpoint{4.394942in}{1.656365in}}%
\pgfusepath{clip}%
\pgfsetroundcap%
\pgfsetroundjoin%
\pgfsetlinewidth{1.003750pt}%
\definecolor{currentstroke}{rgb}{1.000000,1.000000,1.000000}%
\pgfsetstrokecolor{currentstroke}%
\pgfsetdash{}{0pt}%
\pgfpathmoveto{\pgfqpoint{2.924255in}{0.663635in}}%
\pgfpathlineto{\pgfqpoint{2.924255in}{2.320000in}}%
\pgfusepath{stroke}%
\end{pgfscope}%
\begin{pgfscope}%
\definecolor{textcolor}{rgb}{0.150000,0.150000,0.150000}%
\pgfsetstrokecolor{textcolor}%
\pgfsetfillcolor{textcolor}%
\pgftext[x=2.924255in,y=0.531691in,,top]{\color{textcolor}{\sffamily\fontsize{11.000000}{13.200000}\selectfont\catcode`\^=\active\def^{\ifmmode\sp\else\^{}\fi}\catcode`\%=\active\def%{\%}60}}%
\end{pgfscope}%
\begin{pgfscope}%
\pgfpathrectangle{\pgfqpoint{0.948751in}{0.663635in}}{\pgfqpoint{4.394942in}{1.656365in}}%
\pgfusepath{clip}%
\pgfsetroundcap%
\pgfsetroundjoin%
\pgfsetlinewidth{1.003750pt}%
\definecolor{currentstroke}{rgb}{1.000000,1.000000,1.000000}%
\pgfsetstrokecolor{currentstroke}%
\pgfsetdash{}{0pt}%
\pgfpathmoveto{\pgfqpoint{3.516167in}{0.663635in}}%
\pgfpathlineto{\pgfqpoint{3.516167in}{2.320000in}}%
\pgfusepath{stroke}%
\end{pgfscope}%
\begin{pgfscope}%
\definecolor{textcolor}{rgb}{0.150000,0.150000,0.150000}%
\pgfsetstrokecolor{textcolor}%
\pgfsetfillcolor{textcolor}%
\pgftext[x=3.516167in,y=0.531691in,,top]{\color{textcolor}{\sffamily\fontsize{11.000000}{13.200000}\selectfont\catcode`\^=\active\def^{\ifmmode\sp\else\^{}\fi}\catcode`\%=\active\def%{\%}80}}%
\end{pgfscope}%
\begin{pgfscope}%
\pgfpathrectangle{\pgfqpoint{0.948751in}{0.663635in}}{\pgfqpoint{4.394942in}{1.656365in}}%
\pgfusepath{clip}%
\pgfsetroundcap%
\pgfsetroundjoin%
\pgfsetlinewidth{1.003750pt}%
\definecolor{currentstroke}{rgb}{1.000000,1.000000,1.000000}%
\pgfsetstrokecolor{currentstroke}%
\pgfsetdash{}{0pt}%
\pgfpathmoveto{\pgfqpoint{4.108078in}{0.663635in}}%
\pgfpathlineto{\pgfqpoint{4.108078in}{2.320000in}}%
\pgfusepath{stroke}%
\end{pgfscope}%
\begin{pgfscope}%
\definecolor{textcolor}{rgb}{0.150000,0.150000,0.150000}%
\pgfsetstrokecolor{textcolor}%
\pgfsetfillcolor{textcolor}%
\pgftext[x=4.108078in,y=0.531691in,,top]{\color{textcolor}{\sffamily\fontsize{11.000000}{13.200000}\selectfont\catcode`\^=\active\def^{\ifmmode\sp\else\^{}\fi}\catcode`\%=\active\def%{\%}100}}%
\end{pgfscope}%
\begin{pgfscope}%
\pgfpathrectangle{\pgfqpoint{0.948751in}{0.663635in}}{\pgfqpoint{4.394942in}{1.656365in}}%
\pgfusepath{clip}%
\pgfsetroundcap%
\pgfsetroundjoin%
\pgfsetlinewidth{1.003750pt}%
\definecolor{currentstroke}{rgb}{1.000000,1.000000,1.000000}%
\pgfsetstrokecolor{currentstroke}%
\pgfsetdash{}{0pt}%
\pgfpathmoveto{\pgfqpoint{4.699990in}{0.663635in}}%
\pgfpathlineto{\pgfqpoint{4.699990in}{2.320000in}}%
\pgfusepath{stroke}%
\end{pgfscope}%
\begin{pgfscope}%
\definecolor{textcolor}{rgb}{0.150000,0.150000,0.150000}%
\pgfsetstrokecolor{textcolor}%
\pgfsetfillcolor{textcolor}%
\pgftext[x=4.699990in,y=0.531691in,,top]{\color{textcolor}{\sffamily\fontsize{11.000000}{13.200000}\selectfont\catcode`\^=\active\def^{\ifmmode\sp\else\^{}\fi}\catcode`\%=\active\def%{\%}120}}%
\end{pgfscope}%
\begin{pgfscope}%
\pgfpathrectangle{\pgfqpoint{0.948751in}{0.663635in}}{\pgfqpoint{4.394942in}{1.656365in}}%
\pgfusepath{clip}%
\pgfsetroundcap%
\pgfsetroundjoin%
\pgfsetlinewidth{1.003750pt}%
\definecolor{currentstroke}{rgb}{1.000000,1.000000,1.000000}%
\pgfsetstrokecolor{currentstroke}%
\pgfsetdash{}{0pt}%
\pgfpathmoveto{\pgfqpoint{5.291901in}{0.663635in}}%
\pgfpathlineto{\pgfqpoint{5.291901in}{2.320000in}}%
\pgfusepath{stroke}%
\end{pgfscope}%
\begin{pgfscope}%
\definecolor{textcolor}{rgb}{0.150000,0.150000,0.150000}%
\pgfsetstrokecolor{textcolor}%
\pgfsetfillcolor{textcolor}%
\pgftext[x=5.291901in,y=0.531691in,,top]{\color{textcolor}{\sffamily\fontsize{11.000000}{13.200000}\selectfont\catcode`\^=\active\def^{\ifmmode\sp\else\^{}\fi}\catcode`\%=\active\def%{\%}140}}%
\end{pgfscope}%
\begin{pgfscope}%
\definecolor{textcolor}{rgb}{0.150000,0.150000,0.150000}%
\pgfsetstrokecolor{textcolor}%
\pgfsetfillcolor{textcolor}%
\pgftext[x=3.146222in,y=0.336413in,,top]{\color{textcolor}{\sffamily\fontsize{12.000000}{14.400000}\selectfont\catcode`\^=\active\def^{\ifmmode\sp\else\^{}\fi}\catcode`\%=\active\def%{\%}Time (s)}}%
\end{pgfscope}%
\begin{pgfscope}%
\pgfpathrectangle{\pgfqpoint{0.948751in}{0.663635in}}{\pgfqpoint{4.394942in}{1.656365in}}%
\pgfusepath{clip}%
\pgfsetroundcap%
\pgfsetroundjoin%
\pgfsetlinewidth{1.003750pt}%
\definecolor{currentstroke}{rgb}{1.000000,1.000000,1.000000}%
\pgfsetstrokecolor{currentstroke}%
\pgfsetdash{}{0pt}%
\pgfpathmoveto{\pgfqpoint{0.948751in}{0.738925in}}%
\pgfpathlineto{\pgfqpoint{5.343693in}{0.738925in}}%
\pgfusepath{stroke}%
\end{pgfscope}%
\begin{pgfscope}%
\definecolor{textcolor}{rgb}{0.150000,0.150000,0.150000}%
\pgfsetstrokecolor{textcolor}%
\pgfsetfillcolor{textcolor}%
\pgftext[x=0.731839in, y=0.684244in, left, base]{\color{textcolor}{\sffamily\fontsize{11.000000}{13.200000}\selectfont\catcode`\^=\active\def^{\ifmmode\sp\else\^{}\fi}\catcode`\%=\active\def%{\%}0}}%
\end{pgfscope}%
\begin{pgfscope}%
\pgfpathrectangle{\pgfqpoint{0.948751in}{0.663635in}}{\pgfqpoint{4.394942in}{1.656365in}}%
\pgfusepath{clip}%
\pgfsetroundcap%
\pgfsetroundjoin%
\pgfsetlinewidth{1.003750pt}%
\definecolor{currentstroke}{rgb}{1.000000,1.000000,1.000000}%
\pgfsetstrokecolor{currentstroke}%
\pgfsetdash{}{0pt}%
\pgfpathmoveto{\pgfqpoint{0.948751in}{1.247104in}}%
\pgfpathlineto{\pgfqpoint{5.343693in}{1.247104in}}%
\pgfusepath{stroke}%
\end{pgfscope}%
\begin{pgfscope}%
\definecolor{textcolor}{rgb}{0.150000,0.150000,0.150000}%
\pgfsetstrokecolor{textcolor}%
\pgfsetfillcolor{textcolor}%
\pgftext[x=0.391968in, y=1.192423in, left, base]{\color{textcolor}{\sffamily\fontsize{11.000000}{13.200000}\selectfont\catcode`\^=\active\def^{\ifmmode\sp\else\^{}\fi}\catcode`\%=\active\def%{\%}10000}}%
\end{pgfscope}%
\begin{pgfscope}%
\pgfpathrectangle{\pgfqpoint{0.948751in}{0.663635in}}{\pgfqpoint{4.394942in}{1.656365in}}%
\pgfusepath{clip}%
\pgfsetroundcap%
\pgfsetroundjoin%
\pgfsetlinewidth{1.003750pt}%
\definecolor{currentstroke}{rgb}{1.000000,1.000000,1.000000}%
\pgfsetstrokecolor{currentstroke}%
\pgfsetdash{}{0pt}%
\pgfpathmoveto{\pgfqpoint{0.948751in}{1.755283in}}%
\pgfpathlineto{\pgfqpoint{5.343693in}{1.755283in}}%
\pgfusepath{stroke}%
\end{pgfscope}%
\begin{pgfscope}%
\definecolor{textcolor}{rgb}{0.150000,0.150000,0.150000}%
\pgfsetstrokecolor{textcolor}%
\pgfsetfillcolor{textcolor}%
\pgftext[x=0.391968in, y=1.700603in, left, base]{\color{textcolor}{\sffamily\fontsize{11.000000}{13.200000}\selectfont\catcode`\^=\active\def^{\ifmmode\sp\else\^{}\fi}\catcode`\%=\active\def%{\%}20000}}%
\end{pgfscope}%
\begin{pgfscope}%
\pgfpathrectangle{\pgfqpoint{0.948751in}{0.663635in}}{\pgfqpoint{4.394942in}{1.656365in}}%
\pgfusepath{clip}%
\pgfsetroundcap%
\pgfsetroundjoin%
\pgfsetlinewidth{1.003750pt}%
\definecolor{currentstroke}{rgb}{1.000000,1.000000,1.000000}%
\pgfsetstrokecolor{currentstroke}%
\pgfsetdash{}{0pt}%
\pgfpathmoveto{\pgfqpoint{0.948751in}{2.263463in}}%
\pgfpathlineto{\pgfqpoint{5.343693in}{2.263463in}}%
\pgfusepath{stroke}%
\end{pgfscope}%
\begin{pgfscope}%
\definecolor{textcolor}{rgb}{0.150000,0.150000,0.150000}%
\pgfsetstrokecolor{textcolor}%
\pgfsetfillcolor{textcolor}%
\pgftext[x=0.391968in, y=2.208782in, left, base]{\color{textcolor}{\sffamily\fontsize{11.000000}{13.200000}\selectfont\catcode`\^=\active\def^{\ifmmode\sp\else\^{}\fi}\catcode`\%=\active\def%{\%}30000}}%
\end{pgfscope}%
\begin{pgfscope}%
\definecolor{textcolor}{rgb}{0.150000,0.150000,0.150000}%
\pgfsetstrokecolor{textcolor}%
\pgfsetfillcolor{textcolor}%
\pgftext[x=0.336413in,y=1.491818in,,bottom,rotate=90.000000]{\color{textcolor}{\sffamily\fontsize{12.000000}{14.400000}\selectfont\catcode`\^=\active\def^{\ifmmode\sp\else\^{}\fi}\catcode`\%=\active\def%{\%}Writes (op/s)}}%
\end{pgfscope}%
\begin{pgfscope}%
\pgfpathrectangle{\pgfqpoint{0.948751in}{0.663635in}}{\pgfqpoint{4.394942in}{1.656365in}}%
\pgfusepath{clip}%
\pgfsetroundcap%
\pgfsetroundjoin%
\pgfsetlinewidth{1.505625pt}%
\definecolor{currentstroke}{rgb}{0.298039,0.447059,0.690196}%
\pgfsetstrokecolor{currentstroke}%
\pgfsetdash{}{0pt}%
\pgfpathmoveto{\pgfqpoint{1.148521in}{0.738925in}}%
\pgfpathlineto{\pgfqpoint{1.296499in}{0.738925in}}%
\pgfpathlineto{\pgfqpoint{1.444477in}{0.738925in}}%
\pgfpathlineto{\pgfqpoint{1.592455in}{0.741339in}}%
\pgfpathlineto{\pgfqpoint{1.740433in}{0.743752in}}%
\pgfpathlineto{\pgfqpoint{1.888411in}{0.743752in}}%
\pgfpathlineto{\pgfqpoint{2.036388in}{1.208177in}}%
\pgfpathlineto{\pgfqpoint{2.184366in}{1.672602in}}%
\pgfpathlineto{\pgfqpoint{2.332344in}{1.672602in}}%
\pgfpathlineto{\pgfqpoint{2.480322in}{1.902350in}}%
\pgfpathlineto{\pgfqpoint{2.628300in}{2.132098in}}%
\pgfpathlineto{\pgfqpoint{2.776278in}{2.132098in}}%
\pgfpathlineto{\pgfqpoint{2.924255in}{2.188404in}}%
\pgfpathlineto{\pgfqpoint{3.072233in}{2.244711in}}%
\pgfpathlineto{\pgfqpoint{3.220211in}{2.244711in}}%
\pgfpathlineto{\pgfqpoint{3.368189in}{2.192317in}}%
\pgfpathlineto{\pgfqpoint{3.516167in}{2.089055in}}%
\pgfpathlineto{\pgfqpoint{3.664145in}{2.089055in}}%
\pgfpathlineto{\pgfqpoint{3.812123in}{2.026905in}}%
\pgfpathlineto{\pgfqpoint{3.960100in}{1.659796in}}%
\pgfpathlineto{\pgfqpoint{4.108078in}{1.507343in}}%
\pgfpathlineto{\pgfqpoint{4.256056in}{1.194203in}}%
\pgfpathlineto{\pgfqpoint{4.404034in}{0.881062in}}%
\pgfpathlineto{\pgfqpoint{4.552012in}{0.881062in}}%
\pgfpathlineto{\pgfqpoint{4.699990in}{0.809968in}}%
\pgfpathlineto{\pgfqpoint{4.847967in}{0.738925in}}%
\pgfpathlineto{\pgfqpoint{4.995945in}{0.738925in}}%
\pgfpathlineto{\pgfqpoint{5.143923in}{0.738925in}}%
\pgfusepath{stroke}%
\end{pgfscope}%
\begin{pgfscope}%
\pgfpathrectangle{\pgfqpoint{0.948751in}{0.663635in}}{\pgfqpoint{4.394942in}{1.656365in}}%
\pgfusepath{clip}%
\pgfsetroundcap%
\pgfsetroundjoin%
\pgfsetlinewidth{1.505625pt}%
\definecolor{currentstroke}{rgb}{1.000000,0.000000,0.000000}%
\pgfsetstrokecolor{currentstroke}%
\pgfsetdash{}{0pt}%
\pgfpathmoveto{\pgfqpoint{0.948751in}{1.406818in}}%
\pgfpathlineto{\pgfqpoint{5.343693in}{1.406818in}}%
\pgfusepath{stroke}%
\end{pgfscope}%
\begin{pgfscope}%
\pgfsetrectcap%
\pgfsetmiterjoin%
\pgfsetlinewidth{1.254687pt}%
\definecolor{currentstroke}{rgb}{1.000000,1.000000,1.000000}%
\pgfsetstrokecolor{currentstroke}%
\pgfsetdash{}{0pt}%
\pgfpathmoveto{\pgfqpoint{0.948751in}{0.663635in}}%
\pgfpathlineto{\pgfqpoint{0.948751in}{2.320000in}}%
\pgfusepath{stroke}%
\end{pgfscope}%
\begin{pgfscope}%
\pgfsetrectcap%
\pgfsetmiterjoin%
\pgfsetlinewidth{1.254687pt}%
\definecolor{currentstroke}{rgb}{1.000000,1.000000,1.000000}%
\pgfsetstrokecolor{currentstroke}%
\pgfsetdash{}{0pt}%
\pgfpathmoveto{\pgfqpoint{5.343693in}{0.663635in}}%
\pgfpathlineto{\pgfqpoint{5.343693in}{2.320000in}}%
\pgfusepath{stroke}%
\end{pgfscope}%
\begin{pgfscope}%
\pgfsetrectcap%
\pgfsetmiterjoin%
\pgfsetlinewidth{1.254687pt}%
\definecolor{currentstroke}{rgb}{1.000000,1.000000,1.000000}%
\pgfsetstrokecolor{currentstroke}%
\pgfsetdash{}{0pt}%
\pgfpathmoveto{\pgfqpoint{0.948751in}{0.663635in}}%
\pgfpathlineto{\pgfqpoint{5.343693in}{0.663635in}}%
\pgfusepath{stroke}%
\end{pgfscope}%
\begin{pgfscope}%
\pgfsetrectcap%
\pgfsetmiterjoin%
\pgfsetlinewidth{1.254687pt}%
\definecolor{currentstroke}{rgb}{1.000000,1.000000,1.000000}%
\pgfsetstrokecolor{currentstroke}%
\pgfsetdash{}{0pt}%
\pgfpathmoveto{\pgfqpoint{0.948751in}{2.320000in}}%
\pgfpathlineto{\pgfqpoint{5.343693in}{2.320000in}}%
\pgfusepath{stroke}%
\end{pgfscope}%
\begin{pgfscope}%
\pgfsetbuttcap%
\pgfsetmiterjoin%
\definecolor{currentfill}{rgb}{0.917647,0.917647,0.949020}%
\pgfsetfillcolor{currentfill}%
\pgfsetfillopacity{0.800000}%
\pgfsetlinewidth{1.003750pt}%
\definecolor{currentstroke}{rgb}{0.800000,0.800000,0.800000}%
\pgfsetstrokecolor{currentstroke}%
\pgfsetstrokeopacity{0.800000}%
\pgfsetdash{}{0pt}%
\pgfpathmoveto{\pgfqpoint{2.069763in}{0.719191in}}%
\pgfpathlineto{\pgfqpoint{4.222682in}{0.719191in}}%
\pgfpathquadraticcurveto{\pgfqpoint{4.244904in}{0.719191in}}{\pgfqpoint{4.244904in}{0.741413in}}%
\pgfpathlineto{\pgfqpoint{4.244904in}{0.888776in}}%
\pgfpathquadraticcurveto{\pgfqpoint{4.244904in}{0.910998in}}{\pgfqpoint{4.222682in}{0.910998in}}%
\pgfpathlineto{\pgfqpoint{2.069763in}{0.910998in}}%
\pgfpathquadraticcurveto{\pgfqpoint{2.047541in}{0.910998in}}{\pgfqpoint{2.047541in}{0.888776in}}%
\pgfpathlineto{\pgfqpoint{2.047541in}{0.741413in}}%
\pgfpathquadraticcurveto{\pgfqpoint{2.047541in}{0.719191in}}{\pgfqpoint{2.069763in}{0.719191in}}%
\pgfpathlineto{\pgfqpoint{2.069763in}{0.719191in}}%
\pgfpathclose%
\pgfusepath{stroke,fill}%
\end{pgfscope}%
\begin{pgfscope}%
\pgfsetroundcap%
\pgfsetroundjoin%
\pgfsetlinewidth{1.505625pt}%
\definecolor{currentstroke}{rgb}{1.000000,0.000000,0.000000}%
\pgfsetstrokecolor{currentstroke}%
\pgfsetdash{}{0pt}%
\pgfpathmoveto{\pgfqpoint{2.091985in}{0.825907in}}%
\pgfpathlineto{\pgfqpoint{2.203096in}{0.825907in}}%
\pgfpathlineto{\pgfqpoint{2.314207in}{0.825907in}}%
\pgfusepath{stroke}%
\end{pgfscope}%
\begin{pgfscope}%
\definecolor{textcolor}{rgb}{0.150000,0.150000,0.150000}%
\pgfsetstrokecolor{textcolor}%
\pgfsetfillcolor{textcolor}%
\pgftext[x=2.403096in,y=0.787018in,left,base]{\color{textcolor}{\sffamily\fontsize{8.000000}{9.600000}\selectfont\catcode`\^=\active\def^{\ifmmode\sp\else\^{}\fi}\catcode`\%=\active\def%{\%}average write operations per second}}%
\end{pgfscope}%
\end{pgfpicture}%
\makeatother%
\endgroup%

    \caption{Stress test of 2 nodes with 1000000 writes}
    \label{fig:stress-1000000writes-2node}
\end{figure}

\begin{figure}
    \centering
    %% Creator: Matplotlib, PGF backend
%%
%% To include the figure in your LaTeX document, write
%%   \input{<filename>.pgf}
%%
%% Make sure the required packages are loaded in your preamble
%%   \usepackage{pgf}
%%
%% Also ensure that all the required font packages are loaded; for instance,
%% the lmodern package is sometimes necessary when using math font.
%%   \usepackage{lmodern}
%%
%% Figures using additional raster images can only be included by \input if
%% they are in the same directory as the main LaTeX file. For loading figures
%% from other directories you can use the `import` package
%%   \usepackage{import}
%%
%% and then include the figures with
%%   \import{<path to file>}{<filename>.pgf}
%%
%% Matplotlib used the following preamble
%%   \def\mathdefault#1{#1}
%%   \everymath=\expandafter{\the\everymath\displaystyle}
%%   
%%   \usepackage{fontspec}
%%   \setmainfont{DejaVuSerif.ttf}[Path=\detokenize{/Users/nkratky/private/polaris-elasticity-strategies/test/scripts/.venv/lib/python3.11/site-packages/matplotlib/mpl-data/fonts/ttf/}]
%%   \setsansfont{Arial.ttf}[Path=\detokenize{/System/Library/Fonts/Supplemental/}]
%%   \setmonofont{DejaVuSansMono.ttf}[Path=\detokenize{/Users/nkratky/private/polaris-elasticity-strategies/test/scripts/.venv/lib/python3.11/site-packages/matplotlib/mpl-data/fonts/ttf/}]
%%   \makeatletter\@ifpackageloaded{underscore}{}{\usepackage[strings]{underscore}}\makeatother
%%
\begingroup%
\makeatletter%
\begin{pgfpicture}%
\pgfpathrectangle{\pgfpointorigin}{\pgfqpoint{5.600000in}{2.500000in}}%
\pgfusepath{use as bounding box, clip}%
\begin{pgfscope}%
\pgfsetbuttcap%
\pgfsetmiterjoin%
\definecolor{currentfill}{rgb}{1.000000,1.000000,1.000000}%
\pgfsetfillcolor{currentfill}%
\pgfsetlinewidth{0.000000pt}%
\definecolor{currentstroke}{rgb}{1.000000,1.000000,1.000000}%
\pgfsetstrokecolor{currentstroke}%
\pgfsetdash{}{0pt}%
\pgfpathmoveto{\pgfqpoint{0.000000in}{0.000000in}}%
\pgfpathlineto{\pgfqpoint{5.600000in}{0.000000in}}%
\pgfpathlineto{\pgfqpoint{5.600000in}{2.500000in}}%
\pgfpathlineto{\pgfqpoint{0.000000in}{2.500000in}}%
\pgfpathlineto{\pgfqpoint{0.000000in}{0.000000in}}%
\pgfpathclose%
\pgfusepath{fill}%
\end{pgfscope}%
\begin{pgfscope}%
\pgfsetbuttcap%
\pgfsetmiterjoin%
\definecolor{currentfill}{rgb}{0.917647,0.917647,0.949020}%
\pgfsetfillcolor{currentfill}%
\pgfsetlinewidth{0.000000pt}%
\definecolor{currentstroke}{rgb}{0.000000,0.000000,0.000000}%
\pgfsetstrokecolor{currentstroke}%
\pgfsetstrokeopacity{0.000000}%
\pgfsetdash{}{0pt}%
\pgfpathmoveto{\pgfqpoint{0.948751in}{0.663635in}}%
\pgfpathlineto{\pgfqpoint{5.420000in}{0.663635in}}%
\pgfpathlineto{\pgfqpoint{5.420000in}{2.320000in}}%
\pgfpathlineto{\pgfqpoint{0.948751in}{2.320000in}}%
\pgfpathlineto{\pgfqpoint{0.948751in}{0.663635in}}%
\pgfpathclose%
\pgfusepath{fill}%
\end{pgfscope}%
\begin{pgfscope}%
\pgfpathrectangle{\pgfqpoint{0.948751in}{0.663635in}}{\pgfqpoint{4.471249in}{1.656365in}}%
\pgfusepath{clip}%
\pgfsetroundcap%
\pgfsetroundjoin%
\pgfsetlinewidth{1.003750pt}%
\definecolor{currentstroke}{rgb}{1.000000,1.000000,1.000000}%
\pgfsetstrokecolor{currentstroke}%
\pgfsetdash{}{0pt}%
\pgfpathmoveto{\pgfqpoint{1.151990in}{0.663635in}}%
\pgfpathlineto{\pgfqpoint{1.151990in}{2.320000in}}%
\pgfusepath{stroke}%
\end{pgfscope}%
\begin{pgfscope}%
\definecolor{textcolor}{rgb}{0.150000,0.150000,0.150000}%
\pgfsetstrokecolor{textcolor}%
\pgfsetfillcolor{textcolor}%
\pgftext[x=1.151990in,y=0.531691in,,top]{\color{textcolor}{\sffamily\fontsize{11.000000}{13.200000}\selectfont\catcode`\^=\active\def^{\ifmmode\sp\else\^{}\fi}\catcode`\%=\active\def%{\%}0}}%
\end{pgfscope}%
\begin{pgfscope}%
\pgfpathrectangle{\pgfqpoint{0.948751in}{0.663635in}}{\pgfqpoint{4.471249in}{1.656365in}}%
\pgfusepath{clip}%
\pgfsetroundcap%
\pgfsetroundjoin%
\pgfsetlinewidth{1.003750pt}%
\definecolor{currentstroke}{rgb}{1.000000,1.000000,1.000000}%
\pgfsetstrokecolor{currentstroke}%
\pgfsetdash{}{0pt}%
\pgfpathmoveto{\pgfqpoint{1.926232in}{0.663635in}}%
\pgfpathlineto{\pgfqpoint{1.926232in}{2.320000in}}%
\pgfusepath{stroke}%
\end{pgfscope}%
\begin{pgfscope}%
\definecolor{textcolor}{rgb}{0.150000,0.150000,0.150000}%
\pgfsetstrokecolor{textcolor}%
\pgfsetfillcolor{textcolor}%
\pgftext[x=1.926232in,y=0.531691in,,top]{\color{textcolor}{\sffamily\fontsize{11.000000}{13.200000}\selectfont\catcode`\^=\active\def^{\ifmmode\sp\else\^{}\fi}\catcode`\%=\active\def%{\%}20}}%
\end{pgfscope}%
\begin{pgfscope}%
\pgfpathrectangle{\pgfqpoint{0.948751in}{0.663635in}}{\pgfqpoint{4.471249in}{1.656365in}}%
\pgfusepath{clip}%
\pgfsetroundcap%
\pgfsetroundjoin%
\pgfsetlinewidth{1.003750pt}%
\definecolor{currentstroke}{rgb}{1.000000,1.000000,1.000000}%
\pgfsetstrokecolor{currentstroke}%
\pgfsetdash{}{0pt}%
\pgfpathmoveto{\pgfqpoint{2.700474in}{0.663635in}}%
\pgfpathlineto{\pgfqpoint{2.700474in}{2.320000in}}%
\pgfusepath{stroke}%
\end{pgfscope}%
\begin{pgfscope}%
\definecolor{textcolor}{rgb}{0.150000,0.150000,0.150000}%
\pgfsetstrokecolor{textcolor}%
\pgfsetfillcolor{textcolor}%
\pgftext[x=2.700474in,y=0.531691in,,top]{\color{textcolor}{\sffamily\fontsize{11.000000}{13.200000}\selectfont\catcode`\^=\active\def^{\ifmmode\sp\else\^{}\fi}\catcode`\%=\active\def%{\%}40}}%
\end{pgfscope}%
\begin{pgfscope}%
\pgfpathrectangle{\pgfqpoint{0.948751in}{0.663635in}}{\pgfqpoint{4.471249in}{1.656365in}}%
\pgfusepath{clip}%
\pgfsetroundcap%
\pgfsetroundjoin%
\pgfsetlinewidth{1.003750pt}%
\definecolor{currentstroke}{rgb}{1.000000,1.000000,1.000000}%
\pgfsetstrokecolor{currentstroke}%
\pgfsetdash{}{0pt}%
\pgfpathmoveto{\pgfqpoint{3.474716in}{0.663635in}}%
\pgfpathlineto{\pgfqpoint{3.474716in}{2.320000in}}%
\pgfusepath{stroke}%
\end{pgfscope}%
\begin{pgfscope}%
\definecolor{textcolor}{rgb}{0.150000,0.150000,0.150000}%
\pgfsetstrokecolor{textcolor}%
\pgfsetfillcolor{textcolor}%
\pgftext[x=3.474716in,y=0.531691in,,top]{\color{textcolor}{\sffamily\fontsize{11.000000}{13.200000}\selectfont\catcode`\^=\active\def^{\ifmmode\sp\else\^{}\fi}\catcode`\%=\active\def%{\%}60}}%
\end{pgfscope}%
\begin{pgfscope}%
\pgfpathrectangle{\pgfqpoint{0.948751in}{0.663635in}}{\pgfqpoint{4.471249in}{1.656365in}}%
\pgfusepath{clip}%
\pgfsetroundcap%
\pgfsetroundjoin%
\pgfsetlinewidth{1.003750pt}%
\definecolor{currentstroke}{rgb}{1.000000,1.000000,1.000000}%
\pgfsetstrokecolor{currentstroke}%
\pgfsetdash{}{0pt}%
\pgfpathmoveto{\pgfqpoint{4.248959in}{0.663635in}}%
\pgfpathlineto{\pgfqpoint{4.248959in}{2.320000in}}%
\pgfusepath{stroke}%
\end{pgfscope}%
\begin{pgfscope}%
\definecolor{textcolor}{rgb}{0.150000,0.150000,0.150000}%
\pgfsetstrokecolor{textcolor}%
\pgfsetfillcolor{textcolor}%
\pgftext[x=4.248959in,y=0.531691in,,top]{\color{textcolor}{\sffamily\fontsize{11.000000}{13.200000}\selectfont\catcode`\^=\active\def^{\ifmmode\sp\else\^{}\fi}\catcode`\%=\active\def%{\%}80}}%
\end{pgfscope}%
\begin{pgfscope}%
\pgfpathrectangle{\pgfqpoint{0.948751in}{0.663635in}}{\pgfqpoint{4.471249in}{1.656365in}}%
\pgfusepath{clip}%
\pgfsetroundcap%
\pgfsetroundjoin%
\pgfsetlinewidth{1.003750pt}%
\definecolor{currentstroke}{rgb}{1.000000,1.000000,1.000000}%
\pgfsetstrokecolor{currentstroke}%
\pgfsetdash{}{0pt}%
\pgfpathmoveto{\pgfqpoint{5.023201in}{0.663635in}}%
\pgfpathlineto{\pgfqpoint{5.023201in}{2.320000in}}%
\pgfusepath{stroke}%
\end{pgfscope}%
\begin{pgfscope}%
\definecolor{textcolor}{rgb}{0.150000,0.150000,0.150000}%
\pgfsetstrokecolor{textcolor}%
\pgfsetfillcolor{textcolor}%
\pgftext[x=5.023201in,y=0.531691in,,top]{\color{textcolor}{\sffamily\fontsize{11.000000}{13.200000}\selectfont\catcode`\^=\active\def^{\ifmmode\sp\else\^{}\fi}\catcode`\%=\active\def%{\%}100}}%
\end{pgfscope}%
\begin{pgfscope}%
\definecolor{textcolor}{rgb}{0.150000,0.150000,0.150000}%
\pgfsetstrokecolor{textcolor}%
\pgfsetfillcolor{textcolor}%
\pgftext[x=3.184376in,y=0.336413in,,top]{\color{textcolor}{\sffamily\fontsize{12.000000}{14.400000}\selectfont\catcode`\^=\active\def^{\ifmmode\sp\else\^{}\fi}\catcode`\%=\active\def%{\%}Time (s)}}%
\end{pgfscope}%
\begin{pgfscope}%
\pgfpathrectangle{\pgfqpoint{0.948751in}{0.663635in}}{\pgfqpoint{4.471249in}{1.656365in}}%
\pgfusepath{clip}%
\pgfsetroundcap%
\pgfsetroundjoin%
\pgfsetlinewidth{1.003750pt}%
\definecolor{currentstroke}{rgb}{1.000000,1.000000,1.000000}%
\pgfsetstrokecolor{currentstroke}%
\pgfsetdash{}{0pt}%
\pgfpathmoveto{\pgfqpoint{0.948751in}{0.738925in}}%
\pgfpathlineto{\pgfqpoint{5.420000in}{0.738925in}}%
\pgfusepath{stroke}%
\end{pgfscope}%
\begin{pgfscope}%
\definecolor{textcolor}{rgb}{0.150000,0.150000,0.150000}%
\pgfsetstrokecolor{textcolor}%
\pgfsetfillcolor{textcolor}%
\pgftext[x=0.731839in, y=0.684244in, left, base]{\color{textcolor}{\sffamily\fontsize{11.000000}{13.200000}\selectfont\catcode`\^=\active\def^{\ifmmode\sp\else\^{}\fi}\catcode`\%=\active\def%{\%}0}}%
\end{pgfscope}%
\begin{pgfscope}%
\pgfpathrectangle{\pgfqpoint{0.948751in}{0.663635in}}{\pgfqpoint{4.471249in}{1.656365in}}%
\pgfusepath{clip}%
\pgfsetroundcap%
\pgfsetroundjoin%
\pgfsetlinewidth{1.003750pt}%
\definecolor{currentstroke}{rgb}{1.000000,1.000000,1.000000}%
\pgfsetstrokecolor{currentstroke}%
\pgfsetdash{}{0pt}%
\pgfpathmoveto{\pgfqpoint{0.948751in}{1.227959in}}%
\pgfpathlineto{\pgfqpoint{5.420000in}{1.227959in}}%
\pgfusepath{stroke}%
\end{pgfscope}%
\begin{pgfscope}%
\definecolor{textcolor}{rgb}{0.150000,0.150000,0.150000}%
\pgfsetstrokecolor{textcolor}%
\pgfsetfillcolor{textcolor}%
\pgftext[x=0.391968in, y=1.173278in, left, base]{\color{textcolor}{\sffamily\fontsize{11.000000}{13.200000}\selectfont\catcode`\^=\active\def^{\ifmmode\sp\else\^{}\fi}\catcode`\%=\active\def%{\%}10000}}%
\end{pgfscope}%
\begin{pgfscope}%
\pgfpathrectangle{\pgfqpoint{0.948751in}{0.663635in}}{\pgfqpoint{4.471249in}{1.656365in}}%
\pgfusepath{clip}%
\pgfsetroundcap%
\pgfsetroundjoin%
\pgfsetlinewidth{1.003750pt}%
\definecolor{currentstroke}{rgb}{1.000000,1.000000,1.000000}%
\pgfsetstrokecolor{currentstroke}%
\pgfsetdash{}{0pt}%
\pgfpathmoveto{\pgfqpoint{0.948751in}{1.716994in}}%
\pgfpathlineto{\pgfqpoint{5.420000in}{1.716994in}}%
\pgfusepath{stroke}%
\end{pgfscope}%
\begin{pgfscope}%
\definecolor{textcolor}{rgb}{0.150000,0.150000,0.150000}%
\pgfsetstrokecolor{textcolor}%
\pgfsetfillcolor{textcolor}%
\pgftext[x=0.391968in, y=1.662313in, left, base]{\color{textcolor}{\sffamily\fontsize{11.000000}{13.200000}\selectfont\catcode`\^=\active\def^{\ifmmode\sp\else\^{}\fi}\catcode`\%=\active\def%{\%}20000}}%
\end{pgfscope}%
\begin{pgfscope}%
\pgfpathrectangle{\pgfqpoint{0.948751in}{0.663635in}}{\pgfqpoint{4.471249in}{1.656365in}}%
\pgfusepath{clip}%
\pgfsetroundcap%
\pgfsetroundjoin%
\pgfsetlinewidth{1.003750pt}%
\definecolor{currentstroke}{rgb}{1.000000,1.000000,1.000000}%
\pgfsetstrokecolor{currentstroke}%
\pgfsetdash{}{0pt}%
\pgfpathmoveto{\pgfqpoint{0.948751in}{2.206028in}}%
\pgfpathlineto{\pgfqpoint{5.420000in}{2.206028in}}%
\pgfusepath{stroke}%
\end{pgfscope}%
\begin{pgfscope}%
\definecolor{textcolor}{rgb}{0.150000,0.150000,0.150000}%
\pgfsetstrokecolor{textcolor}%
\pgfsetfillcolor{textcolor}%
\pgftext[x=0.391968in, y=2.151347in, left, base]{\color{textcolor}{\sffamily\fontsize{11.000000}{13.200000}\selectfont\catcode`\^=\active\def^{\ifmmode\sp\else\^{}\fi}\catcode`\%=\active\def%{\%}30000}}%
\end{pgfscope}%
\begin{pgfscope}%
\definecolor{textcolor}{rgb}{0.150000,0.150000,0.150000}%
\pgfsetstrokecolor{textcolor}%
\pgfsetfillcolor{textcolor}%
\pgftext[x=0.336413in,y=1.491818in,,bottom,rotate=90.000000]{\color{textcolor}{\sffamily\fontsize{12.000000}{14.400000}\selectfont\catcode`\^=\active\def^{\ifmmode\sp\else\^{}\fi}\catcode`\%=\active\def%{\%}Writes (op/s)}}%
\end{pgfscope}%
\begin{pgfscope}%
\pgfpathrectangle{\pgfqpoint{0.948751in}{0.663635in}}{\pgfqpoint{4.471249in}{1.656365in}}%
\pgfusepath{clip}%
\pgfsetroundcap%
\pgfsetroundjoin%
\pgfsetlinewidth{1.505625pt}%
\definecolor{currentstroke}{rgb}{0.298039,0.447059,0.690196}%
\pgfsetstrokecolor{currentstroke}%
\pgfsetdash{}{0pt}%
\pgfpathmoveto{\pgfqpoint{1.151990in}{0.738925in}}%
\pgfpathlineto{\pgfqpoint{1.345550in}{0.738925in}}%
\pgfpathlineto{\pgfqpoint{1.539111in}{0.738925in}}%
\pgfpathlineto{\pgfqpoint{1.732671in}{1.014789in}}%
\pgfpathlineto{\pgfqpoint{1.926232in}{1.290653in}}%
\pgfpathlineto{\pgfqpoint{2.119793in}{1.290653in}}%
\pgfpathlineto{\pgfqpoint{2.313353in}{1.552531in}}%
\pgfpathlineto{\pgfqpoint{2.506914in}{1.814360in}}%
\pgfpathlineto{\pgfqpoint{2.700474in}{1.814360in}}%
\pgfpathlineto{\pgfqpoint{2.894035in}{1.974568in}}%
\pgfpathlineto{\pgfqpoint{3.087595in}{2.134776in}}%
\pgfpathlineto{\pgfqpoint{3.281156in}{2.134776in}}%
\pgfpathlineto{\pgfqpoint{3.474716in}{2.189743in}}%
\pgfpathlineto{\pgfqpoint{3.668277in}{2.244711in}}%
\pgfpathlineto{\pgfqpoint{3.861838in}{2.244711in}}%
\pgfpathlineto{\pgfqpoint{4.055398in}{1.794848in}}%
\pgfpathlineto{\pgfqpoint{4.248959in}{1.344985in}}%
\pgfpathlineto{\pgfqpoint{4.442519in}{1.344985in}}%
\pgfpathlineto{\pgfqpoint{4.636080in}{1.041979in}}%
\pgfpathlineto{\pgfqpoint{4.829640in}{0.738925in}}%
\pgfpathlineto{\pgfqpoint{5.023201in}{0.738925in}}%
\pgfpathlineto{\pgfqpoint{5.216761in}{0.738925in}}%
\pgfusepath{stroke}%
\end{pgfscope}%
\begin{pgfscope}%
\pgfpathrectangle{\pgfqpoint{0.948751in}{0.663635in}}{\pgfqpoint{4.471249in}{1.656365in}}%
\pgfusepath{clip}%
\pgfsetroundcap%
\pgfsetroundjoin%
\pgfsetlinewidth{1.505625pt}%
\definecolor{currentstroke}{rgb}{1.000000,0.000000,0.000000}%
\pgfsetstrokecolor{currentstroke}%
\pgfsetdash{}{0pt}%
\pgfpathmoveto{\pgfqpoint{0.948751in}{1.439135in}}%
\pgfpathlineto{\pgfqpoint{5.420000in}{1.439135in}}%
\pgfusepath{stroke}%
\end{pgfscope}%
\begin{pgfscope}%
\pgfsetrectcap%
\pgfsetmiterjoin%
\pgfsetlinewidth{1.254687pt}%
\definecolor{currentstroke}{rgb}{1.000000,1.000000,1.000000}%
\pgfsetstrokecolor{currentstroke}%
\pgfsetdash{}{0pt}%
\pgfpathmoveto{\pgfqpoint{0.948751in}{0.663635in}}%
\pgfpathlineto{\pgfqpoint{0.948751in}{2.320000in}}%
\pgfusepath{stroke}%
\end{pgfscope}%
\begin{pgfscope}%
\pgfsetrectcap%
\pgfsetmiterjoin%
\pgfsetlinewidth{1.254687pt}%
\definecolor{currentstroke}{rgb}{1.000000,1.000000,1.000000}%
\pgfsetstrokecolor{currentstroke}%
\pgfsetdash{}{0pt}%
\pgfpathmoveto{\pgfqpoint{5.420000in}{0.663635in}}%
\pgfpathlineto{\pgfqpoint{5.420000in}{2.320000in}}%
\pgfusepath{stroke}%
\end{pgfscope}%
\begin{pgfscope}%
\pgfsetrectcap%
\pgfsetmiterjoin%
\pgfsetlinewidth{1.254687pt}%
\definecolor{currentstroke}{rgb}{1.000000,1.000000,1.000000}%
\pgfsetstrokecolor{currentstroke}%
\pgfsetdash{}{0pt}%
\pgfpathmoveto{\pgfqpoint{0.948751in}{0.663635in}}%
\pgfpathlineto{\pgfqpoint{5.420000in}{0.663635in}}%
\pgfusepath{stroke}%
\end{pgfscope}%
\begin{pgfscope}%
\pgfsetrectcap%
\pgfsetmiterjoin%
\pgfsetlinewidth{1.254687pt}%
\definecolor{currentstroke}{rgb}{1.000000,1.000000,1.000000}%
\pgfsetstrokecolor{currentstroke}%
\pgfsetdash{}{0pt}%
\pgfpathmoveto{\pgfqpoint{0.948751in}{2.320000in}}%
\pgfpathlineto{\pgfqpoint{5.420000in}{2.320000in}}%
\pgfusepath{stroke}%
\end{pgfscope}%
\begin{pgfscope}%
\pgfsetbuttcap%
\pgfsetmiterjoin%
\definecolor{currentfill}{rgb}{0.917647,0.917647,0.949020}%
\pgfsetfillcolor{currentfill}%
\pgfsetfillopacity{0.800000}%
\pgfsetlinewidth{1.003750pt}%
\definecolor{currentstroke}{rgb}{0.800000,0.800000,0.800000}%
\pgfsetstrokecolor{currentstroke}%
\pgfsetstrokeopacity{0.800000}%
\pgfsetdash{}{0pt}%
\pgfpathmoveto{\pgfqpoint{2.107916in}{0.719191in}}%
\pgfpathlineto{\pgfqpoint{4.260835in}{0.719191in}}%
\pgfpathquadraticcurveto{\pgfqpoint{4.283057in}{0.719191in}}{\pgfqpoint{4.283057in}{0.741413in}}%
\pgfpathlineto{\pgfqpoint{4.283057in}{0.888776in}}%
\pgfpathquadraticcurveto{\pgfqpoint{4.283057in}{0.910998in}}{\pgfqpoint{4.260835in}{0.910998in}}%
\pgfpathlineto{\pgfqpoint{2.107916in}{0.910998in}}%
\pgfpathquadraticcurveto{\pgfqpoint{2.085694in}{0.910998in}}{\pgfqpoint{2.085694in}{0.888776in}}%
\pgfpathlineto{\pgfqpoint{2.085694in}{0.741413in}}%
\pgfpathquadraticcurveto{\pgfqpoint{2.085694in}{0.719191in}}{\pgfqpoint{2.107916in}{0.719191in}}%
\pgfpathlineto{\pgfqpoint{2.107916in}{0.719191in}}%
\pgfpathclose%
\pgfusepath{stroke,fill}%
\end{pgfscope}%
\begin{pgfscope}%
\pgfsetroundcap%
\pgfsetroundjoin%
\pgfsetlinewidth{1.505625pt}%
\definecolor{currentstroke}{rgb}{1.000000,0.000000,0.000000}%
\pgfsetstrokecolor{currentstroke}%
\pgfsetdash{}{0pt}%
\pgfpathmoveto{\pgfqpoint{2.130138in}{0.825907in}}%
\pgfpathlineto{\pgfqpoint{2.241250in}{0.825907in}}%
\pgfpathlineto{\pgfqpoint{2.352361in}{0.825907in}}%
\pgfusepath{stroke}%
\end{pgfscope}%
\begin{pgfscope}%
\definecolor{textcolor}{rgb}{0.150000,0.150000,0.150000}%
\pgfsetstrokecolor{textcolor}%
\pgfsetfillcolor{textcolor}%
\pgftext[x=2.441250in,y=0.787018in,left,base]{\color{textcolor}{\sffamily\fontsize{8.000000}{9.600000}\selectfont\catcode`\^=\active\def^{\ifmmode\sp\else\^{}\fi}\catcode`\%=\active\def%{\%}average write operations per second}}%
\end{pgfscope}%
\end{pgfpicture}%
\makeatother%
\endgroup%

    \caption{Stress test of 3 nodes with 1000000 writes}
    \label{fig:stress-1000000writes-3node}
\end{figure}

\subsection{Vertical Elasticity Strategy}
\label{sec:evaluation-vertical-elasticity}

As mentioned in \Cref{sec:vertical-elasticity} the vertical elasticity strategy adjusts the resource claims of K8ssandra according to its CPU and memory utilization.

As it can bee seen in \Cref{fig:simple-limits-vertical} the elasticity strategy controller successfully changes the CPU and memory limits of the K8ssandra cluster once it is operational and idling. \Cref{fig:utilization-vertical} shows the CPU and memory utilization that is used for triggering elasticity processes. Because the CPU utilization stays very low even after scaling takes place, it can be assumed that this metric was not a decisive factor. The memory utilization, however, changes notably. Before starting the elasticity strategy controller the actual memory utilization was off by \(>10\%\) from the target memory utilization. This triggers an elasticity event and the resources are adjusted proportionally.

Interestingly, during reconsiliation the exposed metrics of K8ssandra are not very meaningful. During this process utilization values of far more than 100\% are exposed by the metrics controller. In order to keep the diagram clean, these nonsense-metrics have been filtered out. The reconsiliation process is marked red in \Cref{fig:utilization-vertical}.

\begin{figure}[H]
    \centering
    %% Creator: Matplotlib, PGF backend
%%
%% To include the figure in your LaTeX document, write
%%   \input{<filename>.pgf}
%%
%% Make sure the required packages are loaded in your preamble
%%   \usepackage{pgf}
%%
%% Also ensure that all the required font packages are loaded; for instance,
%% the lmodern package is sometimes necessary when using math font.
%%   \usepackage{lmodern}
%%
%% Figures using additional raster images can only be included by \input if
%% they are in the same directory as the main LaTeX file. For loading figures
%% from other directories you can use the `import` package
%%   \usepackage{import}
%%
%% and then include the figures with
%%   \import{<path to file>}{<filename>.pgf}
%%
%% Matplotlib used the following preamble
%%   \def\mathdefault#1{#1}
%%   \everymath=\expandafter{\the\everymath\displaystyle}
%%   
%%   \usepackage{fontspec}
%%   \setmainfont{DejaVuSerif.ttf}[Path=\detokenize{/Users/nkratky/private/polaris-elasticity-strategies/test/scripts/.venv/lib/python3.11/site-packages/matplotlib/mpl-data/fonts/ttf/}]
%%   \setsansfont{Arial.ttf}[Path=\detokenize{/System/Library/Fonts/Supplemental/}]
%%   \setmonofont{DejaVuSansMono.ttf}[Path=\detokenize{/Users/nkratky/private/polaris-elasticity-strategies/test/scripts/.venv/lib/python3.11/site-packages/matplotlib/mpl-data/fonts/ttf/}]
%%   \makeatletter\@ifpackageloaded{underscore}{}{\usepackage[strings]{underscore}}\makeatother
%%
\begingroup%
\makeatletter%
\begin{pgfpicture}%
\pgfpathrectangle{\pgfpointorigin}{\pgfqpoint{5.600000in}{4.000000in}}%
\pgfusepath{use as bounding box, clip}%
\begin{pgfscope}%
\pgfsetbuttcap%
\pgfsetmiterjoin%
\definecolor{currentfill}{rgb}{1.000000,1.000000,1.000000}%
\pgfsetfillcolor{currentfill}%
\pgfsetlinewidth{0.000000pt}%
\definecolor{currentstroke}{rgb}{1.000000,1.000000,1.000000}%
\pgfsetstrokecolor{currentstroke}%
\pgfsetdash{}{0pt}%
\pgfpathmoveto{\pgfqpoint{0.000000in}{0.000000in}}%
\pgfpathlineto{\pgfqpoint{5.600000in}{0.000000in}}%
\pgfpathlineto{\pgfqpoint{5.600000in}{4.000000in}}%
\pgfpathlineto{\pgfqpoint{0.000000in}{4.000000in}}%
\pgfpathlineto{\pgfqpoint{0.000000in}{0.000000in}}%
\pgfpathclose%
\pgfusepath{fill}%
\end{pgfscope}%
\begin{pgfscope}%
\pgfsetbuttcap%
\pgfsetmiterjoin%
\definecolor{currentfill}{rgb}{0.917647,0.917647,0.949020}%
\pgfsetfillcolor{currentfill}%
\pgfsetlinewidth{0.000000pt}%
\definecolor{currentstroke}{rgb}{0.000000,0.000000,0.000000}%
\pgfsetstrokecolor{currentstroke}%
\pgfsetstrokeopacity{0.000000}%
\pgfsetdash{}{0pt}%
\pgfpathmoveto{\pgfqpoint{0.863783in}{2.546295in}}%
\pgfpathlineto{\pgfqpoint{5.420000in}{2.546295in}}%
\pgfpathlineto{\pgfqpoint{5.420000in}{3.765319in}}%
\pgfpathlineto{\pgfqpoint{0.863783in}{3.765319in}}%
\pgfpathlineto{\pgfqpoint{0.863783in}{2.546295in}}%
\pgfpathclose%
\pgfusepath{fill}%
\end{pgfscope}%
\begin{pgfscope}%
\pgfpathrectangle{\pgfqpoint{0.863783in}{2.546295in}}{\pgfqpoint{4.556217in}{1.219024in}}%
\pgfusepath{clip}%
\pgfsetroundcap%
\pgfsetroundjoin%
\pgfsetlinewidth{1.003750pt}%
\definecolor{currentstroke}{rgb}{1.000000,1.000000,1.000000}%
\pgfsetstrokecolor{currentstroke}%
\pgfsetdash{}{0pt}%
\pgfpathmoveto{\pgfqpoint{1.070884in}{2.546295in}}%
\pgfpathlineto{\pgfqpoint{1.070884in}{3.765319in}}%
\pgfusepath{stroke}%
\end{pgfscope}%
\begin{pgfscope}%
\definecolor{textcolor}{rgb}{0.150000,0.150000,0.150000}%
\pgfsetstrokecolor{textcolor}%
\pgfsetfillcolor{textcolor}%
\pgftext[x=1.070884in,y=2.414351in,,top]{\color{textcolor}{\sffamily\fontsize{11.000000}{13.200000}\selectfont\catcode`\^=\active\def^{\ifmmode\sp\else\^{}\fi}\catcode`\%=\active\def%{\%}0}}%
\end{pgfscope}%
\begin{pgfscope}%
\pgfpathrectangle{\pgfqpoint{0.863783in}{2.546295in}}{\pgfqpoint{4.556217in}{1.219024in}}%
\pgfusepath{clip}%
\pgfsetroundcap%
\pgfsetroundjoin%
\pgfsetlinewidth{1.003750pt}%
\definecolor{currentstroke}{rgb}{1.000000,1.000000,1.000000}%
\pgfsetstrokecolor{currentstroke}%
\pgfsetdash{}{0pt}%
\pgfpathmoveto{\pgfqpoint{1.582244in}{2.546295in}}%
\pgfpathlineto{\pgfqpoint{1.582244in}{3.765319in}}%
\pgfusepath{stroke}%
\end{pgfscope}%
\begin{pgfscope}%
\definecolor{textcolor}{rgb}{0.150000,0.150000,0.150000}%
\pgfsetstrokecolor{textcolor}%
\pgfsetfillcolor{textcolor}%
\pgftext[x=1.582244in,y=2.414351in,,top]{\color{textcolor}{\sffamily\fontsize{11.000000}{13.200000}\selectfont\catcode`\^=\active\def^{\ifmmode\sp\else\^{}\fi}\catcode`\%=\active\def%{\%}200}}%
\end{pgfscope}%
\begin{pgfscope}%
\pgfpathrectangle{\pgfqpoint{0.863783in}{2.546295in}}{\pgfqpoint{4.556217in}{1.219024in}}%
\pgfusepath{clip}%
\pgfsetroundcap%
\pgfsetroundjoin%
\pgfsetlinewidth{1.003750pt}%
\definecolor{currentstroke}{rgb}{1.000000,1.000000,1.000000}%
\pgfsetstrokecolor{currentstroke}%
\pgfsetdash{}{0pt}%
\pgfpathmoveto{\pgfqpoint{2.093604in}{2.546295in}}%
\pgfpathlineto{\pgfqpoint{2.093604in}{3.765319in}}%
\pgfusepath{stroke}%
\end{pgfscope}%
\begin{pgfscope}%
\definecolor{textcolor}{rgb}{0.150000,0.150000,0.150000}%
\pgfsetstrokecolor{textcolor}%
\pgfsetfillcolor{textcolor}%
\pgftext[x=2.093604in,y=2.414351in,,top]{\color{textcolor}{\sffamily\fontsize{11.000000}{13.200000}\selectfont\catcode`\^=\active\def^{\ifmmode\sp\else\^{}\fi}\catcode`\%=\active\def%{\%}400}}%
\end{pgfscope}%
\begin{pgfscope}%
\pgfpathrectangle{\pgfqpoint{0.863783in}{2.546295in}}{\pgfqpoint{4.556217in}{1.219024in}}%
\pgfusepath{clip}%
\pgfsetroundcap%
\pgfsetroundjoin%
\pgfsetlinewidth{1.003750pt}%
\definecolor{currentstroke}{rgb}{1.000000,1.000000,1.000000}%
\pgfsetstrokecolor{currentstroke}%
\pgfsetdash{}{0pt}%
\pgfpathmoveto{\pgfqpoint{2.604964in}{2.546295in}}%
\pgfpathlineto{\pgfqpoint{2.604964in}{3.765319in}}%
\pgfusepath{stroke}%
\end{pgfscope}%
\begin{pgfscope}%
\definecolor{textcolor}{rgb}{0.150000,0.150000,0.150000}%
\pgfsetstrokecolor{textcolor}%
\pgfsetfillcolor{textcolor}%
\pgftext[x=2.604964in,y=2.414351in,,top]{\color{textcolor}{\sffamily\fontsize{11.000000}{13.200000}\selectfont\catcode`\^=\active\def^{\ifmmode\sp\else\^{}\fi}\catcode`\%=\active\def%{\%}600}}%
\end{pgfscope}%
\begin{pgfscope}%
\pgfpathrectangle{\pgfqpoint{0.863783in}{2.546295in}}{\pgfqpoint{4.556217in}{1.219024in}}%
\pgfusepath{clip}%
\pgfsetroundcap%
\pgfsetroundjoin%
\pgfsetlinewidth{1.003750pt}%
\definecolor{currentstroke}{rgb}{1.000000,1.000000,1.000000}%
\pgfsetstrokecolor{currentstroke}%
\pgfsetdash{}{0pt}%
\pgfpathmoveto{\pgfqpoint{3.116324in}{2.546295in}}%
\pgfpathlineto{\pgfqpoint{3.116324in}{3.765319in}}%
\pgfusepath{stroke}%
\end{pgfscope}%
\begin{pgfscope}%
\definecolor{textcolor}{rgb}{0.150000,0.150000,0.150000}%
\pgfsetstrokecolor{textcolor}%
\pgfsetfillcolor{textcolor}%
\pgftext[x=3.116324in,y=2.414351in,,top]{\color{textcolor}{\sffamily\fontsize{11.000000}{13.200000}\selectfont\catcode`\^=\active\def^{\ifmmode\sp\else\^{}\fi}\catcode`\%=\active\def%{\%}800}}%
\end{pgfscope}%
\begin{pgfscope}%
\pgfpathrectangle{\pgfqpoint{0.863783in}{2.546295in}}{\pgfqpoint{4.556217in}{1.219024in}}%
\pgfusepath{clip}%
\pgfsetroundcap%
\pgfsetroundjoin%
\pgfsetlinewidth{1.003750pt}%
\definecolor{currentstroke}{rgb}{1.000000,1.000000,1.000000}%
\pgfsetstrokecolor{currentstroke}%
\pgfsetdash{}{0pt}%
\pgfpathmoveto{\pgfqpoint{3.627684in}{2.546295in}}%
\pgfpathlineto{\pgfqpoint{3.627684in}{3.765319in}}%
\pgfusepath{stroke}%
\end{pgfscope}%
\begin{pgfscope}%
\definecolor{textcolor}{rgb}{0.150000,0.150000,0.150000}%
\pgfsetstrokecolor{textcolor}%
\pgfsetfillcolor{textcolor}%
\pgftext[x=3.627684in,y=2.414351in,,top]{\color{textcolor}{\sffamily\fontsize{11.000000}{13.200000}\selectfont\catcode`\^=\active\def^{\ifmmode\sp\else\^{}\fi}\catcode`\%=\active\def%{\%}1000}}%
\end{pgfscope}%
\begin{pgfscope}%
\pgfpathrectangle{\pgfqpoint{0.863783in}{2.546295in}}{\pgfqpoint{4.556217in}{1.219024in}}%
\pgfusepath{clip}%
\pgfsetroundcap%
\pgfsetroundjoin%
\pgfsetlinewidth{1.003750pt}%
\definecolor{currentstroke}{rgb}{1.000000,1.000000,1.000000}%
\pgfsetstrokecolor{currentstroke}%
\pgfsetdash{}{0pt}%
\pgfpathmoveto{\pgfqpoint{4.139044in}{2.546295in}}%
\pgfpathlineto{\pgfqpoint{4.139044in}{3.765319in}}%
\pgfusepath{stroke}%
\end{pgfscope}%
\begin{pgfscope}%
\definecolor{textcolor}{rgb}{0.150000,0.150000,0.150000}%
\pgfsetstrokecolor{textcolor}%
\pgfsetfillcolor{textcolor}%
\pgftext[x=4.139044in,y=2.414351in,,top]{\color{textcolor}{\sffamily\fontsize{11.000000}{13.200000}\selectfont\catcode`\^=\active\def^{\ifmmode\sp\else\^{}\fi}\catcode`\%=\active\def%{\%}1200}}%
\end{pgfscope}%
\begin{pgfscope}%
\pgfpathrectangle{\pgfqpoint{0.863783in}{2.546295in}}{\pgfqpoint{4.556217in}{1.219024in}}%
\pgfusepath{clip}%
\pgfsetroundcap%
\pgfsetroundjoin%
\pgfsetlinewidth{1.003750pt}%
\definecolor{currentstroke}{rgb}{1.000000,1.000000,1.000000}%
\pgfsetstrokecolor{currentstroke}%
\pgfsetdash{}{0pt}%
\pgfpathmoveto{\pgfqpoint{4.650403in}{2.546295in}}%
\pgfpathlineto{\pgfqpoint{4.650403in}{3.765319in}}%
\pgfusepath{stroke}%
\end{pgfscope}%
\begin{pgfscope}%
\definecolor{textcolor}{rgb}{0.150000,0.150000,0.150000}%
\pgfsetstrokecolor{textcolor}%
\pgfsetfillcolor{textcolor}%
\pgftext[x=4.650403in,y=2.414351in,,top]{\color{textcolor}{\sffamily\fontsize{11.000000}{13.200000}\selectfont\catcode`\^=\active\def^{\ifmmode\sp\else\^{}\fi}\catcode`\%=\active\def%{\%}1400}}%
\end{pgfscope}%
\begin{pgfscope}%
\pgfpathrectangle{\pgfqpoint{0.863783in}{2.546295in}}{\pgfqpoint{4.556217in}{1.219024in}}%
\pgfusepath{clip}%
\pgfsetroundcap%
\pgfsetroundjoin%
\pgfsetlinewidth{1.003750pt}%
\definecolor{currentstroke}{rgb}{1.000000,1.000000,1.000000}%
\pgfsetstrokecolor{currentstroke}%
\pgfsetdash{}{0pt}%
\pgfpathmoveto{\pgfqpoint{5.161763in}{2.546295in}}%
\pgfpathlineto{\pgfqpoint{5.161763in}{3.765319in}}%
\pgfusepath{stroke}%
\end{pgfscope}%
\begin{pgfscope}%
\definecolor{textcolor}{rgb}{0.150000,0.150000,0.150000}%
\pgfsetstrokecolor{textcolor}%
\pgfsetfillcolor{textcolor}%
\pgftext[x=5.161763in,y=2.414351in,,top]{\color{textcolor}{\sffamily\fontsize{11.000000}{13.200000}\selectfont\catcode`\^=\active\def^{\ifmmode\sp\else\^{}\fi}\catcode`\%=\active\def%{\%}1600}}%
\end{pgfscope}%
\begin{pgfscope}%
\definecolor{textcolor}{rgb}{0.150000,0.150000,0.150000}%
\pgfsetstrokecolor{textcolor}%
\pgfsetfillcolor{textcolor}%
\pgftext[x=3.141892in,y=2.219072in,,top]{\color{textcolor}{\sffamily\fontsize{12.000000}{14.400000}\selectfont\catcode`\^=\active\def^{\ifmmode\sp\else\^{}\fi}\catcode`\%=\active\def%{\%}Time (s)}}%
\end{pgfscope}%
\begin{pgfscope}%
\pgfpathrectangle{\pgfqpoint{0.863783in}{2.546295in}}{\pgfqpoint{4.556217in}{1.219024in}}%
\pgfusepath{clip}%
\pgfsetroundcap%
\pgfsetroundjoin%
\pgfsetlinewidth{1.003750pt}%
\definecolor{currentstroke}{rgb}{1.000000,1.000000,1.000000}%
\pgfsetstrokecolor{currentstroke}%
\pgfsetdash{}{0pt}%
\pgfpathmoveto{\pgfqpoint{0.863783in}{2.790100in}}%
\pgfpathlineto{\pgfqpoint{5.420000in}{2.790100in}}%
\pgfusepath{stroke}%
\end{pgfscope}%
\begin{pgfscope}%
\definecolor{textcolor}{rgb}{0.150000,0.150000,0.150000}%
\pgfsetstrokecolor{textcolor}%
\pgfsetfillcolor{textcolor}%
\pgftext[x=0.476936in, y=2.735419in, left, base]{\color{textcolor}{\sffamily\fontsize{11.000000}{13.200000}\selectfont\catcode`\^=\active\def^{\ifmmode\sp\else\^{}\fi}\catcode`\%=\active\def%{\%}800}}%
\end{pgfscope}%
\begin{pgfscope}%
\pgfpathrectangle{\pgfqpoint{0.863783in}{2.546295in}}{\pgfqpoint{4.556217in}{1.219024in}}%
\pgfusepath{clip}%
\pgfsetroundcap%
\pgfsetroundjoin%
\pgfsetlinewidth{1.003750pt}%
\definecolor{currentstroke}{rgb}{1.000000,1.000000,1.000000}%
\pgfsetstrokecolor{currentstroke}%
\pgfsetdash{}{0pt}%
\pgfpathmoveto{\pgfqpoint{0.863783in}{3.277710in}}%
\pgfpathlineto{\pgfqpoint{5.420000in}{3.277710in}}%
\pgfusepath{stroke}%
\end{pgfscope}%
\begin{pgfscope}%
\definecolor{textcolor}{rgb}{0.150000,0.150000,0.150000}%
\pgfsetstrokecolor{textcolor}%
\pgfsetfillcolor{textcolor}%
\pgftext[x=0.391968in, y=3.223029in, left, base]{\color{textcolor}{\sffamily\fontsize{11.000000}{13.200000}\selectfont\catcode`\^=\active\def^{\ifmmode\sp\else\^{}\fi}\catcode`\%=\active\def%{\%}1000}}%
\end{pgfscope}%
\begin{pgfscope}%
\pgfpathrectangle{\pgfqpoint{0.863783in}{2.546295in}}{\pgfqpoint{4.556217in}{1.219024in}}%
\pgfusepath{clip}%
\pgfsetroundcap%
\pgfsetroundjoin%
\pgfsetlinewidth{1.003750pt}%
\definecolor{currentstroke}{rgb}{1.000000,1.000000,1.000000}%
\pgfsetstrokecolor{currentstroke}%
\pgfsetdash{}{0pt}%
\pgfpathmoveto{\pgfqpoint{0.863783in}{3.765319in}}%
\pgfpathlineto{\pgfqpoint{5.420000in}{3.765319in}}%
\pgfusepath{stroke}%
\end{pgfscope}%
\begin{pgfscope}%
\definecolor{textcolor}{rgb}{0.150000,0.150000,0.150000}%
\pgfsetstrokecolor{textcolor}%
\pgfsetfillcolor{textcolor}%
\pgftext[x=0.391968in, y=3.710639in, left, base]{\color{textcolor}{\sffamily\fontsize{11.000000}{13.200000}\selectfont\catcode`\^=\active\def^{\ifmmode\sp\else\^{}\fi}\catcode`\%=\active\def%{\%}1200}}%
\end{pgfscope}%
\begin{pgfscope}%
\definecolor{textcolor}{rgb}{0.150000,0.150000,0.150000}%
\pgfsetstrokecolor{textcolor}%
\pgfsetfillcolor{textcolor}%
\pgftext[x=0.336413in,y=3.155807in,,bottom,rotate=90.000000]{\color{textcolor}{\sffamily\fontsize{12.000000}{14.400000}\selectfont\catcode`\^=\active\def^{\ifmmode\sp\else\^{}\fi}\catcode`\%=\active\def%{\%}CPU Limits (milliCPU)}}%
\end{pgfscope}%
\begin{pgfscope}%
\pgfpathrectangle{\pgfqpoint{0.863783in}{2.546295in}}{\pgfqpoint{4.556217in}{1.219024in}}%
\pgfusepath{clip}%
\pgfsetroundcap%
\pgfsetroundjoin%
\pgfsetlinewidth{1.505625pt}%
\definecolor{currentstroke}{rgb}{0.298039,0.447059,0.690196}%
\pgfsetstrokecolor{currentstroke}%
\pgfsetdash{}{0pt}%
\pgfpathmoveto{\pgfqpoint{1.070884in}{3.521514in}}%
\pgfpathlineto{\pgfqpoint{1.684516in}{3.521514in}}%
\pgfpathlineto{\pgfqpoint{1.697300in}{3.033905in}}%
\pgfpathlineto{\pgfqpoint{5.212899in}{3.033905in}}%
\pgfpathlineto{\pgfqpoint{5.212899in}{3.033905in}}%
\pgfusepath{stroke}%
\end{pgfscope}%
\begin{pgfscope}%
\pgfsetrectcap%
\pgfsetmiterjoin%
\pgfsetlinewidth{1.254687pt}%
\definecolor{currentstroke}{rgb}{1.000000,1.000000,1.000000}%
\pgfsetstrokecolor{currentstroke}%
\pgfsetdash{}{0pt}%
\pgfpathmoveto{\pgfqpoint{0.863783in}{2.546295in}}%
\pgfpathlineto{\pgfqpoint{0.863783in}{3.765319in}}%
\pgfusepath{stroke}%
\end{pgfscope}%
\begin{pgfscope}%
\pgfsetrectcap%
\pgfsetmiterjoin%
\pgfsetlinewidth{1.254687pt}%
\definecolor{currentstroke}{rgb}{1.000000,1.000000,1.000000}%
\pgfsetstrokecolor{currentstroke}%
\pgfsetdash{}{0pt}%
\pgfpathmoveto{\pgfqpoint{5.420000in}{2.546295in}}%
\pgfpathlineto{\pgfqpoint{5.420000in}{3.765319in}}%
\pgfusepath{stroke}%
\end{pgfscope}%
\begin{pgfscope}%
\pgfsetrectcap%
\pgfsetmiterjoin%
\pgfsetlinewidth{1.254687pt}%
\definecolor{currentstroke}{rgb}{1.000000,1.000000,1.000000}%
\pgfsetstrokecolor{currentstroke}%
\pgfsetdash{}{0pt}%
\pgfpathmoveto{\pgfqpoint{0.863783in}{2.546295in}}%
\pgfpathlineto{\pgfqpoint{5.420000in}{2.546295in}}%
\pgfusepath{stroke}%
\end{pgfscope}%
\begin{pgfscope}%
\pgfsetrectcap%
\pgfsetmiterjoin%
\pgfsetlinewidth{1.254687pt}%
\definecolor{currentstroke}{rgb}{1.000000,1.000000,1.000000}%
\pgfsetstrokecolor{currentstroke}%
\pgfsetdash{}{0pt}%
\pgfpathmoveto{\pgfqpoint{0.863783in}{3.765319in}}%
\pgfpathlineto{\pgfqpoint{5.420000in}{3.765319in}}%
\pgfusepath{stroke}%
\end{pgfscope}%
\begin{pgfscope}%
\pgfsetbuttcap%
\pgfsetmiterjoin%
\definecolor{currentfill}{rgb}{0.917647,0.917647,0.949020}%
\pgfsetfillcolor{currentfill}%
\pgfsetlinewidth{0.000000pt}%
\definecolor{currentstroke}{rgb}{0.000000,0.000000,0.000000}%
\pgfsetstrokecolor{currentstroke}%
\pgfsetstrokeopacity{0.000000}%
\pgfsetdash{}{0pt}%
\pgfpathmoveto{\pgfqpoint{0.863783in}{0.663635in}}%
\pgfpathlineto{\pgfqpoint{5.420000in}{0.663635in}}%
\pgfpathlineto{\pgfqpoint{5.420000in}{1.882660in}}%
\pgfpathlineto{\pgfqpoint{0.863783in}{1.882660in}}%
\pgfpathlineto{\pgfqpoint{0.863783in}{0.663635in}}%
\pgfpathclose%
\pgfusepath{fill}%
\end{pgfscope}%
\begin{pgfscope}%
\pgfpathrectangle{\pgfqpoint{0.863783in}{0.663635in}}{\pgfqpoint{4.556217in}{1.219024in}}%
\pgfusepath{clip}%
\pgfsetroundcap%
\pgfsetroundjoin%
\pgfsetlinewidth{1.003750pt}%
\definecolor{currentstroke}{rgb}{1.000000,1.000000,1.000000}%
\pgfsetstrokecolor{currentstroke}%
\pgfsetdash{}{0pt}%
\pgfpathmoveto{\pgfqpoint{1.070884in}{0.663635in}}%
\pgfpathlineto{\pgfqpoint{1.070884in}{1.882660in}}%
\pgfusepath{stroke}%
\end{pgfscope}%
\begin{pgfscope}%
\definecolor{textcolor}{rgb}{0.150000,0.150000,0.150000}%
\pgfsetstrokecolor{textcolor}%
\pgfsetfillcolor{textcolor}%
\pgftext[x=1.070884in,y=0.531691in,,top]{\color{textcolor}{\sffamily\fontsize{11.000000}{13.200000}\selectfont\catcode`\^=\active\def^{\ifmmode\sp\else\^{}\fi}\catcode`\%=\active\def%{\%}0}}%
\end{pgfscope}%
\begin{pgfscope}%
\pgfpathrectangle{\pgfqpoint{0.863783in}{0.663635in}}{\pgfqpoint{4.556217in}{1.219024in}}%
\pgfusepath{clip}%
\pgfsetroundcap%
\pgfsetroundjoin%
\pgfsetlinewidth{1.003750pt}%
\definecolor{currentstroke}{rgb}{1.000000,1.000000,1.000000}%
\pgfsetstrokecolor{currentstroke}%
\pgfsetdash{}{0pt}%
\pgfpathmoveto{\pgfqpoint{1.582244in}{0.663635in}}%
\pgfpathlineto{\pgfqpoint{1.582244in}{1.882660in}}%
\pgfusepath{stroke}%
\end{pgfscope}%
\begin{pgfscope}%
\definecolor{textcolor}{rgb}{0.150000,0.150000,0.150000}%
\pgfsetstrokecolor{textcolor}%
\pgfsetfillcolor{textcolor}%
\pgftext[x=1.582244in,y=0.531691in,,top]{\color{textcolor}{\sffamily\fontsize{11.000000}{13.200000}\selectfont\catcode`\^=\active\def^{\ifmmode\sp\else\^{}\fi}\catcode`\%=\active\def%{\%}200}}%
\end{pgfscope}%
\begin{pgfscope}%
\pgfpathrectangle{\pgfqpoint{0.863783in}{0.663635in}}{\pgfqpoint{4.556217in}{1.219024in}}%
\pgfusepath{clip}%
\pgfsetroundcap%
\pgfsetroundjoin%
\pgfsetlinewidth{1.003750pt}%
\definecolor{currentstroke}{rgb}{1.000000,1.000000,1.000000}%
\pgfsetstrokecolor{currentstroke}%
\pgfsetdash{}{0pt}%
\pgfpathmoveto{\pgfqpoint{2.093604in}{0.663635in}}%
\pgfpathlineto{\pgfqpoint{2.093604in}{1.882660in}}%
\pgfusepath{stroke}%
\end{pgfscope}%
\begin{pgfscope}%
\definecolor{textcolor}{rgb}{0.150000,0.150000,0.150000}%
\pgfsetstrokecolor{textcolor}%
\pgfsetfillcolor{textcolor}%
\pgftext[x=2.093604in,y=0.531691in,,top]{\color{textcolor}{\sffamily\fontsize{11.000000}{13.200000}\selectfont\catcode`\^=\active\def^{\ifmmode\sp\else\^{}\fi}\catcode`\%=\active\def%{\%}400}}%
\end{pgfscope}%
\begin{pgfscope}%
\pgfpathrectangle{\pgfqpoint{0.863783in}{0.663635in}}{\pgfqpoint{4.556217in}{1.219024in}}%
\pgfusepath{clip}%
\pgfsetroundcap%
\pgfsetroundjoin%
\pgfsetlinewidth{1.003750pt}%
\definecolor{currentstroke}{rgb}{1.000000,1.000000,1.000000}%
\pgfsetstrokecolor{currentstroke}%
\pgfsetdash{}{0pt}%
\pgfpathmoveto{\pgfqpoint{2.604964in}{0.663635in}}%
\pgfpathlineto{\pgfqpoint{2.604964in}{1.882660in}}%
\pgfusepath{stroke}%
\end{pgfscope}%
\begin{pgfscope}%
\definecolor{textcolor}{rgb}{0.150000,0.150000,0.150000}%
\pgfsetstrokecolor{textcolor}%
\pgfsetfillcolor{textcolor}%
\pgftext[x=2.604964in,y=0.531691in,,top]{\color{textcolor}{\sffamily\fontsize{11.000000}{13.200000}\selectfont\catcode`\^=\active\def^{\ifmmode\sp\else\^{}\fi}\catcode`\%=\active\def%{\%}600}}%
\end{pgfscope}%
\begin{pgfscope}%
\pgfpathrectangle{\pgfqpoint{0.863783in}{0.663635in}}{\pgfqpoint{4.556217in}{1.219024in}}%
\pgfusepath{clip}%
\pgfsetroundcap%
\pgfsetroundjoin%
\pgfsetlinewidth{1.003750pt}%
\definecolor{currentstroke}{rgb}{1.000000,1.000000,1.000000}%
\pgfsetstrokecolor{currentstroke}%
\pgfsetdash{}{0pt}%
\pgfpathmoveto{\pgfqpoint{3.116324in}{0.663635in}}%
\pgfpathlineto{\pgfqpoint{3.116324in}{1.882660in}}%
\pgfusepath{stroke}%
\end{pgfscope}%
\begin{pgfscope}%
\definecolor{textcolor}{rgb}{0.150000,0.150000,0.150000}%
\pgfsetstrokecolor{textcolor}%
\pgfsetfillcolor{textcolor}%
\pgftext[x=3.116324in,y=0.531691in,,top]{\color{textcolor}{\sffamily\fontsize{11.000000}{13.200000}\selectfont\catcode`\^=\active\def^{\ifmmode\sp\else\^{}\fi}\catcode`\%=\active\def%{\%}800}}%
\end{pgfscope}%
\begin{pgfscope}%
\pgfpathrectangle{\pgfqpoint{0.863783in}{0.663635in}}{\pgfqpoint{4.556217in}{1.219024in}}%
\pgfusepath{clip}%
\pgfsetroundcap%
\pgfsetroundjoin%
\pgfsetlinewidth{1.003750pt}%
\definecolor{currentstroke}{rgb}{1.000000,1.000000,1.000000}%
\pgfsetstrokecolor{currentstroke}%
\pgfsetdash{}{0pt}%
\pgfpathmoveto{\pgfqpoint{3.627684in}{0.663635in}}%
\pgfpathlineto{\pgfqpoint{3.627684in}{1.882660in}}%
\pgfusepath{stroke}%
\end{pgfscope}%
\begin{pgfscope}%
\definecolor{textcolor}{rgb}{0.150000,0.150000,0.150000}%
\pgfsetstrokecolor{textcolor}%
\pgfsetfillcolor{textcolor}%
\pgftext[x=3.627684in,y=0.531691in,,top]{\color{textcolor}{\sffamily\fontsize{11.000000}{13.200000}\selectfont\catcode`\^=\active\def^{\ifmmode\sp\else\^{}\fi}\catcode`\%=\active\def%{\%}1000}}%
\end{pgfscope}%
\begin{pgfscope}%
\pgfpathrectangle{\pgfqpoint{0.863783in}{0.663635in}}{\pgfqpoint{4.556217in}{1.219024in}}%
\pgfusepath{clip}%
\pgfsetroundcap%
\pgfsetroundjoin%
\pgfsetlinewidth{1.003750pt}%
\definecolor{currentstroke}{rgb}{1.000000,1.000000,1.000000}%
\pgfsetstrokecolor{currentstroke}%
\pgfsetdash{}{0pt}%
\pgfpathmoveto{\pgfqpoint{4.139044in}{0.663635in}}%
\pgfpathlineto{\pgfqpoint{4.139044in}{1.882660in}}%
\pgfusepath{stroke}%
\end{pgfscope}%
\begin{pgfscope}%
\definecolor{textcolor}{rgb}{0.150000,0.150000,0.150000}%
\pgfsetstrokecolor{textcolor}%
\pgfsetfillcolor{textcolor}%
\pgftext[x=4.139044in,y=0.531691in,,top]{\color{textcolor}{\sffamily\fontsize{11.000000}{13.200000}\selectfont\catcode`\^=\active\def^{\ifmmode\sp\else\^{}\fi}\catcode`\%=\active\def%{\%}1200}}%
\end{pgfscope}%
\begin{pgfscope}%
\pgfpathrectangle{\pgfqpoint{0.863783in}{0.663635in}}{\pgfqpoint{4.556217in}{1.219024in}}%
\pgfusepath{clip}%
\pgfsetroundcap%
\pgfsetroundjoin%
\pgfsetlinewidth{1.003750pt}%
\definecolor{currentstroke}{rgb}{1.000000,1.000000,1.000000}%
\pgfsetstrokecolor{currentstroke}%
\pgfsetdash{}{0pt}%
\pgfpathmoveto{\pgfqpoint{4.650403in}{0.663635in}}%
\pgfpathlineto{\pgfqpoint{4.650403in}{1.882660in}}%
\pgfusepath{stroke}%
\end{pgfscope}%
\begin{pgfscope}%
\definecolor{textcolor}{rgb}{0.150000,0.150000,0.150000}%
\pgfsetstrokecolor{textcolor}%
\pgfsetfillcolor{textcolor}%
\pgftext[x=4.650403in,y=0.531691in,,top]{\color{textcolor}{\sffamily\fontsize{11.000000}{13.200000}\selectfont\catcode`\^=\active\def^{\ifmmode\sp\else\^{}\fi}\catcode`\%=\active\def%{\%}1400}}%
\end{pgfscope}%
\begin{pgfscope}%
\pgfpathrectangle{\pgfqpoint{0.863783in}{0.663635in}}{\pgfqpoint{4.556217in}{1.219024in}}%
\pgfusepath{clip}%
\pgfsetroundcap%
\pgfsetroundjoin%
\pgfsetlinewidth{1.003750pt}%
\definecolor{currentstroke}{rgb}{1.000000,1.000000,1.000000}%
\pgfsetstrokecolor{currentstroke}%
\pgfsetdash{}{0pt}%
\pgfpathmoveto{\pgfqpoint{5.161763in}{0.663635in}}%
\pgfpathlineto{\pgfqpoint{5.161763in}{1.882660in}}%
\pgfusepath{stroke}%
\end{pgfscope}%
\begin{pgfscope}%
\definecolor{textcolor}{rgb}{0.150000,0.150000,0.150000}%
\pgfsetstrokecolor{textcolor}%
\pgfsetfillcolor{textcolor}%
\pgftext[x=5.161763in,y=0.531691in,,top]{\color{textcolor}{\sffamily\fontsize{11.000000}{13.200000}\selectfont\catcode`\^=\active\def^{\ifmmode\sp\else\^{}\fi}\catcode`\%=\active\def%{\%}1600}}%
\end{pgfscope}%
\begin{pgfscope}%
\definecolor{textcolor}{rgb}{0.150000,0.150000,0.150000}%
\pgfsetstrokecolor{textcolor}%
\pgfsetfillcolor{textcolor}%
\pgftext[x=3.141892in,y=0.336413in,,top]{\color{textcolor}{\sffamily\fontsize{12.000000}{14.400000}\selectfont\catcode`\^=\active\def^{\ifmmode\sp\else\^{}\fi}\catcode`\%=\active\def%{\%}Time (s)}}%
\end{pgfscope}%
\begin{pgfscope}%
\pgfpathrectangle{\pgfqpoint{0.863783in}{0.663635in}}{\pgfqpoint{4.556217in}{1.219024in}}%
\pgfusepath{clip}%
\pgfsetroundcap%
\pgfsetroundjoin%
\pgfsetlinewidth{1.003750pt}%
\definecolor{currentstroke}{rgb}{1.000000,1.000000,1.000000}%
\pgfsetstrokecolor{currentstroke}%
\pgfsetdash{}{0pt}%
\pgfpathmoveto{\pgfqpoint{0.863783in}{0.934530in}}%
\pgfpathlineto{\pgfqpoint{5.420000in}{0.934530in}}%
\pgfusepath{stroke}%
\end{pgfscope}%
\begin{pgfscope}%
\definecolor{textcolor}{rgb}{0.150000,0.150000,0.150000}%
\pgfsetstrokecolor{textcolor}%
\pgfsetfillcolor{textcolor}%
\pgftext[x=0.391968in, y=0.879849in, left, base]{\color{textcolor}{\sffamily\fontsize{11.000000}{13.200000}\selectfont\catcode`\^=\active\def^{\ifmmode\sp\else\^{}\fi}\catcode`\%=\active\def%{\%}4000}}%
\end{pgfscope}%
\begin{pgfscope}%
\pgfpathrectangle{\pgfqpoint{0.863783in}{0.663635in}}{\pgfqpoint{4.556217in}{1.219024in}}%
\pgfusepath{clip}%
\pgfsetroundcap%
\pgfsetroundjoin%
\pgfsetlinewidth{1.003750pt}%
\definecolor{currentstroke}{rgb}{1.000000,1.000000,1.000000}%
\pgfsetstrokecolor{currentstroke}%
\pgfsetdash{}{0pt}%
\pgfpathmoveto{\pgfqpoint{0.863783in}{1.476318in}}%
\pgfpathlineto{\pgfqpoint{5.420000in}{1.476318in}}%
\pgfusepath{stroke}%
\end{pgfscope}%
\begin{pgfscope}%
\definecolor{textcolor}{rgb}{0.150000,0.150000,0.150000}%
\pgfsetstrokecolor{textcolor}%
\pgfsetfillcolor{textcolor}%
\pgftext[x=0.391968in, y=1.421638in, left, base]{\color{textcolor}{\sffamily\fontsize{11.000000}{13.200000}\selectfont\catcode`\^=\active\def^{\ifmmode\sp\else\^{}\fi}\catcode`\%=\active\def%{\%}6000}}%
\end{pgfscope}%
\begin{pgfscope}%
\definecolor{textcolor}{rgb}{0.150000,0.150000,0.150000}%
\pgfsetstrokecolor{textcolor}%
\pgfsetfillcolor{textcolor}%
\pgftext[x=0.336413in,y=1.273148in,,bottom,rotate=90.000000]{\color{textcolor}{\sffamily\fontsize{12.000000}{14.400000}\selectfont\catcode`\^=\active\def^{\ifmmode\sp\else\^{}\fi}\catcode`\%=\active\def%{\%}Memory Limits (MiB)}}%
\end{pgfscope}%
\begin{pgfscope}%
\pgfpathrectangle{\pgfqpoint{0.863783in}{0.663635in}}{\pgfqpoint{4.556217in}{1.219024in}}%
\pgfusepath{clip}%
\pgfsetroundcap%
\pgfsetroundjoin%
\pgfsetlinewidth{1.505625pt}%
\definecolor{currentstroke}{rgb}{0.298039,0.447059,0.690196}%
\pgfsetstrokecolor{currentstroke}%
\pgfsetdash{}{0pt}%
\pgfpathmoveto{\pgfqpoint{1.070884in}{1.572609in}}%
\pgfpathlineto{\pgfqpoint{1.684516in}{1.572609in}}%
\pgfpathlineto{\pgfqpoint{1.697300in}{0.924968in}}%
\pgfpathlineto{\pgfqpoint{5.212899in}{0.924968in}}%
\pgfpathlineto{\pgfqpoint{5.212899in}{0.924968in}}%
\pgfusepath{stroke}%
\end{pgfscope}%
\begin{pgfscope}%
\pgfsetrectcap%
\pgfsetmiterjoin%
\pgfsetlinewidth{1.254687pt}%
\definecolor{currentstroke}{rgb}{1.000000,1.000000,1.000000}%
\pgfsetstrokecolor{currentstroke}%
\pgfsetdash{}{0pt}%
\pgfpathmoveto{\pgfqpoint{0.863783in}{0.663635in}}%
\pgfpathlineto{\pgfqpoint{0.863783in}{1.882660in}}%
\pgfusepath{stroke}%
\end{pgfscope}%
\begin{pgfscope}%
\pgfsetrectcap%
\pgfsetmiterjoin%
\pgfsetlinewidth{1.254687pt}%
\definecolor{currentstroke}{rgb}{1.000000,1.000000,1.000000}%
\pgfsetstrokecolor{currentstroke}%
\pgfsetdash{}{0pt}%
\pgfpathmoveto{\pgfqpoint{5.420000in}{0.663635in}}%
\pgfpathlineto{\pgfqpoint{5.420000in}{1.882660in}}%
\pgfusepath{stroke}%
\end{pgfscope}%
\begin{pgfscope}%
\pgfsetrectcap%
\pgfsetmiterjoin%
\pgfsetlinewidth{1.254687pt}%
\definecolor{currentstroke}{rgb}{1.000000,1.000000,1.000000}%
\pgfsetstrokecolor{currentstroke}%
\pgfsetdash{}{0pt}%
\pgfpathmoveto{\pgfqpoint{0.863783in}{0.663635in}}%
\pgfpathlineto{\pgfqpoint{5.420000in}{0.663635in}}%
\pgfusepath{stroke}%
\end{pgfscope}%
\begin{pgfscope}%
\pgfsetrectcap%
\pgfsetmiterjoin%
\pgfsetlinewidth{1.254687pt}%
\definecolor{currentstroke}{rgb}{1.000000,1.000000,1.000000}%
\pgfsetstrokecolor{currentstroke}%
\pgfsetdash{}{0pt}%
\pgfpathmoveto{\pgfqpoint{0.863783in}{1.882660in}}%
\pgfpathlineto{\pgfqpoint{5.420000in}{1.882660in}}%
\pgfusepath{stroke}%
\end{pgfscope}%
\end{pgfpicture}%
\makeatother%
\endgroup%

    \caption{Adjustment of CPU and memory limits by the vertical elasticity strategy controller}
    \label{fig:simple-limits-vertical}
\end{figure}

\begin{figure}[H]
    \centering
    %% Creator: Matplotlib, PGF backend
%%
%% To include the figure in your LaTeX document, write
%%   \input{<filename>.pgf}
%%
%% Make sure the required packages are loaded in your preamble
%%   \usepackage{pgf}
%%
%% Also ensure that all the required font packages are loaded; for instance,
%% the lmodern package is sometimes necessary when using math font.
%%   \usepackage{lmodern}
%%
%% Figures using additional raster images can only be included by \input if
%% they are in the same directory as the main LaTeX file. For loading figures
%% from other directories you can use the `import` package
%%   \usepackage{import}
%%
%% and then include the figures with
%%   \import{<path to file>}{<filename>.pgf}
%%
%% Matplotlib used the following preamble
%%   \def\mathdefault#1{#1}
%%   \everymath=\expandafter{\the\everymath\displaystyle}
%%   
%%   \usepackage{fontspec}
%%   \setmainfont{DejaVuSerif.ttf}[Path=\detokenize{/usr/local/lib/python3.11/site-packages/matplotlib/mpl-data/fonts/ttf/}]
%%   \setsansfont{Arial.ttf}[Path=\detokenize{/System/Library/Fonts/Supplemental/}]
%%   \setmonofont{DejaVuSansMono.ttf}[Path=\detokenize{/usr/local/lib/python3.11/site-packages/matplotlib/mpl-data/fonts/ttf/}]
%%   \makeatletter\@ifpackageloaded{underscore}{}{\usepackage[strings]{underscore}}\makeatother
%%
\begingroup%
\makeatletter%
\begin{pgfpicture}%
\pgfpathrectangle{\pgfpointorigin}{\pgfqpoint{5.600000in}{3.500000in}}%
\pgfusepath{use as bounding box, clip}%
\begin{pgfscope}%
\pgfsetbuttcap%
\pgfsetmiterjoin%
\definecolor{currentfill}{rgb}{1.000000,1.000000,1.000000}%
\pgfsetfillcolor{currentfill}%
\pgfsetlinewidth{0.000000pt}%
\definecolor{currentstroke}{rgb}{1.000000,1.000000,1.000000}%
\pgfsetstrokecolor{currentstroke}%
\pgfsetdash{}{0pt}%
\pgfpathmoveto{\pgfqpoint{0.000000in}{0.000000in}}%
\pgfpathlineto{\pgfqpoint{5.600000in}{0.000000in}}%
\pgfpathlineto{\pgfqpoint{5.600000in}{3.500000in}}%
\pgfpathlineto{\pgfqpoint{0.000000in}{3.500000in}}%
\pgfpathlineto{\pgfqpoint{0.000000in}{0.000000in}}%
\pgfpathclose%
\pgfusepath{fill}%
\end{pgfscope}%
\begin{pgfscope}%
\pgfsetbuttcap%
\pgfsetmiterjoin%
\definecolor{currentfill}{rgb}{0.917647,0.917647,0.949020}%
\pgfsetfillcolor{currentfill}%
\pgfsetlinewidth{0.000000pt}%
\definecolor{currentstroke}{rgb}{0.000000,0.000000,0.000000}%
\pgfsetstrokecolor{currentstroke}%
\pgfsetstrokeopacity{0.000000}%
\pgfsetdash{}{0pt}%
\pgfpathmoveto{\pgfqpoint{0.736295in}{2.323635in}}%
\pgfpathlineto{\pgfqpoint{5.420000in}{2.323635in}}%
\pgfpathlineto{\pgfqpoint{5.420000in}{3.320000in}}%
\pgfpathlineto{\pgfqpoint{0.736295in}{3.320000in}}%
\pgfpathlineto{\pgfqpoint{0.736295in}{2.323635in}}%
\pgfpathclose%
\pgfusepath{fill}%
\end{pgfscope}%
\begin{pgfscope}%
\pgfpathrectangle{\pgfqpoint{0.736295in}{2.323635in}}{\pgfqpoint{4.683705in}{0.996365in}}%
\pgfusepath{clip}%
\pgfsetroundcap%
\pgfsetroundjoin%
\pgfsetlinewidth{1.003750pt}%
\definecolor{currentstroke}{rgb}{1.000000,1.000000,1.000000}%
\pgfsetstrokecolor{currentstroke}%
\pgfsetdash{}{0pt}%
\pgfpathmoveto{\pgfqpoint{0.949190in}{2.323635in}}%
\pgfpathlineto{\pgfqpoint{0.949190in}{3.320000in}}%
\pgfusepath{stroke}%
\end{pgfscope}%
\begin{pgfscope}%
\definecolor{textcolor}{rgb}{0.150000,0.150000,0.150000}%
\pgfsetstrokecolor{textcolor}%
\pgfsetfillcolor{textcolor}%
\pgftext[x=0.949190in,y=2.191691in,,top]{\color{textcolor}{\sffamily\fontsize{11.000000}{13.200000}\selectfont\catcode`\^=\active\def^{\ifmmode\sp\else\^{}\fi}\catcode`\%=\active\def%{\%}0}}%
\end{pgfscope}%
\begin{pgfscope}%
\pgfpathrectangle{\pgfqpoint{0.736295in}{2.323635in}}{\pgfqpoint{4.683705in}{0.996365in}}%
\pgfusepath{clip}%
\pgfsetroundcap%
\pgfsetroundjoin%
\pgfsetlinewidth{1.003750pt}%
\definecolor{currentstroke}{rgb}{1.000000,1.000000,1.000000}%
\pgfsetstrokecolor{currentstroke}%
\pgfsetdash{}{0pt}%
\pgfpathmoveto{\pgfqpoint{1.658843in}{2.323635in}}%
\pgfpathlineto{\pgfqpoint{1.658843in}{3.320000in}}%
\pgfusepath{stroke}%
\end{pgfscope}%
\begin{pgfscope}%
\definecolor{textcolor}{rgb}{0.150000,0.150000,0.150000}%
\pgfsetstrokecolor{textcolor}%
\pgfsetfillcolor{textcolor}%
\pgftext[x=1.658843in,y=2.191691in,,top]{\color{textcolor}{\sffamily\fontsize{11.000000}{13.200000}\selectfont\catcode`\^=\active\def^{\ifmmode\sp\else\^{}\fi}\catcode`\%=\active\def%{\%}200}}%
\end{pgfscope}%
\begin{pgfscope}%
\pgfpathrectangle{\pgfqpoint{0.736295in}{2.323635in}}{\pgfqpoint{4.683705in}{0.996365in}}%
\pgfusepath{clip}%
\pgfsetroundcap%
\pgfsetroundjoin%
\pgfsetlinewidth{1.003750pt}%
\definecolor{currentstroke}{rgb}{1.000000,1.000000,1.000000}%
\pgfsetstrokecolor{currentstroke}%
\pgfsetdash{}{0pt}%
\pgfpathmoveto{\pgfqpoint{2.368495in}{2.323635in}}%
\pgfpathlineto{\pgfqpoint{2.368495in}{3.320000in}}%
\pgfusepath{stroke}%
\end{pgfscope}%
\begin{pgfscope}%
\definecolor{textcolor}{rgb}{0.150000,0.150000,0.150000}%
\pgfsetstrokecolor{textcolor}%
\pgfsetfillcolor{textcolor}%
\pgftext[x=2.368495in,y=2.191691in,,top]{\color{textcolor}{\sffamily\fontsize{11.000000}{13.200000}\selectfont\catcode`\^=\active\def^{\ifmmode\sp\else\^{}\fi}\catcode`\%=\active\def%{\%}400}}%
\end{pgfscope}%
\begin{pgfscope}%
\pgfpathrectangle{\pgfqpoint{0.736295in}{2.323635in}}{\pgfqpoint{4.683705in}{0.996365in}}%
\pgfusepath{clip}%
\pgfsetroundcap%
\pgfsetroundjoin%
\pgfsetlinewidth{1.003750pt}%
\definecolor{currentstroke}{rgb}{1.000000,1.000000,1.000000}%
\pgfsetstrokecolor{currentstroke}%
\pgfsetdash{}{0pt}%
\pgfpathmoveto{\pgfqpoint{3.078147in}{2.323635in}}%
\pgfpathlineto{\pgfqpoint{3.078147in}{3.320000in}}%
\pgfusepath{stroke}%
\end{pgfscope}%
\begin{pgfscope}%
\definecolor{textcolor}{rgb}{0.150000,0.150000,0.150000}%
\pgfsetstrokecolor{textcolor}%
\pgfsetfillcolor{textcolor}%
\pgftext[x=3.078147in,y=2.191691in,,top]{\color{textcolor}{\sffamily\fontsize{11.000000}{13.200000}\selectfont\catcode`\^=\active\def^{\ifmmode\sp\else\^{}\fi}\catcode`\%=\active\def%{\%}600}}%
\end{pgfscope}%
\begin{pgfscope}%
\pgfpathrectangle{\pgfqpoint{0.736295in}{2.323635in}}{\pgfqpoint{4.683705in}{0.996365in}}%
\pgfusepath{clip}%
\pgfsetroundcap%
\pgfsetroundjoin%
\pgfsetlinewidth{1.003750pt}%
\definecolor{currentstroke}{rgb}{1.000000,1.000000,1.000000}%
\pgfsetstrokecolor{currentstroke}%
\pgfsetdash{}{0pt}%
\pgfpathmoveto{\pgfqpoint{3.787800in}{2.323635in}}%
\pgfpathlineto{\pgfqpoint{3.787800in}{3.320000in}}%
\pgfusepath{stroke}%
\end{pgfscope}%
\begin{pgfscope}%
\definecolor{textcolor}{rgb}{0.150000,0.150000,0.150000}%
\pgfsetstrokecolor{textcolor}%
\pgfsetfillcolor{textcolor}%
\pgftext[x=3.787800in,y=2.191691in,,top]{\color{textcolor}{\sffamily\fontsize{11.000000}{13.200000}\selectfont\catcode`\^=\active\def^{\ifmmode\sp\else\^{}\fi}\catcode`\%=\active\def%{\%}800}}%
\end{pgfscope}%
\begin{pgfscope}%
\pgfpathrectangle{\pgfqpoint{0.736295in}{2.323635in}}{\pgfqpoint{4.683705in}{0.996365in}}%
\pgfusepath{clip}%
\pgfsetroundcap%
\pgfsetroundjoin%
\pgfsetlinewidth{1.003750pt}%
\definecolor{currentstroke}{rgb}{1.000000,1.000000,1.000000}%
\pgfsetstrokecolor{currentstroke}%
\pgfsetdash{}{0pt}%
\pgfpathmoveto{\pgfqpoint{4.497452in}{2.323635in}}%
\pgfpathlineto{\pgfqpoint{4.497452in}{3.320000in}}%
\pgfusepath{stroke}%
\end{pgfscope}%
\begin{pgfscope}%
\definecolor{textcolor}{rgb}{0.150000,0.150000,0.150000}%
\pgfsetstrokecolor{textcolor}%
\pgfsetfillcolor{textcolor}%
\pgftext[x=4.497452in,y=2.191691in,,top]{\color{textcolor}{\sffamily\fontsize{11.000000}{13.200000}\selectfont\catcode`\^=\active\def^{\ifmmode\sp\else\^{}\fi}\catcode`\%=\active\def%{\%}1000}}%
\end{pgfscope}%
\begin{pgfscope}%
\pgfpathrectangle{\pgfqpoint{0.736295in}{2.323635in}}{\pgfqpoint{4.683705in}{0.996365in}}%
\pgfusepath{clip}%
\pgfsetroundcap%
\pgfsetroundjoin%
\pgfsetlinewidth{1.003750pt}%
\definecolor{currentstroke}{rgb}{1.000000,1.000000,1.000000}%
\pgfsetstrokecolor{currentstroke}%
\pgfsetdash{}{0pt}%
\pgfpathmoveto{\pgfqpoint{5.207104in}{2.323635in}}%
\pgfpathlineto{\pgfqpoint{5.207104in}{3.320000in}}%
\pgfusepath{stroke}%
\end{pgfscope}%
\begin{pgfscope}%
\definecolor{textcolor}{rgb}{0.150000,0.150000,0.150000}%
\pgfsetstrokecolor{textcolor}%
\pgfsetfillcolor{textcolor}%
\pgftext[x=5.207104in,y=2.191691in,,top]{\color{textcolor}{\sffamily\fontsize{11.000000}{13.200000}\selectfont\catcode`\^=\active\def^{\ifmmode\sp\else\^{}\fi}\catcode`\%=\active\def%{\%}1200}}%
\end{pgfscope}%
\begin{pgfscope}%
\definecolor{textcolor}{rgb}{0.150000,0.150000,0.150000}%
\pgfsetstrokecolor{textcolor}%
\pgfsetfillcolor{textcolor}%
\pgftext[x=3.078147in,y=1.996413in,,top]{\color{textcolor}{\sffamily\fontsize{12.000000}{14.400000}\selectfont\catcode`\^=\active\def^{\ifmmode\sp\else\^{}\fi}\catcode`\%=\active\def%{\%}Time (s)}}%
\end{pgfscope}%
\begin{pgfscope}%
\pgfpathrectangle{\pgfqpoint{0.736295in}{2.323635in}}{\pgfqpoint{4.683705in}{0.996365in}}%
\pgfusepath{clip}%
\pgfsetroundcap%
\pgfsetroundjoin%
\pgfsetlinewidth{1.003750pt}%
\definecolor{currentstroke}{rgb}{1.000000,1.000000,1.000000}%
\pgfsetstrokecolor{currentstroke}%
\pgfsetdash{}{0pt}%
\pgfpathmoveto{\pgfqpoint{0.736295in}{2.406666in}}%
\pgfpathlineto{\pgfqpoint{5.420000in}{2.406666in}}%
\pgfusepath{stroke}%
\end{pgfscope}%
\begin{pgfscope}%
\definecolor{textcolor}{rgb}{0.150000,0.150000,0.150000}%
\pgfsetstrokecolor{textcolor}%
\pgfsetfillcolor{textcolor}%
\pgftext[x=0.391968in, y=2.351985in, left, base]{\color{textcolor}{\sffamily\fontsize{11.000000}{13.200000}\selectfont\catcode`\^=\active\def^{\ifmmode\sp\else\^{}\fi}\catcode`\%=\active\def%{\%}0.0}}%
\end{pgfscope}%
\begin{pgfscope}%
\pgfpathrectangle{\pgfqpoint{0.736295in}{2.323635in}}{\pgfqpoint{4.683705in}{0.996365in}}%
\pgfusepath{clip}%
\pgfsetroundcap%
\pgfsetroundjoin%
\pgfsetlinewidth{1.003750pt}%
\definecolor{currentstroke}{rgb}{1.000000,1.000000,1.000000}%
\pgfsetstrokecolor{currentstroke}%
\pgfsetdash{}{0pt}%
\pgfpathmoveto{\pgfqpoint{0.736295in}{2.821818in}}%
\pgfpathlineto{\pgfqpoint{5.420000in}{2.821818in}}%
\pgfusepath{stroke}%
\end{pgfscope}%
\begin{pgfscope}%
\definecolor{textcolor}{rgb}{0.150000,0.150000,0.150000}%
\pgfsetstrokecolor{textcolor}%
\pgfsetfillcolor{textcolor}%
\pgftext[x=0.391968in, y=2.767137in, left, base]{\color{textcolor}{\sffamily\fontsize{11.000000}{13.200000}\selectfont\catcode`\^=\active\def^{\ifmmode\sp\else\^{}\fi}\catcode`\%=\active\def%{\%}0.5}}%
\end{pgfscope}%
\begin{pgfscope}%
\pgfpathrectangle{\pgfqpoint{0.736295in}{2.323635in}}{\pgfqpoint{4.683705in}{0.996365in}}%
\pgfusepath{clip}%
\pgfsetroundcap%
\pgfsetroundjoin%
\pgfsetlinewidth{1.003750pt}%
\definecolor{currentstroke}{rgb}{1.000000,1.000000,1.000000}%
\pgfsetstrokecolor{currentstroke}%
\pgfsetdash{}{0pt}%
\pgfpathmoveto{\pgfqpoint{0.736295in}{3.236970in}}%
\pgfpathlineto{\pgfqpoint{5.420000in}{3.236970in}}%
\pgfusepath{stroke}%
\end{pgfscope}%
\begin{pgfscope}%
\definecolor{textcolor}{rgb}{0.150000,0.150000,0.150000}%
\pgfsetstrokecolor{textcolor}%
\pgfsetfillcolor{textcolor}%
\pgftext[x=0.391968in, y=3.182289in, left, base]{\color{textcolor}{\sffamily\fontsize{11.000000}{13.200000}\selectfont\catcode`\^=\active\def^{\ifmmode\sp\else\^{}\fi}\catcode`\%=\active\def%{\%}1.0}}%
\end{pgfscope}%
\begin{pgfscope}%
\definecolor{textcolor}{rgb}{0.150000,0.150000,0.150000}%
\pgfsetstrokecolor{textcolor}%
\pgfsetfillcolor{textcolor}%
\pgftext[x=0.336413in,y=2.821818in,,bottom,rotate=90.000000]{\color{textcolor}{\sffamily\fontsize{12.000000}{14.400000}\selectfont\catcode`\^=\active\def^{\ifmmode\sp\else\^{}\fi}\catcode`\%=\active\def%{\%}CPU Utilization (%)}}%
\end{pgfscope}%
\begin{pgfscope}%
\pgfpathrectangle{\pgfqpoint{0.736295in}{2.323635in}}{\pgfqpoint{4.683705in}{0.996365in}}%
\pgfusepath{clip}%
\pgfsetroundcap%
\pgfsetroundjoin%
\pgfsetlinewidth{1.505625pt}%
\definecolor{currentstroke}{rgb}{0.298039,0.447059,0.690196}%
\pgfsetstrokecolor{currentstroke}%
\pgfsetdash{}{0pt}%
\pgfpathmoveto{\pgfqpoint{0.949190in}{2.461330in}}%
\pgfpathlineto{\pgfqpoint{1.144345in}{2.460406in}}%
\pgfpathlineto{\pgfqpoint{1.179827in}{2.459655in}}%
\pgfpathlineto{\pgfqpoint{1.428206in}{2.459341in}}%
\pgfpathlineto{\pgfqpoint{1.445947in}{2.457568in}}%
\pgfpathlineto{\pgfqpoint{1.534653in}{2.457568in}}%
\pgfpathlineto{\pgfqpoint{1.552395in}{2.454686in}}%
\pgfpathlineto{\pgfqpoint{1.605619in}{2.454686in}}%
\pgfpathlineto{\pgfqpoint{1.623360in}{2.453504in}}%
\pgfpathlineto{\pgfqpoint{1.694325in}{2.453434in}}%
\pgfpathlineto{\pgfqpoint{1.712067in}{2.446276in}}%
\pgfpathlineto{\pgfqpoint{1.836256in}{2.447643in}}%
\pgfpathlineto{\pgfqpoint{1.853997in}{2.462667in}}%
\pgfpathlineto{\pgfqpoint{1.907221in}{2.462685in}}%
\pgfpathlineto{\pgfqpoint{1.924962in}{2.464368in}}%
\pgfpathlineto{\pgfqpoint{1.978186in}{2.464521in}}%
\pgfpathlineto{\pgfqpoint{1.995927in}{2.462376in}}%
\pgfpathlineto{\pgfqpoint{2.013669in}{2.462451in}}%
\pgfpathlineto{\pgfqpoint{2.031410in}{2.461045in}}%
\pgfpathlineto{\pgfqpoint{2.084634in}{2.461241in}}%
\pgfpathlineto{\pgfqpoint{2.102375in}{2.459602in}}%
\pgfpathlineto{\pgfqpoint{2.191082in}{2.459005in}}%
\pgfpathlineto{\pgfqpoint{2.244306in}{2.457727in}}%
\pgfpathlineto{\pgfqpoint{2.297530in}{2.458042in}}%
\pgfpathlineto{\pgfqpoint{2.350754in}{2.456965in}}%
\pgfpathlineto{\pgfqpoint{2.403978in}{2.457254in}}%
\pgfpathlineto{\pgfqpoint{2.457201in}{2.456644in}}%
\pgfpathlineto{\pgfqpoint{2.492684in}{2.456885in}}%
\pgfpathlineto{\pgfqpoint{2.545908in}{2.455912in}}%
\pgfpathlineto{\pgfqpoint{2.634615in}{2.454941in}}%
\pgfpathlineto{\pgfqpoint{2.652356in}{2.453077in}}%
\pgfpathlineto{\pgfqpoint{2.670097in}{2.453507in}}%
\pgfpathlineto{\pgfqpoint{2.687839in}{2.451254in}}%
\pgfpathlineto{\pgfqpoint{2.723321in}{2.451254in}}%
\pgfpathlineto{\pgfqpoint{2.741062in}{2.453592in}}%
\pgfpathlineto{\pgfqpoint{2.829769in}{2.453293in}}%
\pgfpathlineto{\pgfqpoint{3.202336in}{2.452847in}}%
\pgfpathlineto{\pgfqpoint{3.220078in}{2.454610in}}%
\pgfpathlineto{\pgfqpoint{3.273302in}{2.454398in}}%
\pgfpathlineto{\pgfqpoint{3.291043in}{2.452926in}}%
\pgfpathlineto{\pgfqpoint{3.308784in}{2.452827in}}%
\pgfpathlineto{\pgfqpoint{3.326526in}{2.459274in}}%
\pgfpathlineto{\pgfqpoint{3.379750in}{2.458990in}}%
\pgfpathlineto{\pgfqpoint{3.397491in}{2.456905in}}%
\pgfpathlineto{\pgfqpoint{3.787800in}{2.455749in}}%
\pgfpathlineto{\pgfqpoint{3.805541in}{2.458019in}}%
\pgfpathlineto{\pgfqpoint{3.858765in}{2.458208in}}%
\pgfpathlineto{\pgfqpoint{3.876506in}{2.456422in}}%
\pgfpathlineto{\pgfqpoint{4.781313in}{2.455947in}}%
\pgfpathlineto{\pgfqpoint{4.799054in}{2.454030in}}%
\pgfpathlineto{\pgfqpoint{4.852278in}{2.454030in}}%
\pgfpathlineto{\pgfqpoint{4.870019in}{2.456044in}}%
\pgfpathlineto{\pgfqpoint{5.207104in}{2.455924in}}%
\pgfpathlineto{\pgfqpoint{5.207104in}{2.455924in}}%
\pgfusepath{stroke}%
\end{pgfscope}%
\begin{pgfscope}%
\pgfpathrectangle{\pgfqpoint{0.736295in}{2.323635in}}{\pgfqpoint{4.683705in}{0.996365in}}%
\pgfusepath{clip}%
\pgfsetbuttcap%
\pgfsetroundjoin%
\pgfsetlinewidth{1.505625pt}%
\definecolor{currentstroke}{rgb}{1.000000,0.647059,0.000000}%
\pgfsetstrokecolor{currentstroke}%
\pgfsetdash{{5.550000pt}{2.400000pt}}{0.000000pt}%
\pgfpathmoveto{\pgfqpoint{1.026542in}{2.323635in}}%
\pgfpathlineto{\pgfqpoint{1.026542in}{3.320000in}}%
\pgfusepath{stroke}%
\end{pgfscope}%
\begin{pgfscope}%
\pgfsetrectcap%
\pgfsetmiterjoin%
\pgfsetlinewidth{1.254687pt}%
\definecolor{currentstroke}{rgb}{1.000000,1.000000,1.000000}%
\pgfsetstrokecolor{currentstroke}%
\pgfsetdash{}{0pt}%
\pgfpathmoveto{\pgfqpoint{0.736295in}{2.323635in}}%
\pgfpathlineto{\pgfqpoint{0.736295in}{3.320000in}}%
\pgfusepath{stroke}%
\end{pgfscope}%
\begin{pgfscope}%
\pgfsetrectcap%
\pgfsetmiterjoin%
\pgfsetlinewidth{1.254687pt}%
\definecolor{currentstroke}{rgb}{1.000000,1.000000,1.000000}%
\pgfsetstrokecolor{currentstroke}%
\pgfsetdash{}{0pt}%
\pgfpathmoveto{\pgfqpoint{5.420000in}{2.323635in}}%
\pgfpathlineto{\pgfqpoint{5.420000in}{3.320000in}}%
\pgfusepath{stroke}%
\end{pgfscope}%
\begin{pgfscope}%
\pgfsetrectcap%
\pgfsetmiterjoin%
\pgfsetlinewidth{1.254687pt}%
\definecolor{currentstroke}{rgb}{1.000000,1.000000,1.000000}%
\pgfsetstrokecolor{currentstroke}%
\pgfsetdash{}{0pt}%
\pgfpathmoveto{\pgfqpoint{0.736295in}{2.323635in}}%
\pgfpathlineto{\pgfqpoint{5.420000in}{2.323635in}}%
\pgfusepath{stroke}%
\end{pgfscope}%
\begin{pgfscope}%
\pgfsetrectcap%
\pgfsetmiterjoin%
\pgfsetlinewidth{1.254687pt}%
\definecolor{currentstroke}{rgb}{1.000000,1.000000,1.000000}%
\pgfsetstrokecolor{currentstroke}%
\pgfsetdash{}{0pt}%
\pgfpathmoveto{\pgfqpoint{0.736295in}{3.320000in}}%
\pgfpathlineto{\pgfqpoint{5.420000in}{3.320000in}}%
\pgfusepath{stroke}%
\end{pgfscope}%
\begin{pgfscope}%
\pgfsetbuttcap%
\pgfsetmiterjoin%
\definecolor{currentfill}{rgb}{0.917647,0.917647,0.949020}%
\pgfsetfillcolor{currentfill}%
\pgfsetfillopacity{0.800000}%
\pgfsetlinewidth{1.003750pt}%
\definecolor{currentstroke}{rgb}{0.800000,0.800000,0.800000}%
\pgfsetstrokecolor{currentstroke}%
\pgfsetstrokeopacity{0.800000}%
\pgfsetdash{}{0pt}%
\pgfpathmoveto{\pgfqpoint{3.276163in}{3.071281in}}%
\pgfpathlineto{\pgfqpoint{5.342222in}{3.071281in}}%
\pgfpathquadraticcurveto{\pgfqpoint{5.364444in}{3.071281in}}{\pgfqpoint{5.364444in}{3.093503in}}%
\pgfpathlineto{\pgfqpoint{5.364444in}{3.242222in}}%
\pgfpathquadraticcurveto{\pgfqpoint{5.364444in}{3.264444in}}{\pgfqpoint{5.342222in}{3.264444in}}%
\pgfpathlineto{\pgfqpoint{3.276163in}{3.264444in}}%
\pgfpathquadraticcurveto{\pgfqpoint{3.253941in}{3.264444in}}{\pgfqpoint{3.253941in}{3.242222in}}%
\pgfpathlineto{\pgfqpoint{3.253941in}{3.093503in}}%
\pgfpathquadraticcurveto{\pgfqpoint{3.253941in}{3.071281in}}{\pgfqpoint{3.276163in}{3.071281in}}%
\pgfpathlineto{\pgfqpoint{3.276163in}{3.071281in}}%
\pgfpathclose%
\pgfusepath{stroke,fill}%
\end{pgfscope}%
\begin{pgfscope}%
\pgfsetbuttcap%
\pgfsetroundjoin%
\pgfsetlinewidth{1.505625pt}%
\definecolor{currentstroke}{rgb}{1.000000,0.647059,0.000000}%
\pgfsetstrokecolor{currentstroke}%
\pgfsetdash{{5.550000pt}{2.400000pt}}{0.000000pt}%
\pgfpathmoveto{\pgfqpoint{3.298385in}{3.177997in}}%
\pgfpathlineto{\pgfqpoint{3.409497in}{3.177997in}}%
\pgfpathlineto{\pgfqpoint{3.520608in}{3.177997in}}%
\pgfusepath{stroke}%
\end{pgfscope}%
\begin{pgfscope}%
\definecolor{textcolor}{rgb}{0.150000,0.150000,0.150000}%
\pgfsetstrokecolor{textcolor}%
\pgfsetfillcolor{textcolor}%
\pgftext[x=3.609497in,y=3.139108in,left,base]{\color{textcolor}{\sffamily\fontsize{8.000000}{9.600000}\selectfont\catcode`\^=\active\def^{\ifmmode\sp\else\^{}\fi}\catcode`\%=\active\def%{\%}start of elasticity strategy controller}}%
\end{pgfscope}%
\begin{pgfscope}%
\pgfsetbuttcap%
\pgfsetmiterjoin%
\definecolor{currentfill}{rgb}{0.917647,0.917647,0.949020}%
\pgfsetfillcolor{currentfill}%
\pgfsetlinewidth{0.000000pt}%
\definecolor{currentstroke}{rgb}{0.000000,0.000000,0.000000}%
\pgfsetstrokecolor{currentstroke}%
\pgfsetstrokeopacity{0.000000}%
\pgfsetdash{}{0pt}%
\pgfpathmoveto{\pgfqpoint{0.736295in}{0.663635in}}%
\pgfpathlineto{\pgfqpoint{5.420000in}{0.663635in}}%
\pgfpathlineto{\pgfqpoint{5.420000in}{1.660000in}}%
\pgfpathlineto{\pgfqpoint{0.736295in}{1.660000in}}%
\pgfpathlineto{\pgfqpoint{0.736295in}{0.663635in}}%
\pgfpathclose%
\pgfusepath{fill}%
\end{pgfscope}%
\begin{pgfscope}%
\pgfpathrectangle{\pgfqpoint{0.736295in}{0.663635in}}{\pgfqpoint{4.683705in}{0.996365in}}%
\pgfusepath{clip}%
\pgfsetroundcap%
\pgfsetroundjoin%
\pgfsetlinewidth{1.003750pt}%
\definecolor{currentstroke}{rgb}{1.000000,1.000000,1.000000}%
\pgfsetstrokecolor{currentstroke}%
\pgfsetdash{}{0pt}%
\pgfpathmoveto{\pgfqpoint{0.949190in}{0.663635in}}%
\pgfpathlineto{\pgfqpoint{0.949190in}{1.660000in}}%
\pgfusepath{stroke}%
\end{pgfscope}%
\begin{pgfscope}%
\definecolor{textcolor}{rgb}{0.150000,0.150000,0.150000}%
\pgfsetstrokecolor{textcolor}%
\pgfsetfillcolor{textcolor}%
\pgftext[x=0.949190in,y=0.531691in,,top]{\color{textcolor}{\sffamily\fontsize{11.000000}{13.200000}\selectfont\catcode`\^=\active\def^{\ifmmode\sp\else\^{}\fi}\catcode`\%=\active\def%{\%}0}}%
\end{pgfscope}%
\begin{pgfscope}%
\pgfpathrectangle{\pgfqpoint{0.736295in}{0.663635in}}{\pgfqpoint{4.683705in}{0.996365in}}%
\pgfusepath{clip}%
\pgfsetroundcap%
\pgfsetroundjoin%
\pgfsetlinewidth{1.003750pt}%
\definecolor{currentstroke}{rgb}{1.000000,1.000000,1.000000}%
\pgfsetstrokecolor{currentstroke}%
\pgfsetdash{}{0pt}%
\pgfpathmoveto{\pgfqpoint{1.658843in}{0.663635in}}%
\pgfpathlineto{\pgfqpoint{1.658843in}{1.660000in}}%
\pgfusepath{stroke}%
\end{pgfscope}%
\begin{pgfscope}%
\definecolor{textcolor}{rgb}{0.150000,0.150000,0.150000}%
\pgfsetstrokecolor{textcolor}%
\pgfsetfillcolor{textcolor}%
\pgftext[x=1.658843in,y=0.531691in,,top]{\color{textcolor}{\sffamily\fontsize{11.000000}{13.200000}\selectfont\catcode`\^=\active\def^{\ifmmode\sp\else\^{}\fi}\catcode`\%=\active\def%{\%}200}}%
\end{pgfscope}%
\begin{pgfscope}%
\pgfpathrectangle{\pgfqpoint{0.736295in}{0.663635in}}{\pgfqpoint{4.683705in}{0.996365in}}%
\pgfusepath{clip}%
\pgfsetroundcap%
\pgfsetroundjoin%
\pgfsetlinewidth{1.003750pt}%
\definecolor{currentstroke}{rgb}{1.000000,1.000000,1.000000}%
\pgfsetstrokecolor{currentstroke}%
\pgfsetdash{}{0pt}%
\pgfpathmoveto{\pgfqpoint{2.368495in}{0.663635in}}%
\pgfpathlineto{\pgfqpoint{2.368495in}{1.660000in}}%
\pgfusepath{stroke}%
\end{pgfscope}%
\begin{pgfscope}%
\definecolor{textcolor}{rgb}{0.150000,0.150000,0.150000}%
\pgfsetstrokecolor{textcolor}%
\pgfsetfillcolor{textcolor}%
\pgftext[x=2.368495in,y=0.531691in,,top]{\color{textcolor}{\sffamily\fontsize{11.000000}{13.200000}\selectfont\catcode`\^=\active\def^{\ifmmode\sp\else\^{}\fi}\catcode`\%=\active\def%{\%}400}}%
\end{pgfscope}%
\begin{pgfscope}%
\pgfpathrectangle{\pgfqpoint{0.736295in}{0.663635in}}{\pgfqpoint{4.683705in}{0.996365in}}%
\pgfusepath{clip}%
\pgfsetroundcap%
\pgfsetroundjoin%
\pgfsetlinewidth{1.003750pt}%
\definecolor{currentstroke}{rgb}{1.000000,1.000000,1.000000}%
\pgfsetstrokecolor{currentstroke}%
\pgfsetdash{}{0pt}%
\pgfpathmoveto{\pgfqpoint{3.078147in}{0.663635in}}%
\pgfpathlineto{\pgfqpoint{3.078147in}{1.660000in}}%
\pgfusepath{stroke}%
\end{pgfscope}%
\begin{pgfscope}%
\definecolor{textcolor}{rgb}{0.150000,0.150000,0.150000}%
\pgfsetstrokecolor{textcolor}%
\pgfsetfillcolor{textcolor}%
\pgftext[x=3.078147in,y=0.531691in,,top]{\color{textcolor}{\sffamily\fontsize{11.000000}{13.200000}\selectfont\catcode`\^=\active\def^{\ifmmode\sp\else\^{}\fi}\catcode`\%=\active\def%{\%}600}}%
\end{pgfscope}%
\begin{pgfscope}%
\pgfpathrectangle{\pgfqpoint{0.736295in}{0.663635in}}{\pgfqpoint{4.683705in}{0.996365in}}%
\pgfusepath{clip}%
\pgfsetroundcap%
\pgfsetroundjoin%
\pgfsetlinewidth{1.003750pt}%
\definecolor{currentstroke}{rgb}{1.000000,1.000000,1.000000}%
\pgfsetstrokecolor{currentstroke}%
\pgfsetdash{}{0pt}%
\pgfpathmoveto{\pgfqpoint{3.787800in}{0.663635in}}%
\pgfpathlineto{\pgfqpoint{3.787800in}{1.660000in}}%
\pgfusepath{stroke}%
\end{pgfscope}%
\begin{pgfscope}%
\definecolor{textcolor}{rgb}{0.150000,0.150000,0.150000}%
\pgfsetstrokecolor{textcolor}%
\pgfsetfillcolor{textcolor}%
\pgftext[x=3.787800in,y=0.531691in,,top]{\color{textcolor}{\sffamily\fontsize{11.000000}{13.200000}\selectfont\catcode`\^=\active\def^{\ifmmode\sp\else\^{}\fi}\catcode`\%=\active\def%{\%}800}}%
\end{pgfscope}%
\begin{pgfscope}%
\pgfpathrectangle{\pgfqpoint{0.736295in}{0.663635in}}{\pgfqpoint{4.683705in}{0.996365in}}%
\pgfusepath{clip}%
\pgfsetroundcap%
\pgfsetroundjoin%
\pgfsetlinewidth{1.003750pt}%
\definecolor{currentstroke}{rgb}{1.000000,1.000000,1.000000}%
\pgfsetstrokecolor{currentstroke}%
\pgfsetdash{}{0pt}%
\pgfpathmoveto{\pgfqpoint{4.497452in}{0.663635in}}%
\pgfpathlineto{\pgfqpoint{4.497452in}{1.660000in}}%
\pgfusepath{stroke}%
\end{pgfscope}%
\begin{pgfscope}%
\definecolor{textcolor}{rgb}{0.150000,0.150000,0.150000}%
\pgfsetstrokecolor{textcolor}%
\pgfsetfillcolor{textcolor}%
\pgftext[x=4.497452in,y=0.531691in,,top]{\color{textcolor}{\sffamily\fontsize{11.000000}{13.200000}\selectfont\catcode`\^=\active\def^{\ifmmode\sp\else\^{}\fi}\catcode`\%=\active\def%{\%}1000}}%
\end{pgfscope}%
\begin{pgfscope}%
\pgfpathrectangle{\pgfqpoint{0.736295in}{0.663635in}}{\pgfqpoint{4.683705in}{0.996365in}}%
\pgfusepath{clip}%
\pgfsetroundcap%
\pgfsetroundjoin%
\pgfsetlinewidth{1.003750pt}%
\definecolor{currentstroke}{rgb}{1.000000,1.000000,1.000000}%
\pgfsetstrokecolor{currentstroke}%
\pgfsetdash{}{0pt}%
\pgfpathmoveto{\pgfqpoint{5.207104in}{0.663635in}}%
\pgfpathlineto{\pgfqpoint{5.207104in}{1.660000in}}%
\pgfusepath{stroke}%
\end{pgfscope}%
\begin{pgfscope}%
\definecolor{textcolor}{rgb}{0.150000,0.150000,0.150000}%
\pgfsetstrokecolor{textcolor}%
\pgfsetfillcolor{textcolor}%
\pgftext[x=5.207104in,y=0.531691in,,top]{\color{textcolor}{\sffamily\fontsize{11.000000}{13.200000}\selectfont\catcode`\^=\active\def^{\ifmmode\sp\else\^{}\fi}\catcode`\%=\active\def%{\%}1200}}%
\end{pgfscope}%
\begin{pgfscope}%
\definecolor{textcolor}{rgb}{0.150000,0.150000,0.150000}%
\pgfsetstrokecolor{textcolor}%
\pgfsetfillcolor{textcolor}%
\pgftext[x=3.078147in,y=0.336413in,,top]{\color{textcolor}{\sffamily\fontsize{12.000000}{14.400000}\selectfont\catcode`\^=\active\def^{\ifmmode\sp\else\^{}\fi}\catcode`\%=\active\def%{\%}Time (s)}}%
\end{pgfscope}%
\begin{pgfscope}%
\pgfpathrectangle{\pgfqpoint{0.736295in}{0.663635in}}{\pgfqpoint{4.683705in}{0.996365in}}%
\pgfusepath{clip}%
\pgfsetroundcap%
\pgfsetroundjoin%
\pgfsetlinewidth{1.003750pt}%
\definecolor{currentstroke}{rgb}{1.000000,1.000000,1.000000}%
\pgfsetstrokecolor{currentstroke}%
\pgfsetdash{}{0pt}%
\pgfpathmoveto{\pgfqpoint{0.736295in}{0.746666in}}%
\pgfpathlineto{\pgfqpoint{5.420000in}{0.746666in}}%
\pgfusepath{stroke}%
\end{pgfscope}%
\begin{pgfscope}%
\definecolor{textcolor}{rgb}{0.150000,0.150000,0.150000}%
\pgfsetstrokecolor{textcolor}%
\pgfsetfillcolor{textcolor}%
\pgftext[x=0.391968in, y=0.691985in, left, base]{\color{textcolor}{\sffamily\fontsize{11.000000}{13.200000}\selectfont\catcode`\^=\active\def^{\ifmmode\sp\else\^{}\fi}\catcode`\%=\active\def%{\%}0.0}}%
\end{pgfscope}%
\begin{pgfscope}%
\pgfpathrectangle{\pgfqpoint{0.736295in}{0.663635in}}{\pgfqpoint{4.683705in}{0.996365in}}%
\pgfusepath{clip}%
\pgfsetroundcap%
\pgfsetroundjoin%
\pgfsetlinewidth{1.003750pt}%
\definecolor{currentstroke}{rgb}{1.000000,1.000000,1.000000}%
\pgfsetstrokecolor{currentstroke}%
\pgfsetdash{}{0pt}%
\pgfpathmoveto{\pgfqpoint{0.736295in}{1.161818in}}%
\pgfpathlineto{\pgfqpoint{5.420000in}{1.161818in}}%
\pgfusepath{stroke}%
\end{pgfscope}%
\begin{pgfscope}%
\definecolor{textcolor}{rgb}{0.150000,0.150000,0.150000}%
\pgfsetstrokecolor{textcolor}%
\pgfsetfillcolor{textcolor}%
\pgftext[x=0.391968in, y=1.107137in, left, base]{\color{textcolor}{\sffamily\fontsize{11.000000}{13.200000}\selectfont\catcode`\^=\active\def^{\ifmmode\sp\else\^{}\fi}\catcode`\%=\active\def%{\%}0.5}}%
\end{pgfscope}%
\begin{pgfscope}%
\pgfpathrectangle{\pgfqpoint{0.736295in}{0.663635in}}{\pgfqpoint{4.683705in}{0.996365in}}%
\pgfusepath{clip}%
\pgfsetroundcap%
\pgfsetroundjoin%
\pgfsetlinewidth{1.003750pt}%
\definecolor{currentstroke}{rgb}{1.000000,1.000000,1.000000}%
\pgfsetstrokecolor{currentstroke}%
\pgfsetdash{}{0pt}%
\pgfpathmoveto{\pgfqpoint{0.736295in}{1.576970in}}%
\pgfpathlineto{\pgfqpoint{5.420000in}{1.576970in}}%
\pgfusepath{stroke}%
\end{pgfscope}%
\begin{pgfscope}%
\definecolor{textcolor}{rgb}{0.150000,0.150000,0.150000}%
\pgfsetstrokecolor{textcolor}%
\pgfsetfillcolor{textcolor}%
\pgftext[x=0.391968in, y=1.522289in, left, base]{\color{textcolor}{\sffamily\fontsize{11.000000}{13.200000}\selectfont\catcode`\^=\active\def^{\ifmmode\sp\else\^{}\fi}\catcode`\%=\active\def%{\%}1.0}}%
\end{pgfscope}%
\begin{pgfscope}%
\definecolor{textcolor}{rgb}{0.150000,0.150000,0.150000}%
\pgfsetstrokecolor{textcolor}%
\pgfsetfillcolor{textcolor}%
\pgftext[x=0.336413in,y=1.161818in,,bottom,rotate=90.000000]{\color{textcolor}{\sffamily\fontsize{12.000000}{14.400000}\selectfont\catcode`\^=\active\def^{\ifmmode\sp\else\^{}\fi}\catcode`\%=\active\def%{\%}Memory Utilization (%)}}%
\end{pgfscope}%
\begin{pgfscope}%
\pgfpathrectangle{\pgfqpoint{0.736295in}{0.663635in}}{\pgfqpoint{4.683705in}{0.996365in}}%
\pgfusepath{clip}%
\pgfsetbuttcap%
\pgfsetmiterjoin%
\definecolor{currentfill}{rgb}{1.000000,0.000000,0.000000}%
\pgfsetfillcolor{currentfill}%
\pgfsetfillopacity{0.300000}%
\pgfsetlinewidth{1.003750pt}%
\definecolor{currentstroke}{rgb}{1.000000,0.000000,0.000000}%
\pgfsetstrokecolor{currentstroke}%
\pgfsetstrokeopacity{0.300000}%
\pgfsetdash{}{0pt}%
\pgfpathmoveto{\pgfqpoint{1.367885in}{0.663635in}}%
\pgfpathlineto{\pgfqpoint{1.367885in}{1.660000in}}%
\pgfpathlineto{\pgfqpoint{2.411074in}{1.660000in}}%
\pgfpathlineto{\pgfqpoint{2.411074in}{0.663635in}}%
\pgfpathlineto{\pgfqpoint{1.367885in}{0.663635in}}%
\pgfpathclose%
\pgfusepath{stroke,fill}%
\end{pgfscope}%
\begin{pgfscope}%
\pgfpathrectangle{\pgfqpoint{0.736295in}{0.663635in}}{\pgfqpoint{4.683705in}{0.996365in}}%
\pgfusepath{clip}%
\pgfsetroundcap%
\pgfsetroundjoin%
\pgfsetlinewidth{1.505625pt}%
\definecolor{currentstroke}{rgb}{0.298039,0.447059,0.690196}%
\pgfsetstrokecolor{currentstroke}%
\pgfsetdash{}{0pt}%
\pgfpathmoveto{\pgfqpoint{0.949190in}{1.175322in}}%
\pgfpathlineto{\pgfqpoint{1.179827in}{1.175111in}}%
\pgfpathlineto{\pgfqpoint{1.197569in}{1.177375in}}%
\pgfpathlineto{\pgfqpoint{1.374982in}{1.178370in}}%
\pgfpathlineto{\pgfqpoint{1.392723in}{1.418450in}}%
\pgfpathlineto{\pgfqpoint{1.445947in}{1.419151in}}%
\pgfpathlineto{\pgfqpoint{1.463688in}{1.576970in}}%
\pgfpathlineto{\pgfqpoint{2.386236in}{1.576970in}}%
\pgfpathlineto{\pgfqpoint{2.403978in}{1.402559in}}%
\pgfpathlineto{\pgfqpoint{2.457201in}{1.402035in}}%
\pgfpathlineto{\pgfqpoint{2.474943in}{1.400621in}}%
\pgfpathlineto{\pgfqpoint{2.510425in}{1.400621in}}%
\pgfpathlineto{\pgfqpoint{2.528167in}{1.404946in}}%
\pgfpathlineto{\pgfqpoint{2.563649in}{1.404946in}}%
\pgfpathlineto{\pgfqpoint{2.581391in}{1.401377in}}%
\pgfpathlineto{\pgfqpoint{2.953958in}{1.400924in}}%
\pgfpathlineto{\pgfqpoint{2.971699in}{1.402976in}}%
\pgfpathlineto{\pgfqpoint{3.024923in}{1.402976in}}%
\pgfpathlineto{\pgfqpoint{3.042665in}{1.400604in}}%
\pgfpathlineto{\pgfqpoint{3.078147in}{1.400604in}}%
\pgfpathlineto{\pgfqpoint{3.095889in}{1.404172in}}%
\pgfpathlineto{\pgfqpoint{3.131371in}{1.404172in}}%
\pgfpathlineto{\pgfqpoint{3.149113in}{1.402557in}}%
\pgfpathlineto{\pgfqpoint{3.184595in}{1.402557in}}%
\pgfpathlineto{\pgfqpoint{3.202336in}{1.400815in}}%
\pgfpathlineto{\pgfqpoint{3.397491in}{1.401466in}}%
\pgfpathlineto{\pgfqpoint{3.415232in}{1.402735in}}%
\pgfpathlineto{\pgfqpoint{3.610387in}{1.402549in}}%
\pgfpathlineto{\pgfqpoint{3.628128in}{1.405441in}}%
\pgfpathlineto{\pgfqpoint{3.787800in}{1.405651in}}%
\pgfpathlineto{\pgfqpoint{3.805541in}{1.404133in}}%
\pgfpathlineto{\pgfqpoint{4.089402in}{1.404731in}}%
\pgfpathlineto{\pgfqpoint{4.107143in}{1.401121in}}%
\pgfpathlineto{\pgfqpoint{4.213591in}{1.401202in}}%
\pgfpathlineto{\pgfqpoint{4.231332in}{1.402726in}}%
\pgfpathlineto{\pgfqpoint{4.266815in}{1.402726in}}%
\pgfpathlineto{\pgfqpoint{4.284556in}{1.404362in}}%
\pgfpathlineto{\pgfqpoint{4.568417in}{1.405176in}}%
\pgfpathlineto{\pgfqpoint{4.586159in}{1.401940in}}%
\pgfpathlineto{\pgfqpoint{4.692606in}{1.401941in}}%
\pgfpathlineto{\pgfqpoint{4.710348in}{1.407260in}}%
\pgfpathlineto{\pgfqpoint{4.745830in}{1.407261in}}%
\pgfpathlineto{\pgfqpoint{4.763572in}{1.404999in}}%
\pgfpathlineto{\pgfqpoint{4.816796in}{1.404999in}}%
\pgfpathlineto{\pgfqpoint{4.834537in}{1.402270in}}%
\pgfpathlineto{\pgfqpoint{4.976467in}{1.402540in}}%
\pgfpathlineto{\pgfqpoint{4.994209in}{1.404280in}}%
\pgfpathlineto{\pgfqpoint{5.207104in}{1.403701in}}%
\pgfpathlineto{\pgfqpoint{5.207104in}{1.403701in}}%
\pgfusepath{stroke}%
\end{pgfscope}%
\begin{pgfscope}%
\pgfpathrectangle{\pgfqpoint{0.736295in}{0.663635in}}{\pgfqpoint{4.683705in}{0.996365in}}%
\pgfusepath{clip}%
\pgfsetbuttcap%
\pgfsetroundjoin%
\pgfsetlinewidth{1.505625pt}%
\definecolor{currentstroke}{rgb}{0.172549,0.627451,0.172549}%
\pgfsetstrokecolor{currentstroke}%
\pgfsetdash{{5.550000pt}{2.400000pt}}{0.000000pt}%
\pgfpathmoveto{\pgfqpoint{0.736295in}{1.327878in}}%
\pgfpathlineto{\pgfqpoint{5.420000in}{1.327878in}}%
\pgfusepath{stroke}%
\end{pgfscope}%
\begin{pgfscope}%
\pgfpathrectangle{\pgfqpoint{0.736295in}{0.663635in}}{\pgfqpoint{4.683705in}{0.996365in}}%
\pgfusepath{clip}%
\pgfsetbuttcap%
\pgfsetroundjoin%
\pgfsetlinewidth{1.505625pt}%
\definecolor{currentstroke}{rgb}{1.000000,0.647059,0.000000}%
\pgfsetstrokecolor{currentstroke}%
\pgfsetdash{{5.550000pt}{2.400000pt}}{0.000000pt}%
\pgfpathmoveto{\pgfqpoint{1.026542in}{0.663635in}}%
\pgfpathlineto{\pgfqpoint{1.026542in}{1.660000in}}%
\pgfusepath{stroke}%
\end{pgfscope}%
\begin{pgfscope}%
\pgfsetrectcap%
\pgfsetmiterjoin%
\pgfsetlinewidth{1.254687pt}%
\definecolor{currentstroke}{rgb}{1.000000,1.000000,1.000000}%
\pgfsetstrokecolor{currentstroke}%
\pgfsetdash{}{0pt}%
\pgfpathmoveto{\pgfqpoint{0.736295in}{0.663635in}}%
\pgfpathlineto{\pgfqpoint{0.736295in}{1.660000in}}%
\pgfusepath{stroke}%
\end{pgfscope}%
\begin{pgfscope}%
\pgfsetrectcap%
\pgfsetmiterjoin%
\pgfsetlinewidth{1.254687pt}%
\definecolor{currentstroke}{rgb}{1.000000,1.000000,1.000000}%
\pgfsetstrokecolor{currentstroke}%
\pgfsetdash{}{0pt}%
\pgfpathmoveto{\pgfqpoint{5.420000in}{0.663635in}}%
\pgfpathlineto{\pgfqpoint{5.420000in}{1.660000in}}%
\pgfusepath{stroke}%
\end{pgfscope}%
\begin{pgfscope}%
\pgfsetrectcap%
\pgfsetmiterjoin%
\pgfsetlinewidth{1.254687pt}%
\definecolor{currentstroke}{rgb}{1.000000,1.000000,1.000000}%
\pgfsetstrokecolor{currentstroke}%
\pgfsetdash{}{0pt}%
\pgfpathmoveto{\pgfqpoint{0.736295in}{0.663635in}}%
\pgfpathlineto{\pgfqpoint{5.420000in}{0.663635in}}%
\pgfusepath{stroke}%
\end{pgfscope}%
\begin{pgfscope}%
\pgfsetrectcap%
\pgfsetmiterjoin%
\pgfsetlinewidth{1.254687pt}%
\definecolor{currentstroke}{rgb}{1.000000,1.000000,1.000000}%
\pgfsetstrokecolor{currentstroke}%
\pgfsetdash{}{0pt}%
\pgfpathmoveto{\pgfqpoint{0.736295in}{1.660000in}}%
\pgfpathlineto{\pgfqpoint{5.420000in}{1.660000in}}%
\pgfusepath{stroke}%
\end{pgfscope}%
\begin{pgfscope}%
\pgfsetbuttcap%
\pgfsetmiterjoin%
\definecolor{currentfill}{rgb}{0.917647,0.917647,0.949020}%
\pgfsetfillcolor{currentfill}%
\pgfsetfillopacity{0.800000}%
\pgfsetlinewidth{1.003750pt}%
\definecolor{currentstroke}{rgb}{0.800000,0.800000,0.800000}%
\pgfsetstrokecolor{currentstroke}%
\pgfsetstrokeopacity{0.800000}%
\pgfsetdash{}{0pt}%
\pgfpathmoveto{\pgfqpoint{3.276163in}{0.719191in}}%
\pgfpathlineto{\pgfqpoint{5.342222in}{0.719191in}}%
\pgfpathquadraticcurveto{\pgfqpoint{5.364444in}{0.719191in}}{\pgfqpoint{5.364444in}{0.741413in}}%
\pgfpathlineto{\pgfqpoint{5.364444in}{1.205779in}}%
\pgfpathquadraticcurveto{\pgfqpoint{5.364444in}{1.228001in}}{\pgfqpoint{5.342222in}{1.228001in}}%
\pgfpathlineto{\pgfqpoint{3.276163in}{1.228001in}}%
\pgfpathquadraticcurveto{\pgfqpoint{3.253941in}{1.228001in}}{\pgfqpoint{3.253941in}{1.205779in}}%
\pgfpathlineto{\pgfqpoint{3.253941in}{0.741413in}}%
\pgfpathquadraticcurveto{\pgfqpoint{3.253941in}{0.719191in}}{\pgfqpoint{3.276163in}{0.719191in}}%
\pgfpathlineto{\pgfqpoint{3.276163in}{0.719191in}}%
\pgfpathclose%
\pgfusepath{stroke,fill}%
\end{pgfscope}%
\begin{pgfscope}%
\pgfsetbuttcap%
\pgfsetmiterjoin%
\definecolor{currentfill}{rgb}{1.000000,0.000000,0.000000}%
\pgfsetfillcolor{currentfill}%
\pgfsetfillopacity{0.300000}%
\pgfsetlinewidth{1.003750pt}%
\definecolor{currentstroke}{rgb}{1.000000,0.000000,0.000000}%
\pgfsetstrokecolor{currentstroke}%
\pgfsetstrokeopacity{0.300000}%
\pgfsetdash{}{0pt}%
\pgfpathmoveto{\pgfqpoint{3.298385in}{1.104021in}}%
\pgfpathlineto{\pgfqpoint{3.520608in}{1.104021in}}%
\pgfpathlineto{\pgfqpoint{3.520608in}{1.181799in}}%
\pgfpathlineto{\pgfqpoint{3.298385in}{1.181799in}}%
\pgfpathlineto{\pgfqpoint{3.298385in}{1.104021in}}%
\pgfpathclose%
\pgfusepath{stroke,fill}%
\end{pgfscope}%
\begin{pgfscope}%
\definecolor{textcolor}{rgb}{0.150000,0.150000,0.150000}%
\pgfsetstrokecolor{textcolor}%
\pgfsetfillcolor{textcolor}%
\pgftext[x=3.609497in,y=1.104021in,left,base]{\color{textcolor}{\sffamily\fontsize{8.000000}{9.600000}\selectfont\catcode`\^=\active\def^{\ifmmode\sp\else\^{}\fi}\catcode`\%=\active\def%{\%}k8ssandra reconsiliation}}%
\end{pgfscope}%
\begin{pgfscope}%
\pgfsetbuttcap%
\pgfsetroundjoin%
\pgfsetlinewidth{1.505625pt}%
\definecolor{currentstroke}{rgb}{0.172549,0.627451,0.172549}%
\pgfsetstrokecolor{currentstroke}%
\pgfsetdash{{5.550000pt}{2.400000pt}}{0.000000pt}%
\pgfpathmoveto{\pgfqpoint{3.298385in}{0.985738in}}%
\pgfpathlineto{\pgfqpoint{3.409497in}{0.985738in}}%
\pgfpathlineto{\pgfqpoint{3.520608in}{0.985738in}}%
\pgfusepath{stroke}%
\end{pgfscope}%
\begin{pgfscope}%
\definecolor{textcolor}{rgb}{0.150000,0.150000,0.150000}%
\pgfsetstrokecolor{textcolor}%
\pgfsetfillcolor{textcolor}%
\pgftext[x=3.609497in,y=0.946849in,left,base]{\color{textcolor}{\sffamily\fontsize{8.000000}{9.600000}\selectfont\catcode`\^=\active\def^{\ifmmode\sp\else\^{}\fi}\catcode`\%=\active\def%{\%}target memory utilization}}%
\end{pgfscope}%
\begin{pgfscope}%
\pgfsetbuttcap%
\pgfsetroundjoin%
\pgfsetlinewidth{1.505625pt}%
\definecolor{currentstroke}{rgb}{1.000000,0.647059,0.000000}%
\pgfsetstrokecolor{currentstroke}%
\pgfsetdash{{5.550000pt}{2.400000pt}}{0.000000pt}%
\pgfpathmoveto{\pgfqpoint{3.298385in}{0.825907in}}%
\pgfpathlineto{\pgfqpoint{3.409497in}{0.825907in}}%
\pgfpathlineto{\pgfqpoint{3.520608in}{0.825907in}}%
\pgfusepath{stroke}%
\end{pgfscope}%
\begin{pgfscope}%
\definecolor{textcolor}{rgb}{0.150000,0.150000,0.150000}%
\pgfsetstrokecolor{textcolor}%
\pgfsetfillcolor{textcolor}%
\pgftext[x=3.609497in,y=0.787018in,left,base]{\color{textcolor}{\sffamily\fontsize{8.000000}{9.600000}\selectfont\catcode`\^=\active\def^{\ifmmode\sp\else\^{}\fi}\catcode`\%=\active\def%{\%}start of elasticity strategy controller}}%
\end{pgfscope}%
\end{pgfpicture}%
\makeatother%
\endgroup%

    \caption{Utilization of CPU and memory during an vertical scaling action}
    \label{fig:utilization-vertical}
\end{figure}

This elasticity strategy mirrors real-life scenarios. The advantage lies in being able to scale down when demand and therefore CPU and memory utilization is low, thus potentially reducing cost. This obviously only applies when not using dedicated resources. Comparing the results to the prior discussed baseline scenario, it is clear that vertical elasticity offers a benefit, as it reduces the CPU claims by 200 milliCPU and memory claims by more than 2 GB.

\subsection{Horizontal Elasticity Strategy}
\label{sec:evaluation-horizontal-elasticity}

The horizontal elasticity strategy controller scales the target K8ssandra cluster horizontally, thus adding nodes as demand increases. Demand is measured as write throughput by the metrics controller as described in \Cref{sec:metrics-average-write-utilization}.

As in the example stress tests discussed in \Cref{sec:stress-testing}, \texttt{cassandra-stress} was used to generate load on the target K8ssandra cluster. During this load generation process, the horizontal elasticity controller was running. The target write load per node was defined in the SLO mapping as 5000 writes/sec. Depicted in \Cref{fig:horizontal-elasticity} is the average write load per node metric and the corresponding node count during the testing process. It can be seen that the node count does not increase immediatly when the scaling action takes place. That is because when the K8ssandra CRD is updated by the elasticity strategy controller, first the \texttt{k8ssandra-operator} has to recognize the made changes and adjust the configuration accordingly. When the second K8ssandra node is successfully scheduled it still needs time to start and finally register in the cluster. The final action is the Cassandra reconciliation process.

\begin{figure}[ht]
    \centering
    %% Creator: Matplotlib, PGF backend
%%
%% To include the figure in your LaTeX document, write
%%   \input{<filename>.pgf}
%%
%% Make sure the required packages are loaded in your preamble
%%   \usepackage{pgf}
%%
%% Also ensure that all the required font packages are loaded; for instance,
%% the lmodern package is sometimes necessary when using math font.
%%   \usepackage{lmodern}
%%
%% Figures using additional raster images can only be included by \input if
%% they are in the same directory as the main LaTeX file. For loading figures
%% from other directories you can use the `import` package
%%   \usepackage{import}
%%
%% and then include the figures with
%%   \import{<path to file>}{<filename>.pgf}
%%
%% Matplotlib used the following preamble
%%   \def\mathdefault#1{#1}
%%   \everymath=\expandafter{\the\everymath\displaystyle}
%%   
%%   \usepackage{fontspec}
%%   \setmainfont{DejaVuSerif.ttf}[Path=\detokenize{/Users/nkratky/private/polaris-elasticity-strategies/test/scripts/.venv/lib/python3.11/site-packages/matplotlib/mpl-data/fonts/ttf/}]
%%   \setsansfont{Arial.ttf}[Path=\detokenize{/System/Library/Fonts/Supplemental/}]
%%   \setmonofont{DejaVuSansMono.ttf}[Path=\detokenize{/Users/nkratky/private/polaris-elasticity-strategies/test/scripts/.venv/lib/python3.11/site-packages/matplotlib/mpl-data/fonts/ttf/}]
%%   \makeatletter\@ifpackageloaded{underscore}{}{\usepackage[strings]{underscore}}\makeatother
%%
\begingroup%
\makeatletter%
\begin{pgfpicture}%
\pgfpathrectangle{\pgfpointorigin}{\pgfqpoint{5.600000in}{3.000000in}}%
\pgfusepath{use as bounding box, clip}%
\begin{pgfscope}%
\pgfsetbuttcap%
\pgfsetmiterjoin%
\definecolor{currentfill}{rgb}{1.000000,1.000000,1.000000}%
\pgfsetfillcolor{currentfill}%
\pgfsetlinewidth{0.000000pt}%
\definecolor{currentstroke}{rgb}{1.000000,1.000000,1.000000}%
\pgfsetstrokecolor{currentstroke}%
\pgfsetdash{}{0pt}%
\pgfpathmoveto{\pgfqpoint{0.000000in}{0.000000in}}%
\pgfpathlineto{\pgfqpoint{5.600000in}{0.000000in}}%
\pgfpathlineto{\pgfqpoint{5.600000in}{3.000000in}}%
\pgfpathlineto{\pgfqpoint{0.000000in}{3.000000in}}%
\pgfpathlineto{\pgfqpoint{0.000000in}{0.000000in}}%
\pgfpathclose%
\pgfusepath{fill}%
\end{pgfscope}%
\begin{pgfscope}%
\pgfsetbuttcap%
\pgfsetmiterjoin%
\definecolor{currentfill}{rgb}{0.917647,0.917647,0.949020}%
\pgfsetfillcolor{currentfill}%
\pgfsetlinewidth{0.000000pt}%
\definecolor{currentstroke}{rgb}{0.000000,0.000000,0.000000}%
\pgfsetstrokecolor{currentstroke}%
\pgfsetstrokeopacity{0.000000}%
\pgfsetdash{}{0pt}%
\pgfpathmoveto{\pgfqpoint{0.946717in}{0.663635in}}%
\pgfpathlineto{\pgfqpoint{2.880912in}{0.663635in}}%
\pgfpathlineto{\pgfqpoint{2.880912in}{2.820000in}}%
\pgfpathlineto{\pgfqpoint{0.946717in}{2.820000in}}%
\pgfpathlineto{\pgfqpoint{0.946717in}{0.663635in}}%
\pgfpathclose%
\pgfusepath{fill}%
\end{pgfscope}%
\begin{pgfscope}%
\pgfpathrectangle{\pgfqpoint{0.946717in}{0.663635in}}{\pgfqpoint{1.934195in}{2.156365in}}%
\pgfusepath{clip}%
\pgfsetroundcap%
\pgfsetroundjoin%
\pgfsetlinewidth{1.003750pt}%
\definecolor{currentstroke}{rgb}{1.000000,1.000000,1.000000}%
\pgfsetstrokecolor{currentstroke}%
\pgfsetdash{}{0pt}%
\pgfpathmoveto{\pgfqpoint{1.034635in}{0.663635in}}%
\pgfpathlineto{\pgfqpoint{1.034635in}{2.820000in}}%
\pgfusepath{stroke}%
\end{pgfscope}%
\begin{pgfscope}%
\definecolor{textcolor}{rgb}{0.150000,0.150000,0.150000}%
\pgfsetstrokecolor{textcolor}%
\pgfsetfillcolor{textcolor}%
\pgftext[x=1.034635in,y=0.531691in,,top]{\color{textcolor}{\sffamily\fontsize{11.000000}{13.200000}\selectfont\catcode`\^=\active\def^{\ifmmode\sp\else\^{}\fi}\catcode`\%=\active\def%{\%}0}}%
\end{pgfscope}%
\begin{pgfscope}%
\pgfpathrectangle{\pgfqpoint{0.946717in}{0.663635in}}{\pgfqpoint{1.934195in}{2.156365in}}%
\pgfusepath{clip}%
\pgfsetroundcap%
\pgfsetroundjoin%
\pgfsetlinewidth{1.003750pt}%
\definecolor{currentstroke}{rgb}{1.000000,1.000000,1.000000}%
\pgfsetstrokecolor{currentstroke}%
\pgfsetdash{}{0pt}%
\pgfpathmoveto{\pgfqpoint{1.674038in}{0.663635in}}%
\pgfpathlineto{\pgfqpoint{1.674038in}{2.820000in}}%
\pgfusepath{stroke}%
\end{pgfscope}%
\begin{pgfscope}%
\definecolor{textcolor}{rgb}{0.150000,0.150000,0.150000}%
\pgfsetstrokecolor{textcolor}%
\pgfsetfillcolor{textcolor}%
\pgftext[x=1.674038in,y=0.531691in,,top]{\color{textcolor}{\sffamily\fontsize{11.000000}{13.200000}\selectfont\catcode`\^=\active\def^{\ifmmode\sp\else\^{}\fi}\catcode`\%=\active\def%{\%}200}}%
\end{pgfscope}%
\begin{pgfscope}%
\pgfpathrectangle{\pgfqpoint{0.946717in}{0.663635in}}{\pgfqpoint{1.934195in}{2.156365in}}%
\pgfusepath{clip}%
\pgfsetroundcap%
\pgfsetroundjoin%
\pgfsetlinewidth{1.003750pt}%
\definecolor{currentstroke}{rgb}{1.000000,1.000000,1.000000}%
\pgfsetstrokecolor{currentstroke}%
\pgfsetdash{}{0pt}%
\pgfpathmoveto{\pgfqpoint{2.313441in}{0.663635in}}%
\pgfpathlineto{\pgfqpoint{2.313441in}{2.820000in}}%
\pgfusepath{stroke}%
\end{pgfscope}%
\begin{pgfscope}%
\definecolor{textcolor}{rgb}{0.150000,0.150000,0.150000}%
\pgfsetstrokecolor{textcolor}%
\pgfsetfillcolor{textcolor}%
\pgftext[x=2.313441in,y=0.531691in,,top]{\color{textcolor}{\sffamily\fontsize{11.000000}{13.200000}\selectfont\catcode`\^=\active\def^{\ifmmode\sp\else\^{}\fi}\catcode`\%=\active\def%{\%}400}}%
\end{pgfscope}%
\begin{pgfscope}%
\definecolor{textcolor}{rgb}{0.150000,0.150000,0.150000}%
\pgfsetstrokecolor{textcolor}%
\pgfsetfillcolor{textcolor}%
\pgftext[x=1.913814in,y=0.336413in,,top]{\color{textcolor}{\sffamily\fontsize{12.000000}{14.400000}\selectfont\catcode`\^=\active\def^{\ifmmode\sp\else\^{}\fi}\catcode`\%=\active\def%{\%}Time (s)}}%
\end{pgfscope}%
\begin{pgfscope}%
\pgfpathrectangle{\pgfqpoint{0.946717in}{0.663635in}}{\pgfqpoint{1.934195in}{2.156365in}}%
\pgfusepath{clip}%
\pgfsetroundcap%
\pgfsetroundjoin%
\pgfsetlinewidth{1.003750pt}%
\definecolor{currentstroke}{rgb}{1.000000,1.000000,1.000000}%
\pgfsetstrokecolor{currentstroke}%
\pgfsetdash{}{0pt}%
\pgfpathmoveto{\pgfqpoint{0.946717in}{0.761652in}}%
\pgfpathlineto{\pgfqpoint{2.880912in}{0.761652in}}%
\pgfusepath{stroke}%
\end{pgfscope}%
\begin{pgfscope}%
\definecolor{textcolor}{rgb}{0.150000,0.150000,0.150000}%
\pgfsetstrokecolor{textcolor}%
\pgfsetfillcolor{textcolor}%
\pgftext[x=0.729804in, y=0.706971in, left, base]{\color{textcolor}{\sffamily\fontsize{11.000000}{13.200000}\selectfont\catcode`\^=\active\def^{\ifmmode\sp\else\^{}\fi}\catcode`\%=\active\def%{\%}0}}%
\end{pgfscope}%
\begin{pgfscope}%
\pgfpathrectangle{\pgfqpoint{0.946717in}{0.663635in}}{\pgfqpoint{1.934195in}{2.156365in}}%
\pgfusepath{clip}%
\pgfsetroundcap%
\pgfsetroundjoin%
\pgfsetlinewidth{1.003750pt}%
\definecolor{currentstroke}{rgb}{1.000000,1.000000,1.000000}%
\pgfsetstrokecolor{currentstroke}%
\pgfsetdash{}{0pt}%
\pgfpathmoveto{\pgfqpoint{0.946717in}{1.243426in}}%
\pgfpathlineto{\pgfqpoint{2.880912in}{1.243426in}}%
\pgfusepath{stroke}%
\end{pgfscope}%
\begin{pgfscope}%
\definecolor{textcolor}{rgb}{0.150000,0.150000,0.150000}%
\pgfsetstrokecolor{textcolor}%
\pgfsetfillcolor{textcolor}%
\pgftext[x=0.474901in, y=1.188746in, left, base]{\color{textcolor}{\sffamily\fontsize{11.000000}{13.200000}\selectfont\catcode`\^=\active\def^{\ifmmode\sp\else\^{}\fi}\catcode`\%=\active\def%{\%}5000}}%
\end{pgfscope}%
\begin{pgfscope}%
\pgfpathrectangle{\pgfqpoint{0.946717in}{0.663635in}}{\pgfqpoint{1.934195in}{2.156365in}}%
\pgfusepath{clip}%
\pgfsetroundcap%
\pgfsetroundjoin%
\pgfsetlinewidth{1.003750pt}%
\definecolor{currentstroke}{rgb}{1.000000,1.000000,1.000000}%
\pgfsetstrokecolor{currentstroke}%
\pgfsetdash{}{0pt}%
\pgfpathmoveto{\pgfqpoint{0.946717in}{1.725201in}}%
\pgfpathlineto{\pgfqpoint{2.880912in}{1.725201in}}%
\pgfusepath{stroke}%
\end{pgfscope}%
\begin{pgfscope}%
\definecolor{textcolor}{rgb}{0.150000,0.150000,0.150000}%
\pgfsetstrokecolor{textcolor}%
\pgfsetfillcolor{textcolor}%
\pgftext[x=0.389934in, y=1.670520in, left, base]{\color{textcolor}{\sffamily\fontsize{11.000000}{13.200000}\selectfont\catcode`\^=\active\def^{\ifmmode\sp\else\^{}\fi}\catcode`\%=\active\def%{\%}10000}}%
\end{pgfscope}%
\begin{pgfscope}%
\pgfpathrectangle{\pgfqpoint{0.946717in}{0.663635in}}{\pgfqpoint{1.934195in}{2.156365in}}%
\pgfusepath{clip}%
\pgfsetroundcap%
\pgfsetroundjoin%
\pgfsetlinewidth{1.003750pt}%
\definecolor{currentstroke}{rgb}{1.000000,1.000000,1.000000}%
\pgfsetstrokecolor{currentstroke}%
\pgfsetdash{}{0pt}%
\pgfpathmoveto{\pgfqpoint{0.946717in}{2.206975in}}%
\pgfpathlineto{\pgfqpoint{2.880912in}{2.206975in}}%
\pgfusepath{stroke}%
\end{pgfscope}%
\begin{pgfscope}%
\definecolor{textcolor}{rgb}{0.150000,0.150000,0.150000}%
\pgfsetstrokecolor{textcolor}%
\pgfsetfillcolor{textcolor}%
\pgftext[x=0.389934in, y=2.152295in, left, base]{\color{textcolor}{\sffamily\fontsize{11.000000}{13.200000}\selectfont\catcode`\^=\active\def^{\ifmmode\sp\else\^{}\fi}\catcode`\%=\active\def%{\%}15000}}%
\end{pgfscope}%
\begin{pgfscope}%
\pgfpathrectangle{\pgfqpoint{0.946717in}{0.663635in}}{\pgfqpoint{1.934195in}{2.156365in}}%
\pgfusepath{clip}%
\pgfsetroundcap%
\pgfsetroundjoin%
\pgfsetlinewidth{1.003750pt}%
\definecolor{currentstroke}{rgb}{1.000000,1.000000,1.000000}%
\pgfsetstrokecolor{currentstroke}%
\pgfsetdash{}{0pt}%
\pgfpathmoveto{\pgfqpoint{0.946717in}{2.688750in}}%
\pgfpathlineto{\pgfqpoint{2.880912in}{2.688750in}}%
\pgfusepath{stroke}%
\end{pgfscope}%
\begin{pgfscope}%
\definecolor{textcolor}{rgb}{0.150000,0.150000,0.150000}%
\pgfsetstrokecolor{textcolor}%
\pgfsetfillcolor{textcolor}%
\pgftext[x=0.389934in, y=2.634069in, left, base]{\color{textcolor}{\sffamily\fontsize{11.000000}{13.200000}\selectfont\catcode`\^=\active\def^{\ifmmode\sp\else\^{}\fi}\catcode`\%=\active\def%{\%}20000}}%
\end{pgfscope}%
\begin{pgfscope}%
\definecolor{textcolor}{rgb}{0.150000,0.150000,0.150000}%
\pgfsetstrokecolor{textcolor}%
\pgfsetfillcolor{textcolor}%
\pgftext[x=0.334378in,y=1.741818in,,bottom,rotate=90.000000]{\color{textcolor}{\sffamily\fontsize{12.000000}{14.400000}\selectfont\catcode`\^=\active\def^{\ifmmode\sp\else\^{}\fi}\catcode`\%=\active\def%{\%}Average Write Load Per Node}}%
\end{pgfscope}%
\begin{pgfscope}%
\pgfpathrectangle{\pgfqpoint{0.946717in}{0.663635in}}{\pgfqpoint{1.934195in}{2.156365in}}%
\pgfusepath{clip}%
\pgfsetroundcap%
\pgfsetroundjoin%
\pgfsetlinewidth{1.505625pt}%
\definecolor{currentstroke}{rgb}{0.298039,0.447059,0.690196}%
\pgfsetstrokecolor{currentstroke}%
\pgfsetdash{}{0pt}%
\pgfpathmoveto{\pgfqpoint{1.034635in}{0.761652in}}%
\pgfpathlineto{\pgfqpoint{1.050620in}{0.761652in}}%
\pgfpathlineto{\pgfqpoint{1.066605in}{0.761652in}}%
\pgfpathlineto{\pgfqpoint{1.082590in}{0.761652in}}%
\pgfpathlineto{\pgfqpoint{1.098575in}{0.761652in}}%
\pgfpathlineto{\pgfqpoint{1.114560in}{0.761652in}}%
\pgfpathlineto{\pgfqpoint{1.130545in}{0.761652in}}%
\pgfpathlineto{\pgfqpoint{1.146530in}{0.803544in}}%
\pgfpathlineto{\pgfqpoint{1.162515in}{0.803544in}}%
\pgfpathlineto{\pgfqpoint{1.178500in}{0.803544in}}%
\pgfpathlineto{\pgfqpoint{1.194485in}{0.803544in}}%
\pgfpathlineto{\pgfqpoint{1.210471in}{0.800208in}}%
\pgfpathlineto{\pgfqpoint{1.226456in}{0.800208in}}%
\pgfpathlineto{\pgfqpoint{1.242441in}{0.800208in}}%
\pgfpathlineto{\pgfqpoint{1.258426in}{0.800208in}}%
\pgfpathlineto{\pgfqpoint{1.274411in}{0.966691in}}%
\pgfpathlineto{\pgfqpoint{1.290396in}{0.966691in}}%
\pgfpathlineto{\pgfqpoint{1.306381in}{0.966691in}}%
\pgfpathlineto{\pgfqpoint{1.322366in}{0.966691in}}%
\pgfpathlineto{\pgfqpoint{1.338351in}{1.445737in}}%
\pgfpathlineto{\pgfqpoint{1.354336in}{1.445737in}}%
\pgfpathlineto{\pgfqpoint{1.370321in}{1.445737in}}%
\pgfpathlineto{\pgfqpoint{1.386307in}{1.445737in}}%
\pgfpathlineto{\pgfqpoint{1.402292in}{1.627569in}}%
\pgfpathlineto{\pgfqpoint{1.418277in}{1.627569in}}%
\pgfpathlineto{\pgfqpoint{1.434262in}{1.627569in}}%
\pgfpathlineto{\pgfqpoint{1.450247in}{1.627569in}}%
\pgfpathlineto{\pgfqpoint{1.466232in}{1.645603in}}%
\pgfpathlineto{\pgfqpoint{1.482217in}{1.645603in}}%
\pgfpathlineto{\pgfqpoint{1.498202in}{1.645603in}}%
\pgfpathlineto{\pgfqpoint{1.514187in}{1.645603in}}%
\pgfpathlineto{\pgfqpoint{1.530172in}{2.007291in}}%
\pgfpathlineto{\pgfqpoint{1.546157in}{2.007291in}}%
\pgfpathlineto{\pgfqpoint{1.562142in}{2.007291in}}%
\pgfpathlineto{\pgfqpoint{1.578128in}{2.007291in}}%
\pgfpathlineto{\pgfqpoint{1.594113in}{2.136263in}}%
\pgfpathlineto{\pgfqpoint{1.610098in}{2.136263in}}%
\pgfpathlineto{\pgfqpoint{1.626083in}{2.136263in}}%
\pgfpathlineto{\pgfqpoint{1.642068in}{2.136263in}}%
\pgfpathlineto{\pgfqpoint{1.658053in}{2.176474in}}%
\pgfpathlineto{\pgfqpoint{1.674038in}{2.176474in}}%
\pgfpathlineto{\pgfqpoint{1.690023in}{2.176474in}}%
\pgfpathlineto{\pgfqpoint{1.706008in}{2.176474in}}%
\pgfpathlineto{\pgfqpoint{1.721993in}{2.529427in}}%
\pgfpathlineto{\pgfqpoint{1.737978in}{2.529427in}}%
\pgfpathlineto{\pgfqpoint{1.753963in}{2.529427in}}%
\pgfpathlineto{\pgfqpoint{1.769949in}{2.529427in}}%
\pgfpathlineto{\pgfqpoint{1.785934in}{2.652978in}}%
\pgfpathlineto{\pgfqpoint{1.801919in}{2.652978in}}%
\pgfpathlineto{\pgfqpoint{1.817904in}{2.652978in}}%
\pgfpathlineto{\pgfqpoint{1.833889in}{2.652978in}}%
\pgfpathlineto{\pgfqpoint{1.849874in}{2.721983in}}%
\pgfpathlineto{\pgfqpoint{1.865859in}{2.721983in}}%
\pgfpathlineto{\pgfqpoint{1.881844in}{2.721983in}}%
\pgfpathlineto{\pgfqpoint{1.897829in}{2.721983in}}%
\pgfpathlineto{\pgfqpoint{1.913814in}{2.475762in}}%
\pgfpathlineto{\pgfqpoint{1.929799in}{2.475762in}}%
\pgfpathlineto{\pgfqpoint{1.945785in}{2.475762in}}%
\pgfpathlineto{\pgfqpoint{1.961770in}{2.475762in}}%
\pgfpathlineto{\pgfqpoint{1.977755in}{1.658006in}}%
\pgfpathlineto{\pgfqpoint{1.993740in}{1.658006in}}%
\pgfpathlineto{\pgfqpoint{2.009725in}{1.658006in}}%
\pgfpathlineto{\pgfqpoint{2.025710in}{1.658006in}}%
\pgfpathlineto{\pgfqpoint{2.041695in}{1.653545in}}%
\pgfpathlineto{\pgfqpoint{2.057680in}{1.653545in}}%
\pgfpathlineto{\pgfqpoint{2.073665in}{1.653545in}}%
\pgfpathlineto{\pgfqpoint{2.089650in}{1.653545in}}%
\pgfpathlineto{\pgfqpoint{2.105635in}{1.561695in}}%
\pgfpathlineto{\pgfqpoint{2.121620in}{1.561695in}}%
\pgfpathlineto{\pgfqpoint{2.137606in}{1.561695in}}%
\pgfpathlineto{\pgfqpoint{2.153591in}{1.561695in}}%
\pgfpathlineto{\pgfqpoint{2.169576in}{1.423998in}}%
\pgfpathlineto{\pgfqpoint{2.185561in}{1.423998in}}%
\pgfpathlineto{\pgfqpoint{2.201546in}{1.423998in}}%
\pgfpathlineto{\pgfqpoint{2.217531in}{1.423998in}}%
\pgfpathlineto{\pgfqpoint{2.233516in}{1.358510in}}%
\pgfpathlineto{\pgfqpoint{2.249501in}{1.358510in}}%
\pgfpathlineto{\pgfqpoint{2.265486in}{1.358510in}}%
\pgfpathlineto{\pgfqpoint{2.281471in}{1.358510in}}%
\pgfpathlineto{\pgfqpoint{2.297456in}{1.295104in}}%
\pgfpathlineto{\pgfqpoint{2.313441in}{1.295104in}}%
\pgfpathlineto{\pgfqpoint{2.329427in}{1.295104in}}%
\pgfpathlineto{\pgfqpoint{2.345412in}{1.295104in}}%
\pgfpathlineto{\pgfqpoint{2.361397in}{1.162725in}}%
\pgfpathlineto{\pgfqpoint{2.377382in}{1.162725in}}%
\pgfpathlineto{\pgfqpoint{2.393367in}{1.162725in}}%
\pgfpathlineto{\pgfqpoint{2.409352in}{1.162725in}}%
\pgfpathlineto{\pgfqpoint{2.425337in}{1.069570in}}%
\pgfpathlineto{\pgfqpoint{2.441322in}{1.069570in}}%
\pgfpathlineto{\pgfqpoint{2.457307in}{1.069570in}}%
\pgfpathlineto{\pgfqpoint{2.473292in}{1.069570in}}%
\pgfpathlineto{\pgfqpoint{2.489277in}{1.037435in}}%
\pgfpathlineto{\pgfqpoint{2.505263in}{1.037435in}}%
\pgfpathlineto{\pgfqpoint{2.521248in}{1.037435in}}%
\pgfpathlineto{\pgfqpoint{2.537233in}{1.037435in}}%
\pgfpathlineto{\pgfqpoint{2.553218in}{0.957524in}}%
\pgfpathlineto{\pgfqpoint{2.569203in}{0.957524in}}%
\pgfpathlineto{\pgfqpoint{2.585188in}{0.957524in}}%
\pgfpathlineto{\pgfqpoint{2.601173in}{0.957524in}}%
\pgfpathlineto{\pgfqpoint{2.617158in}{0.898503in}}%
\pgfpathlineto{\pgfqpoint{2.633143in}{0.898503in}}%
\pgfpathlineto{\pgfqpoint{2.649128in}{0.898503in}}%
\pgfpathlineto{\pgfqpoint{2.665113in}{0.898503in}}%
\pgfpathlineto{\pgfqpoint{2.681098in}{0.884924in}}%
\pgfpathlineto{\pgfqpoint{2.697084in}{0.884924in}}%
\pgfpathlineto{\pgfqpoint{2.713069in}{0.884924in}}%
\pgfpathlineto{\pgfqpoint{2.729054in}{0.884924in}}%
\pgfpathlineto{\pgfqpoint{2.745039in}{0.770701in}}%
\pgfpathlineto{\pgfqpoint{2.761024in}{0.770701in}}%
\pgfpathlineto{\pgfqpoint{2.777009in}{0.770701in}}%
\pgfpathlineto{\pgfqpoint{2.792994in}{0.770701in}}%
\pgfusepath{stroke}%
\end{pgfscope}%
\begin{pgfscope}%
\pgfpathrectangle{\pgfqpoint{0.946717in}{0.663635in}}{\pgfqpoint{1.934195in}{2.156365in}}%
\pgfusepath{clip}%
\pgfsetbuttcap%
\pgfsetroundjoin%
\pgfsetlinewidth{1.505625pt}%
\definecolor{currentstroke}{rgb}{1.000000,0.647059,0.000000}%
\pgfsetstrokecolor{currentstroke}%
\pgfsetdash{{5.550000pt}{2.400000pt}}{0.000000pt}%
\pgfpathmoveto{\pgfqpoint{1.343466in}{0.663635in}}%
\pgfpathlineto{\pgfqpoint{1.343466in}{2.820000in}}%
\pgfusepath{stroke}%
\end{pgfscope}%
\begin{pgfscope}%
\pgfsetrectcap%
\pgfsetmiterjoin%
\pgfsetlinewidth{1.254687pt}%
\definecolor{currentstroke}{rgb}{1.000000,1.000000,1.000000}%
\pgfsetstrokecolor{currentstroke}%
\pgfsetdash{}{0pt}%
\pgfpathmoveto{\pgfqpoint{0.946717in}{0.663635in}}%
\pgfpathlineto{\pgfqpoint{0.946717in}{2.820000in}}%
\pgfusepath{stroke}%
\end{pgfscope}%
\begin{pgfscope}%
\pgfsetrectcap%
\pgfsetmiterjoin%
\pgfsetlinewidth{1.254687pt}%
\definecolor{currentstroke}{rgb}{1.000000,1.000000,1.000000}%
\pgfsetstrokecolor{currentstroke}%
\pgfsetdash{}{0pt}%
\pgfpathmoveto{\pgfqpoint{2.880912in}{0.663635in}}%
\pgfpathlineto{\pgfqpoint{2.880912in}{2.820000in}}%
\pgfusepath{stroke}%
\end{pgfscope}%
\begin{pgfscope}%
\pgfsetrectcap%
\pgfsetmiterjoin%
\pgfsetlinewidth{1.254687pt}%
\definecolor{currentstroke}{rgb}{1.000000,1.000000,1.000000}%
\pgfsetstrokecolor{currentstroke}%
\pgfsetdash{}{0pt}%
\pgfpathmoveto{\pgfqpoint{0.946717in}{0.663635in}}%
\pgfpathlineto{\pgfqpoint{2.880912in}{0.663635in}}%
\pgfusepath{stroke}%
\end{pgfscope}%
\begin{pgfscope}%
\pgfsetrectcap%
\pgfsetmiterjoin%
\pgfsetlinewidth{1.254687pt}%
\definecolor{currentstroke}{rgb}{1.000000,1.000000,1.000000}%
\pgfsetstrokecolor{currentstroke}%
\pgfsetdash{}{0pt}%
\pgfpathmoveto{\pgfqpoint{0.946717in}{2.820000in}}%
\pgfpathlineto{\pgfqpoint{2.880912in}{2.820000in}}%
\pgfusepath{stroke}%
\end{pgfscope}%
\begin{pgfscope}%
\pgfsetbuttcap%
\pgfsetmiterjoin%
\definecolor{currentfill}{rgb}{0.917647,0.917647,0.949020}%
\pgfsetfillcolor{currentfill}%
\pgfsetlinewidth{0.000000pt}%
\definecolor{currentstroke}{rgb}{0.000000,0.000000,0.000000}%
\pgfsetstrokecolor{currentstroke}%
\pgfsetstrokeopacity{0.000000}%
\pgfsetdash{}{0pt}%
\pgfpathmoveto{\pgfqpoint{3.485805in}{0.663635in}}%
\pgfpathlineto{\pgfqpoint{5.420000in}{0.663635in}}%
\pgfpathlineto{\pgfqpoint{5.420000in}{2.820000in}}%
\pgfpathlineto{\pgfqpoint{3.485805in}{2.820000in}}%
\pgfpathlineto{\pgfqpoint{3.485805in}{0.663635in}}%
\pgfpathclose%
\pgfusepath{fill}%
\end{pgfscope}%
\begin{pgfscope}%
\pgfpathrectangle{\pgfqpoint{3.485805in}{0.663635in}}{\pgfqpoint{1.934195in}{2.156365in}}%
\pgfusepath{clip}%
\pgfsetroundcap%
\pgfsetroundjoin%
\pgfsetlinewidth{1.003750pt}%
\definecolor{currentstroke}{rgb}{1.000000,1.000000,1.000000}%
\pgfsetstrokecolor{currentstroke}%
\pgfsetdash{}{0pt}%
\pgfpathmoveto{\pgfqpoint{3.573723in}{0.663635in}}%
\pgfpathlineto{\pgfqpoint{3.573723in}{2.820000in}}%
\pgfusepath{stroke}%
\end{pgfscope}%
\begin{pgfscope}%
\definecolor{textcolor}{rgb}{0.150000,0.150000,0.150000}%
\pgfsetstrokecolor{textcolor}%
\pgfsetfillcolor{textcolor}%
\pgftext[x=3.573723in,y=0.531691in,,top]{\color{textcolor}{\sffamily\fontsize{11.000000}{13.200000}\selectfont\catcode`\^=\active\def^{\ifmmode\sp\else\^{}\fi}\catcode`\%=\active\def%{\%}0}}%
\end{pgfscope}%
\begin{pgfscope}%
\pgfpathrectangle{\pgfqpoint{3.485805in}{0.663635in}}{\pgfqpoint{1.934195in}{2.156365in}}%
\pgfusepath{clip}%
\pgfsetroundcap%
\pgfsetroundjoin%
\pgfsetlinewidth{1.003750pt}%
\definecolor{currentstroke}{rgb}{1.000000,1.000000,1.000000}%
\pgfsetstrokecolor{currentstroke}%
\pgfsetdash{}{0pt}%
\pgfpathmoveto{\pgfqpoint{4.190691in}{0.663635in}}%
\pgfpathlineto{\pgfqpoint{4.190691in}{2.820000in}}%
\pgfusepath{stroke}%
\end{pgfscope}%
\begin{pgfscope}%
\definecolor{textcolor}{rgb}{0.150000,0.150000,0.150000}%
\pgfsetstrokecolor{textcolor}%
\pgfsetfillcolor{textcolor}%
\pgftext[x=4.190691in,y=0.531691in,,top]{\color{textcolor}{\sffamily\fontsize{11.000000}{13.200000}\selectfont\catcode`\^=\active\def^{\ifmmode\sp\else\^{}\fi}\catcode`\%=\active\def%{\%}200}}%
\end{pgfscope}%
\begin{pgfscope}%
\pgfpathrectangle{\pgfqpoint{3.485805in}{0.663635in}}{\pgfqpoint{1.934195in}{2.156365in}}%
\pgfusepath{clip}%
\pgfsetroundcap%
\pgfsetroundjoin%
\pgfsetlinewidth{1.003750pt}%
\definecolor{currentstroke}{rgb}{1.000000,1.000000,1.000000}%
\pgfsetstrokecolor{currentstroke}%
\pgfsetdash{}{0pt}%
\pgfpathmoveto{\pgfqpoint{4.807659in}{0.663635in}}%
\pgfpathlineto{\pgfqpoint{4.807659in}{2.820000in}}%
\pgfusepath{stroke}%
\end{pgfscope}%
\begin{pgfscope}%
\definecolor{textcolor}{rgb}{0.150000,0.150000,0.150000}%
\pgfsetstrokecolor{textcolor}%
\pgfsetfillcolor{textcolor}%
\pgftext[x=4.807659in,y=0.531691in,,top]{\color{textcolor}{\sffamily\fontsize{11.000000}{13.200000}\selectfont\catcode`\^=\active\def^{\ifmmode\sp\else\^{}\fi}\catcode`\%=\active\def%{\%}400}}%
\end{pgfscope}%
\begin{pgfscope}%
\definecolor{textcolor}{rgb}{0.150000,0.150000,0.150000}%
\pgfsetstrokecolor{textcolor}%
\pgfsetfillcolor{textcolor}%
\pgftext[x=4.452902in,y=0.336413in,,top]{\color{textcolor}{\sffamily\fontsize{12.000000}{14.400000}\selectfont\catcode`\^=\active\def^{\ifmmode\sp\else\^{}\fi}\catcode`\%=\active\def%{\%}Time (s)}}%
\end{pgfscope}%
\begin{pgfscope}%
\pgfpathrectangle{\pgfqpoint{3.485805in}{0.663635in}}{\pgfqpoint{1.934195in}{2.156365in}}%
\pgfusepath{clip}%
\pgfsetroundcap%
\pgfsetroundjoin%
\pgfsetlinewidth{1.003750pt}%
\definecolor{currentstroke}{rgb}{1.000000,1.000000,1.000000}%
\pgfsetstrokecolor{currentstroke}%
\pgfsetdash{}{0pt}%
\pgfpathmoveto{\pgfqpoint{3.485805in}{1.202727in}}%
\pgfpathlineto{\pgfqpoint{5.420000in}{1.202727in}}%
\pgfusepath{stroke}%
\end{pgfscope}%
\begin{pgfscope}%
\definecolor{textcolor}{rgb}{0.150000,0.150000,0.150000}%
\pgfsetstrokecolor{textcolor}%
\pgfsetfillcolor{textcolor}%
\pgftext[x=3.268892in, y=1.148046in, left, base]{\color{textcolor}{\sffamily\fontsize{11.000000}{13.200000}\selectfont\catcode`\^=\active\def^{\ifmmode\sp\else\^{}\fi}\catcode`\%=\active\def%{\%}1}}%
\end{pgfscope}%
\begin{pgfscope}%
\pgfpathrectangle{\pgfqpoint{3.485805in}{0.663635in}}{\pgfqpoint{1.934195in}{2.156365in}}%
\pgfusepath{clip}%
\pgfsetroundcap%
\pgfsetroundjoin%
\pgfsetlinewidth{1.003750pt}%
\definecolor{currentstroke}{rgb}{1.000000,1.000000,1.000000}%
\pgfsetstrokecolor{currentstroke}%
\pgfsetdash{}{0pt}%
\pgfpathmoveto{\pgfqpoint{3.485805in}{2.280909in}}%
\pgfpathlineto{\pgfqpoint{5.420000in}{2.280909in}}%
\pgfusepath{stroke}%
\end{pgfscope}%
\begin{pgfscope}%
\definecolor{textcolor}{rgb}{0.150000,0.150000,0.150000}%
\pgfsetstrokecolor{textcolor}%
\pgfsetfillcolor{textcolor}%
\pgftext[x=3.268892in, y=2.226228in, left, base]{\color{textcolor}{\sffamily\fontsize{11.000000}{13.200000}\selectfont\catcode`\^=\active\def^{\ifmmode\sp\else\^{}\fi}\catcode`\%=\active\def%{\%}2}}%
\end{pgfscope}%
\begin{pgfscope}%
\definecolor{textcolor}{rgb}{0.150000,0.150000,0.150000}%
\pgfsetstrokecolor{textcolor}%
\pgfsetfillcolor{textcolor}%
\pgftext[x=3.213337in,y=1.741818in,,bottom,rotate=90.000000]{\color{textcolor}{\sffamily\fontsize{12.000000}{14.400000}\selectfont\catcode`\^=\active\def^{\ifmmode\sp\else\^{}\fi}\catcode`\%=\active\def%{\%}\# nodes}}%
\end{pgfscope}%
\begin{pgfscope}%
\pgfpathrectangle{\pgfqpoint{3.485805in}{0.663635in}}{\pgfqpoint{1.934195in}{2.156365in}}%
\pgfusepath{clip}%
\pgfsetroundcap%
\pgfsetroundjoin%
\pgfsetlinewidth{1.505625pt}%
\definecolor{currentstroke}{rgb}{0.298039,0.447059,0.690196}%
\pgfsetstrokecolor{currentstroke}%
\pgfsetdash{}{0pt}%
\pgfpathmoveto{\pgfqpoint{3.573723in}{1.202727in}}%
\pgfpathlineto{\pgfqpoint{3.589147in}{1.202727in}}%
\pgfpathlineto{\pgfqpoint{3.604571in}{1.202727in}}%
\pgfpathlineto{\pgfqpoint{3.619995in}{1.202727in}}%
\pgfpathlineto{\pgfqpoint{3.635419in}{1.202727in}}%
\pgfpathlineto{\pgfqpoint{3.650844in}{1.202727in}}%
\pgfpathlineto{\pgfqpoint{3.666268in}{1.202727in}}%
\pgfpathlineto{\pgfqpoint{3.681692in}{1.202727in}}%
\pgfpathlineto{\pgfqpoint{3.697116in}{1.202727in}}%
\pgfpathlineto{\pgfqpoint{3.712540in}{1.202727in}}%
\pgfpathlineto{\pgfqpoint{3.727965in}{1.202727in}}%
\pgfpathlineto{\pgfqpoint{3.743389in}{1.202727in}}%
\pgfpathlineto{\pgfqpoint{3.758813in}{1.202727in}}%
\pgfpathlineto{\pgfqpoint{3.774237in}{1.202727in}}%
\pgfpathlineto{\pgfqpoint{3.789661in}{1.202727in}}%
\pgfpathlineto{\pgfqpoint{3.805086in}{1.202727in}}%
\pgfpathlineto{\pgfqpoint{3.820510in}{1.202727in}}%
\pgfpathlineto{\pgfqpoint{3.835934in}{1.202727in}}%
\pgfpathlineto{\pgfqpoint{3.851358in}{1.202727in}}%
\pgfpathlineto{\pgfqpoint{3.866783in}{1.202727in}}%
\pgfpathlineto{\pgfqpoint{3.882207in}{1.202727in}}%
\pgfpathlineto{\pgfqpoint{3.897631in}{1.202727in}}%
\pgfpathlineto{\pgfqpoint{3.913055in}{1.202727in}}%
\pgfpathlineto{\pgfqpoint{3.928479in}{1.202727in}}%
\pgfpathlineto{\pgfqpoint{3.943904in}{1.202727in}}%
\pgfpathlineto{\pgfqpoint{3.959328in}{1.202727in}}%
\pgfpathlineto{\pgfqpoint{3.974752in}{1.202727in}}%
\pgfpathlineto{\pgfqpoint{3.990176in}{1.202727in}}%
\pgfpathlineto{\pgfqpoint{4.005600in}{1.202727in}}%
\pgfpathlineto{\pgfqpoint{4.021025in}{1.202727in}}%
\pgfpathlineto{\pgfqpoint{4.036449in}{1.202727in}}%
\pgfpathlineto{\pgfqpoint{4.051873in}{1.202727in}}%
\pgfpathlineto{\pgfqpoint{4.067297in}{1.202727in}}%
\pgfpathlineto{\pgfqpoint{4.082721in}{1.202727in}}%
\pgfpathlineto{\pgfqpoint{4.098146in}{1.202727in}}%
\pgfpathlineto{\pgfqpoint{4.113570in}{1.202727in}}%
\pgfpathlineto{\pgfqpoint{4.128994in}{1.202727in}}%
\pgfpathlineto{\pgfqpoint{4.144418in}{1.202727in}}%
\pgfpathlineto{\pgfqpoint{4.159842in}{1.202727in}}%
\pgfpathlineto{\pgfqpoint{4.175267in}{1.202727in}}%
\pgfpathlineto{\pgfqpoint{4.190691in}{1.202727in}}%
\pgfpathlineto{\pgfqpoint{4.206115in}{1.202727in}}%
\pgfpathlineto{\pgfqpoint{4.221539in}{1.202727in}}%
\pgfpathlineto{\pgfqpoint{4.236963in}{1.202727in}}%
\pgfpathlineto{\pgfqpoint{4.252388in}{1.202727in}}%
\pgfpathlineto{\pgfqpoint{4.267812in}{1.202727in}}%
\pgfpathlineto{\pgfqpoint{4.283236in}{1.202727in}}%
\pgfpathlineto{\pgfqpoint{4.298660in}{1.202727in}}%
\pgfpathlineto{\pgfqpoint{4.314084in}{1.202727in}}%
\pgfpathlineto{\pgfqpoint{4.329509in}{1.202727in}}%
\pgfpathlineto{\pgfqpoint{4.344933in}{1.202727in}}%
\pgfpathlineto{\pgfqpoint{4.360357in}{1.202727in}}%
\pgfpathlineto{\pgfqpoint{4.375781in}{1.202727in}}%
\pgfpathlineto{\pgfqpoint{4.391206in}{1.202727in}}%
\pgfpathlineto{\pgfqpoint{4.406630in}{1.202727in}}%
\pgfpathlineto{\pgfqpoint{4.422054in}{2.280909in}}%
\pgfpathlineto{\pgfqpoint{4.437478in}{2.280909in}}%
\pgfpathlineto{\pgfqpoint{4.452902in}{2.280909in}}%
\pgfpathlineto{\pgfqpoint{4.468327in}{2.280909in}}%
\pgfpathlineto{\pgfqpoint{4.483751in}{2.280909in}}%
\pgfpathlineto{\pgfqpoint{4.499175in}{2.280909in}}%
\pgfpathlineto{\pgfqpoint{4.514599in}{2.280909in}}%
\pgfpathlineto{\pgfqpoint{4.530023in}{2.280909in}}%
\pgfpathlineto{\pgfqpoint{4.545448in}{2.280909in}}%
\pgfpathlineto{\pgfqpoint{4.560872in}{2.280909in}}%
\pgfpathlineto{\pgfqpoint{4.576296in}{2.280909in}}%
\pgfpathlineto{\pgfqpoint{4.591720in}{2.280909in}}%
\pgfpathlineto{\pgfqpoint{4.607144in}{2.280909in}}%
\pgfpathlineto{\pgfqpoint{4.622569in}{2.280909in}}%
\pgfpathlineto{\pgfqpoint{4.637993in}{2.280909in}}%
\pgfpathlineto{\pgfqpoint{4.653417in}{2.280909in}}%
\pgfpathlineto{\pgfqpoint{4.668841in}{2.280909in}}%
\pgfpathlineto{\pgfqpoint{4.684265in}{2.280909in}}%
\pgfpathlineto{\pgfqpoint{4.699690in}{2.280909in}}%
\pgfpathlineto{\pgfqpoint{4.715114in}{2.280909in}}%
\pgfpathlineto{\pgfqpoint{4.730538in}{2.280909in}}%
\pgfpathlineto{\pgfqpoint{4.745962in}{2.280909in}}%
\pgfpathlineto{\pgfqpoint{4.761386in}{2.280909in}}%
\pgfpathlineto{\pgfqpoint{4.776811in}{2.280909in}}%
\pgfpathlineto{\pgfqpoint{4.792235in}{2.280909in}}%
\pgfpathlineto{\pgfqpoint{4.807659in}{2.280909in}}%
\pgfpathlineto{\pgfqpoint{4.823083in}{2.280909in}}%
\pgfpathlineto{\pgfqpoint{4.838507in}{2.280909in}}%
\pgfpathlineto{\pgfqpoint{4.853932in}{2.280909in}}%
\pgfpathlineto{\pgfqpoint{4.869356in}{2.280909in}}%
\pgfpathlineto{\pgfqpoint{4.884780in}{2.280909in}}%
\pgfpathlineto{\pgfqpoint{4.900204in}{2.280909in}}%
\pgfpathlineto{\pgfqpoint{4.915628in}{2.280909in}}%
\pgfpathlineto{\pgfqpoint{4.931053in}{2.280909in}}%
\pgfpathlineto{\pgfqpoint{4.946477in}{2.280909in}}%
\pgfpathlineto{\pgfqpoint{4.961901in}{2.280909in}}%
\pgfpathlineto{\pgfqpoint{4.977325in}{2.280909in}}%
\pgfpathlineto{\pgfqpoint{4.992750in}{2.280909in}}%
\pgfpathlineto{\pgfqpoint{5.008174in}{2.280909in}}%
\pgfpathlineto{\pgfqpoint{5.023598in}{2.280909in}}%
\pgfpathlineto{\pgfqpoint{5.039022in}{2.280909in}}%
\pgfpathlineto{\pgfqpoint{5.054446in}{2.280909in}}%
\pgfpathlineto{\pgfqpoint{5.069871in}{2.280909in}}%
\pgfpathlineto{\pgfqpoint{5.085295in}{2.280909in}}%
\pgfpathlineto{\pgfqpoint{5.100719in}{2.280909in}}%
\pgfpathlineto{\pgfqpoint{5.116143in}{2.280909in}}%
\pgfpathlineto{\pgfqpoint{5.131567in}{2.280909in}}%
\pgfpathlineto{\pgfqpoint{5.146992in}{2.280909in}}%
\pgfpathlineto{\pgfqpoint{5.162416in}{2.280909in}}%
\pgfpathlineto{\pgfqpoint{5.177840in}{2.280909in}}%
\pgfpathlineto{\pgfqpoint{5.193264in}{2.280909in}}%
\pgfpathlineto{\pgfqpoint{5.208688in}{2.280909in}}%
\pgfpathlineto{\pgfqpoint{5.224113in}{2.280909in}}%
\pgfpathlineto{\pgfqpoint{5.239537in}{2.280909in}}%
\pgfpathlineto{\pgfqpoint{5.254961in}{2.280909in}}%
\pgfpathlineto{\pgfqpoint{5.270385in}{2.280909in}}%
\pgfpathlineto{\pgfqpoint{5.285809in}{2.280909in}}%
\pgfpathlineto{\pgfqpoint{5.301234in}{2.280909in}}%
\pgfpathlineto{\pgfqpoint{5.316658in}{2.280909in}}%
\pgfpathlineto{\pgfqpoint{5.332082in}{2.280909in}}%
\pgfusepath{stroke}%
\end{pgfscope}%
\begin{pgfscope}%
\pgfsetrectcap%
\pgfsetmiterjoin%
\pgfsetlinewidth{1.254687pt}%
\definecolor{currentstroke}{rgb}{1.000000,1.000000,1.000000}%
\pgfsetstrokecolor{currentstroke}%
\pgfsetdash{}{0pt}%
\pgfpathmoveto{\pgfqpoint{3.485805in}{0.663635in}}%
\pgfpathlineto{\pgfqpoint{3.485805in}{2.820000in}}%
\pgfusepath{stroke}%
\end{pgfscope}%
\begin{pgfscope}%
\pgfsetrectcap%
\pgfsetmiterjoin%
\pgfsetlinewidth{1.254687pt}%
\definecolor{currentstroke}{rgb}{1.000000,1.000000,1.000000}%
\pgfsetstrokecolor{currentstroke}%
\pgfsetdash{}{0pt}%
\pgfpathmoveto{\pgfqpoint{5.420000in}{0.663635in}}%
\pgfpathlineto{\pgfqpoint{5.420000in}{2.820000in}}%
\pgfusepath{stroke}%
\end{pgfscope}%
\begin{pgfscope}%
\pgfsetrectcap%
\pgfsetmiterjoin%
\pgfsetlinewidth{1.254687pt}%
\definecolor{currentstroke}{rgb}{1.000000,1.000000,1.000000}%
\pgfsetstrokecolor{currentstroke}%
\pgfsetdash{}{0pt}%
\pgfpathmoveto{\pgfqpoint{3.485805in}{0.663635in}}%
\pgfpathlineto{\pgfqpoint{5.420000in}{0.663635in}}%
\pgfusepath{stroke}%
\end{pgfscope}%
\begin{pgfscope}%
\pgfsetrectcap%
\pgfsetmiterjoin%
\pgfsetlinewidth{1.254687pt}%
\definecolor{currentstroke}{rgb}{1.000000,1.000000,1.000000}%
\pgfsetstrokecolor{currentstroke}%
\pgfsetdash{}{0pt}%
\pgfpathmoveto{\pgfqpoint{3.485805in}{2.820000in}}%
\pgfpathlineto{\pgfqpoint{5.420000in}{2.820000in}}%
\pgfusepath{stroke}%
\end{pgfscope}%
\begin{pgfscope}%
\pgfsetbuttcap%
\pgfsetmiterjoin%
\definecolor{currentfill}{rgb}{0.917647,0.917647,0.949020}%
\pgfsetfillcolor{currentfill}%
\pgfsetfillopacity{0.800000}%
\pgfsetlinewidth{1.003750pt}%
\definecolor{currentstroke}{rgb}{0.800000,0.800000,0.800000}%
\pgfsetstrokecolor{currentstroke}%
\pgfsetstrokeopacity{0.800000}%
\pgfsetdash{}{0pt}%
\pgfpathmoveto{\pgfqpoint{2.100358in}{2.751281in}}%
\pgfpathlineto{\pgfqpoint{3.499642in}{2.751281in}}%
\pgfpathquadraticcurveto{\pgfqpoint{3.521864in}{2.751281in}}{\pgfqpoint{3.521864in}{2.773503in}}%
\pgfpathlineto{\pgfqpoint{3.521864in}{2.922222in}}%
\pgfpathquadraticcurveto{\pgfqpoint{3.521864in}{2.944444in}}{\pgfqpoint{3.499642in}{2.944444in}}%
\pgfpathlineto{\pgfqpoint{2.100358in}{2.944444in}}%
\pgfpathquadraticcurveto{\pgfqpoint{2.078136in}{2.944444in}}{\pgfqpoint{2.078136in}{2.922222in}}%
\pgfpathlineto{\pgfqpoint{2.078136in}{2.773503in}}%
\pgfpathquadraticcurveto{\pgfqpoint{2.078136in}{2.751281in}}{\pgfqpoint{2.100358in}{2.751281in}}%
\pgfpathlineto{\pgfqpoint{2.100358in}{2.751281in}}%
\pgfpathclose%
\pgfusepath{stroke,fill}%
\end{pgfscope}%
\begin{pgfscope}%
\pgfsetbuttcap%
\pgfsetroundjoin%
\pgfsetlinewidth{1.505625pt}%
\definecolor{currentstroke}{rgb}{1.000000,0.647059,0.000000}%
\pgfsetstrokecolor{currentstroke}%
\pgfsetdash{{5.550000pt}{2.400000pt}}{0.000000pt}%
\pgfpathmoveto{\pgfqpoint{2.122580in}{2.857997in}}%
\pgfpathlineto{\pgfqpoint{2.233691in}{2.857997in}}%
\pgfpathlineto{\pgfqpoint{2.344803in}{2.857997in}}%
\pgfusepath{stroke}%
\end{pgfscope}%
\begin{pgfscope}%
\definecolor{textcolor}{rgb}{0.150000,0.150000,0.150000}%
\pgfsetstrokecolor{textcolor}%
\pgfsetfillcolor{textcolor}%
\pgftext[x=2.433691in,y=2.819108in,left,base]{\color{textcolor}{\sffamily\fontsize{8.000000}{9.600000}\selectfont\catcode`\^=\active\def^{\ifmmode\sp\else\^{}\fi}\catcode`\%=\active\def%{\%}start of scaling action}}%
\end{pgfscope}%
\end{pgfpicture}%
\makeatother%
\endgroup%

    \caption{Average write load per node and amount of nodes during a horizontal scaling action}
    \label{fig:horizontal-elasticity}
\end{figure}

At approximately 290s a sudden drop in the metric can be observed. This is the point when the scaling action becomes effective and the K8ssandra node is ready. Then, after another few moments the metric drops under the set boundary of 5000. Tests of this kind are difficult to run over an extended period of time because of a limitation of \texttt{cassandra-stress}. When the load generator is started, it collects all available nodes in the cluster through Cassandra's communication protocol \texttt{Gossip}. \texttt{Gossip} is the protocol that Cassandra uses internally for its nodes to communicate with each other\footnote{\raggedright\url{https://docs.datastax.com/en/cassandra-oss/3.x/cassandra/architecture/archGossipAbout.html}}. While \texttt{cassandra-stress} is running, new nodes are not recognized and requests are therefore not sent to added nodes. Possible solutions to this will be discussed in \Cref{ch:conclusion}.

\begin{figure}[H]
    \centering
    %% Creator: Matplotlib, PGF backend
%%
%% To include the figure in your LaTeX document, write
%%   \input{<filename>.pgf}
%%
%% Make sure the required packages are loaded in your preamble
%%   \usepackage{pgf}
%%
%% Also ensure that all the required font packages are loaded; for instance,
%% the lmodern package is sometimes necessary when using math font.
%%   \usepackage{lmodern}
%%
%% Figures using additional raster images can only be included by \input if
%% they are in the same directory as the main LaTeX file. For loading figures
%% from other directories you can use the `import` package
%%   \usepackage{import}
%%
%% and then include the figures with
%%   \import{<path to file>}{<filename>.pgf}
%%
%% Matplotlib used the following preamble
%%   \def\mathdefault#1{#1}
%%   \everymath=\expandafter{\the\everymath\displaystyle}
%%   
%%   \usepackage{fontspec}
%%   \setmainfont{DejaVuSerif.ttf}[Path=\detokenize{/usr/local/lib/python3.11/site-packages/matplotlib/mpl-data/fonts/ttf/}]
%%   \setsansfont{Arial.ttf}[Path=\detokenize{/System/Library/Fonts/Supplemental/}]
%%   \setmonofont{DejaVuSansMono.ttf}[Path=\detokenize{/usr/local/lib/python3.11/site-packages/matplotlib/mpl-data/fonts/ttf/}]
%%   \makeatletter\@ifpackageloaded{underscore}{}{\usepackage[strings]{underscore}}\makeatother
%%
\begingroup%
\makeatletter%
\begin{pgfpicture}%
\pgfpathrectangle{\pgfpointorigin}{\pgfqpoint{5.600000in}{5.000000in}}%
\pgfusepath{use as bounding box, clip}%
\begin{pgfscope}%
\pgfsetbuttcap%
\pgfsetmiterjoin%
\definecolor{currentfill}{rgb}{1.000000,1.000000,1.000000}%
\pgfsetfillcolor{currentfill}%
\pgfsetlinewidth{0.000000pt}%
\definecolor{currentstroke}{rgb}{1.000000,1.000000,1.000000}%
\pgfsetstrokecolor{currentstroke}%
\pgfsetdash{}{0pt}%
\pgfpathmoveto{\pgfqpoint{0.000000in}{0.000000in}}%
\pgfpathlineto{\pgfqpoint{5.600000in}{0.000000in}}%
\pgfpathlineto{\pgfqpoint{5.600000in}{5.000000in}}%
\pgfpathlineto{\pgfqpoint{0.000000in}{5.000000in}}%
\pgfpathlineto{\pgfqpoint{0.000000in}{0.000000in}}%
\pgfpathclose%
\pgfusepath{fill}%
\end{pgfscope}%
\begin{pgfscope}%
\pgfsetbuttcap%
\pgfsetmiterjoin%
\definecolor{currentfill}{rgb}{0.917647,0.917647,0.949020}%
\pgfsetfillcolor{currentfill}%
\pgfsetlinewidth{0.000000pt}%
\definecolor{currentstroke}{rgb}{0.000000,0.000000,0.000000}%
\pgfsetstrokecolor{currentstroke}%
\pgfsetstrokeopacity{0.000000}%
\pgfsetdash{}{0pt}%
\pgfpathmoveto{\pgfqpoint{0.946717in}{3.073635in}}%
\pgfpathlineto{\pgfqpoint{2.772727in}{3.073635in}}%
\pgfpathlineto{\pgfqpoint{2.772727in}{4.615329in}}%
\pgfpathlineto{\pgfqpoint{0.946717in}{4.615329in}}%
\pgfpathlineto{\pgfqpoint{0.946717in}{3.073635in}}%
\pgfpathclose%
\pgfusepath{fill}%
\end{pgfscope}%
\begin{pgfscope}%
\pgfpathrectangle{\pgfqpoint{0.946717in}{3.073635in}}{\pgfqpoint{1.826010in}{1.541693in}}%
\pgfusepath{clip}%
\pgfsetroundcap%
\pgfsetroundjoin%
\pgfsetlinewidth{1.003750pt}%
\definecolor{currentstroke}{rgb}{1.000000,1.000000,1.000000}%
\pgfsetstrokecolor{currentstroke}%
\pgfsetdash{}{0pt}%
\pgfpathmoveto{\pgfqpoint{1.029717in}{3.073635in}}%
\pgfpathlineto{\pgfqpoint{1.029717in}{4.615329in}}%
\pgfusepath{stroke}%
\end{pgfscope}%
\begin{pgfscope}%
\definecolor{textcolor}{rgb}{0.150000,0.150000,0.150000}%
\pgfsetstrokecolor{textcolor}%
\pgfsetfillcolor{textcolor}%
\pgftext[x=1.029717in,y=2.941691in,,top]{\color{textcolor}{\sffamily\fontsize{11.000000}{13.200000}\selectfont\catcode`\^=\active\def^{\ifmmode\sp\else\^{}\fi}\catcode`\%=\active\def%{\%}0}}%
\end{pgfscope}%
\begin{pgfscope}%
\pgfpathrectangle{\pgfqpoint{0.946717in}{3.073635in}}{\pgfqpoint{1.826010in}{1.541693in}}%
\pgfusepath{clip}%
\pgfsetroundcap%
\pgfsetroundjoin%
\pgfsetlinewidth{1.003750pt}%
\definecolor{currentstroke}{rgb}{1.000000,1.000000,1.000000}%
\pgfsetstrokecolor{currentstroke}%
\pgfsetdash{}{0pt}%
\pgfpathmoveto{\pgfqpoint{2.048128in}{3.073635in}}%
\pgfpathlineto{\pgfqpoint{2.048128in}{4.615329in}}%
\pgfusepath{stroke}%
\end{pgfscope}%
\begin{pgfscope}%
\definecolor{textcolor}{rgb}{0.150000,0.150000,0.150000}%
\pgfsetstrokecolor{textcolor}%
\pgfsetfillcolor{textcolor}%
\pgftext[x=2.048128in,y=2.941691in,,top]{\color{textcolor}{\sffamily\fontsize{11.000000}{13.200000}\selectfont\catcode`\^=\active\def^{\ifmmode\sp\else\^{}\fi}\catcode`\%=\active\def%{\%}2000}}%
\end{pgfscope}%
\begin{pgfscope}%
\definecolor{textcolor}{rgb}{0.150000,0.150000,0.150000}%
\pgfsetstrokecolor{textcolor}%
\pgfsetfillcolor{textcolor}%
\pgftext[x=1.859722in,y=2.746413in,,top]{\color{textcolor}{\sffamily\fontsize{12.000000}{14.400000}\selectfont\catcode`\^=\active\def^{\ifmmode\sp\else\^{}\fi}\catcode`\%=\active\def%{\%}Time (s)}}%
\end{pgfscope}%
\begin{pgfscope}%
\pgfpathrectangle{\pgfqpoint{0.946717in}{3.073635in}}{\pgfqpoint{1.826010in}{1.541693in}}%
\pgfusepath{clip}%
\pgfsetroundcap%
\pgfsetroundjoin%
\pgfsetlinewidth{1.003750pt}%
\definecolor{currentstroke}{rgb}{1.000000,1.000000,1.000000}%
\pgfsetstrokecolor{currentstroke}%
\pgfsetdash{}{0pt}%
\pgfpathmoveto{\pgfqpoint{0.946717in}{3.143712in}}%
\pgfpathlineto{\pgfqpoint{2.772727in}{3.143712in}}%
\pgfusepath{stroke}%
\end{pgfscope}%
\begin{pgfscope}%
\definecolor{textcolor}{rgb}{0.150000,0.150000,0.150000}%
\pgfsetstrokecolor{textcolor}%
\pgfsetfillcolor{textcolor}%
\pgftext[x=0.729804in, y=3.089032in, left, base]{\color{textcolor}{\sffamily\fontsize{11.000000}{13.200000}\selectfont\catcode`\^=\active\def^{\ifmmode\sp\else\^{}\fi}\catcode`\%=\active\def%{\%}0}}%
\end{pgfscope}%
\begin{pgfscope}%
\pgfpathrectangle{\pgfqpoint{0.946717in}{3.073635in}}{\pgfqpoint{1.826010in}{1.541693in}}%
\pgfusepath{clip}%
\pgfsetroundcap%
\pgfsetroundjoin%
\pgfsetlinewidth{1.003750pt}%
\definecolor{currentstroke}{rgb}{1.000000,1.000000,1.000000}%
\pgfsetstrokecolor{currentstroke}%
\pgfsetdash{}{0pt}%
\pgfpathmoveto{\pgfqpoint{0.946717in}{3.515177in}}%
\pgfpathlineto{\pgfqpoint{2.772727in}{3.515177in}}%
\pgfusepath{stroke}%
\end{pgfscope}%
\begin{pgfscope}%
\definecolor{textcolor}{rgb}{0.150000,0.150000,0.150000}%
\pgfsetstrokecolor{textcolor}%
\pgfsetfillcolor{textcolor}%
\pgftext[x=0.474901in, y=3.460496in, left, base]{\color{textcolor}{\sffamily\fontsize{11.000000}{13.200000}\selectfont\catcode`\^=\active\def^{\ifmmode\sp\else\^{}\fi}\catcode`\%=\active\def%{\%}5000}}%
\end{pgfscope}%
\begin{pgfscope}%
\pgfpathrectangle{\pgfqpoint{0.946717in}{3.073635in}}{\pgfqpoint{1.826010in}{1.541693in}}%
\pgfusepath{clip}%
\pgfsetroundcap%
\pgfsetroundjoin%
\pgfsetlinewidth{1.003750pt}%
\definecolor{currentstroke}{rgb}{1.000000,1.000000,1.000000}%
\pgfsetstrokecolor{currentstroke}%
\pgfsetdash{}{0pt}%
\pgfpathmoveto{\pgfqpoint{0.946717in}{3.886641in}}%
\pgfpathlineto{\pgfqpoint{2.772727in}{3.886641in}}%
\pgfusepath{stroke}%
\end{pgfscope}%
\begin{pgfscope}%
\definecolor{textcolor}{rgb}{0.150000,0.150000,0.150000}%
\pgfsetstrokecolor{textcolor}%
\pgfsetfillcolor{textcolor}%
\pgftext[x=0.389934in, y=3.831960in, left, base]{\color{textcolor}{\sffamily\fontsize{11.000000}{13.200000}\selectfont\catcode`\^=\active\def^{\ifmmode\sp\else\^{}\fi}\catcode`\%=\active\def%{\%}10000}}%
\end{pgfscope}%
\begin{pgfscope}%
\pgfpathrectangle{\pgfqpoint{0.946717in}{3.073635in}}{\pgfqpoint{1.826010in}{1.541693in}}%
\pgfusepath{clip}%
\pgfsetroundcap%
\pgfsetroundjoin%
\pgfsetlinewidth{1.003750pt}%
\definecolor{currentstroke}{rgb}{1.000000,1.000000,1.000000}%
\pgfsetstrokecolor{currentstroke}%
\pgfsetdash{}{0pt}%
\pgfpathmoveto{\pgfqpoint{0.946717in}{4.258105in}}%
\pgfpathlineto{\pgfqpoint{2.772727in}{4.258105in}}%
\pgfusepath{stroke}%
\end{pgfscope}%
\begin{pgfscope}%
\definecolor{textcolor}{rgb}{0.150000,0.150000,0.150000}%
\pgfsetstrokecolor{textcolor}%
\pgfsetfillcolor{textcolor}%
\pgftext[x=0.389934in, y=4.203424in, left, base]{\color{textcolor}{\sffamily\fontsize{11.000000}{13.200000}\selectfont\catcode`\^=\active\def^{\ifmmode\sp\else\^{}\fi}\catcode`\%=\active\def%{\%}15000}}%
\end{pgfscope}%
\begin{pgfscope}%
\definecolor{textcolor}{rgb}{0.150000,0.150000,0.150000}%
\pgfsetstrokecolor{textcolor}%
\pgfsetfillcolor{textcolor}%
\pgftext[x=0.334378in,y=3.844482in,,bottom,rotate=90.000000]{\color{textcolor}{\sffamily\fontsize{12.000000}{14.400000}\selectfont\catcode`\^=\active\def^{\ifmmode\sp\else\^{}\fi}\catcode`\%=\active\def%{\%}Average Write Load Per Node}}%
\end{pgfscope}%
\begin{pgfscope}%
\pgfpathrectangle{\pgfqpoint{0.946717in}{3.073635in}}{\pgfqpoint{1.826010in}{1.541693in}}%
\pgfusepath{clip}%
\pgfsetbuttcap%
\pgfsetroundjoin%
\pgfsetlinewidth{1.505625pt}%
\definecolor{currentstroke}{rgb}{1.000000,0.647059,0.000000}%
\pgfsetstrokecolor{currentstroke}%
\pgfsetdash{{1.500000pt}{2.475000pt}}{0.000000pt}%
\pgfpathmoveto{\pgfqpoint{1.174331in}{3.073635in}}%
\pgfpathlineto{\pgfqpoint{1.174331in}{4.615329in}}%
\pgfusepath{stroke}%
\end{pgfscope}%
\begin{pgfscope}%
\pgfpathrectangle{\pgfqpoint{0.946717in}{3.073635in}}{\pgfqpoint{1.826010in}{1.541693in}}%
\pgfusepath{clip}%
\pgfsetbuttcap%
\pgfsetroundjoin%
\pgfsetlinewidth{1.505625pt}%
\definecolor{currentstroke}{rgb}{1.000000,0.647059,0.000000}%
\pgfsetstrokecolor{currentstroke}%
\pgfsetdash{{1.500000pt}{2.475000pt}}{0.000000pt}%
\pgfpathmoveto{\pgfqpoint{1.482401in}{3.073635in}}%
\pgfpathlineto{\pgfqpoint{1.482401in}{4.615329in}}%
\pgfusepath{stroke}%
\end{pgfscope}%
\begin{pgfscope}%
\pgfpathrectangle{\pgfqpoint{0.946717in}{3.073635in}}{\pgfqpoint{1.826010in}{1.541693in}}%
\pgfusepath{clip}%
\pgfsetbuttcap%
\pgfsetroundjoin%
\pgfsetlinewidth{1.505625pt}%
\definecolor{currentstroke}{rgb}{1.000000,0.647059,0.000000}%
\pgfsetstrokecolor{currentstroke}%
\pgfsetdash{{1.500000pt}{2.475000pt}}{0.000000pt}%
\pgfpathmoveto{\pgfqpoint{1.788433in}{3.073635in}}%
\pgfpathlineto{\pgfqpoint{1.788433in}{4.615329in}}%
\pgfusepath{stroke}%
\end{pgfscope}%
\begin{pgfscope}%
\pgfpathrectangle{\pgfqpoint{0.946717in}{3.073635in}}{\pgfqpoint{1.826010in}{1.541693in}}%
\pgfusepath{clip}%
\pgfsetbuttcap%
\pgfsetroundjoin%
\pgfsetlinewidth{1.505625pt}%
\definecolor{currentstroke}{rgb}{1.000000,0.647059,0.000000}%
\pgfsetstrokecolor{currentstroke}%
\pgfsetdash{{1.500000pt}{2.475000pt}}{0.000000pt}%
\pgfpathmoveto{\pgfqpoint{2.094975in}{3.073635in}}%
\pgfpathlineto{\pgfqpoint{2.094975in}{4.615329in}}%
\pgfusepath{stroke}%
\end{pgfscope}%
\begin{pgfscope}%
\pgfpathrectangle{\pgfqpoint{0.946717in}{3.073635in}}{\pgfqpoint{1.826010in}{1.541693in}}%
\pgfusepath{clip}%
\pgfsetbuttcap%
\pgfsetroundjoin%
\pgfsetlinewidth{1.505625pt}%
\definecolor{currentstroke}{rgb}{1.000000,0.647059,0.000000}%
\pgfsetstrokecolor{currentstroke}%
\pgfsetdash{{1.500000pt}{2.475000pt}}{0.000000pt}%
\pgfpathmoveto{\pgfqpoint{2.400498in}{3.073635in}}%
\pgfpathlineto{\pgfqpoint{2.400498in}{4.615329in}}%
\pgfusepath{stroke}%
\end{pgfscope}%
\begin{pgfscope}%
\pgfpathrectangle{\pgfqpoint{0.946717in}{3.073635in}}{\pgfqpoint{1.826010in}{1.541693in}}%
\pgfusepath{clip}%
\pgfsetroundcap%
\pgfsetroundjoin%
\pgfsetlinewidth{1.505625pt}%
\definecolor{currentstroke}{rgb}{0.298039,0.447059,0.690196}%
\pgfsetstrokecolor{currentstroke}%
\pgfsetdash{}{0pt}%
\pgfpathmoveto{\pgfqpoint{1.029717in}{3.143712in}}%
\pgfpathlineto{\pgfqpoint{1.345424in}{3.143712in}}%
\pgfpathlineto{\pgfqpoint{1.347971in}{3.158604in}}%
\pgfpathlineto{\pgfqpoint{1.355609in}{3.158604in}}%
\pgfpathlineto{\pgfqpoint{1.358155in}{3.230184in}}%
\pgfpathlineto{\pgfqpoint{1.365793in}{3.230184in}}%
\pgfpathlineto{\pgfqpoint{1.368339in}{3.325679in}}%
\pgfpathlineto{\pgfqpoint{1.396345in}{3.326570in}}%
\pgfpathlineto{\pgfqpoint{1.398891in}{3.328107in}}%
\pgfpathlineto{\pgfqpoint{1.406529in}{3.328107in}}%
\pgfpathlineto{\pgfqpoint{1.409075in}{3.441047in}}%
\pgfpathlineto{\pgfqpoint{1.426897in}{3.441047in}}%
\pgfpathlineto{\pgfqpoint{1.429443in}{3.704511in}}%
\pgfpathlineto{\pgfqpoint{1.437081in}{3.704511in}}%
\pgfpathlineto{\pgfqpoint{1.439627in}{3.904847in}}%
\pgfpathlineto{\pgfqpoint{1.447266in}{3.904847in}}%
\pgfpathlineto{\pgfqpoint{1.449812in}{4.022609in}}%
\pgfpathlineto{\pgfqpoint{1.457450in}{4.022609in}}%
\pgfpathlineto{\pgfqpoint{1.459996in}{4.175209in}}%
\pgfpathlineto{\pgfqpoint{1.467634in}{4.175209in}}%
\pgfpathlineto{\pgfqpoint{1.470180in}{4.260429in}}%
\pgfpathlineto{\pgfqpoint{1.477818in}{4.260429in}}%
\pgfpathlineto{\pgfqpoint{1.480364in}{4.384592in}}%
\pgfpathlineto{\pgfqpoint{1.488002in}{4.384592in}}%
\pgfpathlineto{\pgfqpoint{1.490548in}{4.545252in}}%
\pgfpathlineto{\pgfqpoint{1.498186in}{4.545252in}}%
\pgfpathlineto{\pgfqpoint{1.500732in}{4.481242in}}%
\pgfpathlineto{\pgfqpoint{1.508370in}{4.481242in}}%
\pgfpathlineto{\pgfqpoint{1.510916in}{4.367637in}}%
\pgfpathlineto{\pgfqpoint{1.518554in}{4.367637in}}%
\pgfpathlineto{\pgfqpoint{1.521100in}{4.133257in}}%
\pgfpathlineto{\pgfqpoint{1.528738in}{4.133257in}}%
\pgfpathlineto{\pgfqpoint{1.531284in}{4.153455in}}%
\pgfpathlineto{\pgfqpoint{1.538923in}{4.153455in}}%
\pgfpathlineto{\pgfqpoint{1.541469in}{4.102588in}}%
\pgfpathlineto{\pgfqpoint{1.549107in}{4.102588in}}%
\pgfpathlineto{\pgfqpoint{1.551653in}{3.980361in}}%
\pgfpathlineto{\pgfqpoint{1.559291in}{3.980361in}}%
\pgfpathlineto{\pgfqpoint{1.561837in}{3.940022in}}%
\pgfpathlineto{\pgfqpoint{1.569475in}{3.940022in}}%
\pgfpathlineto{\pgfqpoint{1.572021in}{3.782894in}}%
\pgfpathlineto{\pgfqpoint{1.579659in}{3.782894in}}%
\pgfpathlineto{\pgfqpoint{1.582205in}{3.611803in}}%
\pgfpathlineto{\pgfqpoint{1.589843in}{3.611803in}}%
\pgfpathlineto{\pgfqpoint{1.592389in}{3.589019in}}%
\pgfpathlineto{\pgfqpoint{1.600027in}{3.589019in}}%
\pgfpathlineto{\pgfqpoint{1.602573in}{3.409127in}}%
\pgfpathlineto{\pgfqpoint{1.610211in}{3.409127in}}%
\pgfpathlineto{\pgfqpoint{1.612757in}{3.188053in}}%
\pgfpathlineto{\pgfqpoint{1.620395in}{3.188053in}}%
\pgfpathlineto{\pgfqpoint{1.622941in}{3.184504in}}%
\pgfpathlineto{\pgfqpoint{1.630579in}{3.184504in}}%
\pgfpathlineto{\pgfqpoint{1.633126in}{3.143713in}}%
\pgfpathlineto{\pgfqpoint{1.834262in}{3.143712in}}%
\pgfpathlineto{\pgfqpoint{1.836808in}{3.150334in}}%
\pgfpathlineto{\pgfqpoint{1.844446in}{3.150334in}}%
\pgfpathlineto{\pgfqpoint{1.846992in}{3.169542in}}%
\pgfpathlineto{\pgfqpoint{1.854630in}{3.169542in}}%
\pgfpathlineto{\pgfqpoint{1.857176in}{3.193126in}}%
\pgfpathlineto{\pgfqpoint{1.864814in}{3.193126in}}%
\pgfpathlineto{\pgfqpoint{1.867360in}{3.252127in}}%
\pgfpathlineto{\pgfqpoint{1.874998in}{3.252127in}}%
\pgfpathlineto{\pgfqpoint{1.877544in}{3.294064in}}%
\pgfpathlineto{\pgfqpoint{1.885182in}{3.294064in}}%
\pgfpathlineto{\pgfqpoint{1.887728in}{3.335186in}}%
\pgfpathlineto{\pgfqpoint{1.895366in}{3.335186in}}%
\pgfpathlineto{\pgfqpoint{1.897912in}{3.429337in}}%
\pgfpathlineto{\pgfqpoint{1.905550in}{3.429337in}}%
\pgfpathlineto{\pgfqpoint{1.908096in}{3.490808in}}%
\pgfpathlineto{\pgfqpoint{1.915735in}{3.490808in}}%
\pgfpathlineto{\pgfqpoint{1.918281in}{3.486618in}}%
\pgfpathlineto{\pgfqpoint{1.925919in}{3.486618in}}%
\pgfpathlineto{\pgfqpoint{1.928465in}{3.591193in}}%
\pgfpathlineto{\pgfqpoint{1.936103in}{3.591193in}}%
\pgfpathlineto{\pgfqpoint{1.938649in}{3.694351in}}%
\pgfpathlineto{\pgfqpoint{1.946287in}{3.694351in}}%
\pgfpathlineto{\pgfqpoint{1.948833in}{3.687280in}}%
\pgfpathlineto{\pgfqpoint{1.956471in}{3.687280in}}%
\pgfpathlineto{\pgfqpoint{1.959017in}{3.745061in}}%
\pgfpathlineto{\pgfqpoint{1.966655in}{3.745061in}}%
\pgfpathlineto{\pgfqpoint{1.969201in}{3.777427in}}%
\pgfpathlineto{\pgfqpoint{1.976839in}{3.777427in}}%
\pgfpathlineto{\pgfqpoint{1.979385in}{3.826823in}}%
\pgfpathlineto{\pgfqpoint{1.987023in}{3.826823in}}%
\pgfpathlineto{\pgfqpoint{1.989569in}{3.976143in}}%
\pgfpathlineto{\pgfqpoint{1.997207in}{3.976143in}}%
\pgfpathlineto{\pgfqpoint{1.999753in}{4.001743in}}%
\pgfpathlineto{\pgfqpoint{2.007391in}{4.001743in}}%
\pgfpathlineto{\pgfqpoint{2.009937in}{3.962403in}}%
\pgfpathlineto{\pgfqpoint{2.017576in}{3.962403in}}%
\pgfpathlineto{\pgfqpoint{2.020122in}{4.044956in}}%
\pgfpathlineto{\pgfqpoint{2.027760in}{4.044956in}}%
\pgfpathlineto{\pgfqpoint{2.030306in}{4.055442in}}%
\pgfpathlineto{\pgfqpoint{2.037944in}{4.055442in}}%
\pgfpathlineto{\pgfqpoint{2.040490in}{4.012724in}}%
\pgfpathlineto{\pgfqpoint{2.048128in}{4.012724in}}%
\pgfpathlineto{\pgfqpoint{2.050674in}{4.065681in}}%
\pgfpathlineto{\pgfqpoint{2.058312in}{4.065681in}}%
\pgfpathlineto{\pgfqpoint{2.060858in}{4.069955in}}%
\pgfpathlineto{\pgfqpoint{2.068496in}{4.069955in}}%
\pgfpathlineto{\pgfqpoint{2.071042in}{4.028993in}}%
\pgfpathlineto{\pgfqpoint{2.078680in}{4.028993in}}%
\pgfpathlineto{\pgfqpoint{2.081226in}{4.065725in}}%
\pgfpathlineto{\pgfqpoint{2.088864in}{4.065725in}}%
\pgfpathlineto{\pgfqpoint{2.091410in}{3.992072in}}%
\pgfpathlineto{\pgfqpoint{2.099048in}{3.992072in}}%
\pgfpathlineto{\pgfqpoint{2.101594in}{4.012084in}}%
\pgfpathlineto{\pgfqpoint{2.109233in}{4.012084in}}%
\pgfpathlineto{\pgfqpoint{2.111779in}{3.944202in}}%
\pgfpathlineto{\pgfqpoint{2.119417in}{3.944202in}}%
\pgfpathlineto{\pgfqpoint{2.121963in}{3.860907in}}%
\pgfpathlineto{\pgfqpoint{2.129601in}{3.860907in}}%
\pgfpathlineto{\pgfqpoint{2.132147in}{3.806081in}}%
\pgfpathlineto{\pgfqpoint{2.139785in}{3.806081in}}%
\pgfpathlineto{\pgfqpoint{2.142331in}{3.733038in}}%
\pgfpathlineto{\pgfqpoint{2.149969in}{3.733038in}}%
\pgfpathlineto{\pgfqpoint{2.152515in}{3.636018in}}%
\pgfpathlineto{\pgfqpoint{2.160153in}{3.636018in}}%
\pgfpathlineto{\pgfqpoint{2.162699in}{3.577450in}}%
\pgfpathlineto{\pgfqpoint{2.170337in}{3.577450in}}%
\pgfpathlineto{\pgfqpoint{2.172883in}{3.384412in}}%
\pgfpathlineto{\pgfqpoint{2.180521in}{3.384412in}}%
\pgfpathlineto{\pgfqpoint{2.183067in}{3.317805in}}%
\pgfpathlineto{\pgfqpoint{2.190705in}{3.317805in}}%
\pgfpathlineto{\pgfqpoint{2.193251in}{3.284332in}}%
\pgfpathlineto{\pgfqpoint{2.200890in}{3.284332in}}%
\pgfpathlineto{\pgfqpoint{2.203436in}{3.246485in}}%
\pgfpathlineto{\pgfqpoint{2.211074in}{3.246485in}}%
\pgfpathlineto{\pgfqpoint{2.213620in}{3.186206in}}%
\pgfpathlineto{\pgfqpoint{2.221258in}{3.186206in}}%
\pgfpathlineto{\pgfqpoint{2.223804in}{3.160619in}}%
\pgfpathlineto{\pgfqpoint{2.231442in}{3.160619in}}%
\pgfpathlineto{\pgfqpoint{2.233988in}{3.143713in}}%
\pgfpathlineto{\pgfqpoint{2.292546in}{3.143713in}}%
\pgfpathlineto{\pgfqpoint{2.295093in}{3.155726in}}%
\pgfpathlineto{\pgfqpoint{2.302731in}{3.155726in}}%
\pgfpathlineto{\pgfqpoint{2.305277in}{3.172750in}}%
\pgfpathlineto{\pgfqpoint{2.312915in}{3.172750in}}%
\pgfpathlineto{\pgfqpoint{2.315461in}{3.209737in}}%
\pgfpathlineto{\pgfqpoint{2.323099in}{3.209737in}}%
\pgfpathlineto{\pgfqpoint{2.325645in}{3.273511in}}%
\pgfpathlineto{\pgfqpoint{2.333283in}{3.273511in}}%
\pgfpathlineto{\pgfqpoint{2.335829in}{3.327104in}}%
\pgfpathlineto{\pgfqpoint{2.343467in}{3.327104in}}%
\pgfpathlineto{\pgfqpoint{2.346013in}{3.357443in}}%
\pgfpathlineto{\pgfqpoint{2.353651in}{3.357443in}}%
\pgfpathlineto{\pgfqpoint{2.356197in}{3.430893in}}%
\pgfpathlineto{\pgfqpoint{2.363835in}{3.430893in}}%
\pgfpathlineto{\pgfqpoint{2.366381in}{3.483452in}}%
\pgfpathlineto{\pgfqpoint{2.374019in}{3.483452in}}%
\pgfpathlineto{\pgfqpoint{2.376565in}{3.517672in}}%
\pgfpathlineto{\pgfqpoint{2.384203in}{3.517672in}}%
\pgfpathlineto{\pgfqpoint{2.386749in}{3.576414in}}%
\pgfpathlineto{\pgfqpoint{2.394388in}{3.576414in}}%
\pgfpathlineto{\pgfqpoint{2.396934in}{3.607149in}}%
\pgfpathlineto{\pgfqpoint{2.404572in}{3.607149in}}%
\pgfpathlineto{\pgfqpoint{2.407118in}{3.629033in}}%
\pgfpathlineto{\pgfqpoint{2.414756in}{3.629033in}}%
\pgfpathlineto{\pgfqpoint{2.417302in}{3.667188in}}%
\pgfpathlineto{\pgfqpoint{2.424940in}{3.667188in}}%
\pgfpathlineto{\pgfqpoint{2.427486in}{3.714440in}}%
\pgfpathlineto{\pgfqpoint{2.435124in}{3.714440in}}%
\pgfpathlineto{\pgfqpoint{2.437670in}{3.764119in}}%
\pgfpathlineto{\pgfqpoint{2.445308in}{3.764119in}}%
\pgfpathlineto{\pgfqpoint{2.447854in}{3.784660in}}%
\pgfpathlineto{\pgfqpoint{2.455492in}{3.784660in}}%
\pgfpathlineto{\pgfqpoint{2.458038in}{3.776924in}}%
\pgfpathlineto{\pgfqpoint{2.465676in}{3.776924in}}%
\pgfpathlineto{\pgfqpoint{2.468222in}{3.766661in}}%
\pgfpathlineto{\pgfqpoint{2.475860in}{3.766661in}}%
\pgfpathlineto{\pgfqpoint{2.478406in}{3.775350in}}%
\pgfpathlineto{\pgfqpoint{2.486045in}{3.775350in}}%
\pgfpathlineto{\pgfqpoint{2.488591in}{3.731102in}}%
\pgfpathlineto{\pgfqpoint{2.496229in}{3.731102in}}%
\pgfpathlineto{\pgfqpoint{2.498775in}{3.709295in}}%
\pgfpathlineto{\pgfqpoint{2.506413in}{3.709295in}}%
\pgfpathlineto{\pgfqpoint{2.508959in}{3.651785in}}%
\pgfpathlineto{\pgfqpoint{2.516597in}{3.651785in}}%
\pgfpathlineto{\pgfqpoint{2.519143in}{3.589265in}}%
\pgfpathlineto{\pgfqpoint{2.526781in}{3.589265in}}%
\pgfpathlineto{\pgfqpoint{2.529327in}{3.563465in}}%
\pgfpathlineto{\pgfqpoint{2.536965in}{3.563465in}}%
\pgfpathlineto{\pgfqpoint{2.539511in}{3.515025in}}%
\pgfpathlineto{\pgfqpoint{2.547149in}{3.515025in}}%
\pgfpathlineto{\pgfqpoint{2.549695in}{3.433818in}}%
\pgfpathlineto{\pgfqpoint{2.557333in}{3.433818in}}%
\pgfpathlineto{\pgfqpoint{2.559879in}{3.410842in}}%
\pgfpathlineto{\pgfqpoint{2.567517in}{3.410842in}}%
\pgfpathlineto{\pgfqpoint{2.570063in}{3.380870in}}%
\pgfpathlineto{\pgfqpoint{2.577701in}{3.380870in}}%
\pgfpathlineto{\pgfqpoint{2.580248in}{3.313050in}}%
\pgfpathlineto{\pgfqpoint{2.587886in}{3.313050in}}%
\pgfpathlineto{\pgfqpoint{2.590432in}{3.286796in}}%
\pgfpathlineto{\pgfqpoint{2.598070in}{3.286796in}}%
\pgfpathlineto{\pgfqpoint{2.600616in}{3.240662in}}%
\pgfpathlineto{\pgfqpoint{2.608254in}{3.240662in}}%
\pgfpathlineto{\pgfqpoint{2.610800in}{3.175896in}}%
\pgfpathlineto{\pgfqpoint{2.618438in}{3.175896in}}%
\pgfpathlineto{\pgfqpoint{2.620984in}{3.158698in}}%
\pgfpathlineto{\pgfqpoint{2.628622in}{3.158698in}}%
\pgfpathlineto{\pgfqpoint{2.631168in}{3.143712in}}%
\pgfpathlineto{\pgfqpoint{2.689727in}{3.143712in}}%
\pgfpathlineto{\pgfqpoint{2.689727in}{3.143712in}}%
\pgfusepath{stroke}%
\end{pgfscope}%
\begin{pgfscope}%
\pgfsetrectcap%
\pgfsetmiterjoin%
\pgfsetlinewidth{1.254687pt}%
\definecolor{currentstroke}{rgb}{1.000000,1.000000,1.000000}%
\pgfsetstrokecolor{currentstroke}%
\pgfsetdash{}{0pt}%
\pgfpathmoveto{\pgfqpoint{0.946717in}{3.073635in}}%
\pgfpathlineto{\pgfqpoint{0.946717in}{4.615329in}}%
\pgfusepath{stroke}%
\end{pgfscope}%
\begin{pgfscope}%
\pgfsetrectcap%
\pgfsetmiterjoin%
\pgfsetlinewidth{1.254687pt}%
\definecolor{currentstroke}{rgb}{1.000000,1.000000,1.000000}%
\pgfsetstrokecolor{currentstroke}%
\pgfsetdash{}{0pt}%
\pgfpathmoveto{\pgfqpoint{2.772727in}{3.073635in}}%
\pgfpathlineto{\pgfqpoint{2.772727in}{4.615329in}}%
\pgfusepath{stroke}%
\end{pgfscope}%
\begin{pgfscope}%
\pgfsetrectcap%
\pgfsetmiterjoin%
\pgfsetlinewidth{1.254687pt}%
\definecolor{currentstroke}{rgb}{1.000000,1.000000,1.000000}%
\pgfsetstrokecolor{currentstroke}%
\pgfsetdash{}{0pt}%
\pgfpathmoveto{\pgfqpoint{0.946717in}{3.073635in}}%
\pgfpathlineto{\pgfqpoint{2.772727in}{3.073635in}}%
\pgfusepath{stroke}%
\end{pgfscope}%
\begin{pgfscope}%
\pgfsetrectcap%
\pgfsetmiterjoin%
\pgfsetlinewidth{1.254687pt}%
\definecolor{currentstroke}{rgb}{1.000000,1.000000,1.000000}%
\pgfsetstrokecolor{currentstroke}%
\pgfsetdash{}{0pt}%
\pgfpathmoveto{\pgfqpoint{0.946717in}{4.615329in}}%
\pgfpathlineto{\pgfqpoint{2.772727in}{4.615329in}}%
\pgfusepath{stroke}%
\end{pgfscope}%
\begin{pgfscope}%
\definecolor{textcolor}{rgb}{0.150000,0.150000,0.150000}%
\pgfsetstrokecolor{textcolor}%
\pgfsetfillcolor{textcolor}%
\pgftext[x=1.859722in,y=4.698662in,,base]{\color{textcolor}{\sffamily\fontsize{12.000000}{14.400000}\selectfont\catcode`\^=\active\def^{\ifmmode\sp\else\^{}\fi}\catcode`\%=\active\def%{\%}a)}}%
\end{pgfscope}%
\begin{pgfscope}%
\pgfsetbuttcap%
\pgfsetmiterjoin%
\definecolor{currentfill}{rgb}{0.917647,0.917647,0.949020}%
\pgfsetfillcolor{currentfill}%
\pgfsetlinewidth{0.000000pt}%
\definecolor{currentstroke}{rgb}{0.000000,0.000000,0.000000}%
\pgfsetstrokecolor{currentstroke}%
\pgfsetstrokeopacity{0.000000}%
\pgfsetdash{}{0pt}%
\pgfpathmoveto{\pgfqpoint{3.593990in}{3.073635in}}%
\pgfpathlineto{\pgfqpoint{5.420000in}{3.073635in}}%
\pgfpathlineto{\pgfqpoint{5.420000in}{4.615329in}}%
\pgfpathlineto{\pgfqpoint{3.593990in}{4.615329in}}%
\pgfpathlineto{\pgfqpoint{3.593990in}{3.073635in}}%
\pgfpathclose%
\pgfusepath{fill}%
\end{pgfscope}%
\begin{pgfscope}%
\pgfpathrectangle{\pgfqpoint{3.593990in}{3.073635in}}{\pgfqpoint{1.826010in}{1.541693in}}%
\pgfusepath{clip}%
\pgfsetroundcap%
\pgfsetroundjoin%
\pgfsetlinewidth{1.003750pt}%
\definecolor{currentstroke}{rgb}{1.000000,1.000000,1.000000}%
\pgfsetstrokecolor{currentstroke}%
\pgfsetdash{}{0pt}%
\pgfpathmoveto{\pgfqpoint{3.676990in}{3.073635in}}%
\pgfpathlineto{\pgfqpoint{3.676990in}{4.615329in}}%
\pgfusepath{stroke}%
\end{pgfscope}%
\begin{pgfscope}%
\definecolor{textcolor}{rgb}{0.150000,0.150000,0.150000}%
\pgfsetstrokecolor{textcolor}%
\pgfsetfillcolor{textcolor}%
\pgftext[x=3.676990in,y=2.941691in,,top]{\color{textcolor}{\sffamily\fontsize{11.000000}{13.200000}\selectfont\catcode`\^=\active\def^{\ifmmode\sp\else\^{}\fi}\catcode`\%=\active\def%{\%}0}}%
\end{pgfscope}%
\begin{pgfscope}%
\pgfpathrectangle{\pgfqpoint{3.593990in}{3.073635in}}{\pgfqpoint{1.826010in}{1.541693in}}%
\pgfusepath{clip}%
\pgfsetroundcap%
\pgfsetroundjoin%
\pgfsetlinewidth{1.003750pt}%
\definecolor{currentstroke}{rgb}{1.000000,1.000000,1.000000}%
\pgfsetstrokecolor{currentstroke}%
\pgfsetdash{}{0pt}%
\pgfpathmoveto{\pgfqpoint{4.683056in}{3.073635in}}%
\pgfpathlineto{\pgfqpoint{4.683056in}{4.615329in}}%
\pgfusepath{stroke}%
\end{pgfscope}%
\begin{pgfscope}%
\definecolor{textcolor}{rgb}{0.150000,0.150000,0.150000}%
\pgfsetstrokecolor{textcolor}%
\pgfsetfillcolor{textcolor}%
\pgftext[x=4.683056in,y=2.941691in,,top]{\color{textcolor}{\sffamily\fontsize{11.000000}{13.200000}\selectfont\catcode`\^=\active\def^{\ifmmode\sp\else\^{}\fi}\catcode`\%=\active\def%{\%}2000}}%
\end{pgfscope}%
\begin{pgfscope}%
\definecolor{textcolor}{rgb}{0.150000,0.150000,0.150000}%
\pgfsetstrokecolor{textcolor}%
\pgfsetfillcolor{textcolor}%
\pgftext[x=4.506995in,y=2.746413in,,top]{\color{textcolor}{\sffamily\fontsize{12.000000}{14.400000}\selectfont\catcode`\^=\active\def^{\ifmmode\sp\else\^{}\fi}\catcode`\%=\active\def%{\%}Time (s)}}%
\end{pgfscope}%
\begin{pgfscope}%
\pgfpathrectangle{\pgfqpoint{3.593990in}{3.073635in}}{\pgfqpoint{1.826010in}{1.541693in}}%
\pgfusepath{clip}%
\pgfsetroundcap%
\pgfsetroundjoin%
\pgfsetlinewidth{1.003750pt}%
\definecolor{currentstroke}{rgb}{1.000000,1.000000,1.000000}%
\pgfsetstrokecolor{currentstroke}%
\pgfsetdash{}{0pt}%
\pgfpathmoveto{\pgfqpoint{3.593990in}{3.330584in}}%
\pgfpathlineto{\pgfqpoint{5.420000in}{3.330584in}}%
\pgfusepath{stroke}%
\end{pgfscope}%
\begin{pgfscope}%
\definecolor{textcolor}{rgb}{0.150000,0.150000,0.150000}%
\pgfsetstrokecolor{textcolor}%
\pgfsetfillcolor{textcolor}%
\pgftext[x=3.377077in, y=3.275904in, left, base]{\color{textcolor}{\sffamily\fontsize{11.000000}{13.200000}\selectfont\catcode`\^=\active\def^{\ifmmode\sp\else\^{}\fi}\catcode`\%=\active\def%{\%}1}}%
\end{pgfscope}%
\begin{pgfscope}%
\pgfpathrectangle{\pgfqpoint{3.593990in}{3.073635in}}{\pgfqpoint{1.826010in}{1.541693in}}%
\pgfusepath{clip}%
\pgfsetroundcap%
\pgfsetroundjoin%
\pgfsetlinewidth{1.003750pt}%
\definecolor{currentstroke}{rgb}{1.000000,1.000000,1.000000}%
\pgfsetstrokecolor{currentstroke}%
\pgfsetdash{}{0pt}%
\pgfpathmoveto{\pgfqpoint{3.593990in}{3.844482in}}%
\pgfpathlineto{\pgfqpoint{5.420000in}{3.844482in}}%
\pgfusepath{stroke}%
\end{pgfscope}%
\begin{pgfscope}%
\definecolor{textcolor}{rgb}{0.150000,0.150000,0.150000}%
\pgfsetstrokecolor{textcolor}%
\pgfsetfillcolor{textcolor}%
\pgftext[x=3.377077in, y=3.789801in, left, base]{\color{textcolor}{\sffamily\fontsize{11.000000}{13.200000}\selectfont\catcode`\^=\active\def^{\ifmmode\sp\else\^{}\fi}\catcode`\%=\active\def%{\%}2}}%
\end{pgfscope}%
\begin{pgfscope}%
\pgfpathrectangle{\pgfqpoint{3.593990in}{3.073635in}}{\pgfqpoint{1.826010in}{1.541693in}}%
\pgfusepath{clip}%
\pgfsetroundcap%
\pgfsetroundjoin%
\pgfsetlinewidth{1.003750pt}%
\definecolor{currentstroke}{rgb}{1.000000,1.000000,1.000000}%
\pgfsetstrokecolor{currentstroke}%
\pgfsetdash{}{0pt}%
\pgfpathmoveto{\pgfqpoint{3.593990in}{4.358380in}}%
\pgfpathlineto{\pgfqpoint{5.420000in}{4.358380in}}%
\pgfusepath{stroke}%
\end{pgfscope}%
\begin{pgfscope}%
\definecolor{textcolor}{rgb}{0.150000,0.150000,0.150000}%
\pgfsetstrokecolor{textcolor}%
\pgfsetfillcolor{textcolor}%
\pgftext[x=3.377077in, y=4.303699in, left, base]{\color{textcolor}{\sffamily\fontsize{11.000000}{13.200000}\selectfont\catcode`\^=\active\def^{\ifmmode\sp\else\^{}\fi}\catcode`\%=\active\def%{\%}3}}%
\end{pgfscope}%
\begin{pgfscope}%
\definecolor{textcolor}{rgb}{0.150000,0.150000,0.150000}%
\pgfsetstrokecolor{textcolor}%
\pgfsetfillcolor{textcolor}%
\pgftext[x=3.321522in,y=3.844482in,,bottom,rotate=90.000000]{\color{textcolor}{\sffamily\fontsize{12.000000}{14.400000}\selectfont\catcode`\^=\active\def^{\ifmmode\sp\else\^{}\fi}\catcode`\%=\active\def%{\%}\# nodes}}%
\end{pgfscope}%
\begin{pgfscope}%
\pgfpathrectangle{\pgfqpoint{3.593990in}{3.073635in}}{\pgfqpoint{1.826010in}{1.541693in}}%
\pgfusepath{clip}%
\pgfsetbuttcap%
\pgfsetroundjoin%
\pgfsetlinewidth{1.505625pt}%
\definecolor{currentstroke}{rgb}{1.000000,0.647059,0.000000}%
\pgfsetstrokecolor{currentstroke}%
\pgfsetdash{{1.500000pt}{2.475000pt}}{0.000000pt}%
\pgfpathmoveto{\pgfqpoint{3.819851in}{3.073635in}}%
\pgfpathlineto{\pgfqpoint{3.819851in}{4.615329in}}%
\pgfusepath{stroke}%
\end{pgfscope}%
\begin{pgfscope}%
\pgfpathrectangle{\pgfqpoint{3.593990in}{3.073635in}}{\pgfqpoint{1.826010in}{1.541693in}}%
\pgfusepath{clip}%
\pgfsetbuttcap%
\pgfsetroundjoin%
\pgfsetlinewidth{1.505625pt}%
\definecolor{currentstroke}{rgb}{1.000000,0.647059,0.000000}%
\pgfsetstrokecolor{currentstroke}%
\pgfsetdash{{1.500000pt}{2.475000pt}}{0.000000pt}%
\pgfpathmoveto{\pgfqpoint{4.124186in}{3.073635in}}%
\pgfpathlineto{\pgfqpoint{4.124186in}{4.615329in}}%
\pgfusepath{stroke}%
\end{pgfscope}%
\begin{pgfscope}%
\pgfpathrectangle{\pgfqpoint{3.593990in}{3.073635in}}{\pgfqpoint{1.826010in}{1.541693in}}%
\pgfusepath{clip}%
\pgfsetbuttcap%
\pgfsetroundjoin%
\pgfsetlinewidth{1.505625pt}%
\definecolor{currentstroke}{rgb}{1.000000,0.647059,0.000000}%
\pgfsetstrokecolor{currentstroke}%
\pgfsetdash{{1.500000pt}{2.475000pt}}{0.000000pt}%
\pgfpathmoveto{\pgfqpoint{4.426509in}{3.073635in}}%
\pgfpathlineto{\pgfqpoint{4.426509in}{4.615329in}}%
\pgfusepath{stroke}%
\end{pgfscope}%
\begin{pgfscope}%
\pgfpathrectangle{\pgfqpoint{3.593990in}{3.073635in}}{\pgfqpoint{1.826010in}{1.541693in}}%
\pgfusepath{clip}%
\pgfsetbuttcap%
\pgfsetroundjoin%
\pgfsetlinewidth{1.505625pt}%
\definecolor{currentstroke}{rgb}{1.000000,0.647059,0.000000}%
\pgfsetstrokecolor{currentstroke}%
\pgfsetdash{{1.500000pt}{2.475000pt}}{0.000000pt}%
\pgfpathmoveto{\pgfqpoint{4.729335in}{3.073635in}}%
\pgfpathlineto{\pgfqpoint{4.729335in}{4.615329in}}%
\pgfusepath{stroke}%
\end{pgfscope}%
\begin{pgfscope}%
\pgfpathrectangle{\pgfqpoint{3.593990in}{3.073635in}}{\pgfqpoint{1.826010in}{1.541693in}}%
\pgfusepath{clip}%
\pgfsetbuttcap%
\pgfsetroundjoin%
\pgfsetlinewidth{1.505625pt}%
\definecolor{currentstroke}{rgb}{1.000000,0.647059,0.000000}%
\pgfsetstrokecolor{currentstroke}%
\pgfsetdash{{1.500000pt}{2.475000pt}}{0.000000pt}%
\pgfpathmoveto{\pgfqpoint{5.031155in}{3.073635in}}%
\pgfpathlineto{\pgfqpoint{5.031155in}{4.615329in}}%
\pgfusepath{stroke}%
\end{pgfscope}%
\begin{pgfscope}%
\pgfpathrectangle{\pgfqpoint{3.593990in}{3.073635in}}{\pgfqpoint{1.826010in}{1.541693in}}%
\pgfusepath{clip}%
\pgfsetroundcap%
\pgfsetroundjoin%
\pgfsetlinewidth{1.505625pt}%
\definecolor{currentstroke}{rgb}{0.298039,0.447059,0.690196}%
\pgfsetstrokecolor{currentstroke}%
\pgfsetdash{}{0pt}%
\pgfpathmoveto{\pgfqpoint{3.676990in}{3.330584in}}%
\pgfpathlineto{\pgfqpoint{4.237872in}{3.330584in}}%
\pgfpathlineto{\pgfqpoint{4.240387in}{3.844482in}}%
\pgfpathlineto{\pgfqpoint{4.806300in}{3.844482in}}%
\pgfpathlineto{\pgfqpoint{4.808815in}{4.358380in}}%
\pgfpathlineto{\pgfqpoint{5.337000in}{4.358380in}}%
\pgfpathlineto{\pgfqpoint{5.337000in}{4.358380in}}%
\pgfusepath{stroke}%
\end{pgfscope}%
\begin{pgfscope}%
\pgfsetrectcap%
\pgfsetmiterjoin%
\pgfsetlinewidth{1.254687pt}%
\definecolor{currentstroke}{rgb}{1.000000,1.000000,1.000000}%
\pgfsetstrokecolor{currentstroke}%
\pgfsetdash{}{0pt}%
\pgfpathmoveto{\pgfqpoint{3.593990in}{3.073635in}}%
\pgfpathlineto{\pgfqpoint{3.593990in}{4.615329in}}%
\pgfusepath{stroke}%
\end{pgfscope}%
\begin{pgfscope}%
\pgfsetrectcap%
\pgfsetmiterjoin%
\pgfsetlinewidth{1.254687pt}%
\definecolor{currentstroke}{rgb}{1.000000,1.000000,1.000000}%
\pgfsetstrokecolor{currentstroke}%
\pgfsetdash{}{0pt}%
\pgfpathmoveto{\pgfqpoint{5.420000in}{3.073635in}}%
\pgfpathlineto{\pgfqpoint{5.420000in}{4.615329in}}%
\pgfusepath{stroke}%
\end{pgfscope}%
\begin{pgfscope}%
\pgfsetrectcap%
\pgfsetmiterjoin%
\pgfsetlinewidth{1.254687pt}%
\definecolor{currentstroke}{rgb}{1.000000,1.000000,1.000000}%
\pgfsetstrokecolor{currentstroke}%
\pgfsetdash{}{0pt}%
\pgfpathmoveto{\pgfqpoint{3.593990in}{3.073635in}}%
\pgfpathlineto{\pgfqpoint{5.420000in}{3.073635in}}%
\pgfusepath{stroke}%
\end{pgfscope}%
\begin{pgfscope}%
\pgfsetrectcap%
\pgfsetmiterjoin%
\pgfsetlinewidth{1.254687pt}%
\definecolor{currentstroke}{rgb}{1.000000,1.000000,1.000000}%
\pgfsetstrokecolor{currentstroke}%
\pgfsetdash{}{0pt}%
\pgfpathmoveto{\pgfqpoint{3.593990in}{4.615329in}}%
\pgfpathlineto{\pgfqpoint{5.420000in}{4.615329in}}%
\pgfusepath{stroke}%
\end{pgfscope}%
\begin{pgfscope}%
\definecolor{textcolor}{rgb}{0.150000,0.150000,0.150000}%
\pgfsetstrokecolor{textcolor}%
\pgfsetfillcolor{textcolor}%
\pgftext[x=4.506995in,y=4.698662in,,base]{\color{textcolor}{\sffamily\fontsize{12.000000}{14.400000}\selectfont\catcode`\^=\active\def^{\ifmmode\sp\else\^{}\fi}\catcode`\%=\active\def%{\%}b)}}%
\end{pgfscope}%
\begin{pgfscope}%
\pgfsetbuttcap%
\pgfsetmiterjoin%
\definecolor{currentfill}{rgb}{0.917647,0.917647,0.949020}%
\pgfsetfillcolor{currentfill}%
\pgfsetlinewidth{0.000000pt}%
\definecolor{currentstroke}{rgb}{0.000000,0.000000,0.000000}%
\pgfsetstrokecolor{currentstroke}%
\pgfsetstrokeopacity{0.000000}%
\pgfsetdash{}{0pt}%
\pgfpathmoveto{\pgfqpoint{0.946717in}{0.663635in}}%
\pgfpathlineto{\pgfqpoint{2.772727in}{0.663635in}}%
\pgfpathlineto{\pgfqpoint{2.772727in}{2.205329in}}%
\pgfpathlineto{\pgfqpoint{0.946717in}{2.205329in}}%
\pgfpathlineto{\pgfqpoint{0.946717in}{0.663635in}}%
\pgfpathclose%
\pgfusepath{fill}%
\end{pgfscope}%
\begin{pgfscope}%
\pgfpathrectangle{\pgfqpoint{0.946717in}{0.663635in}}{\pgfqpoint{1.826010in}{1.541693in}}%
\pgfusepath{clip}%
\pgfsetroundcap%
\pgfsetroundjoin%
\pgfsetlinewidth{1.003750pt}%
\definecolor{currentstroke}{rgb}{1.000000,1.000000,1.000000}%
\pgfsetstrokecolor{currentstroke}%
\pgfsetdash{}{0pt}%
\pgfpathmoveto{\pgfqpoint{1.029717in}{0.663635in}}%
\pgfpathlineto{\pgfqpoint{1.029717in}{2.205329in}}%
\pgfusepath{stroke}%
\end{pgfscope}%
\begin{pgfscope}%
\definecolor{textcolor}{rgb}{0.150000,0.150000,0.150000}%
\pgfsetstrokecolor{textcolor}%
\pgfsetfillcolor{textcolor}%
\pgftext[x=1.029717in,y=0.531691in,,top]{\color{textcolor}{\sffamily\fontsize{11.000000}{13.200000}\selectfont\catcode`\^=\active\def^{\ifmmode\sp\else\^{}\fi}\catcode`\%=\active\def%{\%}0}}%
\end{pgfscope}%
\begin{pgfscope}%
\pgfpathrectangle{\pgfqpoint{0.946717in}{0.663635in}}{\pgfqpoint{1.826010in}{1.541693in}}%
\pgfusepath{clip}%
\pgfsetroundcap%
\pgfsetroundjoin%
\pgfsetlinewidth{1.003750pt}%
\definecolor{currentstroke}{rgb}{1.000000,1.000000,1.000000}%
\pgfsetstrokecolor{currentstroke}%
\pgfsetdash{}{0pt}%
\pgfpathmoveto{\pgfqpoint{2.048128in}{0.663635in}}%
\pgfpathlineto{\pgfqpoint{2.048128in}{2.205329in}}%
\pgfusepath{stroke}%
\end{pgfscope}%
\begin{pgfscope}%
\definecolor{textcolor}{rgb}{0.150000,0.150000,0.150000}%
\pgfsetstrokecolor{textcolor}%
\pgfsetfillcolor{textcolor}%
\pgftext[x=2.048128in,y=0.531691in,,top]{\color{textcolor}{\sffamily\fontsize{11.000000}{13.200000}\selectfont\catcode`\^=\active\def^{\ifmmode\sp\else\^{}\fi}\catcode`\%=\active\def%{\%}2000}}%
\end{pgfscope}%
\begin{pgfscope}%
\definecolor{textcolor}{rgb}{0.150000,0.150000,0.150000}%
\pgfsetstrokecolor{textcolor}%
\pgfsetfillcolor{textcolor}%
\pgftext[x=1.859722in,y=0.336413in,,top]{\color{textcolor}{\sffamily\fontsize{12.000000}{14.400000}\selectfont\catcode`\^=\active\def^{\ifmmode\sp\else\^{}\fi}\catcode`\%=\active\def%{\%}Time (s)}}%
\end{pgfscope}%
\begin{pgfscope}%
\pgfpathrectangle{\pgfqpoint{0.946717in}{0.663635in}}{\pgfqpoint{1.826010in}{1.541693in}}%
\pgfusepath{clip}%
\pgfsetroundcap%
\pgfsetroundjoin%
\pgfsetlinewidth{1.003750pt}%
\definecolor{currentstroke}{rgb}{1.000000,1.000000,1.000000}%
\pgfsetstrokecolor{currentstroke}%
\pgfsetdash{}{0pt}%
\pgfpathmoveto{\pgfqpoint{0.946717in}{0.792110in}}%
\pgfpathlineto{\pgfqpoint{2.772727in}{0.792110in}}%
\pgfusepath{stroke}%
\end{pgfscope}%
\begin{pgfscope}%
\definecolor{textcolor}{rgb}{0.150000,0.150000,0.150000}%
\pgfsetstrokecolor{textcolor}%
\pgfsetfillcolor{textcolor}%
\pgftext[x=0.517423in, y=0.737429in, left, base]{\color{textcolor}{\sffamily\fontsize{11.000000}{13.200000}\selectfont\catcode`\^=\active\def^{\ifmmode\sp\else\^{}\fi}\catcode`\%=\active\def%{\%}0.00}}%
\end{pgfscope}%
\begin{pgfscope}%
\pgfpathrectangle{\pgfqpoint{0.946717in}{0.663635in}}{\pgfqpoint{1.826010in}{1.541693in}}%
\pgfusepath{clip}%
\pgfsetroundcap%
\pgfsetroundjoin%
\pgfsetlinewidth{1.003750pt}%
\definecolor{currentstroke}{rgb}{1.000000,1.000000,1.000000}%
\pgfsetstrokecolor{currentstroke}%
\pgfsetdash{}{0pt}%
\pgfpathmoveto{\pgfqpoint{0.946717in}{1.113296in}}%
\pgfpathlineto{\pgfqpoint{2.772727in}{1.113296in}}%
\pgfusepath{stroke}%
\end{pgfscope}%
\begin{pgfscope}%
\definecolor{textcolor}{rgb}{0.150000,0.150000,0.150000}%
\pgfsetstrokecolor{textcolor}%
\pgfsetfillcolor{textcolor}%
\pgftext[x=0.517423in, y=1.058615in, left, base]{\color{textcolor}{\sffamily\fontsize{11.000000}{13.200000}\selectfont\catcode`\^=\active\def^{\ifmmode\sp\else\^{}\fi}\catcode`\%=\active\def%{\%}0.25}}%
\end{pgfscope}%
\begin{pgfscope}%
\pgfpathrectangle{\pgfqpoint{0.946717in}{0.663635in}}{\pgfqpoint{1.826010in}{1.541693in}}%
\pgfusepath{clip}%
\pgfsetroundcap%
\pgfsetroundjoin%
\pgfsetlinewidth{1.003750pt}%
\definecolor{currentstroke}{rgb}{1.000000,1.000000,1.000000}%
\pgfsetstrokecolor{currentstroke}%
\pgfsetdash{}{0pt}%
\pgfpathmoveto{\pgfqpoint{0.946717in}{1.434482in}}%
\pgfpathlineto{\pgfqpoint{2.772727in}{1.434482in}}%
\pgfusepath{stroke}%
\end{pgfscope}%
\begin{pgfscope}%
\definecolor{textcolor}{rgb}{0.150000,0.150000,0.150000}%
\pgfsetstrokecolor{textcolor}%
\pgfsetfillcolor{textcolor}%
\pgftext[x=0.517423in, y=1.379801in, left, base]{\color{textcolor}{\sffamily\fontsize{11.000000}{13.200000}\selectfont\catcode`\^=\active\def^{\ifmmode\sp\else\^{}\fi}\catcode`\%=\active\def%{\%}0.50}}%
\end{pgfscope}%
\begin{pgfscope}%
\pgfpathrectangle{\pgfqpoint{0.946717in}{0.663635in}}{\pgfqpoint{1.826010in}{1.541693in}}%
\pgfusepath{clip}%
\pgfsetroundcap%
\pgfsetroundjoin%
\pgfsetlinewidth{1.003750pt}%
\definecolor{currentstroke}{rgb}{1.000000,1.000000,1.000000}%
\pgfsetstrokecolor{currentstroke}%
\pgfsetdash{}{0pt}%
\pgfpathmoveto{\pgfqpoint{0.946717in}{1.755668in}}%
\pgfpathlineto{\pgfqpoint{2.772727in}{1.755668in}}%
\pgfusepath{stroke}%
\end{pgfscope}%
\begin{pgfscope}%
\definecolor{textcolor}{rgb}{0.150000,0.150000,0.150000}%
\pgfsetstrokecolor{textcolor}%
\pgfsetfillcolor{textcolor}%
\pgftext[x=0.517423in, y=1.700987in, left, base]{\color{textcolor}{\sffamily\fontsize{11.000000}{13.200000}\selectfont\catcode`\^=\active\def^{\ifmmode\sp\else\^{}\fi}\catcode`\%=\active\def%{\%}0.75}}%
\end{pgfscope}%
\begin{pgfscope}%
\pgfpathrectangle{\pgfqpoint{0.946717in}{0.663635in}}{\pgfqpoint{1.826010in}{1.541693in}}%
\pgfusepath{clip}%
\pgfsetroundcap%
\pgfsetroundjoin%
\pgfsetlinewidth{1.003750pt}%
\definecolor{currentstroke}{rgb}{1.000000,1.000000,1.000000}%
\pgfsetstrokecolor{currentstroke}%
\pgfsetdash{}{0pt}%
\pgfpathmoveto{\pgfqpoint{0.946717in}{2.076854in}}%
\pgfpathlineto{\pgfqpoint{2.772727in}{2.076854in}}%
\pgfusepath{stroke}%
\end{pgfscope}%
\begin{pgfscope}%
\definecolor{textcolor}{rgb}{0.150000,0.150000,0.150000}%
\pgfsetstrokecolor{textcolor}%
\pgfsetfillcolor{textcolor}%
\pgftext[x=0.517423in, y=2.022174in, left, base]{\color{textcolor}{\sffamily\fontsize{11.000000}{13.200000}\selectfont\catcode`\^=\active\def^{\ifmmode\sp\else\^{}\fi}\catcode`\%=\active\def%{\%}1.00}}%
\end{pgfscope}%
\begin{pgfscope}%
\definecolor{textcolor}{rgb}{0.150000,0.150000,0.150000}%
\pgfsetstrokecolor{textcolor}%
\pgfsetfillcolor{textcolor}%
\pgftext[x=0.461867in,y=1.434482in,,bottom,rotate=90.000000]{\color{textcolor}{\sffamily\fontsize{12.000000}{14.400000}\selectfont\catcode`\^=\active\def^{\ifmmode\sp\else\^{}\fi}\catcode`\%=\active\def%{\%}CPU Utilization (%)}}%
\end{pgfscope}%
\begin{pgfscope}%
\pgfpathrectangle{\pgfqpoint{0.946717in}{0.663635in}}{\pgfqpoint{1.826010in}{1.541693in}}%
\pgfusepath{clip}%
\pgfsetbuttcap%
\pgfsetroundjoin%
\pgfsetlinewidth{1.505625pt}%
\definecolor{currentstroke}{rgb}{1.000000,0.647059,0.000000}%
\pgfsetstrokecolor{currentstroke}%
\pgfsetdash{{1.500000pt}{2.475000pt}}{0.000000pt}%
\pgfpathmoveto{\pgfqpoint{1.174331in}{0.663635in}}%
\pgfpathlineto{\pgfqpoint{1.174331in}{2.205329in}}%
\pgfusepath{stroke}%
\end{pgfscope}%
\begin{pgfscope}%
\pgfpathrectangle{\pgfqpoint{0.946717in}{0.663635in}}{\pgfqpoint{1.826010in}{1.541693in}}%
\pgfusepath{clip}%
\pgfsetbuttcap%
\pgfsetroundjoin%
\pgfsetlinewidth{1.505625pt}%
\definecolor{currentstroke}{rgb}{1.000000,0.647059,0.000000}%
\pgfsetstrokecolor{currentstroke}%
\pgfsetdash{{1.500000pt}{2.475000pt}}{0.000000pt}%
\pgfpathmoveto{\pgfqpoint{1.482401in}{0.663635in}}%
\pgfpathlineto{\pgfqpoint{1.482401in}{2.205329in}}%
\pgfusepath{stroke}%
\end{pgfscope}%
\begin{pgfscope}%
\pgfpathrectangle{\pgfqpoint{0.946717in}{0.663635in}}{\pgfqpoint{1.826010in}{1.541693in}}%
\pgfusepath{clip}%
\pgfsetbuttcap%
\pgfsetroundjoin%
\pgfsetlinewidth{1.505625pt}%
\definecolor{currentstroke}{rgb}{1.000000,0.647059,0.000000}%
\pgfsetstrokecolor{currentstroke}%
\pgfsetdash{{1.500000pt}{2.475000pt}}{0.000000pt}%
\pgfpathmoveto{\pgfqpoint{1.788433in}{0.663635in}}%
\pgfpathlineto{\pgfqpoint{1.788433in}{2.205329in}}%
\pgfusepath{stroke}%
\end{pgfscope}%
\begin{pgfscope}%
\pgfpathrectangle{\pgfqpoint{0.946717in}{0.663635in}}{\pgfqpoint{1.826010in}{1.541693in}}%
\pgfusepath{clip}%
\pgfsetbuttcap%
\pgfsetroundjoin%
\pgfsetlinewidth{1.505625pt}%
\definecolor{currentstroke}{rgb}{1.000000,0.647059,0.000000}%
\pgfsetstrokecolor{currentstroke}%
\pgfsetdash{{1.500000pt}{2.475000pt}}{0.000000pt}%
\pgfpathmoveto{\pgfqpoint{2.094975in}{0.663635in}}%
\pgfpathlineto{\pgfqpoint{2.094975in}{2.205329in}}%
\pgfusepath{stroke}%
\end{pgfscope}%
\begin{pgfscope}%
\pgfpathrectangle{\pgfqpoint{0.946717in}{0.663635in}}{\pgfqpoint{1.826010in}{1.541693in}}%
\pgfusepath{clip}%
\pgfsetbuttcap%
\pgfsetroundjoin%
\pgfsetlinewidth{1.505625pt}%
\definecolor{currentstroke}{rgb}{1.000000,0.647059,0.000000}%
\pgfsetstrokecolor{currentstroke}%
\pgfsetdash{{1.500000pt}{2.475000pt}}{0.000000pt}%
\pgfpathmoveto{\pgfqpoint{2.400498in}{0.663635in}}%
\pgfpathlineto{\pgfqpoint{2.400498in}{2.205329in}}%
\pgfusepath{stroke}%
\end{pgfscope}%
\begin{pgfscope}%
\pgfpathrectangle{\pgfqpoint{0.946717in}{0.663635in}}{\pgfqpoint{1.826010in}{1.541693in}}%
\pgfusepath{clip}%
\pgfsetroundcap%
\pgfsetroundjoin%
\pgfsetlinewidth{1.505625pt}%
\definecolor{currentstroke}{rgb}{0.298039,0.447059,0.690196}%
\pgfsetstrokecolor{currentstroke}%
\pgfsetdash{}{0pt}%
\pgfpathmoveto{\pgfqpoint{1.029717in}{0.866259in}}%
\pgfpathlineto{\pgfqpoint{1.032263in}{0.862796in}}%
\pgfpathlineto{\pgfqpoint{1.050085in}{0.862716in}}%
\pgfpathlineto{\pgfqpoint{1.052631in}{0.865894in}}%
\pgfpathlineto{\pgfqpoint{1.121374in}{0.865024in}}%
\pgfpathlineto{\pgfqpoint{1.123920in}{0.863109in}}%
\pgfpathlineto{\pgfqpoint{1.131558in}{0.863109in}}%
\pgfpathlineto{\pgfqpoint{1.134104in}{0.856711in}}%
\pgfpathlineto{\pgfqpoint{1.141742in}{0.856711in}}%
\pgfpathlineto{\pgfqpoint{1.144288in}{0.865146in}}%
\pgfpathlineto{\pgfqpoint{1.182479in}{0.865396in}}%
\pgfpathlineto{\pgfqpoint{1.185025in}{0.862536in}}%
\pgfpathlineto{\pgfqpoint{1.213031in}{0.863623in}}%
\pgfpathlineto{\pgfqpoint{1.215577in}{0.865027in}}%
\pgfpathlineto{\pgfqpoint{1.223215in}{0.865027in}}%
\pgfpathlineto{\pgfqpoint{1.225761in}{0.868627in}}%
\pgfpathlineto{\pgfqpoint{1.233399in}{0.868627in}}%
\pgfpathlineto{\pgfqpoint{1.235945in}{0.892589in}}%
\pgfpathlineto{\pgfqpoint{1.243583in}{0.892589in}}%
\pgfpathlineto{\pgfqpoint{1.246129in}{0.886992in}}%
\pgfpathlineto{\pgfqpoint{1.253768in}{0.886992in}}%
\pgfpathlineto{\pgfqpoint{1.256314in}{0.880761in}}%
\pgfpathlineto{\pgfqpoint{1.274136in}{0.881468in}}%
\pgfpathlineto{\pgfqpoint{1.276682in}{0.873684in}}%
\pgfpathlineto{\pgfqpoint{1.284320in}{0.873684in}}%
\pgfpathlineto{\pgfqpoint{1.286866in}{0.866329in}}%
\pgfpathlineto{\pgfqpoint{1.294504in}{0.866329in}}%
\pgfpathlineto{\pgfqpoint{1.297050in}{0.869762in}}%
\pgfpathlineto{\pgfqpoint{1.304688in}{0.869762in}}%
\pgfpathlineto{\pgfqpoint{1.307234in}{0.874550in}}%
\pgfpathlineto{\pgfqpoint{1.314872in}{0.874550in}}%
\pgfpathlineto{\pgfqpoint{1.317418in}{0.870937in}}%
\pgfpathlineto{\pgfqpoint{1.325056in}{0.870937in}}%
\pgfpathlineto{\pgfqpoint{1.327602in}{0.872898in}}%
\pgfpathlineto{\pgfqpoint{1.335240in}{0.872898in}}%
\pgfpathlineto{\pgfqpoint{1.337786in}{0.891206in}}%
\pgfpathlineto{\pgfqpoint{1.345424in}{0.891206in}}%
\pgfpathlineto{\pgfqpoint{1.347971in}{0.886765in}}%
\pgfpathlineto{\pgfqpoint{1.355609in}{0.886765in}}%
\pgfpathlineto{\pgfqpoint{1.358155in}{0.890076in}}%
\pgfpathlineto{\pgfqpoint{1.365793in}{0.890076in}}%
\pgfpathlineto{\pgfqpoint{1.368339in}{0.935678in}}%
\pgfpathlineto{\pgfqpoint{1.375977in}{0.935678in}}%
\pgfpathlineto{\pgfqpoint{1.378523in}{0.944991in}}%
\pgfpathlineto{\pgfqpoint{1.386161in}{0.944991in}}%
\pgfpathlineto{\pgfqpoint{1.388707in}{0.947885in}}%
\pgfpathlineto{\pgfqpoint{1.396345in}{0.947885in}}%
\pgfpathlineto{\pgfqpoint{1.398891in}{0.956280in}}%
\pgfpathlineto{\pgfqpoint{1.406529in}{0.956280in}}%
\pgfpathlineto{\pgfqpoint{1.409075in}{0.970287in}}%
\pgfpathlineto{\pgfqpoint{1.416713in}{0.970287in}}%
\pgfpathlineto{\pgfqpoint{1.419259in}{0.906994in}}%
\pgfpathlineto{\pgfqpoint{1.426897in}{0.906994in}}%
\pgfpathlineto{\pgfqpoint{1.429443in}{1.008196in}}%
\pgfpathlineto{\pgfqpoint{1.437081in}{1.008196in}}%
\pgfpathlineto{\pgfqpoint{1.439627in}{0.915053in}}%
\pgfpathlineto{\pgfqpoint{1.447266in}{0.915053in}}%
\pgfpathlineto{\pgfqpoint{1.449812in}{1.043680in}}%
\pgfpathlineto{\pgfqpoint{1.457450in}{1.043680in}}%
\pgfpathlineto{\pgfqpoint{1.459996in}{1.064998in}}%
\pgfpathlineto{\pgfqpoint{1.467634in}{1.064998in}}%
\pgfpathlineto{\pgfqpoint{1.470180in}{1.079168in}}%
\pgfpathlineto{\pgfqpoint{1.477818in}{1.079168in}}%
\pgfpathlineto{\pgfqpoint{1.480364in}{1.082047in}}%
\pgfpathlineto{\pgfqpoint{1.498186in}{1.082969in}}%
\pgfpathlineto{\pgfqpoint{1.500732in}{0.995310in}}%
\pgfpathlineto{\pgfqpoint{1.508370in}{0.995310in}}%
\pgfpathlineto{\pgfqpoint{1.510916in}{1.055334in}}%
\pgfpathlineto{\pgfqpoint{1.518554in}{1.055334in}}%
\pgfpathlineto{\pgfqpoint{1.521100in}{1.039204in}}%
\pgfpathlineto{\pgfqpoint{1.528738in}{1.039204in}}%
\pgfpathlineto{\pgfqpoint{1.531284in}{1.020840in}}%
\pgfpathlineto{\pgfqpoint{1.538923in}{1.020840in}}%
\pgfpathlineto{\pgfqpoint{1.541469in}{1.015728in}}%
\pgfpathlineto{\pgfqpoint{1.549107in}{1.015728in}}%
\pgfpathlineto{\pgfqpoint{1.551653in}{0.991635in}}%
\pgfpathlineto{\pgfqpoint{1.559291in}{0.991635in}}%
\pgfpathlineto{\pgfqpoint{1.561837in}{0.978089in}}%
\pgfpathlineto{\pgfqpoint{1.569475in}{0.978089in}}%
\pgfpathlineto{\pgfqpoint{1.572021in}{0.974174in}}%
\pgfpathlineto{\pgfqpoint{1.579659in}{0.974174in}}%
\pgfpathlineto{\pgfqpoint{1.582205in}{0.945379in}}%
\pgfpathlineto{\pgfqpoint{1.589843in}{0.945379in}}%
\pgfpathlineto{\pgfqpoint{1.592389in}{0.943138in}}%
\pgfpathlineto{\pgfqpoint{1.610211in}{0.942662in}}%
\pgfpathlineto{\pgfqpoint{1.612757in}{0.952734in}}%
\pgfpathlineto{\pgfqpoint{1.620395in}{0.952734in}}%
\pgfpathlineto{\pgfqpoint{1.622941in}{0.947795in}}%
\pgfpathlineto{\pgfqpoint{1.630579in}{0.947795in}}%
\pgfpathlineto{\pgfqpoint{1.633126in}{0.912657in}}%
\pgfpathlineto{\pgfqpoint{1.640764in}{0.912657in}}%
\pgfpathlineto{\pgfqpoint{1.643310in}{0.906420in}}%
\pgfpathlineto{\pgfqpoint{1.650948in}{0.906420in}}%
\pgfpathlineto{\pgfqpoint{1.653494in}{0.959943in}}%
\pgfpathlineto{\pgfqpoint{1.661132in}{0.959943in}}%
\pgfpathlineto{\pgfqpoint{1.663678in}{0.918051in}}%
\pgfpathlineto{\pgfqpoint{1.671316in}{0.918051in}}%
\pgfpathlineto{\pgfqpoint{1.673862in}{0.912174in}}%
\pgfpathlineto{\pgfqpoint{1.681500in}{0.912174in}}%
\pgfpathlineto{\pgfqpoint{1.684046in}{0.909091in}}%
\pgfpathlineto{\pgfqpoint{1.691684in}{0.909091in}}%
\pgfpathlineto{\pgfqpoint{1.694230in}{0.906363in}}%
\pgfpathlineto{\pgfqpoint{1.701868in}{0.906363in}}%
\pgfpathlineto{\pgfqpoint{1.704414in}{0.908082in}}%
\pgfpathlineto{\pgfqpoint{1.712052in}{0.908082in}}%
\pgfpathlineto{\pgfqpoint{1.714598in}{0.909609in}}%
\pgfpathlineto{\pgfqpoint{1.722236in}{0.909609in}}%
\pgfpathlineto{\pgfqpoint{1.724782in}{0.941701in}}%
\pgfpathlineto{\pgfqpoint{1.732421in}{0.941701in}}%
\pgfpathlineto{\pgfqpoint{1.734967in}{0.922381in}}%
\pgfpathlineto{\pgfqpoint{1.742605in}{0.922381in}}%
\pgfpathlineto{\pgfqpoint{1.745151in}{0.916257in}}%
\pgfpathlineto{\pgfqpoint{1.752789in}{0.916257in}}%
\pgfpathlineto{\pgfqpoint{1.755335in}{0.913047in}}%
\pgfpathlineto{\pgfqpoint{1.762973in}{0.913047in}}%
\pgfpathlineto{\pgfqpoint{1.765519in}{0.911260in}}%
\pgfpathlineto{\pgfqpoint{1.783341in}{0.910332in}}%
\pgfpathlineto{\pgfqpoint{1.785887in}{0.905027in}}%
\pgfpathlineto{\pgfqpoint{1.793525in}{0.905027in}}%
\pgfpathlineto{\pgfqpoint{1.796071in}{0.903241in}}%
\pgfpathlineto{\pgfqpoint{1.803709in}{0.903241in}}%
\pgfpathlineto{\pgfqpoint{1.806255in}{0.906615in}}%
\pgfpathlineto{\pgfqpoint{1.824078in}{0.906523in}}%
\pgfpathlineto{\pgfqpoint{1.826624in}{0.902718in}}%
\pgfpathlineto{\pgfqpoint{1.834262in}{0.902718in}}%
\pgfpathlineto{\pgfqpoint{1.836808in}{0.914739in}}%
\pgfpathlineto{\pgfqpoint{1.844446in}{0.914739in}}%
\pgfpathlineto{\pgfqpoint{1.846992in}{0.926129in}}%
\pgfpathlineto{\pgfqpoint{1.854630in}{0.926129in}}%
\pgfpathlineto{\pgfqpoint{1.857176in}{0.932991in}}%
\pgfpathlineto{\pgfqpoint{1.864814in}{0.932991in}}%
\pgfpathlineto{\pgfqpoint{1.867360in}{0.944601in}}%
\pgfpathlineto{\pgfqpoint{1.874998in}{0.944601in}}%
\pgfpathlineto{\pgfqpoint{1.877544in}{0.955987in}}%
\pgfpathlineto{\pgfqpoint{1.885182in}{0.955987in}}%
\pgfpathlineto{\pgfqpoint{1.887728in}{0.938362in}}%
\pgfpathlineto{\pgfqpoint{1.895366in}{0.938362in}}%
\pgfpathlineto{\pgfqpoint{1.897912in}{0.975177in}}%
\pgfpathlineto{\pgfqpoint{1.905550in}{0.975177in}}%
\pgfpathlineto{\pgfqpoint{1.908096in}{0.986757in}}%
\pgfpathlineto{\pgfqpoint{1.915735in}{0.986757in}}%
\pgfpathlineto{\pgfqpoint{1.918281in}{0.997830in}}%
\pgfpathlineto{\pgfqpoint{1.925919in}{0.997830in}}%
\pgfpathlineto{\pgfqpoint{1.928465in}{1.007275in}}%
\pgfpathlineto{\pgfqpoint{1.936103in}{1.007275in}}%
\pgfpathlineto{\pgfqpoint{1.938649in}{0.961563in}}%
\pgfpathlineto{\pgfqpoint{1.946287in}{0.961563in}}%
\pgfpathlineto{\pgfqpoint{1.948833in}{0.964367in}}%
\pgfpathlineto{\pgfqpoint{1.956471in}{0.964367in}}%
\pgfpathlineto{\pgfqpoint{1.959017in}{1.038433in}}%
\pgfpathlineto{\pgfqpoint{1.966655in}{1.038433in}}%
\pgfpathlineto{\pgfqpoint{1.969201in}{1.051416in}}%
\pgfpathlineto{\pgfqpoint{1.987023in}{1.052222in}}%
\pgfpathlineto{\pgfqpoint{1.989569in}{0.984802in}}%
\pgfpathlineto{\pgfqpoint{1.997207in}{0.984802in}}%
\pgfpathlineto{\pgfqpoint{1.999753in}{1.054291in}}%
\pgfpathlineto{\pgfqpoint{2.027760in}{1.054884in}}%
\pgfpathlineto{\pgfqpoint{2.030306in}{1.050929in}}%
\pgfpathlineto{\pgfqpoint{2.048128in}{1.049883in}}%
\pgfpathlineto{\pgfqpoint{2.050674in}{1.057242in}}%
\pgfpathlineto{\pgfqpoint{2.058312in}{1.057242in}}%
\pgfpathlineto{\pgfqpoint{2.060858in}{1.022536in}}%
\pgfpathlineto{\pgfqpoint{2.068496in}{1.022536in}}%
\pgfpathlineto{\pgfqpoint{2.071042in}{1.057241in}}%
\pgfpathlineto{\pgfqpoint{2.088864in}{1.056601in}}%
\pgfpathlineto{\pgfqpoint{2.091410in}{1.044251in}}%
\pgfpathlineto{\pgfqpoint{2.099048in}{1.044251in}}%
\pgfpathlineto{\pgfqpoint{2.101594in}{1.046028in}}%
\pgfpathlineto{\pgfqpoint{2.109233in}{1.046028in}}%
\pgfpathlineto{\pgfqpoint{2.111779in}{1.044659in}}%
\pgfpathlineto{\pgfqpoint{2.119417in}{1.044659in}}%
\pgfpathlineto{\pgfqpoint{2.121963in}{1.027114in}}%
\pgfpathlineto{\pgfqpoint{2.129601in}{1.027114in}}%
\pgfpathlineto{\pgfqpoint{2.132147in}{1.018788in}}%
\pgfpathlineto{\pgfqpoint{2.139785in}{1.018788in}}%
\pgfpathlineto{\pgfqpoint{2.142331in}{1.013025in}}%
\pgfpathlineto{\pgfqpoint{2.149969in}{1.013025in}}%
\pgfpathlineto{\pgfqpoint{2.152515in}{1.002854in}}%
\pgfpathlineto{\pgfqpoint{2.160153in}{1.002854in}}%
\pgfpathlineto{\pgfqpoint{2.162699in}{0.954696in}}%
\pgfpathlineto{\pgfqpoint{2.170337in}{0.954696in}}%
\pgfpathlineto{\pgfqpoint{2.172883in}{0.972913in}}%
\pgfpathlineto{\pgfqpoint{2.180521in}{0.972913in}}%
\pgfpathlineto{\pgfqpoint{2.183067in}{0.958554in}}%
\pgfpathlineto{\pgfqpoint{2.190705in}{0.958554in}}%
\pgfpathlineto{\pgfqpoint{2.193251in}{0.987772in}}%
\pgfpathlineto{\pgfqpoint{2.200890in}{0.987772in}}%
\pgfpathlineto{\pgfqpoint{2.203436in}{0.947067in}}%
\pgfpathlineto{\pgfqpoint{2.211074in}{0.947067in}}%
\pgfpathlineto{\pgfqpoint{2.213620in}{0.971514in}}%
\pgfpathlineto{\pgfqpoint{2.221258in}{0.971514in}}%
\pgfpathlineto{\pgfqpoint{2.223804in}{0.928224in}}%
\pgfpathlineto{\pgfqpoint{2.231442in}{0.928224in}}%
\pgfpathlineto{\pgfqpoint{2.233988in}{0.961534in}}%
\pgfpathlineto{\pgfqpoint{2.241626in}{0.961534in}}%
\pgfpathlineto{\pgfqpoint{2.244172in}{0.956379in}}%
\pgfpathlineto{\pgfqpoint{2.251810in}{0.956379in}}%
\pgfpathlineto{\pgfqpoint{2.254356in}{0.950527in}}%
\pgfpathlineto{\pgfqpoint{2.261994in}{0.950527in}}%
\pgfpathlineto{\pgfqpoint{2.264540in}{0.913183in}}%
\pgfpathlineto{\pgfqpoint{2.272178in}{0.913183in}}%
\pgfpathlineto{\pgfqpoint{2.274724in}{0.910900in}}%
\pgfpathlineto{\pgfqpoint{2.292546in}{0.911098in}}%
\pgfpathlineto{\pgfqpoint{2.295093in}{0.916662in}}%
\pgfpathlineto{\pgfqpoint{2.312915in}{0.917655in}}%
\pgfpathlineto{\pgfqpoint{2.315461in}{0.922202in}}%
\pgfpathlineto{\pgfqpoint{2.323099in}{0.922202in}}%
\pgfpathlineto{\pgfqpoint{2.325645in}{0.930061in}}%
\pgfpathlineto{\pgfqpoint{2.333283in}{0.930061in}}%
\pgfpathlineto{\pgfqpoint{2.335829in}{0.941220in}}%
\pgfpathlineto{\pgfqpoint{2.343467in}{0.941220in}}%
\pgfpathlineto{\pgfqpoint{2.346013in}{0.951623in}}%
\pgfpathlineto{\pgfqpoint{2.353651in}{0.951623in}}%
\pgfpathlineto{\pgfqpoint{2.356197in}{0.961916in}}%
\pgfpathlineto{\pgfqpoint{2.363835in}{0.961916in}}%
\pgfpathlineto{\pgfqpoint{2.366381in}{0.977231in}}%
\pgfpathlineto{\pgfqpoint{2.374019in}{0.977231in}}%
\pgfpathlineto{\pgfqpoint{2.376565in}{0.982889in}}%
\pgfpathlineto{\pgfqpoint{2.384203in}{0.982889in}}%
\pgfpathlineto{\pgfqpoint{2.386749in}{0.993027in}}%
\pgfpathlineto{\pgfqpoint{2.394388in}{0.993027in}}%
\pgfpathlineto{\pgfqpoint{2.396934in}{1.006080in}}%
\pgfpathlineto{\pgfqpoint{2.404572in}{1.006080in}}%
\pgfpathlineto{\pgfqpoint{2.407118in}{1.010945in}}%
\pgfpathlineto{\pgfqpoint{2.414756in}{1.010945in}}%
\pgfpathlineto{\pgfqpoint{2.417302in}{0.963347in}}%
\pgfpathlineto{\pgfqpoint{2.424940in}{0.963347in}}%
\pgfpathlineto{\pgfqpoint{2.427486in}{0.969372in}}%
\pgfpathlineto{\pgfqpoint{2.435124in}{0.969372in}}%
\pgfpathlineto{\pgfqpoint{2.437670in}{1.033217in}}%
\pgfpathlineto{\pgfqpoint{2.445308in}{1.033217in}}%
\pgfpathlineto{\pgfqpoint{2.447854in}{1.035497in}}%
\pgfpathlineto{\pgfqpoint{2.465676in}{1.035738in}}%
\pgfpathlineto{\pgfqpoint{2.468222in}{1.034145in}}%
\pgfpathlineto{\pgfqpoint{2.475860in}{1.034145in}}%
\pgfpathlineto{\pgfqpoint{2.478406in}{1.030803in}}%
\pgfpathlineto{\pgfqpoint{2.486045in}{1.030803in}}%
\pgfpathlineto{\pgfqpoint{2.488591in}{1.021082in}}%
\pgfpathlineto{\pgfqpoint{2.496229in}{1.021082in}}%
\pgfpathlineto{\pgfqpoint{2.498775in}{1.017811in}}%
\pgfpathlineto{\pgfqpoint{2.506413in}{1.017811in}}%
\pgfpathlineto{\pgfqpoint{2.508959in}{1.007034in}}%
\pgfpathlineto{\pgfqpoint{2.516597in}{1.007034in}}%
\pgfpathlineto{\pgfqpoint{2.519143in}{0.994367in}}%
\pgfpathlineto{\pgfqpoint{2.526781in}{0.994367in}}%
\pgfpathlineto{\pgfqpoint{2.529327in}{0.986386in}}%
\pgfpathlineto{\pgfqpoint{2.536965in}{0.986386in}}%
\pgfpathlineto{\pgfqpoint{2.539511in}{0.978107in}}%
\pgfpathlineto{\pgfqpoint{2.547149in}{0.978107in}}%
\pgfpathlineto{\pgfqpoint{2.549695in}{0.986617in}}%
\pgfpathlineto{\pgfqpoint{2.567517in}{0.986297in}}%
\pgfpathlineto{\pgfqpoint{2.570063in}{0.940975in}}%
\pgfpathlineto{\pgfqpoint{2.577701in}{0.940975in}}%
\pgfpathlineto{\pgfqpoint{2.580248in}{0.930225in}}%
\pgfpathlineto{\pgfqpoint{2.587886in}{0.930225in}}%
\pgfpathlineto{\pgfqpoint{2.590432in}{0.980214in}}%
\pgfpathlineto{\pgfqpoint{2.598070in}{0.980214in}}%
\pgfpathlineto{\pgfqpoint{2.600616in}{0.914228in}}%
\pgfpathlineto{\pgfqpoint{2.608254in}{0.914228in}}%
\pgfpathlineto{\pgfqpoint{2.610800in}{0.969652in}}%
\pgfpathlineto{\pgfqpoint{2.618438in}{0.969652in}}%
\pgfpathlineto{\pgfqpoint{2.620984in}{0.894887in}}%
\pgfpathlineto{\pgfqpoint{2.628622in}{0.894887in}}%
\pgfpathlineto{\pgfqpoint{2.631168in}{0.959945in}}%
\pgfpathlineto{\pgfqpoint{2.638806in}{0.959945in}}%
\pgfpathlineto{\pgfqpoint{2.641352in}{0.885372in}}%
\pgfpathlineto{\pgfqpoint{2.648990in}{0.885372in}}%
\pgfpathlineto{\pgfqpoint{2.651536in}{0.889654in}}%
\pgfpathlineto{\pgfqpoint{2.669358in}{0.889336in}}%
\pgfpathlineto{\pgfqpoint{2.671904in}{0.939182in}}%
\pgfpathlineto{\pgfqpoint{2.679543in}{0.939182in}}%
\pgfpathlineto{\pgfqpoint{2.682089in}{0.890077in}}%
\pgfpathlineto{\pgfqpoint{2.689727in}{0.890077in}}%
\pgfpathlineto{\pgfqpoint{2.689727in}{0.890077in}}%
\pgfusepath{stroke}%
\end{pgfscope}%
\begin{pgfscope}%
\pgfpathrectangle{\pgfqpoint{0.946717in}{0.663635in}}{\pgfqpoint{1.826010in}{1.541693in}}%
\pgfusepath{clip}%
\pgfsetbuttcap%
\pgfsetroundjoin%
\pgfsetlinewidth{1.505625pt}%
\definecolor{currentstroke}{rgb}{0.580392,0.403922,0.741176}%
\pgfsetstrokecolor{currentstroke}%
\pgfsetdash{{5.550000pt}{2.400000pt}}{0.000000pt}%
\pgfpathmoveto{\pgfqpoint{0.946717in}{1.562957in}}%
\pgfpathlineto{\pgfqpoint{2.772727in}{1.562957in}}%
\pgfusepath{stroke}%
\end{pgfscope}%
\begin{pgfscope}%
\pgfsetrectcap%
\pgfsetmiterjoin%
\pgfsetlinewidth{1.254687pt}%
\definecolor{currentstroke}{rgb}{1.000000,1.000000,1.000000}%
\pgfsetstrokecolor{currentstroke}%
\pgfsetdash{}{0pt}%
\pgfpathmoveto{\pgfqpoint{0.946717in}{0.663635in}}%
\pgfpathlineto{\pgfqpoint{0.946717in}{2.205329in}}%
\pgfusepath{stroke}%
\end{pgfscope}%
\begin{pgfscope}%
\pgfsetrectcap%
\pgfsetmiterjoin%
\pgfsetlinewidth{1.254687pt}%
\definecolor{currentstroke}{rgb}{1.000000,1.000000,1.000000}%
\pgfsetstrokecolor{currentstroke}%
\pgfsetdash{}{0pt}%
\pgfpathmoveto{\pgfqpoint{2.772727in}{0.663635in}}%
\pgfpathlineto{\pgfqpoint{2.772727in}{2.205329in}}%
\pgfusepath{stroke}%
\end{pgfscope}%
\begin{pgfscope}%
\pgfsetrectcap%
\pgfsetmiterjoin%
\pgfsetlinewidth{1.254687pt}%
\definecolor{currentstroke}{rgb}{1.000000,1.000000,1.000000}%
\pgfsetstrokecolor{currentstroke}%
\pgfsetdash{}{0pt}%
\pgfpathmoveto{\pgfqpoint{0.946717in}{0.663635in}}%
\pgfpathlineto{\pgfqpoint{2.772727in}{0.663635in}}%
\pgfusepath{stroke}%
\end{pgfscope}%
\begin{pgfscope}%
\pgfsetrectcap%
\pgfsetmiterjoin%
\pgfsetlinewidth{1.254687pt}%
\definecolor{currentstroke}{rgb}{1.000000,1.000000,1.000000}%
\pgfsetstrokecolor{currentstroke}%
\pgfsetdash{}{0pt}%
\pgfpathmoveto{\pgfqpoint{0.946717in}{2.205329in}}%
\pgfpathlineto{\pgfqpoint{2.772727in}{2.205329in}}%
\pgfusepath{stroke}%
\end{pgfscope}%
\begin{pgfscope}%
\definecolor{textcolor}{rgb}{0.150000,0.150000,0.150000}%
\pgfsetstrokecolor{textcolor}%
\pgfsetfillcolor{textcolor}%
\pgftext[x=1.859722in,y=2.288662in,,base]{\color{textcolor}{\sffamily\fontsize{12.000000}{14.400000}\selectfont\catcode`\^=\active\def^{\ifmmode\sp\else\^{}\fi}\catcode`\%=\active\def%{\%}c)}}%
\end{pgfscope}%
\begin{pgfscope}%
\pgfsetbuttcap%
\pgfsetmiterjoin%
\definecolor{currentfill}{rgb}{0.917647,0.917647,0.949020}%
\pgfsetfillcolor{currentfill}%
\pgfsetlinewidth{0.000000pt}%
\definecolor{currentstroke}{rgb}{0.000000,0.000000,0.000000}%
\pgfsetstrokecolor{currentstroke}%
\pgfsetstrokeopacity{0.000000}%
\pgfsetdash{}{0pt}%
\pgfpathmoveto{\pgfqpoint{3.593990in}{0.663635in}}%
\pgfpathlineto{\pgfqpoint{5.420000in}{0.663635in}}%
\pgfpathlineto{\pgfqpoint{5.420000in}{2.205329in}}%
\pgfpathlineto{\pgfqpoint{3.593990in}{2.205329in}}%
\pgfpathlineto{\pgfqpoint{3.593990in}{0.663635in}}%
\pgfpathclose%
\pgfusepath{fill}%
\end{pgfscope}%
\begin{pgfscope}%
\pgfpathrectangle{\pgfqpoint{3.593990in}{0.663635in}}{\pgfqpoint{1.826010in}{1.541693in}}%
\pgfusepath{clip}%
\pgfsetroundcap%
\pgfsetroundjoin%
\pgfsetlinewidth{1.003750pt}%
\definecolor{currentstroke}{rgb}{1.000000,1.000000,1.000000}%
\pgfsetstrokecolor{currentstroke}%
\pgfsetdash{}{0pt}%
\pgfpathmoveto{\pgfqpoint{3.676990in}{0.663635in}}%
\pgfpathlineto{\pgfqpoint{3.676990in}{2.205329in}}%
\pgfusepath{stroke}%
\end{pgfscope}%
\begin{pgfscope}%
\definecolor{textcolor}{rgb}{0.150000,0.150000,0.150000}%
\pgfsetstrokecolor{textcolor}%
\pgfsetfillcolor{textcolor}%
\pgftext[x=3.676990in,y=0.531691in,,top]{\color{textcolor}{\sffamily\fontsize{11.000000}{13.200000}\selectfont\catcode`\^=\active\def^{\ifmmode\sp\else\^{}\fi}\catcode`\%=\active\def%{\%}0}}%
\end{pgfscope}%
\begin{pgfscope}%
\pgfpathrectangle{\pgfqpoint{3.593990in}{0.663635in}}{\pgfqpoint{1.826010in}{1.541693in}}%
\pgfusepath{clip}%
\pgfsetroundcap%
\pgfsetroundjoin%
\pgfsetlinewidth{1.003750pt}%
\definecolor{currentstroke}{rgb}{1.000000,1.000000,1.000000}%
\pgfsetstrokecolor{currentstroke}%
\pgfsetdash{}{0pt}%
\pgfpathmoveto{\pgfqpoint{4.695401in}{0.663635in}}%
\pgfpathlineto{\pgfqpoint{4.695401in}{2.205329in}}%
\pgfusepath{stroke}%
\end{pgfscope}%
\begin{pgfscope}%
\definecolor{textcolor}{rgb}{0.150000,0.150000,0.150000}%
\pgfsetstrokecolor{textcolor}%
\pgfsetfillcolor{textcolor}%
\pgftext[x=4.695401in,y=0.531691in,,top]{\color{textcolor}{\sffamily\fontsize{11.000000}{13.200000}\selectfont\catcode`\^=\active\def^{\ifmmode\sp\else\^{}\fi}\catcode`\%=\active\def%{\%}2000}}%
\end{pgfscope}%
\begin{pgfscope}%
\definecolor{textcolor}{rgb}{0.150000,0.150000,0.150000}%
\pgfsetstrokecolor{textcolor}%
\pgfsetfillcolor{textcolor}%
\pgftext[x=4.506995in,y=0.336413in,,top]{\color{textcolor}{\sffamily\fontsize{12.000000}{14.400000}\selectfont\catcode`\^=\active\def^{\ifmmode\sp\else\^{}\fi}\catcode`\%=\active\def%{\%}Time (s)}}%
\end{pgfscope}%
\begin{pgfscope}%
\pgfpathrectangle{\pgfqpoint{3.593990in}{0.663635in}}{\pgfqpoint{1.826010in}{1.541693in}}%
\pgfusepath{clip}%
\pgfsetroundcap%
\pgfsetroundjoin%
\pgfsetlinewidth{1.003750pt}%
\definecolor{currentstroke}{rgb}{1.000000,1.000000,1.000000}%
\pgfsetstrokecolor{currentstroke}%
\pgfsetdash{}{0pt}%
\pgfpathmoveto{\pgfqpoint{3.593990in}{0.792110in}}%
\pgfpathlineto{\pgfqpoint{5.420000in}{0.792110in}}%
\pgfusepath{stroke}%
\end{pgfscope}%
\begin{pgfscope}%
\definecolor{textcolor}{rgb}{0.150000,0.150000,0.150000}%
\pgfsetstrokecolor{textcolor}%
\pgfsetfillcolor{textcolor}%
\pgftext[x=3.164695in, y=0.737429in, left, base]{\color{textcolor}{\sffamily\fontsize{11.000000}{13.200000}\selectfont\catcode`\^=\active\def^{\ifmmode\sp\else\^{}\fi}\catcode`\%=\active\def%{\%}0.00}}%
\end{pgfscope}%
\begin{pgfscope}%
\pgfpathrectangle{\pgfqpoint{3.593990in}{0.663635in}}{\pgfqpoint{1.826010in}{1.541693in}}%
\pgfusepath{clip}%
\pgfsetroundcap%
\pgfsetroundjoin%
\pgfsetlinewidth{1.003750pt}%
\definecolor{currentstroke}{rgb}{1.000000,1.000000,1.000000}%
\pgfsetstrokecolor{currentstroke}%
\pgfsetdash{}{0pt}%
\pgfpathmoveto{\pgfqpoint{3.593990in}{1.113296in}}%
\pgfpathlineto{\pgfqpoint{5.420000in}{1.113296in}}%
\pgfusepath{stroke}%
\end{pgfscope}%
\begin{pgfscope}%
\definecolor{textcolor}{rgb}{0.150000,0.150000,0.150000}%
\pgfsetstrokecolor{textcolor}%
\pgfsetfillcolor{textcolor}%
\pgftext[x=3.164695in, y=1.058615in, left, base]{\color{textcolor}{\sffamily\fontsize{11.000000}{13.200000}\selectfont\catcode`\^=\active\def^{\ifmmode\sp\else\^{}\fi}\catcode`\%=\active\def%{\%}0.25}}%
\end{pgfscope}%
\begin{pgfscope}%
\pgfpathrectangle{\pgfqpoint{3.593990in}{0.663635in}}{\pgfqpoint{1.826010in}{1.541693in}}%
\pgfusepath{clip}%
\pgfsetroundcap%
\pgfsetroundjoin%
\pgfsetlinewidth{1.003750pt}%
\definecolor{currentstroke}{rgb}{1.000000,1.000000,1.000000}%
\pgfsetstrokecolor{currentstroke}%
\pgfsetdash{}{0pt}%
\pgfpathmoveto{\pgfqpoint{3.593990in}{1.434482in}}%
\pgfpathlineto{\pgfqpoint{5.420000in}{1.434482in}}%
\pgfusepath{stroke}%
\end{pgfscope}%
\begin{pgfscope}%
\definecolor{textcolor}{rgb}{0.150000,0.150000,0.150000}%
\pgfsetstrokecolor{textcolor}%
\pgfsetfillcolor{textcolor}%
\pgftext[x=3.164695in, y=1.379801in, left, base]{\color{textcolor}{\sffamily\fontsize{11.000000}{13.200000}\selectfont\catcode`\^=\active\def^{\ifmmode\sp\else\^{}\fi}\catcode`\%=\active\def%{\%}0.50}}%
\end{pgfscope}%
\begin{pgfscope}%
\pgfpathrectangle{\pgfqpoint{3.593990in}{0.663635in}}{\pgfqpoint{1.826010in}{1.541693in}}%
\pgfusepath{clip}%
\pgfsetroundcap%
\pgfsetroundjoin%
\pgfsetlinewidth{1.003750pt}%
\definecolor{currentstroke}{rgb}{1.000000,1.000000,1.000000}%
\pgfsetstrokecolor{currentstroke}%
\pgfsetdash{}{0pt}%
\pgfpathmoveto{\pgfqpoint{3.593990in}{1.755668in}}%
\pgfpathlineto{\pgfqpoint{5.420000in}{1.755668in}}%
\pgfusepath{stroke}%
\end{pgfscope}%
\begin{pgfscope}%
\definecolor{textcolor}{rgb}{0.150000,0.150000,0.150000}%
\pgfsetstrokecolor{textcolor}%
\pgfsetfillcolor{textcolor}%
\pgftext[x=3.164695in, y=1.700987in, left, base]{\color{textcolor}{\sffamily\fontsize{11.000000}{13.200000}\selectfont\catcode`\^=\active\def^{\ifmmode\sp\else\^{}\fi}\catcode`\%=\active\def%{\%}0.75}}%
\end{pgfscope}%
\begin{pgfscope}%
\pgfpathrectangle{\pgfqpoint{3.593990in}{0.663635in}}{\pgfqpoint{1.826010in}{1.541693in}}%
\pgfusepath{clip}%
\pgfsetroundcap%
\pgfsetroundjoin%
\pgfsetlinewidth{1.003750pt}%
\definecolor{currentstroke}{rgb}{1.000000,1.000000,1.000000}%
\pgfsetstrokecolor{currentstroke}%
\pgfsetdash{}{0pt}%
\pgfpathmoveto{\pgfqpoint{3.593990in}{2.076854in}}%
\pgfpathlineto{\pgfqpoint{5.420000in}{2.076854in}}%
\pgfusepath{stroke}%
\end{pgfscope}%
\begin{pgfscope}%
\definecolor{textcolor}{rgb}{0.150000,0.150000,0.150000}%
\pgfsetstrokecolor{textcolor}%
\pgfsetfillcolor{textcolor}%
\pgftext[x=3.164695in, y=2.022174in, left, base]{\color{textcolor}{\sffamily\fontsize{11.000000}{13.200000}\selectfont\catcode`\^=\active\def^{\ifmmode\sp\else\^{}\fi}\catcode`\%=\active\def%{\%}1.00}}%
\end{pgfscope}%
\begin{pgfscope}%
\definecolor{textcolor}{rgb}{0.150000,0.150000,0.150000}%
\pgfsetstrokecolor{textcolor}%
\pgfsetfillcolor{textcolor}%
\pgftext[x=3.109140in,y=1.434482in,,bottom,rotate=90.000000]{\color{textcolor}{\sffamily\fontsize{12.000000}{14.400000}\selectfont\catcode`\^=\active\def^{\ifmmode\sp\else\^{}\fi}\catcode`\%=\active\def%{\%}Memory Utilization (%)}}%
\end{pgfscope}%
\begin{pgfscope}%
\pgfpathrectangle{\pgfqpoint{3.593990in}{0.663635in}}{\pgfqpoint{1.826010in}{1.541693in}}%
\pgfusepath{clip}%
\pgfsetbuttcap%
\pgfsetmiterjoin%
\definecolor{currentfill}{rgb}{1.000000,0.000000,0.000000}%
\pgfsetfillcolor{currentfill}%
\pgfsetfillopacity{0.300000}%
\pgfsetlinewidth{1.003750pt}%
\definecolor{currentstroke}{rgb}{1.000000,0.000000,0.000000}%
\pgfsetstrokecolor{currentstroke}%
\pgfsetstrokeopacity{0.300000}%
\pgfsetdash{}{0pt}%
\pgfpathmoveto{\pgfqpoint{3.847574in}{0.663635in}}%
\pgfpathlineto{\pgfqpoint{3.847574in}{2.205329in}}%
\pgfpathlineto{\pgfqpoint{3.987605in}{2.205329in}}%
\pgfpathlineto{\pgfqpoint{3.987605in}{0.663635in}}%
\pgfpathlineto{\pgfqpoint{3.847574in}{0.663635in}}%
\pgfpathclose%
\pgfusepath{stroke,fill}%
\end{pgfscope}%
\begin{pgfscope}%
\pgfpathrectangle{\pgfqpoint{3.593990in}{0.663635in}}{\pgfqpoint{1.826010in}{1.541693in}}%
\pgfusepath{clip}%
\pgfsetbuttcap%
\pgfsetmiterjoin%
\definecolor{currentfill}{rgb}{1.000000,0.000000,0.000000}%
\pgfsetfillcolor{currentfill}%
\pgfsetfillopacity{0.300000}%
\pgfsetlinewidth{1.003750pt}%
\definecolor{currentstroke}{rgb}{1.000000,0.000000,0.000000}%
\pgfsetstrokecolor{currentstroke}%
\pgfsetstrokeopacity{0.300000}%
\pgfsetdash{}{0pt}%
\pgfpathmoveto{\pgfqpoint{4.211656in}{0.663635in}}%
\pgfpathlineto{\pgfqpoint{4.211656in}{2.205329in}}%
\pgfpathlineto{\pgfqpoint{4.389878in}{2.205329in}}%
\pgfpathlineto{\pgfqpoint{4.389878in}{0.663635in}}%
\pgfpathlineto{\pgfqpoint{4.211656in}{0.663635in}}%
\pgfpathclose%
\pgfusepath{stroke,fill}%
\end{pgfscope}%
\begin{pgfscope}%
\pgfpathrectangle{\pgfqpoint{3.593990in}{0.663635in}}{\pgfqpoint{1.826010in}{1.541693in}}%
\pgfusepath{clip}%
\pgfsetbuttcap%
\pgfsetroundjoin%
\pgfsetlinewidth{1.505625pt}%
\definecolor{currentstroke}{rgb}{1.000000,0.647059,0.000000}%
\pgfsetstrokecolor{currentstroke}%
\pgfsetdash{{1.500000pt}{2.475000pt}}{0.000000pt}%
\pgfpathmoveto{\pgfqpoint{3.821604in}{0.663635in}}%
\pgfpathlineto{\pgfqpoint{3.821604in}{2.205329in}}%
\pgfusepath{stroke}%
\end{pgfscope}%
\begin{pgfscope}%
\pgfpathrectangle{\pgfqpoint{3.593990in}{0.663635in}}{\pgfqpoint{1.826010in}{1.541693in}}%
\pgfusepath{clip}%
\pgfsetbuttcap%
\pgfsetroundjoin%
\pgfsetlinewidth{1.505625pt}%
\definecolor{currentstroke}{rgb}{1.000000,0.647059,0.000000}%
\pgfsetstrokecolor{currentstroke}%
\pgfsetdash{{1.500000pt}{2.475000pt}}{0.000000pt}%
\pgfpathmoveto{\pgfqpoint{4.129674in}{0.663635in}}%
\pgfpathlineto{\pgfqpoint{4.129674in}{2.205329in}}%
\pgfusepath{stroke}%
\end{pgfscope}%
\begin{pgfscope}%
\pgfpathrectangle{\pgfqpoint{3.593990in}{0.663635in}}{\pgfqpoint{1.826010in}{1.541693in}}%
\pgfusepath{clip}%
\pgfsetbuttcap%
\pgfsetroundjoin%
\pgfsetlinewidth{1.505625pt}%
\definecolor{currentstroke}{rgb}{1.000000,0.647059,0.000000}%
\pgfsetstrokecolor{currentstroke}%
\pgfsetdash{{1.500000pt}{2.475000pt}}{0.000000pt}%
\pgfpathmoveto{\pgfqpoint{4.435706in}{0.663635in}}%
\pgfpathlineto{\pgfqpoint{4.435706in}{2.205329in}}%
\pgfusepath{stroke}%
\end{pgfscope}%
\begin{pgfscope}%
\pgfpathrectangle{\pgfqpoint{3.593990in}{0.663635in}}{\pgfqpoint{1.826010in}{1.541693in}}%
\pgfusepath{clip}%
\pgfsetbuttcap%
\pgfsetroundjoin%
\pgfsetlinewidth{1.505625pt}%
\definecolor{currentstroke}{rgb}{1.000000,0.647059,0.000000}%
\pgfsetstrokecolor{currentstroke}%
\pgfsetdash{{1.500000pt}{2.475000pt}}{0.000000pt}%
\pgfpathmoveto{\pgfqpoint{4.742248in}{0.663635in}}%
\pgfpathlineto{\pgfqpoint{4.742248in}{2.205329in}}%
\pgfusepath{stroke}%
\end{pgfscope}%
\begin{pgfscope}%
\pgfpathrectangle{\pgfqpoint{3.593990in}{0.663635in}}{\pgfqpoint{1.826010in}{1.541693in}}%
\pgfusepath{clip}%
\pgfsetbuttcap%
\pgfsetroundjoin%
\pgfsetlinewidth{1.505625pt}%
\definecolor{currentstroke}{rgb}{1.000000,0.647059,0.000000}%
\pgfsetstrokecolor{currentstroke}%
\pgfsetdash{{1.500000pt}{2.475000pt}}{0.000000pt}%
\pgfpathmoveto{\pgfqpoint{5.047771in}{0.663635in}}%
\pgfpathlineto{\pgfqpoint{5.047771in}{2.205329in}}%
\pgfusepath{stroke}%
\end{pgfscope}%
\begin{pgfscope}%
\pgfpathrectangle{\pgfqpoint{3.593990in}{0.663635in}}{\pgfqpoint{1.826010in}{1.541693in}}%
\pgfusepath{clip}%
\pgfsetroundcap%
\pgfsetroundjoin%
\pgfsetlinewidth{1.505625pt}%
\definecolor{currentstroke}{rgb}{0.298039,0.447059,0.690196}%
\pgfsetstrokecolor{currentstroke}%
\pgfsetdash{}{0pt}%
\pgfpathmoveto{\pgfqpoint{3.676990in}{1.456257in}}%
\pgfpathlineto{\pgfqpoint{3.694812in}{1.456482in}}%
\pgfpathlineto{\pgfqpoint{3.697358in}{1.456482in}}%
\pgfpathlineto{\pgfqpoint{3.699904in}{1.454167in}}%
\pgfpathlineto{\pgfqpoint{3.748279in}{1.453987in}}%
\pgfpathlineto{\pgfqpoint{3.750825in}{1.455761in}}%
\pgfpathlineto{\pgfqpoint{3.829752in}{1.455691in}}%
\pgfpathlineto{\pgfqpoint{3.832298in}{1.823900in}}%
\pgfpathlineto{\pgfqpoint{3.839936in}{1.823900in}}%
\pgfpathlineto{\pgfqpoint{3.842482in}{1.826405in}}%
\pgfpathlineto{\pgfqpoint{3.850120in}{1.826405in}}%
\pgfpathlineto{\pgfqpoint{3.851169in}{2.215329in}}%
\pgfpathmoveto{\pgfqpoint{3.984273in}{2.215329in}}%
\pgfpathlineto{\pgfqpoint{3.985059in}{1.905119in}}%
\pgfpathlineto{\pgfqpoint{3.992697in}{1.905119in}}%
\pgfpathlineto{\pgfqpoint{3.995243in}{1.946038in}}%
\pgfpathlineto{\pgfqpoint{4.002881in}{1.946038in}}%
\pgfpathlineto{\pgfqpoint{4.005427in}{1.956183in}}%
\pgfpathlineto{\pgfqpoint{4.013066in}{1.956183in}}%
\pgfpathlineto{\pgfqpoint{4.015612in}{1.954042in}}%
\pgfpathlineto{\pgfqpoint{4.033434in}{1.953436in}}%
\pgfpathlineto{\pgfqpoint{4.035980in}{1.958472in}}%
\pgfpathlineto{\pgfqpoint{4.053802in}{1.957796in}}%
\pgfpathlineto{\pgfqpoint{4.056348in}{1.963118in}}%
\pgfpathlineto{\pgfqpoint{4.063986in}{1.963118in}}%
\pgfpathlineto{\pgfqpoint{4.066532in}{1.977779in}}%
\pgfpathlineto{\pgfqpoint{4.074170in}{1.977779in}}%
\pgfpathlineto{\pgfqpoint{4.076716in}{1.976000in}}%
\pgfpathlineto{\pgfqpoint{4.084354in}{1.976000in}}%
\pgfpathlineto{\pgfqpoint{4.086900in}{1.969789in}}%
\pgfpathlineto{\pgfqpoint{4.094538in}{1.969789in}}%
\pgfpathlineto{\pgfqpoint{4.097084in}{1.980070in}}%
\pgfpathlineto{\pgfqpoint{4.104723in}{1.980070in}}%
\pgfpathlineto{\pgfqpoint{4.107269in}{1.974144in}}%
\pgfpathlineto{\pgfqpoint{4.125091in}{1.974343in}}%
\pgfpathlineto{\pgfqpoint{4.127637in}{1.983816in}}%
\pgfpathlineto{\pgfqpoint{4.135275in}{1.983816in}}%
\pgfpathlineto{\pgfqpoint{4.137821in}{1.852594in}}%
\pgfpathlineto{\pgfqpoint{4.145459in}{1.852594in}}%
\pgfpathlineto{\pgfqpoint{4.148005in}{1.856296in}}%
\pgfpathlineto{\pgfqpoint{4.155643in}{1.856296in}}%
\pgfpathlineto{\pgfqpoint{4.158189in}{1.853202in}}%
\pgfpathlineto{\pgfqpoint{4.165827in}{1.853202in}}%
\pgfpathlineto{\pgfqpoint{4.168373in}{1.849885in}}%
\pgfpathlineto{\pgfqpoint{4.186195in}{1.849886in}}%
\pgfpathlineto{\pgfqpoint{4.191287in}{1.851114in}}%
\pgfpathlineto{\pgfqpoint{4.196379in}{1.851114in}}%
\pgfpathlineto{\pgfqpoint{4.198926in}{1.853070in}}%
\pgfpathlineto{\pgfqpoint{4.206564in}{1.853070in}}%
\pgfpathlineto{\pgfqpoint{4.209110in}{1.848955in}}%
\pgfpathlineto{\pgfqpoint{4.216748in}{1.848955in}}%
\pgfpathlineto{\pgfqpoint{4.218968in}{2.215329in}}%
\pgfpathmoveto{\pgfqpoint{4.390661in}{2.215329in}}%
\pgfpathlineto{\pgfqpoint{4.392424in}{1.623128in}}%
\pgfpathlineto{\pgfqpoint{4.410246in}{1.622524in}}%
\pgfpathlineto{\pgfqpoint{4.412792in}{1.625428in}}%
\pgfpathlineto{\pgfqpoint{4.420430in}{1.625428in}}%
\pgfpathlineto{\pgfqpoint{4.422976in}{1.622397in}}%
\pgfpathlineto{\pgfqpoint{4.440798in}{1.623252in}}%
\pgfpathlineto{\pgfqpoint{4.445890in}{1.624091in}}%
\pgfpathlineto{\pgfqpoint{4.450982in}{1.624091in}}%
\pgfpathlineto{\pgfqpoint{4.453528in}{1.626954in}}%
\pgfpathlineto{\pgfqpoint{4.471350in}{1.626134in}}%
\pgfpathlineto{\pgfqpoint{4.473896in}{1.629369in}}%
\pgfpathlineto{\pgfqpoint{4.481534in}{1.629369in}}%
\pgfpathlineto{\pgfqpoint{4.484081in}{1.720583in}}%
\pgfpathlineto{\pgfqpoint{4.491719in}{1.720583in}}%
\pgfpathlineto{\pgfqpoint{4.494265in}{1.741237in}}%
\pgfpathlineto{\pgfqpoint{4.522271in}{1.740695in}}%
\pgfpathlineto{\pgfqpoint{4.524817in}{1.749738in}}%
\pgfpathlineto{\pgfqpoint{4.532455in}{1.749738in}}%
\pgfpathlineto{\pgfqpoint{4.535001in}{1.754076in}}%
\pgfpathlineto{\pgfqpoint{4.542639in}{1.754076in}}%
\pgfpathlineto{\pgfqpoint{4.545185in}{1.752721in}}%
\pgfpathlineto{\pgfqpoint{4.552823in}{1.752721in}}%
\pgfpathlineto{\pgfqpoint{4.555369in}{1.754723in}}%
\pgfpathlineto{\pgfqpoint{4.563007in}{1.754723in}}%
\pgfpathlineto{\pgfqpoint{4.565553in}{1.756842in}}%
\pgfpathlineto{\pgfqpoint{4.583376in}{1.755963in}}%
\pgfpathlineto{\pgfqpoint{4.585922in}{1.760567in}}%
\pgfpathlineto{\pgfqpoint{4.593560in}{1.760567in}}%
\pgfpathlineto{\pgfqpoint{4.596106in}{1.768671in}}%
\pgfpathlineto{\pgfqpoint{4.603744in}{1.768671in}}%
\pgfpathlineto{\pgfqpoint{4.606290in}{1.763945in}}%
\pgfpathlineto{\pgfqpoint{4.613928in}{1.763945in}}%
\pgfpathlineto{\pgfqpoint{4.616474in}{1.766077in}}%
\pgfpathlineto{\pgfqpoint{4.624112in}{1.766077in}}%
\pgfpathlineto{\pgfqpoint{4.626658in}{1.771198in}}%
\pgfpathlineto{\pgfqpoint{4.634296in}{1.771198in}}%
\pgfpathlineto{\pgfqpoint{4.636842in}{1.769701in}}%
\pgfpathlineto{\pgfqpoint{4.644480in}{1.769701in}}%
\pgfpathlineto{\pgfqpoint{4.647026in}{1.776581in}}%
\pgfpathlineto{\pgfqpoint{4.664848in}{1.777049in}}%
\pgfpathlineto{\pgfqpoint{4.667394in}{1.779755in}}%
\pgfpathlineto{\pgfqpoint{4.675033in}{1.779755in}}%
\pgfpathlineto{\pgfqpoint{4.680125in}{1.778552in}}%
\pgfpathlineto{\pgfqpoint{4.685217in}{1.778552in}}%
\pgfpathlineto{\pgfqpoint{4.690309in}{1.777400in}}%
\pgfpathlineto{\pgfqpoint{4.705585in}{1.776610in}}%
\pgfpathlineto{\pgfqpoint{4.708131in}{1.785847in}}%
\pgfpathlineto{\pgfqpoint{4.715769in}{1.785847in}}%
\pgfpathlineto{\pgfqpoint{4.718315in}{1.777602in}}%
\pgfpathlineto{\pgfqpoint{4.725953in}{1.777602in}}%
\pgfpathlineto{\pgfqpoint{4.728499in}{1.789458in}}%
\pgfpathlineto{\pgfqpoint{4.736137in}{1.789458in}}%
\pgfpathlineto{\pgfqpoint{4.738683in}{1.661705in}}%
\pgfpathlineto{\pgfqpoint{4.746321in}{1.661705in}}%
\pgfpathlineto{\pgfqpoint{4.748867in}{1.659297in}}%
\pgfpathlineto{\pgfqpoint{4.756505in}{1.659297in}}%
\pgfpathlineto{\pgfqpoint{4.759051in}{1.663695in}}%
\pgfpathlineto{\pgfqpoint{4.766689in}{1.663695in}}%
\pgfpathlineto{\pgfqpoint{4.769236in}{1.666360in}}%
\pgfpathlineto{\pgfqpoint{4.776874in}{1.666360in}}%
\pgfpathlineto{\pgfqpoint{4.779420in}{1.661953in}}%
\pgfpathlineto{\pgfqpoint{4.797242in}{1.662133in}}%
\pgfpathlineto{\pgfqpoint{4.799788in}{1.669862in}}%
\pgfpathlineto{\pgfqpoint{4.807426in}{1.669862in}}%
\pgfpathlineto{\pgfqpoint{4.809972in}{1.672239in}}%
\pgfpathlineto{\pgfqpoint{4.817610in}{1.672239in}}%
\pgfpathlineto{\pgfqpoint{4.820156in}{1.662035in}}%
\pgfpathlineto{\pgfqpoint{4.827794in}{1.662035in}}%
\pgfpathlineto{\pgfqpoint{4.832886in}{1.663253in}}%
\pgfpathlineto{\pgfqpoint{4.899083in}{1.663893in}}%
\pgfpathlineto{\pgfqpoint{4.906721in}{1.664380in}}%
\pgfpathlineto{\pgfqpoint{4.929635in}{1.664400in}}%
\pgfpathlineto{\pgfqpoint{4.932181in}{1.752529in}}%
\pgfpathlineto{\pgfqpoint{4.939819in}{1.752529in}}%
\pgfpathlineto{\pgfqpoint{4.942365in}{1.761003in}}%
\pgfpathlineto{\pgfqpoint{4.950003in}{1.761003in}}%
\pgfpathlineto{\pgfqpoint{4.952549in}{1.779409in}}%
\pgfpathlineto{\pgfqpoint{4.960188in}{1.779409in}}%
\pgfpathlineto{\pgfqpoint{4.962734in}{1.782799in}}%
\pgfpathlineto{\pgfqpoint{4.980556in}{1.783057in}}%
\pgfpathlineto{\pgfqpoint{4.983102in}{1.794422in}}%
\pgfpathlineto{\pgfqpoint{4.990740in}{1.794422in}}%
\pgfpathlineto{\pgfqpoint{4.993286in}{1.790988in}}%
\pgfpathlineto{\pgfqpoint{5.000924in}{1.790988in}}%
\pgfpathlineto{\pgfqpoint{5.003470in}{1.795197in}}%
\pgfpathlineto{\pgfqpoint{5.011108in}{1.795197in}}%
\pgfpathlineto{\pgfqpoint{5.013654in}{1.792161in}}%
\pgfpathlineto{\pgfqpoint{5.021292in}{1.792161in}}%
\pgfpathlineto{\pgfqpoint{5.023838in}{1.794286in}}%
\pgfpathlineto{\pgfqpoint{5.062029in}{1.794497in}}%
\pgfpathlineto{\pgfqpoint{5.064575in}{1.798133in}}%
\pgfpathlineto{\pgfqpoint{5.082397in}{1.798220in}}%
\pgfpathlineto{\pgfqpoint{5.084943in}{1.794368in}}%
\pgfpathlineto{\pgfqpoint{5.102765in}{1.795218in}}%
\pgfpathlineto{\pgfqpoint{5.112949in}{1.795584in}}%
\pgfpathlineto{\pgfqpoint{5.118041in}{1.794800in}}%
\pgfpathlineto{\pgfqpoint{5.123133in}{1.794800in}}%
\pgfpathlineto{\pgfqpoint{5.125679in}{1.797588in}}%
\pgfpathlineto{\pgfqpoint{5.133317in}{1.797588in}}%
\pgfpathlineto{\pgfqpoint{5.135863in}{1.665033in}}%
\pgfpathlineto{\pgfqpoint{5.184238in}{1.664633in}}%
\pgfpathlineto{\pgfqpoint{5.189330in}{1.663409in}}%
\pgfpathlineto{\pgfqpoint{5.265711in}{1.663505in}}%
\pgfpathlineto{\pgfqpoint{5.268257in}{1.670644in}}%
\pgfpathlineto{\pgfqpoint{5.286079in}{1.670634in}}%
\pgfpathlineto{\pgfqpoint{5.288625in}{1.665950in}}%
\pgfpathlineto{\pgfqpoint{5.306447in}{1.666008in}}%
\pgfpathlineto{\pgfqpoint{5.308993in}{1.671815in}}%
\pgfpathlineto{\pgfqpoint{5.337000in}{1.671809in}}%
\pgfpathlineto{\pgfqpoint{5.337000in}{1.671809in}}%
\pgfusepath{stroke}%
\end{pgfscope}%
\begin{pgfscope}%
\pgfpathrectangle{\pgfqpoint{3.593990in}{0.663635in}}{\pgfqpoint{1.826010in}{1.541693in}}%
\pgfusepath{clip}%
\pgfsetbuttcap%
\pgfsetroundjoin%
\pgfsetlinewidth{1.505625pt}%
\definecolor{currentstroke}{rgb}{0.172549,0.627451,0.172549}%
\pgfsetstrokecolor{currentstroke}%
\pgfsetdash{{5.550000pt}{2.400000pt}}{0.000000pt}%
\pgfpathmoveto{\pgfqpoint{3.593990in}{1.691431in}}%
\pgfpathlineto{\pgfqpoint{5.420000in}{1.691431in}}%
\pgfusepath{stroke}%
\end{pgfscope}%
\begin{pgfscope}%
\pgfsetrectcap%
\pgfsetmiterjoin%
\pgfsetlinewidth{1.254687pt}%
\definecolor{currentstroke}{rgb}{1.000000,1.000000,1.000000}%
\pgfsetstrokecolor{currentstroke}%
\pgfsetdash{}{0pt}%
\pgfpathmoveto{\pgfqpoint{3.593990in}{0.663635in}}%
\pgfpathlineto{\pgfqpoint{3.593990in}{2.205329in}}%
\pgfusepath{stroke}%
\end{pgfscope}%
\begin{pgfscope}%
\pgfsetrectcap%
\pgfsetmiterjoin%
\pgfsetlinewidth{1.254687pt}%
\definecolor{currentstroke}{rgb}{1.000000,1.000000,1.000000}%
\pgfsetstrokecolor{currentstroke}%
\pgfsetdash{}{0pt}%
\pgfpathmoveto{\pgfqpoint{5.420000in}{0.663635in}}%
\pgfpathlineto{\pgfqpoint{5.420000in}{2.205329in}}%
\pgfusepath{stroke}%
\end{pgfscope}%
\begin{pgfscope}%
\pgfsetrectcap%
\pgfsetmiterjoin%
\pgfsetlinewidth{1.254687pt}%
\definecolor{currentstroke}{rgb}{1.000000,1.000000,1.000000}%
\pgfsetstrokecolor{currentstroke}%
\pgfsetdash{}{0pt}%
\pgfpathmoveto{\pgfqpoint{3.593990in}{0.663635in}}%
\pgfpathlineto{\pgfqpoint{5.420000in}{0.663635in}}%
\pgfusepath{stroke}%
\end{pgfscope}%
\begin{pgfscope}%
\pgfsetrectcap%
\pgfsetmiterjoin%
\pgfsetlinewidth{1.254687pt}%
\definecolor{currentstroke}{rgb}{1.000000,1.000000,1.000000}%
\pgfsetstrokecolor{currentstroke}%
\pgfsetdash{}{0pt}%
\pgfpathmoveto{\pgfqpoint{3.593990in}{2.205329in}}%
\pgfpathlineto{\pgfqpoint{5.420000in}{2.205329in}}%
\pgfusepath{stroke}%
\end{pgfscope}%
\begin{pgfscope}%
\definecolor{textcolor}{rgb}{0.150000,0.150000,0.150000}%
\pgfsetstrokecolor{textcolor}%
\pgfsetfillcolor{textcolor}%
\pgftext[x=4.506995in,y=2.288662in,,base]{\color{textcolor}{\sffamily\fontsize{12.000000}{14.400000}\selectfont\catcode`\^=\active\def^{\ifmmode\sp\else\^{}\fi}\catcode`\%=\active\def%{\%}d)}}%
\end{pgfscope}%
\begin{pgfscope}%
\pgfsetbuttcap%
\pgfsetmiterjoin%
\definecolor{currentfill}{rgb}{0.917647,0.917647,0.949020}%
\pgfsetfillcolor{currentfill}%
\pgfsetfillopacity{0.800000}%
\pgfsetlinewidth{1.003750pt}%
\definecolor{currentstroke}{rgb}{0.800000,0.800000,0.800000}%
\pgfsetstrokecolor{currentstroke}%
\pgfsetstrokeopacity{0.800000}%
\pgfsetdash{}{0pt}%
\pgfpathmoveto{\pgfqpoint{2.017052in}{4.277160in}}%
\pgfpathlineto{\pgfqpoint{3.582948in}{4.277160in}}%
\pgfpathquadraticcurveto{\pgfqpoint{3.605170in}{4.277160in}}{\pgfqpoint{3.605170in}{4.299382in}}%
\pgfpathlineto{\pgfqpoint{3.605170in}{4.922222in}}%
\pgfpathquadraticcurveto{\pgfqpoint{3.605170in}{4.944444in}}{\pgfqpoint{3.582948in}{4.944444in}}%
\pgfpathlineto{\pgfqpoint{2.017052in}{4.944444in}}%
\pgfpathquadraticcurveto{\pgfqpoint{1.994830in}{4.944444in}}{\pgfqpoint{1.994830in}{4.922222in}}%
\pgfpathlineto{\pgfqpoint{1.994830in}{4.299382in}}%
\pgfpathquadraticcurveto{\pgfqpoint{1.994830in}{4.277160in}}{\pgfqpoint{2.017052in}{4.277160in}}%
\pgfpathlineto{\pgfqpoint{2.017052in}{4.277160in}}%
\pgfpathclose%
\pgfusepath{stroke,fill}%
\end{pgfscope}%
\begin{pgfscope}%
\pgfsetbuttcap%
\pgfsetroundjoin%
\pgfsetlinewidth{1.505625pt}%
\definecolor{currentstroke}{rgb}{1.000000,0.647059,0.000000}%
\pgfsetstrokecolor{currentstroke}%
\pgfsetdash{{1.500000pt}{2.475000pt}}{0.000000pt}%
\pgfpathmoveto{\pgfqpoint{2.039274in}{4.859353in}}%
\pgfpathlineto{\pgfqpoint{2.150385in}{4.859353in}}%
\pgfpathlineto{\pgfqpoint{2.261496in}{4.859353in}}%
\pgfusepath{stroke}%
\end{pgfscope}%
\begin{pgfscope}%
\definecolor{textcolor}{rgb}{0.150000,0.150000,0.150000}%
\pgfsetstrokecolor{textcolor}%
\pgfsetfillcolor{textcolor}%
\pgftext[x=2.350385in,y=4.820464in,left,base]{\color{textcolor}{\sffamily\fontsize{8.000000}{9.600000}\selectfont\catcode`\^=\active\def^{\ifmmode\sp\else\^{}\fi}\catcode`\%=\active\def%{\%}scaling event}}%
\end{pgfscope}%
\begin{pgfscope}%
\pgfsetbuttcap%
\pgfsetroundjoin%
\pgfsetlinewidth{1.505625pt}%
\definecolor{currentstroke}{rgb}{0.580392,0.403922,0.741176}%
\pgfsetstrokecolor{currentstroke}%
\pgfsetdash{{5.550000pt}{2.400000pt}}{0.000000pt}%
\pgfpathmoveto{\pgfqpoint{2.039274in}{4.699523in}}%
\pgfpathlineto{\pgfqpoint{2.150385in}{4.699523in}}%
\pgfpathlineto{\pgfqpoint{2.261496in}{4.699523in}}%
\pgfusepath{stroke}%
\end{pgfscope}%
\begin{pgfscope}%
\definecolor{textcolor}{rgb}{0.150000,0.150000,0.150000}%
\pgfsetstrokecolor{textcolor}%
\pgfsetfillcolor{textcolor}%
\pgftext[x=2.350385in,y=4.660634in,left,base]{\color{textcolor}{\sffamily\fontsize{8.000000}{9.600000}\selectfont\catcode`\^=\active\def^{\ifmmode\sp\else\^{}\fi}\catcode`\%=\active\def%{\%}target CPU utilization}}%
\end{pgfscope}%
\begin{pgfscope}%
\pgfsetbuttcap%
\pgfsetmiterjoin%
\definecolor{currentfill}{rgb}{1.000000,0.000000,0.000000}%
\pgfsetfillcolor{currentfill}%
\pgfsetfillopacity{0.300000}%
\pgfsetlinewidth{1.003750pt}%
\definecolor{currentstroke}{rgb}{1.000000,0.000000,0.000000}%
\pgfsetstrokecolor{currentstroke}%
\pgfsetstrokeopacity{0.300000}%
\pgfsetdash{}{0pt}%
\pgfpathmoveto{\pgfqpoint{2.039274in}{4.502160in}}%
\pgfpathlineto{\pgfqpoint{2.261496in}{4.502160in}}%
\pgfpathlineto{\pgfqpoint{2.261496in}{4.579938in}}%
\pgfpathlineto{\pgfqpoint{2.039274in}{4.579938in}}%
\pgfpathlineto{\pgfqpoint{2.039274in}{4.502160in}}%
\pgfpathclose%
\pgfusepath{stroke,fill}%
\end{pgfscope}%
\begin{pgfscope}%
\definecolor{textcolor}{rgb}{0.150000,0.150000,0.150000}%
\pgfsetstrokecolor{textcolor}%
\pgfsetfillcolor{textcolor}%
\pgftext[x=2.350385in,y=4.502160in,left,base]{\color{textcolor}{\sffamily\fontsize{8.000000}{9.600000}\selectfont\catcode`\^=\active\def^{\ifmmode\sp\else\^{}\fi}\catcode`\%=\active\def%{\%}k8ssandra reconsiliation}}%
\end{pgfscope}%
\begin{pgfscope}%
\pgfsetbuttcap%
\pgfsetroundjoin%
\pgfsetlinewidth{1.505625pt}%
\definecolor{currentstroke}{rgb}{0.172549,0.627451,0.172549}%
\pgfsetstrokecolor{currentstroke}%
\pgfsetdash{{5.550000pt}{2.400000pt}}{0.000000pt}%
\pgfpathmoveto{\pgfqpoint{2.039274in}{4.383877in}}%
\pgfpathlineto{\pgfqpoint{2.150385in}{4.383877in}}%
\pgfpathlineto{\pgfqpoint{2.261496in}{4.383877in}}%
\pgfusepath{stroke}%
\end{pgfscope}%
\begin{pgfscope}%
\definecolor{textcolor}{rgb}{0.150000,0.150000,0.150000}%
\pgfsetstrokecolor{textcolor}%
\pgfsetfillcolor{textcolor}%
\pgftext[x=2.350385in,y=4.344988in,left,base]{\color{textcolor}{\sffamily\fontsize{8.000000}{9.600000}\selectfont\catcode`\^=\active\def^{\ifmmode\sp\else\^{}\fi}\catcode`\%=\active\def%{\%}target memory utilization}}%
\end{pgfscope}%
\end{pgfpicture}%
\makeatother%
\endgroup%

  \caption{a) Write operations per second per node during runs of \texttt{cassandra-stress}\qquad b) Amount of K8ssandra nodes during the test. Scaling actions 2 and 4 perform horizontal scaling\qquad c) CPU Utilization of the K8ssandra cluster\qquad d) Memory utilization of the K8ssandra cluster}
    \label{fig:diagonal-elasticity}
\end{figure}

\begin{figure}[H]
    \centering
    %% Creator: Matplotlib, PGF backend
%%
%% To include the figure in your LaTeX document, write
%%   \input{<filename>.pgf}
%%
%% Make sure the required packages are loaded in your preamble
%%   \usepackage{pgf}
%%
%% Also ensure that all the required font packages are loaded; for instance,
%% the lmodern package is sometimes necessary when using math font.
%%   \usepackage{lmodern}
%%
%% Figures using additional raster images can only be included by \input if
%% they are in the same directory as the main LaTeX file. For loading figures
%% from other directories you can use the `import` package
%%   \usepackage{import}
%%
%% and then include the figures with
%%   \import{<path to file>}{<filename>.pgf}
%%
%% Matplotlib used the following preamble
%%   \def\mathdefault#1{#1}
%%   \everymath=\expandafter{\the\everymath\displaystyle}
%%   
%%   \usepackage{fontspec}
%%   \setmainfont{DejaVuSerif.ttf}[Path=\detokenize{/Users/nkratky/private/polaris-elasticity-strategies/test/scripts/.venv/lib/python3.11/site-packages/matplotlib/mpl-data/fonts/ttf/}]
%%   \setsansfont{Arial.ttf}[Path=\detokenize{/System/Library/Fonts/Supplemental/}]
%%   \setmonofont{DejaVuSansMono.ttf}[Path=\detokenize{/Users/nkratky/private/polaris-elasticity-strategies/test/scripts/.venv/lib/python3.11/site-packages/matplotlib/mpl-data/fonts/ttf/}]
%%   \makeatletter\@ifpackageloaded{underscore}{}{\usepackage[strings]{underscore}}\makeatother
%%
\begingroup%
\makeatletter%
\begin{pgfpicture}%
\pgfpathrectangle{\pgfpointorigin}{\pgfqpoint{5.600000in}{4.000000in}}%
\pgfusepath{use as bounding box, clip}%
\begin{pgfscope}%
\pgfsetbuttcap%
\pgfsetmiterjoin%
\definecolor{currentfill}{rgb}{1.000000,1.000000,1.000000}%
\pgfsetfillcolor{currentfill}%
\pgfsetlinewidth{0.000000pt}%
\definecolor{currentstroke}{rgb}{1.000000,1.000000,1.000000}%
\pgfsetstrokecolor{currentstroke}%
\pgfsetdash{}{0pt}%
\pgfpathmoveto{\pgfqpoint{0.000000in}{0.000000in}}%
\pgfpathlineto{\pgfqpoint{5.600000in}{0.000000in}}%
\pgfpathlineto{\pgfqpoint{5.600000in}{4.000000in}}%
\pgfpathlineto{\pgfqpoint{0.000000in}{4.000000in}}%
\pgfpathlineto{\pgfqpoint{0.000000in}{0.000000in}}%
\pgfpathclose%
\pgfusepath{fill}%
\end{pgfscope}%
\begin{pgfscope}%
\pgfsetbuttcap%
\pgfsetmiterjoin%
\definecolor{currentfill}{rgb}{0.917647,0.917647,0.949020}%
\pgfsetfillcolor{currentfill}%
\pgfsetlinewidth{0.000000pt}%
\definecolor{currentstroke}{rgb}{0.000000,0.000000,0.000000}%
\pgfsetstrokecolor{currentstroke}%
\pgfsetstrokeopacity{0.000000}%
\pgfsetdash{}{0pt}%
\pgfpathmoveto{\pgfqpoint{0.863783in}{2.573635in}}%
\pgfpathlineto{\pgfqpoint{5.420000in}{2.573635in}}%
\pgfpathlineto{\pgfqpoint{5.420000in}{3.765319in}}%
\pgfpathlineto{\pgfqpoint{0.863783in}{3.765319in}}%
\pgfpathlineto{\pgfqpoint{0.863783in}{2.573635in}}%
\pgfpathclose%
\pgfusepath{fill}%
\end{pgfscope}%
\begin{pgfscope}%
\pgfpathrectangle{\pgfqpoint{0.863783in}{2.573635in}}{\pgfqpoint{4.556217in}{1.191684in}}%
\pgfusepath{clip}%
\pgfsetroundcap%
\pgfsetroundjoin%
\pgfsetlinewidth{1.003750pt}%
\definecolor{currentstroke}{rgb}{1.000000,1.000000,1.000000}%
\pgfsetstrokecolor{currentstroke}%
\pgfsetdash{}{0pt}%
\pgfpathmoveto{\pgfqpoint{1.070884in}{2.573635in}}%
\pgfpathlineto{\pgfqpoint{1.070884in}{3.765319in}}%
\pgfusepath{stroke}%
\end{pgfscope}%
\begin{pgfscope}%
\definecolor{textcolor}{rgb}{0.150000,0.150000,0.150000}%
\pgfsetstrokecolor{textcolor}%
\pgfsetfillcolor{textcolor}%
\pgftext[x=1.070884in,y=2.441691in,,top]{\color{textcolor}{\sffamily\fontsize{11.000000}{13.200000}\selectfont\catcode`\^=\active\def^{\ifmmode\sp\else\^{}\fi}\catcode`\%=\active\def%{\%}0}}%
\end{pgfscope}%
\begin{pgfscope}%
\pgfpathrectangle{\pgfqpoint{0.863783in}{2.573635in}}{\pgfqpoint{4.556217in}{1.191684in}}%
\pgfusepath{clip}%
\pgfsetroundcap%
\pgfsetroundjoin%
\pgfsetlinewidth{1.003750pt}%
\definecolor{currentstroke}{rgb}{1.000000,1.000000,1.000000}%
\pgfsetstrokecolor{currentstroke}%
\pgfsetdash{}{0pt}%
\pgfpathmoveto{\pgfqpoint{1.698462in}{2.573635in}}%
\pgfpathlineto{\pgfqpoint{1.698462in}{3.765319in}}%
\pgfusepath{stroke}%
\end{pgfscope}%
\begin{pgfscope}%
\definecolor{textcolor}{rgb}{0.150000,0.150000,0.150000}%
\pgfsetstrokecolor{textcolor}%
\pgfsetfillcolor{textcolor}%
\pgftext[x=1.698462in,y=2.441691in,,top]{\color{textcolor}{\sffamily\fontsize{11.000000}{13.200000}\selectfont\catcode`\^=\active\def^{\ifmmode\sp\else\^{}\fi}\catcode`\%=\active\def%{\%}500}}%
\end{pgfscope}%
\begin{pgfscope}%
\pgfpathrectangle{\pgfqpoint{0.863783in}{2.573635in}}{\pgfqpoint{4.556217in}{1.191684in}}%
\pgfusepath{clip}%
\pgfsetroundcap%
\pgfsetroundjoin%
\pgfsetlinewidth{1.003750pt}%
\definecolor{currentstroke}{rgb}{1.000000,1.000000,1.000000}%
\pgfsetstrokecolor{currentstroke}%
\pgfsetdash{}{0pt}%
\pgfpathmoveto{\pgfqpoint{2.326040in}{2.573635in}}%
\pgfpathlineto{\pgfqpoint{2.326040in}{3.765319in}}%
\pgfusepath{stroke}%
\end{pgfscope}%
\begin{pgfscope}%
\definecolor{textcolor}{rgb}{0.150000,0.150000,0.150000}%
\pgfsetstrokecolor{textcolor}%
\pgfsetfillcolor{textcolor}%
\pgftext[x=2.326040in,y=2.441691in,,top]{\color{textcolor}{\sffamily\fontsize{11.000000}{13.200000}\selectfont\catcode`\^=\active\def^{\ifmmode\sp\else\^{}\fi}\catcode`\%=\active\def%{\%}1000}}%
\end{pgfscope}%
\begin{pgfscope}%
\pgfpathrectangle{\pgfqpoint{0.863783in}{2.573635in}}{\pgfqpoint{4.556217in}{1.191684in}}%
\pgfusepath{clip}%
\pgfsetroundcap%
\pgfsetroundjoin%
\pgfsetlinewidth{1.003750pt}%
\definecolor{currentstroke}{rgb}{1.000000,1.000000,1.000000}%
\pgfsetstrokecolor{currentstroke}%
\pgfsetdash{}{0pt}%
\pgfpathmoveto{\pgfqpoint{2.953618in}{2.573635in}}%
\pgfpathlineto{\pgfqpoint{2.953618in}{3.765319in}}%
\pgfusepath{stroke}%
\end{pgfscope}%
\begin{pgfscope}%
\definecolor{textcolor}{rgb}{0.150000,0.150000,0.150000}%
\pgfsetstrokecolor{textcolor}%
\pgfsetfillcolor{textcolor}%
\pgftext[x=2.953618in,y=2.441691in,,top]{\color{textcolor}{\sffamily\fontsize{11.000000}{13.200000}\selectfont\catcode`\^=\active\def^{\ifmmode\sp\else\^{}\fi}\catcode`\%=\active\def%{\%}1500}}%
\end{pgfscope}%
\begin{pgfscope}%
\pgfpathrectangle{\pgfqpoint{0.863783in}{2.573635in}}{\pgfqpoint{4.556217in}{1.191684in}}%
\pgfusepath{clip}%
\pgfsetroundcap%
\pgfsetroundjoin%
\pgfsetlinewidth{1.003750pt}%
\definecolor{currentstroke}{rgb}{1.000000,1.000000,1.000000}%
\pgfsetstrokecolor{currentstroke}%
\pgfsetdash{}{0pt}%
\pgfpathmoveto{\pgfqpoint{3.581196in}{2.573635in}}%
\pgfpathlineto{\pgfqpoint{3.581196in}{3.765319in}}%
\pgfusepath{stroke}%
\end{pgfscope}%
\begin{pgfscope}%
\definecolor{textcolor}{rgb}{0.150000,0.150000,0.150000}%
\pgfsetstrokecolor{textcolor}%
\pgfsetfillcolor{textcolor}%
\pgftext[x=3.581196in,y=2.441691in,,top]{\color{textcolor}{\sffamily\fontsize{11.000000}{13.200000}\selectfont\catcode`\^=\active\def^{\ifmmode\sp\else\^{}\fi}\catcode`\%=\active\def%{\%}2000}}%
\end{pgfscope}%
\begin{pgfscope}%
\pgfpathrectangle{\pgfqpoint{0.863783in}{2.573635in}}{\pgfqpoint{4.556217in}{1.191684in}}%
\pgfusepath{clip}%
\pgfsetroundcap%
\pgfsetroundjoin%
\pgfsetlinewidth{1.003750pt}%
\definecolor{currentstroke}{rgb}{1.000000,1.000000,1.000000}%
\pgfsetstrokecolor{currentstroke}%
\pgfsetdash{}{0pt}%
\pgfpathmoveto{\pgfqpoint{4.208774in}{2.573635in}}%
\pgfpathlineto{\pgfqpoint{4.208774in}{3.765319in}}%
\pgfusepath{stroke}%
\end{pgfscope}%
\begin{pgfscope}%
\definecolor{textcolor}{rgb}{0.150000,0.150000,0.150000}%
\pgfsetstrokecolor{textcolor}%
\pgfsetfillcolor{textcolor}%
\pgftext[x=4.208774in,y=2.441691in,,top]{\color{textcolor}{\sffamily\fontsize{11.000000}{13.200000}\selectfont\catcode`\^=\active\def^{\ifmmode\sp\else\^{}\fi}\catcode`\%=\active\def%{\%}2500}}%
\end{pgfscope}%
\begin{pgfscope}%
\pgfpathrectangle{\pgfqpoint{0.863783in}{2.573635in}}{\pgfqpoint{4.556217in}{1.191684in}}%
\pgfusepath{clip}%
\pgfsetroundcap%
\pgfsetroundjoin%
\pgfsetlinewidth{1.003750pt}%
\definecolor{currentstroke}{rgb}{1.000000,1.000000,1.000000}%
\pgfsetstrokecolor{currentstroke}%
\pgfsetdash{}{0pt}%
\pgfpathmoveto{\pgfqpoint{4.836352in}{2.573635in}}%
\pgfpathlineto{\pgfqpoint{4.836352in}{3.765319in}}%
\pgfusepath{stroke}%
\end{pgfscope}%
\begin{pgfscope}%
\definecolor{textcolor}{rgb}{0.150000,0.150000,0.150000}%
\pgfsetstrokecolor{textcolor}%
\pgfsetfillcolor{textcolor}%
\pgftext[x=4.836352in,y=2.441691in,,top]{\color{textcolor}{\sffamily\fontsize{11.000000}{13.200000}\selectfont\catcode`\^=\active\def^{\ifmmode\sp\else\^{}\fi}\catcode`\%=\active\def%{\%}3000}}%
\end{pgfscope}%
\begin{pgfscope}%
\definecolor{textcolor}{rgb}{0.150000,0.150000,0.150000}%
\pgfsetstrokecolor{textcolor}%
\pgfsetfillcolor{textcolor}%
\pgftext[x=3.141892in,y=2.246413in,,top]{\color{textcolor}{\sffamily\fontsize{12.000000}{14.400000}\selectfont\catcode`\^=\active\def^{\ifmmode\sp\else\^{}\fi}\catcode`\%=\active\def%{\%}Time (s)}}%
\end{pgfscope}%
\begin{pgfscope}%
\pgfpathrectangle{\pgfqpoint{0.863783in}{2.573635in}}{\pgfqpoint{4.556217in}{1.191684in}}%
\pgfusepath{clip}%
\pgfsetroundcap%
\pgfsetroundjoin%
\pgfsetlinewidth{1.003750pt}%
\definecolor{currentstroke}{rgb}{1.000000,1.000000,1.000000}%
\pgfsetstrokecolor{currentstroke}%
\pgfsetdash{}{0pt}%
\pgfpathmoveto{\pgfqpoint{0.863783in}{2.871556in}}%
\pgfpathlineto{\pgfqpoint{5.420000in}{2.871556in}}%
\pgfusepath{stroke}%
\end{pgfscope}%
\begin{pgfscope}%
\definecolor{textcolor}{rgb}{0.150000,0.150000,0.150000}%
\pgfsetstrokecolor{textcolor}%
\pgfsetfillcolor{textcolor}%
\pgftext[x=0.391968in, y=2.816876in, left, base]{\color{textcolor}{\sffamily\fontsize{11.000000}{13.200000}\selectfont\catcode`\^=\active\def^{\ifmmode\sp\else\^{}\fi}\catcode`\%=\active\def%{\%}1800}}%
\end{pgfscope}%
\begin{pgfscope}%
\pgfpathrectangle{\pgfqpoint{0.863783in}{2.573635in}}{\pgfqpoint{4.556217in}{1.191684in}}%
\pgfusepath{clip}%
\pgfsetroundcap%
\pgfsetroundjoin%
\pgfsetlinewidth{1.003750pt}%
\definecolor{currentstroke}{rgb}{1.000000,1.000000,1.000000}%
\pgfsetstrokecolor{currentstroke}%
\pgfsetdash{}{0pt}%
\pgfpathmoveto{\pgfqpoint{0.863783in}{3.467398in}}%
\pgfpathlineto{\pgfqpoint{5.420000in}{3.467398in}}%
\pgfusepath{stroke}%
\end{pgfscope}%
\begin{pgfscope}%
\definecolor{textcolor}{rgb}{0.150000,0.150000,0.150000}%
\pgfsetstrokecolor{textcolor}%
\pgfsetfillcolor{textcolor}%
\pgftext[x=0.391968in, y=3.412718in, left, base]{\color{textcolor}{\sffamily\fontsize{11.000000}{13.200000}\selectfont\catcode`\^=\active\def^{\ifmmode\sp\else\^{}\fi}\catcode`\%=\active\def%{\%}2000}}%
\end{pgfscope}%
\begin{pgfscope}%
\definecolor{textcolor}{rgb}{0.150000,0.150000,0.150000}%
\pgfsetstrokecolor{textcolor}%
\pgfsetfillcolor{textcolor}%
\pgftext[x=0.336413in,y=3.169477in,,bottom,rotate=90.000000]{\color{textcolor}{\sffamily\fontsize{12.000000}{14.400000}\selectfont\catcode`\^=\active\def^{\ifmmode\sp\else\^{}\fi}\catcode`\%=\active\def%{\%}CPU Limits (milliCPU)}}%
\end{pgfscope}%
\begin{pgfscope}%
\pgfpathrectangle{\pgfqpoint{0.863783in}{2.573635in}}{\pgfqpoint{4.556217in}{1.191684in}}%
\pgfusepath{clip}%
\pgfsetbuttcap%
\pgfsetroundjoin%
\pgfsetlinewidth{1.505625pt}%
\definecolor{currentstroke}{rgb}{1.000000,0.647059,0.000000}%
\pgfsetstrokecolor{currentstroke}%
\pgfsetdash{{1.500000pt}{2.475000pt}}{0.000000pt}%
\pgfpathmoveto{\pgfqpoint{1.427349in}{2.573635in}}%
\pgfpathlineto{\pgfqpoint{1.427349in}{3.765319in}}%
\pgfusepath{stroke}%
\end{pgfscope}%
\begin{pgfscope}%
\pgfpathrectangle{\pgfqpoint{0.863783in}{2.573635in}}{\pgfqpoint{4.556217in}{1.191684in}}%
\pgfusepath{clip}%
\pgfsetbuttcap%
\pgfsetroundjoin%
\pgfsetlinewidth{1.505625pt}%
\definecolor{currentstroke}{rgb}{1.000000,0.647059,0.000000}%
\pgfsetstrokecolor{currentstroke}%
\pgfsetdash{{1.500000pt}{2.475000pt}}{0.000000pt}%
\pgfpathmoveto{\pgfqpoint{2.186718in}{2.573635in}}%
\pgfpathlineto{\pgfqpoint{2.186718in}{3.765319in}}%
\pgfusepath{stroke}%
\end{pgfscope}%
\begin{pgfscope}%
\pgfpathrectangle{\pgfqpoint{0.863783in}{2.573635in}}{\pgfqpoint{4.556217in}{1.191684in}}%
\pgfusepath{clip}%
\pgfsetbuttcap%
\pgfsetroundjoin%
\pgfsetlinewidth{1.505625pt}%
\definecolor{currentstroke}{rgb}{1.000000,0.647059,0.000000}%
\pgfsetstrokecolor{currentstroke}%
\pgfsetdash{{1.500000pt}{2.475000pt}}{0.000000pt}%
\pgfpathmoveto{\pgfqpoint{2.941067in}{2.573635in}}%
\pgfpathlineto{\pgfqpoint{2.941067in}{3.765319in}}%
\pgfusepath{stroke}%
\end{pgfscope}%
\begin{pgfscope}%
\pgfpathrectangle{\pgfqpoint{0.863783in}{2.573635in}}{\pgfqpoint{4.556217in}{1.191684in}}%
\pgfusepath{clip}%
\pgfsetbuttcap%
\pgfsetroundjoin%
\pgfsetlinewidth{1.505625pt}%
\definecolor{currentstroke}{rgb}{1.000000,0.647059,0.000000}%
\pgfsetstrokecolor{currentstroke}%
\pgfsetdash{{1.500000pt}{2.475000pt}}{0.000000pt}%
\pgfpathmoveto{\pgfqpoint{3.696671in}{2.573635in}}%
\pgfpathlineto{\pgfqpoint{3.696671in}{3.765319in}}%
\pgfusepath{stroke}%
\end{pgfscope}%
\begin{pgfscope}%
\pgfpathrectangle{\pgfqpoint{0.863783in}{2.573635in}}{\pgfqpoint{4.556217in}{1.191684in}}%
\pgfusepath{clip}%
\pgfsetbuttcap%
\pgfsetroundjoin%
\pgfsetlinewidth{1.505625pt}%
\definecolor{currentstroke}{rgb}{1.000000,0.647059,0.000000}%
\pgfsetstrokecolor{currentstroke}%
\pgfsetdash{{1.500000pt}{2.475000pt}}{0.000000pt}%
\pgfpathmoveto{\pgfqpoint{4.449764in}{2.573635in}}%
\pgfpathlineto{\pgfqpoint{4.449764in}{3.765319in}}%
\pgfusepath{stroke}%
\end{pgfscope}%
\begin{pgfscope}%
\pgfpathrectangle{\pgfqpoint{0.863783in}{2.573635in}}{\pgfqpoint{4.556217in}{1.191684in}}%
\pgfusepath{clip}%
\pgfsetroundcap%
\pgfsetroundjoin%
\pgfsetlinewidth{1.505625pt}%
\definecolor{currentstroke}{rgb}{0.298039,0.447059,0.690196}%
\pgfsetstrokecolor{currentstroke}%
\pgfsetdash{}{0pt}%
\pgfpathmoveto{\pgfqpoint{1.070884in}{3.467398in}}%
\pgfpathlineto{\pgfqpoint{1.434879in}{3.467398in}}%
\pgfpathlineto{\pgfqpoint{1.441155in}{3.169477in}}%
\pgfpathlineto{\pgfqpoint{5.212899in}{3.169477in}}%
\pgfpathlineto{\pgfqpoint{5.212899in}{3.169477in}}%
\pgfusepath{stroke}%
\end{pgfscope}%
\begin{pgfscope}%
\pgfsetrectcap%
\pgfsetmiterjoin%
\pgfsetlinewidth{1.254687pt}%
\definecolor{currentstroke}{rgb}{1.000000,1.000000,1.000000}%
\pgfsetstrokecolor{currentstroke}%
\pgfsetdash{}{0pt}%
\pgfpathmoveto{\pgfqpoint{0.863783in}{2.573635in}}%
\pgfpathlineto{\pgfqpoint{0.863783in}{3.765319in}}%
\pgfusepath{stroke}%
\end{pgfscope}%
\begin{pgfscope}%
\pgfsetrectcap%
\pgfsetmiterjoin%
\pgfsetlinewidth{1.254687pt}%
\definecolor{currentstroke}{rgb}{1.000000,1.000000,1.000000}%
\pgfsetstrokecolor{currentstroke}%
\pgfsetdash{}{0pt}%
\pgfpathmoveto{\pgfqpoint{5.420000in}{2.573635in}}%
\pgfpathlineto{\pgfqpoint{5.420000in}{3.765319in}}%
\pgfusepath{stroke}%
\end{pgfscope}%
\begin{pgfscope}%
\pgfsetrectcap%
\pgfsetmiterjoin%
\pgfsetlinewidth{1.254687pt}%
\definecolor{currentstroke}{rgb}{1.000000,1.000000,1.000000}%
\pgfsetstrokecolor{currentstroke}%
\pgfsetdash{}{0pt}%
\pgfpathmoveto{\pgfqpoint{0.863783in}{2.573635in}}%
\pgfpathlineto{\pgfqpoint{5.420000in}{2.573635in}}%
\pgfusepath{stroke}%
\end{pgfscope}%
\begin{pgfscope}%
\pgfsetrectcap%
\pgfsetmiterjoin%
\pgfsetlinewidth{1.254687pt}%
\definecolor{currentstroke}{rgb}{1.000000,1.000000,1.000000}%
\pgfsetstrokecolor{currentstroke}%
\pgfsetdash{}{0pt}%
\pgfpathmoveto{\pgfqpoint{0.863783in}{3.765319in}}%
\pgfpathlineto{\pgfqpoint{5.420000in}{3.765319in}}%
\pgfusepath{stroke}%
\end{pgfscope}%
\begin{pgfscope}%
\pgfsetbuttcap%
\pgfsetmiterjoin%
\definecolor{currentfill}{rgb}{0.917647,0.917647,0.949020}%
\pgfsetfillcolor{currentfill}%
\pgfsetlinewidth{0.000000pt}%
\definecolor{currentstroke}{rgb}{0.000000,0.000000,0.000000}%
\pgfsetstrokecolor{currentstroke}%
\pgfsetstrokeopacity{0.000000}%
\pgfsetdash{}{0pt}%
\pgfpathmoveto{\pgfqpoint{0.863783in}{0.663635in}}%
\pgfpathlineto{\pgfqpoint{5.420000in}{0.663635in}}%
\pgfpathlineto{\pgfqpoint{5.420000in}{1.855319in}}%
\pgfpathlineto{\pgfqpoint{0.863783in}{1.855319in}}%
\pgfpathlineto{\pgfqpoint{0.863783in}{0.663635in}}%
\pgfpathclose%
\pgfusepath{fill}%
\end{pgfscope}%
\begin{pgfscope}%
\pgfpathrectangle{\pgfqpoint{0.863783in}{0.663635in}}{\pgfqpoint{4.556217in}{1.191684in}}%
\pgfusepath{clip}%
\pgfsetroundcap%
\pgfsetroundjoin%
\pgfsetlinewidth{1.003750pt}%
\definecolor{currentstroke}{rgb}{1.000000,1.000000,1.000000}%
\pgfsetstrokecolor{currentstroke}%
\pgfsetdash{}{0pt}%
\pgfpathmoveto{\pgfqpoint{1.070884in}{0.663635in}}%
\pgfpathlineto{\pgfqpoint{1.070884in}{1.855319in}}%
\pgfusepath{stroke}%
\end{pgfscope}%
\begin{pgfscope}%
\definecolor{textcolor}{rgb}{0.150000,0.150000,0.150000}%
\pgfsetstrokecolor{textcolor}%
\pgfsetfillcolor{textcolor}%
\pgftext[x=1.070884in,y=0.531691in,,top]{\color{textcolor}{\sffamily\fontsize{11.000000}{13.200000}\selectfont\catcode`\^=\active\def^{\ifmmode\sp\else\^{}\fi}\catcode`\%=\active\def%{\%}0}}%
\end{pgfscope}%
\begin{pgfscope}%
\pgfpathrectangle{\pgfqpoint{0.863783in}{0.663635in}}{\pgfqpoint{4.556217in}{1.191684in}}%
\pgfusepath{clip}%
\pgfsetroundcap%
\pgfsetroundjoin%
\pgfsetlinewidth{1.003750pt}%
\definecolor{currentstroke}{rgb}{1.000000,1.000000,1.000000}%
\pgfsetstrokecolor{currentstroke}%
\pgfsetdash{}{0pt}%
\pgfpathmoveto{\pgfqpoint{1.698462in}{0.663635in}}%
\pgfpathlineto{\pgfqpoint{1.698462in}{1.855319in}}%
\pgfusepath{stroke}%
\end{pgfscope}%
\begin{pgfscope}%
\definecolor{textcolor}{rgb}{0.150000,0.150000,0.150000}%
\pgfsetstrokecolor{textcolor}%
\pgfsetfillcolor{textcolor}%
\pgftext[x=1.698462in,y=0.531691in,,top]{\color{textcolor}{\sffamily\fontsize{11.000000}{13.200000}\selectfont\catcode`\^=\active\def^{\ifmmode\sp\else\^{}\fi}\catcode`\%=\active\def%{\%}500}}%
\end{pgfscope}%
\begin{pgfscope}%
\pgfpathrectangle{\pgfqpoint{0.863783in}{0.663635in}}{\pgfqpoint{4.556217in}{1.191684in}}%
\pgfusepath{clip}%
\pgfsetroundcap%
\pgfsetroundjoin%
\pgfsetlinewidth{1.003750pt}%
\definecolor{currentstroke}{rgb}{1.000000,1.000000,1.000000}%
\pgfsetstrokecolor{currentstroke}%
\pgfsetdash{}{0pt}%
\pgfpathmoveto{\pgfqpoint{2.326040in}{0.663635in}}%
\pgfpathlineto{\pgfqpoint{2.326040in}{1.855319in}}%
\pgfusepath{stroke}%
\end{pgfscope}%
\begin{pgfscope}%
\definecolor{textcolor}{rgb}{0.150000,0.150000,0.150000}%
\pgfsetstrokecolor{textcolor}%
\pgfsetfillcolor{textcolor}%
\pgftext[x=2.326040in,y=0.531691in,,top]{\color{textcolor}{\sffamily\fontsize{11.000000}{13.200000}\selectfont\catcode`\^=\active\def^{\ifmmode\sp\else\^{}\fi}\catcode`\%=\active\def%{\%}1000}}%
\end{pgfscope}%
\begin{pgfscope}%
\pgfpathrectangle{\pgfqpoint{0.863783in}{0.663635in}}{\pgfqpoint{4.556217in}{1.191684in}}%
\pgfusepath{clip}%
\pgfsetroundcap%
\pgfsetroundjoin%
\pgfsetlinewidth{1.003750pt}%
\definecolor{currentstroke}{rgb}{1.000000,1.000000,1.000000}%
\pgfsetstrokecolor{currentstroke}%
\pgfsetdash{}{0pt}%
\pgfpathmoveto{\pgfqpoint{2.953618in}{0.663635in}}%
\pgfpathlineto{\pgfqpoint{2.953618in}{1.855319in}}%
\pgfusepath{stroke}%
\end{pgfscope}%
\begin{pgfscope}%
\definecolor{textcolor}{rgb}{0.150000,0.150000,0.150000}%
\pgfsetstrokecolor{textcolor}%
\pgfsetfillcolor{textcolor}%
\pgftext[x=2.953618in,y=0.531691in,,top]{\color{textcolor}{\sffamily\fontsize{11.000000}{13.200000}\selectfont\catcode`\^=\active\def^{\ifmmode\sp\else\^{}\fi}\catcode`\%=\active\def%{\%}1500}}%
\end{pgfscope}%
\begin{pgfscope}%
\pgfpathrectangle{\pgfqpoint{0.863783in}{0.663635in}}{\pgfqpoint{4.556217in}{1.191684in}}%
\pgfusepath{clip}%
\pgfsetroundcap%
\pgfsetroundjoin%
\pgfsetlinewidth{1.003750pt}%
\definecolor{currentstroke}{rgb}{1.000000,1.000000,1.000000}%
\pgfsetstrokecolor{currentstroke}%
\pgfsetdash{}{0pt}%
\pgfpathmoveto{\pgfqpoint{3.581196in}{0.663635in}}%
\pgfpathlineto{\pgfqpoint{3.581196in}{1.855319in}}%
\pgfusepath{stroke}%
\end{pgfscope}%
\begin{pgfscope}%
\definecolor{textcolor}{rgb}{0.150000,0.150000,0.150000}%
\pgfsetstrokecolor{textcolor}%
\pgfsetfillcolor{textcolor}%
\pgftext[x=3.581196in,y=0.531691in,,top]{\color{textcolor}{\sffamily\fontsize{11.000000}{13.200000}\selectfont\catcode`\^=\active\def^{\ifmmode\sp\else\^{}\fi}\catcode`\%=\active\def%{\%}2000}}%
\end{pgfscope}%
\begin{pgfscope}%
\pgfpathrectangle{\pgfqpoint{0.863783in}{0.663635in}}{\pgfqpoint{4.556217in}{1.191684in}}%
\pgfusepath{clip}%
\pgfsetroundcap%
\pgfsetroundjoin%
\pgfsetlinewidth{1.003750pt}%
\definecolor{currentstroke}{rgb}{1.000000,1.000000,1.000000}%
\pgfsetstrokecolor{currentstroke}%
\pgfsetdash{}{0pt}%
\pgfpathmoveto{\pgfqpoint{4.208774in}{0.663635in}}%
\pgfpathlineto{\pgfqpoint{4.208774in}{1.855319in}}%
\pgfusepath{stroke}%
\end{pgfscope}%
\begin{pgfscope}%
\definecolor{textcolor}{rgb}{0.150000,0.150000,0.150000}%
\pgfsetstrokecolor{textcolor}%
\pgfsetfillcolor{textcolor}%
\pgftext[x=4.208774in,y=0.531691in,,top]{\color{textcolor}{\sffamily\fontsize{11.000000}{13.200000}\selectfont\catcode`\^=\active\def^{\ifmmode\sp\else\^{}\fi}\catcode`\%=\active\def%{\%}2500}}%
\end{pgfscope}%
\begin{pgfscope}%
\pgfpathrectangle{\pgfqpoint{0.863783in}{0.663635in}}{\pgfqpoint{4.556217in}{1.191684in}}%
\pgfusepath{clip}%
\pgfsetroundcap%
\pgfsetroundjoin%
\pgfsetlinewidth{1.003750pt}%
\definecolor{currentstroke}{rgb}{1.000000,1.000000,1.000000}%
\pgfsetstrokecolor{currentstroke}%
\pgfsetdash{}{0pt}%
\pgfpathmoveto{\pgfqpoint{4.836352in}{0.663635in}}%
\pgfpathlineto{\pgfqpoint{4.836352in}{1.855319in}}%
\pgfusepath{stroke}%
\end{pgfscope}%
\begin{pgfscope}%
\definecolor{textcolor}{rgb}{0.150000,0.150000,0.150000}%
\pgfsetstrokecolor{textcolor}%
\pgfsetfillcolor{textcolor}%
\pgftext[x=4.836352in,y=0.531691in,,top]{\color{textcolor}{\sffamily\fontsize{11.000000}{13.200000}\selectfont\catcode`\^=\active\def^{\ifmmode\sp\else\^{}\fi}\catcode`\%=\active\def%{\%}3000}}%
\end{pgfscope}%
\begin{pgfscope}%
\definecolor{textcolor}{rgb}{0.150000,0.150000,0.150000}%
\pgfsetstrokecolor{textcolor}%
\pgfsetfillcolor{textcolor}%
\pgftext[x=3.141892in,y=0.336413in,,top]{\color{textcolor}{\sffamily\fontsize{12.000000}{14.400000}\selectfont\catcode`\^=\active\def^{\ifmmode\sp\else\^{}\fi}\catcode`\%=\active\def%{\%}Time (s)}}%
\end{pgfscope}%
\begin{pgfscope}%
\pgfpathrectangle{\pgfqpoint{0.863783in}{0.663635in}}{\pgfqpoint{4.556217in}{1.191684in}}%
\pgfusepath{clip}%
\pgfsetroundcap%
\pgfsetroundjoin%
\pgfsetlinewidth{1.003750pt}%
\definecolor{currentstroke}{rgb}{1.000000,1.000000,1.000000}%
\pgfsetstrokecolor{currentstroke}%
\pgfsetdash{}{0pt}%
\pgfpathmoveto{\pgfqpoint{0.863783in}{0.961556in}}%
\pgfpathlineto{\pgfqpoint{5.420000in}{0.961556in}}%
\pgfusepath{stroke}%
\end{pgfscope}%
\begin{pgfscope}%
\definecolor{textcolor}{rgb}{0.150000,0.150000,0.150000}%
\pgfsetstrokecolor{textcolor}%
\pgfsetfillcolor{textcolor}%
\pgftext[x=0.391968in, y=0.906876in, left, base]{\color{textcolor}{\sffamily\fontsize{11.000000}{13.200000}\selectfont\catcode`\^=\active\def^{\ifmmode\sp\else\^{}\fi}\catcode`\%=\active\def%{\%}4000}}%
\end{pgfscope}%
\begin{pgfscope}%
\pgfpathrectangle{\pgfqpoint{0.863783in}{0.663635in}}{\pgfqpoint{4.556217in}{1.191684in}}%
\pgfusepath{clip}%
\pgfsetroundcap%
\pgfsetroundjoin%
\pgfsetlinewidth{1.003750pt}%
\definecolor{currentstroke}{rgb}{1.000000,1.000000,1.000000}%
\pgfsetstrokecolor{currentstroke}%
\pgfsetdash{}{0pt}%
\pgfpathmoveto{\pgfqpoint{0.863783in}{1.557398in}}%
\pgfpathlineto{\pgfqpoint{5.420000in}{1.557398in}}%
\pgfusepath{stroke}%
\end{pgfscope}%
\begin{pgfscope}%
\definecolor{textcolor}{rgb}{0.150000,0.150000,0.150000}%
\pgfsetstrokecolor{textcolor}%
\pgfsetfillcolor{textcolor}%
\pgftext[x=0.391968in, y=1.502718in, left, base]{\color{textcolor}{\sffamily\fontsize{11.000000}{13.200000}\selectfont\catcode`\^=\active\def^{\ifmmode\sp\else\^{}\fi}\catcode`\%=\active\def%{\%}6000}}%
\end{pgfscope}%
\begin{pgfscope}%
\definecolor{textcolor}{rgb}{0.150000,0.150000,0.150000}%
\pgfsetstrokecolor{textcolor}%
\pgfsetfillcolor{textcolor}%
\pgftext[x=0.336413in,y=1.259477in,,bottom,rotate=90.000000]{\color{textcolor}{\sffamily\fontsize{12.000000}{14.400000}\selectfont\catcode`\^=\active\def^{\ifmmode\sp\else\^{}\fi}\catcode`\%=\active\def%{\%}Memory Limits (MiB)}}%
\end{pgfscope}%
\begin{pgfscope}%
\pgfpathrectangle{\pgfqpoint{0.863783in}{0.663635in}}{\pgfqpoint{4.556217in}{1.191684in}}%
\pgfusepath{clip}%
\pgfsetbuttcap%
\pgfsetroundjoin%
\pgfsetlinewidth{1.505625pt}%
\definecolor{currentstroke}{rgb}{1.000000,0.647059,0.000000}%
\pgfsetstrokecolor{currentstroke}%
\pgfsetdash{{1.500000pt}{2.475000pt}}{0.000000pt}%
\pgfpathmoveto{\pgfqpoint{1.427349in}{0.663635in}}%
\pgfpathlineto{\pgfqpoint{1.427349in}{1.855319in}}%
\pgfusepath{stroke}%
\end{pgfscope}%
\begin{pgfscope}%
\pgfpathrectangle{\pgfqpoint{0.863783in}{0.663635in}}{\pgfqpoint{4.556217in}{1.191684in}}%
\pgfusepath{clip}%
\pgfsetbuttcap%
\pgfsetroundjoin%
\pgfsetlinewidth{1.505625pt}%
\definecolor{currentstroke}{rgb}{1.000000,0.647059,0.000000}%
\pgfsetstrokecolor{currentstroke}%
\pgfsetdash{{1.500000pt}{2.475000pt}}{0.000000pt}%
\pgfpathmoveto{\pgfqpoint{2.186718in}{0.663635in}}%
\pgfpathlineto{\pgfqpoint{2.186718in}{1.855319in}}%
\pgfusepath{stroke}%
\end{pgfscope}%
\begin{pgfscope}%
\pgfpathrectangle{\pgfqpoint{0.863783in}{0.663635in}}{\pgfqpoint{4.556217in}{1.191684in}}%
\pgfusepath{clip}%
\pgfsetbuttcap%
\pgfsetroundjoin%
\pgfsetlinewidth{1.505625pt}%
\definecolor{currentstroke}{rgb}{1.000000,0.647059,0.000000}%
\pgfsetstrokecolor{currentstroke}%
\pgfsetdash{{1.500000pt}{2.475000pt}}{0.000000pt}%
\pgfpathmoveto{\pgfqpoint{2.941067in}{0.663635in}}%
\pgfpathlineto{\pgfqpoint{2.941067in}{1.855319in}}%
\pgfusepath{stroke}%
\end{pgfscope}%
\begin{pgfscope}%
\pgfpathrectangle{\pgfqpoint{0.863783in}{0.663635in}}{\pgfqpoint{4.556217in}{1.191684in}}%
\pgfusepath{clip}%
\pgfsetbuttcap%
\pgfsetroundjoin%
\pgfsetlinewidth{1.505625pt}%
\definecolor{currentstroke}{rgb}{1.000000,0.647059,0.000000}%
\pgfsetstrokecolor{currentstroke}%
\pgfsetdash{{1.500000pt}{2.475000pt}}{0.000000pt}%
\pgfpathmoveto{\pgfqpoint{3.696671in}{0.663635in}}%
\pgfpathlineto{\pgfqpoint{3.696671in}{1.855319in}}%
\pgfusepath{stroke}%
\end{pgfscope}%
\begin{pgfscope}%
\pgfpathrectangle{\pgfqpoint{0.863783in}{0.663635in}}{\pgfqpoint{4.556217in}{1.191684in}}%
\pgfusepath{clip}%
\pgfsetbuttcap%
\pgfsetroundjoin%
\pgfsetlinewidth{1.505625pt}%
\definecolor{currentstroke}{rgb}{1.000000,0.647059,0.000000}%
\pgfsetstrokecolor{currentstroke}%
\pgfsetdash{{1.500000pt}{2.475000pt}}{0.000000pt}%
\pgfpathmoveto{\pgfqpoint{4.449764in}{0.663635in}}%
\pgfpathlineto{\pgfqpoint{4.449764in}{1.855319in}}%
\pgfusepath{stroke}%
\end{pgfscope}%
\begin{pgfscope}%
\pgfpathrectangle{\pgfqpoint{0.863783in}{0.663635in}}{\pgfqpoint{4.556217in}{1.191684in}}%
\pgfusepath{clip}%
\pgfsetroundcap%
\pgfsetroundjoin%
\pgfsetlinewidth{1.505625pt}%
\definecolor{currentstroke}{rgb}{0.298039,0.447059,0.690196}%
\pgfsetstrokecolor{currentstroke}%
\pgfsetdash{}{0pt}%
\pgfpathmoveto{\pgfqpoint{1.070884in}{1.663296in}}%
\pgfpathlineto{\pgfqpoint{1.434879in}{1.663296in}}%
\pgfpathlineto{\pgfqpoint{1.441155in}{0.988528in}}%
\pgfpathlineto{\pgfqpoint{2.457832in}{0.988528in}}%
\pgfpathlineto{\pgfqpoint{2.464107in}{1.276554in}}%
\pgfpathlineto{\pgfqpoint{5.212899in}{1.276554in}}%
\pgfpathlineto{\pgfqpoint{5.212899in}{1.276554in}}%
\pgfusepath{stroke}%
\end{pgfscope}%
\begin{pgfscope}%
\pgfsetrectcap%
\pgfsetmiterjoin%
\pgfsetlinewidth{1.254687pt}%
\definecolor{currentstroke}{rgb}{1.000000,1.000000,1.000000}%
\pgfsetstrokecolor{currentstroke}%
\pgfsetdash{}{0pt}%
\pgfpathmoveto{\pgfqpoint{0.863783in}{0.663635in}}%
\pgfpathlineto{\pgfqpoint{0.863783in}{1.855319in}}%
\pgfusepath{stroke}%
\end{pgfscope}%
\begin{pgfscope}%
\pgfsetrectcap%
\pgfsetmiterjoin%
\pgfsetlinewidth{1.254687pt}%
\definecolor{currentstroke}{rgb}{1.000000,1.000000,1.000000}%
\pgfsetstrokecolor{currentstroke}%
\pgfsetdash{}{0pt}%
\pgfpathmoveto{\pgfqpoint{5.420000in}{0.663635in}}%
\pgfpathlineto{\pgfqpoint{5.420000in}{1.855319in}}%
\pgfusepath{stroke}%
\end{pgfscope}%
\begin{pgfscope}%
\pgfsetrectcap%
\pgfsetmiterjoin%
\pgfsetlinewidth{1.254687pt}%
\definecolor{currentstroke}{rgb}{1.000000,1.000000,1.000000}%
\pgfsetstrokecolor{currentstroke}%
\pgfsetdash{}{0pt}%
\pgfpathmoveto{\pgfqpoint{0.863783in}{0.663635in}}%
\pgfpathlineto{\pgfqpoint{5.420000in}{0.663635in}}%
\pgfusepath{stroke}%
\end{pgfscope}%
\begin{pgfscope}%
\pgfsetrectcap%
\pgfsetmiterjoin%
\pgfsetlinewidth{1.254687pt}%
\definecolor{currentstroke}{rgb}{1.000000,1.000000,1.000000}%
\pgfsetstrokecolor{currentstroke}%
\pgfsetdash{}{0pt}%
\pgfpathmoveto{\pgfqpoint{0.863783in}{1.855319in}}%
\pgfpathlineto{\pgfqpoint{5.420000in}{1.855319in}}%
\pgfusepath{stroke}%
\end{pgfscope}%
\begin{pgfscope}%
\pgfsetbuttcap%
\pgfsetmiterjoin%
\definecolor{currentfill}{rgb}{0.917647,0.917647,0.949020}%
\pgfsetfillcolor{currentfill}%
\pgfsetfillopacity{0.800000}%
\pgfsetlinewidth{1.003750pt}%
\definecolor{currentstroke}{rgb}{0.800000,0.800000,0.800000}%
\pgfsetstrokecolor{currentstroke}%
\pgfsetstrokeopacity{0.800000}%
\pgfsetdash{}{0pt}%
\pgfpathmoveto{\pgfqpoint{4.518121in}{3.752637in}}%
\pgfpathlineto{\pgfqpoint{5.522222in}{3.752637in}}%
\pgfpathquadraticcurveto{\pgfqpoint{5.544444in}{3.752637in}}{\pgfqpoint{5.544444in}{3.774859in}}%
\pgfpathlineto{\pgfqpoint{5.544444in}{3.922222in}}%
\pgfpathquadraticcurveto{\pgfqpoint{5.544444in}{3.944444in}}{\pgfqpoint{5.522222in}{3.944444in}}%
\pgfpathlineto{\pgfqpoint{4.518121in}{3.944444in}}%
\pgfpathquadraticcurveto{\pgfqpoint{4.495898in}{3.944444in}}{\pgfqpoint{4.495898in}{3.922222in}}%
\pgfpathlineto{\pgfqpoint{4.495898in}{3.774859in}}%
\pgfpathquadraticcurveto{\pgfqpoint{4.495898in}{3.752637in}}{\pgfqpoint{4.518121in}{3.752637in}}%
\pgfpathlineto{\pgfqpoint{4.518121in}{3.752637in}}%
\pgfpathclose%
\pgfusepath{stroke,fill}%
\end{pgfscope}%
\begin{pgfscope}%
\pgfsetbuttcap%
\pgfsetroundjoin%
\pgfsetlinewidth{1.505625pt}%
\definecolor{currentstroke}{rgb}{1.000000,0.647059,0.000000}%
\pgfsetstrokecolor{currentstroke}%
\pgfsetdash{{1.500000pt}{2.475000pt}}{0.000000pt}%
\pgfpathmoveto{\pgfqpoint{4.540343in}{3.859353in}}%
\pgfpathlineto{\pgfqpoint{4.651454in}{3.859353in}}%
\pgfpathlineto{\pgfqpoint{4.762565in}{3.859353in}}%
\pgfusepath{stroke}%
\end{pgfscope}%
\begin{pgfscope}%
\definecolor{textcolor}{rgb}{0.150000,0.150000,0.150000}%
\pgfsetstrokecolor{textcolor}%
\pgfsetfillcolor{textcolor}%
\pgftext[x=4.851454in,y=3.820464in,left,base]{\color{textcolor}{\sffamily\fontsize{8.000000}{9.600000}\selectfont\catcode`\^=\active\def^{\ifmmode\sp\else\^{}\fi}\catcode`\%=\active\def%{\%}scaling event}}%
\end{pgfscope}%
\end{pgfpicture}%
\makeatother%
\endgroup%

    \caption{CPU and memory limits while the diagonal elasticity strategy controller is running}
    \label{fig:diagonal-elasticity-limits}
\end{figure}

\subsection{Diagonal Elasticity Strategy}

As explained earlier, the diagonal elasticity strategy combines the capabilites of the vertical and horizontal elasticity strategy into one single elasticity strategy.

\Cref{fig:diagonal-elasticity} summarizes all metrics into a single illustration. The starting configuration was set to be a single K8ssandra node with resources of 2 CPUs and 6GB of memory. After starting the elasticity strategy controller it can be seen in \Cref{fig:diagonal-elasticity-limits} that the controller immediatly reduces both CPU and memory resources. The reason for that can be seen in \Cref{fig:diagonal-elasticity}c and \Cref{fig:diagonal-elasticity}d. Right at the start, both CPU and memory utilization was not within the tolerance range of the target utilization. Therefore both CPU and memory limits were reduced. After the inital adjustment, the CPU utilization was still far away from the targeted amount. That is because the CPU resources hit the statically set lower bounds. The memory utilization however climbed above the targeted amount, therefore it was reduced again during the second scaling action.

Similarly to \Cref{sec:evaluation-vertical-elasticity}, during Cassandra reconsiliation metrics are not very useful. This is again highlighted in red in \Cref{fig:diagonal-elasticity}.

During the second scaling action it can be seen that vertical and horizontal scaling indeed can happen simultaneously. In \Cref{fig:diagonal-elasticity}b the node count increased to 2, whereas in \Cref{fig:diagonal-elasticity-limits} the memory limits increased. Note, that the \texttt{k8ssandra-operator} adjusts those values one at a time. This means that first the second K8ssandra node is started and then both Pods will get its resources updated accordingly.

Horizontal scaling actions are taken when the write load per node reaches a certain threshold, 5000 in this example. In \Cref{fig:diagonal-elasticity}a it can be seen that after the second, third and fourth scaling event, the K8ssandra cluster size is increased, thus an additional node is started. During the fifth and last scaling event no additional node is started, because the statically set maximum amount of nodes is reached. It is also visible, that the total write throughput increases with increasing node count. This can be further illustrated by multiplying the estimated peak write load with the current node count, as depicted in \Cref{fig:diagonal-elasticity_cluster-write-load}. \(18000 * 1 = 18000,\ 12000 * 2 = 24000,\ 9000 * 3 = 27000 \rightarrow 18000 < 24000 < 27000\).

\begin{figure}
    \centering
    %% Creator: Matplotlib, PGF backend
%%
%% To include the figure in your LaTeX document, write
%%   \input{<filename>.pgf}
%%
%% Make sure the required packages are loaded in your preamble
%%   \usepackage{pgf}
%%
%% Also ensure that all the required font packages are loaded; for instance,
%% the lmodern package is sometimes necessary when using math font.
%%   \usepackage{lmodern}
%%
%% Figures using additional raster images can only be included by \input if
%% they are in the same directory as the main LaTeX file. For loading figures
%% from other directories you can use the `import` package
%%   \usepackage{import}
%%
%% and then include the figures with
%%   \import{<path to file>}{<filename>.pgf}
%%
%% Matplotlib used the following preamble
%%   \def\mathdefault#1{#1}
%%   \everymath=\expandafter{\the\everymath\displaystyle}
%%   
%%   \usepackage{fontspec}
%%   \setmainfont{DejaVuSerif.ttf}[Path=\detokenize{/usr/local/lib/python3.11/site-packages/matplotlib/mpl-data/fonts/ttf/}]
%%   \setsansfont{Arial.ttf}[Path=\detokenize{/System/Library/Fonts/Supplemental/}]
%%   \setmonofont{DejaVuSansMono.ttf}[Path=\detokenize{/usr/local/lib/python3.11/site-packages/matplotlib/mpl-data/fonts/ttf/}]
%%   \makeatletter\@ifpackageloaded{underscore}{}{\usepackage[strings]{underscore}}\makeatother
%%
\begingroup%
\makeatletter%
\begin{pgfpicture}%
\pgfpathrectangle{\pgfpointorigin}{\pgfqpoint{4.500000in}{3.000000in}}%
\pgfusepath{use as bounding box, clip}%
\begin{pgfscope}%
\pgfsetbuttcap%
\pgfsetmiterjoin%
\definecolor{currentfill}{rgb}{1.000000,1.000000,1.000000}%
\pgfsetfillcolor{currentfill}%
\pgfsetlinewidth{0.000000pt}%
\definecolor{currentstroke}{rgb}{1.000000,1.000000,1.000000}%
\pgfsetstrokecolor{currentstroke}%
\pgfsetdash{}{0pt}%
\pgfpathmoveto{\pgfqpoint{0.000000in}{0.000000in}}%
\pgfpathlineto{\pgfqpoint{4.500000in}{0.000000in}}%
\pgfpathlineto{\pgfqpoint{4.500000in}{3.000000in}}%
\pgfpathlineto{\pgfqpoint{0.000000in}{3.000000in}}%
\pgfpathlineto{\pgfqpoint{0.000000in}{0.000000in}}%
\pgfpathclose%
\pgfusepath{fill}%
\end{pgfscope}%
\begin{pgfscope}%
\pgfsetbuttcap%
\pgfsetmiterjoin%
\definecolor{currentfill}{rgb}{0.917647,0.917647,0.949020}%
\pgfsetfillcolor{currentfill}%
\pgfsetlinewidth{0.000000pt}%
\definecolor{currentstroke}{rgb}{0.000000,0.000000,0.000000}%
\pgfsetstrokecolor{currentstroke}%
\pgfsetstrokeopacity{0.000000}%
\pgfsetdash{}{0pt}%
\pgfpathmoveto{\pgfqpoint{0.946717in}{0.663635in}}%
\pgfpathlineto{\pgfqpoint{4.320000in}{0.663635in}}%
\pgfpathlineto{\pgfqpoint{4.320000in}{2.820000in}}%
\pgfpathlineto{\pgfqpoint{0.946717in}{2.820000in}}%
\pgfpathlineto{\pgfqpoint{0.946717in}{0.663635in}}%
\pgfpathclose%
\pgfusepath{fill}%
\end{pgfscope}%
\begin{pgfscope}%
\pgfpathrectangle{\pgfqpoint{0.946717in}{0.663635in}}{\pgfqpoint{3.373283in}{2.156365in}}%
\pgfusepath{clip}%
\pgfsetroundcap%
\pgfsetroundjoin%
\pgfsetlinewidth{1.003750pt}%
\definecolor{currentstroke}{rgb}{1.000000,1.000000,1.000000}%
\pgfsetstrokecolor{currentstroke}%
\pgfsetdash{}{0pt}%
\pgfpathmoveto{\pgfqpoint{1.100048in}{0.663635in}}%
\pgfpathlineto{\pgfqpoint{1.100048in}{2.820000in}}%
\pgfusepath{stroke}%
\end{pgfscope}%
\begin{pgfscope}%
\definecolor{textcolor}{rgb}{0.150000,0.150000,0.150000}%
\pgfsetstrokecolor{textcolor}%
\pgfsetfillcolor{textcolor}%
\pgftext[x=1.100048in,y=0.531691in,,top]{\color{textcolor}{\sffamily\fontsize{11.000000}{13.200000}\selectfont\catcode`\^=\active\def^{\ifmmode\sp\else\^{}\fi}\catcode`\%=\active\def%{\%}0}}%
\end{pgfscope}%
\begin{pgfscope}%
\pgfpathrectangle{\pgfqpoint{0.946717in}{0.663635in}}{\pgfqpoint{3.373283in}{2.156365in}}%
\pgfusepath{clip}%
\pgfsetroundcap%
\pgfsetroundjoin%
\pgfsetlinewidth{1.003750pt}%
\definecolor{currentstroke}{rgb}{1.000000,1.000000,1.000000}%
\pgfsetstrokecolor{currentstroke}%
\pgfsetdash{}{0pt}%
\pgfpathmoveto{\pgfqpoint{2.040729in}{0.663635in}}%
\pgfpathlineto{\pgfqpoint{2.040729in}{2.820000in}}%
\pgfusepath{stroke}%
\end{pgfscope}%
\begin{pgfscope}%
\definecolor{textcolor}{rgb}{0.150000,0.150000,0.150000}%
\pgfsetstrokecolor{textcolor}%
\pgfsetfillcolor{textcolor}%
\pgftext[x=2.040729in,y=0.531691in,,top]{\color{textcolor}{\sffamily\fontsize{11.000000}{13.200000}\selectfont\catcode`\^=\active\def^{\ifmmode\sp\else\^{}\fi}\catcode`\%=\active\def%{\%}1000}}%
\end{pgfscope}%
\begin{pgfscope}%
\pgfpathrectangle{\pgfqpoint{0.946717in}{0.663635in}}{\pgfqpoint{3.373283in}{2.156365in}}%
\pgfusepath{clip}%
\pgfsetroundcap%
\pgfsetroundjoin%
\pgfsetlinewidth{1.003750pt}%
\definecolor{currentstroke}{rgb}{1.000000,1.000000,1.000000}%
\pgfsetstrokecolor{currentstroke}%
\pgfsetdash{}{0pt}%
\pgfpathmoveto{\pgfqpoint{2.981410in}{0.663635in}}%
\pgfpathlineto{\pgfqpoint{2.981410in}{2.820000in}}%
\pgfusepath{stroke}%
\end{pgfscope}%
\begin{pgfscope}%
\definecolor{textcolor}{rgb}{0.150000,0.150000,0.150000}%
\pgfsetstrokecolor{textcolor}%
\pgfsetfillcolor{textcolor}%
\pgftext[x=2.981410in,y=0.531691in,,top]{\color{textcolor}{\sffamily\fontsize{11.000000}{13.200000}\selectfont\catcode`\^=\active\def^{\ifmmode\sp\else\^{}\fi}\catcode`\%=\active\def%{\%}2000}}%
\end{pgfscope}%
\begin{pgfscope}%
\pgfpathrectangle{\pgfqpoint{0.946717in}{0.663635in}}{\pgfqpoint{3.373283in}{2.156365in}}%
\pgfusepath{clip}%
\pgfsetroundcap%
\pgfsetroundjoin%
\pgfsetlinewidth{1.003750pt}%
\definecolor{currentstroke}{rgb}{1.000000,1.000000,1.000000}%
\pgfsetstrokecolor{currentstroke}%
\pgfsetdash{}{0pt}%
\pgfpathmoveto{\pgfqpoint{3.922092in}{0.663635in}}%
\pgfpathlineto{\pgfqpoint{3.922092in}{2.820000in}}%
\pgfusepath{stroke}%
\end{pgfscope}%
\begin{pgfscope}%
\definecolor{textcolor}{rgb}{0.150000,0.150000,0.150000}%
\pgfsetstrokecolor{textcolor}%
\pgfsetfillcolor{textcolor}%
\pgftext[x=3.922092in,y=0.531691in,,top]{\color{textcolor}{\sffamily\fontsize{11.000000}{13.200000}\selectfont\catcode`\^=\active\def^{\ifmmode\sp\else\^{}\fi}\catcode`\%=\active\def%{\%}3000}}%
\end{pgfscope}%
\begin{pgfscope}%
\definecolor{textcolor}{rgb}{0.150000,0.150000,0.150000}%
\pgfsetstrokecolor{textcolor}%
\pgfsetfillcolor{textcolor}%
\pgftext[x=2.633358in,y=0.336413in,,top]{\color{textcolor}{\sffamily\fontsize{12.000000}{14.400000}\selectfont\catcode`\^=\active\def^{\ifmmode\sp\else\^{}\fi}\catcode`\%=\active\def%{\%}Time (s)}}%
\end{pgfscope}%
\begin{pgfscope}%
\pgfpathrectangle{\pgfqpoint{0.946717in}{0.663635in}}{\pgfqpoint{3.373283in}{2.156365in}}%
\pgfusepath{clip}%
\pgfsetroundcap%
\pgfsetroundjoin%
\pgfsetlinewidth{1.003750pt}%
\definecolor{currentstroke}{rgb}{1.000000,1.000000,1.000000}%
\pgfsetstrokecolor{currentstroke}%
\pgfsetdash{}{0pt}%
\pgfpathmoveto{\pgfqpoint{0.946717in}{0.761652in}}%
\pgfpathlineto{\pgfqpoint{4.320000in}{0.761652in}}%
\pgfusepath{stroke}%
\end{pgfscope}%
\begin{pgfscope}%
\definecolor{textcolor}{rgb}{0.150000,0.150000,0.150000}%
\pgfsetstrokecolor{textcolor}%
\pgfsetfillcolor{textcolor}%
\pgftext[x=0.729804in, y=0.706971in, left, base]{\color{textcolor}{\sffamily\fontsize{11.000000}{13.200000}\selectfont\catcode`\^=\active\def^{\ifmmode\sp\else\^{}\fi}\catcode`\%=\active\def%{\%}0}}%
\end{pgfscope}%
\begin{pgfscope}%
\pgfpathrectangle{\pgfqpoint{0.946717in}{0.663635in}}{\pgfqpoint{3.373283in}{2.156365in}}%
\pgfusepath{clip}%
\pgfsetroundcap%
\pgfsetroundjoin%
\pgfsetlinewidth{1.003750pt}%
\definecolor{currentstroke}{rgb}{1.000000,1.000000,1.000000}%
\pgfsetstrokecolor{currentstroke}%
\pgfsetdash{}{0pt}%
\pgfpathmoveto{\pgfqpoint{0.946717in}{1.140359in}}%
\pgfpathlineto{\pgfqpoint{4.320000in}{1.140359in}}%
\pgfusepath{stroke}%
\end{pgfscope}%
\begin{pgfscope}%
\definecolor{textcolor}{rgb}{0.150000,0.150000,0.150000}%
\pgfsetstrokecolor{textcolor}%
\pgfsetfillcolor{textcolor}%
\pgftext[x=0.474901in, y=1.085678in, left, base]{\color{textcolor}{\sffamily\fontsize{11.000000}{13.200000}\selectfont\catcode`\^=\active\def^{\ifmmode\sp\else\^{}\fi}\catcode`\%=\active\def%{\%}5000}}%
\end{pgfscope}%
\begin{pgfscope}%
\pgfpathrectangle{\pgfqpoint{0.946717in}{0.663635in}}{\pgfqpoint{3.373283in}{2.156365in}}%
\pgfusepath{clip}%
\pgfsetroundcap%
\pgfsetroundjoin%
\pgfsetlinewidth{1.003750pt}%
\definecolor{currentstroke}{rgb}{1.000000,1.000000,1.000000}%
\pgfsetstrokecolor{currentstroke}%
\pgfsetdash{}{0pt}%
\pgfpathmoveto{\pgfqpoint{0.946717in}{1.519065in}}%
\pgfpathlineto{\pgfqpoint{4.320000in}{1.519065in}}%
\pgfusepath{stroke}%
\end{pgfscope}%
\begin{pgfscope}%
\definecolor{textcolor}{rgb}{0.150000,0.150000,0.150000}%
\pgfsetstrokecolor{textcolor}%
\pgfsetfillcolor{textcolor}%
\pgftext[x=0.389934in, y=1.464385in, left, base]{\color{textcolor}{\sffamily\fontsize{11.000000}{13.200000}\selectfont\catcode`\^=\active\def^{\ifmmode\sp\else\^{}\fi}\catcode`\%=\active\def%{\%}10000}}%
\end{pgfscope}%
\begin{pgfscope}%
\pgfpathrectangle{\pgfqpoint{0.946717in}{0.663635in}}{\pgfqpoint{3.373283in}{2.156365in}}%
\pgfusepath{clip}%
\pgfsetroundcap%
\pgfsetroundjoin%
\pgfsetlinewidth{1.003750pt}%
\definecolor{currentstroke}{rgb}{1.000000,1.000000,1.000000}%
\pgfsetstrokecolor{currentstroke}%
\pgfsetdash{}{0pt}%
\pgfpathmoveto{\pgfqpoint{0.946717in}{1.897772in}}%
\pgfpathlineto{\pgfqpoint{4.320000in}{1.897772in}}%
\pgfusepath{stroke}%
\end{pgfscope}%
\begin{pgfscope}%
\definecolor{textcolor}{rgb}{0.150000,0.150000,0.150000}%
\pgfsetstrokecolor{textcolor}%
\pgfsetfillcolor{textcolor}%
\pgftext[x=0.389934in, y=1.843091in, left, base]{\color{textcolor}{\sffamily\fontsize{11.000000}{13.200000}\selectfont\catcode`\^=\active\def^{\ifmmode\sp\else\^{}\fi}\catcode`\%=\active\def%{\%}15000}}%
\end{pgfscope}%
\begin{pgfscope}%
\pgfpathrectangle{\pgfqpoint{0.946717in}{0.663635in}}{\pgfqpoint{3.373283in}{2.156365in}}%
\pgfusepath{clip}%
\pgfsetroundcap%
\pgfsetroundjoin%
\pgfsetlinewidth{1.003750pt}%
\definecolor{currentstroke}{rgb}{1.000000,1.000000,1.000000}%
\pgfsetstrokecolor{currentstroke}%
\pgfsetdash{}{0pt}%
\pgfpathmoveto{\pgfqpoint{0.946717in}{2.276479in}}%
\pgfpathlineto{\pgfqpoint{4.320000in}{2.276479in}}%
\pgfusepath{stroke}%
\end{pgfscope}%
\begin{pgfscope}%
\definecolor{textcolor}{rgb}{0.150000,0.150000,0.150000}%
\pgfsetstrokecolor{textcolor}%
\pgfsetfillcolor{textcolor}%
\pgftext[x=0.389934in, y=2.221798in, left, base]{\color{textcolor}{\sffamily\fontsize{11.000000}{13.200000}\selectfont\catcode`\^=\active\def^{\ifmmode\sp\else\^{}\fi}\catcode`\%=\active\def%{\%}20000}}%
\end{pgfscope}%
\begin{pgfscope}%
\pgfpathrectangle{\pgfqpoint{0.946717in}{0.663635in}}{\pgfqpoint{3.373283in}{2.156365in}}%
\pgfusepath{clip}%
\pgfsetroundcap%
\pgfsetroundjoin%
\pgfsetlinewidth{1.003750pt}%
\definecolor{currentstroke}{rgb}{1.000000,1.000000,1.000000}%
\pgfsetstrokecolor{currentstroke}%
\pgfsetdash{}{0pt}%
\pgfpathmoveto{\pgfqpoint{0.946717in}{2.655186in}}%
\pgfpathlineto{\pgfqpoint{4.320000in}{2.655186in}}%
\pgfusepath{stroke}%
\end{pgfscope}%
\begin{pgfscope}%
\definecolor{textcolor}{rgb}{0.150000,0.150000,0.150000}%
\pgfsetstrokecolor{textcolor}%
\pgfsetfillcolor{textcolor}%
\pgftext[x=0.389934in, y=2.600505in, left, base]{\color{textcolor}{\sffamily\fontsize{11.000000}{13.200000}\selectfont\catcode`\^=\active\def^{\ifmmode\sp\else\^{}\fi}\catcode`\%=\active\def%{\%}25000}}%
\end{pgfscope}%
\begin{pgfscope}%
\definecolor{textcolor}{rgb}{0.150000,0.150000,0.150000}%
\pgfsetstrokecolor{textcolor}%
\pgfsetfillcolor{textcolor}%
\pgftext[x=0.334378in,y=1.741818in,,bottom,rotate=90.000000]{\color{textcolor}{\sffamily\fontsize{12.000000}{14.400000}\selectfont\catcode`\^=\active\def^{\ifmmode\sp\else\^{}\fi}\catcode`\%=\active\def%{\%}Average Write Load}}%
\end{pgfscope}%
\begin{pgfscope}%
\pgfpathrectangle{\pgfqpoint{0.946717in}{0.663635in}}{\pgfqpoint{3.373283in}{2.156365in}}%
\pgfusepath{clip}%
\pgfsetroundcap%
\pgfsetroundjoin%
\pgfsetlinewidth{1.505625pt}%
\definecolor{currentstroke}{rgb}{0.298039,0.447059,0.690196}%
\pgfsetstrokecolor{currentstroke}%
\pgfsetdash{}{0pt}%
\pgfpathmoveto{\pgfqpoint{1.100048in}{0.761652in}}%
\pgfpathlineto{\pgfqpoint{1.683270in}{0.761652in}}%
\pgfpathlineto{\pgfqpoint{1.687974in}{0.776834in}}%
\pgfpathlineto{\pgfqpoint{1.702084in}{0.776834in}}%
\pgfpathlineto{\pgfqpoint{1.706787in}{0.849810in}}%
\pgfpathlineto{\pgfqpoint{1.720897in}{0.849810in}}%
\pgfpathlineto{\pgfqpoint{1.725601in}{0.947167in}}%
\pgfpathlineto{\pgfqpoint{1.777338in}{0.948075in}}%
\pgfpathlineto{\pgfqpoint{1.782042in}{0.949642in}}%
\pgfpathlineto{\pgfqpoint{1.796152in}{0.949642in}}%
\pgfpathlineto{\pgfqpoint{1.800855in}{1.064784in}}%
\pgfpathlineto{\pgfqpoint{1.833779in}{1.064784in}}%
\pgfpathlineto{\pgfqpoint{1.838483in}{1.333385in}}%
\pgfpathlineto{\pgfqpoint{1.852593in}{1.333385in}}%
\pgfpathlineto{\pgfqpoint{1.857296in}{1.537627in}}%
\pgfpathlineto{\pgfqpoint{1.871406in}{1.537627in}}%
\pgfpathlineto{\pgfqpoint{1.876110in}{1.657685in}}%
\pgfpathlineto{\pgfqpoint{1.890220in}{1.657685in}}%
\pgfpathlineto{\pgfqpoint{1.894923in}{1.813260in}}%
\pgfpathlineto{\pgfqpoint{1.909034in}{1.813260in}}%
\pgfpathlineto{\pgfqpoint{1.913737in}{1.900142in}}%
\pgfpathlineto{\pgfqpoint{1.927847in}{1.900142in}}%
\pgfpathlineto{\pgfqpoint{1.932551in}{2.026726in}}%
\pgfpathlineto{\pgfqpoint{1.946661in}{2.026726in}}%
\pgfpathlineto{\pgfqpoint{1.951364in}{2.190517in}}%
\pgfpathlineto{\pgfqpoint{1.965475in}{2.190517in}}%
\pgfpathlineto{\pgfqpoint{1.970178in}{2.125260in}}%
\pgfpathlineto{\pgfqpoint{1.984288in}{2.125260in}}%
\pgfpathlineto{\pgfqpoint{1.988992in}{2.009439in}}%
\pgfpathlineto{\pgfqpoint{2.003102in}{2.009439in}}%
\pgfpathlineto{\pgfqpoint{2.007805in}{1.770490in}}%
\pgfpathlineto{\pgfqpoint{2.021915in}{1.770490in}}%
\pgfpathlineto{\pgfqpoint{2.026619in}{1.791082in}}%
\pgfpathlineto{\pgfqpoint{2.040729in}{1.791082in}}%
\pgfpathlineto{\pgfqpoint{2.045432in}{1.739223in}}%
\pgfpathlineto{\pgfqpoint{2.059543in}{1.739223in}}%
\pgfpathlineto{\pgfqpoint{2.064246in}{1.614613in}}%
\pgfpathlineto{\pgfqpoint{2.078356in}{1.614613in}}%
\pgfpathlineto{\pgfqpoint{2.083060in}{1.573488in}}%
\pgfpathlineto{\pgfqpoint{2.097170in}{1.573488in}}%
\pgfpathlineto{\pgfqpoint{2.101873in}{1.413296in}}%
\pgfpathlineto{\pgfqpoint{2.115984in}{1.413296in}}%
\pgfpathlineto{\pgfqpoint{2.120687in}{1.238869in}}%
\pgfpathlineto{\pgfqpoint{2.134797in}{1.238869in}}%
\pgfpathlineto{\pgfqpoint{2.139501in}{1.215640in}}%
\pgfpathlineto{\pgfqpoint{2.153611in}{1.215640in}}%
\pgfpathlineto{\pgfqpoint{2.158314in}{1.032241in}}%
\pgfpathlineto{\pgfqpoint{2.172424in}{1.032241in}}%
\pgfpathlineto{\pgfqpoint{2.177128in}{0.806857in}}%
\pgfpathlineto{\pgfqpoint{2.191238in}{0.806857in}}%
\pgfpathlineto{\pgfqpoint{2.195942in}{0.803239in}}%
\pgfpathlineto{\pgfqpoint{2.210052in}{0.803239in}}%
\pgfpathlineto{\pgfqpoint{2.214755in}{0.761653in}}%
\pgfpathlineto{\pgfqpoint{2.586324in}{0.761652in}}%
\pgfpathlineto{\pgfqpoint{2.591028in}{0.768403in}}%
\pgfpathlineto{\pgfqpoint{2.605138in}{0.768403in}}%
\pgfpathlineto{\pgfqpoint{2.609841in}{0.787986in}}%
\pgfpathlineto{\pgfqpoint{2.623952in}{0.787986in}}%
\pgfpathlineto{\pgfqpoint{2.628655in}{0.812029in}}%
\pgfpathlineto{\pgfqpoint{2.642765in}{0.812029in}}%
\pgfpathlineto{\pgfqpoint{2.647469in}{0.872180in}}%
\pgfpathlineto{\pgfqpoint{2.661579in}{0.872180in}}%
\pgfpathlineto{\pgfqpoint{2.666282in}{0.914935in}}%
\pgfpathlineto{\pgfqpoint{2.680392in}{0.914935in}}%
\pgfpathlineto{\pgfqpoint{2.685096in}{0.956859in}}%
\pgfpathlineto{\pgfqpoint{2.699206in}{0.956859in}}%
\pgfpathlineto{\pgfqpoint{2.703909in}{1.052846in}}%
\pgfpathlineto{\pgfqpoint{2.718020in}{1.052846in}}%
\pgfpathlineto{\pgfqpoint{2.722723in}{1.115515in}}%
\pgfpathlineto{\pgfqpoint{2.736833in}{1.115515in}}%
\pgfpathlineto{\pgfqpoint{2.741537in}{1.111243in}}%
\pgfpathlineto{\pgfqpoint{2.755647in}{1.111243in}}%
\pgfpathlineto{\pgfqpoint{2.760350in}{1.217857in}}%
\pgfpathlineto{\pgfqpoint{2.774461in}{1.217857in}}%
\pgfpathlineto{\pgfqpoint{2.779164in}{1.323026in}}%
\pgfpathlineto{\pgfqpoint{2.793274in}{1.323026in}}%
\pgfpathlineto{\pgfqpoint{2.797978in}{1.315818in}}%
\pgfpathlineto{\pgfqpoint{2.812088in}{1.315818in}}%
\pgfpathlineto{\pgfqpoint{2.816791in}{1.374726in}}%
\pgfpathlineto{\pgfqpoint{2.830901in}{1.374726in}}%
\pgfpathlineto{\pgfqpoint{2.835605in}{1.407722in}}%
\pgfpathlineto{\pgfqpoint{2.849715in}{1.407722in}}%
\pgfpathlineto{\pgfqpoint{2.854418in}{1.458081in}}%
\pgfpathlineto{\pgfqpoint{2.868529in}{1.458081in}}%
\pgfpathlineto{\pgfqpoint{2.873232in}{1.610312in}}%
\pgfpathlineto{\pgfqpoint{2.887342in}{1.610312in}}%
\pgfpathlineto{\pgfqpoint{2.892046in}{1.636412in}}%
\pgfpathlineto{\pgfqpoint{2.906156in}{1.636412in}}%
\pgfpathlineto{\pgfqpoint{2.910859in}{1.596305in}}%
\pgfpathlineto{\pgfqpoint{2.924970in}{1.596305in}}%
\pgfpathlineto{\pgfqpoint{2.929673in}{1.680467in}}%
\pgfpathlineto{\pgfqpoint{2.943783in}{1.680467in}}%
\pgfpathlineto{\pgfqpoint{2.948487in}{1.691157in}}%
\pgfpathlineto{\pgfqpoint{2.962597in}{1.691157in}}%
\pgfpathlineto{\pgfqpoint{2.967300in}{1.647606in}}%
\pgfpathlineto{\pgfqpoint{2.981410in}{1.647606in}}%
\pgfpathlineto{\pgfqpoint{2.986114in}{1.701596in}}%
\pgfpathlineto{\pgfqpoint{3.000224in}{1.701596in}}%
\pgfpathlineto{\pgfqpoint{3.004927in}{1.705954in}}%
\pgfpathlineto{\pgfqpoint{3.019038in}{1.705954in}}%
\pgfpathlineto{\pgfqpoint{3.023741in}{1.664193in}}%
\pgfpathlineto{\pgfqpoint{3.037851in}{1.664193in}}%
\pgfpathlineto{\pgfqpoint{3.042555in}{1.701641in}}%
\pgfpathlineto{\pgfqpoint{3.056665in}{1.701641in}}%
\pgfpathlineto{\pgfqpoint{3.061368in}{1.626552in}}%
\pgfpathlineto{\pgfqpoint{3.075479in}{1.626552in}}%
\pgfpathlineto{\pgfqpoint{3.080182in}{1.646955in}}%
\pgfpathlineto{\pgfqpoint{3.094292in}{1.646955in}}%
\pgfpathlineto{\pgfqpoint{3.098996in}{1.577748in}}%
\pgfpathlineto{\pgfqpoint{3.113106in}{1.577748in}}%
\pgfpathlineto{\pgfqpoint{3.117809in}{1.492830in}}%
\pgfpathlineto{\pgfqpoint{3.131919in}{1.492830in}}%
\pgfpathlineto{\pgfqpoint{3.136623in}{1.436935in}}%
\pgfpathlineto{\pgfqpoint{3.150733in}{1.436935in}}%
\pgfpathlineto{\pgfqpoint{3.155436in}{1.362468in}}%
\pgfpathlineto{\pgfqpoint{3.169547in}{1.362468in}}%
\pgfpathlineto{\pgfqpoint{3.174250in}{1.263557in}}%
\pgfpathlineto{\pgfqpoint{3.188360in}{1.263557in}}%
\pgfpathlineto{\pgfqpoint{3.193064in}{1.203846in}}%
\pgfpathlineto{\pgfqpoint{3.207174in}{1.203846in}}%
\pgfpathlineto{\pgfqpoint{3.211877in}{1.007045in}}%
\pgfpathlineto{\pgfqpoint{3.225988in}{1.007045in}}%
\pgfpathlineto{\pgfqpoint{3.230691in}{0.939138in}}%
\pgfpathlineto{\pgfqpoint{3.244801in}{0.939138in}}%
\pgfpathlineto{\pgfqpoint{3.249505in}{0.905013in}}%
\pgfpathlineto{\pgfqpoint{3.263615in}{0.905013in}}%
\pgfpathlineto{\pgfqpoint{3.268318in}{0.866428in}}%
\pgfpathlineto{\pgfqpoint{3.282428in}{0.866428in}}%
\pgfpathlineto{\pgfqpoint{3.287132in}{0.804974in}}%
\pgfpathlineto{\pgfqpoint{3.301242in}{0.804974in}}%
\pgfpathlineto{\pgfqpoint{3.305946in}{0.778888in}}%
\pgfpathlineto{\pgfqpoint{3.320056in}{0.778888in}}%
\pgfpathlineto{\pgfqpoint{3.324759in}{0.761653in}}%
\pgfpathlineto{\pgfqpoint{3.432937in}{0.761653in}}%
\pgfpathlineto{\pgfqpoint{3.437641in}{0.773900in}}%
\pgfpathlineto{\pgfqpoint{3.451751in}{0.773900in}}%
\pgfpathlineto{\pgfqpoint{3.456455in}{0.791255in}}%
\pgfpathlineto{\pgfqpoint{3.470565in}{0.791255in}}%
\pgfpathlineto{\pgfqpoint{3.475268in}{0.828964in}}%
\pgfpathlineto{\pgfqpoint{3.489378in}{0.828964in}}%
\pgfpathlineto{\pgfqpoint{3.494082in}{0.893981in}}%
\pgfpathlineto{\pgfqpoint{3.508192in}{0.893981in}}%
\pgfpathlineto{\pgfqpoint{3.512895in}{0.948619in}}%
\pgfpathlineto{\pgfqpoint{3.527006in}{0.948619in}}%
\pgfpathlineto{\pgfqpoint{3.531709in}{0.979550in}}%
\pgfpathlineto{\pgfqpoint{3.545819in}{0.979550in}}%
\pgfpathlineto{\pgfqpoint{3.550523in}{1.054431in}}%
\pgfpathlineto{\pgfqpoint{3.564633in}{1.054431in}}%
\pgfpathlineto{\pgfqpoint{3.569336in}{1.108016in}}%
\pgfpathlineto{\pgfqpoint{3.583447in}{1.108016in}}%
\pgfpathlineto{\pgfqpoint{3.588150in}{1.142903in}}%
\pgfpathlineto{\pgfqpoint{3.602260in}{1.142903in}}%
\pgfpathlineto{\pgfqpoint{3.606964in}{1.202790in}}%
\pgfpathlineto{\pgfqpoint{3.621074in}{1.202790in}}%
\pgfpathlineto{\pgfqpoint{3.625777in}{1.234124in}}%
\pgfpathlineto{\pgfqpoint{3.639887in}{1.234124in}}%
\pgfpathlineto{\pgfqpoint{3.644591in}{1.256435in}}%
\pgfpathlineto{\pgfqpoint{3.658701in}{1.256435in}}%
\pgfpathlineto{\pgfqpoint{3.663404in}{1.295333in}}%
\pgfpathlineto{\pgfqpoint{3.677515in}{1.295333in}}%
\pgfpathlineto{\pgfqpoint{3.682218in}{1.343507in}}%
\pgfpathlineto{\pgfqpoint{3.696328in}{1.343507in}}%
\pgfpathlineto{\pgfqpoint{3.701032in}{1.394155in}}%
\pgfpathlineto{\pgfqpoint{3.715142in}{1.394155in}}%
\pgfpathlineto{\pgfqpoint{3.719845in}{1.415096in}}%
\pgfpathlineto{\pgfqpoint{3.733956in}{1.415096in}}%
\pgfpathlineto{\pgfqpoint{3.738659in}{1.407210in}}%
\pgfpathlineto{\pgfqpoint{3.752769in}{1.407210in}}%
\pgfpathlineto{\pgfqpoint{3.757473in}{1.396747in}}%
\pgfpathlineto{\pgfqpoint{3.771583in}{1.396747in}}%
\pgfpathlineto{\pgfqpoint{3.776286in}{1.405605in}}%
\pgfpathlineto{\pgfqpoint{3.790396in}{1.405605in}}%
\pgfpathlineto{\pgfqpoint{3.795100in}{1.360494in}}%
\pgfpathlineto{\pgfqpoint{3.809210in}{1.360494in}}%
\pgfpathlineto{\pgfqpoint{3.813913in}{1.338261in}}%
\pgfpathlineto{\pgfqpoint{3.828024in}{1.338261in}}%
\pgfpathlineto{\pgfqpoint{3.832727in}{1.279630in}}%
\pgfpathlineto{\pgfqpoint{3.846837in}{1.279630in}}%
\pgfpathlineto{\pgfqpoint{3.851541in}{1.215892in}}%
\pgfpathlineto{\pgfqpoint{3.865651in}{1.215892in}}%
\pgfpathlineto{\pgfqpoint{3.870354in}{1.189589in}}%
\pgfpathlineto{\pgfqpoint{3.884465in}{1.189589in}}%
\pgfpathlineto{\pgfqpoint{3.889168in}{1.140204in}}%
\pgfpathlineto{\pgfqpoint{3.903278in}{1.140204in}}%
\pgfpathlineto{\pgfqpoint{3.907982in}{1.057413in}}%
\pgfpathlineto{\pgfqpoint{3.922092in}{1.057413in}}%
\pgfpathlineto{\pgfqpoint{3.926795in}{1.033989in}}%
\pgfpathlineto{\pgfqpoint{3.940905in}{1.033989in}}%
\pgfpathlineto{\pgfqpoint{3.945609in}{1.003434in}}%
\pgfpathlineto{\pgfqpoint{3.959719in}{1.003434in}}%
\pgfpathlineto{\pgfqpoint{3.964422in}{0.934291in}}%
\pgfpathlineto{\pgfqpoint{3.978533in}{0.934291in}}%
\pgfpathlineto{\pgfqpoint{3.983236in}{0.907525in}}%
\pgfpathlineto{\pgfqpoint{3.997346in}{0.907525in}}%
\pgfpathlineto{\pgfqpoint{4.002050in}{0.860492in}}%
\pgfpathlineto{\pgfqpoint{4.016160in}{0.860492in}}%
\pgfpathlineto{\pgfqpoint{4.020863in}{0.794463in}}%
\pgfpathlineto{\pgfqpoint{4.034974in}{0.794463in}}%
\pgfpathlineto{\pgfqpoint{4.039677in}{0.776930in}}%
\pgfpathlineto{\pgfqpoint{4.053787in}{0.776930in}}%
\pgfpathlineto{\pgfqpoint{4.058491in}{0.761652in}}%
\pgfpathlineto{\pgfqpoint{4.166669in}{0.761652in}}%
\pgfpathlineto{\pgfqpoint{4.166669in}{0.761652in}}%
\pgfusepath{stroke}%
\end{pgfscope}%
\begin{pgfscope}%
\pgfpathrectangle{\pgfqpoint{0.946717in}{0.663635in}}{\pgfqpoint{3.373283in}{2.156365in}}%
\pgfusepath{clip}%
\pgfsetroundcap%
\pgfsetroundjoin%
\pgfsetlinewidth{1.505625pt}%
\definecolor{currentstroke}{rgb}{0.866667,0.517647,0.321569}%
\pgfsetstrokecolor{currentstroke}%
\pgfsetdash{}{0pt}%
\pgfpathmoveto{\pgfqpoint{1.100048in}{0.761652in}}%
\pgfpathlineto{\pgfqpoint{1.683270in}{0.761652in}}%
\pgfpathlineto{\pgfqpoint{1.687974in}{0.776834in}}%
\pgfpathlineto{\pgfqpoint{1.702084in}{0.776834in}}%
\pgfpathlineto{\pgfqpoint{1.706787in}{0.849810in}}%
\pgfpathlineto{\pgfqpoint{1.720897in}{0.849810in}}%
\pgfpathlineto{\pgfqpoint{1.725601in}{0.947167in}}%
\pgfpathlineto{\pgfqpoint{1.777338in}{0.948075in}}%
\pgfpathlineto{\pgfqpoint{1.782042in}{0.949642in}}%
\pgfpathlineto{\pgfqpoint{1.796152in}{0.949642in}}%
\pgfpathlineto{\pgfqpoint{1.800855in}{1.064784in}}%
\pgfpathlineto{\pgfqpoint{1.833779in}{1.064784in}}%
\pgfpathlineto{\pgfqpoint{1.838483in}{1.333385in}}%
\pgfpathlineto{\pgfqpoint{1.852593in}{1.333385in}}%
\pgfpathlineto{\pgfqpoint{1.857296in}{1.537627in}}%
\pgfpathlineto{\pgfqpoint{1.871406in}{1.537627in}}%
\pgfpathlineto{\pgfqpoint{1.876110in}{1.657685in}}%
\pgfpathlineto{\pgfqpoint{1.890220in}{1.657685in}}%
\pgfpathlineto{\pgfqpoint{1.894923in}{1.813260in}}%
\pgfpathlineto{\pgfqpoint{1.909034in}{1.813260in}}%
\pgfpathlineto{\pgfqpoint{1.913737in}{1.900142in}}%
\pgfpathlineto{\pgfqpoint{1.927847in}{1.900142in}}%
\pgfpathlineto{\pgfqpoint{1.932551in}{2.026726in}}%
\pgfpathlineto{\pgfqpoint{1.946661in}{2.026726in}}%
\pgfpathlineto{\pgfqpoint{1.951364in}{2.190517in}}%
\pgfpathlineto{\pgfqpoint{1.965475in}{2.190517in}}%
\pgfpathlineto{\pgfqpoint{1.970178in}{2.125260in}}%
\pgfpathlineto{\pgfqpoint{1.984288in}{2.125260in}}%
\pgfpathlineto{\pgfqpoint{1.988992in}{2.009439in}}%
\pgfpathlineto{\pgfqpoint{2.003102in}{2.009439in}}%
\pgfpathlineto{\pgfqpoint{2.007805in}{1.770490in}}%
\pgfpathlineto{\pgfqpoint{2.021915in}{1.770490in}}%
\pgfpathlineto{\pgfqpoint{2.026619in}{1.791082in}}%
\pgfpathlineto{\pgfqpoint{2.040729in}{1.791082in}}%
\pgfpathlineto{\pgfqpoint{2.045432in}{1.739223in}}%
\pgfpathlineto{\pgfqpoint{2.059543in}{1.739223in}}%
\pgfpathlineto{\pgfqpoint{2.064246in}{1.614613in}}%
\pgfpathlineto{\pgfqpoint{2.078356in}{1.614613in}}%
\pgfpathlineto{\pgfqpoint{2.083060in}{1.573488in}}%
\pgfpathlineto{\pgfqpoint{2.097170in}{1.573488in}}%
\pgfpathlineto{\pgfqpoint{2.101873in}{1.413296in}}%
\pgfpathlineto{\pgfqpoint{2.115984in}{1.413296in}}%
\pgfpathlineto{\pgfqpoint{2.120687in}{1.238869in}}%
\pgfpathlineto{\pgfqpoint{2.134797in}{1.238869in}}%
\pgfpathlineto{\pgfqpoint{2.139501in}{1.215640in}}%
\pgfpathlineto{\pgfqpoint{2.153611in}{1.215640in}}%
\pgfpathlineto{\pgfqpoint{2.158314in}{1.032241in}}%
\pgfpathlineto{\pgfqpoint{2.172424in}{1.032241in}}%
\pgfpathlineto{\pgfqpoint{2.177128in}{0.806857in}}%
\pgfpathlineto{\pgfqpoint{2.191238in}{0.806857in}}%
\pgfpathlineto{\pgfqpoint{2.195942in}{0.803239in}}%
\pgfpathlineto{\pgfqpoint{2.210052in}{0.803239in}}%
\pgfpathlineto{\pgfqpoint{2.214755in}{0.761654in}}%
\pgfpathlineto{\pgfqpoint{2.586324in}{0.761652in}}%
\pgfpathlineto{\pgfqpoint{2.591028in}{0.775153in}}%
\pgfpathlineto{\pgfqpoint{2.605138in}{0.775153in}}%
\pgfpathlineto{\pgfqpoint{2.609841in}{0.814319in}}%
\pgfpathlineto{\pgfqpoint{2.623952in}{0.814319in}}%
\pgfpathlineto{\pgfqpoint{2.628655in}{0.862407in}}%
\pgfpathlineto{\pgfqpoint{2.642765in}{0.862407in}}%
\pgfpathlineto{\pgfqpoint{2.647469in}{0.982708in}}%
\pgfpathlineto{\pgfqpoint{2.661579in}{0.982708in}}%
\pgfpathlineto{\pgfqpoint{2.666282in}{1.068219in}}%
\pgfpathlineto{\pgfqpoint{2.680392in}{1.068219in}}%
\pgfpathlineto{\pgfqpoint{2.685096in}{1.152066in}}%
\pgfpathlineto{\pgfqpoint{2.699206in}{1.152066in}}%
\pgfpathlineto{\pgfqpoint{2.703909in}{1.344040in}}%
\pgfpathlineto{\pgfqpoint{2.718020in}{1.344040in}}%
\pgfpathlineto{\pgfqpoint{2.722723in}{1.469378in}}%
\pgfpathlineto{\pgfqpoint{2.736833in}{1.469378in}}%
\pgfpathlineto{\pgfqpoint{2.741537in}{1.460834in}}%
\pgfpathlineto{\pgfqpoint{2.755647in}{1.460834in}}%
\pgfpathlineto{\pgfqpoint{2.760350in}{1.674063in}}%
\pgfpathlineto{\pgfqpoint{2.774461in}{1.674063in}}%
\pgfpathlineto{\pgfqpoint{2.779164in}{1.884401in}}%
\pgfpathlineto{\pgfqpoint{2.793274in}{1.884401in}}%
\pgfpathlineto{\pgfqpoint{2.797978in}{1.869983in}}%
\pgfpathlineto{\pgfqpoint{2.812088in}{1.869983in}}%
\pgfpathlineto{\pgfqpoint{2.816791in}{1.987799in}}%
\pgfpathlineto{\pgfqpoint{2.830901in}{1.987799in}}%
\pgfpathlineto{\pgfqpoint{2.835605in}{2.053793in}}%
\pgfpathlineto{\pgfqpoint{2.849715in}{2.053793in}}%
\pgfpathlineto{\pgfqpoint{2.854418in}{2.154511in}}%
\pgfpathlineto{\pgfqpoint{2.868529in}{2.154511in}}%
\pgfpathlineto{\pgfqpoint{2.873232in}{2.458973in}}%
\pgfpathlineto{\pgfqpoint{2.887342in}{2.458973in}}%
\pgfpathlineto{\pgfqpoint{2.892046in}{2.511172in}}%
\pgfpathlineto{\pgfqpoint{2.906156in}{2.511172in}}%
\pgfpathlineto{\pgfqpoint{2.910859in}{2.430957in}}%
\pgfpathlineto{\pgfqpoint{2.924970in}{2.430957in}}%
\pgfpathlineto{\pgfqpoint{2.929673in}{2.599283in}}%
\pgfpathlineto{\pgfqpoint{2.943783in}{2.599283in}}%
\pgfpathlineto{\pgfqpoint{2.948487in}{2.620662in}}%
\pgfpathlineto{\pgfqpoint{2.962597in}{2.620662in}}%
\pgfpathlineto{\pgfqpoint{2.967300in}{2.533561in}}%
\pgfpathlineto{\pgfqpoint{2.981410in}{2.533561in}}%
\pgfpathlineto{\pgfqpoint{2.986114in}{2.641541in}}%
\pgfpathlineto{\pgfqpoint{3.000224in}{2.641541in}}%
\pgfpathlineto{\pgfqpoint{3.004927in}{2.650255in}}%
\pgfpathlineto{\pgfqpoint{3.019038in}{2.650255in}}%
\pgfpathlineto{\pgfqpoint{3.023741in}{2.566734in}}%
\pgfpathlineto{\pgfqpoint{3.037851in}{2.566734in}}%
\pgfpathlineto{\pgfqpoint{3.042555in}{2.641630in}}%
\pgfpathlineto{\pgfqpoint{3.056665in}{2.641630in}}%
\pgfpathlineto{\pgfqpoint{3.061368in}{2.491452in}}%
\pgfpathlineto{\pgfqpoint{3.075479in}{2.491452in}}%
\pgfpathlineto{\pgfqpoint{3.080182in}{2.532258in}}%
\pgfpathlineto{\pgfqpoint{3.094292in}{2.532258in}}%
\pgfpathlineto{\pgfqpoint{3.098996in}{2.393845in}}%
\pgfpathlineto{\pgfqpoint{3.113106in}{2.393845in}}%
\pgfpathlineto{\pgfqpoint{3.117809in}{2.224008in}}%
\pgfpathlineto{\pgfqpoint{3.131919in}{2.224008in}}%
\pgfpathlineto{\pgfqpoint{3.136623in}{2.112217in}}%
\pgfpathlineto{\pgfqpoint{3.150733in}{2.112217in}}%
\pgfpathlineto{\pgfqpoint{3.155436in}{1.963284in}}%
\pgfpathlineto{\pgfqpoint{3.169547in}{1.963284in}}%
\pgfpathlineto{\pgfqpoint{3.174250in}{1.765461in}}%
\pgfpathlineto{\pgfqpoint{3.188360in}{1.765461in}}%
\pgfpathlineto{\pgfqpoint{3.193064in}{1.646041in}}%
\pgfpathlineto{\pgfqpoint{3.207174in}{1.646041in}}%
\pgfpathlineto{\pgfqpoint{3.211877in}{1.252437in}}%
\pgfpathlineto{\pgfqpoint{3.225988in}{1.252437in}}%
\pgfpathlineto{\pgfqpoint{3.230691in}{1.116625in}}%
\pgfpathlineto{\pgfqpoint{3.244801in}{1.116625in}}%
\pgfpathlineto{\pgfqpoint{3.249505in}{1.048374in}}%
\pgfpathlineto{\pgfqpoint{3.263615in}{1.048374in}}%
\pgfpathlineto{\pgfqpoint{3.268318in}{0.971205in}}%
\pgfpathlineto{\pgfqpoint{3.282428in}{0.971205in}}%
\pgfpathlineto{\pgfqpoint{3.287132in}{0.848297in}}%
\pgfpathlineto{\pgfqpoint{3.301242in}{0.848297in}}%
\pgfpathlineto{\pgfqpoint{3.305946in}{0.796125in}}%
\pgfpathlineto{\pgfqpoint{3.320056in}{0.796125in}}%
\pgfpathlineto{\pgfqpoint{3.324759in}{0.761654in}}%
\pgfpathlineto{\pgfqpoint{3.432937in}{0.761654in}}%
\pgfpathlineto{\pgfqpoint{3.437641in}{0.798397in}}%
\pgfpathlineto{\pgfqpoint{3.451751in}{0.798397in}}%
\pgfpathlineto{\pgfqpoint{3.456455in}{0.850462in}}%
\pgfpathlineto{\pgfqpoint{3.470565in}{0.850462in}}%
\pgfpathlineto{\pgfqpoint{3.475268in}{0.963589in}}%
\pgfpathlineto{\pgfqpoint{3.489378in}{0.963589in}}%
\pgfpathlineto{\pgfqpoint{3.494082in}{1.158638in}}%
\pgfpathlineto{\pgfqpoint{3.508192in}{1.158638in}}%
\pgfpathlineto{\pgfqpoint{3.512895in}{1.322553in}}%
\pgfpathlineto{\pgfqpoint{3.527006in}{1.322553in}}%
\pgfpathlineto{\pgfqpoint{3.531709in}{1.415346in}}%
\pgfpathlineto{\pgfqpoint{3.545819in}{1.415346in}}%
\pgfpathlineto{\pgfqpoint{3.550523in}{1.639990in}}%
\pgfpathlineto{\pgfqpoint{3.564633in}{1.639990in}}%
\pgfpathlineto{\pgfqpoint{3.569336in}{1.800743in}}%
\pgfpathlineto{\pgfqpoint{3.583447in}{1.800743in}}%
\pgfpathlineto{\pgfqpoint{3.588150in}{1.905404in}}%
\pgfpathlineto{\pgfqpoint{3.602260in}{1.905404in}}%
\pgfpathlineto{\pgfqpoint{3.606964in}{2.085065in}}%
\pgfpathlineto{\pgfqpoint{3.621074in}{2.085065in}}%
\pgfpathlineto{\pgfqpoint{3.625777in}{2.179069in}}%
\pgfpathlineto{\pgfqpoint{3.639887in}{2.179069in}}%
\pgfpathlineto{\pgfqpoint{3.644591in}{2.246001in}}%
\pgfpathlineto{\pgfqpoint{3.658701in}{2.246001in}}%
\pgfpathlineto{\pgfqpoint{3.663404in}{2.362696in}}%
\pgfpathlineto{\pgfqpoint{3.677515in}{2.362696in}}%
\pgfpathlineto{\pgfqpoint{3.682218in}{2.507217in}}%
\pgfpathlineto{\pgfqpoint{3.696328in}{2.507217in}}%
\pgfpathlineto{\pgfqpoint{3.701032in}{2.659160in}}%
\pgfpathlineto{\pgfqpoint{3.715142in}{2.659160in}}%
\pgfpathlineto{\pgfqpoint{3.719845in}{2.721983in}}%
\pgfpathlineto{\pgfqpoint{3.733956in}{2.721983in}}%
\pgfpathlineto{\pgfqpoint{3.738659in}{2.698326in}}%
\pgfpathlineto{\pgfqpoint{3.752769in}{2.698326in}}%
\pgfpathlineto{\pgfqpoint{3.757473in}{2.666936in}}%
\pgfpathlineto{\pgfqpoint{3.771583in}{2.666936in}}%
\pgfpathlineto{\pgfqpoint{3.776286in}{2.693510in}}%
\pgfpathlineto{\pgfqpoint{3.790396in}{2.693510in}}%
\pgfpathlineto{\pgfqpoint{3.795100in}{2.558179in}}%
\pgfpathlineto{\pgfqpoint{3.809210in}{2.558179in}}%
\pgfpathlineto{\pgfqpoint{3.813913in}{2.491480in}}%
\pgfpathlineto{\pgfqpoint{3.828024in}{2.491480in}}%
\pgfpathlineto{\pgfqpoint{3.832727in}{2.315587in}}%
\pgfpathlineto{\pgfqpoint{3.846837in}{2.315587in}}%
\pgfpathlineto{\pgfqpoint{3.851541in}{2.124372in}}%
\pgfpathlineto{\pgfqpoint{3.865651in}{2.124372in}}%
\pgfpathlineto{\pgfqpoint{3.870354in}{2.045462in}}%
\pgfpathlineto{\pgfqpoint{3.884465in}{2.045462in}}%
\pgfpathlineto{\pgfqpoint{3.889168in}{1.897309in}}%
\pgfpathlineto{\pgfqpoint{3.903278in}{1.897309in}}%
\pgfpathlineto{\pgfqpoint{3.907982in}{1.648936in}}%
\pgfpathlineto{\pgfqpoint{3.922092in}{1.648936in}}%
\pgfpathlineto{\pgfqpoint{3.926795in}{1.578664in}}%
\pgfpathlineto{\pgfqpoint{3.940905in}{1.578664in}}%
\pgfpathlineto{\pgfqpoint{3.945609in}{1.486998in}}%
\pgfpathlineto{\pgfqpoint{3.959719in}{1.486998in}}%
\pgfpathlineto{\pgfqpoint{3.964422in}{1.279570in}}%
\pgfpathlineto{\pgfqpoint{3.978533in}{1.279570in}}%
\pgfpathlineto{\pgfqpoint{3.983236in}{1.199271in}}%
\pgfpathlineto{\pgfqpoint{3.997346in}{1.199271in}}%
\pgfpathlineto{\pgfqpoint{4.002050in}{1.058171in}}%
\pgfpathlineto{\pgfqpoint{4.016160in}{1.058171in}}%
\pgfpathlineto{\pgfqpoint{4.020863in}{0.860086in}}%
\pgfpathlineto{\pgfqpoint{4.034974in}{0.860086in}}%
\pgfpathlineto{\pgfqpoint{4.039677in}{0.807485in}}%
\pgfpathlineto{\pgfqpoint{4.053787in}{0.807485in}}%
\pgfpathlineto{\pgfqpoint{4.058491in}{0.761652in}}%
\pgfpathlineto{\pgfqpoint{4.166669in}{0.761652in}}%
\pgfpathlineto{\pgfqpoint{4.166669in}{0.761652in}}%
\pgfusepath{stroke}%
\end{pgfscope}%
\begin{pgfscope}%
\pgfsetrectcap%
\pgfsetmiterjoin%
\pgfsetlinewidth{1.254687pt}%
\definecolor{currentstroke}{rgb}{1.000000,1.000000,1.000000}%
\pgfsetstrokecolor{currentstroke}%
\pgfsetdash{}{0pt}%
\pgfpathmoveto{\pgfqpoint{0.946717in}{0.663635in}}%
\pgfpathlineto{\pgfqpoint{0.946717in}{2.820000in}}%
\pgfusepath{stroke}%
\end{pgfscope}%
\begin{pgfscope}%
\pgfsetrectcap%
\pgfsetmiterjoin%
\pgfsetlinewidth{1.254687pt}%
\definecolor{currentstroke}{rgb}{1.000000,1.000000,1.000000}%
\pgfsetstrokecolor{currentstroke}%
\pgfsetdash{}{0pt}%
\pgfpathmoveto{\pgfqpoint{4.320000in}{0.663635in}}%
\pgfpathlineto{\pgfqpoint{4.320000in}{2.820000in}}%
\pgfusepath{stroke}%
\end{pgfscope}%
\begin{pgfscope}%
\pgfsetrectcap%
\pgfsetmiterjoin%
\pgfsetlinewidth{1.254687pt}%
\definecolor{currentstroke}{rgb}{1.000000,1.000000,1.000000}%
\pgfsetstrokecolor{currentstroke}%
\pgfsetdash{}{0pt}%
\pgfpathmoveto{\pgfqpoint{0.946717in}{0.663635in}}%
\pgfpathlineto{\pgfqpoint{4.320000in}{0.663635in}}%
\pgfusepath{stroke}%
\end{pgfscope}%
\begin{pgfscope}%
\pgfsetrectcap%
\pgfsetmiterjoin%
\pgfsetlinewidth{1.254687pt}%
\definecolor{currentstroke}{rgb}{1.000000,1.000000,1.000000}%
\pgfsetstrokecolor{currentstroke}%
\pgfsetdash{}{0pt}%
\pgfpathmoveto{\pgfqpoint{0.946717in}{2.820000in}}%
\pgfpathlineto{\pgfqpoint{4.320000in}{2.820000in}}%
\pgfusepath{stroke}%
\end{pgfscope}%
\begin{pgfscope}%
\pgfsetbuttcap%
\pgfsetmiterjoin%
\definecolor{currentfill}{rgb}{0.917647,0.917647,0.949020}%
\pgfsetfillcolor{currentfill}%
\pgfsetfillopacity{0.800000}%
\pgfsetlinewidth{1.003750pt}%
\definecolor{currentstroke}{rgb}{0.800000,0.800000,0.800000}%
\pgfsetstrokecolor{currentstroke}%
\pgfsetstrokeopacity{0.800000}%
\pgfsetdash{}{0pt}%
\pgfpathmoveto{\pgfqpoint{1.024494in}{2.416767in}}%
\pgfpathlineto{\pgfqpoint{2.337244in}{2.416767in}}%
\pgfpathquadraticcurveto{\pgfqpoint{2.359466in}{2.416767in}}{\pgfqpoint{2.359466in}{2.438989in}}%
\pgfpathlineto{\pgfqpoint{2.359466in}{2.742222in}}%
\pgfpathquadraticcurveto{\pgfqpoint{2.359466in}{2.764444in}}{\pgfqpoint{2.337244in}{2.764444in}}%
\pgfpathlineto{\pgfqpoint{1.024494in}{2.764444in}}%
\pgfpathquadraticcurveto{\pgfqpoint{1.002272in}{2.764444in}}{\pgfqpoint{1.002272in}{2.742222in}}%
\pgfpathlineto{\pgfqpoint{1.002272in}{2.438989in}}%
\pgfpathquadraticcurveto{\pgfqpoint{1.002272in}{2.416767in}}{\pgfqpoint{1.024494in}{2.416767in}}%
\pgfpathlineto{\pgfqpoint{1.024494in}{2.416767in}}%
\pgfpathclose%
\pgfusepath{stroke,fill}%
\end{pgfscope}%
\begin{pgfscope}%
\pgfsetroundcap%
\pgfsetroundjoin%
\pgfsetlinewidth{1.505625pt}%
\definecolor{currentstroke}{rgb}{0.298039,0.447059,0.690196}%
\pgfsetstrokecolor{currentstroke}%
\pgfsetdash{}{0pt}%
\pgfpathmoveto{\pgfqpoint{1.046717in}{2.679353in}}%
\pgfpathlineto{\pgfqpoint{1.157828in}{2.679353in}}%
\pgfpathlineto{\pgfqpoint{1.268939in}{2.679353in}}%
\pgfusepath{stroke}%
\end{pgfscope}%
\begin{pgfscope}%
\definecolor{textcolor}{rgb}{0.150000,0.150000,0.150000}%
\pgfsetstrokecolor{textcolor}%
\pgfsetfillcolor{textcolor}%
\pgftext[x=1.357828in,y=2.640464in,left,base]{\color{textcolor}{\sffamily\fontsize{8.000000}{9.600000}\selectfont\catcode`\^=\active\def^{\ifmmode\sp\else\^{}\fi}\catcode`\%=\active\def%{\%}write load per node}}%
\end{pgfscope}%
\begin{pgfscope}%
\pgfsetroundcap%
\pgfsetroundjoin%
\pgfsetlinewidth{1.505625pt}%
\definecolor{currentstroke}{rgb}{0.866667,0.517647,0.321569}%
\pgfsetstrokecolor{currentstroke}%
\pgfsetdash{}{0pt}%
\pgfpathmoveto{\pgfqpoint{1.046717in}{2.522181in}}%
\pgfpathlineto{\pgfqpoint{1.157828in}{2.522181in}}%
\pgfpathlineto{\pgfqpoint{1.268939in}{2.522181in}}%
\pgfusepath{stroke}%
\end{pgfscope}%
\begin{pgfscope}%
\definecolor{textcolor}{rgb}{0.150000,0.150000,0.150000}%
\pgfsetstrokecolor{textcolor}%
\pgfsetfillcolor{textcolor}%
\pgftext[x=1.357828in,y=2.483292in,left,base]{\color{textcolor}{\sffamily\fontsize{8.000000}{9.600000}\selectfont\catcode`\^=\active\def^{\ifmmode\sp\else\^{}\fi}\catcode`\%=\active\def%{\%}write load of cluster}}%
\end{pgfscope}%
\end{pgfpicture}%
\makeatother%
\endgroup%

    \caption{Comparison of write load per node and cluster write load during horizontal scaling}
    \label{fig:diagonal-elasticity_cluster-write-load}
\end{figure}

Because of the in \Cref{sec:evaluation-horizontal-elasticity} addressed drawback of \texttt{cassandra-stress}, which does not detect changes to the cluster architecture, stress tests were cancelled after new nodes were added, and restarted when Cassandra had finished its reconsiliation process.

The advantage of this elasticity strategy is its ability to scale vertically and horizontally independently. This means that during times of low demand resources can be saved or used by other applications. A lower amount of resources also implies lower costs. During high demand times resources can be claimed again to provide a sufficient service level. If K8ssandra reports a high amount of writes the elasticity strategy can the also decide to scale-out horizontally by adding more nodes. As it was shown in \Cref{sec:stress-testing} this increases the total throughput. Because this elasticity strategy combines two elasticity dimensions which perfectly complement each other, diagonal elasticity is clearly superior to horizontal or vertical elasticity alone. As mentioned before, horizontal scale-in is not implemented in this project. This will be further addressed in \Cref{sec:future-work}.


\chapter{Conclusion}
\label{ch:conclusion}

This chapter concludes the thesis by summarizing the results, discussing the limitations and outlining possible future work.

By enabling k8ssandra to scale vertically, horizontally and both combined, thus scaling diagonally, different tasks can be achieved. Vertical scaling reduces to allocated resources when not in use. This in turn reduces cost when using cloud computing infrastructure through its pay-as-you-go pricing model. On the other hand, freeing resources when not in use allows other applications to claim then, making scheduling applications much easier when working with a limited amount of resources.

Horizontal scaling allows k8ssandra to essentially scale its throughput linearly. This was not only shown by the Apache Software Foundation, the developers of Cassandra, but also by industry leading companies such as Netflix\footnote{\raggedright\url{https://netflixtechblog.com/benchmarking-cassandra-scalability-on-aws-over-a-million-writes-per-second-39f45f066c9e}}.

Combining those two dimension into a single elasticity strategy using the Polaris SLO framework, the benefits from both dimensions can be combined. Cost reduction through releasing and claiming resources dynamically and scaling throughput by adding nodes when demand is sufficient.

Nevertheless, limiting factors exist. First and foremost, Cassandra is not designed to be a dynamic application. While it is possible to remove and add nodes to a running k8ssandra cluster, substantial is generated because Cassandra need to reconcile the cluster. During this time, the newly added nodes are not operational and other already existing nodes experience significant load that impairs operability.

\section{Future Work}
\label{sec:future-work}

During implementation various issues arose that were deemed out of scope to solve. These imply the following suggestions:

\begin{itemize}
    \item \textbf{Horizontal scale-in}. As described in \cref{sec:horizontal-elasticity}, the within this thesis implemented version of horizontal scaling only perform scale-out due to the fact that further considerations related to storage have to be made. Some Kubernetes storage drivers support dynamic volume expansion\footnote{\url{https://kubernetes.io/blog/2022/05/05/volume-expansion-ga/}}, therefore this poses an oppertunity for further development.

    \item \textbf{In-place resource resize}. Earlier this year Kubernetes released a feature that allows resource updates to pods without them needing to restart\footnote{\url{https://kubernetes.io/blog/2023/05/12/in-place-pod-resize-alpha/}}. This would be beneficial as restarting k8ssandra nodes takes a long time.

    \item \textbf{Improve stress testing}. Using \texttt{cassandra-stress} as load generation tool has the advantage of being a native Cassandra tool. The downside of this tool is that it is relativly inflexible. As mention in \cref{sec:evaluation-horizontal-elasticity} the cluster architecture is only discovered once during startup. Therefore changes to the architecture are not immediately reflected in the stress test.
    %data gateway

    \item \textbf{Scale to zero}. To provide even more cost effectiveness during times where there is no demand, a scale-to-zero approach could be taken. k8ssandra supports stopping the cluster as a whole. This could be subject to further research.
\end{itemize}


\chapter{Future Work}
\label{ch:future-work}
% horizontal scale-in


\backmatter

% Use an optional list of figures.
% \listoffigures % Starred version, i.e., \listoffigures*, removes the toc entry.

% Use an optional list of tables.
% \cleardoublepage % Start list of tables on the next empty right hand page.
% \listoftables % Starred version, i.e., \listoftables*, removes the toc entry.

% Use an optional list of alogrithms.
% \listofalgorithms
% \addcontentsline{toc}{chapter}{List of Algorithms}

% Add an index.
% \printindex
%
% Add a glossary.
\printglossaries

% Add a bibliography.
\bibliographystyle{ieeetr}
\bibliography{bibliography}

\end{document}
